\documentclass[a4paper,openany]{book}
\usepackage[utf8]{inputenc}

%\includeonly{2012-byelections}

% put all the other packages here:

\usepackage{election06-test}
\usepackage[t]{ebgaramond}
\usepackage{ebgaramond-maths}
	
\usepackage{graphicx}
\usepackage{url}
\usepackage{microtype}

\setcounter{secnumdepth}{-2}
\setcounter{tocdepth}{1}

\usepackage[plainpages=false,pdfpagelabels,pdfauthor={Andrew Teale},pdftitle={Local By-election Results 2010--19},hidelinks]{hyperref}

\renewcommand\resultsyear{2010--19}

\setboolean{maps}{false}

\begin{document}

% Title page

\begin{titlepage}

\begin{center}

\Huge Local By-election Results

2010--19

\bigskip

\Large Andrew Teale

\vfill

\today

\end{center}

\end{titlepage}

% Copyright notice

\begin{center}

\bigskip

First published by Local Elections Archive Project 2020

\bigskip

Local Elections Archive Project
\url{https://www.andrewteale.me.uk/leap/}

\bigskip

\textcopyright\ Andrew Teale, 2020.

All rights reserved.

\bigskip

ISBN 978 1 9998345 3 1

% Permission is granted to copy, distribute and/or modify this document
% under the terms of the GNU Free Documentation License, Version 1.3
% or any later version published by the Free Software Foundation;
% with no Invariant Sections, no Front-Cover Texts, and no Back-Cover Texts.
% A copy of the license is included in the section entitled ``GNU
% Free Documentation License''.

%\bigskip
%
%This file is available for download from
%\url{http://www.andrewteale.me.uk/}
%
%\bigskip
%
%Please advise the author of any corrections which need to be made by
%email: \url{andrewteale@yahoo.co.uk}

\vfill
\end{center}

%\section*{Change Log}

%8 June 2014: Added result for Clydesdale South by-election.

%24 November 2013: Corrected results for Vassall ward, Lambeth (typing error) and Reddish North ward, Stockport (LD candidate incorrectly shown as Labour).
%
%21 November 2013: First version.

\tableofcontents

% Introduction

% Introduction

% \chapter*{Introduction and Abbreviations}
% \addcontentsline{toc}{chapter}{Introduction and Abbreviations}
% %\markright{INTRODUCTION AND ABBREVIATIONS}
% 
% Elections were held on 6th May 2010 to all London boroughs and metropolitan boroughs, and some unitary authorities and shire districts in England. These elections were combined with a general election which was held on the same day. 
%
% The voting system used for all elections covered here was
% first-past-the-post, with multi-member FPTP being used where more than
% one seat was up for election. 
% 
%The results of the general election are shown in Part I. The information in Part I is taken from the Electoral Commission.
%
% All of the seats on the 32 London borough councils were up for election. The vast majority of London boroughs use multi-member wards electing three councillors each; there are also a handful of single-member and two-member wards. Elections to the London boroughs are covered in Part~II, which has been split into two chapters (North and South London).
% 
% The 36 metropolitan boroughs are all elected by thirds. Each ward has
% three councillors, with the winning councillor from the 2006 election being up for
% election in each ward. In some cases two seats were up for election,
% due to the death or resignation of another councillor for the ward
% within six months of the election. Results of these elections are
% contained in Part~III, which each of the former metropolitan counties
% constituting a separate chapter.
% 
% The English unitary authorities and shire districts may have up to
% three councillors in each ward, and may hold elections either all at
% once or by thirds. 
% Only those councils which elect by thirds held
% elections this year; those councils which elect all at once were
% last elected in 2007 and will next be elected in 2011. A few
% districts elect by halves every two years; all of these districts held
% an election this year. Where districts elect by thirds generally not
% all of the wards in the district hold an election every year. A full explanation of the electoral arrangements is
% given at the head of each council's entry.
%
% Due to a botched attempt at local government reorganisation, the 2010 elections to Exeter and Norwich city councils were held on 9th September. It had originally been intended to change these councils to unitary status, which resulted in the scheduled 2010 elections to these councils being cancelled with the intention that the first elections to the new unitary councils would take place in 2011, the councillors elected in 2006 to have their terms extended until 2011. However, when the unitary plans were abandoned by the coalition government following the general election, the High Court ruled that the councillors elected in 2006 had come to the end of their four-year term and could no longer continue in office.
% 
% Unitary election results are shown in Part~IV 
% with shire district results in Part~V. Part~IV is
% divided into eight chapters based on region, while Part~V has one
% chapter for each county.
% 
% For the first time in this series, referendums (Part VI) and by-elections (Part VII) held in 2010 are also included. Scottish local by-elections are held using the Alternative Vote; while details of transfers are shown, for reasons of space some elimination stages have been omitted.
% 
% Finally, at the back you will find an Index of Wards.
% 
% Where a candidate in an election dies, the election in that ward or division is
% cancelled and rearranged for a later date. This happened in the
% following wards or divisions at this election:
% 
% \begin{results}
% \begin{itemize}
% \item Haverstock, Camden\index{Haverstock , Camden@Haverstock, Camden}
% \item Ore, Hastings\index{Ore , Hastings@Ore, Hastings}
% \end{itemize}
% \end{results}
% 
% Here is a list of abbreviations used in this book for major parties
% and selected other parties which fought several councils. This list
% is not exhaustive; parties which put up only a few candidates will
% generally have their abbreviation listed at the head of the entry for
% the retant council. Please note that the ``Lab'' label includes
% candidates who were jointly sponsored by the Labour and Co-operative
% Parties.
% 
% \begin{results}
% BNP - British National Party
% 
% C - Conservative Party
% 
% Grn - Green Party
% 
% Ind - Independent
% 
% Lab - Labour Party
% 
% LD - Liberal Democrat
% 
% Lib - Liberal Party
% 
% Loony - Monster Raving Loony Party
% 
% Respect - Respect, the Unity Coalition
% 
% SocLab - Socialist Labour Party
% 
% UKIP - UK Independence Party
% 
% \end{results}
%
%Errors in a work of this size are inevitable. I take full responsibility for any errors which may have crept in, undertake to correct any errors which I am made aware of, and hope that any errors which you may spot do not substantially affect any use you may make of this book.
%
% I would like to close this section by thanking all those who have
% supplied me with results and sources of information, most notably David Boothroyd, John Cartwright, James Doyle, Keith Edkins, the Electoral Commission, Tom Harris, Paul Harwood, ``hullenedge'', ``Listener'', ``MaxQue'', Philip Mutton, John Swarbrick, Andrew Stidwell, Pete Whitehead and all the members of the Vote UK Forum, and particularly those scores of council
% webpages without which this work would not have been possible. 

 
 
% Here beginneth the content

% 2019 results to go here

\part{2010}
\renewcommand\resultsyear{2010}


\chapter{Referendums in 2010}

\section{Tower Hamlets Mayoral Referendum}
\index{Tower Hamlets!Mayoral Referendum}

One referendum was held in 2010, on 6 May 2010 in the London Borough of Tower Hamlets on the question of whether the borough should have a directly elected Mayor:\label{thref:20100506}

\noindent
\begin{tabular*}{\columnwidth}{@{\extracolsep{\fill}} p{0.545\columnwidth} >{\itshape}l r @{\extracolsep{\fill}}}
& Yes & 60758\\
& No & 39857\\
\end{tabular*}

%\part{By-elections}

\chapter{Parliamentary by-elections}

There were no parliamentary by-elections in 2010.

At the dissolution of Parliament in advance of the 6th May 2010 general election three seats were vacant: Middlesbrough South and East Cleveland (caused by the death of Ashok Kumar, Lab), North West Leicestershire (caused by the death of David Taylor, Lab), and Strangford (caused by the resignation of Iris Robinson, DUP).
\index{Middlesbrough South and East Cleveland , House of Commons@Middlesbrough S. \& East Cleveland, \emph{House of Commons}}
\index{North West Leicestershire , House of Commons@North West Leics., \emph{House of Commons}}
\index{Strangford , House of Commons@Strangford, \emph{House of Commons}}

The general election in the Thirsk and Malton constituency was postponed to 27th May 2010 following the death of the originally nominated UKIP candidate, John Boakes:

\section*{Thirsk and Malton}\label{ThirskMalton}

\index{Thirsk and Malton , House of Commons@Thirsk \& Malton, \emph{House of Commons}}

 \begin{tabular*}{\columnwidth}{@{\extracolsep{\fill}} p{0.545\columnwidth} >{\itshape}l r @{\extracolsep{\fill}}}
McIntosh, A. Ms.*&C&20167\\
Keal, H.&LD&8886\\
Roberts, J.&Lab&5169\\
Horton, T.&UKIP&2502\\
Clark, J.S.&Lib&1418\\ \end{tabular*}\par


\chapter{By-elections to devolved assemblies and the European Parliament}

\section{Greater London Authority}

There were no by-elections in 2010 to the Greater London Authority.

\section{National Assembly for Wales}

There were no by-elections in 2010 to the National Assembly for Wales.

\section{Scottish Parliament}

There were no by-elections in 2010 to the Scottish Parliament.

\section{Northern Ireland Assembly}

Vacancies in the Northern Ireland Assembly are filled by co-option.  The following members were co-opted to the Assembly in 2010:
\begin{itemize}
\item Billy Leonard (SF) was co-opted on 7th January following the resignation of Francie Brolly (East Londonderry).
\item Jonathan Bell (DUP) was co-opted on 22nd January following the resignation of Iris Robinson (Strangford).
\item Conal McDevitt (SDLP) was co-opted on 21st May following the resignation of Carmel Hanna (Belfast South).
\item Paul Givan (DUP) was co-opted on 10th June following the resignation of Jeffrey Donaldson (Lagan Valley).
\item Paul Frew (DUP) was co-opted on 21st June following the resignation of Ian Paisley Jr (Antrim North).
\item Sydney Anderson and Paul Girvan (both DUP) were co-opted on 1st July following the resignations of David Simpson (Upper Bann) and William McCrea (Antrim South) respectively.
\item Chris Lyttle (APNI) was co-opted on 5th July following the resignation of Naomi Long (Belfast East).
\item Simpson Gibson (DUP) was co-opted on 2nd August following the resignation of Jim Shannon (Strangford).
\item William Humphrey (DUP) was co-opted on 13th September following the resignation of Nigel Dodds (Belfast North).
\item P\'ol Callaghan (SDLP) was co-opted on 15th November following the resignation of Mark Durkan (Foyle).
\item Pat Sheehan (SF) was co-opted on 7th December following the resignation of Gerry Adams (Belfast West).
\end{itemize}

\section{European Parliament}

UK vacancies in the European Parliament are filled by the next available person from the party list at the most recent election (which was held in 2009).  The following replacement was made in 2010:
\begin{itemize}
\item Keith Taylor (Grn) replaced Caroline Lucas following her resignation on 17th May (South East).
\end{itemize}

\chapter{Local by-elections and unfilled vacancies}

\begin{resultsiii}

\section{North London}

\subsection{City of London Corporation}
\index{City of London}

\subsubsection*{Bassishaw \hspace*{\fill}\nolinebreak[1]%
\enspace\hspace*{\fill}
\finalhyphendemerits=0
[Wednesday 21st April]}

\index{Bassishaw , City of London@Bassishaw, \emph{City of London}}

Aldermanic election: retirement of Sir David Brewer (Ind).

\noindent
\begin{tabular*}{\columnwidth}{@{\extracolsep{\fill}} p{0.545\columnwidth} >{\itshape}l r @{\extracolsep{\fill}}}
Philip Remnant & Ind & 130\\
Mark Watson-Gandy & Ind & 31\\
\end{tabular*}

\subsubsection*{Queenhithe \hspace*{\fill}\nolinebreak[1]%
\enspace\hspace*{\fill}
\finalhyphendemerits=0
[Tuesday 5th October]}

\index{Queenhithe , City of London@Queenhithe, \emph{City of London}}

Aldermanic election: Gordon Haines (Ind) sought re-election.

\noindent
\begin{tabular*}{\columnwidth}{@{\extracolsep{\fill}} p{0.545\columnwidth} >{\itshape}l r @{\extracolsep{\fill}}}
Gordon Haines & Ind & \emph{unop.}\\
\end{tabular*}

\subsection{Barking and Dagenham}
\index{Barking and Dagenham}

\subsubsection*{Goresbrook \hspace*{\fill}\nolinebreak[1]%
\enspace\hspace*{\fill}
\finalhyphendemerits=0
[8th July]}

\index{Goresbrook , Barking and Dagenham@Goresbrook, \emph{Barking \& Dagenham}}

Disqualification (employed by the council, lollipop lady) of Louise Couling (Lab).

\noindent
\begin{tabular*}{\columnwidth}{@{\extracolsep{\fill}} p{0.545\columnwidth} >{\itshape}l r @{\extracolsep{\fill}}}
Louise Couling & Lab & 881\\
Richard Barnbrook & BNP & 642\\
Felicia Taiwo & LD & 136\\
Paul Ayer & C & 108\\
Warren Northover & Ind & 63\\
Nobby Manning & UKIP & 50\\
Faruk Choudhury & Ind & 11\\
\end{tabular*}

\subsection{Barnet}
\index{Barnet}

At the May 2010 ordinary election there was an unfilled vacancy in Golders Green ward due to the resignation of Chris Harris (C).
\index{Golders Green , Barnet@Golders Green, \emph{Barnet}}

\subsection{Camden}
\index{Camden}

\subsubsection*{Haverstock (3) \hspace*{\fill}\nolinebreak[1]%
\enspace\hspace*{\fill}
\finalhyphendemerits=0
[Tuesday 25th May; 2 LD gains from Lab]}%\label{CamdenHaverstock}

\index{Haverstock , Camden@Haverstock, \emph{Camden}}

Ordinary election postponed from 6th May; death of outgoing councillor Syed Hoque (LD elected as Labour) who had been nominated for re-election.

\noindent
\begin{tabular*}{\columnwidth}{@{\extracolsep{\fill}} p{0.545\columnwidth} >{\itshape}l r @{\extracolsep{\fill}}}
Jill Fraser & LD & 1462\\
Matt Sanders & LD & 1326\\
Rahel Bokth & LD & 1291\\
Sabrina Francis & Lab & 1257\\
Tom Copley & Lab & 1202\\
Joynal Uddin & Lab & 1114\\
Joan Stally & C & 259\\
Tom Frost & C & 250\\
Jane Lawrie & Grn & 246\\
Paul Grader & Grn & 240\\
Robert Ricketts & C & 236\\
Sean Thompson & Grn & 181\\
\end{tabular*}

\subsubsection*{Frognal and Fitzjohns \hspace*{\fill}\nolinebreak[1]%
\enspace\hspace*{\fill}
\finalhyphendemerits=0
[22nd July]}

\index{Frognal and Fitzjohns , Camden@Frognal \& Fitzjohns, \emph{Camden}}

Death of Martin Davies (C).

\noindent
\begin{tabular*}{\columnwidth}{@{\extracolsep{\fill}} p{0.545\columnwidth} >{\itshape}l r @{\extracolsep{\fill}}}
Gio Spinella & C & 1061\\
David Bouchier & LD & 329\\
Jack Smith & Lab & 235\\
Charles Harris & Grn & 71\\
\end{tabular*}

\subsubsection*{Kentish Town \hspace*{\fill}\nolinebreak[1]%
\enspace\hspace*{\fill}
\finalhyphendemerits=0
[28th October]}

\index{Kentish Town , Camden@Kentish Town, \emph{Camden}}

Death of Dave Horan (Lab).

\noindent
\begin{tabular*}{\columnwidth}{@{\extracolsep{\fill}} p{0.545\columnwidth} >{\itshape}l r @{\extracolsep{\fill}}}
Jenny Headlam-Wells & Lab & 1411\\
Nick Russell & LD & 715\\
Naomi Aptowitzer & Grn & 349\\
Will Blair & C & 186\\
\end{tabular*}

\subsection{Hackney}
\index{Hackney}

\subsubsection*{New River \hspace*{\fill}\nolinebreak[1]%
\enspace\hspace*{\fill}
\finalhyphendemerits=0
[16th September]}

\index{New River , Hackney@New River, \emph{Hackney}}

Death of Maureen Middleton (C).

\noindent
\begin{tabular*}{\columnwidth}{@{\extracolsep{\fill}} p{0.545\columnwidth} >{\itshape}l r @{\extracolsep{\fill}}}
Benzion Papier & C & 1567\\
Jonathan Burke & Lab & 1007\\
Stuart Coggins & Grn & 77\\
Benjamin Mathis & LD & 61\\
Darren Fraser & Ind & 26\\
\end{tabular*}

\subsection{Hammersmith and Fulham}
\index{Hammersmith and Fulham}

At the May 2010 ordinary election there was an unfilled vacancy in Avonmore and Brook Green ward due to the resignation of Will Bethell (C).
\index{Avonmore and Brook Green , Hammersmith and Fulham@Avonmore \& Brook Green, \emph{Hammersmith \& Fulham}}

\subsection{Havering}
\index{Havering}

At the May 2010 ordinary election there was an unfilled vacancy in Elm Park ward due to the resignation of David Grantham (C).
\index{Elm Park , Havering@Elm Park, \emph{Havering}}

\subsection{Islington}
\index{Islington}

At the May 2010 ordinary election there was an unfilled vacancy in Bunhill ward due to the death of Donna Boffa (LD).
\index{Bunhill , Islington@Bunhill, \emph{Islington}}

\subsection{Kensington and Chelsea}
\index{Kensington and Chelsea}

\subsubsection*{Holland \hspace*{\fill}\nolinebreak[1]%
\enspace\hspace*{\fill}
\finalhyphendemerits=0
[22nd July]}

\index{Holland , Kensington and Chelsea@Holland, \emph{Kensington \& Chelsea}}

Resignation of Baroness Hanham (C).

\noindent
\begin{tabular*}{\columnwidth}{@{\extracolsep{\fill}} p{0.545\columnwidth} >{\itshape}l r @{\extracolsep{\fill}}}
Rock Feilding-Mellen & C & 649\\
Martin Wilson & LD & 146\\
Jack Bovill & Ind & 70\\
\end{tabular*}

\subsubsection*{Cremorne \hspace*{\fill}\nolinebreak[1]%
\enspace\hspace*{\fill}
\finalhyphendemerits=0
[16th September]}

\index{Cremorne , Kensington and Chelsea@Cremorne, \emph{Kensington \& Chelsea}}

Resignation of Mark Daley (C).

\noindent
\begin{tabular*}{\columnwidth}{@{\extracolsep{\fill}} p{0.545\columnwidth} >{\itshape}l r @{\extracolsep{\fill}}}
Gerard Hargreaves & C & 602\\
Mabel McKeown & Lab & 583\\
Peter Kosta & LD & 180\\
Julia Stephenson & Grn & 51\\
David Coburn & UKIP & 46\\
\end{tabular*}

\subsubsection*{Earl's Court \hspace*{\fill}\nolinebreak[1]%
\enspace\hspace*{\fill}
\finalhyphendemerits=0
[16th September; LD gain from C]}

\index{Earl's Court , Kensington and Chelsea@Earl's Court, \emph{Kensington \& Chelsea}}

Resignation of Barry Phelps (C).

\noindent
\begin{tabular*}{\columnwidth}{@{\extracolsep{\fill}} p{0.545\columnwidth} >{\itshape}l r @{\extracolsep{\fill}}}
Linda Wade & LD & 703\\
Malcolm Spalding & C & 594\\
Joel Bishop & Lab & 151\\
Elizabeth Arbuthnot & Ind & 49\\
Jack Bovill & Ind & 29\\
Michael Enright & Grn & 26\\
Antony Little & UKIP & 18\\
\end{tabular*}

\subsection{Redbridge}
\index{Redbridge}

\subsubsection*{Chadwell \hspace*{\fill}\nolinebreak[1]%
\enspace\hspace*{\fill}
\finalhyphendemerits=0
[8th July]}

\index{Chadwell , Redbridge@Chadwell, \emph{Redbridge}}

Disqualification (employed by the council, schoolteacher) of Mark Gittens (Lab).

\noindent
\begin{tabular*}{\columnwidth}{@{\extracolsep{\fill}} p{0.545\columnwidth} >{\itshape}l r @{\extracolsep{\fill}}}
Wes Streeting & Lab & 800\\
Gary Monro & C & 580\\
John Tyne & LD & 576\\
Wilson Chowdhry & Grn & 413\\
Julian Leppert & BNP & 115\\
Paul Wiffen & UKIP & 54\\
\end{tabular*}

\subsection{Tower Hamlets}
\index{Tower Hamlets}

\subsubsection*{Spitalfields and Banglatown \hspace*{\fill}\nolinebreak[1]%
\enspace\hspace*{\fill}
\finalhyphendemerits=0
[16th December; Respect gain from Lab]}

\index{Spitalfields and Banglatown , Tower Hamlets@Spitalfields \& Banglatown, \emph{Tower Hamlets}}

Election of Lutfur Rahman (Ind elected as Lab) as Mayor of Tower Hamlets.

\noindent
\begin{tabular*}{\columnwidth}{@{\extracolsep{\fill}} p{0.545\columnwidth} >{\itshape}l r @{\extracolsep{\fill}}}
Fozol Miah & Respect & 666\\
Abdul Alim & Lab & 553\\
Matthew Smith & C & 135\\
Margaret Crosbie & Grn & 52\\
Fernando North & LD & 33\\
Jewel Choudhury & Ind & 28\\
\end{tabular*}

\subsection{Westminster}
\index{Westminster}

At the May 2010 ordinary election there was an unfilled vacancy in Bryanston and Dorset Square ward due to the death of Angela Hooper (C).
\index{Bryanston and Dorset Square , Westminster@Bryanston \& Dorset Square, \emph{Westminster}}

\section{South London}

\subsection{Bexley}
\index{Bexley}

At the May 2010 ordinary election there was an unfilled vacancy in Barnehurst ward due to the disqualification (non-attendance) of Bill McEwen (C).
\index{Barnehurst , Bexley@Barnehurst, \emph{Bexley}}

\subsection{Bromley}
\index{Bromley}

At the May 2010 ordinary election there was an unfilled vacancy in Bromley Town ward due to the resignation of Stephen Maly (C).
\index{Bromley Town , Bromley@Bromley Town, \emph{Bromley}}

\subsection{Croydon}
\index{Croydon}

At the May 2010 ordinary election there was an unfilled vacancy in Ashburton ward due to the death of Lindsay Frost (C).
\index{Ashburton , Croydon@Ashburton, \emph{Croydon}}

\subsection{Greenwich}
\index{Greenwich}

At the May 2010 ordinary election there was an unfilled vacancy in Eltham South ward due to the resignation of Elizabeth Truss (C).
\index{Eltham South , Greenwich@Eltham S., \emph{Greenwich}}

\subsection{Lambeth}
\index{Lambeth}

\subsubsection*{Tulse Hill \hspace*{\fill}\nolinebreak[1]%
\enspace\hspace*{\fill}
\finalhyphendemerits=0
[1st July]}

\index{Tulse Hill , Lambeth@Tulse Hill, \emph{Lambeth}}

Resignation of Toren Smith (Lab).

\noindent
\begin{tabular*}{\columnwidth}{@{\extracolsep{\fill}} p{0.545\columnwidth} >{\itshape}l r @{\extracolsep{\fill}}}
Ruth Ling & Lab & 1235\\
Terence Curtis & LD & 745\\
George Graham & Grn & 256\\
Alan Blackburn & C & 94\\
Robin Lambert & UKIP & 36\\
\end{tabular*}

\subsection{Lewisham}
\index{Lewisham}

At the May 2010 ordinary election there was an unfilled vacancy in Whitefoot ward due to the resignation of Cathy Priddey (LD).
\index{Whitefoot , Lewisham@Whitefoot, \emph{Lewisham}}

LPBP = Lewisham People Before Profit

\subsubsection*{Ladywell \hspace*{\fill}\nolinebreak[1]%
\enspace\hspace*{\fill}
\finalhyphendemerits=0
[4th November]}

\index{Ladywell , Lewisham@Ladywell, \emph{Lewisham}}

Resignation of Tim Shand (Lab).

\noindent
\begin{tabular*}{\columnwidth}{@{\extracolsep{\fill}} p{0.545\columnwidth} >{\itshape}l r @{\extracolsep{\fill}}}
Carl Handley & Lab & 1231\\
Ute Michel & Grn & 1041\\
Ingrid Chetram & LD & 314 \\
Helen Mercer & LPBP & 233\\
Benjamin Appleby & C & 153\\
\end{tabular*}

\subsection{Southwark}
\index{Southwark}

At the May 2010 ordinary election there was an unfilled vacancy in The Lane ward due to the resignation of Susan Elan Jones (Lab).
\index{Lane , Southwark@The Lane, \emph{Southwark}}

\subsection{Wandsworth}
\index{Wandsworth}

At the May 2010 ordinary election there was an unfilled vacancy in East Putney ward due to the death of Prof Brian Prichard (C).
\index{East Putney , Wandsworth@East Putney, \emph{Wandsworth}}

\section{Greater Manchester}

\subsection{Bury}
\index{Bury}

\subsubsection*{Pilkington Park \hspace*{\fill}\nolinebreak[1]%
\enspace\hspace*{\fill}
\finalhyphendemerits=0
[6th May]}

\index{Pilkington Park , Bury@Pilkington Park, \emph{Bury}}

Resignation of Peter Redstone (C).

This by-election was combined with the 2010 ordinary election.
%; see page \pageref{PilkingtonParkBury} for the result.

\subsection{Manchester}
\index{Manchester}

\subsubsection*{Hulme \hspace*{\fill}\nolinebreak[1]%
\enspace\hspace*{\fill}
\finalhyphendemerits=0
[4th November]}

\index{Hulme , Manchester@Hulme, \emph{Manchester}}

Resignation of Emily Lomax (Lab).

\noindent
\begin{tabular*}{\columnwidth}{@{\extracolsep{\fill}} p{0.545\columnwidth} >{\itshape}l r @{\extracolsep{\fill}}}
Amina Lone & Lab & 1031\\
Deyika Nzeribe & Grn & 451\\
Grace Baynham & LD & 151\\
Will Stobart & C & 67\\
\end{tabular*}

%\subsubsection*{Baguley \hspace*{\fill}\nolinebreak[1]%
%\enspace\hspace*{\fill}
%\finalhyphendemerits=0
%[tba]}
%
%\index{Baguley , Manchester@Baguley, Manchester}
%
%Death of Eddie McCulley (Lab).
%
%\noindent
%\begin{tabular*}{\columnwidth}{@{\extracolsep{\fill}} p{0.545\columnwidth} >{\itshape}l r @{\extracolsep{\fill}}}
%.
%\end{tabular*}

\subsection{Rochdale}
\index{Rochdale}

\subsubsection*{Healey \hspace*{\fill}\nolinebreak[1]%
\enspace\hspace*{\fill}
\finalhyphendemerits=0
[6th May]}

\index{Healey , Rochdale@Healey, \emph{Rochdale}}

Resignation of Elwyn Watkins (LD).

This by-election was combined with the 2010 ordinary election.
%; see page \pageref{HealeyRochdale} for the result.

\subsection{Tameside}
\index{Tameside}

\subsubsection*{Longdendale \hspace*{\fill}\nolinebreak[1]%
\enspace\hspace*{\fill}
\finalhyphendemerits=0
[30th September]}

\index{Longdendale , Tameside@Longdendale, \emph{Tameside}}

Death of Roy Oldham (Lab).

\noindent
\begin{tabular*}{\columnwidth}{@{\extracolsep{\fill}} p{0.545\columnwidth} >{\itshape}l r @{\extracolsep{\fill}}}
Janet Cooper & Lab & 1275\\
Rob Adlard & C & 1083\\
Melanie Roberts & Grn & 99\\
Anthony Jones & BNP & 80\\
Kevin Misell & UKIP & 67\\
\end{tabular*}

\subsection{Trafford}
\index{Trafford}

\subsubsection*{Bowdon \hspace*{\fill}\nolinebreak[1]%
\enspace\hspace*{\fill}
\finalhyphendemerits=0
[6th May]}

\index{Bowdon , Trafford@Bowdon, \emph{Trafford}}

Resignation of Stephanie Poole (C).

This by-election was combined with the 2010 ordinary election.
%; see page \pageref{BowdonTrafford} for the result.

\section{Merseyside}

\subsection{Knowsley}
\index{Knowsley}

\subsubsection*{St Gabriels \hspace*{\fill}\nolinebreak[1]%
\enspace\hspace*{\fill}
\finalhyphendemerits=0
[6th May]}

\index{Saint Gabriels , Knowsley@St Gabriels, \emph{Knowsley}}

Resignation of Mike Currie (LD).

This by-election was originally scheduled for 25th March but was postponed following the death of the Labour candidate Mike Peers.  It was then combined with the 2010 ordinary election.
%; see page \pageref{StGabrielsKnowsley} for the result.

\subsubsection*{Park \hspace*{\fill}\nolinebreak[1]%
\enspace\hspace*{\fill}
\finalhyphendemerits=0
[16th September]}

\index{Park , Knowsley@Park, \emph{Knowsley}}

Resignation of Margaret Dobbie (Lab).

\noindent
\begin{tabular*}{\columnwidth}{@{\extracolsep{\fill}} p{0.545\columnwidth} >{\itshape}l r @{\extracolsep{\fill}}}
Tony Brennan & Lab & 650\\
John White & LD & 70\\
Gary Robertson & C & 36\\
\end{tabular*}

\subsection{Liverpool}
\index{Liverpool}

EDP = English Democrats Party

\subsubsection*{Fazakerley \hspace*{\fill}\nolinebreak[1]%
\enspace\hspace*{\fill}
\finalhyphendemerits=0
[18th February]}

\index{Fazakerley , Liverpool@Fazakerley, \emph{Liverpool}}

Death of Jack Spriggs (Lab).

\noindent
\begin{tabular*}{\columnwidth}{@{\extracolsep{\fill}} p{0.545\columnwidth} >{\itshape}l r @{\extracolsep{\fill}}}
Louise Armstrong & Lab & 1525\\
Graham Seddon & LD & 807\\
Peter Stafford & BNP & 234\\
Alexander Rudkin & Grn & 84\\
\end{tabular*}

\subsubsection*{Croxteth (2) \hspace*{\fill}\nolinebreak[1]%
\enspace\hspace*{\fill}
\finalhyphendemerits=0
[18th November; 1 Lab gain from LD]}

\index{Croxteth , Liverpool@Croxteth, \emph{Liverpool}}

Death of Rose Bailey (Lab) and resignation of Phil Moffatt (LD).

\noindent
\begin{tabular*}{\columnwidth}{@{\extracolsep{\fill}} p{0.545\columnwidth} >{\itshape}l r @{\extracolsep{\fill}}}
Martin Cummins & Lab & 1447\\
Stephanie Till & Lab & 1424\\
Mark Coughlin & LD & 611\\
Michael Marner & LD & 479\\
Kai Andersen & SocLab & 135\\
Peter Tierney & BNP & 117\\
Barbara Bryan & SocLab & 70\\
Eleanor Pontin & Grn & 63\\
Tony Hammond & UKIP & 50\\
Paul Rimmer & EDP & 35\\
Steven McEllenborough & EDP & 33\\
Norman Coppell & C & 31\\
Brenda Coppell & C & 29\\
Michael Lane & UKIP & 19\\
\end{tabular*}

\subsection{St Helens}
\index{Saint Helens@St Helens}

\subsubsection*{Billinge and Seneley Green \hspace*{\fill}\nolinebreak[1]%
\enspace\hspace*{\fill}
\finalhyphendemerits=0
[14th October]}

\index{Billinge and Seneley Green , Saint Helens@Billinge \& Seneley Green, \emph{St Helens}}

Death of Richard Ward (Lab).

\noindent
\begin{tabular*}{\columnwidth}{@{\extracolsep{\fill}} p{0.545\columnwidth} >{\itshape}l r @{\extracolsep{\fill}}}
Alison Bacon & Lab & 1288\\
Elizabeth Black & C & 624\\
Thomas Gadsden & LD & 229\\
James Winstanley & BNP & 141\\
\end{tabular*}

\subsubsection*{Haydock \hspace*{\fill}\nolinebreak[1]%
\enspace\hspace*{\fill}
\finalhyphendemerits=0
[2nd December]}

\index{Haydock , Saint Helens@Haydock, \emph{St Helens}}

Death of Jim Caunce (Lab).

\noindent
\begin{tabular*}{\columnwidth}{@{\extracolsep{\fill}} p{0.545\columnwidth} >{\itshape}l r @{\extracolsep{\fill}}}
Anthony Burns & Lab & 1234\\
Eric Sheldon & LD & 540\\
John Cunliffe & C & 112\\
James Winstanley & BNP & 82\\
\end{tabular*}

\subsection{Wirral}
\index{Wirral}

\subsubsection*{Pensby and Thingwall \hspace*{\fill}\nolinebreak[1]%
\enspace\hspace*{\fill}
\finalhyphendemerits=0
[6th May]}

\index{Pensby and Thingwall , Wirral@Pensby \& Thingwall, \emph{Wirral}}

Resignation of Sarah Quinn (LD).

This by-election was combined with the 2010 ordinary election.
%; see page \pageref{PensbyThingwallWirral} for the result.

\section{South Yorkshire}

\subsection{Rotherham}
\index{Rotherham}

\subsubsection*{Sitwell \hspace*{\fill}\nolinebreak[1]%
\enspace\hspace*{\fill}
\finalhyphendemerits=0
[5th August]}

\index{Sitwell , Rotherham@Sitwell, \emph{Rotherham}}

Death of Michael Clarke (C).

\noindent
\begin{tabular*}{\columnwidth}{@{\extracolsep{\fill}} p{0.545\columnwidth} >{\itshape}l r @{\extracolsep{\fill}}}
Christopher Middleton & C & 1213\\
Judy Dalton & Lab & 864\\
David Ridgway & Ind & 252\\
John Wilkinson & UKIP & 241\\
Abdul Razaq & LD & 98\\
\end{tabular*}

\subsection{Sheffield}
\index{Sheffield}

\subsubsection*{Woodhouse \hspace*{\fill}\nolinebreak[1]%
\enspace\hspace*{\fill}
\finalhyphendemerits=0
[26th August]}

\index{Woodhouse , Sheffield@Woodhouse, \emph{Sheffield}}

Death of Marjorie Barker (Lab).

\noindent
\begin{tabular*}{\columnwidth}{@{\extracolsep{\fill}} p{0.545\columnwidth} >{\itshape}l r @{\extracolsep{\fill}}}
Jackie Satur & Lab & 1855\\
Joe Otten & LD & 757\\
Jonathan Arnott & UKIP & 491\\
Laurence Hayward & C & 154\\
Jordan Pont & BNP & 143\\
John Grant & Grn & 83\\
Derek Hutchinson & Ind & 58\\
\end{tabular*}

\subsubsection*{Manor Castle \hspace*{\fill}\nolinebreak[1]%
\enspace\hspace*{\fill}
\finalhyphendemerits=0
[21st October]}

\index{Manor Castle , Sheffield@Manor Castle, \emph{Sheffield}}

Death of Jan Wilson (Lab).

\noindent
\begin{tabular*}{\columnwidth}{@{\extracolsep{\fill}} p{0.545\columnwidth} >{\itshape}l r @{\extracolsep{\fill}}}
Terry Fox & Lab & 2092\\
Robbie Cowbury & LD & 303\\
Graham Wroe & Grn & 224\\
Christina Stark & C & 142\\
\end{tabular*}

\section{Tyne and Wear}

\subsection{Gateshead}
\index{Gateshead}

\subsubsection*{Dunston and Teams \hspace*{\fill}\nolinebreak[1]%
\enspace\hspace*{\fill}
\finalhyphendemerits=0
[6th May]}

\index{Dunston and Teams , Gateshead@Dunston \& Teams, \emph{Gateshead}}

Resignation of David Bollands (Lab).

This by-election was combined with the 2010 ordinary election.
%; see page \pageref{DunstonTeamsGateshead} for the result.

\subsubsection*{Lobley Hill and Bensham \hspace*{\fill}\nolinebreak[1]%
\enspace\hspace*{\fill}
\finalhyphendemerits=0
[23rd September]}

\index{Lobley Hill and Bensham , Gateshead@Lobley Hill \& Bensham, \emph{Gateshead}}

Death of Frank Donovan (Lab).

\noindent
\begin{tabular*}{\columnwidth}{@{\extracolsep{\fill}} p{0.545\columnwidth} >{\itshape}l r @{\extracolsep{\fill}}}
Eileen McMaster & Lab & 1120\\
Michael Ruddy & LD & 298\\
Derrick Robson & BNP & 101\\
Val Bond & C & 97\\
\end{tabular*}

\subsubsection*{Saltwell \hspace*{\fill}\nolinebreak[1]%
\enspace\hspace*{\fill}
\finalhyphendemerits=0
[23rd September]}

\index{Saltwell , Gateshead@Saltwell, \emph{Gateshead}}

Resignation of Ian Mearns (Lab).

\noindent
\begin{tabular*}{\columnwidth}{@{\extracolsep{\fill}} p{0.545\columnwidth} >{\itshape}l r @{\extracolsep{\fill}}}
Denise Robson & Lab & 793\\
Laura Turner & LD & 196\\
Alan Bond & C & 86\\
Janet Robson & BNP & 77\\
\end{tabular*}

\subsection{Newcastle upon Tyne}
\index{Newcastle upon Tyne}

\subsubsection*{Denton \hspace*{\fill}\nolinebreak[1]%
\enspace\hspace*{\fill}
\finalhyphendemerits=0
[6th May]}

\index{Denton , Newcastle upon Tyne@Denton, \emph{Newcastle upon Tyne}}

Resignation of Peter Arnold (LD).

This by-election was combined with the 2010 ordinary election.
%; see page \pageref{DentonNewcastleTyne} for the result.

\subsection{North Tyneside}
\index{North Tyneside}

\subsubsection*{Battle Hill \hspace*{\fill}\nolinebreak[1]%
\enspace\hspace*{\fill}
\finalhyphendemerits=0
[30th September]}

\index{Battle Hill , North Tyneside@Battle Hill, \emph{N. Tyneside}}

Resignation of Mary Glindon (Lab).

\noindent
\begin{tabular*}{\columnwidth}{@{\extracolsep{\fill}} p{0.545\columnwidth} >{\itshape}l r @{\extracolsep{\fill}}}
Lesley Spillard & Lab & 1334\\
Dorothy Bradley & LD & 826\\
Wendy Morton & C & 97\\
Dan Ellis & Ind & 43\\
\end{tabular*}

\subsection{South Tyneside}
\index{South Tyneside}

\subsubsection*{Primrose \hspace*{\fill}\nolinebreak[1]%
\enspace\hspace*{\fill}
\finalhyphendemerits=0
[25th February]}

\index{Primrose , South Tyneside@Primrose, \emph{S. Tyneside}}

Death of Barrie Scorer (Lab).

\noindent
\begin{tabular*}{\columnwidth}{@{\extracolsep{\fill}} p{0.545\columnwidth} >{\itshape}l r @{\extracolsep{\fill}}}
Ken Stephenson & Lab & 854\\
Pete Hodgkinson & BNP & 566\\
Aaron Luke & Ind & 213\\
David Rice & Ind & 174\\
Anthony Lanaghan & C & 124\\
Susan Troupe & LD & 100\\
\end{tabular*}

\section{West Midlands}

\subsection{Dudley}
\index{Dudley}

At the May 2010 ordinary election there was an unfilled vacancy in Sedgley ward due to the resignation of John Perry (C).
\index{Sedgley , Dudley@Sedgley, \emph{Dudley}}

\subsection{Sandwell}
\index{Sandwell}

At the May 2010 ordinary election there was an unfilled vacancy in Rowley ward due to the resignation of Bill Thomas (Lab).
\index{Sedgley , Sandwell@Sedgley, \emph{Sandwell}}

NF = National Front

\subsubsection*{Wednesbury North \hspace*{\fill}\nolinebreak[1]%
\enspace\hspace*{\fill}
\finalhyphendemerits=0
[18th November; Lab gain from C]}

\index{Wednesbury North , Sandwell@Wednesbury N., \emph{Sandwell}}

Resignation of Bill Archer (C).

\noindent
\begin{tabular*}{\columnwidth}{@{\extracolsep{\fill}} p{0.545\columnwidth} >{\itshape}l r @{\extracolsep{\fill}}}
Peter Hughes & Lab & 1322\\
Mike Warner & C & 643\\
Ade Woodhouse & NF & 76\\
Mary Wilson & LD & 45\\
Colin Bye & Grn & 42\\
\end{tabular*}

%\subsection{Solihull}
%
%\subsubsection*{Olton \hspace*{\fill}\nolinebreak[1]%
%\enspace\hspace*{\fill}
%\finalhyphendemerits=0
%[20th January]}
%
%\index{Olton , Solihull@Olton, Solihull}
%
%Death of Honor Cox (LD).
%
%\noindent
%\begin{tabular*}{\columnwidth}{@{\extracolsep{\fill}} p{0.545\columnwidth} >{\itshape}l r @{\extracolsep{\fill}}}
%.
%\end{tabular*}

\subsection{Walsall}
\index{Walsall}

\subsubsection*{Bloxwich West \hspace*{\fill}\nolinebreak[1]%
\enspace\hspace*{\fill}
\finalhyphendemerits=0
[15th July; Lab gain from C]}

\index{Bloxwich West , Walsall@Bloxwich W., \emph{Walsall}}

Death of Melvin Pitt (C).

\noindent
\begin{tabular*}{\columnwidth}{@{\extracolsep{\fill}} p{0.545\columnwidth} >{\itshape}l r @{\extracolsep{\fill}}}
Frederick Westley & Lab & 1142\\
Theresa Smith & C & 800\\
Paul Valdmanis & UKIP & 91\\
Christine Cockayne & LD & 71\\
Zoe Henderson & Grn & 28\\
\end{tabular*}

\subsubsection*{Rushall-Shelfield \hspace*{\fill}\nolinebreak[1]%
\enspace\hspace*{\fill}
\finalhyphendemerits=0
[11th November]}

\index{Rushall-Shelfield , Walsall@Rushall-Shelfield, \emph{Walsall}}

Death of Albert Griffiths (C).

\noindent
\begin{tabular*}{\columnwidth}{@{\extracolsep{\fill}} p{0.545\columnwidth} >{\itshape}l r @{\extracolsep{\fill}}}
Lorna Rattigan & C & 639\\
Richard Worrall & Lab & 611\\
Bill Vaughan & BNP & 141\\
Tim Melville & UKIP & 90\\
Mark Beech & Loony & 42\\
\end{tabular*}

\subsection{Wolverhampton}
\index{Wolverhampton}

\subsubsection*{Wednesfield North \hspace*{\fill}\nolinebreak[1]%
\enspace\hspace*{\fill}
\finalhyphendemerits=0
[6th May]}

\index{Wednesfield North , Wolverhampton@Wednesfield N., \emph{Wolverhampton}}

Resignation of Charlotte Quarmby (C).

This by-election was combined with the 2010 ordinary election.
%; see page \pageref{WednesfieldNWolverhampton} for the result.

\subsubsection*{Bilston North \hspace*{\fill}\nolinebreak[1]%
\enspace\hspace*{\fill}
\finalhyphendemerits=0
[29th July; Lab gain from C]}

\index{Bilston North , Wolverhampton@Bilston N., \emph{Wolverhampton}}

Death of Gill Fellows (C).

\noindent
\begin{tabular*}{\columnwidth}{@{\extracolsep{\fill}} p{0.545\columnwidth} >{\itshape}l r @{\extracolsep{\fill}}}
Linda Leach & Lab & 1292\\
Marlene Berry & C & 460\\
Stewart Gardner & BNP & 131\\
Barry Hodgson & UKIP & 55\\
Darren Friel & LD & 52\\
\end{tabular*}

\section{West Yorkshire}

\subsection{Bradford}
\index{Bradford}

\subsubsection*{Worth Valley \hspace*{\fill}\nolinebreak[1]%
\enspace\hspace*{\fill}
\finalhyphendemerits=0
[25th November]}

\index{Worth Valley , Bradford@Worth Valley, \emph{Bradford}}

Resignation of Kris Hopkins (C).

\noindent
\begin{tabular*}{\columnwidth}{@{\extracolsep{\fill}} p{0.545\columnwidth} >{\itshape}l r @{\extracolsep{\fill}}}
Russell Brown & C & 1020\\
Mark Curtis & Lab & 697\\
Robert Swindells & Grn & 235\\
Sharon Purvis & LD & 180\\
\end{tabular*}

\subsection{Leeds}
\index{Leeds}

\subsubsection*{Hyde Park and Woodhouse \hspace*{\fill}\nolinebreak[1]%
\enspace\hspace*{\fill}
\finalhyphendemerits=0
[18th February; Lab gain from LD]}

\index{Hyde Park and Woodhouse , Leeds@Hyde Park \& Woodhouse, \emph{Leeds}}

Death of Kabeer Hussain (LD).

\noindent
\begin{tabular*}{\columnwidth}{@{\extracolsep{\fill}} p{0.545\columnwidth} >{\itshape}l r @{\extracolsep{\fill}}}
Gerry Harper & Lab & 1054\\
Mike Taylor & LD & 671\\
Yasser Khalid & C & 188\\
Adele Beeson & Ind & 150\\
Christopher Foren & Grn & 140\\
\end{tabular*}

\subsubsection*{Guiseley and Rawdon \hspace*{\fill}\nolinebreak[1]%
\enspace\hspace*{\fill}
\finalhyphendemerits=0
[14th October]}

\index{Guiseley and Rawdon , Leeds@Guiseley \& Rawdon, \emph{Leeds}}

Resignation of Stuart Andrew (C).

\noindent
\begin{tabular*}{\columnwidth}{@{\extracolsep{\fill}} p{0.545\columnwidth} >{\itshape}l r @{\extracolsep{\fill}}}
Paul Wadsworth & C & 2075\\
Mike King & Lab & 1708\\
Cindy Cleasby & LD & 818\\
\end{tabular*}

\columnbreak

\subsection{Wakefield}
\index{Wakefield}

\subsubsection*{Airedale and Ferry Fryston \hspace*{\fill}\nolinebreak[1]%
\enspace\hspace*{\fill}
\finalhyphendemerits=0
[21st January]}

\index{Airedale and Ferry Fryston , Wakefield@Airedale \& Ferry Fryston, \emph{Wakefield}}

Death of Graham Phelps (Lab).

\noindent
\begin{tabular*}{\columnwidth}{@{\extracolsep{\fill}} p{0.545\columnwidth} >{\itshape}l r @{\extracolsep{\fill}}}
Les Shaw & Lab & 1330\\
Paul Kirby & LD & 603\\
Stephen Rogerson & BNP & 353\\
Carl Milner & C & 275\\
Jason Smart & Ind & 102\\
\end{tabular*}

\section{Bedfordshire}

\subsection{Bedford}
\index{Bedford}

\subsubsection*{Kempston North \hspace*{\fill}\nolinebreak[1]%
\enspace\hspace*{\fill}
\finalhyphendemerits=0
[24th June]}

\index{Kempston North , Bedford@Kempston N., \emph{Bedford}}

Death of Ray Oliver (Lab).

\noindent
\begin{tabular*}{\columnwidth}{@{\extracolsep{\fill}} p{0.545\columnwidth} >{\itshape}l r @{\extracolsep{\fill}}}
Shan Hunt & Lab & 715\\
Martin Quince & C & 384\\
Ant Caprioli & LD & 272\\
\end{tabular*}

\subsection{Luton}
\index{Luton}

\subsubsection*{South \hspace*{\fill}\nolinebreak[1]%
\enspace\hspace*{\fill}
\finalhyphendemerits=0
[6th May]}

\index{South , Luton@South, \emph{Luton}}

Disqualification (non-attendance) of Michelle Kiansumba (Lab).

\noindent
\begin{tabular*}{\columnwidth}{@{\extracolsep{\fill}} p{0.545\columnwidth} >{\itshape}l r @{\extracolsep{\fill}}}
Keir Gale & Lab & 1493\\
Peter Banks-Smith & C & 1015\\
Richard Hayward & LD & 616\\
Lance Richardson & UKIP & 201\\
Marc Scheimann & Grn & 155\\
\end{tabular*}

\section{Berkshire}

\subsection{Bracknell Forest}
\index{Bracknell Forest}

\subsubsection*{Owlsmoor \hspace*{\fill}\nolinebreak[1]%
\enspace\hspace*{\fill}
\finalhyphendemerits=0
[25th February]}

\index{Owlsmoor , Bracknell Forest@Owlsmoor, \emph{Bracknell Forest}}

Death of Ray Simonds (C).

\noindent
\begin{tabular*}{\columnwidth}{@{\extracolsep{\fill}} p{0.545\columnwidth} >{\itshape}l r @{\extracolsep{\fill}}}
Norman Bowers & C & 508\\
Mark Thompson & LD & 238\\
Guy Gillbe & Lab & 126\\
Peter Forbes & Grn & 66\\
\end{tabular*}

\subsection{West Berkshire}
\index{West Berkshire}

\subsubsection*{Thatcham South and Crookham \hspace*{\fill}\nolinebreak[1]%
\enspace\hspace*{\fill}
\finalhyphendemerits=0
[22nd July]}

\index{Thatcham South and Crookham , West Berkshire@Thatcham S. \& Crookham, \emph{W. Berks.}}

Resignation of Terry Port (LD).

\noindent
\begin{tabular*}{\columnwidth}{@{\extracolsep{\fill}} p{0.545\columnwidth} >{\itshape}l r @{\extracolsep{\fill}}}
Bob Morgan & LD & 936\\
Dominic Boeck & C & 787\\
\end{tabular*}

%\subsection{Windsor and Maidenhead}
%
%\subsubsection*{Park \hspace*{\fill}\nolinebreak[1]%
%\enspace\hspace*{\fill}
%\finalhyphendemerits=0
%[6th January]}
%
%\index{Park , Windsor and Maidenhead@Park, Windsor \& Maidenhead}
%
%Resignation of Richard Gard (C).
%
%\noindent
%\begin{tabular*}{\columnwidth}{@{\extracolsep{\fill}} p{0.545\columnwidth} >{\itshape}l r @{\extracolsep{\fill}}}
%.
%\end{tabular*}

\subsection{Wokingham}
\index{Wokingham}

At the May 2010 ordinary election there was an unfilled vacancy in Evendons ward due to the resignation of Dennis Morgan (C).
\index{Evendons , Wokingham@Evendons, \emph{Wokingham}}

%There was an unfilled vacancy in Evendons ward at the ordinary May elections following the resignation of Dennis Morgan (C) with less than six months of his term remaining.

\subsubsection*{Maiden Erlegh \hspace*{\fill}\nolinebreak[1]%
\enspace\hspace*{\fill}
\finalhyphendemerits=0
[6th May]}

\index{Maiden Erlegh , Wokingham@Maiden Erlegh, \emph{Wokingham}}

Death of Chris Edmunds (C).

This by-election was combined with the 2010 ordinary election.
%; see page \pageref{MaidenErleghWokingham} for the result.

\subsubsection*{Shinfield South \hspace*{\fill}\nolinebreak[1]%
\enspace\hspace*{\fill}
\finalhyphendemerits=0
[6th May]}

\index{Shinfield South , Wokingham@Shinfield S., \emph{Wokingham}}

Death of Malcolm Bryant (C).

This by-election was combined with the 2010 ordinary election.
%; see page \pageref{ShinfieldSWokingham} for the result.

\section{Buckinghamshire}

\subsection{Aylesbury Vale}
\index{Aylesbury Vale}

\subsubsection*{Aylesbury Central \hspace*{\fill}\nolinebreak[1]%
\enspace\hspace*{\fill}
\finalhyphendemerits=0
[11th February]}

\index{Aylesbury Central , Aylesbury Vale@Aylesbury C., \emph{Aylesbury Vale}}

Resignation of Keith Turner (LD).

\noindent
\begin{tabular*}{\columnwidth}{@{\extracolsep{\fill}} p{0.545\columnwidth} >{\itshape}l r @{\extracolsep{\fill}}}
Graham Webster & LD & 354\\
Mark Winn & C & 213\\
Michael Beall & Lab & 67\\
Brian Adams & UKIP & 65\\
\end{tabular*}

\subsubsection*{Luffield Abbey \hspace*{\fill}\nolinebreak[1]%
\enspace\hspace*{\fill}
\finalhyphendemerits=0
[11th February]}

\index{Luffield Abbey , Aylesbury Vale@Luffield Abbey, \emph{Aylesbury Vale}}

Resignation of Stefan Balbuza (C).

\noindent
\begin{tabular*}{\columnwidth}{@{\extracolsep{\fill}} p{0.545\columnwidth} >{\itshape}l r @{\extracolsep{\fill}}}
Pearl Lewis & C & 343\\
John Russell & UKIP & 151\\
Ian Metherell & LD & 133\\
Mark Benson & Ind & 63\\
\end{tabular*}

\subsection{Chiltern}

\subsubsection*{Amersham on the Hill \hspace*{\fill}\nolinebreak[1]%
\enspace\hspace*{\fill}
\finalhyphendemerits=0
[6th May]}

\index{Amersham on the Hill , Chiltern@Amersham on the Hill, \emph{Chiltern}}

Resignation of Ronnie Lamont (C).

\noindent
\begin{tabular*}{\columnwidth}{@{\extracolsep{\fill}} p{0.545\columnwidth} >{\itshape}l r @{\extracolsep{\fill}}}
Nigel Shepherd & C & 1125\\
Howard Maitland-Jones & LD & 1073\\
Peter Harper & Lab & 203\\
\end{tabular*}

\subsubsection*{Ashley Green, Latimer and Chenies \hspace*{\fill}\nolinebreak[1]%
\enspace\hspace*{\fill}
\finalhyphendemerits=0
[21st October]}

\index{Ashley Green, Latimer and Chenies , Chiltern@Ashley Green, Latimer \& Chenies, \emph{Chiltern}}

Death of Graham Sussum (C).

\noindent
\begin{tabular*}{\columnwidth}{@{\extracolsep{\fill}} p{0.545\columnwidth} >{\itshape}l r @{\extracolsep{\fill}}}
Andrew Garth & C & 399\\
Anil Kantaria & LD & 92\\
Peter Ward & Lab & 47\\
Alan Stevens & UKIP & 11\\
\end{tabular*}

\subsubsection*{Great Missenden \hspace*{\fill}\nolinebreak[1]%
\enspace\hspace*{\fill}
\finalhyphendemerits=0
[21st October]}

\index{Great Missenden , Chiltern@Great Missenden, \emph{Chiltern}}

Resignation of Bob Swayne (Ind elected as C).

\noindent
\begin{tabular*}{\columnwidth}{@{\extracolsep{\fill}} p{0.545\columnwidth} >{\itshape}l r @{\extracolsep{\fill}}}
Gilbert Nockles & C & 306\\
Seb Berry & LD & 281\\
Dennis Sluman & UKIP & 90\\
\end{tabular*}

\subsection{Milton Keynes}
\index{Milton Keynes}

\subsubsection*{Stony Stratford \hspace*{\fill}\nolinebreak[1]%
\enspace\hspace*{\fill}
\finalhyphendemerits=0
[6th May]}

\index{Stony Stratford , Milton Keynes@Stony Stratford, \emph{Milton Keynes}}

Resignation of Brin Carstens (C).

This by-election was combined with the 2010 ordinary election.
%; see page \pageref{StonyStratfordMK} for the result.

\subsection{South Bucks}
\index{South Bucks}

\subsubsection*{Burnham Church \hspace*{\fill}\nolinebreak[1]%
\enspace\hspace*{\fill}
\finalhyphendemerits=0
[26th August]}

\index{Burnham Church , South Bucks@Burnham Church, \emph{S. Bucks}}

Resignation of Nicholas Binns (C).

\noindent
\begin{tabular*}{\columnwidth}{@{\extracolsep{\fill}} p{0.545\columnwidth} >{\itshape}l r @{\extracolsep{\fill}}}
Paul Kelly & C & 459\\
Peter Price & UKIP & 276\\
David Linsdall & LD & 80\\
\end{tabular*}

\subsection{Wycombe}
\index{Wycombe}

\subsubsection*{Totteridge \hspace*{\fill}\nolinebreak[1]%
\enspace\hspace*{\fill}
\finalhyphendemerits=0
[6th May; LD gain from C]}

\index{Totteridge , Wycombe@Totteridge, \emph{Wycombe}}

Resignation of Joel Foley (C).

\noindent
\begin{tabular*}{\columnwidth}{@{\extracolsep{\fill}} p{0.545\columnwidth} >{\itshape}l r @{\extracolsep{\fill}}}
Jen Joseph & LD & 1234\\
Lakshan Wanigasooriya & C & 1017\\
Ian Bates & Lab & 524\\
\end{tabular*}

\subsubsection*{Greater Marlow \hspace*{\fill}\nolinebreak[1]%
\enspace\hspace*{\fill}
\finalhyphendemerits=0
[15th July]}

\index{Greater Marlow , Wycombe@Greater Marlow, \emph{Wycombe}}

Disqualification (non-attendance) of Helen Wilkinson (C).

\noindent
\begin{tabular*}{\columnwidth}{@{\extracolsep{\fill}} p{0.545\columnwidth} >{\itshape}l r @{\extracolsep{\fill}}}
Dominic Barnes & C & 609\\
Mike Harris & Ind & 348\\
Kavita Mohan & LD & 195\\
\end{tabular*}

\section{Cambridgeshire}

\subsection{County Council}
\index{Cambridgeshire}

CambSoc = Cambridge Socialists

\subsubsection*{Wisbech North \hspace*{\fill}\nolinebreak[1]%
\enspace\hspace*{\fill}
\finalhyphendemerits=0
[15th April]}

\index{Wisbech North , Cambridgeshire@Wisbech N., \emph{Cambs.}}

Death of Les Sims (C).

\noindent
\begin{tabular*}{\columnwidth}{@{\extracolsep{\fill}} p{0.545\columnwidth} >{\itshape}l r @{\extracolsep{\fill}}}
Samantha Hoy & C & 548\\
David Patrick & LD & 506\\
Barry Diggle & Lab & 287\\
Paul Clapp & UKIP & 233\\
\end{tabular*}

\subsubsection*{East Chesterton \hspace*{\fill}\nolinebreak[1]%
\enspace\hspace*{\fill}
\finalhyphendemerits=0
[16th September]}

\index{East Chesterton , Cambridgeshire@East Chesterton, \emph{Cambs.}}

Resignation of Siep Wijsenbeek (LD).

\noindent
\begin{tabular*}{\columnwidth}{@{\extracolsep{\fill}} p{0.5\columnwidth} >{\itshape}l r @{\extracolsep{\fill}}}
Ian Manning & LD & 832\\
Gerri Bird & Lab & 663\\
Matthew Bradney & C & 334\\
Peter Pope & Grn & 117\\
Anna Gordon & CambSoc & 53\\
Peter Burkinshaw & UKIP & 37\\
\end{tabular*}

\subsection{Cambridge}
\index{Cambridge}

\subsubsection*{East Chesterton \hspace*{\fill}\nolinebreak[1]%
\enspace\hspace*{\fill}
\finalhyphendemerits=0
[6th May]}

\index{East Chesterton , Cambridge@East Chesterton, \emph{Cambridge}}

Resignation of Jennifer Liddle (LD).

This by-election was combined with the 2010 ordinary election.
%; see page \pageref{EastChestertonCambridge} for the result.

\subsubsection*{Petersfield \hspace*{\fill}\nolinebreak[1]%
\enspace\hspace*{\fill}
\finalhyphendemerits=0
[6th May]}

\index{Petersfield , Cambridge@Petersfield, \emph{Cambridge}}

Resignation of Ben Bradnack (Lab).

This by-election was combined with the 2010 ordinary election.
%; see page \pageref{PetersfieldCambridge} for the result.

\subsubsection*{West Chesterton \hspace*{\fill}\nolinebreak[1]%
\enspace\hspace*{\fill}
\finalhyphendemerits=0
[6th May]}

\index{West Chesterton , Cambridge@West Chesterton, \emph{Cambridge}}

Resignation of Diane Armstrong (LD).

This by-election was combined with the 2010 ordinary election.
%; see page \pageref{WestChestertonCambridge} for the result.

\subsubsection*{Coleridge \hspace*{\fill}\nolinebreak[1]%
\enspace\hspace*{\fill}
\finalhyphendemerits=0
[4th November; Lab gain from C]}

\index{Coleridge , Cambridge@Coleridge, \emph{Cambridge}}

Resignation of Chris Howell (C).

\noindent
\begin{tabular*}{\columnwidth}{@{\extracolsep{\fill}} p{0.545\columnwidth} >{\itshape}l r @{\extracolsep{\fill}}}
George Owers & Lab & 900\\
Andrew Bower & C & 734\\
Sarah Barnes & LD & 223\\
Valerie Hopkins & Grn & 137\\
Albert Watts & UKIP & 53\\
\end{tabular*}

\subsection{Fenland}
\index{Fenland}

\subsubsection*{The Mills \hspace*{\fill}\nolinebreak[1]%
\enspace\hspace*{\fill}
\finalhyphendemerits=0
[4th March]}

\index{Mills , Fenland@The Mills, \emph{Fenland}}

Death of Ray German (C).

\noindent
\begin{tabular*}{\columnwidth}{@{\extracolsep{\fill}} p{0.545\columnwidth} >{\itshape}l r @{\extracolsep{\fill}}}
Robert Chambers & C & 301\\
Chris Howes & LD & 264\\
Sandra Rylance & UKIP & 58\\
Max Kelly & Lab & 33\\
\end{tabular*}

\subsubsection*{Kirkgate \hspace*{\fill}\nolinebreak[1]%
\enspace\hspace*{\fill}
\finalhyphendemerits=0
[15th April; LD gain from C]}

\index{Kirkgate , Fenland@Kirkgate, \emph{Fenland}}

Death of Les Sims (C).

\noindent
\begin{tabular*}{\columnwidth}{@{\extracolsep{\fill}} p{0.545\columnwidth} >{\itshape}l r @{\extracolsep{\fill}}}
Dave Patrick & LD & 287\\
Steve Tierney & C & 145\\
Barry Diggle & Lab & 74\\
Paul Clapp & UKIP & 54\\
\end{tabular*}

\subsection{Huntingdonshire}
\index{Huntingdonshire}

\subsubsection*{Fenstanton \hspace*{\fill}\nolinebreak[1]%
\enspace\hspace*{\fill}
\finalhyphendemerits=0
[25th February; LD gain from C]}

\index{Fenstanton , Huntingdonshire@Fenstanton, \emph{Hunts.}}

Resignation of Paul Dakers (C).

\noindent
\begin{tabular*}{\columnwidth}{@{\extracolsep{\fill}} p{0.545\columnwidth} >{\itshape}l r @{\extracolsep{\fill}}}
Colin Saunderson & LD & 391\\
David O'Neill & C & 337\\
Angela Richards & Lab & 37\\
\end{tabular*}

\section{Cheshire}

\subsection{Cheshire East}
\index{Cheshire East}

\subsubsection*{Alderley \hspace*{\fill}\nolinebreak[1]%
\enspace\hspace*{\fill}
\finalhyphendemerits=0
[30th September]}

\index{Alderley , Cheshire East@Alderley, \emph{Ches. E.}}

Death of Liz Gilliland (C).

\noindent
\begin{tabular*}{\columnwidth}{@{\extracolsep{\fill}} p{0.545\columnwidth} >{\itshape}l r @{\extracolsep{\fill}}}
Matthew Lloyd & C & 1647\\
Oliver Romain & LD & 779\\
\end{tabular*}

\subsection{Halton}
\index{Halton}

\subsubsection*{Mersey \hspace*{\fill}\nolinebreak[1]%
\enspace\hspace*{\fill}
\finalhyphendemerits=0
[6th May]}

\index{Mersey , Halton@Mersey, \emph{Halton}}

Death of Sue Blackmore (LD).

This by-election was combined with the 2010 ordinary election.
%; see page \pageref{MerseyHalton} for the result.

\subsection{Warrington}
\index{Warrington}

\subsubsection*{Fairfield and Howley \hspace*{\fill}\nolinebreak[1]%
\enspace\hspace*{\fill}
\finalhyphendemerits=0
[6th May]}

\index{Fairfield and Howley , Warrington@Fairfield \& Howley, \emph{Warrington}}

Resignation of Yvonne Fovargue (Lab).

This by-election was combined with the 2010 ordinary election.
%; see page \pageref{FairfieldHowleyWarrington} for the result.

\subsubsection*{Bewsey and Whitecross \hspace*{\fill}\nolinebreak[1]%
\enspace\hspace*{\fill}
\finalhyphendemerits=0
[9th December; Lab gain from LD]}

\index{Bewsey and Whitecross , Warrington@Bewsey \& Whitecross, \emph{Warrington}}

Resignation of Jo Crotty (LD).

\noindent
\begin{tabular*}{\columnwidth}{@{\extracolsep{\fill}} p{0.545\columnwidth} >{\itshape}l r @{\extracolsep{\fill}}}
Jeff Richards & Lab & 1032\\
Ann Raymond & LD & 221\\
Lance Reah & C & 118\\
Lyndsay McAteer & Grn & 47\\
John Mulhall & Ind & 33\\
\end{tabular*}

\section{Cumbria}

\subsection{County Council}
\index{Cumbria}

\subsubsection*{Longtown and Bewcastle \hspace*{\fill}\nolinebreak[1]%
\enspace\hspace*{\fill}
\finalhyphendemerits=0
[6th May]}

\index{Longtown and Bewcastle , Cumbria@Longtown \& Bewcastle, \emph{Cumbria}}

Resignation of Amanda Long (C).

\noindent
\begin{tabular*}{\columnwidth}{@{\extracolsep{\fill}} p{0.545\columnwidth} >{\itshape}l r @{\extracolsep{\fill}}}
Val Tarbitt & C & 1718\\
Ian Highmore & LD & 1158\\
Helen Horne & Lab & 495\\
Tony Carvell & BNP & 175\\
\end{tabular*}

\subsubsection*{Aspatria and Wharrels \hspace*{\fill}\nolinebreak[1]%
\enspace\hspace*{\fill}
\finalhyphendemerits=0
[9th September]}

\index{Aspatria and Wharrels , Cumbria@Aspatria \& Wharrels, \emph{Cumbria}}

Death of Jim Buchanan (C).

\noindent
\begin{tabular*}{\columnwidth}{@{\extracolsep{\fill}} p{0.545\columnwidth} >{\itshape}l r @{\extracolsep{\fill}}}
Mike Johnson & C & 823\\
Helen Graham & Grn & 342\\
\end{tabular*}

\subsection{Allerdale}
\index{Allerdale}

\subsubsection*{Christchurch \hspace*{\fill}\nolinebreak[1]%
\enspace\hspace*{\fill}
\finalhyphendemerits=0
[19th August]}

\index{Christchurch , Allerdale@Christchurch, \emph{Allerdale}}

Disqualification (non-attendance) of Les Lytollis (C).

\noindent
\begin{tabular*}{\columnwidth}{@{\extracolsep{\fill}} p{0.545\columnwidth} >{\itshape}l r @{\extracolsep{\fill}}}
Roy Swindells & C & 466\\
Roger Peck & LD & 131\\
Bob Edwards & Grn & 108\\
\end{tabular*}

\subsection{Carlisle}
\index{Carlisle}

EDP = English Democrats Party

\subsubsection*{Irthing \hspace*{\fill}\nolinebreak[1]%
\enspace\hspace*{\fill}
\finalhyphendemerits=0
[6th May]}

\index{Irthing , Carlisle@Irthing, \emph{Carlisle}}

Resignation of Ray Knapton (C).

This by-election was combined with the 2010 ordinary election.
%; see page \pageref{IrthingCarlisle} for the result.

\subsubsection*{Stanwix Urban \hspace*{\fill}\nolinebreak[1]%
\enspace\hspace*{\fill}
\finalhyphendemerits=0
[16th September]}

\index{Stanwix Urban , Carlisle@Stanwix Urban, \emph{Carlisle}}

Resignation of John Stevenson (C).

\noindent
\begin{tabular*}{\columnwidth}{@{\extracolsep{\fill}} p{0.545\columnwidth} >{\itshape}l r @{\extracolsep{\fill}}}
Paul Nedved & C & 888\\
Jackie Franklin & Lab & 488\\
Hazel Bowmaker & Grn & 96\\
Adam Pearson & EDP & 85\\
\end{tabular*}

\subsection{South Lakeland}
\index{South Lakeland}

\subsubsection*{Staveley-in-Cartmel \hspace*{\fill}\nolinebreak[1]%
\enspace\hspace*{\fill}
\finalhyphendemerits=0
[6th May]}

\index{Staveley-in-Cartmel , South Lakeland@Staveley-in-Cartmel, \emph{S. Lakeland}}

Resignation of Clive Leal (LD).

This by-election was combined with the 2010 ordinary election.
%; see page \pageref{StaveleyCartmelSLakeland} for the result.

\subsubsection*{Lyth Valley \hspace*{\fill}\nolinebreak[1]%
\enspace\hspace*{\fill}
\finalhyphendemerits=0
[4th November; C gain from LD]}

\index{Lyth Valley , South Lakeland@Lyth Valley, \emph{S. Lakeland}}

Resignation of Rosie Ballantyne-Smith (LD).

\noindent
\begin{tabular*}{\columnwidth}{@{\extracolsep{\fill}} p{0.545\columnwidth} >{\itshape}l r @{\extracolsep{\fill}}}
John Holmes & C & 474\\
Jane Hall & LD & 451\\
Marilyn Molloy & Lab & 32\\
\end{tabular*}

\section{Derbyshire}

\subsection{Chesterfield}
\index{Chesterfield}

\subsubsection*{Brimington South \hspace*{\fill}\nolinebreak[1]%
\enspace\hspace*{\fill}
\finalhyphendemerits=0
[6th May; Lab gain from LD]}

\index{Brimington South , Chesterfield@Brimington S., \emph{Chesterfield}}

Resignation of Helen Walsh (LD).

\noindent
\begin{tabular*}{\columnwidth}{@{\extracolsep{\fill}} p{0.545\columnwidth} >{\itshape}l r @{\extracolsep{\fill}}}
John Haywood & Lab & 1500\\
Stephen Hartley & LD & 1134\\
Gary Hatton & C & 499\\
\end{tabular*}

\subsection{Derby}
\index{Derby}

At the May 2010 ordinary election there was an unfilled vacancy in Abbey ward due to the resignation of Deirdre Mitchell (LD).
\index{Abbey , Derby@Abbey, \emph{Derby}}

%There was an unfilled vacancy in Abbey ward at the ordinary May elections following the resignation of Deirdre Mitchell (LD) with less than six months of her term remaining.

\subsection{Erewash}
\index{Erewash}

\subsubsection*{West Hallam and Dale Abbey \hspace*{\fill}\nolinebreak[1]%
\enspace\hspace*{\fill}
\finalhyphendemerits=0
[21st January]}

\index{West Hallam and Dale Abbey , Erewash@West Hallam \& Dale Abbey, \emph{Erewash}}

Resignation of John Fildes (C).

\noindent
\begin{tabular*}{\columnwidth}{@{\extracolsep{\fill}} p{0.545\columnwidth} >{\itshape}l r @{\extracolsep{\fill}}}
Bruce Broughton & C & 692\\
Gary Hamson & LD & 506\\
James Dawson & Lab & 149\\
\end{tabular*}

\subsection{High Peak}
\index{High Peak}

\subsubsection*{New Mills West \hspace*{\fill}\nolinebreak[1]%
\enspace\hspace*{\fill}
\finalhyphendemerits=0
[6th May]}

\index{New Mills West , High Peak@New Mills W., \emph{High Peak}}

Resignation of Stephen Sharp (LD).

\noindent
\begin{tabular*}{\columnwidth}{@{\extracolsep{\fill}} p{0.545\columnwidth} >{\itshape}l r @{\extracolsep{\fill}}}
Janet Carter & LD & 819\\
Jacqueline Gadd & C & 647\\
Alan Barrow & Lab & 504\\
Lancelot Dawson & Ind & 235\\
Hazel Body & Grn & 218\\
\end{tabular*}

\subsection{North East Derbyshire}
\index{North East Derbyshire}

\subsubsection*{Holmewood and Heath \hspace*{\fill}\nolinebreak[1]%
\enspace\hspace*{\fill}
\finalhyphendemerits=0
[4th February; Lab gain from LD]}

\index{Holmewood and Heath , North East Derbyshire@Holmewood \& Heath, \emph{N.E. Derbys.}}

Resignation of Jan Robinson (LD).

\noindent
\begin{tabular*}{\columnwidth}{@{\extracolsep{\fill}} p{0.545\columnwidth} >{\itshape}l r @{\extracolsep{\fill}}}
Lee Stone & Lab & 373\\
Derek Jason & C & 209\\
\end{tabular*}

\subsubsection*{Gosforth Valley \hspace*{\fill}\nolinebreak[1]%
\enspace\hspace*{\fill}
\finalhyphendemerits=0
[23rd September; C gain from LD]}

\index{Gosforth Valley , North East Derbyshire@Gosforth Valley, \emph{N.E. Derbys}}

Resignation of Walter Pryce (LD).

\noindent
\begin{tabular*}{\columnwidth}{@{\extracolsep{\fill}} p{0.545\columnwidth} >{\itshape}l r @{\extracolsep{\fill}}}
John McGrory & C & 416\\
Robert Gachagan & Lab & 354\\
Wendy Temple & LD & 350\\
\end{tabular*}

\section{Devon}

\subsection{East Devon}
\index{East Devon}

\subsubsection*{Seaton \hspace*{\fill}\nolinebreak[1]%
\enspace\hspace*{\fill}
\finalhyphendemerits=0
[6th May]}

\index{Seaton , East Devon@Seaton, \emph{E, Devon}}

Resignation of Margaret Rogers (LD).

\noindent
\begin{tabular*}{\columnwidth}{@{\extracolsep{\fill}} p{0.545\columnwidth} >{\itshape}l r @{\extracolsep{\fill}}}
Peter Burrows & LD & 2414\\
John Meakin & C & 1795\\
\end{tabular*}

\subsection{Exeter}
\index{Exeter}

For the results of the twelve by-elections held on 9th September, following the disqualification of the councillors elected in May 2006 after the High Court ruled that their terms had expired, and the further by-election in Topsham on that day following the resignation of Mark Starling (C), see
\url{https://www.andrewteale.me.uk/u172}.
%pages \pageref{ExeterStart} to \pageref{ExeterEnd}.

\subsubsection*{Pennsylvania \hspace*{\fill}\nolinebreak[1]%
\enspace\hspace*{\fill}
\finalhyphendemerits=0
[6th May]}

\index{Pennsylvania , Exeter@Pennsylvania, \emph{Exeter}}

Resignation of Alexander Bond (C).

\noindent
\begin{tabular*}{\columnwidth}{@{\extracolsep{\fill}} p{0.545\columnwidth} >{\itshape}l r @{\extracolsep{\fill}}}
	Tim Payne & LD & 1391\\
	David Thompson & C & 913\\
	Bernard Dugdale & Lab & 491\\
	David Smith & UKIP & 152\\
	Isaac Price-Sosner & Grn & 88\\
\end{tabular*}

Due to Exeter's failed conversion to unitary status this by-election ended up being the ordinary election in Pennsylvania for the term 2010--14.

\subsection{Mid Devon}
\index{Mid Devon}

\subsubsection*{Yeo \hspace*{\fill}\nolinebreak[1]%
\enspace\hspace*{\fill}
\finalhyphendemerits=0
[25th February; C gain from LD]}

\index{Yeo , Mid Devon@Yeo, \emph{Mid Devon}}

Death of David Pullen (LD).

\noindent
\begin{tabular*}{\columnwidth}{@{\extracolsep{\fill}} p{0.545\columnwidth} >{\itshape}l r @{\extracolsep{\fill}}}
Derek Coren & C & 714\\
Peter Heal & LD & 489\\
\end{tabular*}

\subsection{Plymouth}
\index{Plymouth}

\subsubsection*{Efford and Lipson \hspace*{\fill}\nolinebreak[1]%
\enspace\hspace*{\fill}
\finalhyphendemerits=0
[6th May]}

\index{Efford and Lipson , Plymouth@Efford \& Lipson, \emph{Plymouth}}

Resignation of Bernard Miller (Lab).

This by-election was combined with the 2010 ordinary election.
%; see page \pageref{EffordLipsonPlymouth} for the result.

\subsection{South Hams}
\index{South Hams}

\subsubsection*{Ivybridge Filham \hspace*{\fill}\nolinebreak[1]%
\enspace\hspace*{\fill}
\finalhyphendemerits=0
[18th February; LD gain from C]}

\index{Ivybridge Filham , South Hams@Ivybridge Filham, \emph{S. Hams}}

Resignation of Terry Hewitt (C).

\noindent
\begin{tabular*}{\columnwidth}{@{\extracolsep{\fill}} p{0.545\columnwidth} >{\itshape}l r @{\extracolsep{\fill}}}
Tony Barber & LD & 379\\
Alan Wright & C & 356\\
Helen Eassom & Lab & 121\\
\end{tabular*}

\subsection{Teignbridge}
\index{Teignbridge}

\subsubsection*{Ipplepen \hspace*{\fill}\nolinebreak[1]%
\enspace\hspace*{\fill}
\finalhyphendemerits=0
[23rd September; LD gain from C]}

\index{Ipplepen , Teignbridge@Ipplepen, \emph{Teignbridge}}

Death of Victor Elliott (C).

\noindent
\begin{tabular*}{\columnwidth}{@{\extracolsep{\fill}} p{0.545\columnwidth} >{\itshape}l r @{\extracolsep{\fill}}}
Alistair Dewhirst & LD & 756\\
Phil Coombes & C & 458\\
\end{tabular*}

\subsection{Torbay}
\index{Torbay}

\subsubsection*{St Mary's-with-Summercombe \hspace*{\fill}\nolinebreak[1]%
\enspace\hspace*{\fill}
\finalhyphendemerits=0
[22nd July; LD gain from C]}

\index{Saint Mary's-with-Summercombe , Torbay@St Mary's-with-Summercombe, \emph{Torbay}}

Death of Stuart John (C).

\noindent
\begin{tabular*}{\columnwidth}{@{\extracolsep{\fill}} p{0.545\columnwidth} >{\itshape}l r @{\extracolsep{\fill}}}
Andrew Baldrey & LD & 801\\
Nicholas Henderson & C & 365\\
Rosemary Clarke & Lab & 195\\
Jen Walsh & UKIP & 159\\
\end{tabular*}

\section{Dorset}

\subsection{East Dorset}
\index{East Dorset}

\subsubsection*{Corfe Mullen South \hspace*{\fill}\nolinebreak[1]%
\enspace\hspace*{\fill}
\finalhyphendemerits=0
[15th July]}

\index{Corfe Mullen South , East Dorset@Corfe Mullen S., \emph{E. Dorset}}

Resignation of Stewart Hearn (LD).

\noindent
\begin{tabular*}{\columnwidth}{@{\extracolsep{\fill}} p{0.545\columnwidth} >{\itshape}l r @{\extracolsep{\fill}}}
Phil Harknett & LD & 478\\
Brian Lane & C & 350\\
Josephine Evans & UKIP & 34\\
\end{tabular*}

\subsection{Poole}
\index{Poole}

Poole = Poole People

\subsubsection*{Broadstone \hspace*{\fill}\nolinebreak[1]%
\enspace\hspace*{\fill}
\finalhyphendemerits=0
[6th May]}

\index{Broadstone , Poole@Broadstone, \emph{Poole}}

Resignation of Daniel Martin (LD).

\noindent
\begin{tabular*}{\columnwidth}{@{\extracolsep{\fill}} p{0.545\columnwidth} >{\itshape}l r @{\extracolsep{\fill}}}
Roy Godfrey & LD & 3018\\
Ruth Bessant & C & 2971\\
Jack Edwards & UKIP & 295\\
John Hewinson & Lab & 277\\
\end{tabular*}

\subsubsection*{Newtown \hspace*{\fill}\nolinebreak[1]%
\enspace\hspace*{\fill}
\finalhyphendemerits=0
[9th September]}

\index{Newtown , Poole@Newtown, \emph{Poole}}

Resignation of Michael Plummer (LD).

\noindent
\begin{tabular*}{\columnwidth}{@{\extracolsep{\fill}} p{0.545\columnwidth} >{\itshape}l r @{\extracolsep{\fill}}}
Jo Clements & LD & 809\\
Tony Reeves & C & 481\\
Jason Sanderson & Lab & 205\\
Diana Butler & UKIP & 114\\
William Kimmet & BNP & 66\\
\end{tabular*}

\subsubsection*{Poole Town \hspace*{\fill}\nolinebreak[1]%
\enspace\hspace*{\fill}
\finalhyphendemerits=0
[2nd December; Poole gain from C]}

\index{Poole Town , Poole@Poole Town, \emph{Poole}}

Death of Brian Leverett (C).

\noindent
\begin{tabular*}{\columnwidth}{@{\extracolsep{\fill}} p{0.545\columnwidth} >{\itshape}l r @{\extracolsep{\fill}}}
Mark Howell & Poole & 463\\
Anthony Reeves & C & 438\\
Peter England & LD & 213\\
Jason Sanderson & Lab & 201\\
Avril King & UKIP & 55\\
William Kimmet & BNP & 32\\
\end{tabular*}

\section{County Durham}

\subsection{Darlington}
\index{Darlington}

\subsubsection*{Cockerton West \hspace*{\fill}\nolinebreak[1]%
\enspace\hspace*{\fill}
\finalhyphendemerits=0
[8th July]}

\index{Cockerton West , Darlington@Cockerton W., \emph{Darlington}}

Resignation of Jenny Chapman (Lab).

\noindent
\begin{tabular*}{\columnwidth}{@{\extracolsep{\fill}} p{0.545\columnwidth} >{\itshape}l r @{\extracolsep{\fill}}}
Jan Cossins & Lab & 388\\
Brian Jefferson & LD & 347\\
David Davies & C & 84\\
Paul Thompson & BNP & 41\\
\end{tabular*}

\subsection{Durham}
\index{Durham}

\subsubsection*{Easington \hspace*{\fill}\nolinebreak[1]%
\enspace\hspace*{\fill}
\finalhyphendemerits=0
[11th February]}

\index{Easington , Durham@Easington, \emph{Durham}}

Death of Richard Burnip (Lab).

\noindent
\begin{tabular*}{\columnwidth}{@{\extracolsep{\fill}} p{0.545\columnwidth} >{\itshape}l r @{\extracolsep{\fill}}}
Alan Barker & Lab & 702\\
Terry Murray & Ind & 311\\
Carole Harrison & LD & 126\\
Margaret Reid & C & 120\\
\end{tabular*}

\subsubsection*{Brandon \hspace*{\fill}\nolinebreak[1]%
\enspace\hspace*{\fill}
\finalhyphendemerits=0
[30th September]}

\index{Brandon , Durham@Brandon, \emph{Durham}}

Death of Ronnie Rodgers (Lab).

\noindent
\begin{tabular*}{\columnwidth}{@{\extracolsep{\fill}} p{0.545\columnwidth} >{\itshape}l r @{\extracolsep{\fill}}}
John Turnbull & Lab & 1204\\
Maureen Smith & LD & 538\\
Mark Krajewski & C & 140\\
\end{tabular*}

\subsubsection*{Deneside \hspace*{\fill}\nolinebreak[1]%
\enspace\hspace*{\fill}
\finalhyphendemerits=0
[14th October]}

\index{Deneside , Durham@Deneside, \emph{Durham}}

Death of Albert Nugent (Lab).

\noindent
\begin{tabular*}{\columnwidth}{@{\extracolsep{\fill}} p{0.545\columnwidth} >{\itshape}l r @{\extracolsep{\fill}}}
Jennifer Bell & Lab & 917\\
Margaret Reid & C & 196\\
\end{tabular*}

%\subsection{Stockton-on-Tees}
%
%IBIS = Ingleby Barwick Independent Society
%
%Thornaby = Thornaby Independent Association
%
%\subsubsection*{Billingham East \hspace*{\fill}\nolinebreak[1]%
%\enspace\hspace*{\fill}
%\finalhyphendemerits=0
%[tba]}
%
%\index{Billingham East , Stockton-on-Tees@Billingham E., Stockton-on-Tees}
%
%Resignation of Alex Cunningham (Lab).
%
%\noindent
%\begin{tabular*}{\columnwidth}{@{\extracolsep{\fill}} p{0.545\columnwidth} >{\itshape}l r @{\extracolsep{\fill}}}
%.
%\end{tabular*}
%
%\subsubsection*{Ingleby Barwick East \hspace*{\fill}\nolinebreak[1]%
%\enspace\hspace*{\fill}
%\finalhyphendemerits=0
%[tba]}
%
%\index{Ingleby Barwick East , Stockton-on-Tees@Ingleby Barwick E., Stockton-on-Tees}
%
%Resignation of Andrew Larkin (IBIS).
%
%\noindent
%\begin{tabular*}{\columnwidth}{@{\extracolsep{\fill}} p{0.545\columnwidth} >{\itshape}l r @{\extracolsep{\fill}}}
%.
%\end{tabular*}
%
%\subsubsection*{Mandale and Victoria \hspace*{\fill}\nolinebreak[1]%
%\enspace\hspace*{\fill}
%\finalhyphendemerits=0
%[tba]}
%
%\index{Mandale and Victoria , Stockton-on-Tees@Mandale \& Victoria, Stockton-on-Tees}
%
%Death of Allison Trainer (Thornaby).
%
%\noindent
%\begin{tabular*}{\columnwidth}{@{\extracolsep{\fill}} p{0.545\columnwidth} >{\itshape}l r @{\extracolsep{\fill}}}
%.
%\end{tabular*}

\section{East Sussex}

\subsection{Brighton and Hove}
\index{Brighton and Hove}

\subsubsection*{St Peter's and North Laine  \hspace*{\fill}\nolinebreak[1]%
\enspace\hspace*{\fill}
\finalhyphendemerits=0
[8th July]}

\index{Saint Peter's and North Laine , Brighton and Hove@St Peter's \& N. Laine, \emph{Brighton \& Hove}}

Resignation of Keith Taylor (Grn).

\noindent
\begin{tabular*}{\columnwidth}{@{\extracolsep{\fill}} p{0.545\columnwidth} >{\itshape}l r @{\extracolsep{\fill}}}
Lizzie Deane & Grn & 1816\\
Tom French & Lab & 880\\
Rob Buckwell & C & 365\\
Trefor Hunter & LD & 103\\
Gerald O'Brien & Ind & 32\\
\end{tabular*}

\subsection{Hastings}
\index{Hastings}

\subsubsection*{Hollington \hspace*{\fill}\nolinebreak[1]%
\enspace\hspace*{\fill}
\finalhyphendemerits=0
[6th May]}

\index{Hollington , Hastings@Hollington, \emph{Hastings}}

Resignation of Terry Soan (Lab).

This by-election was combined with the 2010 ordinary election.
%; see page \pageref{HollingtonHastings} for the result.

\subsubsection*{Ore \hspace*{\fill}\nolinebreak[1]%
\enspace\hspace*{\fill}
\finalhyphendemerits=0
[17th June; Lab gain from C]}

\index{Ore , Hastings@Ore, \emph{Hastings}}

Ordinary election postponed from 6th May; death of outgoing councillor Roy Tucker (C) who had been nominated for re-election.

\noindent
\begin{tabular*}{\columnwidth}{@{\extracolsep{\fill}} p{0.545\columnwidth} >{\itshape}l r @{\extracolsep{\fill}}}
Michael Wincott & Lab & 608\\
Stuart Padget & C & 475\\
Anne Scott & LD & 158\\
Nick Prince & BNP & 33\\
\end{tabular*}

\subsection{Wealden}
\index{Wealden}

\subsubsection*{Heathfield North and Central \hspace*{\fill}\nolinebreak[1]%
\enspace\hspace*{\fill}
\finalhyphendemerits=0
[23rd September]}

\index{Heathfield North and Central , Wealden@Heathfield N. \& C., \emph{Wealden}}

Death of Niki Oakes (C).

\noindent
\begin{tabular*}{\columnwidth}{@{\extracolsep{\fill}} p{0.545\columnwidth} >{\itshape}l r @{\extracolsep{\fill}}}
Peter Newnham & C & 802\\
Jim Benson & LD & 357\\
\end{tabular*}

\section{East Yorkshire}

\subsection{East Riding}
\index{East Riding}

\subsubsection*{Willerby and Kirk Ella \hspace*{\fill}\nolinebreak[1]%
\enspace\hspace*{\fill}
\finalhyphendemerits=0
[6th May; LD gain from C]}

\index{Willerby and Kirk Ella , East Riding@Willerby \& Kirk Ella, \emph{E. Riding}}

Resignation of Kerry Hemming-Tayler (C).

\noindent
\begin{tabular*}{\columnwidth}{@{\extracolsep{\fill}} p{0.545\columnwidth} >{\itshape}l r @{\extracolsep{\fill}}}
Fred Smith & LD & 5107\\
Josh Newlove & Lab & 1298\\
\end{tabular*}

\subsection{Kingston upon Hull}
\index{Kingston upon Hull}

\subsubsection*{Bransholme West \hspace*{\fill}\nolinebreak[1]%
\enspace\hspace*{\fill}
\finalhyphendemerits=0
[6th May]}

\index{Bransholme West , Kingston upon Hull@Bransholme W., \emph{Kingston upon Hull}}

Resignation of Gordon Wilson (Lab).

This by-election was combined with the 2010 ordinary election.
%; see page \pageref{BransholmeWHull} for the result.

\subsubsection*{Bricknell \hspace*{\fill}\nolinebreak[1]%
\enspace\hspace*{\fill}
\finalhyphendemerits=0
[6th May]}

\index{Bricknell , Kingston upon Hull@Bricknell, \emph{Kingston upon Hull}}

Resignation of Andrew Percy (C).

This by-election was combined with the 2010 ordinary election.
%; see page \pageref{BricknellHull} for the result.

\section{Essex}

\subsection{Basildon}
\index{Basildon}

\subsubsection*{Nethermayne \hspace*{\fill}\nolinebreak[1]%
\enspace\hspace*{\fill}
\finalhyphendemerits=0
[22nd July]}

\index{Nethermayne , Basildon@Nethermayne, \emph{Basildon}}

Resignation of Ben Williams (LD).

\noindent
\begin{tabular*}{\columnwidth}{@{\extracolsep{\fill}} p{0.545\columnwidth} >{\itshape}l r @{\extracolsep{\fill}}}
Phil Jenkins & LD & 605\\
David Kirkman & Lab & 461\\
Stephen Foster & C & 372\\
Kerry Smith & UKIP & 280\\
Irene Bateman & BNP & 70\\
Jason Richardson & Ind & 18\\
\end{tabular*}

\subsection{Braintree}
\index{Braintree}

\subsubsection*{Braintree South \hspace*{\fill}\nolinebreak[1]%
\enspace\hspace*{\fill}
\finalhyphendemerits=0
[24th June]}

\index{Braintree South , Braintree@Braintree S., \emph{Braintree}}

Disqualification (non-attendance) of Russell Wilkins (C).

\noindent
\begin{tabular*}{\columnwidth}{@{\extracolsep{\fill}} p{0.545\columnwidth} >{\itshape}l r @{\extracolsep{\fill}}}
Abi Olumbori & C & 351\\
Gordon Currie & Lab & 316\\
David Toombs & LD & 216\\
Andy Beatty & Ind & 138\\
Wendy Partridge & Grn & 44\\
\end{tabular*}

\subsection{Brentwood}
\index{Brentwood}

\subsubsection*{Pilgrims Hatch \hspace*{\fill}\nolinebreak[1]%
\enspace\hspace*{\fill}
\finalhyphendemerits=0
[6th May]}

\index{Pilgrims Hatch , Brentwood@Pilgrims Hatch, \emph{Brentwood}}

Resignation of Mathew Aspinell (LD).

This by-election was combined with the 2010 ordinary election.
%; see page \pageref{PilgrimsHatchBrentwood} for the result.

\subsection{Epping Forest}
\index{Epping Forest}

\subsubsection*{\sloppyword{Chipping Ongar, Greensted and Marden Ash} \hspace*{\fill}\nolinebreak[1]%
\enspace\hspace*{\fill}
\finalhyphendemerits=0
[6th May]}

\index{Chipping Ongar, Greensted and Marden Ash , Epping Forest@Chipping Ongar, Greensted \& Marden Ash, \emph{Epping Forest}}

Resignation of Glyn Pritchard (C).

This by-election was combined with the 2010 ordinary election.
%; see page \pageref{ChippingOngarGreenstedMardenAshEppingForest} for the result.

\subsection{Rochford}
\index{Rochford}

EDP = English Democrats Party

\subsubsection*{Wheatley \hspace*{\fill}\nolinebreak[1]%
\enspace\hspace*{\fill}
\finalhyphendemerits=0
[15th July]}

\index{Wheatley , Rochford@Wheatley, \emph{Rochford}}

Death of John Pullen (C).

\noindent
\begin{tabular*}{\columnwidth}{@{\extracolsep{\fill}} p{0.545\columnwidth} >{\itshape}l r @{\extracolsep{\fill}}}
Aron Priest & C & 417\\
John Hayter & EDP & 142\\
Sid Cumberland & LD & 78\\
David Bodimeade & Lab & 47\\
\end{tabular*}

\subsection{Southend-on-Sea}
\index{Southend-on-Sea}

At the May 2010 ordinary election there was an unfilled vacancy in Milton ward due to the death of Ann Robertson (C).
\index{Milton , Southend-on-Sea@Milton, \emph{Southend-on-Sea}}

%There was an unfilled vacancy in Milton ward at the ordinary May elections following the death of Ann Robertson (C) with less than six months of her term remaining.

\subsection{Tendring}
\index{Tendring}

Tendring = Tendring First

\subsubsection*{Golf Green \hspace*{\fill}\nolinebreak[1]%
\enspace\hspace*{\fill}
\finalhyphendemerits=0
[8th April; Lab gain from Tendring]}

\index{Golf Green , Tendring@Golf Green, \emph{Tendring}}

Death of Ray Smith (Tendring).

\noindent
\begin{tabular*}{\columnwidth}{@{\extracolsep{\fill}} p{0.53\columnwidth} >{\itshape}l r @{\extracolsep{\fill}}}
Dan Casey & Lab & 409\\
John Chittock & C & 404\\
Louise Stanley & Tendring & 313\\
Keith Beaumont & BNP & 139\\
Christopher Judd & Ind & 120\\
John Candler & LD & 63\\
\end{tabular*}

\section{Gloucestershire}

%\subsection{County Council}
%
%\subsubsection*{Rodborough \hspace*{\fill}\nolinebreak[1]%
%\enspace\hspace*{\fill}
%\finalhyphendemerits=0
%[tba]}
%
%\index{Rodborough , Gloucestershire@Rodborough, Glos.}
%
%Death of Stephen Glanfield (C).
%
%\noindent
%\begin{tabular*}{\columnwidth}{@{\extracolsep{\fill}} p{0.545\columnwidth} >{\itshape}l r @{\extracolsep{\fill}}}
%.
%\end{tabular*}

\subsection{Cheltenham}
\index{Cheltenham}

\subsubsection*{Oakley \hspace*{\fill}\nolinebreak[1]%
\enspace\hspace*{\fill}
\finalhyphendemerits=0
[6th May]}

\index{Oakley , Cheltenham@Oakley, \emph{Cheltenham}}

Resignation of Martin Dunne (LD).

This by-election was combined with the 2010 ordinary election.
%; see page \pageref{OakleyCheltenham} for the result.

\subsubsection*{Springbank \hspace*{\fill}\nolinebreak[1]%
\enspace\hspace*{\fill}
\finalhyphendemerits=0
[6th May]}

\index{Springbank , Cheltenham@Springbank, \emph{Cheltenham}}

Resignation of Simon Wheeler (LD).

This by-election was combined with the 2010 ordinary election.
%; see page \pageref{SpringbankCheltenham} for the result.

\subsubsection*{Springbank \hspace*{\fill}\nolinebreak[1]%
\enspace\hspace*{\fill}
\finalhyphendemerits=0
[28th October]}

\index{Springbank , Cheltenham@Springbank, \emph{Cheltenham}}

Death of John Morris (LD).

\noindent
\begin{tabular*}{\columnwidth}{@{\extracolsep{\fill}} p{0.545\columnwidth} >{\itshape}l r @{\extracolsep{\fill}}}
Christopher Coleman & LD & 722\\
Mireille Weller & C & 188\\
Clive Harriss & Lab & 142\\
Jon Stubbings & Grn & 35\\
\end{tabular*}

\subsection{Gloucester}
\index{Gloucester}

At the May 2010 ordinary election there was an unfilled vacancy in Barton and Tredworth ward due to the death of Yakub Pandor (C).
\index{Barton and Tredworth , Gloucester@Barton \& Tredworth, \emph{Gloucester}}

\subsection{Stroud}
\index{Stroud}

\subsubsection*{Berkeley \hspace*{\fill}\nolinebreak[1]%
\enspace\hspace*{\fill}
\finalhyphendemerits=0
[6th May]}

\index{Berkeley , Stroud@Berkeley, \emph{Stroud}}

Resignation of Tim Archer (C).

This by-election was combined with the 2010 ordinary election.
%; see page \pageref{BerkeleyStroud} for the result.

%\subsubsection*{Amberley and Woodchester \hspace*{\fill}\nolinebreak[1]%
%\enspace\hspace*{\fill}
%\finalhyphendemerits=0
%[tba]}
%
%\index{Amberley and Woodchester , Stroud@Amberley \& Woodchester, Stroud}
%
%Death of Stephen Glanfield (C).
%
%\noindent
%\begin{tabular*}{\columnwidth}{@{\extracolsep{\fill}} p{0.545\columnwidth} >{\itshape}l r @{\extracolsep{\fill}}}
%.
%\end{tabular*}

\section{Hampshire}

\subsection{County Council}
\index{Hampshire}

\subsubsection*{Andover South \hspace*{\fill}\nolinebreak[1]%
\enspace\hspace*{\fill}
\finalhyphendemerits=0
[21st October]}

\index{Andover South , Hampshire@Andover S., \emph{Hants.}}

Death of David Kirk (C).

\noindent
\begin{tabular*}{\columnwidth}{@{\extracolsep{\fill}} p{0.545\columnwidth} >{\itshape}l r @{\extracolsep{\fill}}}
David Drew & C & 1183\\
Leonard Gates & LD & 1111\\
John Newland & Lab & 245\\
Anthony McCabe & UKIP & 233\\
\end{tabular*}

\subsection{Fareham}
\index{Fareham}

\subsubsection*{Fareham West \hspace*{\fill}\nolinebreak[1]%
\enspace\hspace*{\fill}
\finalhyphendemerits=0
[9th December; LD gain from C]}

\index{Fareham West , Fareham@Fareham W., \emph{Fareham}}

Resignation of Diana Harrison (C).

\noindent
\begin{tabular*}{\columnwidth}{@{\extracolsep{\fill}} p{0.545\columnwidth} >{\itshape}l r @{\extracolsep{\fill}}}
Nick Gregory & LD & 933\\
Stephen Day & C & 687\\
Michael Prior & Lab & 124\\
Steve Richards & UKIP & 93\\
Peter Doggett & Grn & 35\\
\end{tabular*}

\subsection{Gosport}
\index{Gosport}

At the May 2010 ordinary election there was an unfilled vacancy in Christchurch ward due to the death of Jackie Carr (LD).
\index{Christchurch , Gosport@Christchurch, \emph{Gosport}}

%There was an unfilled vacancy in Christchurch ward at the ordinary May elections following the death of Jackie Carr (LD) with less than six months of his term remaining.

\subsection{Havant}
\index{Havant}

\subsubsection*{Hayling East \hspace*{\fill}\nolinebreak[1]%
\enspace\hspace*{\fill}
\finalhyphendemerits=0
[6th May]}

\index{Hayling East , Havant@Hayling E., \emph{Havant}}

Resignation of Sheila Pearce (C).

This by-election was combined with the 2010 ordinary election.
%; see page \pageref{HaylingEHavant} for the result.

%\subsection{New Forest}
%
%\subsubsection*{Marchwood \hspace*{\fill}\nolinebreak[1]%
%\enspace\hspace*{\fill}
%\finalhyphendemerits=0
%[tba]}
%
%\index{Marchwood , New Forest@Marchwood, New Forest}
%
%Death of Alan Shotter (C).
%
%\noindent
%\begin{tabular*}{\columnwidth}{@{\extracolsep{\fill}} p{0.545\columnwidth} >{\itshape}l r @{\extracolsep{\fill}}}
%.
%\end{tabular*}

\subsection{Rushmoor}
\index{Rushmoor}

\subsubsection*{Wellington \hspace*{\fill}\nolinebreak[1]%
\enspace\hspace*{\fill}
\finalhyphendemerits=0
[23rd September]}

\index{Wellington , Rushmoor@Wellington, \emph{Rushmoor}}

Resignation of Francis Williams (C).

\noindent
\begin{tabular*}{\columnwidth}{@{\extracolsep{\fill}} p{0.545\columnwidth} >{\itshape}l r @{\extracolsep{\fill}}}
Attika Choudhary & C & 270\\
Mitch Manning & LD & 238\\
Sam Wines & Lab & 184\\
Eddie Poole & UKIP & 50\\
Roger Watkins & Ind & 12\\
\end{tabular*}

\subsection{Test Valley}
\index{Test Valley}

\subsubsection*{Anna \hspace*{\fill}\nolinebreak[1]%
\enspace\hspace*{\fill}
\finalhyphendemerits=0
[6th May]}

\index{Anna , Test Valley@Anna, \emph{Test Valley}}

Resignation of Arthur Peters (C).

\noindent
\begin{tabular*}{\columnwidth}{@{\extracolsep{\fill}} p{0.545\columnwidth} >{\itshape}l r @{\extracolsep{\fill}}}
Maureen Flood & C & 1804\\
Anthony Evans & LD & 836\\
Anthony McCabe & UKIP & 202\\
\end{tabular*}

\subsection{Winchester}
\index{Winchester}

\subsubsection*{St Paul \hspace*{\fill}\nolinebreak[1]%
\enspace\hspace*{\fill}
\finalhyphendemerits=0
[14th October]}

\index{Saint Paul , Winchester@St Paul, \emph{Winchester}}

Resignation of Karen Barratt (LD).

\noindent
\begin{tabular*}{\columnwidth}{@{\extracolsep{\fill}} p{0.545\columnwidth} >{\itshape}l r @{\extracolsep{\fill}}}
Robert Hutchison & LD & 968\\
Helen Osborne & C & 606\\
Nigel Fox & Lab & 247\\
\end{tabular*}

%\subsubsection*{Oliver's Battery and Badger Farm \hspace*{\fill}\nolinebreak[1]%
%\enspace\hspace*{\fill}
%\finalhyphendemerits=0
%[tba]}
%
%\index{Oliver's Battery and Badger Farm , Winchester@Oliver's Battery \& Badger Farm, Winchester}
%
%Death of David Spender (LD).
%
%\noindent
%\begin{tabular*}{\columnwidth}{@{\extracolsep{\fill}} p{0.545\columnwidth} >{\itshape}l r @{\extracolsep{\fill}}}
%.
%\end{tabular*}

\columnbreak

\section{Herefordshire}
\index{Herefordshire}

IOCH = It's Our County (Herefordshire)

\subsubsection*{Hope End \hspace*{\fill}\nolinebreak[1]%
\enspace\hspace*{\fill}
\finalhyphendemerits=0
[6th May]}

\index{Hope End , Herefordshire@Hope End, \emph{Herefs.}}

Resignation of Rees Mills (C).

\noindent
\begin{tabular*}{\columnwidth}{@{\extracolsep{\fill}} p{0.545\columnwidth} >{\itshape}l r @{\extracolsep{\fill}}}
Tony Johnson & C & 1703\\
Barry Ashton & LD & 1651\\
\end{tabular*}

\subsubsection*{Ledbury \hspace*{\fill}\nolinebreak[1]%
\enspace\hspace*{\fill}
\finalhyphendemerits=0
[6th May]}

\index{Ledbury , Herefordshire@Ledbury, \emph{Herefs.}}

Resignation of Kay Swinburne (C).

\noindent
\begin{tabular*}{\columnwidth}{@{\extracolsep{\fill}} p{0.545\columnwidth} >{\itshape}l r @{\extracolsep{\fill}}}
Phil Bettington & C & 2438\\
Michael Gogan & LD & 1769\\
Robert Wilson & Ind & 713\\
\end{tabular*}

\subsubsection*{St Nicholas \hspace*{\fill}\nolinebreak[1]%
\enspace\hspace*{\fill}
\finalhyphendemerits=0
[14th October; IOCH gain from Ind]}

\index{Saint Nicholas , Herefordshire@St Nicholas, \emph{Herefs.}}

Death of David Benjamin (Ind).

\noindent
\begin{tabular*}{\columnwidth}{@{\extracolsep{\fill}} p{0.545\columnwidth} >{\itshape}l r @{\extracolsep{\fill}}}
Justin Lavender & IOCH & 589\\
Anthony Murphy & LD & 385\\
Colin Mears & Ind & 173\\
Vivienne Jones & C & 156\\
\end{tabular*}

\section{Hertfordshire}

\subsection{County Council}
\index{Hertfordshire}

\subsubsection*{St Albans South \hspace*{\fill}\nolinebreak[1]%
\enspace\hspace*{\fill}
\finalhyphendemerits=0
[3rd June]}

\index{Saint Albans South , Hertfordshire@St Albans S., \emph{Herts.}}

Death of Mike Ellis (LD).

\noindent
\begin{tabular*}{\columnwidth}{@{\extracolsep{\fill}} p{0.545\columnwidth} >{\itshape}l r @{\extracolsep{\fill}}}
Martin Frearson & LD & 1482\\
Salih Gaygusuz & C & 1250\\
Iain Grant & Lab & 540\\
Kate Metcalf & Ind & 249\\
\end{tabular*}

\subsection{Dacorum}
\index{Dacorum}

\subsubsection*{Adeyfield West \hspace*{\fill}\nolinebreak[1]%
\enspace\hspace*{\fill}
\finalhyphendemerits=0
[11th March]}

\index{Adeyfield West , Dacorum@Adeyfield W., \emph{Dacorum}}

Death of Keith Reid (C).

\noindent
\begin{tabular*}{\columnwidth}{@{\extracolsep{\fill}} p{0.545\columnwidth} >{\itshape}l r @{\extracolsep{\fill}}}
Dan Wood & C & 486\\
Sue White & Lab & 429\\
Steve Wilson & LD & 362\\
Janet Price & BNP & 203\\
\end{tabular*}

\subsubsection*{Aldbury and Wigginton \hspace*{\fill}\nolinebreak[1]%
\enspace\hspace*{\fill}
\finalhyphendemerits=0
[26th August; LD gain from C]}

\index{Aldbury and Wigginton , Dacorum@Aldbury \& Wigginton, \emph{Dacorum}}

Disqualification (non-attendance) of Mike Edwards (C).

\noindent
\begin{tabular*}{\columnwidth}{@{\extracolsep{\fill}} p{0.545\columnwidth} >{\itshape}l r @{\extracolsep{\fill}}}
Rosemarie Hollinghurst & LD & 593\\
Paul Richardson & C & 305\\
Thomas Wright & Lab & 18\\
\end{tabular*}

\subsection{East Hertfordshire}
\index{East Hertfordshire}

\subsubsection*{Hunsdon \hspace*{\fill}\nolinebreak[1]%
\enspace\hspace*{\fill}
\finalhyphendemerits=0
[11th November]}

\index{Hunsdon , East Hertfordshire@Hunsdon, \emph{E. Herts.}}

Resignation of Deborah Clark (Ind).

\noindent
\begin{tabular*}{\columnwidth}{@{\extracolsep{\fill}} p{0.545\columnwidth} >{\itshape}l r @{\extracolsep{\fill}}}
Michael Newman & Ind & 339\\
Geoffrey Williamson & C & 206\\
Linda Harvey & Lab & 31\\
\end{tabular*}

\subsubsection*{Sawbridgeworth \hspace*{\fill}\nolinebreak[1]%
\enspace\hspace*{\fill}
\finalhyphendemerits=0
[23rd December]}

\index{Sawbridgeworth , East Hertfordshire@Sawbridgeworth, \emph{E. Herts.}}

Resignation of Nigel Clark (Ind).

\noindent
\begin{tabular*}{\columnwidth}{@{\extracolsep{\fill}} p{0.545\columnwidth} >{\itshape}l r @{\extracolsep{\fill}}}
Eric Buckmaster & Ind & 441\\
William Mortimer & C & 343\\
Peter Mitchell & Lab & 99\\
Michael Shaw & LD & 95\\
\end{tabular*}

\subsection{North Hertfordshire}
\index{North Hertfordshire}

\subsubsection*{Royston Palace \hspace*{\fill}\nolinebreak[1]%
\enspace\hspace*{\fill}
\finalhyphendemerits=0
[6th May]}

\index{Royston Palace , North Hertfordshire@Royston Palace, \emph{N. Herts}}

Resignation of Liz Beardwell (LD).

This by-election was combined with the 2010 ordinary election.
%; see page \pageref{RoystonPalaceNHerts} for the result.

\subsection{St Albans}
\index{Saint Albans@St Albans}

\subsubsection*{Ashley \hspace*{\fill}\nolinebreak[1]%
\enspace\hspace*{\fill}
\finalhyphendemerits=0
[3rd June]}

\index{Ashley , Saint Albans@Ashley, \emph{St Albans}}

Death of Mike Ellis (LD).

\noindent
\begin{tabular*}{\columnwidth}{@{\extracolsep{\fill}} p{0.545\columnwidth} >{\itshape}l r @{\extracolsep{\fill}}}
Andy Grant & LD & 774\\
John Paton & Lab & 354\\
Christopher Baker & C & 342\\
Graham Ward & Grn & 93\\
\end{tabular*}

\subsection{Watford}
\index{Watford}

\subsubsection*{Central \hspace*{\fill}\nolinebreak[1]%
\enspace\hspace*{\fill}
\finalhyphendemerits=0
[14th October]}

\index{Central , Watford@Central, \emph{Watford}}

Resignation of Sheila Smillie (LD).

\noindent
\begin{tabular*}{\columnwidth}{@{\extracolsep{\fill}} p{0.545\columnwidth} >{\itshape}l r @{\extracolsep{\fill}}}
Helen Lynch & LD & 696\\
Fred Grindrod & Lab & 622\\
Youseffe Fahmy & C & 158\\
Dorothy Nixon & Grn & 79\\
Dan Channing & UKIP & 24\\
\end{tabular*}

\subsection{Welwyn Hatfield}
\index{Welwyn Hatfield}

\subsubsection*{Hatfield Villages \hspace*{\fill}\nolinebreak[1]%
\enspace\hspace*{\fill}
\finalhyphendemerits=0
[6th May]}

\index{Hatfield Villages , Welwyn Hatfield@Hatfield Villages, \emph{Welwyn Hatfield}}

Resignation of Mark Gilding (C).

This by-election was combined with the 2010 ordinary election.
%; see page \pageref{HatfieldVillagesWelwynHatfield} for the result.

\subsubsection*{Northaw and Cuffley \hspace*{\fill}\nolinebreak[1]%
\enspace\hspace*{\fill}
\finalhyphendemerits=0
[6th May]}

\index{Northaw and Cuffley , Welwyn Hatfield@Northaw \& Cuffley, \emph{Welwyn Hatfield}}

Death of John Mansfield (C).

This by-election was combined with the 2010 ordinary election.
%; see page \pageref{NorthawCuffleyWelwynHatfield} for the result.

\section{Isle of Wight}
\index{Isle of Wight}

\subsubsection*{Ryde South \hspace*{\fill}\nolinebreak[1]%
\enspace\hspace*{\fill}
\finalhyphendemerits=0
[27th May; C gain from LD]}

\index{Ryde South , Isle of Wight@Ryde S., \emph{Isle of Wight}}

Disqualification (sentenced to three months in prison, suspended, assaulting a police officer) of Adrian Whittaker (Ind elected as LD).

\noindent
\begin{tabular*}{\columnwidth}{@{\extracolsep{\fill}} p{0.545\columnwidth} >{\itshape}l r @{\extracolsep{\fill}}}
Gary Taylor & C & 274\\
Deborah Gardiner & Lab & 201\\
Tony Zeid & LD & 164\\
Ian Jenkins & Ind & 97\\
\end{tabular*}

\subsubsection*{Chale, Niton and Whitwell \hspace*{\fill}\nolinebreak[1]%
\enspace\hspace*{\fill}
\finalhyphendemerits=0
[11th November]}

\index{Chale, Niton and Whitwell , Isle of Wight@Chale, Niton \& Whitwell, \emph{Isle of Wight}}

Resignation of William Myatt-Willington (C).

\noindent
\begin{tabular*}{\columnwidth}{@{\extracolsep{\fill}} p{0.545\columnwidth} >{\itshape}l r @{\extracolsep{\fill}}}
Dave Stewart & C & 510\\
Malcolm Groves & LD & 365\\
Josh Cooper & Lab & 76\\
\end{tabular*}

\section{Kent}

EDP = English Democrats Party

\subsection{County Council}
\index{Kent}

\subsubsection*{Dover Town \hspace*{\fill}\nolinebreak[1]%
\enspace\hspace*{\fill}
\finalhyphendemerits=0
[16th December; Lab gain from C]}

\index{Dover Town , Kent@Dover Town, \emph{Kent}}

Death of Roger Frayne (C).

\noindent
\begin{tabular*}{\columnwidth}{@{\extracolsep{\fill}} p{0.545\columnwidth} >{\itshape}l r @{\extracolsep{\fill}}}
Gordon Cowan & Lab & 1491\\
Patrick Sherratt & C & 1348\\
Victor Matcham & UKIP & 404\\
John Trickey & LD & 170\\
\end{tabular*}

%\subsubsection*{Romney Marsh \hspace*{\fill}\nolinebreak[1]%
%\enspace\hspace*{\fill}
%\finalhyphendemerits=0
%[tba]}
%
%\index{Romney Marsh , Dover@Romney Marsh, Dover}
%
%Death of Willie Richardson (C).
%
%\noindent
%\begin{tabular*}{\columnwidth}{@{\extracolsep{\fill}} p{0.545\columnwidth} >{\itshape}l r @{\extracolsep{\fill}}}
%.
%\end{tabular*}

%\subsubsection*{Tonbridge \hspace*{\fill}\nolinebreak[1]%
%\enspace\hspace*{\fill}
%\finalhyphendemerits=0
%[20th January]}
%
%\index{Tonbridge , Dover@Tonbridge, Dover}
%
%Death of Godfrey Horne (C).
%
%\noindent
%\begin{tabular*}{\columnwidth}{@{\extracolsep{\fill}} p{0.545\columnwidth} >{\itshape}l r @{\extracolsep{\fill}}}
%.
%\end{tabular*}

\subsection{Dover}
\index{Dover}

\subsubsection*{Lydden and Temple Ewell \hspace*{\fill}\nolinebreak[1]%
\enspace\hspace*{\fill}
\finalhyphendemerits=0
[16th December]}

\index{Lydden and Temple Ewell , Dover@Lydden \& Temple Ewell, \emph{Dover}}

Death of Roger Frayne (C).

\noindent
\begin{tabular*}{\columnwidth}{@{\extracolsep{\fill}} p{0.545\columnwidth} >{\itshape}l r @{\extracolsep{\fill}}}
Geoffrey Lymer & C & 239\\
Peter Walker & Lab & 90\\
Victor Matcham & UKIP & 55\\
\end{tabular*}

\subsection{Gravesham}
\index{Gravesham}

\subsubsection*{Meopham South and Vigo \hspace*{\fill}\nolinebreak[1]%
\enspace\hspace*{\fill}
\finalhyphendemerits=0
[18th March]}

\index{Meopham South and Vigo , Gravesham@Meopham S. \& Vigo, \emph{Gravesham}}

Death of Raymonde Collins (C).

\noindent
\begin{tabular*}{\columnwidth}{@{\extracolsep{\fill}} p{0.545\columnwidth} >{\itshape}l r @{\extracolsep{\fill}}}
Derek Shelbrooke & C & 515\\
Geoffrey Clark & UKIP & 122\\
Douglas Christie & Lab & 114\\
Ann O'Brien & LD & 114\\
\end{tabular*}

\subsection{Medway}
\index{Medway}

\subsubsection*{River \hspace*{\fill}\nolinebreak[1]%
\enspace\hspace*{\fill}
\finalhyphendemerits=0
[12th August; C gain from Lab]}

\index{River , Medway@River, \emph{Medway}}

Resignation of Bill Esterston (Lab).

\noindent
\begin{tabular*}{\columnwidth}{@{\extracolsep{\fill}} p{0.545\columnwidth} >{\itshape}l r @{\extracolsep{\fill}}}
David Craggs & C & 617\\
John Jones & Lab & 544\\
Garry Harrison & LD & 104\\
Steven Keevil & Grn & 45\\
Brian Ravenscroft & BNP & 39\\
Ron Sands & EDP & 33\\
\end{tabular*}

\subsubsection*{River \hspace*{\fill}\nolinebreak[1]%
\enspace\hspace*{\fill}
\finalhyphendemerits=0
[21st October; Lab gain from C]}

\index{River , Medway@River, \emph{Medway}}

Resignation of David Craggs (C).

\noindent
\begin{tabular*}{\columnwidth}{@{\extracolsep{\fill}} p{0.545\columnwidth} >{\itshape}l r @{\extracolsep{\fill}}}
John Jones & Lab & 695\\
Andrew Mackness & C & 631\\
Garry Harrison & LD & 92\\
Anthony Cook & UKIP & 42\\
Steven Keevil & Grn & 36\\
Ron Sands & EDP & 31\\
\end{tabular*}

\subsection{Swale}
\index{Swale}

\subsubsection*{Queenborough and Halfway \hspace*{\fill}\nolinebreak[1]%
\enspace\hspace*{\fill}
\finalhyphendemerits=0
[6th May]}

\index{Queenborough and Halfway , Swale@Queenborough \& Halfway, \emph{Swale}}

Resignation of Paul Hayes (C).

This by-election was combined with the 2010 ordinary election.
%; see page \pageref{QueenboroughHalfwaySwale} for the result.

\subsection{Tunbridge Wells}
\index{Tunbridge Wells}

\subsubsection*{St James' \hspace*{\fill}\nolinebreak[1]%
\enspace\hspace*{\fill}
\finalhyphendemerits=0
[7th October]}

\index{Saint James' , Tunbridge Wells@St James', \emph{Tunbridge Wells}}

Resignation of Mary Lewis (LD).

\noindent
\begin{tabular*}{\columnwidth}{@{\extracolsep{\fill}} p{0.545\columnwidth} >{\itshape}l r @{\extracolsep{\fill}}}
Benjamin Chapelard & LD & 649\\
Robert Rutherford & C & 294\\
Richard Leslie & Grn & 103\\
\end{tabular*}

\subsubsection*{Sherwood \hspace*{\fill}\nolinebreak[1]%
\enspace\hspace*{\fill}
\finalhyphendemerits=0
[16th December]}

\index{Sherwood , Tunbridge Wells@Sherwood, \emph{Tunbridge Wells}}

Death of Ted Jolley (C).

\noindent
\begin{tabular*}{\columnwidth}{@{\extracolsep{\fill}} p{0.545\columnwidth} >{\itshape}l r @{\extracolsep{\fill}}}
Robert Backhouse & C & 422\\
Alan Bullion & LD & 174\\
Ian Carvell & Lab & 124\\
Victor Webb & UKIP & 92\\
Joanna Stanley & EDP & 75\\
\end{tabular*}

\section{Lancashire}

\subsection{County Council}
\index{Lancashire}

\subsubsection*{Burnley Central East \hspace*{\fill}\nolinebreak[1]%
\enspace\hspace*{\fill}
\finalhyphendemerits=0
[6th May; Lab gain from LD]}

\index{Burnley Central East , Lancashire@Burnley C.E., \emph{Lancs.}}

Death of Bill Bennett (LD).

\noindent
\begin{tabular*}{\columnwidth}{@{\extracolsep{\fill}} p{0.545\columnwidth} >{\itshape}l r @{\extracolsep{\fill}}}
Misfar Hassan & Lab & 3157\\
Martin Smith & LD & 2279\\
Paul McDevitt & BNP & 868\\
Matthew Isherwood & C & 815\\
\end{tabular*}

\subsection{Blackburn with Darwen}
\index{Blackburn with Darwen}

\subsubsection*{Queen's Park \hspace*{\fill}\nolinebreak[1]%
\enspace\hspace*{\fill}
\finalhyphendemerits=0
[4th February; Lab gain from LD]}

\index{Queen's Park , Blackburn with Darwen@Queen's Park, \emph{Blackburn with Darwen}}

Resignation of Yusuf Sidat (Ind elected as LD).

\noindent
\begin{tabular*}{\columnwidth}{@{\extracolsep{\fill}} p{0.545\columnwidth} >{\itshape}l r @{\extracolsep{\fill}}}
Mustafa Desai & Lab & 638\\
Imtiaz Patel&LD&366\\
Asghar Ali&C&174\\
\end{tabular*}

%\subsection{Blackpool}
%
%\subsubsection*{Stanley \hspace*{\fill}\nolinebreak[1]%
%\enspace\hspace*{\fill}
%\finalhyphendemerits=0
%[tba]}
%
%\index{Stanley , Blackpool@Stanley, Blackpool}
%
%Death of Jean Kenrick (C).
%
%\noindent
%\begin{tabular*}{\columnwidth}{@{\extracolsep{\fill}} p{0.545\columnwidth} >{\itshape}l r @{\extracolsep{\fill}}}
%.
%\end{tabular*}

\subsection{Burnley}
\index{Burnley}

\subsubsection*{Queensgate \hspace*{\fill}\nolinebreak[1]%
\enspace\hspace*{\fill}
\finalhyphendemerits=0
[6th May]}

\index{Queensgate , Burnley@Queensgate, \emph{Burnley}}

Death of Bill Bennett (LD).

This by-election was combined with the 2010 ordinary election.
%; see page \pageref{QueensgateBurnley} for the result.

\subsection{Fylde}
\index{Fylde}

\subsubsection*{Ribby-with-Wrea \hspace*{\fill}\nolinebreak[1]%
\enspace\hspace*{\fill}
\finalhyphendemerits=0
[1st July; C gain from Ind]}

\index{Ribby-with-Wrea , Fylde@Ribby-with-Wrea, \emph{Fylde}}

Death of Lyndsay Greening (Ind).

\noindent
\begin{tabular*}{\columnwidth}{@{\extracolsep{\fill}} p{0.545\columnwidth} >{\itshape}l r @{\extracolsep{\fill}}}
Frank Andrews & C & 292\\
Janet Wardell & Ind & 256\\
David Hobson & Grn & 11\\
\end{tabular*}

\subsubsection*{Kilnhouse \hspace*{\fill}\nolinebreak[1]%
\enspace\hspace*{\fill}
\finalhyphendemerits=0
[9th September; LD gain from C]}

\index{Kilnhouse , Fylde@Kilnhouse, \emph{Fylde}}

Death of John Prestwich (C).

\noindent
\begin{tabular*}{\columnwidth}{@{\extracolsep{\fill}} p{0.545\columnwidth} >{\itshape}l r @{\extracolsep{\fill}}}
Karen Henshaw & LD & 529\\
Matthew Lardner & C & 459\\
Peter Stephenson & Lab & 165\\
Ian Roberts & Grn & 29\\
\end{tabular*}

\subsection{Hyndburn}
\index{Hyndburn}

\subsubsection*{Peel \hspace*{\fill}\nolinebreak[1]%
\enspace\hspace*{\fill}
\finalhyphendemerits=0
[1st July]}

\index{Peel , Hyndburn@Peel, \emph{Hyndburn}}

Resignation of Graham Jones (Lab).

\noindent
\begin{tabular*}{\columnwidth}{@{\extracolsep{\fill}} p{0.545\columnwidth} >{\itshape}l r @{\extracolsep{\fill}}}
Wendy Dwyer & Lab & 592\\
Daniel Cassidy & C & 189\\
\end{tabular*}

\subsubsection*{Baxenden \hspace*{\fill}\nolinebreak[1]%
\enspace\hspace*{\fill}
\finalhyphendemerits=0
[18th November]}

\index{Baxenden , Hyndburn@Baxenden, \emph{Hyndburn}}

Death of John Griffiths (C).

\noindent
\begin{tabular*}{\columnwidth}{@{\extracolsep{\fill}} p{0.545\columnwidth} >{\itshape}l r @{\extracolsep{\fill}}}
Terence Hurn & C & 693\\
David Hartley & Lab & 434\\
Lesley Wolstencroft & Ind & 47\\
Munzoor Anwar & UKIP & 17\\
\end{tabular*}

\subsection{Lancaster}
\index{Lancaster}

MBI = Morecambe Bay Independents

\subsubsection*{John O'Gaunt \hspace*{\fill}\nolinebreak[1]%
\enspace\hspace*{\fill}
\finalhyphendemerits=0
[1st April]}

\index{John O'Gaunt , Lancaster@John O'Gaunt, \emph{Lancaster}}

Resignation of Jim Blakely (Lab).

\noindent
\begin{tabular*}{\columnwidth}{@{\extracolsep{\fill}} p{0.545\columnwidth} >{\itshape}l r @{\extracolsep{\fill}}}
Elizabeth Scott & Lab & 603\\
Harry Armistead & LD & 389\\
Ian Chamberlain & Grn & 339\\
Billy Hill & C & 301\\
Fred McGlade & UKIP & 83\\
\end{tabular*}

\subsubsection*{Skerton West \hspace*{\fill}\nolinebreak[1]%
\enspace\hspace*{\fill}
\finalhyphendemerits=0
[6th May]}

\index{Skerton West , Lancaster@Skerton W., \emph{Lancaster}}

Resignation of Robert Smith (Lab).

\noindent
\begin{tabular*}{\columnwidth}{@{\extracolsep{\fill}} p{0.545\columnwidth} >{\itshape}l r @{\extracolsep{\fill}}}
John Harrison & Lab & 1127\\
Richard Rollins & C & 731\\
David Taylor & LD & 535\\
Paul Andrews & Grn & 184\\
\end{tabular*}

\subsubsection*{Harbour \hspace*{\fill}\nolinebreak[1]%
\enspace\hspace*{\fill}
\finalhyphendemerits=0
[7th October]}

\index{Harbour , Lancaster@Harbour, \emph{Lancaster}}

Death of John Barnes (MBI).

\noindent
\begin{tabular*}{\columnwidth}{@{\extracolsep{\fill}} p{0.545\columnwidth} >{\itshape}l r @{\extracolsep{\fill}}}
Geoff Walker & MBI & 287\\
Elliot Layfield & C & 161\\
Fred McGlade & UKIP & 86\\
Ian Clift & LD & 68\\
\end{tabular*}

\subsection{Pendle}
\index{Pendle}

At the May 2010 ordinary election there was an unfilled vacancy in Brierfield ward due to the resignation of Naseem Shabnam (LD).
\index{Brierfield , Pendle@Brierfield, \emph{Pendle}}

%There was an unfilled vacancy in Brierfield ward at the ordinary May elections following the resignation of Naseem Shabnam (LD) with less than six months of her term remaining.

\subsection{Preston}
\index{Preston}

\subsubsection*{Riversway \hspace*{\fill}\nolinebreak[1]%
\enspace\hspace*{\fill}
\finalhyphendemerits=0
[6th May]}

\index{Riversway , Preston@Riversway, \emph{Preston}}

Resignation of Jack Davenport (Lab).

This by-election was combined with the 2010 ordinary election.
%; see page \pageref{RiverswayPreston} for the result.

\subsubsection*{Riversway \hspace*{\fill}\nolinebreak[1]%
\enspace\hspace*{\fill}
\finalhyphendemerits=0
[15th July; Lab gain from LD]}

\index{Riversway , Preston@Riversway, \emph{Preston}}

Resignation of Líam Pennington (LD).

\noindent
\begin{tabular*}{\columnwidth}{@{\extracolsep{\fill}} p{0.545\columnwidth} >{\itshape}l r @{\extracolsep{\fill}}}
Linda Crompton & Lab & 890\\
Stephen Wilkinson & LD & 388\\
Adam Vardey & Grn & 56\\
\end{tabular*}

\subsubsection*{Cadley \hspace*{\fill}\nolinebreak[1]%
\enspace\hspace*{\fill}
\finalhyphendemerits=0
[16th September]}

\index{Cadley , Preston@Cadley, \emph{Preston}}

Resignation of Mike Onyon (LD).

\noindent
\begin{tabular*}{\columnwidth}{@{\extracolsep{\fill}} p{0.545\columnwidth} >{\itshape}l r @{\extracolsep{\fill}}}
John Potter & LD & 721\\
John Young & Lab & 476\\
David Walker & C & 465\\
\end{tabular*}

\subsection{Rossendale}
\index{Rossendale}

CFirst = Community First

\subsubsection*{Healey and Whitworth \hspace*{\fill}\nolinebreak[1]%
\enspace\hspace*{\fill}
\finalhyphendemerits=0
[7th October]}

\index{Healey and Whitworth , Rossendale@Healey \& Whitworth, \emph{Rossendale}}

Resignation of Roger Wilson (Lab).

\noindent
\begin{tabular*}{\columnwidth}{@{\extracolsep{\fill}} p{0.545\columnwidth} >{\itshape}l r @{\extracolsep{\fill}}}
Sean Serridge & Lab & 346\\
David Bradbury & CFirst & 165\\
Marie Gibbons & C & 156\\
Clive Laight & LD & 10\\
\end{tabular*}

\subsubsection*{Irwell \hspace*{\fill}\nolinebreak[1]%
\enspace\hspace*{\fill}
\finalhyphendemerits=0
[7th October; Lab gain from C]}

\index{Irwell , Rossendale@Irwell, \emph{Rossendale}}

Resignation of Michael Christie (C).

\noindent
\begin{tabular*}{\columnwidth}{@{\extracolsep{\fill}} p{0.545\columnwidth} >{\itshape}l r @{\extracolsep{\fill}}}
Helen Jackson & Lab & 277\\
Elizabeth Heath & C & 220\\
Bill Jackson & LD & 183\\
Tina Durkin & CFirst & 64\\
\end{tabular*}

\subsection{South Ribble}
\index{South Ribble}

\subsubsection*{Moss Side \hspace*{\fill}\nolinebreak[1]%
\enspace\hspace*{\fill}
\finalhyphendemerits=0
[6th May]}

\index{Moss Side , South Ribble@Moss Side, \emph{S. Ribble}}

Resignation of Frank Duxbury (C).

\noindent
\begin{tabular*}{\columnwidth}{@{\extracolsep{\fill}} p{0.545\columnwidth} >{\itshape}l r @{\extracolsep{\fill}}}
Caroline Moon & C & 1088\\
Ken Jones & Lab & 731\\
\end{tabular*}

\subsection{West Lancashire}
\index{West Lancashire}

\subsubsection*{Scott \hspace*{\fill}\nolinebreak[1]%
\enspace\hspace*{\fill}
\finalhyphendemerits=0
[6th May]}

\index{Scott , West Lancashire@Scott, \emph{W. Lancs.}}

Resignation of Geoff Hammond (C).

This by-election was combined with the 2010 ordinary election.
%; see page \pageref{ScottWLancs} for the result.

\subsubsection*{Skelmersdale South \hspace*{\fill}\nolinebreak[1]%
\enspace\hspace*{\fill}
\finalhyphendemerits=0
[2nd September]}

\index{Skelmersdale South , West Lancashire@Skelmersdale S., \emph{W. Lancs.}}

Death of Doreen Saxon (Lab).

\noindent
\begin{tabular*}{\columnwidth}{@{\extracolsep{\fill}} p{0.545\columnwidth} >{\itshape}l r @{\extracolsep{\fill}}}
Nicola Roberts & Lab & 644\\
Sarah Ainscough & C & 122\\
Mike Brennan & UKIP & 105\\
\end{tabular*}

\subsubsection*{Upholland \hspace*{\fill}\nolinebreak[1]%
\enspace\hspace*{\fill}
\finalhyphendemerits=0
[2nd September]}

\index{Upholland , West Lancashire@Upholland, \emph{W. Lancs.}}

Death of Terry Rice (Lab).

\noindent
\begin{tabular*}{\columnwidth}{@{\extracolsep{\fill}} p{0.545\columnwidth} >{\itshape}l r @{\extracolsep{\fill}}}
John Fillis & Lab & 749\\
David Sudworth & C & 597\\
Gary McNulty & UKIP & 118\\
\end{tabular*}

%\subsubsection*{Derby \hspace*{\fill}\nolinebreak[1]%
%\enspace\hspace*{\fill}
%\finalhyphendemerits=0
%[tba]}
%
%\index{Derby , West Lancashire@Derby, W. Lancs.}
%
%Death of David Swiffen (C).
%
%\noindent
%\begin{tabular*}{\columnwidth}{@{\extracolsep{\fill}} p{0.545\columnwidth} >{\itshape}l r @{\extracolsep{\fill}}}
%.
%\end{tabular*}

\subsection{Wyre}
\index{Wyre}

\subsubsection*{Tithebarn \hspace*{\fill}\nolinebreak[1]%
\enspace\hspace*{\fill}
\finalhyphendemerits=0
[7th October]}

\index{Tithebarn , Wyre@Tithebarn, \emph{Wyre}}

Death of Peter Hawley (C).

\noindent
\begin{tabular*}{\columnwidth}{@{\extracolsep{\fill}} p{0.545\columnwidth} >{\itshape}l r @{\extracolsep{\fill}}}
Lesley McKay & C & 847\\
Alan Morgan & Lab & 307\\
\end{tabular*}

\section{Leicestershire}

\subsection{Blaby}
\index{Blaby}

\subsubsection*{Saxondale \hspace*{\fill}\nolinebreak[1]%
\enspace\hspace*{\fill}
\finalhyphendemerits=0
[7th October]}

\index{Saxondale , Blaby@Saxondale, \emph{Blaby}}

Death of Alexandra Dilks (LD).

\noindent
\begin{tabular*}{\columnwidth}{@{\extracolsep{\fill}} p{0.545\columnwidth} >{\itshape}l r @{\extracolsep{\fill}}}
Christine Merrill & LD & 701\\
Nadina Kalsi & C & 311\\
Alan Methven & Lab & 243\\
Gary Reynolds & BNP & 94\\
\end{tabular*}

\subsubsection*{Blaby South \hspace*{\fill}\nolinebreak[1]%
\enspace\hspace*{\fill}
\finalhyphendemerits=0
[25th November]}

\index{Blaby South , Blaby@Blaby S., \emph{Blaby}}

Resignation of David Pollard (LD).

\noindent
\begin{tabular*}{\columnwidth}{@{\extracolsep{\fill}} p{0.545\columnwidth} >{\itshape}l r @{\extracolsep{\fill}}}
Antony Moseley & LD & 381\\
Marian Broomhead & C & 264\\
Inga Windley & Lab & 153\\
Peter Jarvis & BNP & 68\\
\end{tabular*}

\subsection{Charnwood}
\index{Charnwood}

\subsubsection*{Birstall Watermead \hspace*{\fill}\nolinebreak[1]%
\enspace\hspace*{\fill}
\finalhyphendemerits=0
[18th February]}

\index{Birstall Watermead , Charnwood@Birstall Watermead, \emph{Charnwood}}

Resignation of Rick Astill (C).

\noindent
\begin{tabular*}{\columnwidth}{@{\extracolsep{\fill}} p{0.545\columnwidth} >{\itshape}l r @{\extracolsep{\fill}}}
Iain Bentley & C & 674\\
Hayley Winrow & Lab & 452\\
Maurice Oatley & BNP & 288\\
\end{tabular*}

\subsection{Harborough}
\index{Harborough}

\subsubsection*{Market Harborough---Great Bowden and Arden \hspace*{\fill}\nolinebreak[1]%
\enspace\hspace*{\fill}
\finalhyphendemerits=0
[28th January; LD gain from C]}

\index{Market Harborough Great Bowden and Arden , Harborough@Market Harborough---Great Bowden \& Arden, \emph{Harborough}}

Resignation of Alistair Swatridge (C).

\noindent
\begin{tabular*}{\columnwidth}{@{\extracolsep{\fill}} p{0.545\columnwidth} >{\itshape}l r @{\extracolsep{\fill}}}
Phil Knowles & LD & 966\\
Barry Champion & C & 598\\
\end{tabular*}

\subsection{Leicester}
\index{Leicester}

EDP = English Democrats Party

\subsubsection*{Eyres Monsell \hspace*{\fill}\nolinebreak[1]%
\enspace\hspace*{\fill}
\finalhyphendemerits=0
[6th May]}

\index{Eyres Monsell , Leicester@Eyres Monsell, \emph{Leicester}}

Resignation of Kim Blower (Ind elected as Lab).

\noindent
\begin{tabular*}{\columnwidth}{@{\extracolsep{\fill}} p{0.545\columnwidth} >{\itshape}l r @{\extracolsep{\fill}}}
Virginia Cleaver & Lab & 1446\\
Jon Humberstone & C & 999\\
Scott Kennedy-Lount & LD & 908\\
Adrian Waudby & BNP & 745\\
\end{tabular*}

\subsubsection*{Castle \hspace*{\fill}\nolinebreak[1]%
\enspace\hspace*{\fill}
\finalhyphendemerits=0
[15th July; Lab gain from Grn]}

\index{Castle , Leicester@Castle, \emph{Leicester}}

Death of Phil Gordon (Grn).

\noindent
\begin{tabular*}{\columnwidth}{@{\extracolsep{\fill}} p{0.545\columnwidth} >{\itshape}l r @{\extracolsep{\fill}}}
Neil Clayton & Lab & 766\\
Mo Taylor & Grn & 625\\
Owen Jones & C & 411\\
Troy Lavers & LD & 150\\
Roy Rudham & EDP & 33\\
David Bowley & Ind & 11\\
Gareth Henry & Lib & 11\\
\end{tabular*}

\section{Lincolnshire}

\subsection{East Lindsey}
\index{East Lindsey}

\subsubsection*{North Holme \hspace*{\fill}\nolinebreak[1]%
\enspace\hspace*{\fill}
\finalhyphendemerits=0
[17th June; Lab gain from LD]}

\index{North Holme , East Lindsey@North Holme, \emph{E. Lindsey}}

Death of Clive Finch (LD).

\noindent
\begin{tabular*}{\columnwidth}{@{\extracolsep{\fill}} p{0.545\columnwidth} >{\itshape}l r @{\extracolsep{\fill}}}
Philip Sturman & Lab & 142\\
Fran Treanor & C & 111\\
Jez Boothman & BNP & 102\\
Mary Finch & LD & 88\\
Susan Locking & Ind & 72\\
\end{tabular*}

\subsection{Lincoln}
\index{Lincoln}

\subsubsection*{Moorland \hspace*{\fill}\nolinebreak[1]%
\enspace\hspace*{\fill}
\finalhyphendemerits=0
[6th May]}

\index{Moorland , Lincoln@Moorland, \emph{Lincoln}}

Resignation of Oliver Peake (C).

This by-election was combined with the 2010 ordinary election.
%; see page \pageref{MoorlandLincoln} for the result.

\subsection{North Kesteven}
\index{North Kesteven}

\subsubsection*{Branston \hspace*{\fill}\nolinebreak[1]%
\enspace\hspace*{\fill}
\finalhyphendemerits=0
[6th May]}

\index{Branston , North Kesteven@Branston, \emph{N. Kesteven}}

Death of Marjorie Duncan (C).

\noindent
\begin{tabular*}{\columnwidth}{@{\extracolsep{\fill}} p{0.545\columnwidth} >{\itshape}l r @{\extracolsep{\fill}}}
Raymond Cucksey & C & 1309\\
Peter Lundgren & Ind & 1202\\
Michael Clayton & BNP & 325\\
\end{tabular*}

\section{Norfolk}

%\subsection{County Council}
%
%\subsubsection*{Humbleyard \hspace*{\fill}\nolinebreak[1]%
%\enspace\hspace*{\fill}
%\finalhyphendemerits=0
%[13th January]}
%
%\index{Humbleyard , Norfolk@Humbleyard, Norfolk}
%
%Resignation of Daniel Cox (C).
%
%\noindent
%\begin{tabular*}{\columnwidth}{@{\extracolsep{\fill}} p{0.545\columnwidth} >{\itshape}l r @{\extracolsep{\fill}}}
%?
%\end{tabular*}

\subsection{Broadland}
\index{Broadland}

\subsubsection*{Taverham North \hspace*{\fill}\nolinebreak[1]%
\enspace\hspace*{\fill}
\finalhyphendemerits=0
[18th March; LD gain from C]}

\index{Taverham North , Broadland@Taverham N., \emph{Broadland}}

Resignation of Heath Dicks (C).

\noindent
\begin{tabular*}{\columnwidth}{@{\extracolsep{\fill}} p{0.545\columnwidth} >{\itshape}l r @{\extracolsep{\fill}}}
Nich Starling & LD & 630\\
John Griffin & C & 471\\
Jennifer Parkhouse & Grn & 54\\
\end{tabular*}

\subsection{Great Yarmouth}
\index{Great Yarmouth}

\subsubsection*{Claydon \hspace*{\fill}\nolinebreak[1]%
\enspace\hspace*{\fill}
\finalhyphendemerits=0
[6th May]}

\index{Claydon , Great Yarmouth@Claydon, \emph{Great Yarmouth}}

Death of Richard Barker (Lab).

This by-election was combined with the 2010 ordinary election.
%; see page \pageref{ClaydonGreatYarmouth} for the result.

\columnbreak

\subsection{King's Lynn and West Norfolk}
\index{King's Lynn and West Norfolk}

\subsubsection*{Mershe Lande \hspace*{\fill}\nolinebreak[1]%
\enspace\hspace*{\fill}
\finalhyphendemerits=0
[6th May; C gain from Ind]}

\index{Mershe Lande , King's Lynn and West Norfolk@Mershe Lande, \emph{King's Lynn \& W. Norfolk}}

Resignation of David Markinson (Ind).

\noindent
\begin{tabular*}{\columnwidth}{@{\extracolsep{\fill}} p{0.545\columnwidth} >{\itshape}l r @{\extracolsep{\fill}}}
Jonathan Johns & C & 548\\
Carol Coleman & Ind & 372\\
John Nicholas-Letch & LD & 212\\
John Collop & Lab & 173\\
\end{tabular*}

\subsection{North Norfolk}
\index{North Norfolk}

\subsubsection*{Lancaster South \hspace*{\fill}\nolinebreak[1]%
\enspace\hspace*{\fill}
\finalhyphendemerits=0
[6th May]}

\index{Lancaster South , North Norfolk@Lancaster S., \emph{N. Norfolk}}

Resignation of Stephanie Towers (LD).

\noindent
\begin{tabular*}{\columnwidth}{@{\extracolsep{\fill}} p{0.545\columnwidth} >{\itshape}l r @{\extracolsep{\fill}}}
John Lisher & LD & 1066\\
Thomas Fitzpatrick & C & 784\\
Monika Wiedmann & Grn & 122\\
\end{tabular*}

\subsection{Norwich}
\index{Norwich}

For the results of the thirteen by-elections held on 9th September, following the disqualification of the councillors elected in 2006 after the High Court ruled that their terms had expired, see
\url{https://www.andrewteale.me.uk/u279}.
%pages \pageref{NorwichStart} to \pageref{NorwichEnd}.

\section{Northamptonshire}

\subsection{Daventry}
\index{Daventry}

\subsubsection*{Abbey North \hspace*{\fill}\nolinebreak[1]%
\enspace\hspace*{\fill}
\finalhyphendemerits=0
[6th May]}

\index{Abbey North , Daventry@Abbey N., \emph{Daventry}}

Resignation of Barry Howard (C).

This by-election was combined with the 2010 ordinary election.
%; see page \pageref{AbbeyNorthDaventry} for the result.

\subsection{Wellingborough}
\index{Wellingborough}

EDP = English Democrats Party

\subsubsection*{Redwell West \hspace*{\fill}\nolinebreak[1]%
\enspace\hspace*{\fill}
\finalhyphendemerits=0
[11th March]}

\index{Redwell West , Wellingborough@Redwell W., \emph{Wellingborough}}

Resignation of Andrew Bigley (C).

\noindent
\begin{tabular*}{\columnwidth}{@{\extracolsep{\fill}} p{0.545\columnwidth} >{\itshape}l r @{\extracolsep{\fill}}}
John Raymond & C & 570\\
Graham Sherwood & Lab & 169\\
David Robinson & BNP & 84\\
Penelope Wilkins & LD & 72\\
Terence Spencer & EDP & 62\\
Jonathan Hornett & Grn & 23\\
\end{tabular*}

\subsubsection*{Earls Barton \hspace*{\fill}\nolinebreak[1]%
\enspace\hspace*{\fill}
\finalhyphendemerits=0
[17th June; C gain from Ind]}

\index{Earls Barton , Wellingborough@Earls Barton, \emph{Wellingborough}}

Disqualification (non-attendance) of Lucy Payne (Ind).

\noindent
\begin{tabular*}{\columnwidth}{@{\extracolsep{\fill}} p{0.545\columnwidth} >{\itshape}l r @{\extracolsep{\fill}}}
Andrew Atkinson & C & 639\\
Peter Wright & Lab & 425\\
Daniel Jones & LD & 229\\
Jonathan Hornett & Grn & 38\\
\end{tabular*}

\section{Northumberland}
\index{Northumberland}

\subsubsection*{Ponteland East \hspace*{\fill}\nolinebreak[1]%
\enspace\hspace*{\fill}
\finalhyphendemerits=0
[4th November]}

\index{Ponteland East , Northumberland@Ponteland E., \emph{Northd.}}

Death of Mel Armstrong (C).

\noindent
\begin{tabular*}{\columnwidth}{@{\extracolsep{\fill}} p{0.545\columnwidth} >{\itshape}l r @{\extracolsep{\fill}}}
Eileen Armstrong & C & 843\\
Andrew Duffield & LD & 403\\
Andrew Avery & Lab & 100\\
\end{tabular*}

\section{North Yorkshire}

\subsection{Harrogate}
\index{Harrogate}

\subsubsection*{Woodfield \hspace*{\fill}\nolinebreak[1]%
\enspace\hspace*{\fill}
\finalhyphendemerits=0
[7th January]}

\index{Woodfield , Harrogate@Woodfield, \emph{Harrogate}}

Death of John Wren (C elected as LD).

\noindent
\begin{tabular*}{\columnwidth}{@{\extracolsep{\fill}} p{0.545\columnwidth} >{\itshape}l r @{\extracolsep{\fill}}}
Greta Knight & LD & 688\\
Heather Adderley & C & 246\\
Steven Gill & BNP & 92\\
Daniel Maguire & Lab & 73\\
\end{tabular*}

\subsection{Middlesbrough}
\index{Middlesbrough}

\subsubsection*{Ayresome \hspace*{\fill}\nolinebreak[1]%
\enspace\hspace*{\fill}
\finalhyphendemerits=0
[9th September; Ind gain from Lab]}

\index{Ayresome , Middlesbrough@Ayresome, \emph{Middlesbrough}}

Resignation of Barbara Dunne (Lab).

\noindent
\begin{tabular*}{\columnwidth}{@{\extracolsep{\fill}} p{0.545\columnwidth} >{\itshape}l r @{\extracolsep{\fill}}}
William Hawthorne & Ind & 456\\
Stephen Cass & Lab & 414\\
James Ruddock & C & 73\\
\end{tabular*}

\subsection{Redcar and Cleveland}
\index{Redcar and Cleveland}

\subsubsection*{Brotton \hspace*{\fill}\nolinebreak[1]%
\enspace\hspace*{\fill}
\finalhyphendemerits=0
[26th August]}

\index{Brotton , Redcar and Cleveland@Brotton, \emph{Redcar \& Cleveland}}

Resignation of Peter Scott (Lab).

\noindent
\begin{tabular*}{\columnwidth}{@{\extracolsep{\fill}} p{0.545\columnwidth} >{\itshape}l r @{\extracolsep{\fill}}}
Doreen Rudland & Lab & 565\\
Barry Hunt & Ind & 351\\
Val Miller & LD & 315\\
Donald Agar & C & 220\\
Michael George & BNP & 33\\
\end{tabular*}

\subsection{Ryedale}
\index{Ryedale}

\subsubsection*{Norton West \hspace*{\fill}\nolinebreak[1]%
\enspace\hspace*{\fill}
\finalhyphendemerits=0
[6th May; LD gain from Ind]}

\index{Norton West , Ryedale@Norton W., \emph{Ryedale}}

Death of David Jackson (Ind).

\noindent
\begin{tabular*}{\columnwidth}{@{\extracolsep{\fill}} p{0.545\columnwidth} >{\itshape}l r @{\extracolsep{\fill}}}
Hugh Spencer & LD & 579\\
Judith Denniss & C & 455\\
Trevor Moss & BNP & 107\\
\end{tabular*}

\subsection{Scarborough}
\index{Scarborough}

\subsubsection*{Mayfield \hspace*{\fill}\nolinebreak[1]%
\enspace\hspace*{\fill}
\finalhyphendemerits=0
[11th February]}

\index{Mayfield , Scarborough@Mayfield, \emph{Scarborough}}

Death of Peter Booth (C).

\noindent
\begin{tabular*}{\columnwidth}{@{\extracolsep{\fill}} p{0.545\columnwidth} >{\itshape}l r @{\extracolsep{\fill}}}
Steven Leadley & C & 436\\
Derek Robinson & Ind & 282\\
Janet Peake & Lab & 238\\
Bill Miller & Ind & 112\\
Christopher Cawood & LD & 65\\
\end{tabular*}

\columnbreak

\subsubsection*{Eastfield \hspace*{\fill}\nolinebreak[1]%
\enspace\hspace*{\fill}
\finalhyphendemerits=0
[Friday 24th September]}

\index{Eastfield , Scarborough@Eastfield, \emph{Scarborough}}

Death of Brian O'Flynn (LD).

\noindent
\begin{tabular*}{\columnwidth}{@{\extracolsep{\fill}} p{0.545\columnwidth} >{\itshape}l r @{\extracolsep{\fill}}}
Johan Zegstroo & LD & 182\\
John Warburton & Lab & 159\\
Nicholas Butterworth & Ind & 110\\
Carl Morris & C & 100\\
Trisha Scott & BNP & 27\\
Michael Cutler & Grn & 13\\
\end{tabular*}

\subsection{York}
\index{York}

\subsubsection*{Hull Road \hspace*{\fill}\nolinebreak[1]%
\enspace\hspace*{\fill}
\finalhyphendemerits=0
[14th October]}

\index{Hull Road , York@Hull Rd., \emph{York}}

Resignation of Roger Pierce (Lab).

\noindent
\begin{tabular*}{\columnwidth}{@{\extracolsep{\fill}} p{0.545\columnwidth} >{\itshape}l r @{\extracolsep{\fill}}}
Mick Hoban & Lab & 860\\
Robin Dickson & C & 296\\
Rachel Williams & LD & 183\\
John Cossham & Grn & 84\\
Trevor Brown & BNP & 42\\
\end{tabular*}

\section{Nottinghamshire}

MIF = Mansfield Independent Forum

\subsection{County Council}
\index{Nottinghamshire}

\subsubsection*{Mansfield South \hspace*{\fill}\nolinebreak[1]%
\enspace\hspace*{\fill}
\finalhyphendemerits=0
[25th February; Lab gain from MIF]}

\index{Mansfield South , Nottinghamshire@Mansfield S., \emph{Notts.}}

Death of Tom Appleby (MIF).

\noindent
\begin{tabular*}{\columnwidth}{@{\extracolsep{\fill}} p{0.545\columnwidth} >{\itshape}l r @{\extracolsep{\fill}}}
Chris Winterton & Lab & 1342\\
Andrew Tristram & MIF & 1108\\
Drew Stafford & C & 774\\
Nick Bennett & UKIP & 489\\
Danielle Gent & LD & 295\\
\end{tabular*}

\subsubsection*{Worksop West \hspace*{\fill}\nolinebreak[1]%
\enspace\hspace*{\fill}
\finalhyphendemerits=0
[16th September; Lab gain from C]}

\index{Worksop West , Nottinghamshire@Worksop W., \emph{Notts.}}

Death of Michael Bennett (C).

\noindent
\begin{tabular*}{\columnwidth}{@{\extracolsep{\fill}} p{0.545\columnwidth} >{\itshape}l r @{\extracolsep{\fill}}}
Kevin Greaves & Lab & 1457\\
Alec Thorpe & C & 755\\
Leon Duveen & LD & 88\\
Ronald Dawes & Ind & 56\\
Terry Coleman & Ind & 13\\
\end{tabular*}

\subsection{Ashfield}
\index{Ashfield}

\subsubsection*{Hucknall Central \hspace*{\fill}\nolinebreak[1]%
\enspace\hspace*{\fill}
\finalhyphendemerits=0
[11th February; Lab gain from C]}

\index{Hucknall Central , Ashfield@Hucknall C., \emph{Ashfield}}

Death of Gordon Riley (C).

\noindent
\begin{tabular*}{\columnwidth}{@{\extracolsep{\fill}} p{0.545\columnwidth} >{\itshape}l r @{\extracolsep{\fill}}}
Trevor Locke & Lab & 675\\
Mick Murphy & C & 437\\
Ken Cotham & LD & 357\\
Ron Nixon & UKIP & 158\\
Edward Holmes & BNP & 131\\
\end{tabular*}

\subsection{Bassetlaw}
\index{Bassetlaw}

At the May 2010 ordinary election there was an unfilled vacancy in Worksop North ward due to the resignation of Steven Burton (C).
\index{Worksop North , Bassetlaw@Worksop N., \emph{Bassetlaw}}

%There was an unfilled vacancy in Worksop North ward at the ordinary May elections following the resignation of Steven Burton (C) with less than six months of his term remaining.

\subsubsection*{Worksop South \hspace*{\fill}\nolinebreak[1]%
\enspace\hspace*{\fill}
\finalhyphendemerits=0
[16th September; Lab gain from C]}

\index{Worksop South , Bassetlaw@Worksop S., \emph{Bassetlaw}}

Death of Michael Bennett (C).

\noindent
\begin{tabular*}{\columnwidth}{@{\extracolsep{\fill}} p{0.545\columnwidth} >{\itshape}l r @{\extracolsep{\fill}}}
Sylvia May & Lab & 815\\
Alex Thorpe & C & 669\\
Leon Duveen & LD & 84\\
\end{tabular*}

\subsubsection*{Harworth \hspace*{\fill}\nolinebreak[1]%
\enspace\hspace*{\fill}
\finalhyphendemerits=0
[21st October]}

\index{Harworth , Bassetlaw@Harworth, \emph{Bassetlaw}}

Resignation of John Clayton (Lab).

\noindent
\begin{tabular*}{\columnwidth}{@{\extracolsep{\fill}} p{0.545\columnwidth} >{\itshape}l r @{\extracolsep{\fill}}}
Gloria Evans & Lab & 1345\\
Tracey Taylor & C & 182\\
Richard Bennett & Ind & 68\\
Mark Hunter & LD & 39\\
\end{tabular*}

\subsection{Broxtowe}
\index{Broxtowe}

\subsubsection*{Eastwood South \hspace*{\fill}\nolinebreak[1]%
\enspace\hspace*{\fill}
\finalhyphendemerits=0
[25th February; LD gain from Lab]}

\index{Eastwood South , Broxtowe@Eastwood S., \emph{Broxtowe}}

Death of Jim Kenny (Lab).

\noindent
\begin{tabular*}{\columnwidth}{@{\extracolsep{\fill}} p{0.545\columnwidth} >{\itshape}l r @{\extracolsep{\fill}}}
Keith Longdon & LD & 985\\
Ken Woodhead & Lab & 484\\
Adrian Limb & C & 387\\
\end{tabular*}

\subsubsection*{Cossall and Kimberley \hspace*{\fill}\nolinebreak[1]%
\enspace\hspace*{\fill}
\finalhyphendemerits=0
[19th August]}

\index{Cossall and Kimberley , Broxtowe@Cossall \& Kimberley, \emph{Broxtowe}}

Disqualification (non-attendance) of Tony Rutherford (C).

\noindent
\begin{tabular*}{\columnwidth}{@{\extracolsep{\fill}} p{0.545\columnwidth} >{\itshape}l r @{\extracolsep{\fill}}}
Shane Easom & C & 757\\
Mary McGuckin & Lab & 731\\
Elaine Cockburn & LD & 524\\
\end{tabular*}

\subsection{Mansfield}
\index{Mansfield}

\subsubsection*{Portland \hspace*{\fill}\nolinebreak[1]%
\enspace\hspace*{\fill}
\finalhyphendemerits=0
[25th February; Lab gain from MIF]}

\index{Portland , Mansfield@Portland, \emph{Mansfield}}

Death of Tom Appleby (MIF).

\noindent
\begin{tabular*}{\columnwidth}{@{\extracolsep{\fill}} p{0.545\columnwidth} >{\itshape}l r @{\extracolsep{\fill}}}
Brian Lohan & Lab & 407\\
Deirdre Appleby & MIF & 263\\
Fraser McFarland & C & 116\\
Nicholas Spencer & LD & 66\\
Philip Smith & UKIP & 37\\
\end{tabular*}

\subsubsection*{Sherwood \hspace*{\fill}\nolinebreak[1]%
\enspace\hspace*{\fill}
\finalhyphendemerits=0
[25th February; Lab gain from MIF]}

\index{Sherwood , Mansfield@Sherwood, \emph{Mansfield}}

Death of Norman Cook (MIF).

\noindent
\begin{tabular*}{\columnwidth}{@{\extracolsep{\fill}} p{0.545\columnwidth} >{\itshape}l r @{\extracolsep{\fill}}}
Julia Yemm & Lab & 406\\
Eric Milnes & MIF & 171\\
Brian Hyatt & C & 156\\
Andrew Hamilton & UKIP & 93\\
Anna Ellis & LD & 49\\
\end{tabular*}

\subsection{Newark and Sherwood}
\index{Newark and Sherwood}

\subsubsection*{Rainworth \hspace*{\fill}\nolinebreak[1]%
\enspace\hspace*{\fill}
\finalhyphendemerits=0
[1st July]}

\index{Rainworth , Newark and Sherwood@Rainworth, \emph{Newark \& Sherwood}}

Death of Allen Tift (Lab).

\noindent
\begin{tabular*}{\columnwidth}{@{\extracolsep{\fill}} p{0.545\columnwidth} >{\itshape}l r @{\extracolsep{\fill}}}
Linda Tift & Lab & 735\\
Mike Warner & C & 219\\
Alan Armin & Ind & 144\\
\end{tabular*}

\subsection{Rushcliffe}
\index{Rushcliffe}

\subsubsection*{Wiverton \hspace*{\fill}\nolinebreak[1]%
\enspace\hspace*{\fill}
\finalhyphendemerits=0
[6th May; C gain from Ind]}

\index{Wiverton , Rushcliffe@Wiverton, \emph{Rushcliffe}}

Resignation of David Barlow (Ind).

\noindent
\begin{tabular*}{\columnwidth}{@{\extracolsep{\fill}} p{0.545\columnwidth} >{\itshape}l r @{\extracolsep{\fill}}}
Sarah Bailey & C & 1264\\
Sue Hull & LD & 1076\\
\end{tabular*}

\section{Oxfordshire}

\subsection{Cherwell}
\index{Cherwell}

\subsubsection*{Kidlington North \hspace*{\fill}\nolinebreak[1]%
\enspace\hspace*{\fill}
\finalhyphendemerits=0
[22nd July]}

\index{Kidlington North , Cherwell@Kidlington N., \emph{Cherwell}}

Death of John Wyse (LD).

\noindent
\begin{tabular*}{\columnwidth}{@{\extracolsep{\fill}} p{0.545\columnwidth} >{\itshape}l r @{\extracolsep{\fill}}}
Alaric Rose & LD & 526\\
Eddie Stevens & C & 419\\
Catherine Arakelian & Lab & 216\\
David Fairweather & UKIP & 86\\
\end{tabular*}

\subsection{Oxford}
\index{Oxford}

\subsubsection*{Barton and Sandhills \hspace*{\fill}\nolinebreak[1]%
\enspace\hspace*{\fill}
\finalhyphendemerits=0
[21st October; Lab gain from LD]}

\index{Barton and Sandhills , Oxford@Barton \& Sandhills, \emph{Oxford}}

Resignation of Patrick Murray (LD).

\noindent
\begin{tabular*}{\columnwidth}{@{\extracolsep{\fill}} p{0.545\columnwidth} >{\itshape}l r @{\extracolsep{\fill}}}
Michael Rowley & Lab & 837\\
Hugh Baylis & LD & 334\\
Mary-Jane Sareva & Grn & 119\\
Siddharth Deva & C & 86\\
Julia Gasper & UKIP & 48\\
Michael McAndrews & Ind & 42\\
\end{tabular*}

\subsection{South Oxfordshire}
\index{South Oxfordshire}

Henley = Henley Residents Group

\subsubsection*{Henley South \hspace*{\fill}\nolinebreak[1]%
\enspace\hspace*{\fill}
\finalhyphendemerits=0
[11th February]}

\index{Henley South , South Oxfordshire@Henley S., \emph{S. Oxon.}}

Death of Terry Buckett (Henley).

\noindent
\begin{tabular*}{\columnwidth}{@{\extracolsep{\fill}} p{0.545\columnwidth} >{\itshape}l r @{\extracolsep{\fill}}}
Elizabeth Hodgkin & Henley & 642\\
David Silvester & C & 472\\
George Levy & LD & 110\\
\end{tabular*}

\subsubsection*{Crowmarsh \hspace*{\fill}\nolinebreak[1]%
\enspace\hspace*{\fill}
\finalhyphendemerits=0
[3rd June; LD gain from C]}

\index{Crowmarsh , South Oxfordshire@Crowmarsh, \emph{S. Oxon.}}

Death of Nicholas Odd (C).

\noindent
\begin{tabular*}{\columnwidth}{@{\extracolsep{\fill}} p{0.545\columnwidth} >{\itshape}l r @{\extracolsep{\fill}}}
John Griffin & LD & 327\\
Kristina Crabbe & C & 258\\
\end{tabular*}

\subsection{West Oxfordshire}
\index{West Oxfordshire}

At the May 2010 ordinary election there was an unfilled vacancy in Carterton North West ward due to the death of Dave King (C).
\index{Carterton North West , West Oxfordshire@Carterton N.W., \emph{W. Oxon.}}

%There was an unfilled vacancy in Carterton North West ward at the ordinary May elections following the death of Dave King (C) with less than six months of his term remaining.

\section{Shropshire}

\subsection{Shropshire}
\index{Shropshire}

\subsubsection*{Clee \hspace*{\fill}\nolinebreak[1]%
\enspace\hspace*{\fill}
\finalhyphendemerits=0
[1st July]}

\index{Clee , Shropshire@Clee, \emph{Shrops.}}

Resignation of Paul Andrews (C).

\noindent
\begin{tabular*}{\columnwidth}{@{\extracolsep{\fill}} p{0.545\columnwidth} >{\itshape}l r @{\extracolsep{\fill}}}
Richard Huffer & LD & 946\\
Hayley Fernihough & C & 506\\
Graeme Perks & Ind & 116\\
\end{tabular*}

\subsection{Telford and Wrekin}
\index{Telford and Wrekin}

TWPA = Telford and Wrekin Peoples Alliance

\subsubsection*{College \hspace*{\fill}\nolinebreak[1]%
\enspace\hspace*{\fill}
\finalhyphendemerits=0
[11th February; C gain from Ind]}

\index{College , Telford and Wrekin@College, \emph{Telford \& Wrekin}}

Resignation of Gary Davies (Ind).

\noindent
\begin{tabular*}{\columnwidth}{@{\extracolsep{\fill}} p{0.545\columnwidth} >{\itshape}l r @{\extracolsep{\fill}}}
David Chaplin & C & 252\\
Patrick McCarthy & Ind & 237\\
Michael Ion & Lab & 219\\
Nusrat Janjua & Ind & 127\\
Richard Choudhary & TWPA & 24\\
\end{tabular*}

\subsubsection*{The Nedge \hspace*{\fill}\nolinebreak[1]%
\enspace\hspace*{\fill}
\finalhyphendemerits=0
[11th February; C gain from Lab]}

\index{Nedge , Telford and Wrekin@The Nedge, \emph{Telford \& Wrekin}}

Death of Ute Sambrook (Lab).

\noindent
\begin{tabular*}{\columnwidth}{@{\extracolsep{\fill}} p{0.545\columnwidth} >{\itshape}l r @{\extracolsep{\fill}}}
Harvey Unwin & C & 760\\
Shaun Davies & Lab & 688\\
Ray Knight & UKIP & 237\\
Terence Gould & BNP & 103\\
\end{tabular*}

\section{Somerset}

\subsection{Bath and North East Somerset}
\index{Bath and North East Somerset}

\subsubsection*{Radstock \hspace*{\fill}\nolinebreak[1]%
\enspace\hspace*{\fill}
\finalhyphendemerits=0
[29th July; LD gain from Ind]}

\index{Radstock , Bath and North East Somerset@Radstock, \emph{Bath \& N.E. Somerset}}

Death of Allan Hall (Ind).

\noindent
\begin{tabular*}{\columnwidth}{@{\extracolsep{\fill}} p{0.545\columnwidth} >{\itshape}l r @{\extracolsep{\fill}}}
Simon Allen & LD & 542\\
Lesley Mansell & Lab & 386\\
Keith Pate & Ind & 370\\
Deirdre Horstmann & C & 55\\
\end{tabular*}

\subsection{Mendip}
\index{Mendip}

\subsubsection*{Shepton West \hspace*{\fill}\nolinebreak[1]%
\enspace\hspace*{\fill}
\finalhyphendemerits=0
[1st July; LD gain from C]}

\index{Shepton West , Mendip@Shepton W., \emph{Mendip}}

Disqualification (non-attendance) of Ashley Taylor (C).

\noindent
\begin{tabular*}{\columnwidth}{@{\extracolsep{\fill}} p{0.545\columnwidth} >{\itshape}l r @{\extracolsep{\fill}}}
Garfield Kennedy & LD & 459\\
Judy Bartlett & C & 358\\
Chris Inchley & Lab & 241\\
Chris Briton & Grn & 44\\
\end{tabular*}

\subsubsection*{Ashwick, Chilcompton and Stratton \hspace*{\fill}\nolinebreak[1]%
\enspace\hspace*{\fill}
\finalhyphendemerits=0
[21st October]}

\index{Ashwick, Chilcompton and Stratton , Mendip@Ashwick, Chilcompton \& Stratton, \emph{Mendip}}

Resignation of Maxine Oxford (C).

\noindent
\begin{tabular*}{\columnwidth}{@{\extracolsep{\fill}} p{0.545\columnwidth} >{\itshape}l r @{\extracolsep{\fill}}}
Steven Priscott & C & 491\\
Tom Buckley & LD & 440\\
Roger Anderson & Lab & 111\\
Chris Briton & Grn & 46\\
\end{tabular*}

\subsection{Sedgemoor}
\index{Sedgemoor}

\subsubsection*{Woolavington \hspace*{\fill}\nolinebreak[1]%
\enspace\hspace*{\fill}
\finalhyphendemerits=0
[30th September; C gain from Lab]}

\index{Woolavington , Sedgemoor@Woolavington, \emph{Sedgemoor}}

Death of Roger Lavers (Lab).

\noindent
\begin{tabular*}{\columnwidth}{@{\extracolsep{\fill}} p{0.545\columnwidth} >{\itshape}l r @{\extracolsep{\fill}}}
Alison Hamlin & C & 264\\
Tina Marsh & LD & 184\\
Ian Tucker & Lab & 141\\
\end{tabular*}

\subsection{South Somerset}
\index{South Somerset}

\subsubsection*{Tower \hspace*{\fill}\nolinebreak[1]%
\enspace\hspace*{\fill}
\finalhyphendemerits=0
[6th May]}

\index{Tower , South Somerset@Tower, \emph{S. Somerset}}

Resignation of Maili Felton (C).

\noindent
\begin{tabular*}{\columnwidth}{@{\extracolsep{\fill}} p{0.545\columnwidth} >{\itshape}l r @{\extracolsep{\fill}}}
Michael Beech & C & 962\\
Gordon Czaprewski & LD & 620\\
\end{tabular*}

\subsection{Taunton Deane}
\index{Taunton Deane}

\subsubsection*{Lyngford \hspace*{\fill}\nolinebreak[1]%
\enspace\hspace*{\fill}
\finalhyphendemerits=0
[28th January]}

\index{Lyngford , Taunton Deane@Lyngford, \emph{Taunton Deane}}

Disqualification (non-attendance) of Julie Wood (LD).

\noindent
\begin{tabular*}{\columnwidth}{@{\extracolsep{\fill}} p{0.545\columnwidth} >{\itshape}l r @{\extracolsep{\fill}}}
Ben Swaine & LD & 390\\
John Gage & C & 253\\
Martin Jevon & Lab & 190\\
Charlene Sherriff & UKIP & 59\\
\end{tabular*}

\section{Staffordshire}

\subsection{Cannock Chase}
\index{Cannock Chase}

\subsubsection*{Heath Hayes East and Wimblebury \hspace*{\fill}\nolinebreak[1]%
\enspace\hspace*{\fill}
\finalhyphendemerits=0
[26th August; Lab gain from C]}

\index{Heath Hayes East and Wimblebury , Cannock Chase@Heath Hayes E. \& Wimblebury, \emph{Cannock Chase}}

Death of John Jillings (C).

\noindent
\begin{tabular*}{\columnwidth}{@{\extracolsep{\fill}} p{0.545\columnwidth} >{\itshape}l r @{\extracolsep{\fill}}}
Diane Todd & Lab & 418\\
Lisa Pearce & C & 238\\
Andrew Clark & LD & 153\\
Ron Turville & Ind & 30\\
\end{tabular*}

\subsection{East Staffordshire}
\index{East Staffordshire}

\subsubsection*{Bagots \hspace*{\fill}\nolinebreak[1]%
\enspace\hspace*{\fill}
\finalhyphendemerits=0
[6th May]}

\index{Bagots , East Staffordshire@Bagots, \emph{E. Staffs.}}

Resignation of Alex Fox (C).

\noindent
\begin{tabular*}{\columnwidth}{@{\extracolsep{\fill}} p{0.545\columnwidth} >{\itshape}l r @{\extracolsep{\fill}}}
Greg Hall & C & 1177\\
Charles Dean-Young & LD & 414\\
\end{tabular*}

\subsubsection*{Abbey \hspace*{\fill}\nolinebreak[1]%
\enspace\hspace*{\fill}
\finalhyphendemerits=0
[21st October]}

\index{Abbey , East Staffordshire@Abbey, \emph{E. Staffs.}}

Resignation of Kathy Lamb (C).

\noindent
\begin{tabular*}{\columnwidth}{@{\extracolsep{\fill}} p{0.545\columnwidth} >{\itshape}l r @{\extracolsep{\fill}}}
Colin Whittaker & C & 604\\
Liz Harman & Lab & 84\\
\end{tabular*}

\subsection{Newcastle-under-Lyme}
\index{Newcastle-under-Lyme}

\subsubsection*{Newchapel \hspace*{\fill}\nolinebreak[1]%
\enspace\hspace*{\fill}
\finalhyphendemerits=0
[4th February]}

\index{Newchapel , Newcastle-under-Lyme@Newchapel, \emph{Newcastle-under-Lyme}}

Death of Nora Salt (C).

\noindent
\begin{tabular*}{\columnwidth}{@{\extracolsep{\fill}} p{0.545\columnwidth} >{\itshape}l r @{\extracolsep{\fill}}}
Susan Short & C & 208\\
Carol Lovatt & UKIP & 148\\
Kyle Robinson & Lab & 138\\
Adrian Rhodes & LD & 127\\
\end{tabular*}

\subsubsection*{Westlands \hspace*{\fill}\nolinebreak[1]%
\enspace\hspace*{\fill}
\finalhyphendemerits=0
[6th May]}

\index{Westlands , Newcastle-under-Lyme@Westlands, \emph{Newcastle-under-Lyme}}

Resignation of Mary Moss (C).

This by-election was combined with the 2010 ordinary election.
%; see page \pageref{WestlandsNewcastleLyme} for the result.

\subsection{South Staffordshire}
\index{South Staffordshire}

\subsubsection*{Huntington and Hatherton \hspace*{\fill}\nolinebreak[1]%
\enspace\hspace*{\fill}
\finalhyphendemerits=0
[1st July]}

\index{Huntington and Hatherton , South Staffordshire@Huntington \& Hatherton, \emph{S. Staffs.}}

Death of Patricia Williams (C).

\noindent
\begin{tabular*}{\columnwidth}{@{\extracolsep{\fill}} p{0.545\columnwidth} >{\itshape}l r @{\extracolsep{\fill}}}
David Williams & C & 237\\
Ron Kenyon & Lab & 235\\
David Percox & Ind & 64\\
\end{tabular*}

\subsection{Stafford}
\index{Stafford}

\subsubsection*{Gnosall and Woodseaves \hspace*{\fill}\nolinebreak[1]%
\enspace\hspace*{\fill}
\finalhyphendemerits=0
[6th May]}

\index{Gnosall and Woodseaves , Stafford@Gnosall \& Woodseaves, \emph{Stafford}}

Death of James Kelly (C).

\noindent
\begin{tabular*}{\columnwidth}{@{\extracolsep{\fill}} p{0.545\columnwidth} >{\itshape}l r @{\extracolsep{\fill}}}
Ann Kelly & C & 2336\\
David Kirby & LD & 1240\\
John Gale & Grn & 255\\
\end{tabular*}

\subsection{Stoke-on-Trent}
\index{Stoke-on-Trent}

At the May 2010 ordinary election there was an unfilled vacancy in Blurton ward due to the death of Derek Hall (City Independent).
\index{Blurton , Stoke-on-Trent@Blurton, \emph{Stoke-on-Trent}}

%There was an unfilled vacancy in Blurton ward at the ordinary May elections following the death of Derek Hall (City Independent) with less than six months of his term remaining.

\section{Suffolk}

\subsection{County Council}
\index{Suffolk}

\subsubsection*{Aldeburgh and Leiston \hspace*{\fill}\nolinebreak[1]%
\enspace\hspace*{\fill}
\finalhyphendemerits=0
[6th May]}

\index{Aldeburgh and Leiston , Suffolk@Aldeburgh \& Leiston, \emph{Suffolk}}

Resignation of Ron Ward (C).

\noindent
\begin{tabular*}{\columnwidth}{@{\extracolsep{\fill}} p{0.545\columnwidth} >{\itshape}l r @{\extracolsep{\fill}}}
Richard Smith & C & 2240\\
Joan Girling & Ind & 1439\\
Terry Hodgson & Lab & 1243\\
\end{tabular*}

\subsubsection*{Tower \hspace*{\fill}\nolinebreak[1]%
\enspace\hspace*{\fill}
\finalhyphendemerits=0
[11th November]}

\index{Tower , Suffolk@Tower, \emph{Suffolk}}

Resignation of Paul Farmer (C).

\noindent
\begin{tabular*}{\columnwidth}{@{\extracolsep{\fill}} p{0.545\columnwidth} >{\itshape}l r @{\extracolsep{\fill}}}
Stefan Oliver & C & 1005\\
David Nettleton & Ind & 950\\
Kevin Hind & Lab & 759\\
Pippa Judd & Grn & 479\\
David Chappell & LD & 300\\
\end{tabular*}

\subsection{Babergh}
\index{Babergh}

\subsubsection*{Great Cornard North \hspace*{\fill}\nolinebreak[1]%
\enspace\hspace*{\fill}
\finalhyphendemerits=0
[22nd July; Lab gain from C]}

\index{Great Cornard North , Babergh@Great Cornard N., \emph{Babergh}}

Resignation of Carole Todd (C).

\noindent
\begin{tabular*}{\columnwidth}{@{\extracolsep{\fill}} p{0.545\columnwidth} >{\itshape}l r @{\extracolsep{\fill}}}
Anthony Bavington & Lab & 340\\
Martin Fryer & C & 201\\
Richard Platt & LD & 141\\
Leon Stedman & UKIP & 72\\
\end{tabular*}

\subsection{Ipswich}
\index{Ipswich}

\subsubsection*{Alexandra \hspace*{\fill}\nolinebreak[1]%
\enspace\hspace*{\fill}
\finalhyphendemerits=0
[6th May]}

\index{Alexandra , Ipswich@Alexandra, \emph{Ipswich}}

Resignation of Louise Gooch (LD).

This by-election was combined with the 2010 ordinary election.
%; see page \pageref{AlexandraIpswich} for the result.

\subsection{Mid Suffolk}
\index{Mid Suffolk}

\subsubsection*{Haughley and Wetherden \hspace*{\fill}\nolinebreak[1]%
\enspace\hspace*{\fill}
\finalhyphendemerits=0
[25th March; Grn gain from C]}

\index{Haughley and Wetherden , Mid Suffolk@Haughley \& Wetherden, \emph{Mid Suffolk}}

Resignation of Bruce Cameron-Laker (C).

\noindent
\begin{tabular*}{\columnwidth}{@{\extracolsep{\fill}} p{0.545\columnwidth} >{\itshape}l r @{\extracolsep{\fill}}}
Rachel Eburne & Grn & 444\\
Samantha Powell & C & 176\\
Chris Vecchi & LD & 51\\
David Hill & Lab & 32\\
Christopher Streatfield & UKIP & 25\\
\end{tabular*}

%\subsection{Waveney}
%
%\subsubsection*{Southwold and Reydon \hspace*{\fill}\nolinebreak[1]%
%\enspace\hspace*{\fill}
%\finalhyphendemerits=0
%[tba]}
%
%\index{Southwold and Reydon , Waveney@Southwold \& Reydon, Waveney}
%
%Resignation of Simon Tobin (C).
%
%\noindent
%\begin{tabular*}{\columnwidth}{@{\extracolsep{\fill}} p{0.545\columnwidth} >{\itshape}l r @{\extracolsep{\fill}}}
%?
%\end{tabular*}
%
\section{Surrey}

\subsection{County Council}
\index{Surrey}

Peace = Peace Party

\subsubsection*{Walton South and Oatlands \hspace*{\fill}\nolinebreak[1]%
\enspace\hspace*{\fill}
\finalhyphendemerits=0
[6th May]}

\index{Walton South and Oatlands , Surrey@Walton S. \& Oatlands, \emph{Surrey}}

Resignation of Roy Taylor (LD).

\noindent
\begin{tabular*}{\columnwidth}{@{\extracolsep{\fill}} p{0.545\columnwidth} >{\itshape}l r @{\extracolsep{\fill}}}
Anthony Samuels & C & 5231\\
Vicki MacLeod & LD & 2802\\
\end{tabular*}

\subsubsection*{Worplesdon \hspace*{\fill}\nolinebreak[1]%
\enspace\hspace*{\fill}
\finalhyphendemerits=0
[15th July]}

\index{Worplesdon , Surrey@Worplesdon, \emph{Surrey}}

Death of Mike Nevins (C).

\noindent
\begin{tabular*}{\columnwidth}{@{\extracolsep{\fill}} p{0.545\columnwidth} >{\itshape}l r @{\extracolsep{\fill}}}
Nigel Sutcliffe & C & 1844\\
Paul Cragg & LD & 1286\\
Martin Phillips & Lab & 193\\
Mazhar Manzoor & UKIP & 78\\
John Morris & Peace & 39\\
\end{tabular*}

\subsection{Elmbridge}
\index{Elmbridge}

StGHI = St George's Hill Independents

\subsubsection*{Hersham South \hspace*{\fill}\nolinebreak[1]%
\enspace\hspace*{\fill}
\finalhyphendemerits=0
[6th May]}

\index{Hersham South , Elmbridge@Hersham S., \emph{Elmbridge}}

Resignation of Lara Conaway (C).

This by-election was combined with the 2010 ordinary election.
%; see page \pageref{HershamSElmbridge} for the result.

\subsubsection*{Hersham North \hspace*{\fill}\nolinebreak[1]%
\enspace\hspace*{\fill}
\finalhyphendemerits=0
[21st October]}

\index{Hersham North , Elmbridge@Hersham N., \emph{Elmbridge}}

Resignation of Doug Packer (C).

\noindent
\begin{tabular*}{\columnwidth}{@{\extracolsep{\fill}} p{0.545\columnwidth} >{\itshape}l r @{\extracolsep{\fill}}}
Simon Desborough & C & 463\\
Roy Green & Ind & 453\\
Peter Jepson & Lab & 135\\
\end{tabular*}

\subsubsection*{St George's Hill \hspace*{\fill}\nolinebreak[1]%
\enspace\hspace*{\fill}
\finalhyphendemerits=0
[21st October]}

\index{Saint George's Hill , Elmbridge@St George's Hill, \emph{Elmbridge}}

Death of John Bartlett (StGHI).

\noindent
\begin{tabular*}{\columnwidth}{@{\extracolsep{\fill}} p{0.545\columnwidth} >{\itshape}l r @{\extracolsep{\fill}}}
Peter Harman & StGHI & 515\\
Melissa Lake & C & 412\\
Robert Evans & Lab & 36\\
\end{tabular*}

\subsection{Guildford}
\index{Guildford}

\subsubsection*{Pirbright \hspace*{\fill}\nolinebreak[1]%
\enspace\hspace*{\fill}
\finalhyphendemerits=0
[15th July]}

\index{Pirbright , Guildford@Pirbright, \emph{Guildford}}

Death of Mike Nevins (C).

\noindent
\begin{tabular*}{\columnwidth}{@{\extracolsep{\fill}} p{0.545\columnwidth} >{\itshape}l r @{\extracolsep{\fill}}}
Gordon Jackson & C & 364\\
Philip Douetil & LD & 199\\
\end{tabular*}

\subsection{Mole Valley}
\index{Mole Valley}

\subsubsection*{Capel, Leigh and Newdigate \hspace*{\fill}\nolinebreak[1]%
\enspace\hspace*{\fill}
\finalhyphendemerits=0
[21st October; LD gain from C]}

\index{Capel, Leigh and Newdigate , Mole Valley@Capel, Leigh \& Newdigate, \emph{Mole Valley}}

Resignation of Philip Warren (C).

\noindent
\begin{tabular*}{\columnwidth}{@{\extracolsep{\fill}} p{0.575\columnwidth} >{\itshape}l r @{\extracolsep{\fill}}}
Iain Murdoch & LD & 618\\
\sloppyword{Corinna Osborne-Patterson} & C & 558\\
Tim Bowling & UKIP & 97\\
Jacqui Hamlin & Grn & 61\\
\end{tabular*}

\subsection{Reigate and Banstead}
\index{Reigate and Banstead}

\subsubsection*{Redhill East (2) \hspace*{\fill}\nolinebreak[1]%
\enspace\hspace*{\fill}
\finalhyphendemerits=0
[6th May]}

\index{Redhill East , Reigate and Banstead@Redhill E., \emph{Reigate \& Banstead}}

Death of Frank Moore and resignation of Daniel Poulter (both C).

This double by-election was combined with the 2010 ordinary election.
%; see page \pageref{RedhillEReigateBanstead} for the result.

\subsection{Tandridge}
\index{Tandridge}

\subsubsection*{Whyteleafe \hspace*{\fill}\nolinebreak[1]%
\enspace\hspace*{\fill}
\finalhyphendemerits=0
[Tuesday 2nd February]}

\index{Whyteleafe , Tandridge@Whyteleafe, \emph{Tandridge}}

Resignation of Jeffrey Gray (LD).

\noindent
\begin{tabular*}{\columnwidth}{@{\extracolsep{\fill}} p{0.545\columnwidth} >{\itshape}l r @{\extracolsep{\fill}}}
David Lee & LD & 444\\
Chris Krishnan & C & 236\\
Jeffrey Bolter & UKIP & 99\\
\end{tabular*}

\subsection{Woking}
\index{Woking}

\subsubsection*{Byfleet \hspace*{\fill}\nolinebreak[1]%
\enspace\hspace*{\fill}
\finalhyphendemerits=0
[6th May]}

\index{Byfleet , Woking@Byfleet, \emph{Woking}}

Resignation of Simon Hutton (C).

This by-election was combined with the 2010 ordinary election.
%; see page \pageref{ByfleetWoking} for the result.

\section{Warwickshire}

\subsection{County Council}
\index{Warwickshire}

\subsubsection*{Nuneaton St Nicholas \hspace*{\fill}\nolinebreak[1]%
\enspace\hspace*{\fill}
\finalhyphendemerits=0
[6th May]}

\index{Nuneaton Saint Nicholas , Warwickshire@Nuneaton St Nicholas, \emph{Warks.}}

Death of David Bryden (C).

\noindent
\begin{tabular*}{\columnwidth}{@{\extracolsep{\fill}} p{0.545\columnwidth} >{\itshape}l r @{\extracolsep{\fill}}}
Jeffrey Clarke & C & 3195\\
Paul Hickling & Lab & 1616\\
Keith Kondakor & Grn & 589\\
Thomas Wilson & Ind & 280\\
\end{tabular*}

\subsubsection*{Studley \hspace*{\fill}\nolinebreak[1]%
\enspace\hspace*{\fill}
\finalhyphendemerits=0
[6th May]}

\index{Studley , Warwickshire@Studley, \emph{Warks.}}

Resignation of Martin Barry (LD).

\noindent
\begin{tabular*}{\columnwidth}{@{\extracolsep{\fill}} p{0.545\columnwidth} >{\itshape}l r @{\extracolsep{\fill}}}
Clive Rickhards & LD & 1906\\
Justin Kerridge & C & 1755\\
Jacqueline Abbott & Lab & 480\\
Karen Varga & Grn & 128\\
\end{tabular*}

\subsection{Nuneaton and Bedworth}
\index{Nuneaton and Bedworth}

At the May 2010 ordinary election there was an unfilled vacancy in St Nicholas ward due to the death of David Bryden (C).
\index{Saint Nicholas , Nuneaton and Bedworth@St Nicholas, \emph{Nuneaton \& Bedworth}}

%There was an unfilled vacancy in St Nicholas ward at the ordinary May elections following the death of David Bryden (C) with less than six months of his term remaining.

\subsection{Rugby}
\index{Rugby}

At the May 2010 ordinary election there was an unfilled vacancy in Dunchurch and Knightlow ward due to the death of Ron Ravenhall (LD).
\index{Dunchurch and Knightlow , Rugby@Dunchurch \& Knightlow, \emph{Rugby}}
%There was an unfilled vacancy in Dunchurch and Knightlow ward at the ordinary May elections following the death of Ron Ravenhall (LD) with less than six months of his term remaining.

\subsubsection*{Dunchurch and Knightlow \hspace*{\fill}\nolinebreak[1]%
\enspace\hspace*{\fill}
\finalhyphendemerits=0
[2nd December]}

\index{Dunchurch and Knightlow , Rugby@Dunchurch \& Knightlow, \emph{Rugby}}

Resignation of Mike Galsworthy (C).

\noindent
\begin{tabular*}{\columnwidth}{@{\extracolsep{\fill}} p{0.545\columnwidth} >{\itshape}l r @{\extracolsep{\fill}}}
Ian Lowe & C & 832\\
Robert Aird & LD & 682\\
Robert McNally & Lab & 149\\
George Hougez & Grn & 20\\
\end{tabular*}

\subsection{Stratford-on-Avon}
\index{Stratford-on-Avon}

\subsubsection*{Henley \hspace*{\fill}\nolinebreak[1]%
\enspace\hspace*{\fill}
\finalhyphendemerits=0
[6th May]}

\index{Henley , Stratford-on-Avon@Henley, \emph{Stratford-on-Avon}}

Resignation of Laurence Marshall (C).

This by-election was combined with the 2010 ordinary election.
%; see page \pageref{HenleyArdenStratfordAvon} for the result.

\subsubsection*{Shipston \hspace*{\fill}\nolinebreak[1]%
\enspace\hspace*{\fill}
\finalhyphendemerits=0
[6th May]}

\index{Shipston , Stratford-on-Avon@Shipston, \emph{Stratford-on-Avon}}

Resignation of Bob White (LD).

This by-election was combined with the 2010 ordinary election.
%; see page \pageref{ShipstonStratfordAvon} for the result.

\subsection{Warwick}
\index{Warwick}

\subsubsection*{Warwick South \hspace*{\fill}\nolinebreak[1]%
\enspace\hspace*{\fill}
\finalhyphendemerits=0
[22nd July]}

\index{Warwick South , Warwick@Warwick S., \emph{Warwick}}

Resignation of Chris White (C).

\noindent
\begin{tabular*}{\columnwidth}{@{\extracolsep{\fill}} p{0.545\columnwidth} >{\itshape}l r @{\extracolsep{\fill}}}
Linda Bromley & C & 1107\\
Chris McKeown & Lab & 648\\
Chris Spedding & LD & 294\\
\end{tabular*}

\section{West Sussex}

Justice = Justice Party

\subsection{County Council}
\index{West Sussex}

\subsubsection*{East Grinstead South and Ashurst Wood \hspace*{\fill}\nolinebreak[1]%
\enspace\hspace*{\fill}
\finalhyphendemerits=0
[6th May]}

\index{East Grinstead South and Ashurst Wood , West Sussex@East Grinstead S. \& Ashurst Wood, \emph{W. Sussex}}

Resignation of Lee Quinn (C).

\noindent
\begin{tabular*}{\columnwidth}{@{\extracolsep{\fill}} p{0.545\columnwidth} >{\itshape}l r @{\extracolsep{\fill}}}
John O'Brien & C & 2878\\
Catrin Ingham & LD & 2771\\
Robert Wall & Ind & 328\\
\end{tabular*}

\subsubsection*{Maidenbower \hspace*{\fill}\nolinebreak[1]%
\enspace\hspace*{\fill}
\finalhyphendemerits=0
[7th October]}

\index{Maidenbower , West Sussex@Maidenbower, \emph{W. Sussex}}

Resignation of Henry Smith (C).

\noindent
\begin{tabular*}{\columnwidth}{@{\extracolsep{\fill}} p{0.545\columnwidth} >{\itshape}l r @{\extracolsep{\fill}}}
Robert Lanzer & C & 1036\\
Peter Smith & Lab & 417\\
Sulu Pandya & LD & 82\\
John Mac Canna & UKIP & 61\\
Arshad Khan & Justice & 12\\
\end{tabular*}

\subsection{Adur}
\index{Adur}

\subsubsection*{Eastbrook \hspace*{\fill}\nolinebreak[1]%
\enspace\hspace*{\fill}
\finalhyphendemerits=0
[6th May]}

\index{Eastbrook , Adur@Eastbrook, \emph{Adur}}

Resignation of Emma Evans (C).

This by-election was combined with the 2010 ordinary election.
%; see page \pageref{EastbrookAdur} for the result.

\subsection{Chichester}
\index{Chichester}

\subsubsection*{Plaistow \hspace*{\fill}\nolinebreak[1]%
\enspace\hspace*{\fill}
\finalhyphendemerits=0
[11th February]}

\index{Plaistow , Chichester@Plaistow, \emph{Chichester}}

Resignation of Brian Hooton (C).

\noindent
\begin{tabular*}{\columnwidth}{@{\extracolsep{\fill}} p{0.545\columnwidth} >{\itshape}l r @{\extracolsep{\fill}}}
Philippa Hardwick & C & 504\\
Raymond Cooper & LD & 301\\
Andrew Emerson & BNP & 69\\
\end{tabular*}

\subsubsection*{Rogate \hspace*{\fill}\nolinebreak[1]%
\enspace\hspace*{\fill}
\finalhyphendemerits=0
[17th June]}

\index{Rogate , Chichester@Rogate, \emph{Chichester}}

Resignation of William Mason (C).

\noindent
\begin{tabular*}{\columnwidth}{@{\extracolsep{\fill}} p{0.545\columnwidth} >{\itshape}l r @{\extracolsep{\fill}}}
Nigel Johnson-Hill & C & \emph{unop.}\\
\end{tabular*}

\subsubsection*{Plaistow \hspace*{\fill}\nolinebreak[1]%
\enspace\hspace*{\fill}
\finalhyphendemerits=0
[25th November]}

\index{Plaistow , Chichester@Plaistow, \emph{Chichester}}

Resignation of John Andrews (C).

\noindent
\begin{tabular*}{\columnwidth}{@{\extracolsep{\fill}} p{0.545\columnwidth} >{\itshape}l r @{\extracolsep{\fill}}}
Linda Westmore & C & 416\\
Ray Cooper & LD & 289\\
Gail Weingartner & UKIP & 62\\
\end{tabular*}

\subsection{Crawley}
\index{Crawley}

\subsubsection*{Tilgate \hspace*{\fill}\nolinebreak[1]%
\enspace\hspace*{\fill}
\finalhyphendemerits=0
[7th October; Lab gain from C]}

\index{Tilgate , Crawley@Tilgate, \emph{Crawley}}

Resignation of Lawrence Taylor (C).

\noindent
\begin{tabular*}{\columnwidth}{@{\extracolsep{\fill}} p{0.545\columnwidth} >{\itshape}l r @{\extracolsep{\fill}}}
Colin Lloyd & Lab & 764\\
Ray Ward & C & 656\\
Maurice Day & UKIP & 79\\
Arshad Khan & Justice & 6\\
\end{tabular*}

\subsection{Mid Sussex}
\index{Mid Sussex}

\subsubsection*{Cuckfield \hspace*{\fill}\nolinebreak[1]%
\enspace\hspace*{\fill}
\finalhyphendemerits=0
[6th May]}

\index{Cuckfield , Mid Sussex@Cuckfield, \emph{Mid Sussex}}

Death of Brenda Binge (C).

\noindent
\begin{tabular*}{\columnwidth}{@{\extracolsep{\fill}} p{0.545\columnwidth} >{\itshape}l r @{\extracolsep{\fill}}}
Robert Salisbury & C & 1597\\
Stephen Blanch & LD & 1362\\
\end{tabular*}

\subsubsection*{Haywards Heath Heath \hspace*{\fill}\nolinebreak[1]%
\enspace\hspace*{\fill}
\finalhyphendemerits=0
[6th May; LD gain from C]}

\index{Haywards Heath Heath , Mid Sussex@Haywards Heath Heath, \emph{Mid Sussex}}

Resignation of Margaret Baker (C).

\noindent
\begin{tabular*}{\columnwidth}{@{\extracolsep{\fill}} p{0.545\columnwidth} >{\itshape}l r @{\extracolsep{\fill}}}
Sue Ng & LD & 1350\\
Ruth de Mierre & C & 1310\\
Alan Yates & Lab & 221\\
\end{tabular*}

\subsubsection*{High Weald \hspace*{\fill}\nolinebreak[1]%
\enspace\hspace*{\fill}
\finalhyphendemerits=0
[6th May]}

\index{High Weald , Mid Sussex@High Weald, \emph{Mid Sussex}}

Resignation of James Temple-Smithson (C).

\noindent
\begin{tabular*}{\columnwidth}{@{\extracolsep{\fill}} p{0.545\columnwidth} >{\itshape}l r @{\extracolsep{\fill}}}
Simon McMenemy & C & 1670\\
Anne-Marie Lucraft & LD & 922\\
Paul Brown & Grn & 395\\
\end{tabular*}

\subsubsection*{Hurstpierpoint and Downs \hspace*{\fill}\nolinebreak[1]%
\enspace\hspace*{\fill}
\finalhyphendemerits=0
[6th May]}

\index{Hurstpierpoint and Downs , Mid Sussex@Hurstpierpoint \& Downs, \emph{Mid Sussex}}

Resignation of Christopher Maidment (C).

\noindent
\begin{tabular*}{\columnwidth}{@{\extracolsep{\fill}} p{0.545\columnwidth} >{\itshape}l r @{\extracolsep{\fill}}}
Anna Deflippo & C & 2362\\
Rodney Jackson & LD & 1848\\
\end{tabular*}

\subsubsection*{Haywards Heath Franklands \hspace*{\fill}\nolinebreak[1]%
\enspace\hspace*{\fill}
\finalhyphendemerits=0
[29th July]}

\index{Haywards Heath Franklands , Mid Sussex@Haywards Heath Franklands, \emph{Mid Sussex}}

Death of Clive Chapman (C).

\noindent
\begin{tabular*}{\columnwidth}{@{\extracolsep{\fill}} p{0.545\columnwidth} >{\itshape}l r @{\extracolsep{\fill}}}
John de Mierre & C & 545\\
Raha Kazemi & LD & 464\\
Colin Bates & Ind & 63\\
\end{tabular*}

\subsection{Worthing}
\index{Worthing}

At the May 2010 ordinary election there was an unfilled vacancy in Salvington ward due to the resignation of Heather Mercer (C).
\index{Salvington , Worthing@Salvington, \emph{Worthing}}

%There was an unfilled vacancy in Salvington ward at the ordinary May elections following the resignation of Heather Mercer (C) with less than six months of her term remaining.

\subsubsection*{Marine \hspace*{\fill}\nolinebreak[1]%
\enspace\hspace*{\fill}
\finalhyphendemerits=0
[6th May]}

\index{Marine , Worthing@Marine, \emph{Worthing}}

Resignation of Keith Mercer (C).

This by-election was combined with the 2010 ordinary election.
%; see page \pageref{MarineWorthing} for the result.

\section{Wiltshire}

\subsection{Swindon}
\index{Swindon}

\subsubsection*{Moredon \hspace*{\fill}\nolinebreak[1]%
\enspace\hspace*{\fill}
\finalhyphendemerits=0
[4th November; Lab gain from C]}

\index{Moredon , Swindon@Moredon, \emph{Swindon}}

Resignation of Steph Exell (C).

\noindent
\begin{tabular*}{\columnwidth}{@{\extracolsep{\fill}} p{0.545\columnwidth} >{\itshape}l r @{\extracolsep{\fill}}}
Jenny Millin & Lab & 887\\
Toby Elliott & C & 755\\
William Oram & UKIP & 129\\
Chris Ward & LD & 98\\
\end{tabular*}

\subsection{Wiltshire}
\index{Wiltshire}

\subsubsection*{Bromham, Rowde and Potterne \hspace*{\fill}\nolinebreak[1]%
\enspace\hspace*{\fill}
\finalhyphendemerits=0
[Tuesday 21st December]}

\index{Bromham, Rowde and Potterne , Wiltshire@Bromham, Rowde \& Potterne, \emph{Wilts.}}

Resignation of Philip Brown (C).

\noindent
\begin{tabular*}{\columnwidth}{@{\extracolsep{\fill}} p{0.545\columnwidth} >{\itshape}l r @{\extracolsep{\fill}}}
Liz Bryant & C & 561\\
Paul Mortimer & LD & 358\\
Andrew Jones & Lab & 74\\
Pat Bryant & Ind & 55\\
\end{tabular*}

\section{Worcestershire}

\subsection{County Council}
\index{Worcestershire}

\subsubsection*{Bowbrook \hspace*{\fill}\nolinebreak[1]%
\enspace\hspace*{\fill}
\finalhyphendemerits=0
[25th November]}

\index{Bowbrook , Worcestershire@Bowbrook, \emph{Worcs.}}

Death of Ted Sheldon (C).

\noindent
\begin{tabular*}{\columnwidth}{@{\extracolsep{\fill}} p{0.545\columnwidth} >{\itshape}l r @{\extracolsep{\fill}}}
Tony Miller & C & 1088\\
Margaret Rowley & LD & 536\\
Chris Barton & Lab & 213\\
\end{tabular*}

\subsubsection*{Alvechurch \hspace*{\fill}\nolinebreak[1]%
\enspace\hspace*{\fill}
\finalhyphendemerits=0
[16th December]}

\index{Alvechurch , Worcestershire@Alvechurch, \emph{Worcs.}}

Resignation of George Lord (C).

\noindent
\begin{tabular*}{\columnwidth}{@{\extracolsep{\fill}} p{0.545\columnwidth} >{\itshape}l r @{\extracolsep{\fill}}}
June Griffiths & C & 637\\
Christopher Bloore & Lab & 189\\
Dee Morton & Ind & 157\\
Howard Allen & LD & 83\\
Kenneth Wheatley & Ind & 79\\
Steven Morson & UKIP & 65\\
\end{tabular*}

\subsection{Bromsgrove}
\index{Bromsgrove}

\subsubsection*{Marlbrook \hspace*{\fill}\nolinebreak[1]%
\enspace\hspace*{\fill}
\finalhyphendemerits=0
[16th December]}

\index{Marlbrook , Bromsgrove@Marlbrook, \emph{Bromsgrove}}

Resignation of George Lord (C).

\noindent
\begin{tabular*}{\columnwidth}{@{\extracolsep{\fill}} p{0.545\columnwidth} >{\itshape}l r @{\extracolsep{\fill}}}
John Ruck & C & 284\\
Martin Knight & Lab & 236\\
Charlie Bateman & Ind & 138\\
Steven Morson & UKIP & 68\\
Janet King & LD & 67\\
Peter Harvey & Grn & 14\\
Kenneth Wheatley & Ind & 4\\
\end{tabular*}

\subsection{Redditch}
\index{Redditch}

\subsubsection*{Crabbs Cross \hspace*{\fill}\nolinebreak[1]%
\enspace\hspace*{\fill}
\finalhyphendemerits=0
[6th May]}

\index{Crabbs Cross , Redditch@Crabbs Cross, \emph{Redditch}}

Death of Jack Field (C).

This by-election was combined with the 2010 ordinary election.
%; see page \pageref{CrabbsXRedditch} for the result.

\subsection{Worcester}
\index{Worcester}

At the May 2010 ordinary election there was an unfilled vacancy in Bedwardine ward due to the death of Barry Mackenzie-Williams (C).
\index{Bedwardine , Worcester@Bedwardine, \emph{Worcester}}

%There was an unfilled vacancy in Bedwardine ward at the ordinary May elections following the death of Barry Mackenzie-Williams (C) with less than six months of his term remaining.

\subsection{Wychavon}
\index{Wychavon}

\subsubsection*{Evesham South \hspace*{\fill}\nolinebreak[1]%
\enspace\hspace*{\fill}
\finalhyphendemerits=0
[18th February]}

\index{Evesham South , Wychavon@Evesham S., \emph{Wychavon}}

Death of Ron Cartwright (C).

\noindent
\begin{tabular*}{\columnwidth}{@{\extracolsep{\fill}} p{0.545\columnwidth} >{\itshape}l r @{\extracolsep{\fill}}}
Gerry O'Donnell & C & 358\\
Diana Brown & LD & 176\\
John White & UKIP & 150\\
\end{tabular*}

\section{Glamorgan}

\subsection{Bridgend}
\index{Bridgend}

\subsubsection*{Pendre \hspace*{\fill}\nolinebreak[1]%
\enspace\hspace*{\fill}
\finalhyphendemerits=0
[18th February; Lab gain from LD]}

\index{Pendre , Bridgend@Pendre, \emph{Bridgend}}

Resignation of Mike Simmonds (LD).

\noindent
\begin{tabular*}{\columnwidth}{@{\extracolsep{\fill}} p{0.545\columnwidth} >{\itshape}l r @{\extracolsep{\fill}}}
Richard Young & Lab & 200\\
Anita Davies & LD & 193\\
Diane Armstrong & Ind & 68\\
Alan Wathan & C & 60\\
Tim Thomas & PC & 25\\
\end{tabular*}

\subsection{Neath Port Talbot}
\index{Neath Port Talbot}

NPTIP = Neath Port Talbot Independent Party

\subsubsection*{Neath North \hspace*{\fill}\nolinebreak[1]%
\enspace\hspace*{\fill}
\finalhyphendemerits=0
[6th May]}

\index{Neath North , Neath Port Talbot@Neath N., \emph{Neath Port Talbot}}

Resignation of Derek Vaughan MEP (Lab).

\noindent
\begin{tabular*}{\columnwidth}{@{\extracolsep{\fill}} p{0.545\columnwidth} >{\itshape}l r @{\extracolsep{\fill}}}
Alan Lockyer & Lab & 923\\
Breandan MacCathail & PC & 305\\
Brian Warlow & Ind & 296\\
Tony Phillips & NPTIP & 239\\
Stephen Carpenter & C & 208\\
\end{tabular*}

\subsubsection*{Neath North \hspace*{\fill}\nolinebreak[1]%
\enspace\hspace*{\fill}
\finalhyphendemerits=0
[14th October]}

\index{Neath North , Neath Port Talbot@Neath N., \emph{Neath Port Talbot}}

Death of Mannie Loaring (Lab).

\noindent
\begin{tabular*}{\columnwidth}{@{\extracolsep{\fill}} p{0.545\columnwidth} >{\itshape}l r @{\extracolsep{\fill}}}
Mark Protheroe & Lab & 437\\
David Howells & NPTIP & 144\\
Breandan MacCathail & PC & 132\\
Mathew McCarthy & LD & 51\\
\end{tabular*}

\subsection{Rhondda Cynon Taf}
\index{Rhondda Cynon Taf}

\subsubsection*{Cymmer \hspace*{\fill}\nolinebreak[1]%
\enspace\hspace*{\fill}
\finalhyphendemerits=0
[22nd July]}

\index{Cymmer , Rhondda Cynon Taf@Cymmer, \emph{Rhondda Cynon Taf}}

Death of Eurwen Davies (Lab).

\noindent
\begin{tabular*}{\columnwidth}{@{\extracolsep{\fill}} p{0.545\columnwidth} >{\itshape}l r @{\extracolsep{\fill}}}
Chris Williams & Lab & 740\\
Nicole Griffiths & PC & 470\\
Paul Wasley & Ind & 142\\
Steven Rogers & C & 41\\
Kevin Jakeaway & Grn & 23\\
\end{tabular*}

\subsubsection*{Treherbert \hspace*{\fill}\nolinebreak[1]%
\enspace\hspace*{\fill}
\finalhyphendemerits=0
[14th October; Lab gain from PC]}

\index{Treherbert , Rhondda Cynon Taf@Treherbert, \emph{Rhondda Cynon Taf}}

Resignation of Percy Jones (PC).

\noindent
\begin{tabular*}{\columnwidth}{@{\extracolsep{\fill}} p{0.545\columnwidth} >{\itshape}l r @{\extracolsep{\fill}}}
Luke Bouchard & Lab & 883\\
Irene Pearce & PC & 855\\
\end{tabular*}

\subsection{Swansea}
\index{Swansea}

\subsubsection*{Newton \hspace*{\fill}\nolinebreak[1]%
\enspace\hspace*{\fill}
\finalhyphendemerits=0
[21st October; C gain from LD]}

\index{Newton , Swansea@Newton, \emph{Swansea}}

Resignation of Susan Waller (LD).

\noindent
\begin{tabular*}{\columnwidth}{@{\extracolsep{\fill}} p{0.545\columnwidth} >{\itshape}l r @{\extracolsep{\fill}}}
Miles Thomas & C & 545\\
Simon Arthur & LD & 299\\
Pam Erasmus & Lab & 187\\
Peter Birch & Ind & 108\\
Rob Lowe & PC & 31\\
\end{tabular*}

\section{Gwent}

\subsection{Blaenau Gwent}
\index{Blaenau Gwent}

\subsubsection*{Tredegar Central and West \hspace*{\fill}\nolinebreak[1]%
\enspace\hspace*{\fill}
\finalhyphendemerits=0
[8th July]}

\index{Tredegar Central and West , Blaenau Gwent@Tredegar C. \& W., \emph{Blaenau Gwent}}

Death of Derek Morris (Lab).

\noindent
\begin{tabular*}{\columnwidth}{@{\extracolsep{\fill}} p{0.545\columnwidth} >{\itshape}l r @{\extracolsep{\fill}}}
Bernard Willis & Lab & 797\\
Tony Gregory & Ind & 334\\
\end{tabular*}

\subsection{Caerphilly}
\index{Caerphilly}

\subsubsection*{Blackwood \hspace*{\fill}\nolinebreak[1]%
\enspace\hspace*{\fill}
\finalhyphendemerits=0
[8th July; Lab gain from Ind]}

\index{Blackwood , Caerphilly@Blackwood, \emph{Caerphilly}}

Resignation of Kevin Etheridge (Ind).

\noindent
\begin{tabular*}{\columnwidth}{@{\extracolsep{\fill}} p{0.545\columnwidth} >{\itshape}l r @{\extracolsep{\fill}}}
Nigel Dix & Lab & 545\\
Claire Caple & Ind & 469\\
Andrew Farina-Childs & PC & 355\\
Ian Chivers & C & 170\\
\end{tabular*}

\section{Mid and West Wales}

\subsection{Carmarthenshire}
\index{Carmarthenshire}

\subsubsection*{Cenarth \hspace*{\fill}\nolinebreak[1]%
\enspace\hspace*{\fill}
\finalhyphendemerits=0
[4th November; PC gain from Ind]}

\index{Cenarth , Carmarthenshire@Cenarth, \emph{Carmarthenshire}}

Death of Haydn Jones (Ind).

\noindent
\begin{tabular*}{\columnwidth}{@{\extracolsep{\fill}} p{0.545\columnwidth} >{\itshape}l r @{\extracolsep{\fill}}}
Hazel Evans & PC & 638\\
Henrietta Hensher & C & 141\\
\end{tabular*}

\subsection{Ceredigion}
\index{Ceredigion}

\subsubsection*{Ciliau Aeron \hspace*{\fill}\nolinebreak[1]%
\enspace\hspace*{\fill}
\finalhyphendemerits=0
[Tuesday 30th November]}

\index{Ciliau Aeron , Ceredigion@Ciliau Aeron, \emph{Ceredigion}}

Disqualification of Moelfryn Maskell (PC) by the Adjudication Panel for Wales.

\noindent
\begin{tabular*}{\columnwidth}{@{\extracolsep{\fill}} p{0.545\columnwidth} >{\itshape}l r @{\extracolsep{\fill}}}
John Lumley & PC & 367\\
Sonia Williams & LD & 247\\
Luke Evetts & C & 43\\
\end{tabular*}

\section{North Wales}

\subsection{Conwy}
\index{Conwy}

\subsubsection*{Mostyn \hspace*{\fill}\nolinebreak[1]%
\enspace\hspace*{\fill}
\finalhyphendemerits=0
[24th June]}

\index{Mostyn , Conwy@Mostyn, \emph{Conwy}}

Death of Alun Barrett (Lab).

\noindent
\begin{tabular*}{\columnwidth}{@{\extracolsep{\fill}} p{0.545\columnwidth} >{\itshape}l r @{\extracolsep{\fill}}}
Jobi Hold & Lab & 348\\
Gary Burchett & C & 310\\
Janet Jones & Ind & 198\\
\end{tabular*}

\subsubsection*{Eglwysbach \hspace*{\fill}\nolinebreak[1]%
\enspace\hspace*{\fill}
\finalhyphendemerits=0
[18th November]}

\index{Eglwysbach , Conwy@Eglwysbach, \emph{Conwy}}

Resignation of Angharad Booth-Taylor (PC).

\noindent
\begin{tabular*}{\columnwidth}{@{\extracolsep{\fill}} p{0.545\columnwidth} >{\itshape}l r @{\extracolsep{\fill}}}
Michael Rayner & PC & 386\\
David Williams & C & 145\\
\end{tabular*}

%\subsubsection*{Marl \hspace*{\fill}\nolinebreak[1]%
%\enspace\hspace*{\fill}
%\finalhyphendemerits=0
%[tba]}
%
%\index{Marl , Conwy@Marl, Conwy}
%
%Resignation of Linda Hurr (C).
%
%\noindent
%\begin{tabular*}{\columnwidth}{@{\extracolsep{\fill}} p{0.545\columnwidth} >{\itshape}l r @{\extracolsep{\fill}}}
%?
%\end{tabular*}

\subsection{Gwynedd}
\index{Gwynedd}

LlG = Llais Gwynedd

\subsubsection*{Diffwys and Maenofferen \hspace*{\fill}\nolinebreak[1]%
\enspace\hspace*{\fill}
\finalhyphendemerits=0
[15th July]}

\index{Diffwys and Maenofferen , Gwynedd@Diffwys \& Maenofferen, \emph{Gwynedd}}

Resignation of Gwilym Roberts (LlG).

\noindent
\begin{tabular*}{\columnwidth}{@{\extracolsep{\fill}} p{0.545\columnwidth} >{\itshape}l r @{\extracolsep{\fill}}}
Richard Jones & LlG & 185\\
Paul Thomas & PC & 181\\
\end{tabular*}

\subsubsection*{Bowydd and Rhiw \hspace*{\fill}\nolinebreak[1]%
\enspace\hspace*{\fill}
\finalhyphendemerits=0
[30th September; PC gain from LlG]}

\index{Bowydd and Rhiw , Gwynedd@Bowydd \& Rhiw, \emph{Gwynedd}}

Resignation of Dafydd Hughes (LlG).

\noindent
\begin{tabular*}{\columnwidth}{@{\extracolsep{\fill}} p{0.545\columnwidth} >{\itshape}l r @{\extracolsep{\fill}}}
Paul Thomas & PC & 338\\
Donna Morgan & LlG & 246\\
\end{tabular*}

\subsubsection*{Seiont \hspace*{\fill}\nolinebreak[1]%
\enspace\hspace*{\fill}
\finalhyphendemerits=0
[7th October; LlG gain from Ind]}

\index{Seiont , Gwynedd@Seiont, \emph{Gwynedd}}

Death of Bob Anderson (Ind).

\noindent
\begin{tabular*}{\columnwidth}{@{\extracolsep{\fill}} p{0.545\columnwidth} >{\itshape}l r @{\extracolsep{\fill}}}
James Cooke & LlG & 399\\
Menna Thomas & PC & 279\\
Tecwyn Thomas & Lab & 184\\
Gareth Edwards & Ind & 91\\
Llinos Thomas & C & 23\\
\end{tabular*}

\subsection{Isle of Anglesey}
\index{Isle of Anglesey}

\subsubsection*{Rhosneigr \hspace*{\fill}\nolinebreak[1]%
\enspace\hspace*{\fill}
\finalhyphendemerits=0
[18th November]}

\index{Rhosneigr , Isle of Anglesey@Rhosneigr, \emph{Isle of Anglesey}}

Resignation of Phil Fowlie (Ind).

\noindent
\begin{tabular*}{\columnwidth}{@{\extracolsep{\fill}} p{0.545\columnwidth} >{\itshape}l r @{\extracolsep{\fill}}}
Richard Dew & Ind & 319\\
Martin Peet & C & 58\\
\end{tabular*}

\section{Clyde Councils}

\subsection{Glasgow}
\index{Glasgow}

\subsubsection*{Drumchapel/Anniesland \hspace*{\fill}\nolinebreak[1]%
\enspace\hspace*{\fill}
\finalhyphendemerits=0
[6th May]}

\index{Drumchapel/Anniesland , Glasgow@Drumchapel/\hspace{0pt}Anniesland, \emph{Glasgow}}

Resignation of Steven Purcell (Lab).

\noindent
\begin{tabular*}{\columnwidth}{@{\extracolsep{\fill}} p{0.545\columnwidth} >{\itshape}l r @{\extracolsep{\fill}}}
Christopher Hughes & Lab & 5710\\
Frank Rankin & SNP & 2197\\
Paul McGarry & LD & 1143\\
Richard Sullivan & C & 710\\
Eileen Duke & Grn & 375\\
\end{tabular*}

\columnbreak

\subsection{South Lanarkshire}
\index{South Lanarkshire}

EKA = East Kilbride Alliance

\subsubsection*{East Kilbride West \hspace*{\fill}\nolinebreak[1]%
\enspace\hspace*{\fill}
\finalhyphendemerits=0
[28th October]}

\index{East Kilbride West , South Lanarkshire@East Kilbride W., \emph{S. Lanarks.}}

Resignation of Michael McCann MP (Lab).

\noindent
\begin{tabular*}{\columnwidth}{@{\extracolsep{\fill}} p{0.545\columnwidth} >{\itshape}l r @{\extracolsep{\fill}}}
\emph{First preferences}\\
Alan Scott & Lab & 847\\
Pat McGuire & SNP & 571\\
Ian Harrow & C & 403\\
Raymond Burke & Grn & 82\\
Brian Jones & EKA & 71\\
Gordon Smith & LD & 70\\
\end{tabular*}

\emph{Exclude Burke, Jones and Smith}: Scott 892 McGuire 641 Harrow 455

%\noindent
%\begin{tabular*}{\columnwidth}{@{\extracolsep{\fill}} p{0.545\columnwidth} >{\itshape}l r @{\extracolsep{\fill}}}
%\multicolumn{3}{@{\extracolsep{\fill}}l}{\emph{Exclude Burke, Jones and Smith}}\\
%Alan Scott & Lab & 892\\
%Pat McGuire & SNP & 641\\
%Ian Harrow & C & 455\\
%\end{tabular*}

\noindent
\begin{tabular*}{\columnwidth}{@{\extracolsep{\fill}} p{0.545\columnwidth} >{\itshape}l r @{\extracolsep{\fill}}}
\emph{Exclude Harrow}\\
Alan Scott & Lab & 973\\
Pat McGuire & SNP & 761\\
\end{tabular*}

\section{Forth Councils}

\subsection{Edinburgh}
\index{Edinburgh}

Pirate = Pirate Party Scotland

\subsubsection*{Liberton/Gilmerton \hspace*{\fill}\nolinebreak[1]%
\enspace\hspace*{\fill}
\finalhyphendemerits=0
[9th September]}

\index{Liberton/Gilmerton , Edinburgh@Liberton\slash Gilmerton, \emph{Edinburgh}}

Resignation of Ian Murray (Lab).

\noindent
\begin{tabular*}{\columnwidth}{@{\extracolsep{\fill}} p{0.545\columnwidth} >{\itshape}l r @{\extracolsep{\fill}}}
\emph{First preferences}\\
Bill Cook & Lab & 2974\\
Richard Lewis & SNP & 1382\\
Stephanie Murray & C & 1020\\
John Knox & LD & 722\\
Peter McColl & Grn & 201\\
Colin Fox & SSP & 169\\
Mev Brown & Ind & 128\\
Philip Hunt & Pirate & 43\\
\end{tabular*}

\emph{Exclude Brown, Fox, Hunt and McColl}: Cook 3121 Lewis 1484 Murray 1055 Knox 816

%\noindent
%\begin{tabular*}{\columnwidth}{@{\extracolsep{\fill}} p{0.545\columnwidth} >{\itshape}l r @{\extracolsep{\fill}}}
%\multicolumn{3}{@{\extracolsep{\fill}}l}{\emph{Exclude Brown, Fox, Hunt and McColl}}\\
%Bill Cook & Lab & 3121\\
%Richard Lewis & SNP & 1494\\
%Stephanie Murray & C & 1055\\
%John Knox & LD & 816\\
%\end{tabular*}

\noindent
\begin{tabular*}{\columnwidth}{@{\extracolsep{\fill}} p{0.545\columnwidth} >{\itshape}l r @{\extracolsep{\fill}}}
\emph{Exclude Knox}\\
Bill Cook & Lab & 3308\\
Richard Lewis & SNP & 1698\\
Stephanie Murray & C & 1262\\
\end{tabular*}

\columnbreak

\section{Highland Councils}

\subsection{Moray}
\index{Moray}

SSCUP = Scottish Senior Citizens Unity Party

\subsubsection*{Forres \hspace*{\fill}\nolinebreak[1]%
\enspace\hspace*{\fill}
\finalhyphendemerits=0
[11th November; Ind gain from C]}

\index{Forres , Moray@Forres, \emph{Moray}}

Resignation of Iain Young (C).

\noindent
\begin{tabular*}{\columnwidth}{@{\extracolsep{\fill}} p{0.545\columnwidth} >{\itshape}l r @{\extracolsep{\fill}}}
\emph{First preferences}\\
Aaron McLean & SNP & 773\\
Lorna Creswell & Ind & 562\\
Paul McBain & C & 463\\
Anne Skene & Ind & 463\\
Fabio Villani & Grn & 401\\
Mark Cascarino & Lab & 195\\
Janet Kennedy & Ind & 192\\
Andy Anderson & SSCUP & 132\\
Jane Cotton & Ind & 30\\
\end{tabular*}

\emph{Exclude Anderson and Cotton}: McLean 789 Creswell 615 Skene 491 McBain 476 Villani 405 Cascarino 201 Kennedy 201

%\noindent
%\begin{tabular*}{\columnwidth}{@{\extracolsep{\fill}} p{0.545\columnwidth} >{\itshape}l r @{\extracolsep{\fill}}}
%\multicolumn{3}{@{\extracolsep{\fill}}l}{\emph{Exclude Anderson and Cotton}}\\
%Aaron McLean & SNP & 789\\
%Lorna Creswell & Ind & 615\\
%Anne Skene & Ind & 491\\
%Paul McBain & C & 476\\
%Fabio Villani & Grn & 405\\
%Mark Cascarino & Lab & 201\\
%Janet Kennedy & Ind & 201\\
%\end{tabular*}

\sloppyword{\emph{Exclude Cascarino and Kennedy}: McLean 837 Creswell 726 Skene 572 McBain 491 Villani 459}

%\noindent
%\begin{tabular*}{\columnwidth}{@{\extracolsep{\fill}} p{0.545\columnwidth} >{\itshape}l r @{\extracolsep{\fill}}}
%\multicolumn{3}{@{\extracolsep{\fill}}l}{\emph{Exclude Cascarino and Kennedy}}\\
%Aaron McLean & SNP & 837\\
%Lorna Creswell & Ind & 726\\
%Anne Skene & Ind & 572\\
%Paul McBain & C & 491\\
%Fabio Villani & Grn & 459\\
%\end{tabular*}

\emph{Exclude Villani}: McLean 899 Creswell 878 Skene 695 McBain 511

%\noindent
%\begin{tabular*}{\columnwidth}{@{\extracolsep{\fill}} p{0.545\columnwidth} >{\itshape}l r @{\extracolsep{\fill}}}
%\emph{Exclude Villani}\\
%Aaron McLean & SNP & 899\\
%Lorna Creswell & Ind & 878\\
%Anne Skene & Ind & 695\\
%Paul McBain & C & 511\\
%\end{tabular*}

\emph{Exclude McBain}: Creswell 999 McLean 950 Skene 860

%\noindent
%\begin{tabular*}{\columnwidth}{@{\extracolsep{\fill}} p{0.545\columnwidth} >{\itshape}l r @{\extracolsep{\fill}}}
%\emph{Exclude McBain}\\
%Lorna Creswell & Ind & 999\\
%Aaron McLean & SNP & 950\\
%Anne Skene & Ind & 860\\
%\end{tabular*}

\noindent
\begin{tabular*}{\columnwidth}{@{\extracolsep{\fill}} p{0.545\columnwidth} >{\itshape}l r @{\extracolsep{\fill}}}
\emph{Exclude Skene}\\
Lorna Creswell & Ind & 1399\\
Aaron McLean & SNP & 1045\\
\end{tabular*}

\section*{}

\section*{}

\section*{}

\columnbreak

\section{Tay Councils}

\subsection{Perth and Kinross}
\index{Perth and Kinross}

\subsubsection*{Strathallan \hspace*{\fill}\nolinebreak[1]%
\enspace\hspace*{\fill}
\finalhyphendemerits=0
[6th May]}

\index{Strathallan , Perth and Kinross@Strathallan, \emph{Perth \& Kinross}}

Death of John Law (SNP).

\noindent
\begin{tabular*}{\columnwidth}{@{\extracolsep{\fill}} p{0.545\columnwidth} >{\itshape}l r @{\extracolsep{\fill}}}
\emph{First preferences}\\
John Blackie & C & 1713\\
Tom Gray & SNP & 1555\\
Neil Gaunt & LD & 1042\\
Alistair Munro & Lab & 754\\
Chris Rennie & Ind & 61\\
\end{tabular*}

\emph{Exclude Munro and Rennie}: Blackie 1785 Gray 1769 Gaunt 1321

%\noindent
%\begin{tabular*}{\columnwidth}{@{\extracolsep{\fill}} p{0.545\columnwidth} >{\itshape}l r @{\extracolsep{\fill}}}
%\emph{Exclude Munro and Rennie}\\
%John Blackie & C & 1785\\
%Tom Gray & SNP & 1769\\
%Neil Gaunt & LD & 1321\\
%\end{tabular*}

\noindent
\begin{tabular*}{\columnwidth}{@{\extracolsep{\fill}} p{0.545\columnwidth} >{\itshape}l r @{\extracolsep{\fill}}}
\emph{Exclude Gaunt}\\
Tom Gray & SNP & 2299\\
John Blackie & C & 2208\\
\end{tabular*}

\end{resultsiii}

\part{2011}
\renewcommand\resultsyear{2011}

\chapter{Referendums in 2011}

\section{Welsh Devolution Referendum}
\index{Referendums!Welsh Devolution (2011)}

A referendum was held in Wales on 3 March 2011 on whether to extend the law-making powers of the National Assembly for Wales.

\emph{Do you want the Assembly now to be able to make laws on all matters in the 20 subject areas it has powers for?}

\noindent
\begin{tabular*}{\columnwidth}{@{\extracolsep{\fill}} p{0.545\columnwidth} >{\itshape}l r @{\extracolsep{\fill}}}
& Yes & 517\,132\\
& No & 297\,380\\
\end{tabular*}

\section{Alternative Vote Referendum}
\index{Referendums!Alternative Vote}

A nationwide referendum was held on 5 May 2011 on whether to replace the first-past-the-post system of electing Members of Parliament with the Alternative Vote method.

\emph{At present, the UK uses the "first past the post" system to elect MPs to the House of Commons. Should the "alternative vote" system be used instead?}

\noindent
\begin{tabular*}{\columnwidth}{@{\extracolsep{\fill}} p{0.545\columnwidth} >{\itshape}l r @{\extracolsep{\fill}}}
& Yes & 6\,152\,607\\
& No & 13\,013\,123\\
\end{tabular*}

%\part{By-elections}

\chapter{Parliamentary by-elections}

There were six parliamentary by-elections in 2011:

\vfill

\subsection*{Oldham East and Saddleworth \hspace*{\fill}\nolinebreak[1]%
\enspace\hspace*{\fill}
\finalhyphendemerits=0
[13th January]}

\index{Oldham East and Saddleworth , House of Commons@Oldham E. \& Saddleworth, \emph{House of Commons}}

Election of Phil Woolas (Lab) declared void on petition (making false statements of fact concerning a candidate).

Elvis = Bus-Pass Elvis Party

Pirate = Pirate Party of the United Kingdom

\noindent
\begin{tabular*}{\columnwidth}{@{\extracolsep{\fill}} p{0.545\columnwidth} >{\itshape}l r @{\extracolsep{\fill}}}
Debbie Abrahams & Lab & 14718\\
Elwyn Watkins & LD & 11160\\
Kashif Ali & C & 4481\\
Paul Nuttall & UKIP & 2029\\
Derek Adams & BNP & 1560\\
Peter Allen & Grn & 530\\
Nick ``The Flying Brick'' Delves & Loony & 145\\
Stephen Morris & EDP & 144\\
Loz Kaye & Pirate & 96\\
David Bishop & Elvis & 67\\
\end{tabular*}

\vfill

\subsection*{Barnsley Central \hspace*{\fill}\nolinebreak[1]%
\enspace\hspace*{\fill}
\finalhyphendemerits=0
[3rd March]}

\index{Barnsley Central , House of Commons@Barnsley C., \emph{House of Commons}}

Resignation of Eric Illsley (Lab).

\noindent
\begin{tabular*}{\columnwidth}{@{\extracolsep{\fill}} p{0.545\columnwidth} >{\itshape}l r @{\extracolsep{\fill}}}
Dan Jarvis & Lab & 14724\\
Jane Collins & UKIP & 2953\\
James Hockney & C & 1999\\
Enis Dalton & BNP & 1463\\
Tony Devoy & Ind & 1266\\
Dominic Carman & LD & 1012\\
Kevin Riddiough & EDP & 544\\
Howling Laud Hope & Loony & 198\\
Michael Val Davies & Ind & 60\\
\end{tabular*}

\vfill

\subsection*{Leicester South \hspace*{\fill}\nolinebreak[1]%
\enspace\hspace*{\fill}
\finalhyphendemerits=0
[5th May]}

\index{Leicester South , House of Commons@Leicester S., \emph{House of Commons}}

Resignation of Sir Peter Soulsby (Lab).

\noindent
\begin{tabular*}{\columnwidth}{@{\extracolsep{\fill}} p{0.545\columnwidth} >{\itshape}l r @{\extracolsep{\fill}}}
Jon Ashworth & Lab & 19771\\
Zuffar Haq & LD & 7693\\
Jane Hunt & C & 5169\\
Abhijit Pandya & UKIP & 994\\
Howling Laud Hope & Loony & 553\\
\end{tabular*}

\subsection*{Belfast West \hspace*{\fill}\nolinebreak[1]%
\enspace\hspace*{\fill}
\finalhyphendemerits=0
[9th June]}

\index{Belfast West , House of Commons@Belfast W., \emph{House of Commons}}

Resignation of Gerry Adams (SF).

PBP = People Before Profit

\noindent
\begin{tabular*}{\columnwidth}{@{\extracolsep{\fill}} p{0.545\columnwidth} >{\itshape}l r @{\extracolsep{\fill}}}
Paul Maskey & SF & 16211\\
Alex Attwood & SDLP & 3088\\
Gerry Carroll & PBP & 1751\\
Brian Kingston & DUP & 1393\\
Bill Manwaring & UUP & 386\\
Aaron McIntyre & All & 122\\
\end{tabular*}

\subsection*{Inverclyde \hspace*{\fill}\nolinebreak[1]%
\enspace\hspace*{\fill}
\finalhyphendemerits=0
[30th June]}

\index{Inverclyde , House of Commons@Inverclyde, \emph{House of Commons}}

Death of David Cairns (Lab).

\noindent
\begin{tabular*}{\columnwidth}{@{\extracolsep{\fill}} p{0.545\columnwidth} >{\itshape}l r @{\extracolsep{\fill}}}
Iain McKenzie & Lab & 15118\\
Anne McLaughlin & SNP & 9280\\
David Wilson & C & 2784\\
Sophie Bridger & LD & 627\\
Mitch Sorbie & UKIP & 288\\
\end{tabular*}

\subsection*{Feltham and Heston \hspace*{\fill}\nolinebreak[1]%
\enspace\hspace*{\fill}
\finalhyphendemerits=0
[15th December]}

\index{Feltham and Heston , House of Commons@Feltham \& Heston, \emph{House of Commons}}

Death of Alan Keen (Lab).

Elvis = Bus-Pass Elvis Party

PBP = People Before Profit

\noindent
\begin{tabular*}{\columnwidth}{@{\extracolsep{\fill}} p{0.545\columnwidth} >{\itshape}l r @{\extracolsep{\fill}}}
Seema Malhotra & Lab & 12639\\
Mark Bowen & C & 6436\\
Roger Crouch & LD & 1364\\
Andrew Charalambous & UKIP & 1276\\
Dave Furness & BNP & 540\\
Daniel Goldsmith & Grn & 426\\
Roger Cooper & EDP & 322\\
George Hallam & PBP & 128\\
David Bishop & Elvis & 93\\
\end{tabular*}


\chapter{By-elections to devolved assemblies and the European Parliament}

\section{Greater London Authority}

There were no by-elections in 2011 to the Greater London Authority.

\section{National Assembly for Wales}

There were no by-elections in 2011 to the National Assembly for Wales.

Brynle Williams (C, North Wales) died on 1st April 2011 and his seat was vacant at the time of the 2011 ordinary election.

Aled Roberts (LD, North Wales) was returned in the 2011 ordinary election while disqualified as a member of the Valuation Tribunal for Wales.  Upon his resignation from that Tribunal, the Assembly voted to reinstate him as a member.

John Dixon (LD, South Wales Central) was returned in the 2011 ordinary election while disqualified as a member of the Care Council for Wales.  He was replaced in the Assembly by the next candidate on the list, Eluned Parrott.

\section{Scottish Parliament}

There were no by-elections in 2011 to the Scottish Parliament.

\section{Northern Ireland Assembly}

Vacancies in the Northern Ireland Assembly are filled by co-option.  There were no co-options to the Assembly in 2011.

\section{European Parliament}

UK vacancies in the European Parliament are filled by the next available person from the party list at the most recent election (which was held in 2009).  No replacements were made in 2011.

\chapter{Local by-elections and unfilled vacancies}

\begin{resultsiii}

\section{North London}

\subsection*{City of London Corporation}
\index{City of London}

\subsubsection*{Cordwainer \hspace*{\fill}\nolinebreak[1]%
\enspace\hspace*{\fill}
\finalhyphendemerits=0
[Monday 9th May]}

\index{Cordwainer , City of London@Cordwainer, \emph{City of London}}

Aldermanic election: Roger Gifford (Ind) sought re-election.

\noindent
\begin{tabular*}{\columnwidth}{@{\extracolsep{\fill}} p{0.545\columnwidth} >{\itshape}l r @{\extracolsep{\fill}}}
Roger Gifford & Ind & \emph{unop.}\\
\end{tabular*}

\subsubsection*{Langbourn \hspace*{\fill}\nolinebreak[1]%
\enspace\hspace*{\fill}
\finalhyphendemerits=0
[Monday 9th May]}

\index{Langbourn , City of London@Langbourn, \emph{City of London}}

Aldermanic election: David Wootton (Ind) sought re-election.

\noindent
\begin{tabular*}{\columnwidth}{@{\extracolsep{\fill}} p{0.545\columnwidth} >{\itshape}l r @{\extracolsep{\fill}}}
David Wootton & Ind & \emph{unop.}\\
\end{tabular*}

\subsubsection*{Aldersgate \hspace*{\fill}\nolinebreak[1]%
\enspace\hspace*{\fill}
\finalhyphendemerits=0
[Tuesday 28th June]}

\index{Aldersgate , City of London@Aldersgate, \emph{City of London}}

Aldermanic election: Nicholas Anstee (Ind) sought re-election.

\noindent
\begin{tabular*}{\columnwidth}{@{\extracolsep{\fill}} p{0.545\columnwidth} >{\itshape}l r @{\extracolsep{\fill}}}
Nicholas Anstee & Ind & 262\\
Martin Dudley & Ind & 141\\
\end{tabular*}

\subsubsection*{Castle Baynard \hspace*{\fill}\nolinebreak[1]%
\enspace\hspace*{\fill}
\finalhyphendemerits=0
[Wednesday 10th August]}

\index{Castle Baynard , City of London@Castle Baynard, \emph{City of London}}

Aldermanic election: Sir Ian Luder (Ind) sought re-election.

\noindent
\begin{tabular*}{\columnwidth}{@{\extracolsep{\fill}} p{0.545\columnwidth} >{\itshape}l r @{\extracolsep{\fill}}}
Ian Luder & Ind & \emph{unop.}\\
\end{tabular*}

\subsubsection*{Portsoken \hspace*{\fill}\nolinebreak[1]%
\enspace\hspace*{\fill}
\finalhyphendemerits=0
[Monday 5th December]}

\index{Portsoken , City of London@Portsoken, \emph{City of London}}

Aldermanic election: Michael Bear (Ind) sought re-election.

\noindent
\begin{tabular*}{\columnwidth}{@{\extracolsep{\fill}} p{0.545\columnwidth} >{\itshape}l r @{\extracolsep{\fill}}}
Michael Bear & Ind & \emph{unop.}\\
\end{tabular*}

\subsection*{Brent}
\index{Brent}

\subsubsection*{Kenton \hspace*{\fill}\nolinebreak[1]%
\enspace\hspace*{\fill}
\finalhyphendemerits=0
[17th February]}

\index{Kenton , Brent@Kenton, \emph{Brent}}

Death of Arthur Steel (C).

\noindent
\begin{tabular*}{\columnwidth}{@{\extracolsep{\fill}} p{0.545\columnwidth} >{\itshape}l r @{\extracolsep{\fill}}}
Suresh Kansagra & C & 1063\\
Ellie Southwood & Lab & 907\\
Robert Dunwell & Ind & 185\\
Chunilal Chavada & LD & 179\\
Alan Mathison & Grn & 75\\
\end{tabular*}

\subsubsection*{Wembley Central \hspace*{\fill}\nolinebreak[1]%
\enspace\hspace*{\fill}
\finalhyphendemerits=0
[22nd December]}

\index{Wembley Central , Brent@Wembley C., \emph{Brent}}

Resignation of Jayesh Mistry (Lab).

\noindent
\begin{tabular*}{\columnwidth}{@{\extracolsep{\fill}} p{0.545\columnwidth} >{\itshape}l r @{\extracolsep{\fill}}}
Krupa Sheth & Lab & 1402\\
Afifa Pervez & LD & 1022\\
Madhuri Davda & C & 340\\
Martin Francis & Grn & 130\\
\end{tabular*}

\subsection*{Camden}
\index{Camden}

\subsubsection*{Highgate \hspace*{\fill}\nolinebreak[1]%
\enspace\hspace*{\fill}
\finalhyphendemerits=0
[15th September]}

\index{Highgate , Camden@Highgate, \emph{Camden}}

Resignation of Michael Nicolaides (Lab).

\noindent
\begin{tabular*}{\columnwidth}{@{\extracolsep{\fill}} p{0.56\columnwidth} >{\itshape}l r @{\extracolsep{\fill}}}
Sally Gimson & Lab & 1178\\
Alexis Rowell & Grn & 947\\
Antony Denyer & C & 593\\
Martin Hay & LD & 111\\
\end{tabular*}

\subsection*{Enfield}
\index{Enfield}

Chr = Christian Party

\subsubsection*{Bush Hill Park \hspace*{\fill}\nolinebreak[1]%
\enspace\hspace*{\fill}
\finalhyphendemerits=0
[28th July]}

\index{Bush Hill Park , Enfield@Bush Hill Park, \emph{Enfield}}

Resignation of Eleftherios Savva (C).

\noindent
\begin{tabular*}{\columnwidth}{@{\extracolsep{\fill}} p{0.545\columnwidth} >{\itshape}l r @{\extracolsep{\fill}}}
Lee Chamberlain & C & 1108\\
Ivor Wiggett & Lab & 668\\
Tony Kingsnorth & Ind & 230\\
Paul Smith & LD & 177\\
Douglas Coker & Grn & 100\\
Gwyneth Rolph & UKIP & 70\\
Stephen Squire & BNP & 61\\
Ben Weald & EDP & 29\\
\end{tabular*}

\subsection*{Harrow}
\index{Harrow}

\subsubsection*{Canons \hspace*{\fill}\nolinebreak[1]%
\enspace\hspace*{\fill}
\finalhyphendemerits=0
[2nd June]}

\index{Canons , Harrow@Canons, \emph{Harrow}}

Death of John Cowan (C).

\noindent
\begin{tabular*}{\columnwidth}{@{\extracolsep{\fill}} p{0.545\columnwidth} >{\itshape}l r @{\extracolsep{\fill}}}
Amir Moshenson & C & 1495\\
Nitin Parekh & Lab & 774\\
Darren Diamond & LD & 236\\
\end{tabular*}

\subsubsection*{Stanmore Park \hspace*{\fill}\nolinebreak[1]%
\enspace\hspace*{\fill}
\finalhyphendemerits=0
[28th July]}

\index{Stanmore Park , Harrow@Stanmore Park, \emph{Harrow}}

Resignation of Mark Versallion (C).

\noindent
\begin{tabular*}{\columnwidth}{@{\extracolsep{\fill}} p{0.545\columnwidth} >{\itshape}l r @{\extracolsep{\fill}}}
Marilyn Ashton & C & 1395\\
Niraj Dattani & Lab & 509\\
Eric Silver & Ind & 299\\
Sylvia Warshaw & LD & 98\\
Linda Robinson & Grn & 53\\
Herbert Crossman & UKIP & 48\\
\end{tabular*}

\subsection*{Islington}
\index{Islington}

\subsubsection*{St Peter's \hspace*{\fill}\nolinebreak[1]%
\enspace\hspace*{\fill}
\finalhyphendemerits=0
[11th August]}

\index{Saint Peter's , Islington@St Peter's, \emph{Islington}}

Resignation of Shelley Coupland (Lab).

\noindent
\begin{tabular*}{\columnwidth}{@{\extracolsep{\fill}} p{0.545\columnwidth} >{\itshape}l r @{\extracolsep{\fill}}}
Alice Perry & Lab & 1167\\
David Sant & LD & 440\\
Richard Bunting & C & 381\\
Caroline Ahern & Grn & 176\\
Martin Rutherford & Ind & 59\\
\end{tabular*}

\subsubsection*{St Mary's \hspace*{\fill}\nolinebreak[1]%
\enspace\hspace*{\fill}
\finalhyphendemerits=0
[10th November]}

\index{Saint Mary's , Islington@St Mary's, \emph{Islington}}

Resignation of Joan Coupland (Ind elected as Lab).

\noindent
\begin{tabular*}{\columnwidth}{@{\extracolsep{\fill}} p{0.545\columnwidth} >{\itshape}l r @{\extracolsep{\fill}}}
Gary Poole & Lab & 1138\\
Emily Fieran-Reed & LD & 641\\
Caroline Russell & Grn & 317\\
Oriel Hutchinson & C & 282\\
Walter Barfoot & BNP & 22\\
\end{tabular*}

\council{Kensington and Chelsea}

\subsubsection*{Brompton \hspace*{\fill}\nolinebreak[1]%
\enspace\hspace*{\fill}
\finalhyphendemerits=0
[9th June]}

\index{Brompton , Kensington and Chelsea@Brompton, \emph{Kensington \& Chelsea}}

Death of Iain Hanham (C).

\noindent
\begin{tabular*}{\columnwidth}{@{\extracolsep{\fill}} p{0.545\columnwidth} >{\itshape}l r @{\extracolsep{\fill}}}
Louis Mosley & C & 728\\
Mark Sautter & Lab & 101\\
Mary Harris & LD & 86\\
\end{tabular*}

\subsubsection*{Queen's Gate \hspace*{\fill}\nolinebreak[1]%
\enspace\hspace*{\fill}
\finalhyphendemerits=0
[9th June]}

\index{Queen's Gate , Kensington and Chelsea@Queen's Gate, \emph{Kensington \& Chelsea}}

Death of Andrew Dalton (C).

\noindent
\begin{tabular*}{\columnwidth}{@{\extracolsep{\fill}} p{0.545\columnwidth} >{\itshape}l r @{\extracolsep{\fill}}}
Sam Mackover & C & 663\\
John Blamey & LD & 100\\
Keith Stirling & Lab & 82\\
David Coburn & UKIP & 54\\
\end{tabular*}

\subsubsection*{Norland \hspace*{\fill}\nolinebreak[1]%
\enspace\hspace*{\fill}
\finalhyphendemerits=0
[6th October]}

\index{Norland , Kensington and Chelsea@Norland, \emph{Kensington \& Chelsea}}

Resignation of Andrew Lamont (C).

\noindent
\begin{tabular*}{\columnwidth}{@{\extracolsep{\fill}} p{0.545\columnwidth} >{\itshape}l r @{\extracolsep{\fill}}}
Catherine Faulks & C & 675\\
Beinazir Lasharie & Lab & 438\\
Peter Kosta & LD & 358\\
David Coburn & UKIP & 69\\
\end{tabular*}

\subsection*{Redbridge}
\index{Redbridge}

\subsubsection*{Aldborough \hspace*{\fill}\nolinebreak[1]%
\enspace\hspace*{\fill}
\finalhyphendemerits=0
[10th November]}

\index{Aldborough , Redbridge@Aldborough, \emph{Redbridge}}

Resignation of Mike Figg (Lab).

\noindent
\begin{tabular*}{\columnwidth}{@{\extracolsep{\fill}} p{0.545\columnwidth} >{\itshape}l r @{\extracolsep{\fill}}}
Debbie Thiara & Lab & 1436\\
Melvyn Marks & C & 1071\\
Christopher Greaves & LD & 87\\
Paul Wiffen & UKIP & 83\\
Clive Durdle & Grn & 64\\
Daniel Warville & BNP & 34\\
\end{tabular*}

\columnbreak

\section{South London}

\subsection*{Bromley}
\index{Bromley}

\subsubsection*{Shortlands \hspace*{\fill}\nolinebreak[1]%
\enspace\hspace*{\fill}
\finalhyphendemerits=0
[20th October]}

\index{Shortlands , Bromley@Shortlands, \emph{Bromley}}

Resignation of George Taylor (C).

\noindent
\begin{tabular*}{\columnwidth}{@{\extracolsep{\fill}} p{0.545\columnwidth} >{\itshape}l r @{\extracolsep{\fill}}}
David Jefferys & C & 1480\\
Anuja Prashar & LD & 490\\
Gareth Abbit & Lab & 256\\
Emmett Jenner & UKIP & 153\\
Anna Martin & Grn & 75\\
Michael Payne & BNP & 35\\
\end{tabular*}

\subsection*{Kingston upon Thames}
\index{Kingston upon Thames}

CPA = Christian Peoples Alliance

\subsubsection*{Surbiton Hill \hspace*{\fill}\nolinebreak[1]%
\enspace\hspace*{\fill}
\finalhyphendemerits=0
[15th September]}

\index{Surbiton Hill , Kingston upon Thames@Surbiton Hill, \emph{Kingston upon Thames}}

Resignation of Umesh Parekh (LD).

\noindent
\begin{tabular*}{\columnwidth}{@{\extracolsep{\fill}} p{0.56\columnwidth} >{\itshape}l r @{\extracolsep{\fill}}}
John Ayles & LD & 997\\
Nick Kilby & C & 895\\
Katie Hill & Lab & 349\\
Paul Pickhaver & CPA & 171\\
Chris Walker & Grn & 81\\
James Riding & Ind & 21\\
\end{tabular*}

\subsubsection*{Coombe Vale (2) \hspace*{\fill}\nolinebreak[1]%
\enspace\hspace*{\fill}
\finalhyphendemerits=0
[15th December]}

\index{Coombe Vale , Kingston upon Thames@Coombe Vale, \emph{Kingston upon Thames}}

\sloppyword{Resignations of Robert-John Tasker and James White (both C).}

\noindent
\begin{tabular*}{\columnwidth}{@{\extracolsep{\fill}} p{0.56\columnwidth} >{\itshape}l r @{\extracolsep{\fill}}}
Lynne Finnerty & C & 1340\\
Julie Pickering & C & 1308\\
Kamala Jugan & LD & 908\\
Rupert Nichol & LD & 778\\
Nick Brown & Lab & 526\\
Ian Parker & Lab & 502\\
Chris Walker & Grn & 122\\
Tariq Shabbeer & Grn & 108\\
Philippa Hayward & CPA & 94\\
Roger Glencross & CPA & 76\\
Michael Watson & UKIP & 70\\
\end{tabular*}

\subsection*{Lewisham}
\index{Lewisham}

\subsubsection*{Bellingham \hspace*{\fill}\nolinebreak[1]%
\enspace\hspace*{\fill}
\finalhyphendemerits=0
[24th March]}

\index{Bellingham , Lewisham@Bellingham, \emph{Lewisham}}

Death of Ron Stockbridge (Lab).

\noindent
\begin{tabular*}{\columnwidth}{@{\extracolsep{\fill}} p{0.54\columnwidth} >{\itshape}l r @{\extracolsep{\fill}}}
Jacq Paschoud & Lab & 1100\\
Simon Nundy & C & 340\\
Jenni Steele & LD & 334\\
Ian Page & SocAlt & 264\\
Phil Laurie & Grn & 100\\
\end{tabular*}

\subsection*{Southwark}
\index{Southwark}

\subsubsection*{Brunswick Park \hspace*{\fill}\nolinebreak[1]%
\enspace\hspace*{\fill}
\finalhyphendemerits=0
[10th March]}

\index{Brunswick Park , Southwark@Brunswick Park, \emph{Southwark}}

Resignation of John Friary (Lab).

\noindent
\begin{tabular*}{\columnwidth}{@{\extracolsep{\fill}} p{0.545\columnwidth} >{\itshape}l r @{\extracolsep{\fill}}}
Mark Williams & Lab & 1981\\
Kate Heywood & LD & 630\\
Jenny Bentall & Grn & 231\\
Simon Kitchen & C & 129\\
Brian Kelly & TUSC & 70\\
\end{tabular*}

\subsubsection*{The Lane \hspace*{\fill}\nolinebreak[1]%
\enspace\hspace*{\fill}
\finalhyphendemerits=0
[5th May]}

\index{Lane , Southwark@The Lane, \emph{Southwark}}

Resignation of Keadean Rhoden (Ind elected as Lab).

\noindent
\begin{tabular*}{\columnwidth}{@{\extracolsep{\fill}} p{0.545\columnwidth} >{\itshape}l r @{\extracolsep{\fill}}}
Rowenna Davis & Lab & 2670\\
Anna Plodowski & Grn & 472\\
Alex Berhanu & LD & 471\\
Simon Fox & C & 423\\
Brian Kelly & TUSC & 107\\
\end{tabular*}

\subsubsection*{Peckham \hspace*{\fill}\nolinebreak[1]%
\enspace\hspace*{\fill}
\finalhyphendemerits=0
[7th July]}

\index{Peckham , Southwark@Peckham, \emph{Southwark}}

Death of Tayo Situ (Lab).

\noindent
\begin{tabular*}{\columnwidth}{@{\extracolsep{\fill}} p{0.545\columnwidth} >{\itshape}l r @{\extracolsep{\fill}}}
Chris Brown & Lab & 1754\\
Jennifer Blake & LD & 554\\
Diana Atuona & C & 86\\
Brian Kelly & TUSC & 63\\
Jason Evers & Grn & 46\\
\end{tabular*}

\subsection*{Wandsworth}
\index{Wandsworth}

\subsubsection*{Thamesfield \hspace*{\fill}\nolinebreak[1]%
\enspace\hspace*{\fill}
\finalhyphendemerits=0
[30th June]}

\index{Thamesfield , Wandsworth@Thamesfield, \emph{Wandsworth}}

Resignation of Edward Lister (C).

\noindent
\begin{tabular*}{\columnwidth}{@{\extracolsep{\fill}} p{0.56\columnwidth} >{\itshape}l r @{\extracolsep{\fill}}}
Mike Ryder & C & 1497\\
Christian Klapp & Lab & 1022\\
Lisa Smart & LD & 545\\
Marian Hoffman & Grn & 202\\
\end{tabular*}

\section{Greater Manchester}

\subsection*{Bolton}
\index{Bolton}

\subsubsection*{Horwich North East \hspace*{\fill}\nolinebreak[1]%
\enspace\hspace*{\fill}
\finalhyphendemerits=0
[5th May]}

\index{Horwich North East , Bolton@Horwich N.E., \emph{Bolton}}

Resignation of Barbara Ronson (LD).

This by-election was combined with the 2011 ordinary election.

\subsection*{Manchester}
\index{Manchester}

\subsubsection*{Baguley \hspace*{\fill}\nolinebreak[1]%
\enspace\hspace*{\fill}
\finalhyphendemerits=0
[20th January]}

\index{Baguley , Manchester@Baguley, \emph{Manchester}}

Death of Eddie McCulley (Lab).

\noindent
\begin{tabular*}{\columnwidth}{@{\extracolsep{\fill}} p{0.545\columnwidth} >{\itshape}l r @{\extracolsep{\fill}}}
Tracy Rawlins & Lab & 996\\
Ralph Ellerton & C & 160\\
Christopher Cassidy & UKIP & 76\\
Yvonne Donaghey & LD & 52\\
Bernard Todd & BNP & 52\\
Mike Dagley & Grn & 51\\
Honor Donnelly & Ind & 19\\
\end{tabular*}

\subsubsection*{Burnage \hspace*{\fill}\nolinebreak[1]%
\enspace\hspace*{\fill}
\finalhyphendemerits=0
[5th May]}

\index{Burnage , Manchester@Burnage, \emph{Manchester}}

Resignation of John Cameron (Lab).

This by-election was combined with the 2011 ordinary election.
%; see page \pageref{BurnageManchester} for the result.

\subsection*{Salford}
\index{Salford}

\subsubsection*{Walkden North \hspace*{\fill}\nolinebreak[1]%
\enspace\hspace*{\fill}
\finalhyphendemerits=0
[3rd March]}

\index{Walkden North , Salford@Walkden N., \emph{Salford}}

Death of Vincent Devine (Lab).

\noindent
\begin{tabular*}{\columnwidth}{@{\extracolsep{\fill}} p{0.545\columnwidth} >{\itshape}l r @{\extracolsep{\fill}}}
Brenden Ryan & Lab & 1291\\
Chris Bates & C & 209\\
Laurence Depares & EDP & 125\\
Keith Fairhurst & BNP & 92\\
Susan Carson & LD & 62\\
\end{tabular*}

\subsubsection*{Eccles \hspace*{\fill}\nolinebreak[1]%
\enspace\hspace*{\fill}
\finalhyphendemerits=0
[20th October]}

\index{Eccles , Salford@Eccles, \emph{Salford}}

Death of John Cullen (Lab).

\noindent
\begin{tabular*}{\columnwidth}{@{\extracolsep{\fill}} p{0.545\columnwidth} >{\itshape}l r @{\extracolsep{\fill}}}
Michael Wheeler & Lab & 1227\\
Nicholas Johnson & C & 701\\
Kay Pollitt & BNP & 147\\
Valerie Kelly & LD & 125\\
Alan Valentine & Ind & 53\\
\end{tabular*}

\subsection*{Stockport}
\index{Stockport}

\subsubsection*{Reddish North \hspace*{\fill}\nolinebreak[1]%
\enspace\hspace*{\fill}
\finalhyphendemerits=0
[5th May]}

\index{Reddish North , Stockport@Reddish N., \emph{Stockport}}

Resignation of Peter Scott (Lab).

This by-election was combined with the 2011 ordinary election.
%; see page \pageref{ReddishNorthStockport} for the result.

\subsection*{Trafford}
\index{Trafford}

\subsubsection*{Altrincham \hspace*{\fill}\nolinebreak[1]%
\enspace\hspace*{\fill}
\finalhyphendemerits=0
[5th May]}

\index{Altrincham , Trafford@Altrincham, \emph{Trafford}}

Resignation of Susan Williams (C).

This by-election was combined with the 2011 ordinary election.
%; see page \pageref{AltrinchamTrafford} for the result.

\subsection*{Wigan}
\index{Wigan}

WiganIC = Wigan Independent Conservative

\subsubsection*{Wigan Central \hspace*{\fill}\nolinebreak[1]%
\enspace\hspace*{\fill}
\finalhyphendemerits=0
[3rd March; Lab gain from C]}

\index{Wigan Central , Wigan@Wigan C., \emph{Wigan}}

Resignation of Henry Cadman (C).

\noindent
\begin{tabular*}{\columnwidth}{@{\extracolsep{\fill}} p{0.48\columnwidth} >{\itshape}l r @{\extracolsep{\fill}}}
Lawrence Hunt & Lab & 1165\\
Robin Gibson & C & 652\\
Gareth Fairhurst & WiganIC & 393\\
Keith Jones & UKIP & 189\\
\end{tabular*}

\section{Merseyside}

\subsection*{Knowsley}
\index{Knowsley}

\subsubsection*{Page Moss \hspace*{\fill}\nolinebreak[1]%
\enspace\hspace*{\fill}
\finalhyphendemerits=0
[18th August]}

\index{Page Moss , Knowsley@Page Moss, \emph{Knowsley}}

Resignation of Tommy Russell (Lab).

\noindent
\begin{tabular*}{\columnwidth}{@{\extracolsep{\fill}} p{0.545\columnwidth} >{\itshape}l r @{\extracolsep{\fill}}}
Dave Tulley & Lab & 541\\
Matt Hughes & LD & 57\\
Sean Watson & Ind & 22\\
Marie Rea & Grn & 21\\
Robert Avery & C & 15\\
\end{tabular*}

\subsection*{Wirral}
\index{Wirral}

\subsubsection*{Greasby, Frankby and Irby \hspace*{\fill}\nolinebreak[1]%
\enspace\hspace*{\fill}
\finalhyphendemerits=0
[5th May]}

\index{Greasby, Frankby and Irby , Wirral@Greasby, Frankby \& Irby, \emph{Wirral}}

Resignation of Gill Gardiner (LD).

This by-election was combined with the 2011 ordinary election.
%; see page \pageref{GreasbyFrankbyIrbyWirral} for the result.

\columnbreak

\section{South Yorkshire}

\subsection*{Barnsley}
\index{Barnsley}

\subsubsection*{St Helens \hspace*{\fill}\nolinebreak[1]%
\enspace\hspace*{\fill}
\finalhyphendemerits=0
[13th October]}

\index{Saint Helens , Barnsley@St Helens, \emph{Barnsley}}

Resignation of Roy Butterwood (Lab).

\noindent
\begin{tabular*}{\columnwidth}{@{\extracolsep{\fill}} p{0.545\columnwidth} >{\itshape}l r @{\extracolsep{\fill}}}
Dave Leech & Lab & 1257\\
Danny Cooke & BNP & 174\\
Kevin Riddiough & EDP & 146\\
Clive Watkinson & C & 61\\
Edward Goughwaite & Ind & 21\\
\end{tabular*}

\section{Tyne and Wear}

\subsection*{Newcastle upon Tyne}
\index{Newcastle upon Tyne}

NuTCFP = Newcastle upon Tyne Community First Party (It's Time to Put Newcastle First)

\subsubsection*{Elswick \hspace*{\fill}\nolinebreak[1]%
\enspace\hspace*{\fill}
\finalhyphendemerits=0
[5th May]}

\index{Elswick , Newcastle upon Tyne@Elswick, \emph{Newcastle upon Tyne}}

Death of Doreen James (Lab).

This by-election was combined with the 2011 ordinary election.
%; see page \pageref{ElswickNewcastleTyne} for the result.

\subsubsection*{Byker \hspace*{\fill}\nolinebreak[1]%
\enspace\hspace*{\fill}
\finalhyphendemerits=0
[16th June]}

\index{Byker , Newcastle upon Tyne@Byker, \emph{Newcastle upon Tyne}}

Ordinary election postponed from 5th May: death of candidate Alice Gingell (C).

\noindent
\begin{tabular*}{\columnwidth}{@{\extracolsep{\fill}} p{0.49\columnwidth} >{\itshape}l r @{\extracolsep{\fill}}}
	Veronica Dunn & Lab & 1206\\
	Ken Booth & BNP & 144\\
	Tracy Connell & LD & 102\\
	Alan Mattinson & Ind & 78\\
	James Bartle & C & 76\\
	Angela McKenna & NuTCFP & 55\\
\end{tabular*}

\subsubsection*{Westerhope \hspace*{\fill}\nolinebreak[1]%
\enspace\hspace*{\fill}
\finalhyphendemerits=0
[16th June]}

\index{Westerhope , Newcastle upon Tyne@Westerhope, \emph{Newcastle upon Tyne}}

Ordinary election on 5th May abandoned: death of candidate Neil Hamilton (LD) on polling day.

\noindent
\begin{tabular*}{\columnwidth}{@{\extracolsep{\fill}} p{0.49\columnwidth} >{\itshape}l r @{\extracolsep{\fill}}}
	Linda Hobson & Lab & 1106\\
	Pat Hillicks & Ind & 883\\
	Ernie Shorton & NuTCFP & 532\\
	PJ Morrissey & LD & 492\\
	William Holloway & C & 240\\
	Anita Cooper & BNP & 81\\
\end{tabular*}

\subsection*{Sunderland}
\index{Sunderland}

\subsubsection*{Sandhill \hspace*{\fill}\nolinebreak[1]%
\enspace\hspace*{\fill}
\finalhyphendemerits=0
[5th May]}

\index{Sandhill , Sunderland@Sandhill, \emph{Sunderland}}

Resignation of Jim Scott (Lab).

This by-election was combined with the 2011 ordinary election.
%; see page \pageref{SandhillSunderland} for the result.

\columnbreak

\section{West Midlands}

\subsection*{Birmingham}
\index{Birmingham}

\subsubsection*{Sparkbrook \hspace*{\fill}\nolinebreak[1]%
\enspace\hspace*{\fill}
\finalhyphendemerits=0
[10th November; Lab gain from Respect]}

\index{Sparkbrook , Birmingham@\sloppyword{Sparkbrook, \emph{Birmingham}}}

Resignation of Salma Yaqoob (Respect).

\noindent
\begin{tabular*}{\columnwidth}{@{\extracolsep{\fill}} p{0.52\columnwidth} >{\itshape}l r @{\extracolsep{\fill}}}
Victoria Quinn & Lab & 3932\\
Mohammed Ishtiaq & Respect & 2301\\
Adil Rashid & LD & 395\\
Peter Tinsley & Grn & 179\\
Sahar Rezazadeh & C & 133\\
\end{tabular*}

\subsection*{Coventry}
\index{Coventry}

\subsubsection*{Lower Stoke \hspace*{\fill}\nolinebreak[1]%
\enspace\hspace*{\fill}
\finalhyphendemerits=0
[6th October]}

\index{Lower Stoke , Coventry@Lower Stoke, \emph{Coventry}}

Death of Jack Harrison (Lab).

\noindent
\begin{tabular*}{\columnwidth}{@{\extracolsep{\fill}} p{0.54\columnwidth} >{\itshape}l r @{\extracolsep{\fill}}}
Catherine Miks & Lab & 1366\\
Jaswant Singh Birdi & C & 563\\
Robert McArdle & SocAlt & 254\\
Keith Oxford & BNP & 149\\
Laura Vesty & Grn & 114\\
Mark Widdop & LD & 79\\
\end{tabular*}

\subsection*{Solihull}
\index{Solihull}

SMRA = Solihull and Meriden Residents Association

At the May 2011 ordinary election there was an unfilled vacancy in Shirley West ward due to the disqualification (non-attendance) of Brynn Tudor (LD).

\index{Shirley West , Solihull@Shirley W., \emph{Solihull}}

\subsubsection*{Olton \hspace*{\fill}\nolinebreak[1]%
\enspace\hspace*{\fill}
\finalhyphendemerits=0
[20th January]}

\index{Olton , Solihull@Olton, \emph{Solihull}}

Death of Honor Cox (LD).

\noindent
\begin{tabular*}{\columnwidth}{@{\extracolsep{\fill}} p{0.53\columnwidth} >{\itshape}l r @{\extracolsep{\fill}}}
Claire O'Kane & LD & 1188\\
David Price & C & 1179\\
Andrew Mullinex & Lab & 280\\
Hayley Watts & SMRA & 228\\
Ian Jamieson & Grn & 115\\
\end{tabular*}

\subsection*{Walsall}
\index{Walsall}

\subsubsection*{Aldridge Central and South \hspace*{\fill}\nolinebreak[1]%
\enspace\hspace*{\fill}
\finalhyphendemerits=0
[5th May]}

\index{Aldridge Central and South , Walsall@Aldridge C. \& S., \emph{Walsall}}

Resignation of John O'Hare (C).

This by-election was combined with the 2011 ordinary election.
%; see page \pageref{AldridgeCentralSouthWalsall} for the result.

\subsubsection*{Bloxwich East \hspace*{\fill}\nolinebreak[1]%
\enspace\hspace*{\fill}
\finalhyphendemerits=0
[27th October; Lab gain from C]}

\index{Bloxwich East , Walsall@Bloxwich E., \emph{Walsall}}

Death of Bill Tweddle (C).

\noindent
\begin{tabular*}{\columnwidth}{@{\extracolsep{\fill}} p{0.545\columnwidth} >{\itshape}l r @{\extracolsep{\fill}}}
Julie Fitzpatrick & Lab & 922\\
Les Beeley & C & 834\\
Derek Bennett & UKIP & 98\\
Chris Newey & EDP & 49\\
Leandra Gebrakedan & Grn & 16\\
\end{tabular*}

\subsubsection*{Birchills Leamore \hspace*{\fill}\nolinebreak[1]%
\enspace\hspace*{\fill}
\finalhyphendemerits=0
[22nd December]}

\index{Birchills Leamore , Walsall@Birchills Leamore, \emph{Walsall}}

Death of Joan Barton (Lab).

\noindent
\begin{tabular*}{\columnwidth}{@{\extracolsep{\fill}} p{0.56\columnwidth} >{\itshape}l r @{\extracolsep{\fill}}}
Tina Jukes & Lab & 835\\
Aftab Kamran & C & 512\\
Chris Newey & EDP & 130\\
Liz Hazell & UKIP & 59\\
Leandra Gebrakedan & Grn & 46\\
\end{tabular*}

\subsection*{Wolverhampton}
\index{Wolverhampton}

\subsubsection*{Graiseley \hspace*{\fill}\nolinebreak[1]%
\enspace\hspace*{\fill}
\finalhyphendemerits=0
[15th September]}

\index{Graiseley , Wolverhampton@Graiseley, \emph{Wolverhampton}}

Death of Mohan Passi (Lab).

\noindent
\begin{tabular*}{\columnwidth}{@{\extracolsep{\fill}} p{0.56\columnwidth} >{\itshape}l r @{\extracolsep{\fill}}}
Jacqueline Sweetman & Lab & 1527\\
John Mellor & C & 591\\
Eileen Birch & LD & 177\\
Don Cooper & UKIP & 65\\
\end{tabular*}

\section{West Yorkshire}

\subsection*{Bradford}
\index{Bradford}

\subsubsection*{Idle and Thackley \hspace*{\fill}\nolinebreak[1]%
\enspace\hspace*{\fill}
\finalhyphendemerits=0
[5th May]}

\index{Idle and Thackley , Bradford@Idle \& Thackley, \emph{Bradford}}

Resignation of Ed Hall (LD).

This by-election was combined with the 2011 ordinary election.
%; see page \pageref{IdleThackleyBradford} for the result.

\subsubsection*{Great Horton \hspace*{\fill}\nolinebreak[1]%
\enspace\hspace*{\fill}
\finalhyphendemerits=0
[24th November]}

\index{Great Horton , Bradford@Great Horton, \emph{Bradford}}

Resignation of Rev Paul Flowers (Lab).

\noindent
\begin{tabular*}{\columnwidth}{@{\extracolsep{\fill}} p{0.545\columnwidth} >{\itshape}l r
@{\extracolsep{\fill}}}
Abdul Jabar & Lab & 1993\\
Mehrban Hussain & C & 705\\
Mary Slingsby & LD & 337\\
Jason Smith & UKIP & 294\\
Celia Hickson & Grn & 73\\
\end{tabular*}

\subsection*{Wakefield}
\index{Wakefield}

\subsubsection*{Horbury and South Ossett \hspace*{\fill}\nolinebreak[1]%
\enspace\hspace*{\fill}
\finalhyphendemerits=0
[7th July; Lab gain from C]}

\index{Horbury and South Ossett , Wakefield@Horbury \& South Ossett, \emph{Wakefield}}

Resignation of Elizabeth Hick (C).

\noindent
\begin{tabular*}{\columnwidth}{@{\extracolsep{\fill}} p{0.545\columnwidth} >{\itshape}l r @{\extracolsep{\fill}}}
Janet Holmes & Lab & 1776\\
Richard Wakefield & C & 1061\\
David Dews & UKIP & 232\\
Mark Goodair & LD & 200\\
Norman Tate & Ind & 93\\
Mark Harrop & Ind & 88\\
\end{tabular*}

\section{Bedfordshire}

\subsection*{Bedford}
\index{Bedford}

\index{Harpur , Bedford@Harpur, \emph{Bedford}}

At the May 2011 ordinary election there was an unfilled vacancy in Harpur ward due to the death of Brian Dillingham (C).

\council{Central Bedfordshire}
%\index{Central Bedfordshire@Central Beds.}

\index{Dunstable Downs , Central Bedfordshire@Dunstable Downs, \emph{C. Beds.}}

At the May 2011 ordinary election there was an unfilled vacancy in Dunstable Downs ward due to the death of Tony Green (C).

\section{Berkshire}

\subsection*{Reading}
\index{Reading}

\index{Whitley , Reading@Whitley, \emph{Reading}}

At the May 2011 ordinary election there was an unfilled vacancy in Whitley ward due to the death of Jim Hanley (Lab).

\subsection*{Slough}
\index{Slough}

\index{Foxborough , Slough@Foxborough, \emph{Slough}}

At the May 2011 ordinary election there was an unfilled vacancy in Foxborough ward due to the death of Sonja Jenkins (LD).

\council{West Berkshire}

\index{Mortimer , West Berkshire@Mortimer, \emph{W. Berks.}}

At the May 2011 ordinary election there was an unfilled vacancy in Mortimer ward due to the death of Keith Lock (LD).

\council{Windsor and Maidenhead}

\subsubsection*{Park \hspace*{\fill}\nolinebreak[1]%
\enspace\hspace*{\fill}
\finalhyphendemerits=0
[6th January]}

\index{Park , Windsor and Maidenhead@Park, \emph{Windsor \& Maidenhead}}

Resignation of Richard Gadd (C).

\noindent
\begin{tabular*}{\columnwidth}{@{\extracolsep{\fill}} p{0.545\columnwidth} >{\itshape}l r @{\extracolsep{\fill}}}
Natasha Lavender & C & 637\\
Richard Fagence & LD & 156\\
Laura Binnie & Lab & 149\\
Derek Prime & Ind & 47\\
\end{tabular*}

\subsubsection*{Eton and Castle \hspace*{\fill}\nolinebreak[1]%
\enspace\hspace*{\fill}
\finalhyphendemerits=0
[11th August; LD gain from C]}

\index{Eton and Castle , Windsor and Maidenhead@Eton \& Castle, \emph{Windsor \& Maidenhead}}

Resignation of Liam Maxwell (C).

\noindent
\begin{tabular*}{\columnwidth}{@{\extracolsep{\fill}} p{0.545\columnwidth} >{\itshape}l r @{\extracolsep{\fill}}}
George Fussey & LD & 208\\
Adam Demeter & C & 182\\
George Davidson & Lab & 32\\
John-Paul Rye & UKIP & 17\\
\end{tabular*}

\subsection*{Wokingham}
\index{Wokingham}

\subsubsection*{\sloppyword{Remenham, Wargrave and Ruscombe} \hspace*{\fill}\nolinebreak[1]%
\enspace\hspace*{\fill}
\finalhyphendemerits=0
[21st July]}

\index{Remenham, Wargrave and Ruscombe , Wokingham@\sloppyword{Remenham, Wargrave \& Ruscombe, \emph{Wokingham}}}

Resignation of Claire Stretton (C).

\noindent
\begin{tabular*}{\columnwidth}{@{\extracolsep{\fill}} p{0.545\columnwidth} >{\itshape}l r @{\extracolsep{\fill}}}
John Halsall & C & 850\\
Martin Alder & LD & 272\\
Matthew Dent & Lab & 94\\
Andy Heape & UKIP & 55\\
Martyn Foss & Grn & 19\\
\end{tabular*}

\section{Bristol}
\index{Bristol}

\subsubsection*{Southmead \hspace*{\fill}\nolinebreak[1]%
\enspace\hspace*{\fill}
\finalhyphendemerits=0
[8th September; Lab gain from LD]}

\index{Southmead , Bristol@Southmead, \emph{Bristol}}

Resignation of Jacqueline Bowles (LD).

\noindent
\begin{tabular*}{\columnwidth}{@{\extracolsep{\fill}} p{0.56\columnwidth} >{\itshape}l r @{\extracolsep{\fill}}}
Brenda Massey & Lab & 1109\\
Rondo Brace & C & 765\\
Barry Cash & LD & 365\\
Chris Millman & Grn & 120\\
Stephen Wright & EDP & 77\\
\end{tabular*}

\section{Buckinghamshire}

\subsection*{Wycombe}
\index{Wycombe}

\subsubsection*{Hazlemere South \hspace*{\fill}\nolinebreak[1]%
\enspace\hspace*{\fill}
\finalhyphendemerits=0
[24th November; LD gain from C]}

\index{Hazlemere South , Wycombe@Hazlemere S., \emph{Wycombe}}

Resignation of Bob Bate (C).

\noindent
\begin{tabular*}{\columnwidth}{@{\extracolsep{\fill}} p{0.545\columnwidth} >{\itshape}l r @{\extracolsep{\fill}}}
Alex Slater & LD & 412\\
Brian Mapletoft & UKIP & 365\\
Lawrence Wood & C & 228\\
Alan De'Ath & Lab & 88\\
\end{tabular*}

\section{Cambridgeshire}

\subsection*{County Council}
\index{Cambridgeshire}

\subsubsection*{March North \hspace*{\fill}\nolinebreak[1]%
\enspace\hspace*{\fill}
\finalhyphendemerits=0
[3rd March]}

\index{March North , Cambridgeshire@March N., \emph{Cambs.}}

Death of John West (C).

\noindent
\begin{tabular*}{\columnwidth}{@{\extracolsep{\fill}} p{0.545\columnwidth} >{\itshape}l r @{\extracolsep{\fill}}}
Steve County & C & 616\\
Louis Sugden & Lab & 282\\
William McAdam & LD & 277\\
\end{tabular*}

\subsubsection*{Arbury \hspace*{\fill}\nolinebreak[1]%
\enspace\hspace*{\fill}
\finalhyphendemerits=0
[5th May; Lab gain from LD]}

\index{Arbury , Cambridgeshire@Arbury, \emph{Cambs.}}

Resignation of Rupert Moss-Eccardt (LD).

\noindent
\begin{tabular*}{\columnwidth}{@{\extracolsep{\fill}} p{0.545\columnwidth} >{\itshape}l r @{\extracolsep{\fill}}}
Paul Sales & Lab & 1214\\
Amy Ellis & LD & 1078\\
Shapour Meftah & C & 496\\
Martin Bonner & Grn & 411\\
\end{tabular*}

\subsection*{Cambridge}
\index{Cambridge}

\subsubsection*{Cherry Hinton \hspace*{\fill}\nolinebreak[1]%
\enspace\hspace*{\fill}
\finalhyphendemerits=0
[5th May]}

\index{Cherry Hinton , Cambridge@Cherry Hinton, \emph{Cambridge}}

Resignation of Stuart Newbold (Lab).

This by-election was combined with the 2011 ordinary election.
%; see page \pageref{CherryHintonCambridge} for the result.

\council{East Cambridgeshire}

\index{Fordham Villages , East Cambridgeshire@\sloppyword{Fordham Villages, \emph{E. Cambs.}}}

At the May 2011 ordinary election there was an unfilled vacancy in Fordham Villages ward due to the death of John Abbott (LD).

\subsection*{Fenland}
\index{Fenland}

\index{March North , Fenland@March N., \emph{Fenland}}

At the May 2011 ordinary election there was an unfilled vacancy in March North ward due to the death of John West (C).

\subsubsection*{Staithe \hspace*{\fill}\nolinebreak[1]%
\enspace\hspace*{\fill}
\finalhyphendemerits=0
[22nd September]}

\index{Staithe , Fenland@Staithe, \emph{Fenland}}

Death of Roger Green (C).

\noindent
\begin{tabular*}{\columnwidth}{@{\extracolsep{\fill}} p{0.545\columnwidth} >{\itshape}l r @{\extracolsep{\fill}}}
David Hodgson & C & 228\\
John White & Lab & 166\\
Robert McLaren & LD & 90\\
William Schooling & UKIP & 39\\
Phil Webb & Ind & 20\\
\end{tabular*}

\subsection*{Huntingdonshire}
\index{Huntingdonshire}

\subsubsection*{The Hemingfords \hspace*{\fill}\nolinebreak[1]%
\enspace\hspace*{\fill}
\finalhyphendemerits=0
[5th May]}

\index{Hemingfords, The , Huntingdonshire@The Hemingfords, \emph{Hunts.}}

Resignation of Chris Stephens (C).

This by-election was combined with the 2011 ordinary election.
%; see page \pageref{TheHemingfordsHuntingdonshire} for the result.

\subsubsection*{St Neots Eaton Socon \hspace*{\fill}\nolinebreak[1]%
\enspace\hspace*{\fill}
\finalhyphendemerits=0
[5th May]}

\index{Saint Neots Eaton Socon , Huntingdonshire@St Neots Eaton Socon, \emph{Hunts.}}

Resignation of Mandy Thomas (C).

This by-election was combined with the 2011 ordinary election.
%; see page \pageref{StNeotsEatonSoconHuntingdonshire} for the result.

\council{South Cambridgeshire}

\subsubsection*{Bourn \hspace*{\fill}\nolinebreak[1]%
\enspace\hspace*{\fill}
\finalhyphendemerits=0
[17th February]}

\index{Bourn , South Cambridgeshire@Bourn, \emph{S. Cambs.}}

Resignation of David Morgan (C).

\noindent
\begin{tabular*}{\columnwidth}{@{\extracolsep{\fill}} p{0.545\columnwidth} >{\itshape}l r @{\extracolsep{\fill}}}
Clayton Hudson & C & 874\\
Nick Glynn & LD & 345\\
Gavin Clayton & Lab & 337\\
\end{tabular*}

\section{Cheshire}

\council{Cheshire East}

\subsubsection*{Crewe South (2) \hspace*{\fill}\nolinebreak[1]%
\enspace\hspace*{\fill}
\finalhyphendemerits=0
[16th June]}

\index{Crewe South , Cheshire East@Crewe S., \emph{Ches. E.}}

Ordinary election postponed from 5th May; death of outgoing councillor Betty Howell (LD) who had been nominated for re-election.

\noindent
\begin{tabular*}{\columnwidth}{@{\extracolsep{\fill}} p{0.545\columnwidth} >{\itshape}l r @{\extracolsep{\fill}}}
	Dorothy Flude & Lab & 970\\
	Steven Hogben & Lab & 899\\
	Jubeyar Ahmed & C & 507\\
	Steve Turnbull & C & 489\\
	Lisa Smith & LD & 147\\
	Robert Icke & LD & 146\\
\end{tabular*}

\council{Cheshire West and Chester}
%\index{Cheshire West and Chester@Cheshire W. \& Chester}

\subsubsection*{Ellesmere Port Town \hspace*{\fill}\nolinebreak[1]%
\enspace\hspace*{\fill}
\finalhyphendemerits=0
[20th October]}

\index{Ellesmere Port Town , Cheshire West and Chester@\sloppyword{Ellesmere Port Town, \emph{Ches. W. \& Chester}}}

Death of Derek Bateman (Lab).

\noindent
\begin{tabular*}{\columnwidth}{@{\extracolsep{\fill}} p{0.545\columnwidth} >{\itshape}l r @{\extracolsep{\fill}}}
Lynn Clare & Lab & 686\\
Graham Pritchard & C & 102\\
Kenny Spain & SocLab & 65\\
Andrew Roberts & UKIP & 64\\
Hilary Chrusciezl & LD & 41\\
\end{tabular*}

\subsection*{Warrington}
\index{Warrington}

\subsubsection*{Westbrook \hspace*{\fill}\nolinebreak[1]%
\enspace\hspace*{\fill}
\finalhyphendemerits=0
[5th May]}

\index{Westbrook , Warrington@Westbrook, \emph{Warrington}}

Resignation of Natalie Douglas (Lab).

This by-election was combined with the 2011 ordinary election.
%; see page \pageref{WestbrookWarrington} for the result.

\subsubsection*{Poulton North \hspace*{\fill}\nolinebreak[1]%
\enspace\hspace*{\fill}
\finalhyphendemerits=0
[28th July]}

\index{Poulton North , Warrington@Poulton N., \emph{Warrington}}

Resignation of Sharon Wilson (LD).

\noindent
\begin{tabular*}{\columnwidth}{@{\extracolsep{\fill}} p{0.545\columnwidth} >{\itshape}l r @{\extracolsep{\fill}}}
Colin Oliver & LD & 1106\\
Ashley Pemberton & Lab & 895\\
Mark Chapman & C & 190\\
James Ashington & UKIP & 97\\
\end{tabular*}

\subsubsection*{Poulton North \hspace*{\fill}\nolinebreak[1]%
\enspace\hspace*{\fill}
\finalhyphendemerits=0
[17th November]}

\index{Poulton North , Warrington@Poulton N., \emph{Warrington}}

Death of Colin Oliver (LD).

\noindent
\begin{tabular*}{\columnwidth}{@{\extracolsep{\fill}} p{0.545\columnwidth} >{\itshape}l r @{\extracolsep{\fill}}}
Sandra Bradshaw & LD & 776\\
Billy Lines-Rowlands & Lab & 733\\
Mark Chapman & C & 147\\
James Ashington & UKIP & 79\\
\end{tabular*}

\columnbreak

\section{Cornwall}
\index{Cornwall}

MK = Mebyon Kernow

\subsubsection*{Camborne North \hspace*{\fill}\nolinebreak[1]%
\enspace\hspace*{\fill}
\finalhyphendemerits=0
[13th January; Lab gain from C]}

\index{Camborne North , Cornwall@Camborne N., \emph{Cornwall}}

Resignation of Bill Jenkin (Ind elected as C).

\noindent
\begin{tabular*}{\columnwidth}{@{\extracolsep{\fill}} p{0.545\columnwidth} >{\itshape}l r @{\extracolsep{\fill}}}
Jude Robinson & Lab & 230\\
Dennis Pascoe & C & 203\\
Anna Pascoe & LD & 152\\
Paul Holmes & Lib & 61\\
Mike Champion & MK & 32\\
Jacqui Merrick & Grn & 31\\
\end{tabular*}

\subsubsection*{Bude North and Stratton \hspace*{\fill}\nolinebreak[1]%
\enspace\hspace*{\fill}
\finalhyphendemerits=0
[27th October]}

\index{Bude North and Stratton , Cornwall@Bude N. \& Stratton, \emph{Cornwall}}

Resignation of Nathan Bale (LD).

\noindent
\begin{tabular*}{\columnwidth}{@{\extracolsep{\fill}} p{0.545\columnwidth} >{\itshape}l r @{\extracolsep{\fill}}}
David Parsons & LD & 958\\
Trevor Macey & C & 395\\
Adrian Jones & Lab & 120\\
Louise Emo & Ind & 93\\
\end{tabular*}

\subsubsection*{Wendron \hspace*{\fill}\nolinebreak[1]%
\enspace\hspace*{\fill}
\finalhyphendemerits=0
[24th November; MK gain from Ind]}

\index{Wendron , Cornwall@Wendron, \emph{Cornwall}}

Death of Mike Clayton (Ind).

\noindent
\begin{tabular*}{\columnwidth}{@{\extracolsep{\fill}} p{0.545\columnwidth} >{\itshape}l r
@{\extracolsep{\fill}}}
Loveday Jenkin & MK & 427\\
John Martin & LD & 262\\
Linda Taylor & C & 227\\
Phil Martin & Ind & 117\\
Robert Webber & Lab & 80\\
\end{tabular*}

\section{Cumbria}

\subsection*{County Council}
\index{Cumbria}

\subsubsection*{Keswick and Derwent \hspace*{\fill}\nolinebreak[1]%
\enspace\hspace*{\fill}
\finalhyphendemerits=0
[5th May; C gain from LD]}

\index{Keswick and Derwent , Cumbria@Keswick \& Derwent, \emph{Cumbria}}

Resignation of Elizabeth Barraclough (LD).

\noindent
\begin{tabular*}{\columnwidth}{@{\extracolsep{\fill}} p{0.545\columnwidth} >{\itshape}l r @{\extracolsep{\fill}}}
Ronald Munby & C & 827\\
David Robinson & Ind & 650\\
Martin Pugmire & LD & 619\\
Brian Cope & Lab & 464\\
\end{tabular*}

\subsection*{Allerdale}
\index{Allerdale}

\index{Christchurch , Allerdale@Christchurch, \emph{Allerdale}}

At the May 2011 ordinary election there was an unfilled vacancy in Christchurch ward due to the death of Roy Swindells (C).

\subsubsection*{Keswick \hspace*{\fill}\nolinebreak[1]%
\enspace\hspace*{\fill}
\finalhyphendemerits=0
[1st September; LD gain from Ind]}

\index{Keswick , Allerdale@Keswick, \emph{Allerdale}}

Resignation of David Robinson (Ind).

\noindent
\begin{tabular*}{\columnwidth}{@{\extracolsep{\fill}} p{0.56\columnwidth} >{\itshape}l r @{\extracolsep{\fill}}}
Martin Pugmire & LD & 757\\
Tony Lywood & Lab & 448\\
Flic Crowley & Grn & 63\\
\end{tabular*}

\subsection*{South Lakeland}
\index{South Lakeland}

\subsubsection*{Ambleside and Grasmere \hspace*{\fill}\nolinebreak[1]%
\enspace\hspace*{\fill}
\finalhyphendemerits=0
[5th May]}

\index{Ambleside and Grasmere , South Lakeland@Ambleside \& Grasmere, \emph{S. Lakeland}}

Resignation of David Vatcher (LD).

This by-election was combined with the 2011 ordinary election.
%; see page \pageref{AmblesideGrasmereSouthLakeland} for the result.

\subsubsection*{Levens \hspace*{\fill}\nolinebreak[1]%
\enspace\hspace*{\fill}
\finalhyphendemerits=0
[5th May]}

\index{Levens , South Lakeland@Levens, \emph{S. Lakeland}}

Death of Brenda Woof (LD).

This by-election was combined with the 2011 ordinary election.
%; see page \pageref{LevensSouthLakeland} for the result.

\section{Derbyshire}

\subsection*{County Council}
\index{Derbyshire}

\subsubsection*{Eckington \hspace*{\fill}\nolinebreak[1]%
\enspace\hspace*{\fill}
\finalhyphendemerits=0
[16th June]}

\index{Eckington , Derbyshire@Eckington, \emph{Derbys.}}

Resignation of Steve Pickering (Lab).

\noindent
\begin{tabular*}{\columnwidth}{@{\extracolsep{\fill}} p{0.56\columnwidth} >{\itshape}l r @{\extracolsep{\fill}}}
Brian Ridgway & Lab & 1142\\
Carolyn Renwick & C & 406\\
\end{tabular*}

\subsubsection*{Sawley \hspace*{\fill}\nolinebreak[1]%
\enspace\hspace*{\fill}
\finalhyphendemerits=0
[7th July; Lab gain from Ind]}

\index{Sawley , Derbyshire@Sawley, \emph{Derbys.}}

Death of Bill Camm (Ind).

\noindent
\begin{tabular*}{\columnwidth}{@{\extracolsep{\fill}} p{0.56\columnwidth} >{\itshape}l r @{\extracolsep{\fill}}}
Cheryl Pidgeon & Lab & 1351\\
Chris Corbett & C & 1197\\
Fiona Aanonson & LD & 692\\
\end{tabular*}

\subsection*{Amber Valley}
\index{Amber Valley}

\subsubsection*{Heage and Ambergate \hspace*{\fill}\nolinebreak[1]%
\enspace\hspace*{\fill}
\finalhyphendemerits=0
[5th May]}

\index{Heage and Ambergate , Amber Valley@Heage \& Ambergate, \emph{Amber Valley}}

Resignation of Juliette Blake (C).

This by-election was combined with the 2011 ordinary election.
%; see page \pageref{HeageAmbergateAmberValley} for the result.

\subsubsection*{Shipley Park, Horsley and Horsley Woodhouse \hspace*{\fill}\nolinebreak[1]%
\enspace\hspace*{\fill}
\finalhyphendemerits=0
[5th May]}

\index{Shipley Park, Horsley and Horsley Woodhouse , Amber Valley@Shipley Park, Horsley \& \sloppyword{Horsley Woodhouse, \emph{Amber Valley}}}

Resignation of Nigel Mills (C).

This by-election was combined with the 2011 ordinary election.
%; see page \pageref{ShipleyParkHorsleyHorsleyWoodhouseAmberValley} for the result.

\subsection*{Bolsover}
\index{Bolsover}

\subsubsection*{Shirebrook South West \hspace*{\fill}\nolinebreak[1]%
\enspace\hspace*{\fill}
\finalhyphendemerits=0
[25th August; Lab gain from Ind]}

\index{Shirebrook South West , Bolsover@Shirebrook S.W., \emph{Bolsover}}

Death of Alan Waring (Ind).

\noindent
\begin{tabular*}{\columnwidth}{@{\extracolsep{\fill}} p{0.545\columnwidth} >{\itshape}l r @{\extracolsep{\fill}}}
Sandra Peake & Lab & 200\\
Ian Musgrove & Grn & 103\\
Bob Newholm & C & 76\\
Tom Key & BNP & 43\\
\end{tabular*}

\subsection*{Derby}
\index{Derby}

\subsubsection*{Abbey \hspace*{\fill}\nolinebreak[1]%
\enspace\hspace*{\fill}
\finalhyphendemerits=0
[5th May]}

\index{Abbey , Derby@Abbey, \emph{Derby}}

Resignation of David Batey (LD).

This by-election was combined with the 2011 ordinary election.
%; see page \pageref{AbbeyDerby} for the result.

\subsubsection*{Normanton \hspace*{\fill}\nolinebreak[1]%
\enspace\hspace*{\fill}
\finalhyphendemerits=0
[5th May]}

\index{Normanton , Derby@Normanton, \emph{Derby}}

Resignation of Chris Williamson (Lab).

This by-election was combined with the 2011 ordinary election.
%; see page \pageref{NormantonDerby} for the result.

\subsection*{Derbyshire Dales}
\index{Derbyshire Dales}

\subsubsection*{Stanton \hspace*{\fill}\nolinebreak[1]%
\enspace\hspace*{\fill}
\finalhyphendemerits=0
[23rd June]}

\index{Stanton , Derbyshire Dales@Stanton, \emph{Derbys. Dales}}

No candidate returned in the 2011 ordinary election (one candidate was nominated, who then withdrew).

\noindent
\begin{tabular*}{\columnwidth}{@{\extracolsep{\fill}} p{0.545\columnwidth} >{\itshape}l r @{\extracolsep{\fill}}}
Joanne Wild & C & 246\\
Julie Morrison & Lab & 173\\
Anthony Allwood & LD & 84\\
\end{tabular*}

\section{Devon}

\subsection*{Mid Devon}
\index{Mid Devon}

\subsubsection*{Clare and Shuttern \hspace*{\fill}\nolinebreak[1]%
\enspace\hspace*{\fill}
\finalhyphendemerits=0
[15th December]}

\index{Clare and Shuttern , Mid Devon@Clare \& Shuttern, \emph{Mid Devon}}

Resignation of Nigel Burrough (C).

\noindent
\begin{tabular*}{\columnwidth}{@{\extracolsep{\fill}} p{0.56\columnwidth} >{\itshape}l r @{\extracolsep{\fill}}}
Polly Colthorpe & C & 385\\
Terry Knagg & Ind & 166\\
Tony McIntyre & UKIP & 54\\
\end{tabular*}

\subsection*{North Devon}
\index{North Devon}

\subsubsection*{Fremington \hspace*{\fill}\nolinebreak[1]%
\enspace\hspace*{\fill}
\finalhyphendemerits=0
[11th August]}

\index{Fremington , North Devon@Fremington, \emph{N. Devon}}

Resignation of Joanne Bell (Ind).

\noindent
\begin{tabular*}{\columnwidth}{@{\extracolsep{\fill}} p{0.545\columnwidth} >{\itshape}l r @{\extracolsep{\fill}}}
Chris Turner & Ind & 501\\
John Gill & C & 308\\
Tony Wood & Ind & 196\\
Neil Basil & Grn & 64\\
\end{tabular*}

\subsection*{Plymouth}
\index{Plymouth}

\index{Plymstock Dunstone , Plymouth@Plymstock Dunstone, \emph{Plymouth}}

At the May 2011 ordinary election there was an unfilled vacancy in Plymstock Dunstone ward due to the death of David Viney (C).

\subsubsection*{St Peter and the Waterfront \hspace*{\fill}\nolinebreak[1]%
\enspace\hspace*{\fill}
\finalhyphendemerits=0
[5th May]}

\index{Saint Peter and the Waterfront , Plymouth@\sloppyword{St Peter \& the Waterfront, \emph{Plymouth}}}

Resignation of Sally Stephens (C).

This by-election was combined with the 2011 ordinary election.
%; see page \pageref{StPeterWaterfrontPlymouth} for the result.

\subsection*{Torbay}
\index{Torbay}

\index{Roundham-with-Hyde , Torbay@Roundham-with-Hyde, \emph{Torbay}}

\index{Shiphay-with-the-Willows , Torbay@\sloppyword{Shiphay-with-the-Willows, \emph{Torbay}}}

At the May 2011 ordinary election there were unfilled vacancies in Roundham-with-Hyde and Shiphay-with-the-Willows wards due to the deaths of Kevin Carroll and Roger Kerslake (both C) respectively.

\subsubsection*{Cockington-with-Chelston \hspace*{\fill}\nolinebreak[1]%
\enspace\hspace*{\fill}
\finalhyphendemerits=0
[23rd June; LD gain from C]}

\index{Cockington-with-Chelston , Torbay@\sloppyword{Cockington-with-Chelston,} \emph{Torbay}}

Election of Gordon Oliver (C) as Mayor of Torbay.

\noindent
\begin{tabular*}{\columnwidth}{@{\extracolsep{\fill}} p{0.545\columnwidth} >{\itshape}l r @{\extracolsep{\fill}}}
Mark Pountney & LD & 1048\\
Sylvia Faryna & C & 614\\
Leonora Critchlow & Lab & 357\\
Susie Colley & Ind & 129\\
Mark Dent & Ind & 61\\
Thomas Cooper & Grn & 55\\
\end{tabular*}

\section{Dorset}

\subsection*{Purbeck}
\index{Purbeck}

\subsubsection*{Lytchett Matravers \hspace*{\fill}\nolinebreak[1]%
\enspace\hspace*{\fill}
\finalhyphendemerits=0
[7th July; C gain from LD]}

\index{Lytchett Matravers , Purbeck@Lytchett Matravers, \emph{Purbeck}}

Resignation of Mark Gracey (LD).

\noindent
\begin{tabular*}{\columnwidth}{@{\extracolsep{\fill}} p{0.545\columnwidth} >{\itshape}l r @{\extracolsep{\fill}}}
Peter Webb & C & 669\\
John Taylor & LD & 599\\
\end{tabular*}

\section{Durham}

\subsection*{Hartlepool}
\index{Hartlepool}

\index{Owton , Hartlepool@Owton, \emph{Hartlepool}}

At the May 2011 ordinary election there was an unfilled vacancy in Owton ward due to the death of Bob Flintoff (LD).

\subsection*{Stockton-on-Tees}
\index{Stockton-on-Tees}

\index{Billingham East , Stockton-on-Tees@Billingham E., \emph{Stockton-on-Tees}}
\index{Ingleby Barwick East , Stockton-on-Tees@\sloppyword{Ingleby Barwick E., \emph{Stockton-on-Tees}}}
\index{Mandale and Victoria , Stockton-on-Tees@\sloppyword{Mandale \& Victoria, \emph{Stockton-on-Tees}}}

At the May 2011 ordinary election there were unfilled vacancies in Billingham East ward due to the resignation of Alex Cunningham (Lab); in Ingleby Barwick East ward due to the resignation of Andrew Larkin (Ingleby Barwick Independent Society); and in Mandale and Victoria ward due to the death of Allison Trainer (Thornaby Independent Association).

\section{East Sussex}

\council{Brighton and Hove}

\index{Hangleton and Knoll , Brighton and Hove@Hangleton \& Knoll, \emph{Brighton \& Hove}}

At the May 2011 ordinary election there was an unfilled vacancy in Hangleton and Knoll ward due to the death of David Smart (C).

ECP = European Citizens Party

\subsubsection*{Westbourne \hspace*{\fill}\nolinebreak[1]%
\enspace\hspace*{\fill}
\finalhyphendemerits=0
[22nd December]}

\index{Westbourne , Brighton and Hove@Westbourne, \emph{Brighton \& Hove}}

Resignation of Brian Oxley (C).

\noindent
\begin{tabular*}{\columnwidth}{@{\extracolsep{\fill}} p{0.545\columnwidth} >{\itshape}l r @{\extracolsep{\fill}}}
Graham Cox & C & 1027\\
Nigel Jenner & Lab & 826\\
Louisa Grennbaum & Grn & 645\\
Gareth Jones & LD & 45\\
Paul Perrin & UKIP & 36\\
Pip Tindall & TUSC & 20\\
Susan Collard & ECP & 13\\
\end{tabular*}

\subsection*{Wealden}
\index{Wealden}

\index{Hailsham South and West , Wealden@Hailsham S. \& W., \emph{Wealden}}

\index{Uckfield Ridgewood , Wealden@Uckfield Ridgewood, \emph{Wealden}}

At the May 2011 ordinary election there were unfilled vacancies in Hailsham South and West, and Uckfield Ridgewood wards due to the resignations of Nick Ellwood (Ind) and Robert Sweetland (LD) respectively.

\section{East Yorkshire}

\subsection*{Kingston upon Hull}
\index{Kingston upon Hull}

\index{Marfleet , Kingston upon Hull@Marfleet, \emph{Kingston upon Hull}}

At the May 2011 ordinary election there was an unfilled vacancy in Marfleet ward due to the death of Brenda Petch (Lab).

\section{Essex}

\subsection*{County Council}
\index{Essex}

\subsubsection*{Harlow West \hspace*{\fill}\nolinebreak[1]%
\enspace\hspace*{\fill}
\finalhyphendemerits=0
[5th May; Lab gain from C]}

\index{Harlow West , Essex@Harlow W., \emph{Essex}}

\sloppyword{Disqualification (non-attendance) of Lee Dangerfield (C).}

\noindent
\begin{tabular*}{\columnwidth}{@{\extracolsep{\fill}} p{0.545\columnwidth} >{\itshape}l r @{\extracolsep{\fill}}}
Tony Durcan & Lab & 5320\\
Mark Gough & C & 4564\\
John Strachan & LD & 1100\\
\end{tabular*}

\subsubsection*{Chelmsford Central \hspace*{\fill}\nolinebreak[1]%
\enspace\hspace*{\fill}
\finalhyphendemerits=0
[9th June; C gain from LD]}

\index{Chelmsford Central , Essex@Chelmsford C., \emph{Essex}}

Death of Margaret Hutchon (LD).

\noindent
\begin{tabular*}{\columnwidth}{@{\extracolsep{\fill}} p{0.545\columnwidth} >{\itshape}l r @{\extracolsep{\fill}}}
Dick Madden & C & 1496\\
Graham Pooley & LD & 1323\\
Russell Kennedy & Lab & 610\\
\end{tabular*}

\subsubsection*{Stock \hspace*{\fill}\nolinebreak[1]%
\enspace\hspace*{\fill}
\finalhyphendemerits=0
[8th September]}

\index{Stock , Essex@Stock, \emph{Essex}}

Disqualification (sentenced to nine months' imprisonment, false accounting) of Lord Hanningfield (C).

\noindent
\begin{tabular*}{\columnwidth}{@{\extracolsep{\fill}} p{0.56\columnwidth} >{\itshape}l r @{\extracolsep{\fill}}}
Ian Grundy & C & 1820\\
Jesse Pryke & UKIP & 736\\
Maurice Austin & Lab & 275\\
Marian Elsden & LD & 160\\
Reza Hossain & Grn & 80\\
\end{tabular*}

\subsection*{Basildon}
\index{Basildon}

\subsubsection*{Fryerns \hspace*{\fill}\nolinebreak[1]%
\enspace\hspace*{\fill}
\finalhyphendemerits=0
[5th May]}

\index{Fryerns , Basildon@Fryerns, \emph{Basildon}}

Resignation of Paul Kirkman (Lab).

This by-election was combined with the 2011 ordinary election.
%; see page \pageref{FryernsBasildon} for the result.

\subsection*{Brentwood}
\index{Brentwood}

\subsubsection*{\sloppyword{Ingatestone, Fryerning and Mountnessing} \hspace*{\fill}\nolinebreak[1]%
\enspace\hspace*{\fill}
\finalhyphendemerits=0
[5th May]}

\index{Ingatestone, Fryerning and Mountnessing , Brentwood@Ingatestone, Fryerning \& \sloppyword{Mountnessing, \emph{Brentwood}}}

Resignation of Richard Harrison (C).

This by-election was combined with the 2011 ordinary election.
%; see page \pageref{IngatestoneFryerningMountnessingBrentwood} for the result.

\subsection*{Chelmsford}
\index{Chelmsford}

\index{Great Baddow West , Chelmsford@Great Baddow W., \emph{Chelmsford}}

At the May 2011 ordinary election there was an unfilled vacancy in Great Baddow West ward due to the death of Margaret Hutchon (LD).

\subsection*{Harlow}
\index{Harlow}

\index{Staple Tye , Harlow@Staple Tye, \emph{Harlow}}

\index{Sumners and Kingsmoor , Harlow@Sumners \& Kingsmoor, \emph{Harlow}}

At the May 2011 ordinary election there were unfilled vacancies in Staple Tye, and Sumners and Kingsmoor wards due to the resignations of Lee and Sarah Dangerfield (both C) respectively.

\subsection*{Rochford}
\index{Rochford}

\subsubsection*{Hullbridge \hspace*{\fill}\nolinebreak[1]%
\enspace\hspace*{\fill}
\finalhyphendemerits=0
[16th June; Grn gain from C]}

\index{Hullbridge , Rochford@Hullbridge, \emph{Rochford}}

Death of Peter Robinson (C).

\noindent
\begin{tabular*}{\columnwidth}{@{\extracolsep{\fill}} p{0.56\columnwidth} >{\itshape}l r @{\extracolsep{\fill}}}
Diane Hoy & Grn & 757\\
Mark Hale & C & 555\\
Angelina Marriott & Lab & 182\\
Carl Whitwell & UKIP & 76\\
\end{tabular*}

\subsubsection*{Rayleigh Central \hspace*{\fill}\nolinebreak[1]%
\enspace\hspace*{\fill}
\finalhyphendemerits=0
[1st December]}

\index{Rayleigh Central , Rochford@Rayleigh C., \emph{Rochford}}

Death of Tony Humphries (C).

\noindent
\begin{tabular*}{\columnwidth}{@{\extracolsep{\fill}} p{0.56\columnwidth} >{\itshape}l r @{\extracolsep{\fill}}}
Cheryl Roe & C & 406\\
John Hayter & EDP & 218\\
Elena Black & LD & 117\\
\end{tabular*}

\section{Gloucestershire}

\subsection*{County Council}
\index{Gloucestershire}

\subsubsection*{Rodborough \hspace*{\fill}\nolinebreak[1]%
\enspace\hspace*{\fill}
\finalhyphendemerits=0
[3rd February; Lab gain from C]}

\index{Rodborough , Gloucestershire@Rodborough, \emph{Glos.}}

Death of Stephen Glanfield (C).

\noindent
\begin{tabular*}{\columnwidth}{@{\extracolsep{\fill}} p{0.545\columnwidth} >{\itshape}l r @{\extracolsep{\fill}}}
Brian Oosthuysen & Lab & 793\\
Nigel Cooper & C & 790\\
Christine Headley & LD & 660\\
Phil Blomberg & Grn & 260\\
\end{tabular*}

\council{South Gloucestershire}

\index{Emersons Green , South Gloucestershire@Emersons Green, \emph{S. Glos.}}

At the May 2011 ordinary election there was an unfilled vacancy in Emersons Green ward due to the death of Ian Morris (C).

\subsection*{Stroud}
\index{Stroud}

\index{Uplands , Stroud@Uplands, \emph{Stroud}}

At the May 2011 ordinary election there was an unfilled vacancy in Uplands ward due to the death of Linda Townley (Ind).

\subsubsection*{Amberley and Woodchester \hspace*{\fill}\nolinebreak[1]%
\enspace\hspace*{\fill}
\finalhyphendemerits=0
[3rd February]}

\index{Amberley and Woodchester , Stroud@Amberley \& Woodchester, \emph{Stroud}}

Death of Stephen Glanfield (C).

\noindent
\begin{tabular*}{\columnwidth}{@{\extracolsep{\fill}} p{0.6\columnwidth} >{\itshape}l r @{\extracolsep{\fill}}}
Rhiannon Wigzell & C & 366\\
Jo Smith & Lab & 177\\
Adrian Walker-Smith & LD & 124\\
\end{tabular*}

\subsubsection*{Nailsworth \hspace*{\fill}\nolinebreak[1]%
\enspace\hspace*{\fill}
\finalhyphendemerits=0
[5th May]}

\index{Nailsworth , Stroud@Nailsworth, \emph{Stroud}}

Resignation of Fi MacMillan (Grn).

This by-election was combined with the 2011 ordinary election.
%; see page \pageref{NailsworthStroud} for the result.

\columnbreak

\section{Hampshire}

\subsection*{County Council}
\index{Hampshire}

JACP = Justice and Anti-Corruption Party

\subsubsection*{Lee \hspace*{\fill}\nolinebreak[1]%
\enspace\hspace*{\fill}
\finalhyphendemerits=0
[5th May]}

\index{Lee , Hampshire@Lee, \emph{Hants.}}

Resignation of Margaret Snaith-Tempia (C).

\noindent
\begin{tabular*}{\columnwidth}{@{\extracolsep{\fill}} p{0.545\columnwidth} >{\itshape}l r @{\extracolsep{\fill}}}
Graham Burgess & C & 3080\\
Angela Whitbread & LD & 1227\\
Graham Giles & Lab & 858\\
\end{tabular*}

\subsubsection*{Headley \hspace*{\fill}\nolinebreak[1]%
\enspace\hspace*{\fill}
\finalhyphendemerits=0
[15th September]}

\index{Headley , Hampshire@Headley, \emph{Hants.}}

Resignation of Sam James (C).

\noindent
\begin{tabular*}{\columnwidth}{@{\extracolsep{\fill}} p{0.56\columnwidth} >{\itshape}l r @{\extracolsep{\fill}}}
Ferris Cowper & C & 1588\\
Maureen Comber & LD & 290\\
John Tough & Lab & 258\\
Neville Taylor & Grn & 178\\
Don Jerrard & JACP & 146\\
\end{tabular*}

\council{Basingstoke and Deane}

\subsubsection*{Popley East \hspace*{\fill}\nolinebreak[1]%
\enspace\hspace*{\fill}
\finalhyphendemerits=0
[5th May]}

\index{Popley East , Basingstoke and Deane@Popley E., \emph{Basingstoke \& Deane}}

Resignation of Mary Brian (Lab).

This by-election was combined with the 2011 ordinary election.
%; see page \pageref{PopleyEastBasingstokeDeane} for the result.

\council{East Hampshire}

\index{Whitehill Pinewood , East Hampshire@Whitehill Pinewood, \emph{E. Hants.}}

At the May 2011 ordinary election there was an unfilled vacancy in Whitehill Pinewood ward due to the resignation of Ian Dowdle (C).

\subsubsection*{Bramshott and Liphook (2) \hspace*{\fill}\nolinebreak[1]%
\enspace\hspace*{\fill}
\finalhyphendemerits=0
[15th September]}

\index{Bramshott and Liphook , East Hampshire@Bramshott \& Liphook, \emph{E. Hants.}}

Resignations of Sam and Anna James (both C).

\noindent
\begin{tabular*}{\columnwidth}{@{\extracolsep{\fill}} p{0.56\columnwidth} >{\itshape}l r @{\extracolsep{\fill}}}
Lynn Ashton & C & 796\\
Bill Mouland & C & 743\\
Michael Croucher & LD & 404\\
Eve Hope & LD & 371\\
Frank Jones & Lab & 183\\
Neville Taylor & Grn & 126\\
John Tough & Lab & 117\\
\end{tabular*}

\subsection*{New Forest}
\index{New Forest}

\index{Marchwood , New Forest@Marchwood, \emph{New Forest}}

At the May 2011 ordinary election there was an unfilled vacancy in Marchwood ward due to the death of Alan Shotter (C).

\subsection*{Winchester}
\index{Winchester}

\subsubsection*{\sloppyword{Oliver's Battery and Badger Farm} \hspace*{\fill}\nolinebreak[1]%
\enspace\hspace*{\fill}
\finalhyphendemerits=0
[27th January]}

\index{Oliver's Battery and Badger Farm , Winchester@Oliver's Battery \& Badger Farm, \emph{Winchester}}

Death of David Spender (LD).

\noindent
\begin{tabular*}{\columnwidth}{@{\extracolsep{\fill}} p{0.545\columnwidth} >{\itshape}l r @{\extracolsep{\fill}}}
Brian Laming & LD & 894\\
Leanne Wheeler & C & 604\\
Hum Qureshi & Lab & 162\\
\end{tabular*}

\section{Hertfordshire}

\subsection*{County Council}
\index{Hertfordshire}

\subsubsection*{Bishop's Stortford West \hspace*{\fill}\nolinebreak[1]%
\enspace\hspace*{\fill}
\finalhyphendemerits=0
[5th May]}

\index{Bishop's Stortford West , Hertfordshire@Bishop's Stortford W., \emph{Herts.}}

Death of Duncan Peek (C).

\noindent
\begin{tabular*}{\columnwidth}{@{\extracolsep{\fill}} p{0.545\columnwidth} >{\itshape}l r @{\extracolsep{\fill}}}
Colin Woodward & C & 2483\\
Robert Taylor & LD & 1256\\
Alexander Young & Lab & 977\\
\end{tabular*}

\subsection*{Broxbourne}
\index{Broxbourne}

\subsubsection*{Cheshunt Central \hspace*{\fill}\nolinebreak[1]%
\enspace\hspace*{\fill}
\finalhyphendemerits=0
[30th June]}

\index{Cheshunt Central , Broxbourne@Cheshunt C., \emph{Broxbourne}}

Resignation of Jason Brimson (C).

\noindent
\begin{tabular*}{\columnwidth}{@{\extracolsep{\fill}} p{0.56\columnwidth} >{\itshape}l r @{\extracolsep{\fill}}}
Tony Siracusa & C & 742\\
Richard Greenhill & Lab & 481\\
David Platt & UKIP & 88\\
Joanne Welch & Ind & 62\\
Peter Huse & LD & 24\\
\end{tabular*}

\council{East Hertfordshire}

\index{Bishop's Stortford Silverleys , East Hertfordshire@Bishop's Stortford Silverleys, \emph{E. Herts.}}

\index{Hertford Sele , East Hertfordshire@Hertford Sele, \emph{E. Herts.}}

At the May 2011 ordinary election there were unfilled vacancies in Bishop's Stortford Silverleys and Hertford Sele wards due to the deaths of Duncan Peek and John Hedley (both C) respectively.

\subsection*{St Albans}
\index{Saint Albans@St Albans}

\subsubsection*{St Peters \hspace*{\fill}\nolinebreak[1]%
\enspace\hspace*{\fill}
\finalhyphendemerits=0
[5th May]}

\index{Saint Peters , Saint Albans@St Peters, \emph{St Albans}}

Resignation of Martin Morris (LD).

This by-election was combined with the 2011 ordinary election.
%; see page \pageref{StPetersStAlbans} for the result.

\subsection*{Watford}
\index{Watford}

\subsubsection*{Nascot \hspace*{\fill}\nolinebreak[1]%
\enspace\hspace*{\fill}
\finalhyphendemerits=0
[29th September]}

\index{Nascot , Watford@Nascot, \emph{Watford}}

Resignation of Andrew Forrest (LD).

\noindent
\begin{tabular*}{\columnwidth}{@{\extracolsep{\fill}} p{0.56\columnwidth} >{\itshape}l r @{\extracolsep{\fill}}}
Jeanette Aron & LD & 1021\\
Penny Edwards & C & 818\\
Omar Ismail & Lab & 134\\
Sally Ivins & Grn & 133\\
\end{tabular*}

\subsubsection*{Vicarage \hspace*{\fill}\nolinebreak[1]%
\enspace\hspace*{\fill}
\finalhyphendemerits=0
[6th October]}

\index{Vicarage , Watford@Vicarage, \emph{Watford}}

Resignation of Tanveer Taj (Lab).

\noindent
\begin{tabular*}{\columnwidth}{@{\extracolsep{\fill}} p{0.545\columnwidth} >{\itshape}l r @{\extracolsep{\fill}}}
Mo Mills & Lab & 917\\
Philip Gough & LD & 403\\
Mohammad Azam & Ind & 190\\
Paul Baker & Ind & 129\\
Dave Ealey & C & 95\\
\end{tabular*}

\subsection*{Welwyn Hatfield}
\index{Welwyn Hatfield}

\subsubsection*{Howlands \hspace*{\fill}\nolinebreak[1]%
\enspace\hspace*{\fill}
\finalhyphendemerits=0
[5th May]}

\index{Howlands , Welwyn Hatfield@Howlands, \emph{Welwyn Hatfield}}

Resignation of Hannah Ball (elected as Hannah Berry, C).

This by-election was combined with the 2011 ordinary election.
%; see page \pageref{HowlandsWelwynHatfield} for the result.

\section{Isle of Wight}
\index{Isle of Wight}

\subsubsection*{Binstead and Fishbourne \hspace*{\fill}\nolinebreak[1]%
\enspace\hspace*{\fill}
\finalhyphendemerits=0
[23rd June; Ind gain from C]}

\index{Binstead and Fishbourne , Isle of Wight@Binstead \& Fishbourne, \emph{Isle of Wight}}

Death of Ivan Bulwer (C).

\noindent
\begin{tabular*}{\columnwidth}{@{\extracolsep{\fill}} p{0.545\columnwidth} >{\itshape}l r @{\extracolsep{\fill}}}
Ivor Warlow & Ind & 428\\
Ian Cobb & C & 424\\
Tim Wakeley & LD & 139\\
Daryll Pitcher & UKIP & 93\\
Mick Lyons & Lab & 66\\
\end{tabular*}

\subsubsection*{West Wight \hspace*{\fill}\nolinebreak[1]%
\enspace\hspace*{\fill}
\finalhyphendemerits=0
[17th November; C gain from Ind]}

\index{West Wight, Isle of Wight@West Wight, \emph{Isle of Wight}}

Death of Stuart Dyer (Ind).

\noindent
\begin{tabular*}{\columnwidth}{@{\extracolsep{\fill}} p{0.545\columnwidth} >{\itshape}l r @{\extracolsep{\fill}}}
Stuart Hutchinson & C & 640\\
Mike Carr & LD & 116\\
Rose Lynden-Bell & UKIP & 78\\
\end{tabular*}

\section{Kent}

\subsection*{County Council}
\index{Kent}

\subsubsection*{Tonbridge \hspace*{\fill}\nolinebreak[1]%
\enspace\hspace*{\fill}
\finalhyphendemerits=0
[20th January]}

\index{Tonbridge , Kent@Tonbridge, \emph{Kent}}

Death of Godfrey Horne (C).

\noindent
\begin{tabular*}{\columnwidth}{@{\extracolsep{\fill}} p{0.545\columnwidth} >{\itshape}l r @{\extracolsep{\fill}}}
Alice Hohler & C & 3229\\
Emily Williams & Lab & 1216\\
Garry Bridge & LD & 561\\
Hazel Dawe & Grn & 366\\
David Waller & UKIP & 337\\
\end{tabular*}

\subsubsection*{Romney Marsh \hspace*{\fill}\nolinebreak[1]%
\enspace\hspace*{\fill}
\finalhyphendemerits=0
[10th February]}

\index{Romney Marsh , Kent@Romney Marsh, \emph{Kent}}

Death of Willie Richardson (C).

\noindent
\begin{tabular*}{\columnwidth}{@{\extracolsep{\fill}} p{0.545\columnwidth} >{\itshape}l r @{\extracolsep{\fill}}}
Carole Waters & C & 2222\\
Doug Suckling & Lab & 748\\
Val Loseby & LD & 479\\
David Cammegh & UKIP & 420\\
Rochelle Saunders & Ind & 238\\
\end{tabular*}

\subsection*{Ashford}
\index{Ashford}

\index{Kennington , Ashford@Kennington, \emph{Ashford}}

At the May 2011 ordinary election there was an unfilled vacancy in Kennington ward due to the death of Maj John Kemp (C).

Ashford = Ashford Independent

\subsubsection*{Beaver \hspace*{\fill}\nolinebreak[1]%
\enspace\hspace*{\fill}
\finalhyphendemerits=0
[24th November]}

\index{Beaver , Ashford@Beaver, \emph{Ashford}}

Death of Brendan Naughton (Lab).

\noindent
\begin{tabular*}{\columnwidth}{@{\extracolsep{\fill}} p{0.545\columnwidth} >{\itshape}l r
@{\extracolsep{\fill}}}
Rebecca Rutter & Lab & 336\\
Jem Cerit & C & 249\\
Jack Cowen & LD & 173\\
Laura Lawrence & Ashford & 111\\
Mark Reed & Grn & 26\\
\end{tabular*}

\subsection*{Dartford}
\index{Dartford}

\index{Stone , Dartford@Stone, \emph{Dartford}}

At the May 2011 ordinary election there was an unfilled vacancy in Stone ward due to the resignation of Derek Lawson (Lab).

\subsection*{Gravesham}
\index{Gravesham}

\subsubsection*{Meopham North \hspace*{\fill}\nolinebreak[1]%
\enspace\hspace*{\fill}
\finalhyphendemerits=0
[13th October]}

\index{Meopham North , Gravesham@Meopham N., \emph{Gravesham}}

Resignation of Laura Hryniewicz (C).

\noindent
\begin{tabular*}{\columnwidth}{@{\extracolsep{\fill}} p{0.545\columnwidth} >{\itshape}l r @{\extracolsep{\fill}}}
John Cubitt & C & 648\\
Geoffrey Clark & UKIP & 462\\
David Gibson & LD & 148\\
Andrew Mylett & Lab & 112\\
\end{tabular*}

\subsection*{Maidstone}
\index{Maidstone}

\subsubsection*{Bearsted \hspace*{\fill}\nolinebreak[1]%
\enspace\hspace*{\fill}
\finalhyphendemerits=0
[5th May]}

\index{Bearsted , Maidstone@Bearsted, \emph{Maidstone}}

Death of Heather Langley (C).

This by-election was combined with the 2011 ordinary election.
%; see page \pageref{BearstedMaidstone} for the result.

\subsubsection*{East \hspace*{\fill}\nolinebreak[1]%
\enspace\hspace*{\fill}
\finalhyphendemerits=0
[5th May]}

\index{East , Maidstone@East, \emph{Maidstone}}

Resignation of Patrick Sellar (LD).

This by-election was combined with the 2011 ordinary election.
%; see page \pageref{EastMaidstone} for the result.

\subsection*{Sevenoaks}
\index{Sevenoaks}

\index{Swanley Saint Mary's , Sevenoaks@Swanley St Mary's, \emph{Sevenoaks}}

At the May 2011 ordinary election there was an unfilled vacancy in Swanley St Mary's ward due to the resignation of Paul Golding (BNP).

\subsection*{Shepway}
\index{Shepway}

\index{Romney Marsh , Shepway@Romney Marsh, \emph{Shepway}}

At the May 2011 ordinary election there was an unfilled vacancy in Romney Marsh ward due to the disqualification (non-attendance) of Tony Clifton-Holt (C).

\subsubsection*{Lydd \hspace*{\fill}\nolinebreak[1]%
\enspace\hspace*{\fill}
\finalhyphendemerits=0
[10th February]}

\index{Lydd , Shepway@Lydd, \emph{Shepway}}

Death of Willie Richardson (C).

\noindent
\begin{tabular*}{\columnwidth}{@{\extracolsep{\fill}} p{0.545\columnwidth} >{\itshape}l r @{\extracolsep{\fill}}}
Tony Hills & C & 591\\
Donald Russell & Lab & 247\\
Ted Last & LD & 184\\
Rochelle Saunders & Ind & 177\\
\end{tabular*}

\subsection*{Thanet}
\index{Thanet}

\index{Bradstowe , Thanet@Bradstowe, \emph{Thanet}}

At the May 2011 ordinary election there was an unfilled vacancy in Bradstowe ward due to the disqualification (non-attendance) of Ewen Cameron (Ind elected as C).

\subsection*{Tunbridge Wells}
\index{Tunbridge Wells}

\subsubsection*{Pembury \hspace*{\fill}\nolinebreak[1]%
\enspace\hspace*{\fill}
\finalhyphendemerits=0
[17th March; LD gain from C]}

\index{Pembury , Tunbridge Wells@Pembury, \emph{Tunbridge Wells}}

Resignation of Mike Tompsett (C).

\noindent
\begin{tabular*}{\columnwidth}{@{\extracolsep{\fill}} p{0.545\columnwidth} >{\itshape}l r @{\extracolsep{\fill}}}
Claire Brown & LD & 578\\
Robert Rutherford & C & 450\\
Victor Webb & UKIP & 297\\
\end{tabular*}

\subsubsection*{Brenchley and Horsmonden \hspace*{\fill}\nolinebreak[1]%
\enspace\hspace*{\fill}
\finalhyphendemerits=0
[5th May]}

\index{Brenchley and Horsmonden , Tunbridge Wells@Brenchley \& Horsmonden, \sloppyword{\emph{Tunbridge Wells}}}

Resignation of Marcella Callow (C).

This by-election was combined with the 2011 ordinary election.
%; see page \pageref{BrenchleyHorsmondenTunbridgeWells} for the result.

\section{Lancashire}

\subsection*{County Council}
\index{Lancashire}

\subsubsection*{Wyreside \hspace*{\fill}\nolinebreak[1]%
\enspace\hspace*{\fill}
\finalhyphendemerits=0
[27th October]}

\index{Wyreside , Lancashire@Wyreside, \emph{Lancs.}}

Death of Bob Mutch (C).

\noindent
\begin{tabular*}{\columnwidth}{@{\extracolsep{\fill}} p{0.545\columnwidth} >{\itshape}l r @{\extracolsep{\fill}}}
Vivien Taylor & C & 2178\\
Kevin Higginson & Lab & 877\\
Simon Noble & UKIP & 361\\
Sue White & Grn & 339\\
\end{tabular*}

\subsection*{Blackburn with Darwen}
\index{Blackburn with Darwen}

\subsubsection*{Beardwood with Lammack \hspace*{\fill}\nolinebreak[1]%
\enspace\hspace*{\fill}
\finalhyphendemerits=0
[28th July]}

\index{Beardwood with Lammack , Blackburn with Darwen@Beardwood with Lammack, \emph{Blackburn with Darwen}}

Resignation of Sheila Williams (C).

\noindent
\begin{tabular*}{\columnwidth}{@{\extracolsep{\fill}} p{0.545\columnwidth} >{\itshape}l r @{\extracolsep{\fill}}}
Julie Daley & C & 1097\\
Ashley Whalley & Lab & 572\\
Salim Lorgat & LD & 51\\
\end{tabular*}

\subsection*{Blackpool}
\index{Blackpool}

\index{Bloomfield , Blackpool@Bloomfield, \emph{Blackpool}}
\index{Stanley , Blackpool@Stanley, \emph{Blackpool}}

At the May 2011 ordinary election there were unfilled vacancies in Bloomfield and Stanley wards due to the deaths of Doreen Holt (LD) and Jean Kenrick (C) respectively.

\subsection*{Burnley}
\index{Burnley}

\index{Bank Hall , Burnley@Bank Hall, \emph{Burnley}}
\index{Rosehill with Burnley Wood , Burnley@Rosehill with Burnley Wood, \emph{Burnley}}

At the May 2011 ordinary election there were unfilled vacancies in Bank Hall and Rosehill with Burnley Wood wards due to the disqualification (non-attendance) of Imtiaz Hussain (Lab) and the death of Frank Ashworth (LD) respectively.

\subsubsection*{Rosegrove with Lowerhouse \hspace*{\fill}\nolinebreak[1]%
\enspace\hspace*{\fill}
\finalhyphendemerits=0
[10th March; Lab gain from LD]}

\index{Rosegrove with Lowerhouse , Burnley@Rosegrove with Lowerhouse, \emph{Burnley}}

Resignation of Julie Johnson (LD).

\noindent
\begin{tabular*}{\columnwidth}{@{\extracolsep{\fill}} p{0.545\columnwidth} >{\itshape}l r @{\extracolsep{\fill}}}
Beatrice Foster & Lab & 521\\
Paul McDevitt & BNP & 288\\
Kate Mottershead & LD & 261\\
Matthew Isherwood & C & 81\\
Andrew Hennessey & Ind & 58\\
\end{tabular*}

\subsection*{Ribble Valley}
\index{Ribble Valley}

\index{Salthill , Ribble Valley@Salthill, \emph{Ribble Valley}}

At the May 2011 ordinary election there was an unfilled vacancy in Salthill ward due to the disqualification (non-attendance) of Simon Farnsworth (C).

\subsubsection*{Salthill \hspace*{\fill}\nolinebreak[1]%
\enspace\hspace*{\fill}
\finalhyphendemerits=0
[17th November; C gain from LD]}

\index{Salthill , Ribble Valley@Salthill, \emph{Ribble Valley}}

Resignation of David Berryman (LD).

\noindent
\begin{tabular*}{\columnwidth}{@{\extracolsep{\fill}} p{0.545\columnwidth} >{\itshape}l r @{\extracolsep{\fill}}}
Ian Brown & C & 208\\
Simon O'Rourke & LD & 204\\
Steve Rush & UKIP & 159\\
Mike Rose & Lab & 40\\
\end{tabular*}

\subsection*{South Ribble}
\index{South Ribble}

\index{Earnshaw Bridge , South Ribble@Earnshaw Bridge, \emph{S. Ribble}}

At the May 2011 ordinary election there was an unfilled vacancy in Earnshaw Bridge ward due to the resignation of Irvine Edwards (C).

\subsubsection*{Bamber Bridge East \hspace*{\fill}\nolinebreak[1]%
\enspace\hspace*{\fill}
\finalhyphendemerits=0
[13th October]}

\index{Bamber Bridge East , South Ribble@Bamber Bridge E., \emph{S. Ribble}}

Resignation of Jim Owen (Lab).

\noindent
\begin{tabular*}{\columnwidth}{@{\extracolsep{\fill}} p{0.545\columnwidth} >{\itshape}l r @{\extracolsep{\fill}}}
Mick Higgins & Lab & 481\\
Barbara Nathan & C & 393\\
\end{tabular*}

\council{West Lancashire}

\index{Derby , West Lancashire@Derby, \emph{W. Lancs.}}

At the May 2011 ordinary election there was an unfilled vacancy in Derby ward due to the death of David Swiffen (C).

\subsubsection*{Rufford \hspace*{\fill}\nolinebreak[1]%
\enspace\hspace*{\fill}
\finalhyphendemerits=0
[5th May]}

\index{Rufford , West Lancashire@Rufford, \emph{W. Lancs.}}

Death of Joan Colling (C).

This by-election was combined with the 2011 ordinary election.
%; see page \pageref{RuffordWestLancashire} for the result.

\section{Leicestershire}

\subsection*{County Council}
\index{Leicestershire}

\subsubsection*{Syston Ridgeway \hspace*{\fill}\nolinebreak[1]%
\enspace\hspace*{\fill}
\finalhyphendemerits=0
[3rd November]}

\index{Syston Ridgeway , Leicestershire@Syston Ridgeway, \emph{Leics.}}

Death of Mike Preston (C).

\noindent
\begin{tabular*}{\columnwidth}{@{\extracolsep{\fill}} p{0.545\columnwidth} >{\itshape}l r @{\extracolsep{\fill}}}
Stephen Hampson & C & 981\\
Colin Lovell & Lab & 490\\
Cathy Duffy & BNP & 279\\
Richard Miller & LD & 132\\
\end{tabular*}

\subsection*{Melton}
\index{Melton}

\subsubsection*{Frisby-on-the-Wreake \hspace*{\fill}\nolinebreak[1]%
\enspace\hspace*{\fill}
\finalhyphendemerits=0
[15th December; Ind gain from C]}

\index{Frisby-on-the-Wreake , Melton@Frisby-on-the-Wreake, \emph{Melton}}

Death of Nigel Angrave (C).

\noindent
\begin{tabular*}{\columnwidth}{@{\extracolsep{\fill}} p{0.545\columnwidth} >{\itshape}l r @{\extracolsep{\fill}}}
Edward Hutchinson & Ind & 212\\
Leigh Higgins & C & 187\\
Shaheer Mohammed & Lab & 89\\
Mark Twittey & Ind & 62\\
\end{tabular*}

\council{North West Leicestershire}

\index{Hugglescote , North West Leicestershire@Hugglescote, \emph{N.W. Leics.}}
\index{Measham , North West Leicestershire@Measham, \emph{N.W. Leics.}}

At the May 2011 ordinary election there were unfilled vacancies in Hugglescote and Measham wards due to the resignation of Graham Partner (BNP) and the disqualification (non-attendance) of Jason Summerfield (C) respectively.

\section{Lincolnshire}

\subsection*{County Council}
\index{Lincolnshire}

LincsInd = Lincolnshire Independent

\subsubsection*{Sleaford West and Leasingham \hspace*{\fill}\nolinebreak[1]%
\enspace\hspace*{\fill}
\finalhyphendemerits=0
[13th October]}

\index{Sleaford West and Leasingham , Lincolnshire@Sleaford W. \& Leasingham, \emph{Lincs.}}

Death of Barry Singleton (C).

\noindent
\begin{tabular*}{\columnwidth}{@{\extracolsep{\fill}} p{0.52\columnwidth} >{\itshape}l r @{\extracolsep{\fill}}}
Andrew Hagues & C & 614\\
David Suiter & LincsInd & 454\\
Jim Clarke & Lab & 315\\
\sloppyword{David Harding-Price} & LD & 45\\
\end{tabular*}

\subsection*{South Holland}
\index{South Holland}

At the May 2011 ordinary election there was an unfilled vacancy in Spalding St Paul's ward due to the disqualification (non-attendance) of Jane Jones (C).

\subsection*{South Kesteven}
\index{South Kesteven}

\subsubsection*{Deeping St James (3) \hspace*{\fill}\nolinebreak[1]%
\enspace\hspace*{\fill}
\finalhyphendemerits=0
[23rd June]}

\index{Deeping Saint James , South Kesteven@Deeping St James, \emph{S. Kesteven}}

\sloppyword{Ordinary election postponed from 5th May; death of outgoing councillor Ken Joynson (LD) who had been nominated for re-election.}

\noindent
\begin{tabular*}{\columnwidth}{@{\extracolsep{\fill}} p{0.545\columnwidth} >{\itshape}l r @{\extracolsep{\fill}}}
	Fair Deal Phil Dilks & Ind & 792\\
	Judy Stevens & Ind & 647\\
	Ray Auger & C & 631\\
	Robert Thomas & C & 483\\
	Ashley Baxter & Grn & 350\\
	Mike Exton & C & 318\\
	Michael Bossingham & Grn & 289\\
	Philip Hammersley & LD & 242\\
	Janire Morris & LD & 70\\
	Peter Morris & LD & 67\\
\end{tabular*}

%See page \pageref{DeepingStJamesSouthKesteven} for the result.

\section{Norfolk}

\subsection*{County Council}
\index{Norfolk}

\subsubsection*{Humbleyard \hspace*{\fill}\nolinebreak[1]%
\enspace\hspace*{\fill}
\finalhyphendemerits=0
[13th January]}

\index{Humbleyard , Norfolk@Humbleyard, \emph{Norfolk}}

Resignation of Daniel Cox (C).

\noindent
\begin{tabular*}{\columnwidth}{@{\extracolsep{\fill}} p{0.545\columnwidth} >{\itshape}l r @{\extracolsep{\fill}}}
Judith Virgo & C & 1015\\
Jaqueline Sutton & LD & 438\\
Marian Chapman & Lab & 424\\
Janet Kitchener & Grn & 170\\
Richard Coke & UKIP & 133\\
\end{tabular*}

\subsubsection*{Lothingland \hspace*{\fill}\nolinebreak[1]%
\enspace\hspace*{\fill}
\finalhyphendemerits=0
[5th May]}

\index{Lothingland , Norfolk@Lothingland, \emph{Norfolk}}

Resignation of Gerald Cook (C).

\noindent
\begin{tabular*}{\columnwidth}{@{\extracolsep{\fill}} p{0.6\columnwidth} >{\itshape}l r @{\extracolsep{\fill}}}
Barry Stone & C & 1611\\
Trevor Wainwright & Lab & 1076\\
John Cooper & Ind & 418\\
Michael Brackenbury & Grn & 184\\
\end{tabular*}

\subsubsection*{Old Catton \hspace*{\fill}\nolinebreak[1]%
\enspace\hspace*{\fill}
\finalhyphendemerits=0
[14th July]}

\index{Old Catton , Norfolk@Old Catton, \emph{Norfolk}}

Resignation of Stuart Dunn (C).

\noindent
\begin{tabular*}{\columnwidth}{@{\extracolsep{\fill}} p{0.545\columnwidth} >{\itshape}l r @{\extracolsep{\fill}}}
Judy Leggett & C & 664\\
Bob Fowkes & LD & 414\\
Chrissie Rumsby & Lab & 337\\
Glenn Tingle & UKIP & 107\\
Jennifer Parkhouse & Grn & 75\\
\end{tabular*}

\subsubsection*{Lakenham \hspace*{\fill}\nolinebreak[1]%
\enspace\hspace*{\fill}
\finalhyphendemerits=0
[24th November; Lab gain from LD]}

\index{Lakenham , Norfolk@Lakenham, \emph{Norfolk}}

Resignation of Fiona Williamson (LD).

\noindent
\begin{tabular*}{\columnwidth}{@{\extracolsep{\fill}} p{0.545\columnwidth} >{\itshape}l r @{\extracolsep{\fill}}}
Susan Whitaker & Lab & 1051\\
David Fairbairn & LD & 611\\
Paul Neale & Grn & 492\\
Mathew Morris & C & 160\\
Steve Emmens & UKIP & 133\\
\end{tabular*}

\subsection*{Great Yarmouth}
\index{Great Yarmouth}

\subsubsection*{St Andrews \hspace*{\fill}\nolinebreak[1]%
\enspace\hspace*{\fill}
\finalhyphendemerits=0
[23rd June; Lab gain from C]}

\index{Saint Andrews , Great Yarmouth@St Andrews, \emph{Great Yarmouth}}

Death of Gerry Cook (C).

\noindent
\begin{tabular*}{\columnwidth}{@{\extracolsep{\fill}} p{0.545\columnwidth} >{\itshape}l r @{\extracolsep{\fill}}}
Marlene Fairhead & Lab & 424\\
Carl Smith & C & 401\\
\end{tabular*}

\subsubsection*{Bradwell South and Hopton \hspace*{\fill}\nolinebreak[1]%
\enspace\hspace*{\fill}
\finalhyphendemerits=0
[20th October]}

\index{Bradwell South and Hopton , Great Yarmouth@Bradwell S. \& Hopton, \emph{Gt Yarmouth}}

Death of Mike Butcher (C).

\noindent
\begin{tabular*}{\columnwidth}{@{\extracolsep{\fill}} p{0.545\columnwidth} >{\itshape}l r @{\extracolsep{\fill}}}
Martin Plane & C & 557\\
Hilary Wainwright & Lab & 407\\
Colin Aldred & UKIP & 159\\
\end{tabular*}

\council{King's Lynn and West Norfolk}

\index{Hunstanton , King's Lynn and West Norfolk@Hunstanton, \emph{King's Lynn \& W. Norfolk}}

At the May 2011 ordinary election there was an unfilled vacancy in Hunstanton ward due to the death of Richard Searle (C).

\subsection*{North Norfolk}
\index{North Norfolk}

\index{Raynhams, The , North Norfolk@The Raynhams, \emph{N. Norfolk}}

At the May 2011 ordinary election there was an unfilled vacancy in The Raynhams ward due to the resignation of Dawn Wakefield (LD).

\subsection*{South Norfolk}
\index{South Norfolk}

\index{Rustens , South Norfolk@Rustens, \emph{S. Norfolk}}

At the May 2011 ordinary election there was an unfilled vacancy in Rustens ward due to the resignation of Daniel Cox (C).

\section{North Yorkshire}

\subsection*{County Council}
\index{North Yorkshire}

\subsubsection*{Thornton Dale \& The Wolds \hspace*{\fill}\nolinebreak[1]%
\enspace\hspace*{\fill}
\finalhyphendemerits=0
[27th October]}

\index{Thornton Dale and Wolds , North Yorkshire@Thornton Dale \& The Wolds, \emph{N. Yorks.}}

Resignation of Ron Haigh (C).

\noindent
\begin{tabular*}{\columnwidth}{@{\extracolsep{\fill}} p{0.545\columnwidth} >{\itshape}l r @{\extracolsep{\fill}}}
Janet Sanderson & C & 1122\\
Mike Beckett & LD & 574\\
\end{tabular*}

\council{Redcar and Cleveland}

\subsubsection*{Zetland (2) \hspace*{\fill}\nolinebreak[1]%
\enspace\hspace*{\fill}
\finalhyphendemerits=0
[17th November]}

\index{Zetland , Redcar and Cleveland@Zetland, \emph{Redcar \& Cleveland}}

\sloppyword{Resignations of Ron Harrison and Jim Rogers (both LD).}

\noindent
\begin{tabular*}{\columnwidth}{@{\extracolsep{\fill}} p{0.58\columnwidth} >{\itshape}l r @{\extracolsep{\fill}}}
Ron Harrison & LD & 661\\
Josh Mason & LD & 633\\
Norma Hensby & Lab & 531\\
Celia Elliott & Lab & 512\\
Michael Bateman & C & 217\\
Brian Hughes-Mundy & C & 142\\
Ian Neil & UKIP & 50\\
\end{tabular*}

\subsection*{Selby}
\index{Selby}

\index{Selby West , Selby@Selby W., \emph{Selby}}

At the May 2011 ordinary election there was an unfilled vacancy in Selby West ward due to the death of David Fagan (C).

\columnbreak

\section{Nottinghamshire}

\subsection*{Bassetlaw}
\index{Bassetlaw}

\subsubsection*{Worksop North East \hspace*{\fill}\nolinebreak[1]%
\enspace\hspace*{\fill}
\finalhyphendemerits=0
[10th February; Lab gain from C]}

\index{Worksop North East , Bassetlaw@Worksop N.E., \emph{Bassetlaw}}

\sloppyword{Disquali{fi}cation (non-attendance) of Bill Graham (C).}

\noindent
\begin{tabular*}{\columnwidth}{@{\extracolsep{\fill}} p{0.545\columnwidth} >{\itshape}l r @{\extracolsep{\fill}}}
John Anderton & Lab & 1198\\
Barry Bowles & C & 317\\
Geoff Coe & Ind & 75\\
Mark Hunter & LD & 28\\
\end{tabular*}

\subsection*{Gedling}
\index{Gedling}

\subsubsection*{Phoenix \hspace*{\fill}\nolinebreak[1]%
\enspace\hspace*{\fill}
\finalhyphendemerits=0
[15th September; LD gain from Lab]}

\index{Phoenix , Gedling@Phoenix, \emph{Gedling}}

Resignation of Ian Howarth (Lab).

\noindent
\begin{tabular*}{\columnwidth}{@{\extracolsep{\fill}} p{0.56\columnwidth} >{\itshape}l r @{\extracolsep{\fill}}}
Andrew Ellwood & LD & 566\\
Allan Leadbeater & Lab & 445\\
James Faulconbridge & C & 98\\
Lee Waters & UKIP & 42\\
\end{tabular*}

\subsection*{Mansfield}
\index{Mansfield}

MIF = Mansfield Independent Forum

\subsubsection*{Park Hall \hspace*{\fill}\nolinebreak[1]%
\enspace\hspace*{\fill}
\finalhyphendemerits=0
[20th October]}

\index{Park Hall , Mansfield@Park Hall, \emph{Mansfield}}

Death of Dorothy Beastall (Lab).

\noindent
\begin{tabular*}{\columnwidth}{@{\extracolsep{\fill}} p{0.545\columnwidth} >{\itshape}l r @{\extracolsep{\fill}}}
Ann Norman & Lab & 416\\
Linda Davidson & MIF & 164\\
Mark Quick & LD & 157\\
Andrea Hamilton & UKIP & 25\\
Fraser McFarland & C & 20\\
\end{tabular*}

\council{Newark and Sherwood}

\index{Southwell East , Newark and Sherwood@Southwell E., \emph{Newark \& Sherwood}}

At the May 2011 ordinary election there was an unfilled vacancy in Southwell East ward due to the resignation of Pauline Jenkins (LD).

\subsection*{Nottingham}
\index{Nottingham}

Elvis = Bus-Pass Elvis Party

\subsubsection*{Bridge \hspace*{\fill}\nolinebreak[1]%
\enspace\hspace*{\fill}
\finalhyphendemerits=0
[20th October]}

\index{Bridge , Nottingham@Bridge, \emph{Nottingham}}

Death of Ian MacLennan (Lab).

\noindent
\begin{tabular*}{\columnwidth}{@{\extracolsep{\fill}} p{0.545\columnwidth} >{\itshape}l r @{\extracolsep{\fill}}}
Michael Edwards & Lab & 1152\\
Saghir Akhtar & LD & 892\\
Michael Ilyas & C & 172\\
Andrew Taylor & UKIP & 50\\
David Bishop & Elvis & 27\\
\end{tabular*}

\subsection*{Rushcliffe}
\index{Rushcliffe}

\subsubsection*{Manvers (2) \hspace*{\fill}\nolinebreak[1]%
\enspace\hspace*{\fill}
\finalhyphendemerits=0
[16th June]}

\index{Manvers , Rushcliffe@Manvers, \emph{Rushcliffe}}

Ordinary election postponed from 5th May: death of outgoing councillor Wally Smith (C) who had been nominated for re-election.

\noindent
\begin{tabular*}{\columnwidth}{@{\extracolsep{\fill}} p{0.545\columnwidth} >{\itshape}l r @{\extracolsep{\fill}}}
	James Fearon & C & 730\\
	David Smith & C & 692\\
	Yvonne Thompson & Lab & 323\\
	John Thorn & Lab & 290\\
\end{tabular*}

\section{Oxfordshire}

\subsection*{Cherwell}
\index{Cherwell}

\subsubsection*{Bicester North \hspace*{\fill}\nolinebreak[1]%
\enspace\hspace*{\fill}
\finalhyphendemerits=0
[29th September]}

\index{Bicester North , Cherwell@Bicester N., \emph{Cherwell}}

Death of Carole Stewart (C).

\noindent
\begin{tabular*}{\columnwidth}{@{\extracolsep{\fill}} p{0.56\columnwidth} >{\itshape}l r @{\extracolsep{\fill}}}
Melanie Magee & C & 443\\
Kevin Walsh & Lab & 130\\
John Innes & LD & 99\\
\end{tabular*}

\council{South Oxfordshire}

\index{Didcot Northbourne , South Oxfordshire@Didcot Northbourne, \emph{S. Oxon.}}

\index{Sandford , South Oxfordshire@Sandford, \emph{S. Oxon.}}

At the May 2011 ordinary election there were unfilled vacancies in Didcot Northbourne and Sandford wards due to the resignations of Lyndon Elias and Pamela Tomlinson (both C) respectively.

\section{Rutland}
\index{Rutland}

\index{Cottesmore , Rutland@Cottesmore, \emph{Rutland}}

At the May 2011 ordinary election there was an unfilled vacancy in Cottesmore ward due to the resignation of Susie Iannantuoni (C).

\section{Shropshire}

\subsection*{Shropshire}
\index{Shropshire}

\subsubsection*{Quarry and Coton Hill \hspace*{\fill}\nolinebreak[1]%
\enspace\hspace*{\fill}
\finalhyphendemerits=0
[17th February; LD gain from C]}

\index{Quarry and Coton Hill , Shropshire@Quarry \& Coton Hill, \emph{Shrops.}}

Resignation of Maxwell Winchester (C).

\noindent
\begin{tabular*}{\columnwidth}{@{\extracolsep{\fill}} p{0.545\columnwidth} >{\itshape}l r @{\extracolsep{\fill}}}
Andrew Bannerman & LD & 356\\
Judie McCoy & C & 269\\
John Lewis & Lab & 197\\
James Collins & Ind & 30\\
\end{tabular*}

\subsubsection*{Bishop's Castle \hspace*{\fill}\nolinebreak[1]%
\enspace\hspace*{\fill}
\finalhyphendemerits=0
[29th September]}

\index{Bishop's Castle , Shropshire@Bishop's Castle, \emph{Shrops.}}

Resignation of Peter Phillips (LD).

\noindent
\begin{tabular*}{\columnwidth}{@{\extracolsep{\fill}} p{0.56\columnwidth} >{\itshape}l r @{\extracolsep{\fill}}}
Charlotte Barnes & LD & 801\\
Georgie Ellis & C & 544\\
Jean Kingdon & Lab & 80\\
Mike Tucker & Grn & 74\\
\end{tabular*}

\section{Somerset}

\subsection*{County Council}
\index{Somerset}

\subsubsection*{Shepton Mallet \hspace*{\fill}\nolinebreak[1]%
\enspace\hspace*{\fill}
\finalhyphendemerits=0
[5th May]}

\index{Shepton Mallet , Somerset@Shepton Mallet, \emph{Somerset}}

Resignation of Margaret Robinson (C).

\noindent
\begin{tabular*}{\columnwidth}{@{\extracolsep{\fill}} p{0.545\columnwidth} >{\itshape}l r @{\extracolsep{\fill}}}
John Parham & C & 1307\\
Garfield Kennedy & LD & 892\\
Chris Inchley & Lab & 711\\
Ian Forster & Grn & 256\\
\end{tabular*}

\subsubsection*{South Petherton \hspace*{\fill}\nolinebreak[1]%
\enspace\hspace*{\fill}
\finalhyphendemerits=0
[4th August; LD gain from C]}

\index{South Petherton , Somerset@South Petherton, \emph{Somerset}}

Resignation of Anne Larpent (C).

\noindent
\begin{tabular*}{\columnwidth}{@{\extracolsep{\fill}} p{0.545\columnwidth} >{\itshape}l r @{\extracolsep{\fill}}}
Paul Maxwell & LD & 1333\\
Paul Thompson & C & 943\\
Ian Greenfield & Grn & 108\\
Godfrey Davey & UKIP & 104\\
\end{tabular*}

\subsubsection*{Brent \hspace*{\fill}\nolinebreak[1]%
\enspace\hspace*{\fill}
\finalhyphendemerits=0
[1st December]}

\index{Brent , Somerset@Brent, \emph{Somerset}}

Death of Alan Ham (C).

\noindent
\begin{tabular*}{\columnwidth}{@{\extracolsep{\fill}} p{0.56\columnwidth} >{\itshape}l r @{\extracolsep{\fill}}}
John Denbee & C & 1285\\
Helen Groves & LD & 932\\
\end{tabular*}

\council{Bath and North East Somerset}

\index{Lambridge , Bath and North East Somerset@Lambridge, \emph{Bath \& N.E. Somerset}}

At the May 2011 ordinary election there was an unfilled vacancy in Lambridge ward due to the death of Richard Maybury (C).

\subsection*{North Somerset}
\index{North Somerset}

\subsubsection*{Backwell \hspace*{\fill}\nolinebreak[1]%
\enspace\hspace*{\fill}
\finalhyphendemerits=0
[8th September]}

\index{Backwell , North Somerset@Backwell, \emph{N. Somerset}}

Death of Tom Collinson (Ind).

\noindent
\begin{tabular*}{\columnwidth}{@{\extracolsep{\fill}} p{0.56\columnwidth} >{\itshape}l r @{\extracolsep{\fill}}}
Geoff Coombs & Ind & 816\\
Peter Burden & C & 314\\
Terry Connell & Lab & 124\\
Nick Alderton & LD & 69\\
\end{tabular*}

\section{Staffordshire}

\subsection*{County Council}
\index{Staffordshire}

\subsubsection*{Burton Town \hspace*{\fill}\nolinebreak[1]%
\enspace\hspace*{\fill}
\finalhyphendemerits=0
[16th June]}

\index{Burton Town , Staffordshire@Burton Town, \emph{Staffs.}}

Death of Peter Beresford (Lab).

\noindent
\begin{tabular*}{\columnwidth}{@{\extracolsep{\fill}} p{0.56\columnwidth} >{\itshape}l r @{\extracolsep{\fill}}}
Ron Clarke & Lab & 1233\\
Ahmet Orta & C & 884\\
Michael Rodgers & LD & 525\\
Peter McGuiggan & UKIP & 182\\
\end{tabular*}

\subsubsection*{Churnet Valley \hspace*{\fill}\nolinebreak[1]%
\enspace\hspace*{\fill}
\finalhyphendemerits=0
[7th July]}

\index{Churnet Valley , Staffordshire@Churnet Valley, \emph{Staffs.}}

Death of Barrie Mycock (C).

\noindent
\begin{tabular*}{\columnwidth}{@{\extracolsep{\fill}} p{0.545\columnwidth} >{\itshape}l r @{\extracolsep{\fill}}}
Mike Worthington & C & 1063\\
Mahfooz Ahmad & Lab & 491\\
Darren Federici & UKIP & 316\\
Nicholas Brewin & LD & 173\\
\end{tabular*}

\council{East Staffordshire}

\subsubsection*{Yoxall \hspace*{\fill}\nolinebreak[1]%
\enspace\hspace*{\fill}
\finalhyphendemerits=0
[3rd November]}

\index{Yoxall , East Staffordshire@Yoxall, \emph{E. Staffs.}}

Resignation of David Brookes (C).

\noindent
\begin{tabular*}{\columnwidth}{@{\extracolsep{\fill}} p{0.545\columnwidth} >{\itshape}l r @{\extracolsep{\fill}}}
Beryl Behague & C & 478\\
John McKiernan & Lab & 89\\
\end{tabular*}

\subsection*{Lichfield}
\index{Lichfield}

\index{Colton and Mavesyn Ridware , Lichfield@\sloppyword{Colton \& Mavesyn Ridware, \emph{Lichfield}}}

At the May 2011 ordinary election there was an unfilled vacancy in Colton and Mavesyn Ridware ward due to the death of Tony Hill (C).

\subsection*{Newcastle-under-Lyme}
\index{Newcastle-under-Lyme}

\subsubsection*{Kidsgrove \hspace*{\fill}\nolinebreak[1]%
\enspace\hspace*{\fill}
\finalhyphendemerits=0
[5th May]}

\index{Kidsgrove , Newcastle-under-Lyme@Kidsgrove, \emph{Newcastle-under-Lyme}}

Resignation of Tina Morrey (LD).

This by-election was combined with the 2011 ordinary election.
%; see page \pageref{KidsgroveNewcastleLyme} for the result.

\subsubsection*{Seabridge \hspace*{\fill}\nolinebreak[1]%
\enspace\hspace*{\fill}
\finalhyphendemerits=0
[23rd June]}

\index{Seabridge , Newcastle-under-Lyme@Seabridge, \emph{Newcastle-under-Lyme}}

Ordinary election postponed from 5th May: death of UKIP candidate Paul Gregory.

\noindent
\begin{tabular*}{\columnwidth}{@{\extracolsep{\fill}} p{0.545\columnwidth} >{\itshape}l r @{\extracolsep{\fill}}}
	Peter Hailstones & C & 557\\
	Colin Brookes & Lab & 497\\
	George Harvey & UKIP & 112\\
	Hilary Jones & LD & 52\\
\end{tabular*}


\subsubsection*{Newchapel \hspace*{\fill}\nolinebreak[1]%
\enspace\hspace*{\fill}
\finalhyphendemerits=0
[27th October; Lab gain from C]}

\index{Newchapel , Newcastle-under-Lyme@Newchapel, \emph{Newcastle-under-Lyme}}

Resignation of Christian Barber (C).

\noindent
\begin{tabular*}{\columnwidth}{@{\extracolsep{\fill}} p{0.545\columnwidth} >{\itshape}l r @{\extracolsep{\fill}}}
Elsie Bates & Lab & 248\\
Carl Thomson & C & 160\\
Tricia Harrison & UKIP & 118\\
Colin Brown & LD & 17\\
\end{tabular*}

\council{South Staffordshire}

\index{Bilbrook , South Staffordshire@Bilbrook, \emph{S. Staffs.}}

At the May 2011 ordinary election there was an unfilled vacancy in Bilbrook ward due to the death of Ken Mackie (C).

\subsection*{Stafford}
\index{Stafford}

\index{Barlaston and Oulton , Stafford@Barlaston \& Oulton, \emph{Stafford}}

\index{Milwich , Stafford@Milwich, \emph{Stafford}}

At the May 2011 ordinary election there were unfilled vacancies in Barlaston and Oulton, and Milwich wards due to the deaths of Peter Proctor and Doug Davis (both C) respectively.

\subsubsection*{Walton \hspace*{\fill}\nolinebreak[1]%
\enspace\hspace*{\fill}
\finalhyphendemerits=0
[20th October; Ind gain from C]}

\index{Walton , Stafford@Walton, \emph{Stafford}}

Death of Mike Carey (C).

\noindent
\begin{tabular*}{\columnwidth}{@{\extracolsep{\fill}} p{0.545\columnwidth} >{\itshape}l r
@{\extracolsep{\fill}}}
Jill Hood & Ind & 569\\
John O'Leary & C & 371\\
Lloyd Brown & Lab & 306\\
\end{tabular*}

\subsection*{Staffordshire Moorlands}
\index{Staffordshire Moorlands}

\index{Churnet , Staffordshire Moorlands@Churnet, \emph{Staffs. Moorlands}}

At the May 2011 ordinary election there was an unfilled vacancy in Churnet ward due to the death of Lionel Richardson (Ind).

\section{Suffolk}

\subsection*{County Council}
\index{Suffolk}

\subsubsection*{Wilford \hspace*{\fill}\nolinebreak[1]%
\enspace\hspace*{\fill}
\finalhyphendemerits=0
[5th May]}

\index{Wilford , Suffolk@Wilford, \emph{Suffolk}}

Death of Rosie Clarke (C).

\noindent
\begin{tabular*}{\columnwidth}{@{\extracolsep{\fill}} p{0.545\columnwidth} >{\itshape}l r @{\extracolsep{\fill}}}
Andrew Reid & C & 1913\\
Christine Hancock & LD & 796\\
Roy Burgon & Lab & 758\\
\end{tabular*}

\subsection*{Ipswich}
\index{Ipswich}

\subsubsection*{St Margaret's \hspace*{\fill}\nolinebreak[1]%
\enspace\hspace*{\fill}
\finalhyphendemerits=0
[10th November; LD gain from C]}

\index{Saint Margaret's , Ipswich@St Margaret's, \emph{Ipswich}}

Resignation of Sarah Stokes (C).

\noindent
\begin{tabular*}{\columnwidth}{@{\extracolsep{\fill}} p{0.545\columnwidth} >{\itshape}l r @{\extracolsep{\fill}}}
Cathy French & LD & 942\\
Stephen Ion & C & 871\\
Glen Chisholm & Lab & 439\\
\end{tabular*}

\subsection*{Waveney}
\index{Waveney}

\index{Oulton Broad , Waveney@Oulton Broad, \emph{Waveney}}

\index{Southwold and Reydon , Waveney@Southwold \& Reydon, \emph{Waveney}}

At the May 2011 ordinary election there were unfilled vacancies in Oulton Broad, and Southwold and Reydon wards due to the resignations of Sandy Keller and Simon Tobin (both C) respectively.

\subsubsection*{Worlingham \hspace*{\fill}\nolinebreak[1]%
\enspace\hspace*{\fill}
\finalhyphendemerits=0
[Monday 19th December]}

\index{Worlingham , Waveney@Worlingham, \emph{Waveney}}

Resignation of Andrew Draper (Ind elected as C).

\noindent
\begin{tabular*}{\columnwidth}{@{\extracolsep{\fill}} p{0.545\columnwidth} >{\itshape}l r @{\extracolsep{\fill}}}
Norman Brooks & C & 708\\
Sylvia Robbins & Lab & 586\\
Sue Bergin & Grn & 137\\
Stuart Foulger & UKIP & 64\\
Doug Farmer & LD & 48\\
\end{tabular*}

\section{Surrey}

\subsection*{County Council}
\index{Surrey}

\subsubsection*{Cranleigh and Ewhurst \hspace*{\fill}\nolinebreak[1]%
\enspace\hspace*{\fill}
\finalhyphendemerits=0
[5th May]}

\index{Cranleigh and Ewhurst , Surrey@Cranleigh \& Ewhurst, \emph{Surrey}}

Resignation of Jonathan Lord (C).

\noindent
\begin{tabular*}{\columnwidth}{@{\extracolsep{\fill}} p{0.545\columnwidth} >{\itshape}l r @{\extracolsep{\fill}}}
Alan Young & C & 2830\\
Diane James & Ind & 1093\\
Richard Cole & LD & 933\\
Lynda MacDermott & Lab & 532\\
\end{tabular*}

\subsubsection*{Shalford \hspace*{\fill}\nolinebreak[1]%
\enspace\hspace*{\fill}
\finalhyphendemerits=0
[5th May]}

\index{Shalford , Surrey@Shalford, \emph{Surrey}}

Resignation of Tony Rooth (C).

\noindent
\begin{tabular*}{\columnwidth}{@{\extracolsep{\fill}} p{0.545\columnwidth} >{\itshape}l r @{\extracolsep{\fill}}}
Simon Gimson & C & 3602\\
Andrew Barnes & LD & 1087\\
Michael Jeram & Lab & 701\\
\end{tabular*}

\subsubsection*{St Johns and Brookwood \hspace*{\fill}\nolinebreak[1]%
\enspace\hspace*{\fill}
\finalhyphendemerits=0
[19th May]}

\index{Saint Johns and Brookwood , Surrey@St Johns \& Brookwood, \emph{Surrey}}

Resignation of Elizabeth Compton (C).

\noindent
\begin{tabular*}{\columnwidth}{@{\extracolsep{\fill}} p{0.545\columnwidth} >{\itshape}l r @{\extracolsep{\fill}}}
Linda Kemeny & C & 1343\\
Tina Liddington & LD & 1058\\
Janice Worgan & Lab & 188\\
Duncan Clarke & UKIP & 155\\
\end{tabular*}

\subsection*{Mole Valley}
\index{Mole Valley}

\subsubsection*{Bookham South \hspace*{\fill}\nolinebreak[1]%
\enspace\hspace*{\fill}
\finalhyphendemerits=0
[20th October]}

\index{Bookham South , Mole Valley@Bookham S., \emph{Mole Valley}}

Death of Anne Howarth (LD).

\noindent
\begin{tabular*}{\columnwidth}{@{\extracolsep{\fill}} p{0.545\columnwidth} >{\itshape}l r
@{\extracolsep{\fill}}}
Stella Brooks & LD & 1043\\
Gail Collett & C & 940\\
Bob Cane & UKIP & 228\\
\end{tabular*}

\subsection*{Woking}
\index{Woking}

\subsubsection*{Goldsworth East \hspace*{\fill}\nolinebreak[1]%
\enspace\hspace*{\fill}
\finalhyphendemerits=0
[5th May]}

\index{Goldsworth East , Woking@Goldsworth E., \emph{Woking}}

Resignation of Rosie Sharpley (LD).

This by-election was combined with the 2011 ordinary election.
%; see page \pageref{GoldsworthEastWoking} for the result.

\section{Warwickshire}

\council{Nuneaton and Bedworth}

\subsubsection*{Bar Pool \hspace*{\fill}\nolinebreak[1]%
\enspace\hspace*{\fill}
\finalhyphendemerits=0
[5th May]}

\index{Bar Pool , Nuneaton and Bedworth@Bar Pool, \emph{Nuneaton \& Bedworth}}

Resignation of Frank McGale (Lab).

\noindent
\begin{tabular*}{\columnwidth}{@{\extracolsep{\fill}} p{0.545\columnwidth} >{\itshape}l r @{\extracolsep{\fill}}}
Victoria Fowler & Lab & 1034\\
Michael Bannister & C & 519\\
Alwyn Deacon & BNP & 204\\
Andrew Crichton & LD & 142\\
\sloppyword{Andreas Hammerschmiedt} & UKIP & 65\\
Thomas Sidwell & TUSC & 38\\
\end{tabular*}

\subsection*{Stratford-on-Avon}
\index{Stratford-on-Avon}

\subsubsection*{Harbury \hspace*{\fill}\nolinebreak[1]%
\enspace\hspace*{\fill}
\finalhyphendemerits=0
[5th May]}

\index{Harbury , Stratford-on-Avon@Harbury, \emph{Stratford-on-Avon}}

Resignation of Richard Tonge (LD).

This by-election was combined with the 2011 ordinary election.
%; see page \pageref{HarburyStratfordAvon} for the result.

\subsubsection*{Long Itchington \hspace*{\fill}\nolinebreak[1]%
\enspace\hspace*{\fill}
\finalhyphendemerits=0
[5th May]}

\index{Long Itchington , Stratford-on-Avon@Long Itchington, \emph{Stratford-on-Avon}}

Resignation of Bob Stevens (LD).

This by-election was combined with the 2011 ordinary election.
%; see page \pageref{LongItchingtonStratfordAvon} for the result.

\section{West Sussex}

\subsection*{Adur}
\index{Adur}

\subsubsection*{Cokeham \hspace*{\fill}\nolinebreak[1]%
\enspace\hspace*{\fill}
\finalhyphendemerits=0
[Tuesday 18th October]}

\index{Cokeham , Adur@Cokeham, \emph{Adur}}

Death of Norman Wright (C).

\noindent
\begin{tabular*}{\columnwidth}{@{\extracolsep{\fill}} p{0.545\columnwidth} >{\itshape}l r @{\extracolsep{\fill}}}
Nicholas Pigott & C & 288\\
Barry Mear & Lab & 282\\
David Bamber & UKIP & 91\\
Jennie Tindall & Grn & 35\\
Cyril Cannings & LD & 31\\
\end{tabular*}

\subsection*{Arun}
\index{Arun}

\index{Barnham , Arun@Barnham, \emph{Arun}}

At the May 2011 ordinary election there was an unfilled vacancy in Barnham ward due to the resignation of Melissa Briggs (Ind).

\subsection*{Mid Sussex}
\index{Mid Sussex}

\index{Haywards Heath Bentswood , Mid Sussex@\sloppyword{Haywards Heath Bentswood, \emph{Mid Sussex}}}

At the May 2011 ordinary election there was an unfilled vacancy in Haywards Heath Bentswood ward due to the death of Paddy Henry (Lab).

\section{Worcestershire}

\subsection*{Bromsgrove}
\index{Bromsgrove}

\index{Wythall South , Bromsgrove@Wythall S., \emph{Bromsgrove}}

At the May 2011 ordinary election there was an unfilled vacancy in Wythall South ward due to the resignation of John Duddy (C).

\subsection*{Redditch}
\index{Redditch}

\subsubsection*{Greenlands \hspace*{\fill}\nolinebreak[1]%
\enspace\hspace*{\fill}
\finalhyphendemerits=0
[5th May]}

\index{Greenlands , Redditch@Greenlands, \emph{Redditch}}

Resignation of Graham Vickery (Lab).

This by-election was combined with the 2011 ordinary election.
%; see page \pageref{GreenlandsRedditch} for the result.

\subsection*{Wychavon}
\index{Wychavon}

\index{Elmley Castle and Somerville , Wychavon@Elmley Castle \& Somerville, \emph{Wychavon}}

At the May 2011 ordinary election there was an unfilled vacancy in Elmley Castle and Somerville ward due to the death of Anna Mackison (C).

\subsubsection*{Elmley Castle and Somerville \hspace*{\fill}\nolinebreak[1]%
\enspace\hspace*{\fill}
\finalhyphendemerits=0
[23rd June]}

\index{Elmley Castle and Somerville , Wychavon@Elmley Castle \& Somerville, \emph{Wychavon}}

No candidates nominated in 2011 ordinary election.

\noindent
\begin{tabular*}{\columnwidth}{@{\extracolsep{\fill}} p{0.545\columnwidth} >{\itshape}l r @{\extracolsep{\fill}}}
Roma Kirke & C & 435\\
Jayne Lewis & LD & 197\\
\end{tabular*}

\section{Glamorgan}

\subsection*{Cardiff}
\index{Cardiff}

\subsubsection*{Riverside \hspace*{\fill}\nolinebreak[1]%
\enspace\hspace*{\fill}
\finalhyphendemerits=0
[3rd March; Lab gain from PC]}

\index{Riverside , Cardiff@Riverside, \emph{Cardiff}}

Resignation of Gwenllian Lansdown (PC).

\noindent
\begin{tabular*}{\columnwidth}{@{\extracolsep{\fill}} p{0.545\columnwidth} >{\itshape}l r @{\extracolsep{\fill}}}
Iona Gordon & Lab & 1700\\
Steve Garrett & PC & 1099\\
James Roach & C & 369\\
Yvan Maurel & Grn & 277\\
Gwilym Owen & LD & 187\\
\end{tabular*}

\section{Gwent}

\subsection*{Torfaen}
\index{Torfaen}

\subsubsection*{Pontypool \hspace*{\fill}\nolinebreak[1]%
\enspace\hspace*{\fill}
\finalhyphendemerits=0
[24th March; Lab gain from Ind]}

\index{Pontypool , Torfaen@Pontypool, \emph{Torfaen}}

Resignation of Fred Wildgust (Ind).

\noindent
\begin{tabular*}{\columnwidth}{@{\extracolsep{\fill}} p{0.545\columnwidth} >{\itshape}l r @{\extracolsep{\fill}}}
John Killick & Lab & 178\\
Mike Harris & Ind & 161\\
Ian Williams & Ind & 80\\
Paul Valente & PC & 50\\
Nigel Harris & Ind & 46\\
Sarah Bousie & Ind & 25\\
Adrian Lang & C & 18\\
Gaynor James & Ind & 12\\
\end{tabular*}

\subsubsection*{Snatchwood \hspace*{\fill}\nolinebreak[1]%
\enspace\hspace*{\fill}
\finalhyphendemerits=0
[Tuesday 26th July; Lab gain from People's Voice]}

\index{Snatchwood , Torfaen@Snatchwood, \emph{Torfaen}}

Death of Tom Gould (Ind elected as People's Voice).

\noindent
\begin{tabular*}{\columnwidth}{@{\extracolsep{\fill}} p{0.545\columnwidth} >{\itshape}l r @{\extracolsep{\fill}}}
Barry Taylor & Lab & 239\\
Mike Harris & Ind & 161\\
Steve Joy & Ind & 41\\
Nigel Harris & Ind & 37\\
Caroline Powell & PC & 12\\
Richard Overton & C & 9\\
\end{tabular*}

\section{Mid and West Wales}

\subsection*{Carmarthenshire}
\index{Carmarthenshire}

\subsubsection*{Llanegwad \hspace*{\fill}\nolinebreak[1]%
\enspace\hspace*{\fill}
\finalhyphendemerits=0
[23rd June; PC gain from Ind]}

\index{Llanegwad , Carmarthenshire@Llanegwad, \emph{Carmarthenshire.}}

Disqualification (non-attendance) of Dilwyn Williams (Ind).

\noindent
\begin{tabular*}{\columnwidth}{@{\extracolsep{\fill}} p{0.545\columnwidth} >{\itshape}l r @{\extracolsep{\fill}}}
John Charles & PC & 494\\
Clive Pugh & Ind & 455\\
\end{tabular*}

\section{North Wales}

\subsection*{Conwy}
\index{Conwy}

\subsubsection*{Marl \hspace*{\fill}\nolinebreak[1]%
\enspace\hspace*{\fill}
\finalhyphendemerits=0
[20th January; LD gain from C]}

\index{Marl , Conwy@Marl, \emph{Conwy}}

Resignation of Linda Hurr (C).

\noindent
\begin{tabular*}{\columnwidth}{@{\extracolsep{\fill}} p{0.545\columnwidth} >{\itshape}l r @{\extracolsep{\fill}}}
Susan Shotter & LD & 389\\
Julie Fallon & C & 270\\
Mike Pritchard & Lab & 216\\
Jason Landy & Ind & 87\\
\end{tabular*}

\subsubsection*{Uwchaled \hspace*{\fill}\nolinebreak[1]%
\enspace\hspace*{\fill}
\finalhyphendemerits=0
[16th June; Ind gain from PC]}

\index{Uwchaled , Conwy@Uwchaled, \emph{Conwy}}

Death of Wil Charles Edwards (PC).

\noindent
\begin{tabular*}{\columnwidth}{@{\extracolsep{\fill}} p{0.56\columnwidth} >{\itshape}l r @{\extracolsep{\fill}}}
Pete Groudd & Ind & 477\\
Renny Crossley & C & 30\\
\end{tabular*}

\subsection*{Gwynedd}
\index{Gwynedd}

LlG = Llais Gwynedd

\subsubsection*{Arllechwedd \hspace*{\fill}\nolinebreak[1]%
\enspace\hspace*{\fill}
\finalhyphendemerits=0
[16th June; PC gain from LD]}

\index{Arllechwedd , Gwynedd@Arllechwedd, \emph{Gwynedd}}

Death of John Robert Jones (LD).

\noindent
\begin{tabular*}{\columnwidth}{@{\extracolsep{\fill}} p{0.545\columnwidth} >{\itshape}l r @{\extracolsep{\fill}}}
Dafydd Meurig & PC & 255\\
\sloppyword{Edmond Douglas-Pennant} & LD & 93\\
David Edwards & Lab & 72\\
Jennie Lewis & C & 35\\
\end{tabular*}

\subsubsection*{Glyder \hspace*{\fill}\nolinebreak[1]%
\enspace\hspace*{\fill}
\finalhyphendemerits=0
[21st July]}

\index{Glyder , Gwynedd@Glyder, \emph{Gwynedd}}

Death of Dai Rees Jones (PC).

\noindent
\begin{tabular*}{\columnwidth}{@{\extracolsep{\fill}} p{0.56\columnwidth} >{\itshape}l r @{\extracolsep{\fill}}}
Elin Walker Jones & PC & 207\\
Doug Madge & LD & 194\\
Jennie Lewis & C & 65\\
Martyn Singleton & Lab & 60\\
\end{tabular*}

\subsubsection*{Diffwys and Maenofferen \hspace*{\fill}\nolinebreak[1]%
\enspace\hspace*{\fill}
\finalhyphendemerits=0
[29th September]}

\index{Diffwys and Maenofferen , Gwynedd@Diffwys \& Maenofferen, \emph{Gwynedd}}

Resignation of Richard Lloyd Jones (LlG).

\noindent
\begin{tabular*}{\columnwidth}{@{\extracolsep{\fill}} p{0.64\columnwidth} >{\itshape}l r @{\extracolsep{\fill}}}
Mandy Williams-Davies & PC & 210\\
Catrin Roberts & LlG & 153\\
\end{tabular*}

\subsubsection*{Penrhyndeudraeth \hspace*{\fill}\nolinebreak[1]%
\enspace\hspace*{\fill}
\finalhyphendemerits=0
[29th September]}

\index{Penrhyndeudraeth , Gwynedd@\sloppyword{Penrhyndeudraeth, \emph{Gwynedd}}}

Resignation of Dewi Lewis (PC).

\noindent
\begin{tabular*}{\columnwidth}{@{\extracolsep{\fill}} p{0.56\columnwidth} >{\itshape}l r @{\extracolsep{\fill}}}
Gareth Thomas & PC & 515\\
Rhian Jones & LlG & 233\\
Dafydd Thomas & Ind & 91\\
\end{tabular*}

\subsection*{Wrexham}
\index{Wrexham}

\subsubsection*{Marchwiel \hspace*{\fill}\nolinebreak[1]%
\enspace\hspace*{\fill}
\finalhyphendemerits=0
[16th June]}

\index{Marchwiel , Wrexham@Marchwiel, \emph{Wrexham}}

Resignation of June Fearnall (Ind).

\noindent
\begin{tabular*}{\columnwidth}{@{\extracolsep{\fill}} p{0.56\columnwidth} >{\itshape}l r @{\extracolsep{\fill}}}
John Pritchard & Ind & 302\\
John Bell & C & 172\\
\end{tabular*}

\subsubsection*{Ruabon \hspace*{\fill}\nolinebreak[1]%
\enspace\hspace*{\fill}
\finalhyphendemerits=0
[17th November; Lab gain from PC]}

\index{Ruabon , Wrexham@Ruabon, \emph{Wrexham}}

Death of Barrie Price (PC).

\noindent
\begin{tabular*}{\columnwidth}{@{\extracolsep{\fill}} p{0.545\columnwidth} >{\itshape}l r @{\extracolsep{\fill}}}
Dana-Louise Davies & Lab & 231\\
Pol Wong & PC & 230\\
Andy Kendrick & Ind & 155\\
Adam Owen & C & 59\\
\end{tabular*}

\section[Aberdeenshire Councils]{\sloppyword{Aberdeenshire Councils}}

\subsection*{Aberdeen}
\index{Aberdeen}

\subsubsection*{Dyce/Bucksburn/Danestone \hspace*{\fill}\nolinebreak[1]%
\enspace\hspace*{\fill}
\finalhyphendemerits=0
[19th May; SNP gain from LD]}

\index{Dyce/Bucksburn/Danestone , Aberdeen@\sloppyword{Dyce\slash Bucksburn\slash Danestone, \emph{Aberdeen}}}

Death of Ron Clark (LD).

\noindent
\begin{tabular*}{\columnwidth}{@{\extracolsep{\fill}} p{0.545\columnwidth} >{\itshape}l r @{\extracolsep{\fill}}}
Neil MacGregor & SNP & 2090\\
Graeme Lawrence & Lab & 941\\
Kristian Chapman & LD & 446\\
Ross Thomson & C & 352\\
Angela Joss & Ind & 150\\
Rhonda Reekie & Grn & 88\\
\end{tabular*}

\subsubsection*{{Airyhall\slash{}Broomhill\slash{}Garthdee} \hspace*{\fill}\nolinebreak[1]%
\enspace\hspace*{\fill}
\finalhyphendemerits=0
[23rd June; SNP gain from LD]}

\index{Airyhall/Broomhill/Garthdee , Aberdeen@\sloppyword{Airyhall\slash Broomhill\slash Garthdee, \emph{Aberdeen}}}

Resignation of Scott Cassie (Ind elected as LD).

\noindent
\begin{tabular*}{\columnwidth}{@{\extracolsep{\fill}} p{0.545\columnwidth} >{\itshape}l r @{\extracolsep{\fill}}}
\emph{First preferences}\\
Gordon Townson & SNP & 1112\\
Angela Taylor & Lab & 783\\
Bill Berry & C & 649\\
Gregor McAbery & LD & 554\\
Richie Brian & Grn & 101\\
Graham Bennett & Ind & 98\\
Hamish MacKay & Ind & 32\\
Dave MacDonald & NF & 25\\
\end{tabular*}

%\noindent
%\begin{tabular*}{\columnwidth}{@{\extracolsep{\fill}} p{0.545\columnwidth} >{\itshape}l r @{\extracolsep{\fill}}}
\emph{Brian, Bennett,
MacKay and MacDonald eliminated}: Townson 1161 Taylor 829 Berry 678 McAbery 589
%Gordon Townson & SNP & 1161\\
%Angela Taylor & Lab & 829\\
%Bill Berry & C & 676\\
%Gregor McAbery & LD & 599\\
%\end{tabular*}

%\noindent
%\begin{tabular*}{\columnwidth}{@{\extracolsep{\fill}} p{0.545\columnwidth} >{\itshape}l r @{\extracolsep{\fill}}}
\emph{McAbery eliminated}: Townson 1263 Taylor 956 Berry 801
%Gordon Townson & SNP & 1263\\
%Angela Taylor & Lab & 956\\
%Bill Berry & C & 801\\
%\end{tabular*}

\noindent
\begin{tabular*}{\columnwidth}{@{\extracolsep{\fill}} p{0.545\columnwidth} >{\itshape}l r @{\extracolsep{\fill}}}
\emph{Berry eliminated}\\
Gordon Townson & SNP & 1441\\
Angela Taylor & Lab & 1068\\
\end{tabular*}

\columnbreak

\section{Ayrshire Councils}

\subsection*{North Ayrshire}
\index{North Ayrshire}

ASPP = All Scotland Pensioners Party

\subsubsection*{Saltcoats and Stevenston \hspace*{\fill}\nolinebreak[1]%
\enspace\hspace*{\fill}
\finalhyphendemerits=0
[25th August]}

\index{Saltcoats and Stevenston , North Ayrshire@Saltcoats \& Stevenston, \emph{N. Ayrshire}}

Resignation of David Munn (Lab).

\noindent
\begin{tabular*}{\columnwidth}{@{\extracolsep{\fill}} p{0.53\columnwidth} >{\itshape}l r @{\extracolsep{\fill}}}
\emph{First preferences}\\
Jim Montgomerie & Lab & 1914\\
Nan Wallace & SNP & 1306\\
Chris Barr & C & 284\\
Jimmy Miller & ASPP & 211\\
Gerard Pollock & Ind & 114\\
Gordon Bain & LD & 56\\
Louise McDaid & SocLab & 43\\
\end{tabular*}

%\noindent
%\begin{tabular*}{\columnwidth}{@{\extracolsep{\fill}} p{0.545\columnwidth} >{\itshape}l r @{\extracolsep{\fill}}}
\emph{Bain and McDaid eliminated}: Montgomerie 1936 Wallace 1326 Barr 297 Miller 222 Pollock 123
%Jim Montgomerie & Lab & 1936\\
%Nan Wallace & SNP & 1326\\
%Chris Barr & C & 297\\
%Jimmy Miller & ASPP & 222\\
%Gerard Pollock & Ind & 123\\
%\end{tabular*}

\noindent
\begin{tabular*}{\columnwidth}{@{\extracolsep{\fill}} p{0.545\columnwidth} >{\itshape}l r @{\extracolsep{\fill}}}
\emph{Pollock eliminated}\\
Jim Montgomerie & Lab & 1963\\
Nan Wallace & SNP & 1363\\
Chris Barr & C & 308\\
Jimmy Miller & ASPP & 240\\
\end{tabular*}

\section{Border Councils}

\council{Dumfries and Galloway}

\subsubsection*{Abbey \hspace*{\fill}\nolinebreak[1]%
\enspace\hspace*{\fill}
\finalhyphendemerits=0
[16th June; Lab gain from C]}

\index{Abbey , Dumfries and Galloway@Abbey, \emph{Dumfries \& Galloway}}

Resignation of Michael Thomson (C).

\noindent
\begin{tabular*}{\columnwidth}{@{\extracolsep{\fill}} p{0.545\columnwidth} >{\itshape}l r @{\extracolsep{\fill}}}
\emph{First preferences}\\
Kath Lord & C & 1236\\
Tom McAughtrie & Lab & 1196\\
Yowann Byghan & SNP & 678\\
\end{tabular*}

\noindent
\begin{tabular*}{\columnwidth}{@{\extracolsep{\fill}} p{0.545\columnwidth} >{\itshape}l r @{\extracolsep{\fill}}}
\emph{Byghan eliminated}\\
Tom McAughtrie & Lab & 1448\\
Kath Lord & C & 1391\\
\end{tabular*}

\section{Clyde Councils}

\subsection*{Glasgow}
\index{Glasgow}

Britnca = Britannica

\subsubsection*{Hillhead \hspace*{\fill}\nolinebreak[1]%
\enspace\hspace*{\fill}
\finalhyphendemerits=0
[17th November]}

\index{Hillhead , Glasgow@Hillhead, \emph{Glsagow}}

Death of George Roberts (SNP).

\noindent
\begin{tabular*}{\columnwidth}{@{\extracolsep{\fill}} p{0.52\columnwidth} >{\itshape}l r @{\extracolsep{\fill}}}
\emph{First preferences}\\
Ken Andrew & SNP & 1026\\
Martin McElroy & Lab & 945\\
Stuart Leckie & Grn & 435\\
Maya Forrest & C & 372\\
Ewan Hoyle & LD & 307\\
Neil Craig & UKIP & 36\\
Charles Baillie & Britnca & 11\\
\end{tabular*}

%\noindent
%\begin{tabular*}{\columnwidth}{@{\extracolsep{\fill}} p{0.545\columnwidth} >{\itshape}l r @{\extracolsep{\fill}}}
%\multicolumn{3}{@{\extracolsep{\fill}}l}{
	\emph{Hoyle, Craig and Baillie eliminated}: Andrew 1079 McElroy 992 Leckie 556 Forrest 441
%}\\
%Ken Andrew & SNP & 1079\\
%Martin McElroy & Lab & 992\\
%Stuart Leckie & Grn & 556\\
%Maya Forrest & C & 441\\
%\end{tabular*}

%\noindent
%\begin{tabular*}{\columnwidth}{@{\extracolsep{\fill}} p{0.545\columnwidth} >{\itshape}l r @{\extracolsep{\fill}}}
\emph{Forrest eliminated}: Andrew 1174 McElroy 1057 Leckie 639
%Ken Andrew & SNP & 1174\\
%Martin McElroy & Lab & 1057\\
%Stuart Leckie & Grn & 639\\
%\end{tabular*}

\noindent
\begin{tabular*}{\columnwidth}{@{\extracolsep{\fill}} p{0.545\columnwidth} >{\itshape}l r @{\extracolsep{\fill}}}
\emph{Leckie eliminated}\\
Ken Andrew & SNP & 1386\\
Martin McElroy & Lab & 1276\\
\end{tabular*}

\subsection*{North Lanarkshire}
\index{North Lanarkshire}

\subsubsection*{Coatbridge North and Glenboig \hspace*{\fill}\nolinebreak[1]%
\enspace\hspace*{\fill}
\finalhyphendemerits=0
[27th October]}

\index{Coatbridge North and Glenboig , North Lanarkshire@Coatbridge N. \& Glenboig, \emph{N. Lanarks.}}

Death of Tony Clarke (Lab).

\noindent
\begin{tabular*}{\columnwidth}{@{\extracolsep{\fill}} p{0.545\columnwidth} >{\itshape}l r @{\extracolsep{\fill}}}
Michael McPake & Lab & 1527\\
Julie McAnulty & SNP & 1139\\
Robert Burgess & C & 174\\
Graham Dale & LD & 78\\
\end{tabular*}

\subsection*{Renfrewshire}
\index{Renfrewshire}

SSP = Scottish Socialist Party

\subsubsection*{Paisley South \hspace*{\fill}\nolinebreak[1]%
\enspace\hspace*{\fill}
\finalhyphendemerits=0
[17th March; Lab gain from SNP]}

\index{Paisley South , Renfrewshire@Paisley S., \emph{Renfs.}}

Death of Jim Mitchell (SNP).

\noindent
\begin{tabular*}{\columnwidth}{@{\extracolsep{\fill}} p{0.545\columnwidth} >{\itshape}l r @{\extracolsep{\fill}}}
\emph{First preferences}\\
Roy Glen & Lab & 2081\\
David McCartney & SNP & 1366\\
Alison Cook & C & 388\\
Gary Pearson & Ind & 164\\
Ross Stalker & LD & 134\\
Jimmy Kerr & SSP & 82\\
\end{tabular*}

\noindent
\begin{tabular*}{\columnwidth}{@{\extracolsep{\fill}} p{0.545\columnwidth} >{\itshape}l r @{\extracolsep{\fill}}}
\emph{Kerr eliminated}\\
Roy Glen & Lab & 2102\\
David McCartney & SNP & 1382\\
Alison Cook & C & 392\\
Gary Pearson & Ind & 179\\
Ross Stalker & LD & 139\\
\end{tabular*}

\subsection*{South Lanarkshire}
\index{South Lanarkshire}

\subsubsection*{Hamilton West and Earnock \hspace*{\fill}\nolinebreak[1]%
\enspace\hspace*{\fill}
\finalhyphendemerits=0
[8th December; SNP gain from Ind]}

\index{Hamilton West and Earnock , South Lanarkshire@Hamilton W. \& Earnock, \emph{S. Lanarks.}}

Death of Tommy Gilligan (Ind).

\noindent
\begin{tabular*}{\columnwidth}{@{\extracolsep{\fill}} p{0.545\columnwidth} >{\itshape}l r @{\extracolsep{\fill}}}
John Menzies & SNP & 822\\
Stuart Gallacher & Lab & 607\\
Connar McBain & C & 214\\
\end{tabular*}

\subsection*{West Dunbartonshire}
\index{West Dunbartonshire}

\subsubsection*{Kilpatrick \hspace*{\fill}\nolinebreak[1]%
\enspace\hspace*{\fill}
\finalhyphendemerits=0
[3rd March]}

\index{Kilpatrick , West Dunbartonshire@Kilpatrick, \emph{W. Duns.}}

Resignation of Margaret Bootland (Lab).

\noindent
\begin{tabular*}{\columnwidth}{@{\extracolsep{\fill}} p{0.545\columnwidth} >{\itshape}l r @{\extracolsep{\fill}}}
Lawrence O'Neill & Lab & 1382\\
Frank McNiff & SNP & 758\\
Douglas Boyle & C & 161\\
\end{tabular*}

\section{Forth Councils}

\subsection*{Edinburgh}
\index{Edinburgh}

\subsubsection*{City Centre \hspace*{\fill}\nolinebreak[1]%
\enspace\hspace*{\fill}
\finalhyphendemerits=0
[18th August]}

\index{City Centre , Edinburgh@City Centre, \emph{Edinburgh}}

Resignation of David Beckett (SNP).

\noindent
\begin{tabular*}{\columnwidth}{@{\extracolsep{\fill}} p{0.545\columnwidth} >{\itshape}l r @{\extracolsep{\fill}}}
\emph{First preferences}\\
Iain McGill & C & 837\\
Alasdair Rankin & SNP & 797\\
Karen Doran & Lab & 682\\
Melanie Main & Grn & 494\\
John Carson & Ind & 394\\
Alistair Hodgson & LD & 251\\
\end{tabular*}

%\noindent
%\begin{tabular*}{\columnwidth}{@{\extracolsep{\fill}} p{0.545\columnwidth} >{\itshape}l r @{\extracolsep{\fill}}}
\emph{Hodgson eliminated}: McGill 904 Rankin 825 Doran 716 Main 578 Carson 402
%Iain McGill & C & 904\\
%Alasdair Rankin & SNP & 825\\
%Karen Doran & Lab & 716\\
%Melanie Main & Grn & 576\\
%John Carson & Ind & 402\\
%\end{tabular*}

%\noindent
%\begin{tabular*}{\columnwidth}{@{\extracolsep{\fill}} p{0.545\columnwidth} >{\itshape}l r @{\extracolsep{\fill}}}
\emph{Carson eliminated}: McGill 1043 Rankin 893 Doran 745 Main 635
%Iain McGill & C & 1043\\
%Alasdair Rankin & SNP & 893\\
%Karen Doran & Lab & 745\\
%Melanie Main & Grn & 635\\
%\end{tabular*}

%\noindent
%\begin{tabular*}{\columnwidth}{@{\extracolsep{\fill}} p{0.545\columnwidth} >{\itshape}l r @{\extracolsep{\fill}}}
\emph{Main eliminated}: McGill 1110 Rankin 1081 Doran 968
%Iain McGill & C & 1110\\
%Alasdair Rankin & SNP & 1081\\
%Karen Doran & Lab & 968\\
%\end{tabular*}

\noindent
\begin{tabular*}{\columnwidth}{@{\extracolsep{\fill}} p{0.545\columnwidth} >{\itshape}l r @{\extracolsep{\fill}}}
\emph{Doran eliminated}\\
Alasdair Rankin & SNP & 1368\\
Iain McGill & C & 1264\\
\end{tabular*}

\subsection*{Falkirk}
\index{Falkirk}

\subsubsection*{Bo'ness and Blackness \hspace*{\fill}\nolinebreak[1]%
\enspace\hspace*{\fill}
\finalhyphendemerits=0
[9th June]}

\index{Bo'ness and Blackness , Falkirk@Bo'ness \& Blackness, \emph{Falkirk}}

Death of John Constable (SNP).

\noindent
\begin{tabular*}{\columnwidth}{@{\extracolsep{\fill}} p{0.545\columnwidth} >{\itshape}l r @{\extracolsep{\fill}}}
Sandy Turner & SNP & 1621\\
David Aitchison & Lab & 893\\
Lynn Munro & C & 231\\
Gerry Lawton & Ind & 59\\
\end{tabular*}

\section{Highland Councils}

\council{Argyll and Bute}

\subsubsection*{Oban North and Lorn \hspace*{\fill}\nolinebreak[1]%
\enspace\hspace*{\fill}
\finalhyphendemerits=0
[3rd November]}

\index{Oban North and Lorn , Argyll and Bute@Oban N. \& Lorn, \emph{Argyll \& Bute}}

Death of Donald MacDonald (SNP).

\noindent
\begin{tabular*}{\columnwidth}{@{\extracolsep{\fill}} p{0.545\columnwidth} >{\itshape}l r @{\extracolsep{\fill}}}
\emph{First preferences}\\
Louise Glen-Lee & SNP & 1081\\
Roy Rutherford & C & 505\\
Gwyneth Neal & Ind & 438\\
Graham Dale & LD & 260\\
George Doyle & Ind & 165\\
\end{tabular*}

%\noindent
%\begin{tabular*}{\columnwidth}{@{\extracolsep{\fill}} p{0.545\columnwidth} >{\itshape}l r @{\extracolsep{\fill}}}
\emph{Doyle eliminated}: Glen-Lee 1117 Rutherford 523 Neal 496 Dale 273
%Louise Glen-Lee & SNP & 1117\\
%Roy Rutherford & C & 523\\
%Gwyneth Neal & Ind & 496\\
%Graham Dale & LD & 273\\
%\end{tabular*}

\noindent
\begin{tabular*}{\columnwidth}{@{\extracolsep{\fill}} p{0.545\columnwidth} >{\itshape}l r @{\extracolsep{\fill}}}
\emph{Dale eliminated}\\
Louise Glen-Lee & SNP & 1179\\
Roy Rutherford & C & 591\\
Gwyneth Neal & Ind & 561\\
\end{tabular*}

\subsection*{Highland}
\index{Highland}

\subsubsection*{Wick \hspace*{\fill}\nolinebreak[1]%
\enspace\hspace*{\fill}
\finalhyphendemerits=0
[7th April; SNP gain from Ind]}

\index{Wick , Highland@Wick, \emph{Highland}}

Resignation of Katrina MacNab (Ind).

\noindent
\begin{tabular*}{\columnwidth}{@{\extracolsep{\fill}} p{0.545\columnwidth} >{\itshape}l r @{\extracolsep{\fill}}}
\emph{First preferences}\\
Gail Ross & SNP & 979\\
Neil MacDonald & Lab & 409\\
Niall Smith & Ind & 207\\
Claire Clarke & LD & 206\\
Jim Oag & Ind & 184\\
Laurel Bush & Ind & 73\\
Michael Carr & C & 33\\
\end{tabular*}

%\noindent
%\begin{tabular*}{\columnwidth}{@{\extracolsep{\fill}} p{0.545\columnwidth} >{\itshape}l r @{\extracolsep{\fill}}}
\emph{Bush and Carr eliminated}: Ross 999 MacDonald 421 Clarke 219 Smith 217 Oag 202
%Gail Ross & SNP & 999\\
%Neil MacDonald & Lab & 421\\
%Claire Clarke & LD & 219\\
%Niall Smith & Ind & 217\\
%Jim Oag & Ind & 202\\
%\end{tabular*}

\noindent
\begin{tabular*}{\columnwidth}{@{\extracolsep{\fill}} p{0.545\columnwidth} >{\itshape}l r @{\extracolsep{\fill}}}
\emph{Oag eliminated}\\
Gail Ross & SNP & 1049\\
Neil MacDonald & Lab & 463\\
Niall Smith & Ind & 245\\
Claire Clarke & LD & 236\\
\end{tabular*}

\columnbreak

\subsubsection*{Tain and Easter Ross \hspace*{\fill}\nolinebreak[1]%
\enspace\hspace*{\fill}
\finalhyphendemerits=0
[9th June]}

\index{Tain and Easter Ross , Highland@Tain \& Easter Ross, \emph{Highland}}

Death of Alan Torrance (SNP elected as Ind).

\noindent
\begin{tabular*}{\columnwidth}{@{\extracolsep{\fill}} p{0.545\columnwidth} >{\itshape}l r @{\extracolsep{\fill}}}
\emph{First preferences}\\
Derek Louden & SNP & 837\\
Fiona Robertson & Ind & 811\\
Ruairidh Mackenzie & Ind & 467\\
Antony Gardner & LD & 307\\
Michael Herd & Ind & 97\\
\end{tabular*}

%\noindent
%\begin{tabular*}{\columnwidth}{@{\extracolsep{\fill}} p{0.545\columnwidth} >{\itshape}l r @{\extracolsep{\fill}}}
\emph{Gardner and Herd eliminated}: Robertson 933 Louden 928 Mackenzie 547
%Fiona Robertson & Ind & 933\\
%Derek Louden & SNP & 928\\
%Ruairidh Mackenzie & Ind & 547\\
%\end{tabular*}

\noindent
\begin{tabular*}{\columnwidth}{@{\extracolsep{\fill}} p{0.545\columnwidth} >{\itshape}l r @{\extracolsep{\fill}}}
\emph{Mackenzie eliminated}\\
Fiona Robertson & Ind & 1204\\
Derek Louden & SNP & 1037\\
\end{tabular*}

\subsubsection*{Inverness South \hspace*{\fill}\nolinebreak[1]%
\enspace\hspace*{\fill}
\finalhyphendemerits=0
[3rd November; LD gain from Lab]}

\index{Inverness South , Highland@Inverness S., \emph{Highland}}

Resignation of John Holden (Lab).

\noindent
\begin{tabular*}{\columnwidth}{@{\extracolsep{\fill}} p{0.545\columnwidth} >{\itshape}l r @{\extracolsep{\fill}}}
\emph{First preferences}\\
Ken Gowans & SNP & 885\\
Carolyn Caddick & LD & 747\\
Katherine MacKenzie & Lab & 308\\
David Bonsor & C & 290\\
Gale Falconer & Grn & 157\\
Donald Boyd & Chr & 126\\
David McGrath & Ind & 94\\
\end{tabular*}

%\noindent
%\begin{tabular*}{\columnwidth}{@{\extracolsep{\fill}} p{0.545\columnwidth} >{\itshape}l r @{\extracolsep{\fill}}}
\emph{McGrath eliminated}: Gowans 903 Caddick 761 MacKenzie 319 Bonsor 300 Falconer 172 Boyd 130
%Ken Gowans & SNP & 903\\
%Carolyn Caddick & LD & 761\\
%Katherine MacKenzie & Lab & 319\\
%David Bonsor & C & 300\\
%Gale Falconer & Grn & 172\\
%Donald Boyd & Chr & 130\\
%\end{tabular*}

%\noindent
%\begin{tabular*}{\columnwidth}{@{\extracolsep{\fill}} p{0.545\columnwidth} >{\itshape}l r @{\extracolsep{\fill}}}
\emph{Boyd eliminated}: Gowans 922 Caddick 777 Bonsor 336 MacKenzie 327 Falconer 189
%Ken Gowans & SNP & 922\\
%Carolyn Caddick & LD & 777\\
%David Bonsor & C & 336\\
%Katherine MacKenzie & Lab & 327\\
%Gale Falconer & Grn & 189\\
%\end{tabular*}

%\noindent
%\begin{tabular*}{\columnwidth}{@{\extracolsep{\fill}} p{0.545\columnwidth} >{\itshape}l r @{\extracolsep{\fill}}}
\emph{Falconer eliminated}: Gowans 967 Caddick 830 MacKenzie 357 Bonsor 339
%Ken Gowans & SNP & 967\\
%Carolyn Caddick & LD & 830\\
%Katherine MacKenzie & Lab & 357\\
%David Bonsor & C & 339\\
%\end{tabular*}

%\noindent
%\begin{tabular*}{\columnwidth}{@{\extracolsep{\fill}} p{0.545\columnwidth} >{\itshape}l r @{\extracolsep{\fill}}}
\emph{Bonsor eliminated}: Gowans 1005 Caddick 971 MacKenzie 379
%Ken Gowans & SNP & 1005\\
%Carolyn Caddick & LD & 971\\
%Katherine MacKenzie & Lab & 379\\
%\end{tabular*}

\noindent
\begin{tabular*}{\columnwidth}{@{\extracolsep{\fill}} p{0.545\columnwidth} >{\itshape}l r @{\extracolsep{\fill}}}
\emph{Mackenzie eliminated}\\
Carolyn Caddick & LD & 1091\\
Ken Gowans & SNP & 1084\\
\end{tabular*}

\section{Island Councils}

\council{Shetland}

\subsubsection*{Shetland Central \hspace*{\fill}\nolinebreak[1]%
\enspace\hspace*{\fill}
\finalhyphendemerits=0
[15th December]}

\index{Shetland Central , Shetland@Shetland C., \emph{Shetland}}

Resignation of Iris Hawkins (Ind).

\noindent
\begin{tabular*}{\columnwidth}{@{\extracolsep{\fill}} p{0.545\columnwidth} >{\itshape}l r @{\extracolsep{\fill}}}
\emph{First preferences}\\
Davie Sandison & Ind & 332\\
Stephen Morgan & Ind & 116\\
Ian Scott & Ind & 107\\
Robert Williamson & Ind & 75\\
Clive Richardson & C & 29\\
Scotty van der Tol & Ind & 27\\
\end{tabular*}

\noindent
\begin{tabular*}{\columnwidth}{@{\extracolsep{\fill}} p{0.545\columnwidth} >{\itshape}l r @{\extracolsep{\fill}}}
\multicolumn{3}{@{\extracolsep{\fill}}l}{\emph{Richardson and van der Tol eliminated}}\\
%\sloppyword{\emph{Exclude Richardson and van der Tol}}\\
Davie Sandison & Ind & 352\\
Stephen Morgan & Ind & 124\\
Ian Scott & Ind & 115\\
Robert Williamson & Ind & 81\\
\end{tabular*}

\columnbreak

\section{Tay Councils}

\subsection*{Angus}
\index{Dundee}

\subsubsection*{Carnoustie and District \hspace*{\fill}\nolinebreak[1]%
\enspace\hspace*{\fill}
\finalhyphendemerits=0
[3rd February; Ind gain from SNP]}

\index{Carnoustie and District , Angus@Carnoustie \& District, \emph{Angus}}

Resignation of Ralph Palmer (SNP).

\noindent
\begin{tabular*}{\columnwidth}{@{\extracolsep{\fill}} p{0.545\columnwidth} >{\itshape}l r @{\extracolsep{\fill}}}
\emph{First preferences}\\
Ed Oswald & SNP & 1289\\
Brian Boyd & Ind & 1252\\
Ron Thoms & Lab & 258\\
Eddie Wilmott & C & 217\\
Charles Goodall & LD & 93\\
\end{tabular*}

%\noindent
%\begin{tabular*}{\columnwidth}{@{\extracolsep{\fill}} p{0.545\columnwidth} >{\itshape}l r @{\extracolsep{\fill}}}
\emph{Goodall eliminated}: Oswald 1302 Boyd 1290 Thoms 266 Wilmott 234
%Ed Oswald & SNP & 1302\\
%Brian Boyd & Ind & 1290\\
%Ron Thoms & Lab & 266\\
%Eddie Wilmott & C & 234\\
%\end{tabular*}

%\noindent
%\begin{tabular*}{\columnwidth}{@{\extracolsep{\fill}} p{0.545\columnwidth} >{\itshape}l r @{\extracolsep{\fill}}}
\emph{Wilmott eliminated}: Boyd 1370 Oswald 1346 Thoms 287
%Brian Boyd & Ind & 1370\\
%Ed Oswald & SNP & 1346\\
%Ron Thoms & Lab & 287\\
%\end{tabular*}

\noindent
\begin{tabular*}{\columnwidth}{@{\extracolsep{\fill}} p{0.545\columnwidth} >{\itshape}l r @{\extracolsep{\fill}}}
\emph{Thoms eliminated}\\
Brian Boyd & Ind & 1454\\
Ed Oswald & SNP & 1426\\
\end{tabular*}

\council{Perth and Kinross}

\subsubsection*{Highland \hspace*{\fill}\nolinebreak[1]%
\enspace\hspace*{\fill}
\finalhyphendemerits=0
[15th September]}

\index{Highland , Perth and Kinross@Highland, \emph{Perth \& Kinross}}

Resignation of Ken Lyall (SNP).

\noindent
\begin{tabular*}{\columnwidth}{@{\extracolsep{\fill}} p{0.56\columnwidth} >{\itshape}l r @{\extracolsep{\fill}}}
Mike Williamson & SNP & 1449\\
Graham Rees & C & 596\\
Victor Clements & LD & 321\\
William Leszke & Ind & 269\\
Chris Rennie & Ind & 27\\
\end{tabular*}

\end{resultsiii}

\part{2012}
\renewcommand\resultsyear{2012}

\chapter{Referendums in 2012}

\section{Salford Mayoral Referendum}\index{Salford!Mayoral Referendum}

A referendum was held in Salford on 26 January 2012 on the question of whether the city should have a directly elected mayor.

\noindent
\begin{tabular*}{\columnwidth}{@{\extracolsep{\fill}} p{0.545\columnwidth} >{\itshape}l r @{\extracolsep{\fill}}}
& Yes & 17344\\
& No & 13653\\
\end{tabular*}

\section{May 2012 Mayoral Referendums}

\subsection*{Core Cities: establishment of Mayor}

Referendums were held in ten English cities on 3 May 2012 on the question of whether the city should have a directly elected mayor.

\begin{results}

\subsubsection*{Birmingham}\index{Birmingham!Mayoral Referendum}

\noindent
\begin{tabular*}{\columnwidth}{@{\extracolsep{\fill}} p{0.545\columnwidth} >{\itshape}l r @{\extracolsep{\fill}}}
& Yes & 88085\\
& No & 120611\\
\end{tabular*}

\subsubsection*{Bradford}\index{Bradford!Mayoral Referendum}

\noindent
\begin{tabular*}{\columnwidth}{@{\extracolsep{\fill}} p{0.545\columnwidth} >{\itshape}l r @{\extracolsep{\fill}}}
& Yes & 53949\\
& No & 66283\\
\end{tabular*}

\subsubsection*{Bristol}\index{Bristol!Mayoral Referendum}

\noindent
\begin{tabular*}{\columnwidth}{@{\extracolsep{\fill}} p{0.545\columnwidth} >{\itshape}l r @{\extracolsep{\fill}}}
& Yes & 41032\\
& No & 35880\\
\end{tabular*}

\subsubsection*{Coventry}\index{Coventry!Mayoral Referendum}

\noindent
\begin{tabular*}{\columnwidth}{@{\extracolsep{\fill}} p{0.545\columnwidth} >{\itshape}l r @{\extracolsep{\fill}}}
& Yes & 22619\\
& No & 39483\\
\end{tabular*}

\subsubsection*{Leeds}\index{Leeds!Mayoral Referendum}

\noindent
\begin{tabular*}{\columnwidth}{@{\extracolsep{\fill}} p{0.545\columnwidth} >{\itshape}l r @{\extracolsep{\fill}}}
& Yes & 62440\\
& No & 107910\\
\end{tabular*}

\subsubsection*{Manchester}\index{Manchester!Mayoral Referendum}

\noindent
\begin{tabular*}{\columnwidth}{@{\extracolsep{\fill}} p{0.545\columnwidth} >{\itshape}l r @{\extracolsep{\fill}}}
& Yes & 42677\\
& No & 48593\\
\end{tabular*}

\subsubsection*{Newcastle upon Tyne}\index{Newcastle upon Tyne!Mayoral Referendum}

\noindent
\begin{tabular*}{\columnwidth}{@{\extracolsep{\fill}} p{0.545\columnwidth} >{\itshape}l r @{\extracolsep{\fill}}}
& Yes & 24630\\
& No & 40089\\
\end{tabular*}

\subsubsection*{Nottingham}\index{Nottingham!Mayoral Referendum}

\noindent
\begin{tabular*}{\columnwidth}{@{\extracolsep{\fill}} p{0.545\columnwidth} >{\itshape}l r @{\extracolsep{\fill}}}
& Yes & 20943\\
& No & 28320\\
\end{tabular*}

\subsubsection*{Sheffield}\index{Sheffield!Mayoral Referendum}

\noindent
\begin{tabular*}{\columnwidth}{@{\extracolsep{\fill}} p{0.545\columnwidth} >{\itshape}l r @{\extracolsep{\fill}}}
& Yes & 44571\\
& No & 82890\\
\end{tabular*}

\subsubsection*{Wakefield}\index{Wakefield!Mayoral Referendum}

\noindent
\begin{tabular*}{\columnwidth}{@{\extracolsep{\fill}} p{0.545\columnwidth} >{\itshape}l r @{\extracolsep{\fill}}}
& Yes & 27610\\
& No & 45357\\
\end{tabular*}

\end{results}

Similar referendums were legislated for in Leicester and Liverpool but were cancelled when the respective city councils voted to introduce the mayoral system without a referendum.

\section{Doncaster: abolition of Mayor}

A referendum was held in Doncaster 3 May 2012 on the question of whether the borough's directly elected mayoralty should be abolished.

\noindent
\begin{tabular*}{\columnwidth}{@{\extracolsep{\fill}} p{0.545\columnwidth} >{\itshape}l r @{\extracolsep{\fill}}}
& Yes & 25879\\
& No & 42196\\
\end{tabular*}

\section{Hartlepool Mayoral Abolition Referendum}

A referendum was held in Hartlepool on 15 November 2012 on the question of whether the borough's directly elected mayoralty should be abolished.

\noindent
\begin{tabular*}{\columnwidth}{@{\extracolsep{\fill}} p{0.545\columnwidth} >{\itshape}l r @{\extracolsep{\fill}}}
& Yes & 7366\\
& No & 5177\\
\end{tabular*}

%\part{By-elections}

\chapter{Parliamentary by-elections}

There were seven parliamentary by-elections in 2012.

9/11 = ``nine eleven was an inside job''

Cannabis = Cannabis Law Reform

CommLg = Communist League

Dem2015 = Democracy 2015

DemNat = Democratic Nationalists

Elvis = Church of the Militant Elvis

PDP = People's Democratic Party

Peace = Peace Party

Pirate = Pirate Party of the United Kingdom

UPP = United People's Party

YPP = Young People's Party

\vfill

\section*{Bradford West \hspace*{\fill}\nolinebreak[1]%
\enspace\hspace*{\fill}
\finalhyphendemerits=0
[29th March; Respect gain from Lab]}

\index{Bradford West , House of Commons@Bradford W., \emph{House of Commons}}

Resignation of Marsha Singh (Lab).

\noindent
\begin{tabular*}{\columnwidth}{@{\extracolsep{\fill}} p{0.545\columnwidth} >{\itshape}l r @{\extracolsep{\fill}}}
George Galloway & Respect & 18341\\
Imran Hussain & Lab & 8201\\
Jackie Whiteley & C & 2746\\
Jeanette Sunderland & LD & 1505\\
Sonja McNally & UKIP & 1085\\
Dawud Islam & Grn & 481\\
Neil Craig & DemNat & 344\\
Howling Laud Hope & Loony & 111\\
\end{tabular*}

\vfill

\section*{Cardiff South and Penarth \hspace*{\fill}\nolinebreak[1]%
\enspace\hspace*{\fill}
\finalhyphendemerits=0
[15th November]}

\index{Cardiff South and Penarth , House of Commons@Cardiff S. \& Penarth, \emph{House of Commons}}

Resignation of Alun Michael (Lab).

\noindent
\begin{tabular*}{\columnwidth}{@{\extracolsep{\fill}} p{0.545\columnwidth} >{\itshape}l r @{\extracolsep{\fill}}}
Stephen Doughty & Lab & 9193\\
Craig Williams & C & 3859\\
Bablin Molik & LD & 2103\\
Luke Nicholas & PC & 1854\\
Simon Zeigler & UKIP & 1179\\
Anthony Slaughter & Grn & 800\\
Andrew Jordan & SocLab & 235\\
Robert Griffiths & Comm & 213\\
\end{tabular*}

\eject

\section*{Corby \hspace*{\fill}\nolinebreak[1]%
\enspace\hspace*{\fill}
\finalhyphendemerits=0
[15th November; Lab gain from C]}

\index{Corby , House of Commons@Corby, \emph{House of Commons}}

Resignation of Louise Mensch (C, elected as Louise Bagshawe).

\noindent
\begin{tabular*}{\columnwidth}{@{\extracolsep{\fill}} p{0.545\columnwidth} >{\itshape}l r @{\extracolsep{\fill}}}
Andy Sawford & Lab & 17267\\
Christine Emmett & C & 9476\\
Margot Parker & UKIP & 5108\\
Jill Hope & LD & 1770\\
Gordon Riddell & BNP & 614\\
David Wickham & EDP & 432\\
Jonathan Hornett & Grn & 378\\
Ian Gillman & Ind & 212\\
Peter Reynolds & Cannabis & 137\\
David Bishop & Elvis & 99\\
Mr Mozzarella & Ind & 73\\
Rohen Kapur & YPP & 39\\
Adam Lotun & Dem2015 & 35\\
Christopher Scotton & UPP & 25\\
\end{tabular*}

\vfill

\section*{Manchester Central \hspace*{\fill}\nolinebreak[1]%
\enspace\hspace*{\fill}
\finalhyphendemerits=0
[15th November]}

\index{Manchester Central , House of Commons@Manchester C., \emph{House of Commons}}

Resignation of Tony Lloyd (Lab).

\noindent
\begin{tabular*}{\columnwidth}{@{\extracolsep{\fill}} p{0.545\columnwidth} >{\itshape}l r @{\extracolsep{\fill}}}
Lucy Powell & Lab & 11507\\
Marc Ramsbottom & LD & 1571\\
Matthew Sephton & C & 754\\
Chris Cassidy & UKIP & 749\\
Tom Dylan & Grn & 652\\
Eddy O'Sullivan & BNP & 492\\
Loz Kaye & Pirate & 308\\
Alex Davidson & TUSC & 220\\
Catherine Higgins & Respect & 182\\
Howling Laud Hope & Loony & 78\\
Lee Holmes & PDP & 71\\
Peter Clifford & CommLg & 64\\
\end{tabular*}

\vfill

\section*{Croydon North \hspace*{\fill}\nolinebreak[1]%
\enspace\hspace*{\fill}
\finalhyphendemerits=0
[29th November]}

\index{Croydon North , House of Commons@Croydon N., \emph{House of Commons}}

Death of Malcolm Wicks (Lab).

\noindent
\begin{tabular*}{\columnwidth}{@{\extracolsep{\fill}} p{0.545\columnwidth} >{\itshape}l r @{\extracolsep{\fill}}}
Steve Reed & Lab & 15892\\
Andrew Stranack & C & 4137\\
Winston McKenzie & UKIP & 1400\\
Marisha Ray & LD & 860\\
Shasha Khan & Grn & 855\\
Lee Jasper & Respect & 707\\
Stephen Hammond & CPA & 192\\
Richard Edmonds & NF & 161\\
Ben Stevenson & Comm & 119\\
John Cartwright & Loony & 110\\
Simon Lane & 9/11 & 66\\
Robin Smith & YPP & 63\\
\end{tabular*}

\eject

\section*{Middlesbrough \hspace*{\fill}\nolinebreak[1]%
\enspace\hspace*{\fill}
\finalhyphendemerits=0
[29th November]}

\index{Middlesbrough , House of Commons@Middlesbrough, \emph{House of Commons}}

Death of Sir Stuart Bell (Lab).

\noindent
\begin{tabular*}{\columnwidth}{@{\extracolsep{\fill}} p{0.545\columnwidth} >{\itshape}l r @{\extracolsep{\fill}}}
Andy McDonald & Lab & 10201\\
Richard Elvin & UKIP & 1990\\
George Selmer & LD & 1672\\
Ben Houchen & C & 1063\\
Imdad Hussain & Peace & 1060\\
Peter Foreman & BNP & 328\\
John Malcolm & TUSC & 277\\
Mark Heslehurst & Ind & 275\\
\end{tabular*}

\section*{Rotherham \hspace*{\fill}\nolinebreak[1]%
\enspace\hspace*{\fill}
\finalhyphendemerits=0
[29th November]}

\index{Rotherham , House of Commons@Rotherham, \emph{House of Commons}}

Resignation of Denis MacShane (Lab).

\noindent
\begin{tabular*}{\columnwidth}{@{\extracolsep{\fill}} p{0.545\columnwidth} >{\itshape}l r @{\extracolsep{\fill}}}
Sarah Champion & Lab & 9966\\
Jane Collins & UKIP & 4648\\
Marlene Guest & BNP & 1804\\
Yvonne Ridley & Respect & 1778\\
Simon Wilson & C & 1157\\
David Wildgoose & EDP & 703\\
Simon Copley & Ind & 582\\
Michael Beckett & LD & 451\\
Ralph Dyson & TUSC & 281\\
Paul Dickson & Ind & 51\\
Clint Bristow & Ind & 29\\
\end{tabular*}

\chapter{By-elections to devolved assemblies, the European Parliament, and police and crime commissionerships}

\section{Greater London Authority}

There were no by-elections in 2012 to the Greater London Authority.

\section{National Assembly for Wales}

There were no by-elections in 2012 to the National Assembly for Wales.

\section{Scottish Parliament}

There were no by-elections in 2012 to the Scottish Parliament.

John Park (Lab, Mid Scotland and Fife list) resigned on 7th December and was replaced by Jayne Baxter.

\section{Northern Ireland Assembly}

Vacancies in the Northern Ireland Assembly are filled by co-option.  %No co-options were made in 2012.
%
The following members were co-opted to the Assembly in 2012:
\begin{itemize}
\item Seán Rogers (SDLP) replaced Margaret Ritchie following her resignation on 31st March (South Down).
\item Chris Hazzard (SDLP) replaced Willie Clarke following his resignation on 13th April (South Down).
\item Maeve McLaughlin (SF) replaced Martina Anderson following her resignation on 11th June (Foyle).
\item Megan Fearon (SF) replaced Conor Murphy following his resignation on 2nd July (Newry and Armagh).
\item Declan McAleer (SF) replaced Pat Doherty following his resignation on 2nd July (West Tyrone).
\item Rosie McCorley (SF) replaced Paul Maskey following his resignation on 2nd July (Belfast West).
\item Bronwyn McGahan (SF) replaced Michelle Gildernew following her resignation on 7th July (Fermanagh and South Tyrone).
\end{itemize}

\section{European Parliament}

UK vacancies in the European Parliament are filled by the next available person from the party list at the most recent election (which was held in 2009).
%No replacements were made in 2012.
The following replacements were made in 2012:
\begin{itemize}
\item Rebecca Taylor (LD) replaced Diana Wallis following her resignation on 31st January (Yorkshire and the Humber).
\item Phil Bennion (LD) replaced Liz Lynne following her resignation on 3rd February (West Midlands).
\item Martina Anderson (SF) was co-opted to replace Bairbre de Brún following her resignation on 3rd May (Northern Ireland).
\end{itemize}

\section{Police and crime commissioners}

There were no by-elections in 2012 for vacant police and crime commissioner posts.

\chapter{Local by-elections and unfilled vacancies}

\begin{resultsiii}

\section{North London}

\council{City of London}

\subsubsection*{Aldgate \hspace*{\fill}\nolinebreak[1]%
\enspace\hspace*{\fill}
\finalhyphendemerits=0
[18th April]}

\index{Aldgate , City of London@Aldgate, \emph{City of London}}

Aldermanic election: retirement of Lord Levene of Portsoken (Ind).

\noindent
\begin{tabular*}{\columnwidth}{@{\extracolsep{\fill}} p{0.545\columnwidth} >{\itshape}l r @{\extracolsep{\fill}}}
Peter Hewitt & Ind & \emph{unop.}\\
\end{tabular*}


\subsubsection*{Farringdon Within \hspace*{\fill}\nolinebreak[1]%
\enspace\hspace*{\fill}
\finalhyphendemerits=0
[26th July]}

\index{Farringdon Within , City of London@Farringdon Wn., \emph{City of London}}

Resignation of Robert Hughes-Penny (Ind).

\noindent
\begin{tabular*}{\columnwidth}{@{\extracolsep{\fill}} p{0.545\columnwidth} >{\itshape}l r @{\extracolsep{\fill}}}
Mark Clarke & Ind & 102\\
Trevor Brignall & Ind & 32\\
Bryn Phillips & Ind & 23\\
Spencer Marshall & Ind & 15\\
\end{tabular*}

\subsubsection*{Candlewick \hspace*{\fill}\nolinebreak[1]%
\enspace\hspace*{\fill}
\finalhyphendemerits=0
[3rd October]}

\index{Candlewick , City of London@Candlewick, \emph{City of London}}

Aldermanic election: Fiona Woolf (Ind) sought re-election.

\noindent
\begin{tabular*}{\columnwidth}{@{\extracolsep{\fill}} p{0.545\columnwidth} >{\itshape}l r @{\extracolsep{\fill}}}
Fiona Woolf & Ind & \emph{unop.}\\
\end{tabular*}

\subsubsection*{Bridge and Bridge Without \hspace*{\fill}\nolinebreak[1]%
\enspace\hspace*{\fill}
\finalhyphendemerits=0
[8th October]}

\index{Bridge and Bridge Without , City of London@Bridge \& Bridge Wt., \emph{City of London}}

Aldermanic election: Alan Yarrow (Ind) sought re-election.

\noindent
\begin{tabular*}{\columnwidth}{@{\extracolsep{\fill}} p{0.545\columnwidth} >{\itshape}l r @{\extracolsep{\fill}}}
Alan Yarrow & Ind & \emph{unop.}\\
\end{tabular*}

\subsubsection*{Billingsgate \hspace*{\fill}\nolinebreak[1]%
\enspace\hspace*{\fill}
\finalhyphendemerits=0
[30th November]}

\index{Billingsgate , City of London@Billingsgate, \emph{City of London}}

Aldermanic election: John White (Ind) retired.

\noindent
\begin{tabular*}{\columnwidth}{@{\extracolsep{\fill}} p{0.545\columnwidth} >{\itshape}l r @{\extracolsep{\fill}}}
Matthew Richardson & Ind & \emph{unop.}\\
\end{tabular*}

\subsubsection*{Tower \hspace*{\fill}\nolinebreak[1]%
\enspace\hspace*{\fill}
\finalhyphendemerits=0
[3rd December]}

\index{Tower , City of London@Tower, \emph{City of London}}

Aldermanic election: Sir Paul Judge (Ind) sought re-election.

\noindent
\begin{tabular*}{\columnwidth}{@{\extracolsep{\fill}} p{0.545\columnwidth} >{\itshape}l r @{\extracolsep{\fill}}}
Sir Paul Judge & Ind & \emph{unop.}\\
\end{tabular*}

\council{Barking and Dagenham}

\subsubsection*{Goresbrook \hspace*{\fill}\nolinebreak[1]%
\enspace\hspace*{\fill}
\finalhyphendemerits=0
[19th April]}

\index{Goresbrook , Barking and Dagenham@Goresbrook, \emph{Barking \& Dagenham}}

Resignation of Louise Couling (Lab).

\noindent
\begin{tabular*}{\columnwidth}{@{\extracolsep{\fill}} p{0.545\columnwidth} >{\itshape}l r @{\extracolsep{\fill}}}
Simon Bremner & Lab & 1113\\
Bob Taylor & BNP & 593\\
John Dias-Broughton & UKIP & 91\\
Mohammed Riaz & C & 81\\
Robert Hills & LD & 48\\
\end{tabular*}



\council{Barnet}

\subsubsection*{East Finchley \hspace*{\fill}\nolinebreak[1]%
\enspace\hspace*{\fill}
\finalhyphendemerits=0
[Wednesday 11th April]}

\index{East Finchley , Barnet@East Finchley, \emph{Barnet}}

Resignation of Andrew McNeil (Lab).

\noindent
\begin{tabular*}{\columnwidth}{@{\extracolsep{\fill}} p{0.545\columnwidth} >{\itshape}l r @{\extracolsep{\fill}}}
Arjun Mittra & Lab & 2117\\
Anshul Gupta & C & 543\\
Jane Gibson & LD & 461\\
\end{tabular*}

\subsubsection*{Brunswick Park \hspace*{\fill}\nolinebreak[1]%
\enspace\hspace*{\fill}
\finalhyphendemerits=0
[31st May; Lab gain from C]}

\index{Brunswick Park , Barnet@Brunswick Park, \emph{Barnet}}

Death of Lynne Hillan (C).

\noindent
\begin{tabular*}{\columnwidth}{@{\extracolsep{\fill}} p{0.545\columnwidth} >{\itshape}l r @{\extracolsep{\fill}}}
Andreas Ioannidis & Lab & 1769\\
Shaheen Mahmood & C & 1598\\
Yahaya Kiingi & LD & 97\\
\end{tabular*}

\council{Brent}

\subsubsection*{Dollis Hill \hspace*{\fill}\nolinebreak[1]%
\enspace\hspace*{\fill}
\finalhyphendemerits=0
[22nd March]}

\index{Dollis Hill , Brent@Dollis Hill, \emph{Brent}}

Death of Alec Castle (LD).

\noindent
\begin{tabular*}{\columnwidth}{@{\extracolsep{\fill}} p{0.545\columnwidth} >{\itshape}l r @{\extracolsep{\fill}}}
Alison Hopkins & LD & 1205\\
Parvez Ahmed & Lab & 1168\\
Samer Ahmedali & C & 140\\
Pete Murry & Grn & 79\\
\end{tabular*}

\subsubsection*{Barnhill \hspace*{\fill}\nolinebreak[1]%
\enspace\hspace*{\fill}
\finalhyphendemerits=0
[3rd May]}

\index{Barnhill , Brent@Barnhill, \emph{Brent}}

Resignation of Judith Beckman (Lab).

\noindent
\begin{tabular*}{\columnwidth}{@{\extracolsep{\fill}} p{0.545\columnwidth} >{\itshape}l r @{\extracolsep{\fill}}}
Michael Pavey & Lab & 2366\\
Kanta Pindoria & C & 1180\\
Martin Francis & Grn & 457\\
Venilal Vaghela & Ind & 156\\
\end{tabular*}



\council{Camden}

\subsubsection*{Camden Town with Primrose Hill \hspace*{\fill}\nolinebreak[1]%
\enspace\hspace*{\fill}
\finalhyphendemerits=0
[3rd May]}

\index{Camden Town with Primrose Hill , Camden@Camden Town with Primrose Hill, \emph{Camden}}

Resignation of Thomas Neumark (Lab).

\noindent
\begin{tabular*}{\columnwidth}{@{\extracolsep{\fill}} p{0.545\columnwidth} >{\itshape}l r @{\extracolsep{\fill}}}
Lazzaro Pietragnoli & Lab & 1847\\
Nigel Rumble & C & 823\\
Chris Richards & LD & 748\\
Peter Lyons & Grn & 450\\
Joe Gardner & Ind & 98\\
\end{tabular*}

\subsubsection*{Hampstead Town \hspace*{\fill}\nolinebreak[1]%
\enspace\hspace*{\fill}
\finalhyphendemerits=0
[27th September]}

\index{Hampstead Town , Camden@Hampstead Town, \emph{Camden}}

Resignation of Kirsty Roberts (C).

\noindent
\begin{tabular*}{\columnwidth}{@{\extracolsep{\fill}} p{0.545\columnwidth} >{\itshape}l r @{\extracolsep{\fill}}}
Simon Marcus & C & 1040\\
Jeffrey Fine & LD & 695\\
Maddy Raman & Lab & 512\\
Sophie Dix & Grn & 207\\
\end{tabular*}

\council{Hackney}

\subsubsection*{Hackney Central \hspace*{\fill}\nolinebreak[1]%
\enspace\hspace*{\fill}
\finalhyphendemerits=0
[3rd May]}

\index{Hackney Central , Hackney@Hackney C., \emph{Hackney}}

Resignation of Alan Laing (Lab).

\noindent
\begin{tabular*}{\columnwidth}{@{\extracolsep{\fill}} p{0.545\columnwidth} >{\itshape}l r @{\extracolsep{\fill}}}
Ben Hayhurst & Lab & 2438\\
Mustafa Korel & Grn & 545\\
Pauline Pearce & LD & 398\\
Andrew Boff & C & 196\\
\end{tabular*}



\council{Hammersmith and Fulham}

\subsubsection*{Town \hspace*{\fill}\nolinebreak[1]%
\enspace\hspace*{\fill}
\finalhyphendemerits=0
[12th July]}

\index{Town , Hammersmith and Fulham@Town, \emph{Hammersmith \& Fulham}}

Resignation of Stephen Greenhalgh (C).

\noindent
\begin{tabular*}{\columnwidth}{@{\extracolsep{\fill}} p{0.545\columnwidth} >{\itshape}l r @{\extracolsep{\fill}}}
Andrew Brown & C & 768\\
Ben Coleman & Lab & 416\\
Paul Kennedy & LD & 331\\
Andrew Elston & UKIP & 39\\
\end{tabular*}

\columnbreak

\council{Islington}

\subsubsection*{Holloway \hspace*{\fill}\nolinebreak[1]%
\enspace\hspace*{\fill}
\finalhyphendemerits=0
[3rd May]}

\index{Holloway , Islington@Holloway, \emph{Islington}}

Resignation of Lucy Rigby (Lab).

\noindent
\begin{tabular*}{\columnwidth}{@{\extracolsep{\fill}} p{0.545\columnwidth} >{\itshape}l r @{\extracolsep{\fill}}}
Rakhia Ismail & Lab & 2352\\
Jonathan Edwards & C & 671\\
Claire Poyner & Grn & 613\\
David Kelly & LD & 490\\
\end{tabular*}



\council{Kensington and Chelsea}

\subsubsection*{Brompton \hspace*{\fill}\nolinebreak[1]%
\enspace\hspace*{\fill}
\finalhyphendemerits=0
[28th June]}

\index{Brompton , Kensington and Chelsea@Brompton, \emph{Kensington \& Chelsea}}

Death of Baroness Ritchie of Brompton (C).

\noindent
\begin{tabular*}{\columnwidth}{@{\extracolsep{\fill}} p{0.545\columnwidth} >{\itshape}l r @{\extracolsep{\fill}}}
Abbas Barkhordar & C & 650\\
Mark Sautter & Lab & 103\\
Moya Denman & LD & 101\\
David Coburn & UKIP & 71\\
\end{tabular*}

\council{Tower Hamlets}

\subsubsection*{Spitalfields and Banglatown \hspace*{\fill}\nolinebreak[1]%
\enspace\hspace*{\fill}
\finalhyphendemerits=0
[19th April; Ind gain from Lab]}

\index{Spitalfields and Banglatown , Tower Hamlets@Spitalfields \& Banglatown, \emph{Tower Hamlets}}

Disqualification (sentenced to sixteen weeks in prison, benefit fraud) of Shelina Akhtar (Ind elected as Lab).

\noindent
\begin{tabular*}{\columnwidth}{@{\extracolsep{\fill}} p{0.545\columnwidth} >{\itshape}l r @{\extracolsep{\fill}}}
Gulam Robbani & Ind & 1030\\
Ala Uddin & Lab & 987\\
Matthew Smith & C & 140\\
Kirsty Blake & Grn & 99\\
Richard Macmillan & LD & 39\\
\end{tabular*}

\subsubsection*{Weavers \hspace*{\fill}\nolinebreak[1]%
\enspace\hspace*{\fill}
\finalhyphendemerits=0
[3rd May]}

\index{Weavers , Tower Hamlets@Weavers, \emph{Tower Hamlets}}

Resignation of Anna Lynch (Lab).

\noindent
\begin{tabular*}{\columnwidth}{@{\extracolsep{\fill}} p{0.52\columnwidth} >{\itshape}l r @{\extracolsep{\fill}}}
John Pierce & Lab & 1544\\
Abjol Miah & Respect & 1260\\
Caroline Kerswell & C & 415\\
Alan Duffell & Grn & 373\\
Azizur Khan & LD & 208\\
Oli Rothschild & Ind & 36\\
\end{tabular*}

\council{Waltham Forest}

\subsubsection*{Larkswood \hspace*{\fill}\nolinebreak[1]%
\enspace\hspace*{\fill}
\finalhyphendemerits=0
[12th July]}

\index{Larkswood , Waltham Forest@Larkswood, \emph{Waltham Forest}}

Resignation of Ed Northover (C).

\noindent
\begin{tabular*}{\columnwidth}{@{\extracolsep{\fill}} p{0.545\columnwidth} >{\itshape}l r @{\extracolsep{\fill}}}
John Moss & C & 1392\\
Peter Woodrow & Lab & 472\\
Graham Woolnough & LD & 79\\
Bill Measure & Grn & 70\\
James O'Rourke & Lib & 64\\
\end{tabular*}

\council{Westminster}

\subsubsection*{Hyde Park \hspace*{\fill}\nolinebreak[1]%
\enspace\hspace*{\fill}
\finalhyphendemerits=0
[3rd May]}

\index{Hyde Park , Westminster@Hyde Park, \emph{Westminster}}

Resignation of Colin Barrow (C).

\noindent
\begin{tabular*}{\columnwidth}{@{\extracolsep{\fill}} p{0.545\columnwidth} >{\itshape}l r @{\extracolsep{\fill}}}
Antonia Cox & C & 1448\\
Jack Gordon & Lab & 563\\
Mark Cridge & Grn & 182\\
Martin Thompson & LD & 178\\
Earl of Bradford & UKIP & 96\\
Abdulla Dharamsey & Ind & 40\\
\end{tabular*}



\section{South London}

\council{Bromley}

\subsubsection*{Bromley Town \hspace*{\fill}\nolinebreak[1]%
\enspace\hspace*{\fill}
\finalhyphendemerits=0
[3rd May]}

\index{Bromley Town , Bromley@Bromley Town, \emph{Bromley}}

Resignation of Diana MacMull (C).

\noindent
\begin{tabular*}{\columnwidth}{@{\extracolsep{\fill}} p{0.545\columnwidth} >{\itshape}l r @{\extracolsep{\fill}}}
Nicky Dykes & C & 2484\\
Sam Webber & LD & 1137\\
Angela Stack & Lab & 1051\\
Owen Brolly & UKIP & 397\\
Ann Garrett & Grn & 404\\
\end{tabular*}

\council{Kingston upon Thames}

\subsubsection*{Coombe Hill \hspace*{\fill}\nolinebreak[1]%
\enspace\hspace*{\fill}
\finalhyphendemerits=0
[3rd May]}

\index{Coombe Hill , Kingston upon Thames@\sloppyword{Coombe Hill, \emph{Kingston upon Thames}}}

Resignation of David Edwards (C).

\noindent
\begin{tabular*}{\columnwidth}{@{\extracolsep{\fill}} p{0.545\columnwidth} >{\itshape}l r @{\extracolsep{\fill}}}
Gaj Wallooppillai & C & 1601\\
Laurie South & Lab & 519\\
David Knowles & LD & 409\\
Jean Vidler & Grn & 235\\
Michael Watson & UKIP & 148\\
Rajesh Dewan & CPA & 66\\
\end{tabular*}

\subsubsection*{Grove \hspace*{\fill}\nolinebreak[1]%
\enspace\hspace*{\fill}
\finalhyphendemerits=0
[5th July]}

\index{Grove , Kingston upon Thames@Grove, \emph{Kingston upon Thames}}

Resignation of Marc Woodall (LD).

\noindent
\begin{tabular*}{\columnwidth}{@{\extracolsep{\fill}} p{0.545\columnwidth} >{\itshape}l r @{\extracolsep{\fill}}}
Rebekah Moll & LD & 710\\
Adrian Amer & C & 687\\
Laurie South & Lab & 440\\
Ryan Coley & Grn & 123\\
Michael Watson & UKIP & 56\\
David Child & BNP & 23\\
Jonathan Rudd & CPA & 20\\
\end{tabular*}



\council{Lewisham}

LPBP = Lewisham People Before Profit

\subsubsection*{Whitefoot \hspace*{\fill}\nolinebreak[1]%
\enspace\hspace*{\fill}
\finalhyphendemerits=0
[11th October; Lab gain from LD]}

\index{Whitefoot , Lewisham@Whitefoot, \emph{Lewisham}}

Resignation of Pete Pattisson (LD).

\noindent
\begin{tabular*}{\columnwidth}{@{\extracolsep{\fill}} p{0.545\columnwidth} >{\itshape}l r @{\extracolsep{\fill}}}
Mark Ingleby & Lab & 924\\
Janet Hurst & LD & 646\\
Simon Nundy & C & 258\\
John Hamilton & LPBP & 241\\
Paul Oakley & UKIP & 100\\
Ute Michel & Grn & 36\\
\end{tabular*}



\council{Merton}

\subsubsection*{Wimbledon Park \hspace*{\fill}\nolinebreak[1]%
\enspace\hspace*{\fill}
\finalhyphendemerits=0
[3rd May]}

\index{Wimbledon Park , Merton@Wimbledon Park, \emph{Merton}}

Resignation of Lord Ahmad of Wimbledon (C).

\noindent
\begin{tabular*}{\columnwidth}{@{\extracolsep{\fill}} p{0.545\columnwidth} >{\itshape}l r @{\extracolsep{\fill}}}
Linda Taylor & C & 1837\\
Louise Deegan & Lab & 931\\
Dave Busby & LD & 838\\
\sloppyword{Richmond Crowhurst} & Grn & 253\\
\end{tabular*}



\council{Richmond upon Thames}

\subsubsection*{North Richmond \hspace*{\fill}\nolinebreak[1]%
\enspace\hspace*{\fill}
\finalhyphendemerits=0
[3rd May]}

\index{North Richmond , Richmond upon Thame@\sloppyword{North Richmond, \emph{Richmond upon Thames}}}

Resignation of Richard Montague (C).

\noindent
\begin{tabular*}{\columnwidth}{@{\extracolsep{\fill}} p{0.545\columnwidth} >{\itshape}l r @{\extracolsep{\fill}}}
Stephen Speak & C & 1733\\
Jane Dodds & LD & 1587\\
Brian Caton & Lab & 364\\
James Page & Grn & 206\\
\sloppyword{Marc Cranfield-Adams} & Ind & 123\\
\end{tabular*}

\council{Southwark}

\subsubsection*{East Walworth \hspace*{\fill}\nolinebreak[1]%
\enspace\hspace*{\fill}
\finalhyphendemerits=0
[29th November; LD gain from Lab]}

\index{East Walworth , Southwark@East Walworth, \emph{Southwark}}

Death of Helen Morrissey (Lab).

\noindent
\begin{tabular*}{\columnwidth}{@{\extracolsep{\fill}} p{0.545\columnwidth} >{\itshape}l r @{\extracolsep{\fill}}}
Ben Johnson & LD & 1250\\
Rebecca Lury & Lab & 1003\\
Stuart Millson & C & 94\\
\end{tabular*}

\council{Sutton}

\subsubsection*{Worcester Park \hspace*{\fill}\nolinebreak[1]%
\enspace\hspace*{\fill}
\finalhyphendemerits=0
[16th February]}

\index{Worcester Park , Sutton@Worcester Park, \emph{Sutton}}

Resignation of Jennifer Campbell-Klomps (LD).

\noindent
\begin{tabular*}{\columnwidth}{@{\extracolsep{\fill}} p{0.545\columnwidth} >{\itshape}l r @{\extracolsep{\fill}}}
Roger Roberts & LD & 1367\\
Simon Densley & C & 977\\
Hilary Hosking & Lab & 315\\
David Pickles & UKIP & 190\\
George Dow & Grn & 46\\
\end{tabular*}

\subsubsection*{Stonecot \hspace*{\fill}\nolinebreak[1]%
\enspace\hspace*{\fill}
\finalhyphendemerits=0
[6th December]}

\index{Stonecot , Sutton@Stonecot, \emph{Sutton}}

Resignation of Brendan Hudson (LD).

\noindent
\begin{tabular*}{\columnwidth}{@{\extracolsep{\fill}} p{0.545\columnwidth} >{\itshape}l r @{\extracolsep{\fill}}}
Nick Emmerson & LD & 1034\\
Graham Jarvis & C & 402\\
Bonnie Craven & Lab & 289\\
Jeremy Wraith & UKIP & 182\\
Joan Hartfield & Grn & 32\\
\end{tabular*}

\council{Wandsworth}

\subsubsection*{Southfields \hspace*{\fill}\nolinebreak[1]%
\enspace\hspace*{\fill}
\finalhyphendemerits=0
[29th March]}

\index{Southfields , Wandsworth@Southfields, \emph{Wandsworth}}

Resignation of Lucy Allan (C).

\noindent
\begin{tabular*}{\columnwidth}{@{\extracolsep{\fill}} p{0.545\columnwidth} >{\itshape}l r @{\extracolsep{\fill}}}
Kim Caddy & C & 1841\\
Josh Kaile & Lab & 1511\\
John Munro & LD & 220\\
Bruce Mackenzie & Grn & 100\\
Strachan McDonald & UKIP & 40\\
Choudry Abid & Ind & 38\\
\end{tabular*}



\section{Greater Manchester}

\council{Bolton}

\subsubsection*{Bradshaw (2) \hspace*{\fill}\nolinebreak[1]%
\enspace\hspace*{\fill}
\finalhyphendemerits=0
[3rd May]}

\index{Bradshaw , Bolton@Bradshaw, \emph{Bolton}}

Resignations of Diana and Paul Brierley (both C).

Combined with the 2012 ordinary election.
%; see page \pageref{BradshawBolton} for the result.

\council{Bury}

At the May 2012 ordinary election there was an unfilled vacancy in East ward due to the death of John Byrne (Lab).
\index{East , Bury@East, \emph{Bury}}

\subsubsection*{Church \hspace*{\fill}\nolinebreak[1]%
\enspace\hspace*{\fill}
\finalhyphendemerits=0
[15th November]}

\index{Church , Bury@Church, \emph{Bury}}

Resignation of Bob Bibby (C).

\noindent
\begin{tabular*}{\columnwidth}{@{\extracolsep{\fill}} p{0.545\columnwidth} >{\itshape}l r @{\extracolsep{\fill}}}
Susan Nuttall & C & 1371\\
Sarah Kerrison & Lab & 1108\\
Stephen Evans & UKIP & 309\\
Kamran Islam & LD & 35\\
\end{tabular*}

\subsubsection*{North Manor \hspace*{\fill}\nolinebreak[1]%
\enspace\hspace*{\fill}
\finalhyphendemerits=0
[15th November]}

\index{North Manor , Bury@North Manor, \emph{Bury}}

Resignation of Jim Taylor (C).

\noindent
\begin{tabular*}{\columnwidth}{@{\extracolsep{\fill}} p{0.545\columnwidth} >{\itshape}l r @{\extracolsep{\fill}}}
James Daly & C & 1324\\
Jean Treadgold & Lab & 643\\
Peter Entwistle & UKIP & 251\\
Stewart Hay & Grn & 126\\
Ewan Arthur & LD & 93\\
\end{tabular*}



\council{Manchester}

At the May 2012 ordinary election there was an unfilled vacancy in Baguley ward due to the death of Tony Burns (Lab).
\index{Baguley , Manchester@Baguley, \emph{Manchester}}

\subsubsection*{Ardwick \hspace*{\fill}\nolinebreak[1]%
\enspace\hspace*{\fill}
\finalhyphendemerits=0
[15th November]}

\index{Ardwick , Manchester@Ardwick, \emph{Manchester}}

Death of Tom O'Callaghan (Lab).

\noindent
\begin{tabular*}{\columnwidth}{@{\extracolsep{\fill}} p{0.545\columnwidth} >{\itshape}l r @{\extracolsep{\fill}}}
Tina Hewitson & Lab & 1904\\
Karl Wardlaw & Grn & 120\\
Liaqat Ali & LD & 94\\
Jamie Williams & C & 92\\
Allison Newsham & UKIP & 61\\
Shari Holden & TUSC & 52\\
Steven Carden & BNP & 43\\
\end{tabular*}



\council{Oldham}

\subsubsection*{Failsworth East \hspace*{\fill}\nolinebreak[1]%
\enspace\hspace*{\fill}
\finalhyphendemerits=0
[14th June]}

\index{Failsworth East , Oldham@Failsworth E., \emph{Oldham}}

Death of Barbara Dawson (Lab).

\noindent
\begin{tabular*}{\columnwidth}{@{\extracolsep{\fill}} p{0.545\columnwidth} >{\itshape}l r @{\extracolsep{\fill}}}
Norman Briggs & Lab & 1199\\
Carrol Ashton & UKIP & 209\\
Ron Wise & LD & 109\\
\end{tabular*}

\subsubsection*{Failsworth West \hspace*{\fill}\nolinebreak[1]%
\enspace\hspace*{\fill}
\finalhyphendemerits=0
[15th November]}

\index{Failsworth West , Oldham@Failsworth W., \emph{Oldham}}

Death of Glenys Butterworth (Lab).

\noindent
\begin{tabular*}{\columnwidth}{@{\extracolsep{\fill}} p{0.545\columnwidth} >{\itshape}l r @{\extracolsep{\fill}}}
Elaine Garry & Lab & 832\\
Warren Bates & UKIP & 489\\
Lewis Quigg & C & 122\\
Jean Betteridge & Grn & 68\\
Martin Dinoff & LD & 26\\
\end{tabular*}



\section{Merseyside}

\council{Liverpool}

\subsubsection*{Allerton and Hunts Cross \hspace*{\fill}\nolinebreak[1]%
\enspace\hspace*{\fill}
\finalhyphendemerits=0
[5th July]}

\index{Allerton and Hunts Cross , Liverpool@Allerton \& Hunts Cross, \emph{Liverpool}}

Resignation of Vera Best (LD).

\noindent
\begin{tabular*}{\columnwidth}{@{\extracolsep{\fill}} p{0.545\columnwidth} >{\itshape}l r @{\extracolsep{\fill}}}
Rachael O'Byrne & Lab & 1450\\
Mirna Juarez & LD & 564\\
Chris Hall & C & 240\\
Christopher Hulme & Lib & 165\\
Maggie Williams & Grn & 77\\
Lynne Wild & TUSC & 31\\
\end{tabular*}

\subsubsection*{Riverside \hspace*{\fill}\nolinebreak[1]%
\enspace\hspace*{\fill}
\finalhyphendemerits=0
[5th July]}

\index{Riverside , Liverpool@Riverside, \emph{Liverpool}}

Election of Joe Anderson (Lab) as Mayor of Liverpool.

\noindent
\begin{tabular*}{\columnwidth}{@{\extracolsep{\fill}} p{0.545\columnwidth} >{\itshape}l r @{\extracolsep{\fill}}}
Hetty Wood & Lab & 1424\\
Peter Cranie & Grn & 163\\
Chris McDermott & TUSC & 115\\
Nicola Beckett & LD & 81\\
Alma McGing & C & 70\\
\end{tabular*}

\subsubsection*{Knotty Ash \hspace*{\fill}\nolinebreak[1]%
\enspace\hspace*{\fill}
\finalhyphendemerits=0
[15th November]}

\index{Knotty Ash , Liverpool@Knotty Ash, \emph{Liverpool}}

Resignation of Jacqui Nasuh (Lab).

\noindent
\begin{tabular*}{\columnwidth}{@{\extracolsep{\fill}} p{0.545\columnwidth} >{\itshape}l r @{\extracolsep{\fill}}}
Gerard Taylor & Lab & 1213\\
Stephen Maddiston & LD & 149\\
Ann Hines & Lib & 131\\
\sloppyword{Adam Heatherington} & UKIP & 101\\
Derek Grue & EDP & 50\\
Charlotte Cosgrove & TUSC & 48\\
Jack Stallworthy & C & 40\\
Jonathan Deamer & Grn & 36\\
\end{tabular*}



\section[South Yorkshire]{\sloppyword{South Yorkshire}}

\council{Rotherham}

At the May 2012 ordinary election there was an unfilled vacancy in Brinsworth and Catcliffe ward due to the disqualification (non-attenadance) of John Gamble (NF elected as BNP).
\index{Brinsworth and Catcliffe , Rotherham@Brinsworth \& Catcliffe, \emph{Rotherham}}



\section{Tyne and Wear}

\council{Newcastle upon Tyne}

NuTCFP = Newcastle upon Tyne Community First Party (It's Time to Put Newcastle First)

\subsubsection*{Dene \hspace*{\fill}\nolinebreak[1]%
\enspace\hspace*{\fill}
\finalhyphendemerits=0
[3rd May]}

\index{Dene , Newcastle upon Tyne@Dene, \emph{Newcastle upon Tyne}}

Resignation of Sharon Bailey (LD).

Combined with the 2012 ordinary election.
%; see page \pageref{DeneNewcastleTyne} for the result.

\subsubsection*{Ouseburn \hspace*{\fill}\nolinebreak[1]%
\enspace\hspace*{\fill}
\finalhyphendemerits=0
[15th November]}

\index{Ouseburn , Newcastle upon Tyne@Ouseburn, \emph{Newcastle upon Tyne}}

Resignation of Ian Preston (Lab).

\noindent
\begin{tabular*}{\columnwidth}{@{\extracolsep{\fill}} p{0.5\columnwidth} >{\itshape}l r @{\extracolsep{\fill}}}
Stephen Powers & Lab & 714\\
Mark Nelson & LD & 665\\
Ian Fraser & NuTCFP & 73\\
Joshua Chew & C & 49\\
\end{tabular*}

\council{North Tyneside}

\subsubsection*{Wallsend \hspace*{\fill}\nolinebreak[1]%
\enspace\hspace*{\fill}
\finalhyphendemerits=0
[15th November; LD gain from Lab]}

\index{Wallsend , North Tyneside@Wallsend, \emph{N. Tyneside}}

Resignation of Jules Rutherford (Lab).

\noindent
\begin{tabular*}{\columnwidth}{@{\extracolsep{\fill}} p{0.545\columnwidth} >{\itshape}l r @{\extracolsep{\fill}}}
Michael Huscroft & LD & 1158\\
Ron Bales & Lab & 693\\
Barbara Bake & C & 72\\
Martin Collins & Grn & 66\\
\end{tabular*}

\council{South Tyneside}

At the May 2012 ordinary election there was an unfilled vacancy in Harton ward due to the death of Jim Capstick (South Tyneside Progressives).
\index{Harton , South Tyneside@Harton, \emph{S. Tyneside}}

\subsubsection*{Simonside and Rekendyke \hspace*{\fill}\nolinebreak[1]%
\enspace\hspace*{\fill}
\finalhyphendemerits=0
[3rd May]}

\index{Simonside and Rekendyke , South Tyneside@Simonside \& Rekendyke, \emph{S. Tyneside}}

Death of Joan Meeks (Lab).

Combined with the 2012 ordinary election.
%; see page \pageref{SimonsideRekendykeSouthTyneside} for the result.

\council{Sunderland}

\subsubsection*{Sandhill \hspace*{\fill}\nolinebreak[1]%
\enspace\hspace*{\fill}
\finalhyphendemerits=0
[3rd May]}

\index{Sandhill , Sunderland@Sandhill, \emph{Sunderland}}

Death of John Gallagher (Lab).

Combined with the 2012 ordinary election.
%; see page \pageref{SandhillSunderland} for the result.



\section{West Midlands}

\council{Dudley}

\subsubsection*{\sloppyword{Kingswinford North and Wall Heath} \hspace*{\fill}\nolinebreak[1]%
\enspace\hspace*{\fill}
\finalhyphendemerits=0
[3rd May]}

\index{Kingswinford North and Wall Heath , Dudley@Kingswinford N. \& Wall Heath, \emph{Dudley}}

Resignation of Paul Woodall (C).

Combined with the 2012 ordinary election.
%; see page \pageref{KingswinfordNorthWallHeathDudley} for the result.

\subsubsection*{Wollaston and Stourbridge Town \hspace*{\fill}\nolinebreak[1]%
\enspace\hspace*{\fill}
\finalhyphendemerits=0
[3rd May]}

\index{Wollaston and Stourbridge Town , Dudley@Wollaston \& Stourbridge Town, \emph{Dudley}}

Death of Margaret Cowell (C).

Combined with the 2012 ordinary election.
%; see page \pageref{WollastonStourbridgeTownDudley} for the result.

\subsubsection*{Norton \hspace*{\fill}\nolinebreak[1]%
\enspace\hspace*{\fill}
\finalhyphendemerits=0
[21st June]}

\index{Norton , Dudley@Norton, \emph{Dudley}}

Death of Angus Adams (C).

\noindent
\begin{tabular*}{\columnwidth}{@{\extracolsep{\fill}} p{0.545\columnwidth} >{\itshape}l r @{\extracolsep{\fill}}}
Colin Elcock & C & 1375\\
Adnan Rashid & Lab & 633\\
Christopher Bramall & LD & 259\\
Glen Wilson & UKIP & 229\\
Benjamin Sweeney & Grn & 143\\
Kevin Inman & NF & 47\\
\end{tabular*}

\council{Walsall}

\subsubsection*{Bloxwich West \hspace*{\fill}\nolinebreak[1]%
\enspace\hspace*{\fill}
\finalhyphendemerits=0
[15th November]}

\index{Bloxwich West , Walsall@Bloxwich W., \emph{Walsall}}

Resignation of Sue Fletcher-Hall (Lab).

\noindent
\begin{tabular*}{\columnwidth}{@{\extracolsep{\fill}} p{0.545\columnwidth} >{\itshape}l r @{\extracolsep{\fill}}}
Patti Lane & Lab & 1049\\
Abi Pitt & C & 783\\
Liz Hazell & UKIP & 195\\
Christine Cockayne & LD & 61\\
\end{tabular*}

\council{Wolverhampton}

\subsubsection*{Park \hspace*{\fill}\nolinebreak[1]%
\enspace\hspace*{\fill}
\finalhyphendemerits=0
[15th November]}

\index{Park , Wolverhampton@Park, \emph{Wolverhampton}}

Death of Manohar Minhas (Lab).

\noindent
\begin{tabular*}{\columnwidth}{@{\extracolsep{\fill}} p{0.545\columnwidth} >{\itshape}l r @{\extracolsep{\fill}}}
Craig Collingswood & Lab & 1023\\
Jenny Brewer & C & 482\\
Roger Gray & LD & 179\\
David Mackintosh & UKIP & 81\\
\end{tabular*}



\section{West Yorkshire}

\council{Bradford}

At the May 2012 ordinary election there was an unfilled vacancy in Keighley West ward due to the resignation of Robert Payne (C).
\index{Keighley West , Bradford@Keighley W., \emph{Bradford}}

\subsubsection*{Wharfedale \hspace*{\fill}\nolinebreak[1]%
\enspace\hspace*{\fill}
\finalhyphendemerits=0
[15th November]}

\index{Wharfedale , Bradford@Wharfedale, \emph{Bradford}}

Resignation of Matt Palmer (C).

\noindent
\begin{tabular*}{\columnwidth}{@{\extracolsep{\fill}} p{0.545\columnwidth} >{\itshape}l r @{\extracolsep{\fill}}}
Jackie Whiteley & C & 1353\\
David Green & Lab & 485\\
Janet Souyave & Grn & 320\\
Paul Treadwell & LD & 222\\
Samuel Fletcher & UKIP & 124\\
\end{tabular*}

\council{Calderdale}

\subsubsection*{Warley \hspace*{\fill}\nolinebreak[1]%
\enspace\hspace*{\fill}
\finalhyphendemerits=0
[19th July]}

\index{Warley , Calderdale@Warley, \emph{Calderdale}}

Resignation of Keith Hutson (LD).

\noindent
\begin{tabular*}{\columnwidth}{@{\extracolsep{\fill}} p{0.545\columnwidth} >{\itshape}l r @{\extracolsep{\fill}}}
James Baker & LD & 1066\\
Jonathan Timbers & Lab & 896\\
Christopher Pearson & C & 454\\
Charles Gate & Grn & 140\\
\end{tabular*}

\council{Wakefield}

At the May 2012 ordinary election there was an unfilled vacancy in Hemsworth ward due to the resignation of Ian Womersley (Ind).
\index{Hemsworth , Wakefield@Hemsworth, \emph{Wakefield}}

\section{Bedfordshire}

\council{Central Bedfordshire}

\subsubsection*{Stotfold and Langford \hspace*{\fill}\nolinebreak[1]%
\enspace\hspace*{\fill}
\finalhyphendemerits=0
[16th August]}

\index{Stotfold and Langford , Central Bedfordshire@Stotfold \& Langford, \emph{C. Beds.}}

Death of Jonathan Clarke (C).

\noindent
\begin{tabular*}{\columnwidth}{@{\extracolsep{\fill}} p{0.545\columnwidth} >{\itshape}l r @{\extracolsep{\fill}}}
Gillian Clarke & C & 1021\\
\sloppyword{Satinderjit Singh Dhaliwal} & Lab & 446\\
George Konstantinidis & UKIP & 221\\
Anthony Baines & LD & 147\\
\end{tabular*}

\subsubsection*{Biggleswade South \hspace*{\fill}\nolinebreak[1]%
\enspace\hspace*{\fill}
\finalhyphendemerits=0
[15th November]}

\index{Biggleswade South , Central Bedfordshire@Biggleswade S., \emph{C. Beds.}}

Death of Peter Vickers (C).

\noindent
\begin{tabular*}{\columnwidth}{@{\extracolsep{\fill}} p{0.545\columnwidth} >{\itshape}l r @{\extracolsep{\fill}}}
Tim Woodward & C & 642\\
Sheila Grayston & Lab & 523\\
Jonathan Medlock & Ind & 437\\
Gee Leach & LD & 73\\
\end{tabular*}

\subsubsection*{Silsoe and Shillington \hspace*{\fill}\nolinebreak[1]%
\enspace\hspace*{\fill}
\finalhyphendemerits=0
[15th November; Ind gain from C]}

\index{Silsoe and Shillington , Central Bedfordshire@Silsoe \& Shillington, \emph{C. Beds.}}

Resignation of Iain MacKilligan (C).

\noindent
\begin{tabular*}{\columnwidth}{@{\extracolsep{\fill}} p{0.545\columnwidth} >{\itshape}l r @{\extracolsep{\fill}}}
Alison Graham & Ind & 527\\
Martin Hawkins & C & 236\\
Steven Wildman & UKIP & 73\\
Janet Nunn & LD & 66\\
Gareth Ellis & Grn & 56\\
\end{tabular*}



\section{Berkshire}

\council{Slough}

\subsubsection*{Baylis and Stoke \hspace*{\fill}\nolinebreak[1]%
\enspace\hspace*{\fill}
\finalhyphendemerits=0
[8th March]}

\index{Baylis and Stoke , Slough@Baylis \& Stoke, \emph{Slough}}

Disqualification (non-attendance) of Azhar Qureshi (Lab).

\noindent
\begin{tabular*}{\columnwidth}{@{\extracolsep{\fill}} p{0.545\columnwidth} >{\itshape}l r @{\extracolsep{\fill}}}
Mohammed Nazir & Lab & 1300\\
Pervez Choudhury & Ind & 764\\
Allan Deverill & UKIP & 72\\
Ivan Dukes & Ind & 68\\
\end{tabular*}

\council{Windsor and Maidenhead}

\subsubsection*{Pinkneys Green \hspace*{\fill}\nolinebreak[1]%
\enspace\hspace*{\fill}
\finalhyphendemerits=0
[25th October; LD gain from C]}

\index{Pinkneys Green , Windsor and Maidenhead@Pinkneys Green, \emph{Windsor \& Maidenhead}}

Death of Wilson Hendry (C).

\noindent
\begin{tabular*}{\columnwidth}{@{\extracolsep{\fill}} p{0.545\columnwidth} >{\itshape}l r @{\extracolsep{\fill}}}
Simon Werner & LD & 839\\
\sloppyword{Catherine Hollingsworth} & C & 831\\
George Chamberlaine & UKIP & 152\\
Patrick McDonald & Lab & 121\\
\end{tabular*}

\section{Buckinghamshire}

\council{Chiltern}

\subsubsection*{Central \hspace*{\fill}\nolinebreak[1]%
\enspace\hspace*{\fill}
\finalhyphendemerits=0
[15th November]}

\index{Central , Chiltern@Central, \emph{Chiltern}}

Resignation of Martyn Groves (C).

\noindent
\begin{tabular*}{\columnwidth}{@{\extracolsep{\fill}} p{0.545\columnwidth} >{\itshape}l r @{\extracolsep{\fill}}}
Jonathan Rush & C & 495\\
David Rafferty & LD & 62\\
Stephen Agar & Lab & 60\\
Alan Stevens & UKIP & 52\\
\end{tabular*}

\council{South Bucks}

\subsubsection*{Iver Heath \hspace*{\fill}\nolinebreak[1]%
\enspace\hspace*{\fill}
\finalhyphendemerits=0
[14th June]}

\index{Iver Heath , South Bucks@Iver Heath, \emph{S. Bucks}}

Resignation of Julian Wilson (C).

\noindent
\begin{tabular*}{\columnwidth}{@{\extracolsep{\fill}} p{0.545\columnwidth} >{\itshape}l r @{\extracolsep{\fill}}}
Luisa Sullivan & C & 404\\
Adam Pamment & UKIP & 196\\
Peter Chapman & LD & 62\\
\end{tabular*}



\section{Cambridgeshire}

\council{Fenland}

\subsubsection*{St Marys \hspace*{\fill}\nolinebreak[1]%
\enspace\hspace*{\fill}
\finalhyphendemerits=0
[25th October]}

\index{Saint Marys , Fenland@St Marys, \emph{Fenland}}

Death of Ken Peachey (C).

\noindent
\begin{tabular*}{\columnwidth}{@{\extracolsep{\fill}} p{0.545\columnwidth} >{\itshape}l r @{\extracolsep{\fill}}}
Gary Swan & C & 397\\
Roy Gerstner & Ind & 160\\
Colin Gale & Lab & 78\\
\end{tabular*}

\council{Huntingdonshire}

\subsubsection*{Earith \hspace*{\fill}\nolinebreak[1]%
\enspace\hspace*{\fill}
\finalhyphendemerits=0
[21st June]}

\index{Earith , Huntingdonshire@Earith, \emph{Hunts.}}

Death of Philip Godfrey (C).

\noindent
\begin{tabular*}{\columnwidth}{@{\extracolsep{\fill}} p{0.545\columnwidth} >{\itshape}l r @{\extracolsep{\fill}}}
Robin Carter & C & 524\\
Alan Fitzgerald & UKIP & 437\\
Iain Ramsbottom & Lab & 96\\
Tony Hulme & LD & 92\\
Lord Toby Jug & Loony & 54\\
\end{tabular*}



\section{Cheshire}

\council{Halton}

At the May 2012 ordinary election there were unfilled vacancies in Grange and Riverside wards due to the deaths of John Swain and Dave Leadbetter (both Lab) respectively.
\index{Grange , Halton@Grange, \emph{Halton}}
\index{Riverside , Halton@Riverside, \emph{Halton}}



\section{Cornwall}

\council{Cornwall}

MK = Mebyon Kernow

\subsubsection*{St Keverne and Meneage \hspace*{\fill}\nolinebreak[1]%
\enspace\hspace*{\fill}
\finalhyphendemerits=0
[20th September; C gain from Ind]}

\index{Saint Keverne and Meneage , Cornwall@St Keverne \& Meneage, \emph{Cornwall}}

Death of Pam Lyne (Ind).

\noindent
\begin{tabular*}{\columnwidth}{@{\extracolsep{\fill}} p{0.545\columnwidth} >{\itshape}l r @{\extracolsep{\fill}}}
Walter Sanger & C & 585\\
Nicholas Driver & LD & 279\\
Sandra Martin & UKIP & 141\\
Steven Richards & Lab & 52\\
\end{tabular*}

\subsubsection*{Gwinear-Gwithian and St Erth \hspace*{\fill}\nolinebreak[1]%
\enspace\hspace*{\fill}
\finalhyphendemerits=0
[6th December]}

\index{Gwinear-Gwithian and Saint Erth , Cornwall@Gwinear-Gwithian \& St Erth, \emph{Cornwall}}

Death of Ray Tovey (C).

\noindent
\begin{tabular*}{\columnwidth}{@{\extracolsep{\fill}} p{0.545\columnwidth} >{\itshape}l r @{\extracolsep{\fill}}}
Anthony Pascoe & C & 332\\
Sheila Furneaux & Ind & 167\\
Michael Roberts & Ind & 163\\
Yvonne Bates & LD & 121\\
Malcolm Hurst & Lab & 76\\
John Gillingham & MK & 58\\
Derek Elliott & Ind & 24\\
\end{tabular*}



\section{Cumbria}

\subsection*{County Council}\index{Cumbria}

\subsubsection*{Castle \hspace*{\fill}\nolinebreak[1]%
\enspace\hspace*{\fill}
\finalhyphendemerits=0
[1st March; Lab gain from LD]}

\index{Castle , Cumbria@Castle, \emph{Cumbria}}

Death of Jim Tootle (LD).

\noindent
\begin{tabular*}{\columnwidth}{@{\extracolsep{\fill}} p{0.545\columnwidth} >{\itshape}l r @{\extracolsep{\fill}}}
Willie Whalen & Lab & 407\\
Olwyn Luckley & LD & 369\\
Keith Meller & C & 93\\
Michael Owen & UKIP & 72\\
Neil Boothman & Grn & 54\\
\end{tabular*}

\subsubsection*{Kendal Strickland and Fell \hspace*{\fill}\nolinebreak[1]%
\enspace\hspace*{\fill}
\finalhyphendemerits=0
[3rd May]}

\index{Kendal Strickland and Fell , Cumbria@Kendal Strickland \& Fell, \emph{Cumbria}}

Death of Brendan Jameson (LD).

\noindent
\begin{tabular*}{\columnwidth}{@{\extracolsep{\fill}} p{0.545\columnwidth} >{\itshape}l r @{\extracolsep{\fill}}}
John McCreesh & LD & 1157\\
Paul Braithwaite & Lab & 542\\
Patrick Birchall & C & 268\\
Malcolm Nightingale & UKIP & 110\\
\end{tabular*}

\subsubsection*{Aspatria and Wharrels \hspace*{\fill}\nolinebreak[1]%
\enspace\hspace*{\fill}
\finalhyphendemerits=0
[24th May]}

\index{Aspatria and Wharrels , Cumbria@Aspatria \& Wharrels, \emph{Cumbria}}

Resignation of Mike Johnson (C).

\noindent
\begin{tabular*}{\columnwidth}{@{\extracolsep{\fill}} p{0.545\columnwidth} >{\itshape}l r @{\extracolsep{\fill}}}
Jim Lister & C & 625\\
Brian Cope & Lab & 520\\
Bill Finlay & Ind & 390\\
Phill Roberts & LD & 206\\
David Bober & Grn & 58\\
\end{tabular*}

\council{Carlisle}

At the May 2012 ordinary election there was an unfilled vacancy in Castle ward due to the death of Jim Tootle (LD).
\index{Castle , Carlisle@Castle, \emph{Carlisle}}

\subsubsection*{Harraby \hspace*{\fill}\nolinebreak[1]%
\enspace\hspace*{\fill}
\finalhyphendemerits=0
[21st June]}

\index{Harraby , Carlisle@Harraby, \emph{Carlisle}}

Death of Dave Weedall (Lab).

\noindent
\begin{tabular*}{\columnwidth}{@{\extracolsep{\fill}} p{0.545\columnwidth} >{\itshape}l r @{\extracolsep{\fill}}}
Don Forrester & Lab & 637\\
Keith Meller & C & 180\\
Eddie Haughan & UKIP & 90\\
Michael Gee & LD & 71\\
James Tucker & Grn & 31\\
\end{tabular*}

\council{Eden}

\subsubsection*{Penrith Pategill \hspace*{\fill}\nolinebreak[1]%
\enspace\hspace*{\fill}
\finalhyphendemerits=0
[15th November; LD gain from C]}

\index{Penrith Pategill , Eden@Penrith Pategill, \emph{Eden}}

Resignation of David Harding (C).

\noindent
\begin{tabular*}{\columnwidth}{@{\extracolsep{\fill}} p{0.545\columnwidth} >{\itshape}l r @{\extracolsep{\fill}}}
John Tompkins & LD & 221\\
John Bateman & C & 86\\
Jamie Ayers & Lab & 77\\
Ian Holt & BNP & 28\\
\end{tabular*}

\council{South Lakeland}

At the May 2012 ordinary election there was an unfilled vacancy in Ulverston North ward due to the death of Colin Hodgson (C).
\index{Ulverston North , South Lakeland@Ulverston N., \emph{S. Lakeland}}

\subsubsection*{Windermere Town \hspace*{\fill}\nolinebreak[1]%
\enspace\hspace*{\fill}
\finalhyphendemerits=0
[9th February]}

\index{Windermere Town , South Lakeland@Windermere Town, \emph{S. Lakeland}}

Resignation of Sandra Britton (LD).

\noindent
\begin{tabular*}{\columnwidth}{@{\extracolsep{\fill}} p{0.545\columnwidth} >{\itshape}l r @{\extracolsep{\fill}}}
Jo Stephenson & LD & 418\\
Sandra Lilley & C & 85\\
Penny Henderson & Lab & 50\\
Robert Gibson & UKIP & 20\\
\end{tabular*}

\subsubsection*{Kendal Fell \hspace*{\fill}\nolinebreak[1]%
\enspace\hspace*{\fill}
\finalhyphendemerits=0
[3rd May]}

\index{Kendal Fell , South Lakeland@Kendal Fell, \emph{S. Lakeland}}

Death of Brendan Jameson (LD).

Combined with the 2012 ordinary election.
%; see page \pageref{KendalFellSouthLakeland} for the result.



\section{Derbyshire}

\council{Amber Valley}

\subsubsection*{Kilburn, Denby and Holbrook \hspace*{\fill}\nolinebreak[1]%
\enspace\hspace*{\fill}
\finalhyphendemerits=0
[3rd May]}

\index{Kilburn, Denby and Holbrook , Amber Valley@Kilburn, Denby \& Holbrook, \emph{Amber Valley}}

Death of Mel Hall (C).

Combined with the 2012 ordinary election.
%; see page \pageref{KilburnDenbyHolbrookAmberValley} for the result.

\council{Chesterfield}

\subsubsection*{St Helen's \hspace*{\fill}\nolinebreak[1]%
\enspace\hspace*{\fill}
\finalhyphendemerits=0
[5th July]}

\index{Saint Helen's , Chesterfield@St Helen's, \emph{Chesterfield}}

Death of Trevor Reynolds (Lab).

\noindent
\begin{tabular*}{\columnwidth}{@{\extracolsep{\fill}} p{0.545\columnwidth} >{\itshape}l r @{\extracolsep{\fill}}}
Tom Murphy & Lab & 579\\
Tony Rogers & LD & 412\\
Keith Lomas & UKIP & 205\\
Kate Barker & Grn & 54\\
Shelley Dale & C & 23\\
\end{tabular*}

\council{Derbyshire Dales}

\subsubsection*{Bradwell \hspace*{\fill}\nolinebreak[1]%
\enspace\hspace*{\fill}
\finalhyphendemerits=0
[12th January]}

\index{Bradwell , Derbyshire Dales@Bradwell, \emph{Derbys. Dales}}

Death of Janet Goodison (C).

\noindent
\begin{tabular*}{\columnwidth}{@{\extracolsep{\fill}} p{0.545\columnwidth} >{\itshape}l r @{\extracolsep{\fill}}}
Christopher Furness & C & \emph{unop.}\\
\end{tabular*}

\council{High Peak}

\subsubsection*{Buxton Central \hspace*{\fill}\nolinebreak[1]%
\enspace\hspace*{\fill}
\finalhyphendemerits=0
[29th March]}

\index{Buxton Central , High Peak@Buxton C., \emph{High Peak}}

Resignation of Philip Ashmore (Lab).

\noindent
\begin{tabular*}{\columnwidth}{@{\extracolsep{\fill}} p{0.545\columnwidth} >{\itshape}l r @{\extracolsep{\fill}}}
Jean Todd & Lab & 416\\
Bob Morris & C & 396\\
Derek Webb & LD & 70\\
Louise Glasscoe & Ind & 42\\
\end{tabular*}



\section{Devon}

\council{Plymouth}

At the May 2012 ordinary election there was an unfilled vacancy in Plympton Erle ward due to the death of John Lock (C).
\index{Plympton Erle , Plymouth@Plympton Erle, \emph{Plymouth}}

\council{West Devon}

\subsubsection*{Tavistock North \hspace*{\fill}\nolinebreak[1]%
\enspace\hspace*{\fill}
\finalhyphendemerits=0
[22nd March; Ind gain from C]}

\index{Tavistock North , West Devon@Tavistock N., \emph{W. Devon}}

Resignation of Darren Lake (C).

\noindent
\begin{tabular*}{\columnwidth}{@{\extracolsep{\fill}} p{0.545\columnwidth} >{\itshape}l r @{\extracolsep{\fill}}}
Jeff Moody & Ind & 407\\
Colin Rogers & C & 256\\
Adam Bridgewater & LD & 225\\
Moira Brown & Lab & 125\\
Andrew Mudge & UKIP & 78\\
Daniel Worth & Ind & 57\\
\end{tabular*}

\subsubsection*{Tavistock North \hspace*{\fill}\nolinebreak[1]%
\enspace\hspace*{\fill}
\finalhyphendemerits=0
[21st June]}

\index{Tavistock North , West Devon@Tavistock N., \emph{W. Devon}}

Resignation of Mike Harper (C).

\noindent
\begin{tabular*}{\columnwidth}{@{\extracolsep{\fill}} p{0.545\columnwidth} >{\itshape}l r @{\extracolsep{\fill}}}
John Sheldon & C & 349\\
Caroline Keane & Ind & 206\\
Moira Brown & Lab & 196\\
Joyce Hillson & LD & 187\\
\end{tabular*}

\columnbreak

\section{Dorset}

\council{Bournemouth}

\subsubsection*{Redhill and Northbourne \hspace*{\fill}\nolinebreak[1]%
\enspace\hspace*{\fill}
\finalhyphendemerits=0
[31st May]}

\index{Redhill and Northbourne , Bournemouth@Redhill \& Northbourne, \emph{Bournemouth}}

Resignation of Peter Charon (C).

\noindent
\begin{tabular*}{\columnwidth}{@{\extracolsep{\fill}} p{0.545\columnwidth} >{\itshape}l r @{\extracolsep{\fill}}}
\sloppyword{David d'Orton-Gibson} & C & 675\\
Jo Kennedy & Lab & 539\\
Pat Lewis & LD & 424\\
Claire Smith & Ind & 398\\
Nicholas Atkinson & UKIP & 327\\
Sandra Hale & Grn & 54\\
Peter Woodley & Ind & 51\\
Colin Smith & BNP & 42\\
Paul Graham & Ind & 15\\
\end{tabular*}

\subsubsection*{Littledown and Iford \hspace*{\fill}\nolinebreak[1]%
\enspace\hspace*{\fill}
\finalhyphendemerits=0
[15th November]}

\index{Littledown and Iford , Bournemouth@Littledown \& Iford, \emph{Bournemouth}}

Resignation of Nick King (C).

\noindent
\begin{tabular*}{\columnwidth}{@{\extracolsep{\fill}} p{0.545\columnwidth} >{\itshape}l r @{\extracolsep{\fill}}}
Gill Seymour & C & 810\\
Debbie Sharman & Lab & 247\\
David Hughes & UKIP & 229\\
Peter Pull & LD & 214\\
\end{tabular*}

\council{Poole}

Poole = Poole People

\subsubsection*{Branksome East \hspace*{\fill}\nolinebreak[1]%
\enspace\hspace*{\fill}
\finalhyphendemerits=0
[15th November]}

\index{Branksome East , Poole@Branksome E., \emph{Poole}}

Resignation of Stephen Rollo-Smith (C).

\noindent
\begin{tabular*}{\columnwidth}{@{\extracolsep{\fill}} p{0.545\columnwidth} >{\itshape}l r @{\extracolsep{\fill}}}
Jane Thomas & C & 553\\
Jenny Paton & LD & 254\\
Peter Portnoi & Poole & 245\\
Diana Butler & UKIP & 132\\
\sloppyword{Hazel Malcolm-Walker} & Lab & 104\\
William Kimmet & BNP & 28\\
\end{tabular*}



\section{Durham}

\council{Darlington}

EFP = England First Party

\subsubsection*{Harrowgate Hill \hspace*{\fill}\nolinebreak[1]%
\enspace\hspace*{\fill}
\finalhyphendemerits=0
[12th April; C gain from Lab]}

\index{Harrowgate Hill , Darlington@Harrowgate Hill, \emph{Darlington}}

Disqualification (sentenced to 22 months' imprisonment, sexual assault, possession of child pornography) of Mark Burton (Lab).

\noindent
\begin{tabular*}{\columnwidth}{@{\extracolsep{\fill}} p{0.545\columnwidth} >{\itshape}l r @{\extracolsep{\fill}}}
Gill Cartwright & C & 694\\
Helen Crumbie & Lab & 607\\
Hilary Allen & LD & 142\\
Daniel Fairclough & UKIP & 95\\
Paul Thompson & EFP & 47\\
\end{tabular*}

\subsubsection*{Hurworth \hspace*{\fill}\nolinebreak[1]%
\enspace\hspace*{\fill}
\finalhyphendemerits=0
[12th April]}

\index{Hurworth , Darlington@Hurworth, \emph{Darlington}}

Disqualification (non-attendance) of Martin Swainston (LD).

\noindent
\begin{tabular*}{\columnwidth}{@{\extracolsep{\fill}} p{0.545\columnwidth} >{\itshape}l r @{\extracolsep{\fill}}}
Martin Swainston & LD & 474\\
\sloppyword{Christopher Brownbridge} & C & 436\\
Steve Rose & Lab & 144\\
David Davies & UKIP & 31\\
\end{tabular*}

\council{Durham}

\subsubsection*{Peterlee West \hspace*{\fill}\nolinebreak[1]%
\enspace\hspace*{\fill}
\finalhyphendemerits=0
[21st June; Lab gain from LD]}

\index{Peterlee West , Durham@Peterlee W., \emph{Durham}}

Death of Barbara Sloan (LD).

\noindent
\begin{tabular*}{\columnwidth}{@{\extracolsep{\fill}} p{0.545\columnwidth} >{\itshape}l r @{\extracolsep{\fill}}}
Jimmy Alvey & Lab & 767\\
Karen Hawley & Ind & 181\\
Wendy Bentley & LD & 99\\
Harvey Morgan & C & 47\\
\end{tabular*}

\council{Hartlepool}

PHF = Putting Hartlepool First

\subsubsection*{Seaton \hspace*{\fill}\nolinebreak[1]%
\enspace\hspace*{\fill}
\finalhyphendemerits=0
[25th October; PHF gain from Ind]}

\index{Seaton , Hartlepool@Seaton, \emph{Hartlepool}}

Death of Mike Turner (Ind).

\noindent
\begin{tabular*}{\columnwidth}{@{\extracolsep{\fill}} p{0.545\columnwidth} >{\itshape}l r @{\extracolsep{\fill}}}
Kelly Atkinson & PHF & 441\\
Ann Marshall & Lab & 261\\
David Young & Ind & 193\\
Tom Hind & UKIP & 128\\
Shane Moore & C & 94\\
Jim Tighe & LD & 31\\
\end{tabular*}



\section{East Sussex}

\council{Brighton and Hove}

\subsubsection*{East Brighton \hspace*{\fill}\nolinebreak[1]%
\enspace\hspace*{\fill}
\finalhyphendemerits=0
[18th October]}

\index{East Brighton , Brighton and Hove@East Brighton, \emph{Brighton \& Hove}}

Resignation of Craig Turton (Lab).

\noindent
\begin{tabular*}{\columnwidth}{@{\extracolsep{\fill}} p{0.545\columnwidth} >{\itshape}l r @{\extracolsep{\fill}}}
Chaun Wilson & Lab & 1596\\
Joe Miller & C & 531\\
Carlie Goldsmith & Grn & 456\\
Sabina Choudhury & UKIP & 148\\
Dominic Sokalski & LD & 59\\
Jon Redford & TUSC & 55\\
\end{tabular*}

\council{Eastbourne}

\subsubsection*{Meads \hspace*{\fill}\nolinebreak[1]%
\enspace\hspace*{\fill}
\finalhyphendemerits=0
[31st May]}

\index{Meads , Eastbourne@Meads, \emph{Eastbourne}}

Resignation of Nigel Goodyear (C).

\noindent
\begin{tabular*}{\columnwidth}{@{\extracolsep{\fill}} p{0.545\columnwidth} >{\itshape}l r @{\extracolsep{\fill}}}
Caroline Ansell & C & 1780\\
Gerard Thompson & LD & 465\\
Ian Cameron & UKIP & 321\\
Dennis Scard & Lab & 320\\
Keith Gell & Ind & 98\\
\end{tabular*}

\council{Wealden}

\subsubsection*{Frant\slash Withyham \hspace*{\fill}\nolinebreak[1]%
\enspace\hspace*{\fill}
\finalhyphendemerits=0
[14th June]}

\index{Frant\slash Withyham , Wealden@Frant\slash Withyham, \emph{Wealden}}

Death of John Padfield (C).

\noindent
\begin{tabular*}{\columnwidth}{@{\extracolsep{\fill}} p{0.545\columnwidth} >{\itshape}l r @{\extracolsep{\fill}}}
William Rutherford & C & 775\\
Tony Rothstein & Lab & 183\\
Chris Rycroft & LD & 116\\
\end{tabular*}

\subsubsection*{Polegate North \hspace*{\fill}\nolinebreak[1]%
\enspace\hspace*{\fill}
\finalhyphendemerits=0
[2nd August]}

\index{Polegate North , Wealden@Polegate N., \emph{Wealden}}

Death of Roy Martin (LD).

\noindent
\begin{tabular*}{\columnwidth}{@{\extracolsep{\fill}} p{0.545\columnwidth} >{\itshape}l r @{\extracolsep{\fill}}}
Don Broadbent & LD & 494\\
Joe O'Riordan & Ind & 267\\
Edward Board & C & 263\\
Bernie Goodwin & UKIP & 212\\
Alex Mthobi & Lab & 76\\
\end{tabular*}

\subsubsection*{Heathfield North and Central \hspace*{\fill}\nolinebreak[1]%
\enspace\hspace*{\fill}
\finalhyphendemerits=0
[15th November]}

\index{Heathfield North and Central , Wealden@Heathfield N. \& C., \emph{Wealden}}

Death of Peter Newnham (C).

\noindent
\begin{tabular*}{\columnwidth}{@{\extracolsep{\fill}} p{0.545\columnwidth} >{\itshape}l r @{\extracolsep{\fill}}}
Raymond Cade & C & 685\\
Bob Bowlder & UKIP & 301\\
Craig Austen-White & Lab & 225\\
Jim Benson & LD & 184\\
\end{tabular*}



\section{East Yorkshire}

\council{East Riding}

\subsubsection*{South East Holderness \hspace*{\fill}\nolinebreak[1]%
\enspace\hspace*{\fill}
\finalhyphendemerits=0
[3rd May]}

\index{South East Holderness , East Riding@South East Holderness, \emph{E. Riding}}

Death of Richard Stead (C).

\noindent
\begin{tabular*}{\columnwidth}{@{\extracolsep{\fill}} p{0.545\columnwidth} >{\itshape}l r @{\extracolsep{\fill}}}
Kevan Hough & C & 1187\\
Jed Lee & Lab & 1011\\
John Windas & Ind & 971\\
\end{tabular*}



\section{Essex}

\subsection*{County Council}\index{Essex}

\subsubsection*{Chelmsford North \hspace*{\fill}\nolinebreak[1]%
\enspace\hspace*{\fill}
\finalhyphendemerits=0
[28th June]}

\index{Chelmsford North , Essex@Chelmsford N., \emph{Essex}}

Death of Tom Smith-Hughes (LD).

\noindent
\begin{tabular*}{\columnwidth}{@{\extracolsep{\fill}} p{0.545\columnwidth} >{\itshape}l r @{\extracolsep{\fill}}}
Stephen Robinson & LD & 1614\\
Robert Pontin & C & 941\\
Nastassia Player & Lab & 711\\
Leslie Retford & UKIP & 435\\
Reza Hossain & Grn & 134\\
\end{tabular*}

\council{Basildon}

\subsubsection*{Billericay West \hspace*{\fill}\nolinebreak[1]%
\enspace\hspace*{\fill}
\finalhyphendemerits=0
[3rd May]}

\index{Billericay West , Basildon@Billericay W., \emph{Basildon}}

Death of Stephen Horgan (C).

Combined with the 2012 ordinary election.
%; see page \pageref{BillericayWBasildon} for the result.

\council{Braintree}

\subsubsection*{Braintree East \hspace*{\fill}\nolinebreak[1]%
\enspace\hspace*{\fill}
\finalhyphendemerits=0
[15th March; Lab gain from C]}

\index{Braintree East , Braintree@Braintree E., \emph{Braintree}}

Resignation of David Messer (C).

\noindent
\begin{tabular*}{\columnwidth}{@{\extracolsep{\fill}} p{0.545\columnwidth} >{\itshape}l r @{\extracolsep{\fill}}}
Eric Lynch & Lab & 554\\
Stephen Nimmons & C & 338\\
Phil Palij & UKIP & 131\\
Wendy Partridge & Grn & 76\\
Paul Lemon & Ind & 32\\
\end{tabular*}

\subsubsection*{Braintree South \hspace*{\fill}\nolinebreak[1]%
\enspace\hspace*{\fill}
\finalhyphendemerits=0
[15th March; Lab gain from C]}

\index{Braintree South , Braintree@Braintree S., \emph{Braintree}}

Resignation of Stephen Sandbrook (C).

\noindent
\begin{tabular*}{\columnwidth}{@{\extracolsep{\fill}} p{0.545\columnwidth} >{\itshape}l r @{\extracolsep{\fill}}}
Martin Green & Lab & 596\\
Abi Olumbori & C & 476\\
Timothy Reeve & Grn & 116\\
\end{tabular*}

\subsubsection*{Great Notley and Braintree West \hspace*{\fill}\nolinebreak[1]%
\enspace\hspace*{\fill}
\finalhyphendemerits=0
[15th March]}

\index{Great Notley and Braintree West , Braintree@Great Notley \& Braintree W., \emph{Braintree}}

Resignation of Claire Sandbrook (C).

\noindent
\begin{tabular*}{\columnwidth}{@{\extracolsep{\fill}} p{0.545\columnwidth} >{\itshape}l r @{\extracolsep{\fill}}}
Frankie Ricci & C & 532\\
Juliet Walton & Lab & 232\\
Gordon Helm & UKIP & 155\\
Lynne Maynard & Grn & 61\\
\end{tabular*}



\council{Brentwood}

\subsubsection*{Herongate, Ingrave and West Horndon \hspace*{\fill}\nolinebreak[1]%
\enspace\hspace*{\fill}
\finalhyphendemerits=0
[3rd May]}

\index{Herongate, Ingrave and West Horndon , Brentwood@Herongate, Ingrave \& West Horndon, \emph{Brentwood}}

Resignation of Gordon MacLellan (C).

Combined with the 2012 ordinary election.
%; see page \pageref{HerongateIngraveWestHorndonBrentwood} for the result.

\subsubsection*{Shenfield \hspace*{\fill}\nolinebreak[1]%
\enspace\hspace*{\fill}
\finalhyphendemerits=0
[6th December; LD gain from C]}

\index{Shenfield , Brentwood@Shenfield, \emph{Brentwood}}

Resignation of Lionel Lee (C).

\noindent
\begin{tabular*}{\columnwidth}{@{\extracolsep{\fill}} p{0.545\columnwidth} >{\itshape}l r @{\extracolsep{\fill}}}
Liz Cohen & LD & 728\\
Steve May & C & 557\\
David Watt & UKIP & 119\\
Richard Millwood & Lab & 31\\
\end{tabular*}



\council{Chelmsford}

\subsubsection*{Patching Hall \hspace*{\fill}\nolinebreak[1]%
\enspace\hspace*{\fill}
\finalhyphendemerits=0
[28th June]}

\index{Patching Hall , Chelmsford@Patching Hall, \emph{Chelmsford}}

Death of Tom Smith-Hughes (LD).

\noindent
\begin{tabular*}{\columnwidth}{@{\extracolsep{\fill}} p{0.545\columnwidth} >{\itshape}l r @{\extracolsep{\fill}}}
Paul Bentham & LD & 842\\
Stephen Fowell & C & 488\\
Chris Vince & Lab & 309\\
Ian Nicholls & UKIP & 263\\
Reza Hossain & Grn & 84\\
\end{tabular*}



\council{Epping Forest}

\subsubsection*{Epping Lindsey and Thornwood Common \hspace*{\fill}\nolinebreak[1]%
\enspace\hspace*{\fill}
\finalhyphendemerits=0
[3rd May]}

\index{Epping Lindsey and Thornwood Common , Epping Forest@Epping Lindsey \& Thornwood Common, \emph{Epping Forest}}

Resignation of Sarah Packford (C).

Combined with the 2012 ordinary election.
%; see page \pageref{EppingLindseyThornwoodCommonEppingForest} for the result.

\council{Harlow}

At the May 2012 ordinary election there was an unfilled vacancy in Mark Hall ward due to the death of Nick Macy (LD).
\index{Mark Hall , Harlow@Mark Hall, \emph{Harlow}}

\subsubsection*{Bush Fair \hspace*{\fill}\nolinebreak[1]%
\enspace\hspace*{\fill}
\finalhyphendemerits=0
[3rd May]}

\index{Bush Fair , Harlow@Bush Fair, \emph{Harlow}}

Resignation of Helen Hart (Lab).

Combined with the 2012 ordinary election.
%; see page \pageref{BushFairHarlow} for the result.

\subsubsection*{Toddbrook \hspace*{\fill}\nolinebreak[1]%
\enspace\hspace*{\fill}
\finalhyphendemerits=0
[15th November]}

\index{Toddbrook , Harlow@Toddbrook, \emph{Harlow}}

Resignation of Bob Davis (Lab).

\noindent
\begin{tabular*}{\columnwidth}{@{\extracolsep{\fill}} p{0.545\columnwidth} >{\itshape}l r @{\extracolsep{\fill}}}
Christine O'Dell & Lab & 604\\
Clive Souter & C & 383\\
Bill Pryor & UKIP & 111\\
Roy Jackson & LD & 53\\
\end{tabular*}



\council{Maldon}

\subsubsection*{Tollesbury \hspace*{\fill}\nolinebreak[1]%
\enspace\hspace*{\fill}
\finalhyphendemerits=0
[5th July; Lab gain from C]}

\index{Tollesbury , Maldon@Tollesbury, \emph{Maldon}}

Death of Russell Porter (C).

\noindent
\begin{tabular*}{\columnwidth}{@{\extracolsep{\fill}} p{0.545\columnwidth} >{\itshape}l r @{\extracolsep{\fill}}}
Stevan Slodzik & Lab & 271\\
Andrew St Joseph & C & 257\\
Gerald Munson & Ind & 75\\
\end{tabular*}



\council{Tendring}

HOSRA = Holland-on-Sea Residents Association

\subsubsection*{St Bartholomews \hspace*{\fill}\nolinebreak[1]%
\enspace\hspace*{\fill}
\finalhyphendemerits=0
[22nd March]}

\index{Saint Bartholomews , Tendring@St Bartholomews, \emph{Tendring}}

Death of Mary Bragg (HOSRA).

\noindent
\begin{tabular*}{\columnwidth}{@{\extracolsep{\fill}} p{0.49\columnwidth} >{\itshape}l r @{\extracolsep{\fill}}}
Chris Cowlin & HOSRA & 1087\\
Linda Mead & C & 289\\
Keith Henderson & Lab & 120\\
\end{tabular*}

\subsubsection*{St Bartholomews \hspace*{\fill}\nolinebreak[1]%
\enspace\hspace*{\fill}
\finalhyphendemerits=0
[16th August]}

\index{Saint Bartholomews , Tendring@St Bartholomews, \emph{Tendring}}

Resignation of Chris Cowlin (HOSRA).

\noindent
\begin{tabular*}{\columnwidth}{@{\extracolsep{\fill}} p{0.49\columnwidth} >{\itshape}l r @{\extracolsep{\fill}}}
Colin Winfield & HOSRA & 1042\\
Graham Townley & C & 145\\
\end{tabular*}



\council{Thurrock}

At the May 2012 ordinary election there was an unfilled vacancy in Corringham and Fobbing ward due to the resignation of Ian Harrison (Ind C elected as C).
\index{Corringham and Fobbing , Thurrock@Corringham \& Fobbing, \emph{Thurrock}}

\section{Gloucestershire}

\council{Cotswold}

\subsubsection*{Fosseridge \hspace*{\fill}\nolinebreak[1]%
\enspace\hspace*{\fill}
\finalhyphendemerits=0
[31st May]}

\index{Fosseridge , Cotswold@Fosseridge, \emph{Cotswold}}

Death of Frank Beard (Lab).

\noindent
\begin{tabular*}{\columnwidth}{@{\extracolsep{\fill}} p{0.545\columnwidth} >{\itshape}l r @{\extracolsep{\fill}}}
Julian Beale & C & 397\\
Danny Loveridge & LD & 168\\
Chris Turner & Ind & 98\\
\end{tabular*}

\council{Forest of Dean}

\subsubsection*{Cinderford West \hspace*{\fill}\nolinebreak[1]%
\enspace\hspace*{\fill}
\finalhyphendemerits=0
[12th January]}

\index{Cinderford West , Forest of Dean@Cinderford W., \emph{Forest of Dean}}

Death of Frank Beard (Lab).

\noindent
\begin{tabular*}{\columnwidth}{@{\extracolsep{\fill}} p{0.545\columnwidth} >{\itshape}l r @{\extracolsep{\fill}}}
Roger Sterry & Lab & 496\\
Aaron Freeman & C & 236\\
Colin Guyton & UKIP & 119\\
Colin Davies & LD & 89\\
\end{tabular*}

\council{South Gloucestershire}

\subsubsection*{Dodington \hspace*{\fill}\nolinebreak[1]%
\enspace\hspace*{\fill}
\finalhyphendemerits=0
[6th September]}

\index{Dodington , South Gloucestershire@Dodington, \emph{S. Glos.}}

Resignation of Dafydd Holbrook (LD).

\noindent
\begin{tabular*}{\columnwidth}{@{\extracolsep{\fill}} p{0.545\columnwidth} >{\itshape}l r @{\extracolsep{\fill}}}
Tony Davis & LD & 787\\
Michael McGrath & Lab & 243\\
Aaron Foot & UKIP & 213\\
Kate Duffy & C & 139\\
\end{tabular*}

\council{Stroud}

At the May 2012 ordinary election there was an unfilled vacancy in Severn ward due to the death of Norman Smith (C).
\index{Severn , Stroud@Severn, \emph{Stroud}}

\subsubsection*{Coaley and Uley \hspace*{\fill}\nolinebreak[1]%
\enspace\hspace*{\fill}
\finalhyphendemerits=0
[3rd May]}

\index{Coaley and Uley , Stroud@Coaley \& Uley, \emph{Stroud}}

Resignation of Graham Trave (C).

Combined with the 2012 ordinary election.
%; see page \pageref{CoaleyUleyStroud} for the result.



\section{Hampshire}

\subsection*{County Council}\index{Hampshire}

\subsubsection*{\sloppyword{Winchester Southern Parishes} \hspace*{\fill}\nolinebreak[1]%
\enspace\hspace*{\fill}
\finalhyphendemerits=0
[9th February]}

\index{Winchester Southern Parishes , Hampshire@\sloppyword{Winchester Southern Parishes, \emph{Hants.}}}

Death of Frederick Allgood (C).

\noindent
\begin{tabular*}{\columnwidth}{@{\extracolsep{\fill}} p{0.545\columnwidth} >{\itshape}l r @{\extracolsep{\fill}}}
Patricia Stallard & C & 1661\\
Vivian Achwal & LD & 1038\\
Stephen Harris & UKIP & 133\\
John Vivian & Grn & 130\\
David Picton-Jones & Lab & 124\\
\end{tabular*}

\council{Basingstoke and Deane}

\subsubsection*{Popley East \hspace*{\fill}\nolinebreak[1]%
\enspace\hspace*{\fill}
\finalhyphendemerits=0
[3rd May]}

\index{Popley East , Basingstoke and Deane@Popley E., \emph{Basingstoke \& Deane}}

Resignation of Andrew McCormick (Lab).

Combined with the 2012 ordinary election.
%; see page \pageref{PopleyEastBasingstokeDeane} for the result.



\council{East Hampshire}

\subsubsection*{Horndean Downs \hspace*{\fill}\nolinebreak[1]%
\enspace\hspace*{\fill}
\finalhyphendemerits=0
[12th July]}

\index{Horndean Downs , East Hampshire@Horndean Downs, \emph{E. Hants.}}

Resignation of Julia Marshall (C).

\noindent
\begin{tabular*}{\columnwidth}{@{\extracolsep{\fill}} p{0.545\columnwidth} >{\itshape}l r @{\extracolsep{\fill}}}
Guy Shepherd & C & 315\\
Terry Port & LD & 274\\
Katie Green & Lab & 73\\
David Alexander & UKIP & 63\\
\end{tabular*}



\council{Fareham}

\subsubsection*{Portchester East \hspace*{\fill}\nolinebreak[1]%
\enspace\hspace*{\fill}
\finalhyphendemerits=0
[15th March]}

\index{Portchester East , Fareham@Portchester E., \emph{Fareham}}

Resignation of Chris Brown (LD).

\noindent
\begin{tabular*}{\columnwidth}{@{\extracolsep{\fill}} p{0.545\columnwidth} >{\itshape}l r @{\extracolsep{\fill}}}
Geoff Fazackarley & LD & 1216\\
Alison Walker & C & 840\\
Richard Ryan & Lab & 323\\
John Vivian & Grn & 90\\
Manny Martins & Ind & 77\\
\end{tabular*}



\council{Havant}

At the May 2012 ordinary election there was an unfilled vacancy in Purbrook ward due to the death of David Farrow (C).
\index{Purbrook , Havant@Purbrook, \emph{Havant}}

\subsubsection*{Battins \hspace*{\fill}\nolinebreak[1]%
\enspace\hspace*{\fill}
\finalhyphendemerits=0
[15th November]}

\index{Battins , Havant@Battins, \emph{Havant}}

Resignation of Katie Ray (LD).

\noindent
\begin{tabular*}{\columnwidth}{@{\extracolsep{\fill}} p{0.545\columnwidth} >{\itshape}l r @{\extracolsep{\fill}}}
Faith Ponsonby & LD & 452\\
Virginia Steel & Lab & 188\\
Kris Sapcote & C & 153\\
Ray Finch & UKIP & 94\\
\end{tabular*}



\council{New Forest}

\subsubsection*{Milford \hspace*{\fill}\nolinebreak[1]%
\enspace\hspace*{\fill}
\finalhyphendemerits=0
[19th July]}

\index{Milford , New Forest@Milford, \emph{New Forest}}

Resignation of Michael Pemberton (C).

\noindent
\begin{tabular*}{\columnwidth}{@{\extracolsep{\fill}} p{0.545\columnwidth} >{\itshape}l r @{\extracolsep{\fill}}}
Sophie Beeton & C & 963\\
Caroline Hexter & Lab & 240\\
\end{tabular*}



\council{Portsmouth}

At the May 2012 ordinary election there was an unfilled vacancy in Paulsgrove ward due to the resignation of Mike Blake (C).
\index{Paulsgrove , Portsmouth@Paulsgrove, \emph{Portsmouth}}

\council{Southampton}

\subsubsection*{Bitterne Park \hspace*{\fill}\nolinebreak[1]%
\enspace\hspace*{\fill}
\finalhyphendemerits=0
[3rd May]}

\index{Bitterne Park , Southampton@Bitterne Park, \emph{Southampton}}

Resignation of Phil Williams (C).

Combined with the 2012 ordinary election.
%; see page \pageref{BitterneParkSouthampton} for the result.

\subsubsection*{Peartree \hspace*{\fill}\nolinebreak[1]%
\enspace\hspace*{\fill}
\finalhyphendemerits=0
[3rd May]}

\index{Peartree , Southampton@Peartree, \emph{Southampton}}

Resignation of Gerry Drake (LD).

Combined with the 2012 ordinary election.
%; see page \pageref{PeartreeSouthampton} for the result.



\section{Herefordshire}
\index{Herefordshire}

IOCH = It's Our County (Herefordshire)

\subsubsection*{St Nicholas \hspace*{\fill}\nolinebreak[1]%
\enspace\hspace*{\fill}
\finalhyphendemerits=0
[20th September]}

\index{Saint Nicholas , Herefordshire@St Nicholas, \emph{Herefs.}}

Death of Julie Woodward (IOCH).

\noindent
\begin{tabular*}{\columnwidth}{@{\extracolsep{\fill}} p{0.545\columnwidth} >{\itshape}l r @{\extracolsep{\fill}}}
Anthony Powers & IOCH & 604\\
David Hurds & LD & 222\\
Mark McEvilly & C & 204\\
Stan Gyford & Lab & 104\\
\end{tabular*}



\section{Hertfordshire}

\subsection*{County Council}\index{Hertfordshire}

\subsubsection*{Waltham Cross \hspace*{\fill}\nolinebreak[1]%
\enspace\hspace*{\fill}
\finalhyphendemerits=0
[22nd March]}

\index{Waltham Cross , Hertfordshire@Waltham Cross, \emph{Herts.}}

Death of Terry Price (C).

\noindent
\begin{tabular*}{\columnwidth}{@{\extracolsep{\fill}} p{0.545\columnwidth} >{\itshape}l r @{\extracolsep{\fill}}}
Dee Hart & C & 1389\\
Malcolm Aitken & Lab & 837\\
Albert Nicholas & UKIP & 159\\
Peter Huse & LD & 76\\
\end{tabular*}

\subsubsection*{Meriden Tudor \hspace*{\fill}\nolinebreak[1]%
\enspace\hspace*{\fill}
\finalhyphendemerits=0
[3rd May]}

\index{Meriden Tudor , Hertfordshire@Meriden Tudor, \emph{Herts.}}

Resignation of Audrey Oaten (LD).

\noindent
\begin{tabular*}{\columnwidth}{@{\extracolsep{\fill}} p{0.545\columnwidth} >{\itshape}l r @{\extracolsep{\fill}}}
Kareen Hastrick & LD & 1231\\
Diana Ivory & Lab & 791\\
Richard Southern & C & 534\\
Nicholas Lincoln & UKIP & 351\\
Paula Broadhurst & Grn & 154\\
\end{tabular*}

\subsubsection*{Hemel Hempstead Town \hspace*{\fill}\nolinebreak[1]%
\enspace\hspace*{\fill}
\finalhyphendemerits=0
[21st June]}

\index{Hemel Hempstead Town , Hertfordshire@Hemel Hempstead Town, \emph{Herts.}}

Resignation of Steve Holmes (C).

\noindent
\begin{tabular*}{\columnwidth}{@{\extracolsep{\fill}} p{0.545\columnwidth} >{\itshape}l r @{\extracolsep{\fill}}}
William Wyatt-Lowe & C & 1413\\
Mike Bromberg & Lab & 693\\
Chris Angell & LD & 456\\
Paul Harris & Grn & 180\\
Howard Koch & UKIP & 151\\
Rodney Tucker & Ind & 61\\
\end{tabular*}

\council{Dacorum}

\subsubsection*{Woodhall Farm \hspace*{\fill}\nolinebreak[1]%
\enspace\hspace*{\fill}
\finalhyphendemerits=0
[21st June]}

\index{Hemel Hempstead Town , Hertfordshire@Hemel Hempstead Town, \emph{Herts.}}

Resignation of Steve Holmes (C).

\noindent
\begin{tabular*}{\columnwidth}{@{\extracolsep{\fill}} p{0.545\columnwidth} >{\itshape}l r @{\extracolsep{\fill}}}
Rosie Sutton & C & 566\\
Paul Eastwood & Lab & 406\\
Nitesh Dave & LD & 70\\
Simon Deacon & EDP & 47\\
Howard Koch & UKIP & 43\\
Paul de Hoest & Grn & 24\\
\end{tabular*}



\council{Hertsmere}

\subsubsection*{Potters Bar Furzefield \hspace*{\fill}\nolinebreak[1]%
\enspace\hspace*{\fill}
\finalhyphendemerits=0
[3rd May]}

\index{Potters Bar Furzefield , Hertsmere@Potters Bar Furzefield, \emph{Hertsmere}}

Death of Ron Morris (C).

Combined with the 2012 ordinary election.
%; see page \pageref{PottersBarFurzefieldHertsmere} for the result.



\council{North Hertfordshire}

\subsubsection*{Hitchwood, Offa and Hoo \hspace*{\fill}\nolinebreak[1]%
\enspace\hspace*{\fill}
\finalhyphendemerits=0
[15th November]}

\index{Hitchwood, Offa and Hoo , North Hertfordshire@Hitchwood, Offa \& Hoo, \emph{N. Herts.}}

Death of David Miller (C).

\noindent
\begin{tabular*}{\columnwidth}{@{\extracolsep{\fill}} p{0.545\columnwidth} >{\itshape}l r @{\extracolsep{\fill}}}
Faye Barnard & C & 774\\
Peter Robbins & UKIP & 217\\
Jackie McDonald & Lab & 189\\
Peter Johnson & LD & 110\\
George Howe & Grn & 72\\
\end{tabular*}

\subsubsection*{Letchworth South East \hspace*{\fill}\nolinebreak[1]%
\enspace\hspace*{\fill}
\finalhyphendemerits=0
[15th November]}

\index{Letchworth South East , North Hertfordshire@Letchworth S.E., \emph{N. Herts.}}

Resignation of Richard Harman (C).

\noindent
\begin{tabular*}{\columnwidth}{@{\extracolsep{\fill}} p{0.545\columnwidth} >{\itshape}l r @{\extracolsep{\fill}}}
Julian Cunningham & C & 761\\
\sloppyword{Martin Stears-Handscomb} & Lab & 399\\
John Barry & UKIP & 184\\
Margaret Higbid & LD & 88\\
Mario May & Grn & 51\\
\end{tabular*}



\subsection*{St Albans}\index{Saint Albans@St Albans}

\subsubsection*{Batchwood \hspace*{\fill}\nolinebreak[1]%
\enspace\hspace*{\fill}
\finalhyphendemerits=0
[19th January; Lab gain from LD]}

\index{Batchwood , Saint Albans@Batchwood, \emph{St Albans}}

Resignation of Amanda Archer (LD).

\noindent
\begin{tabular*}{\columnwidth}{@{\extracolsep{\fill}} p{0.545\columnwidth} >{\itshape}l r @{\extracolsep{\fill}}}
Roma Mills & Lab & 1002\\
David Partridge & LD & 395\\
Tim Smith & C & 347\\
Naomi Love & Grn & 76\\
\end{tabular*}

\subsubsection*{London Colney \hspace*{\fill}\nolinebreak[1]%
\enspace\hspace*{\fill}
\finalhyphendemerits=0
[3rd May]}

\index{London Colney , Saint Albans@London Colney, \emph{St Albans}}

Resignation of Irene Willcocks (C).

Combined with the 2012 ordinary election.
%; see page \pageref{LondonColneyStAlbans} for the result.

\council{Watford}

\subsubsection*{Central \hspace*{\fill}\nolinebreak[1]%
\enspace\hspace*{\fill}
\finalhyphendemerits=0
[15th November]}

\index{Central , Watford@Central, \emph{Watford}}

Resignation of Chris Leslie (LD).

\noindent
\begin{tabular*}{\columnwidth}{@{\extracolsep{\fill}} p{0.545\columnwidth} >{\itshape}l r @{\extracolsep{\fill}}}
Lizz Ayre & LD & 716\\
Avril Haley & Lab & 550\\
Binita Mehta & C & 151\\
Su Murray & Grn & 83\\
Renie Price & UKIP & 59\\
\end{tabular*}



\section{Kent}

\subsection*{County Council}\index{Kent}

\subsubsection*{Tunbridge Wells East \hspace*{\fill}\nolinebreak[1]%
\enspace\hspace*{\fill}
\finalhyphendemerits=0
[14th June]}

\index{Tunbridge Wells East , Kent@Tunbridge Wells E., \emph{Kent}}

Death of Kevin Lynes (C).

\noindent
\begin{tabular*}{\columnwidth}{@{\extracolsep{\fill}} p{0.545\columnwidth} >{\itshape}l r @{\extracolsep{\fill}}}
James Tansley & C & 1171\\
David Neve & LD & 1022\\
Christopher Hoare & UKIP & 1000\\
Ian Carvell & Lab & 321\\
Hazel Dawe & Grn & 109\\
\end{tabular*}

\subsubsection*{Maidstone Central \hspace*{\fill}\nolinebreak[1]%
\enspace\hspace*{\fill}
\finalhyphendemerits=0
[18th October]}

\index{Maidstone Central , Kent@Maidstone C., \emph{Kent}}

Death of Malcolm Robertson (LD).

\noindent
\begin{tabular*}{\columnwidth}{@{\extracolsep{\fill}} p{0.545\columnwidth} >{\itshape}l r @{\extracolsep{\fill}}}
Robert Bird & LD & 2169\\
Paul Butcher & C & 1301\\
Paul Harper & Lab & 943\\
John Stanford & UKIP & 510\\
Stuart Jeffery & Grn & 393\\
Michael Walters & EDP & 89\\
\end{tabular*}

\subsubsection*{Gravesham Rural \hspace*{\fill}\nolinebreak[1]%
\enspace\hspace*{\fill}
\finalhyphendemerits=0
[20th December]}

\index{Gravesham Rural , Kent@Gravesham Rural, \emph{Kent}}

Death of Mike Snelling (C).

\noindent
\begin{tabular*}{\columnwidth}{@{\extracolsep{\fill}} p{0.545\columnwidth} >{\itshape}l r @{\extracolsep{\fill}}}
Bryan Sweetland & C & 1780\\
Geoffrey Clark & UKIP & 634\\
Douglas Christie & Lab & 397\\
Gill McGill & LD & 91\\
\end{tabular*}

\council{Ashford}

Ashford = Ashford Independents

\subsubsection*{Tenterden South \hspace*{\fill}\nolinebreak[1]%
\enspace\hspace*{\fill}
\finalhyphendemerits=0
[12th July]}

\index{Tenterden South , Ashford@Tenterden S., \emph{Ashford}}

Death of Peter Goddard (C).

\noindent
\begin{tabular*}{\columnwidth}{@{\extracolsep{\fill}} p{0.545\columnwidth} >{\itshape}l r @{\extracolsep{\fill}}}
Jill Hutchinson & C & 277\\
Roy Isworth & Ashford & 179\\
Norman Taylor & UKIP & 72\\
Chris Took & LD & 40\\
Dara Farrell & Lab & 36\\
\end{tabular*}

\council{Canterbury}

\subsubsection*{Blean Forest \hspace*{\fill}\nolinebreak[1]%
\enspace\hspace*{\fill}
\finalhyphendemerits=0
[20th September]}

\index{Blean Forest , Canterbury@Blean Forest, \emph{Canterbury}}

Death of Hazel McCabe (C).

\noindent
\begin{tabular*}{\columnwidth}{@{\extracolsep{\fill}} p{0.545\columnwidth} >{\itshape}l r @{\extracolsep{\fill}}}
Ben Fitter & C & 342\\
Carol Goldstein & Lab & 185\\
Dan Smith & LD & 121\\
Russell Page & Grn & 64\\
Howard Farmer & UKIP & 38\\
John Hippisley & Ind & 24\\
\end{tabular*}



\council{Dartford}

SGRA = Swanscombe and Greenhithe Residents Association

\subsubsection*{Castle \hspace*{\fill}\nolinebreak[1]%
\enspace\hspace*{\fill}
\finalhyphendemerits=0
[27th September]}

\index{Castle , Dartford@Castle, \emph{Dartford}}

Death of Sheila East (C).

\noindent
\begin{tabular*}{\columnwidth}{@{\extracolsep{\fill}} p{0.545\columnwidth} >{\itshape}l r @{\extracolsep{\fill}}}
Paul Cutler & C & 191\\
Hayley Reece & Lab & 111\\
Stephen Wilders & UKIP & 60\\
Vic Openshaw & SGRA & 50\\
Frances Moore & EDP & 32\\
\end{tabular*}



\council{Gravesham}

\subsubsection*{Meopham North \hspace*{\fill}\nolinebreak[1]%
\enspace\hspace*{\fill}
\finalhyphendemerits=0
[20th December]}

\index{Meopham North , Gravesham@Meopham N., \emph{Gravesham}}

Death of Mike Snelling (C).

\noindent
\begin{tabular*}{\columnwidth}{@{\extracolsep{\fill}} p{0.545\columnwidth} >{\itshape}l r @{\extracolsep{\fill}}}
Julia Burgoyne & C & 419\\
Geoffrey Clark & UKIP & 204\\
Douglas Christie & Lab & 108\\
Martin Wilson & LD & 36\\
\end{tabular*}



\council{Maidstone}

\subsubsection*{Heath \hspace*{\fill}\nolinebreak[1]%
\enspace\hspace*{\fill}
\finalhyphendemerits=0
[3rd May]}

\index{Heath , Maidstone@Heath, \emph{Maidstone}}

Resignation of Jenni Sharp (LD).

Combined with the 2012 ordinary election.
%; see page \pageref{HeathMaidstone} for the result.

\subsubsection*{Allington \hspace*{\fill}\nolinebreak[1]%
\enspace\hspace*{\fill}
\finalhyphendemerits=0
[18th October]}

\index{Allington , Maidstone@Allington, \emph{Maidstone}}

Death of Malcolm Robertson (LD).

\noindent
\begin{tabular*}{\columnwidth}{@{\extracolsep{\fill}} p{0.545\columnwidth} >{\itshape}l r @{\extracolsep{\fill}}}
Belinda Watson & LD & 885\\
Barry Ginley & C & 434\\
Marianna Poliszczuk & Lab & 179\\
Gareth Kendall & UKIP & 162\\
Jill Shepherd & Grn & 30\\
\end{tabular*}



\council{Sevenoaks}

\subsubsection*{Cowden and Hever \hspace*{\fill}\nolinebreak[1]%
\enspace\hspace*{\fill}
\finalhyphendemerits=0
[29th March]}

\index{Cowden and Hever , Sevenoaks@Cowden \& Hever, \emph{Sevenoaks}}

Resignation of Gerry Ryan (C).

\noindent
\begin{tabular*}{\columnwidth}{@{\extracolsep{\fill}} p{0.545\columnwidth} >{\itshape}l r @{\extracolsep{\fill}}}
Christopher Neal & C & 296\\
Lorraine Millgate & UKIP & 81\\
\end{tabular*}

\subsubsection*{Crockenhill and Well Hill \hspace*{\fill}\nolinebreak[1]%
\enspace\hspace*{\fill}
\finalhyphendemerits=0
[29th March; Lab gain from Ind]}

\index{Crockenhill and Well Hill , Sevenoaks@Crockenhill \& Well Hill, \emph{Sevenoaks}}

Death of Colin Dibsdall (Ind).

\noindent
\begin{tabular*}{\columnwidth}{@{\extracolsep{\fill}} p{0.545\columnwidth} >{\itshape}l r @{\extracolsep{\fill}}}
Jenny Dibsdall & Lab & 304\\
Adrian Crossley & C & 177\\
Christopher Heath & UKIP & 40\\
\end{tabular*}



\council{Shepway}

PF = People First

\subsubsection*{Folkestone Park \hspace*{\fill}\nolinebreak[1]%
\enspace\hspace*{\fill}
\finalhyphendemerits=0
[22nd November; LD gain from C]}

\index{Folkestone Park , Shepway@Folkestone Park, \emph{Shepway}}

Resignation of Tristan Allen (C).

\noindent
\begin{tabular*}{\columnwidth}{@{\extracolsep{\fill}} p{0.545\columnwidth} >{\itshape}l r @{\extracolsep{\fill}}}
Lynne Beaumont & LD & 461\\
Leo Griggs & C & 320\\
Patricia Copping & PF & 200\\
Hod Birkby & UKIP & 153\\
Nicola Keen & Lab & 111\\
Derek Moran & Grn & 29\\
\end{tabular*}



\council{Swale}

\subsubsection*{Kemsley \hspace*{\fill}\nolinebreak[1]%
\enspace\hspace*{\fill}
\finalhyphendemerits=0
[8th March]}

\index{Kemsley , Swale@Kemsley, \emph{Swale}}

Death of Brenda Simpson (C).

\noindent
\begin{tabular*}{\columnwidth}{@{\extracolsep{\fill}} p{0.545\columnwidth} >{\itshape}l r @{\extracolsep{\fill}}}
Mike Whiting & C & 384\\
Richard Raycraft & Lab & 312\\
Derek Carnell & UKIP & 279\\
Berick Tomes & LD & 166\\
\end{tabular*}



\council{Thanet}

\subsubsection*{Westgate-on-Sea \hspace*{\fill}\nolinebreak[1]%
\enspace\hspace*{\fill}
\finalhyphendemerits=0
[5th July; Lab gain from C]}

\index{Westgate-on-Sea , Thanet@Westgate-on-Sea, \emph{Thanet}}

Death of Brian Goodwin (C).

\noindent
\begin{tabular*}{\columnwidth}{@{\extracolsep{\fill}} p{0.545\columnwidth} >{\itshape}l r @{\extracolsep{\fill}}}
Jodie Hibbert & Lab & 482\\
James Maskell & C & 377\\
Ash Ashbee & Ind & 316\\
Jeffrey Elenor & UKIP & 298\\
Bill Furness & LD & 64\\
Claire Mendelsohn & Ind & 22\\
\end{tabular*}



\council{Tonbridge and Malling}

\subsubsection*{West Malling and Leybourne \hspace*{\fill}\nolinebreak[1]%
\enspace\hspace*{\fill}
\finalhyphendemerits=0
[5th July]}

\index{West Malling and Leybourne , Tonbridge and Malling@West Malling \& Leybourne, \emph{Tonbridge \& Malling}}

Death of Mark Worrall (C).

\noindent
\begin{tabular*}{\columnwidth}{@{\extracolsep{\fill}} p{0.545\columnwidth} >{\itshape}l r @{\extracolsep{\fill}}}
Sophie Shrubsole & C & 769\\
Yvonne Smyth & LD & 472\\
Peter Stevens & UKIP & 127\\
Kathleen Garlick & Lab & 123\\
Michael Walters & EDP & 57\\
Howard Porter & Grn & 43\\
\end{tabular*}



\council{Tunbridge Wells}

\subsubsection*{Broadwater \hspace*{\fill}\nolinebreak[1]%
\enspace\hspace*{\fill}
\finalhyphendemerits=0
[3rd May]}

\index{Broadwater , Tunbridge Wells@Broadwater, \emph{Tunbridge Wells}}

Death of Peter Crawford (LD).

Combined with the 2012 ordinary election.
%; see page \pageref{BroadwaterTunbridgeWells} for the result.

\subsubsection*{Pembury \hspace*{\fill}\nolinebreak[1]%
\enspace\hspace*{\fill}
\finalhyphendemerits=0
[3rd May]}

\index{Pembury , Tunbridge Wells@Pembury, \emph{Tunbridge Wells}}

Resignation of Claire Brown (LD).

Combined with the 2012 ordinary election.
%; see page \pageref{PemburyTunbridgeWells} for the result.



\section{Lancashire}

\council{Blackpool}

\subsubsection*{Bloomfield \hspace*{\fill}\nolinebreak[1]%
\enspace\hspace*{\fill}
\finalhyphendemerits=0
[3rd May]}

\index{Bloomfield , Blackpool@Bloomfield, \emph{Blackpool}}

Death of Mary Smith (Lab).

\noindent
\begin{tabular*}{\columnwidth}{@{\extracolsep{\fill}} p{0.545\columnwidth} >{\itshape}l r @{\extracolsep{\fill}}}
John Jones & Lab & 731\\
Ged Walsh & C & 216\\
Beverley Keenan & Ind & 193\\
Sue Close & LD & 73\\
\end{tabular*}

\subsubsection*{Marton \hspace*{\fill}\nolinebreak[1]%
\enspace\hspace*{\fill}
\finalhyphendemerits=0
[3rd May; Lab gain from C]}

\index{Marton , Blackpool@Marton, \emph{Blackpool}}

Death of Jim Houldsworth (C).

\noindent
\begin{tabular*}{\columnwidth}{@{\extracolsep{\fill}} p{0.545\columnwidth} >{\itshape}l r @{\extracolsep{\fill}}}
Jim Elmes & Lab & 917\\
Peter Collins & C & 644\\
Philip Mitchell & Grn & 103\\
Kevan Benfold & LD & 76\\
\end{tabular*}

\council{Burnley}

\subsubsection*{Trinity \hspace*{\fill}\nolinebreak[1]%
\enspace\hspace*{\fill}
\finalhyphendemerits=0
[13th September]}

\index{Trinity , Burnley@Trinity, \emph{Burnley}}

Death of Tony Lambert (Lab).

\noindent
\begin{tabular*}{\columnwidth}{@{\extracolsep{\fill}} p{0.545\columnwidth} >{\itshape}l r @{\extracolsep{\fill}}}
Tony Martin & Lab & 493\\
Stephanie Forrest & LD & 256\\
Tom Watson & C & 96\\
Derek Dawson & BNP & 95\\
Craig Ramplee & UKIP & 35\\
Steven Smith & NF & 26\\
\end{tabular*}

\council{Chorley}

\subsubsection*{Adlington and Anderton \hspace*{\fill}\nolinebreak[1]%
\enspace\hspace*{\fill}
\finalhyphendemerits=0
[3rd May]}

\index{Adlington and Anderton , Chorley@Adlington \& Anderton, \emph{Chorley}}

Resignation of Catherine Hoyle (Lab).

Combined with the 2012 ordinary election.
%; see page \pageref{AdlingtonAndertonChorley} for the result.

\council{Fylde}

Integ = Integrity UK

\subsubsection*{Heyhouses \hspace*{\fill}\nolinebreak[1]%
\enspace\hspace*{\fill}
\finalhyphendemerits=0
[5th July]}

\index{Heyhouses , Fylde@Heyhouses, \emph{Fylde}}

Resignation of Peter Wood (C).

\noindent
\begin{tabular*}{\columnwidth}{@{\extracolsep{\fill}} p{0.545\columnwidth} >{\itshape}l r @{\extracolsep{\fill}}}
Barbara Nash & C & 401\\
Palmira Stafford & Ind & 313\\
Carol Gilligan & LD & 163\\
Ian Roberts & Grn & 150\\
Bill Dickson & UKIP & 147\\
Bill Whitehead & Integ & 25\\
\end{tabular*}

\council{Pendle}

\subsubsection*{Boulsworth \hspace*{\fill}\nolinebreak[1]%
\enspace\hspace*{\fill}
\finalhyphendemerits=0
[3rd May]}

\index{Boulsworth , Pendle@Boulsworth, \emph{Pendle}}

Resignation of George Askew (C).

Combined with the 2012 ordinary election.
%; see page \pageref{BoulsworthPendle} for the result.

\council{Preston}

\subsubsection*{Preston Rural North \hspace*{\fill}\nolinebreak[1]%
\enspace\hspace*{\fill}
\finalhyphendemerits=0
[3rd May]}

\index{Preston Rural North , Preston@Preston Rural N., \emph{Preston}}

Resignation of Kate Calder (C).

Combined with the 2012 ordinary election.
%; see page \pageref{PrestonRuralNorthPreston} for the result.

\subsubsection*{St George's \hspace*{\fill}\nolinebreak[1]%
\enspace\hspace*{\fill}
\finalhyphendemerits=0
[3rd May]}

\index{Saint George's , Preston@St George's, \emph{Preston}}

Resignation of Taalib Shamsuddin (Lab).

Combined with the 2012 ordinary election.
%; see page \pageref{SaintGeorgesPreston} for the result.



\section{Leicestershire}

\council{Blaby}

\subsubsection*{Ravenhurst and Fosse \hspace*{\fill}\nolinebreak[1]%
\enspace\hspace*{\fill}
\finalhyphendemerits=0
[28th June]}

\index{Ravenhurst and Fosse , Blaby@Ravenhurst \& Fosse, \emph{Blaby}}

Death of Phil Fox (Lab).

\noindent
\begin{tabular*}{\columnwidth}{@{\extracolsep{\fill}} p{0.545\columnwidth} >{\itshape}l r @{\extracolsep{\fill}}}
Sam Maxwell & Lab & 1083\\
Michael Potter & C & 501\\
\end{tabular*}

\subsubsection*{Pastures \hspace*{\fill}\nolinebreak[1]%
\enspace\hspace*{\fill}
\finalhyphendemerits=0
[27th September]}

\index{Pastures , Blaby@Pastures, \emph{Blaby}}

Resignation of Ted Webster-Williams (C).

\noindent
\begin{tabular*}{\columnwidth}{@{\extracolsep{\fill}} p{0.545\columnwidth} >{\itshape}l r @{\extracolsep{\fill}}}
Michael Potter & C & 442\\
Ann Malthouse & Lab & 281\\
\end{tabular*}

\council{Charnwood}

\subsubsection*{Sileby \hspace*{\fill}\nolinebreak[1]%
\enspace\hspace*{\fill}
\finalhyphendemerits=0
[28th June]}

\index{Sileby , Charnwood@Sileby, \emph{Charnwood}}

Death of Roy Brown (C).

\noindent
\begin{tabular*}{\columnwidth}{@{\extracolsep{\fill}} p{0.545\columnwidth} >{\itshape}l r @{\extracolsep{\fill}}}
Ken Jones & C & 703\\
Richard Watson & Lab & 450\\
Stephen Denham & BNP & 93\\
Stephen Coltman & LD & 29\\
\end{tabular*}

\subsubsection*{Loughborough Southfields \hspace*{\fill}\nolinebreak[1]%
\enspace\hspace*{\fill}
\finalhyphendemerits=0
[13th September; C gain from Lab]}

\index{Loughborough Southfields , Charnwood@Loughborough Southfields, \emph{Charnwood}}

Resignation of Graeme Smith (Lab).

\noindent
\begin{tabular*}{\columnwidth}{@{\extracolsep{\fill}} p{0.545\columnwidth} >{\itshape}l r @{\extracolsep{\fill}}}
Ted Parton & C & 538\\
Mary Draycott & Lab & 516\\
Diana Brass & LD & 54\\
\end{tabular*}

\council{Melton}

\subsubsection*{Melton Egerton \hspace*{\fill}\nolinebreak[1]%
\enspace\hspace*{\fill}
\finalhyphendemerits=0
[15th November; Ind gain from Lab]}

\index{Melton Egerton , Melton@Melton Egerton, \emph{Melton}}

Resignation of Steve Dungworth (Lab).

\noindent
\begin{tabular*}{\columnwidth}{@{\extracolsep{\fill}} p{0.545\columnwidth} >{\itshape}l r @{\extracolsep{\fill}}}
Mark Twittey & Ind & 253\\
Mike Brown & Lab & 232\\
Paul Phizacklea & C & 92\\
\end{tabular*}

\council{North West Leicestershire}

\subsubsection*{Ibstock and Heather \hspace*{\fill}\nolinebreak[1]%
\enspace\hspace*{\fill}
\finalhyphendemerits=0
[16th February]}

\index{Ibstock and Heather , North West Leicestershire@Ibstock \& Heather, \emph{N.W. Leics.}}

Resignation of Stacey Harris (Lab).

\noindent
\begin{tabular*}{\columnwidth}{@{\extracolsep{\fill}} p{0.545\columnwidth} >{\itshape}l r @{\extracolsep{\fill}}}
Dave de Lacy & Lab & 480\\
Kim Wyatt & LD & 372\\
Russell Boam & C & 357\\
Ivan Hammonds & Ind & 125\\
Sue Morrell & Grn & 32\\
Jakob Whiten & UKIP & 26\\
\end{tabular*}

\council{Oadby and Wigston}

\subsubsection*{Oadby Woodlands \hspace*{\fill}\nolinebreak[1]%
\enspace\hspace*{\fill}
\finalhyphendemerits=0
[16th February; C gain from LD]}

\index{Oadby Woodlands , Oadby and Wigston@Oadby Woodlands, \emph{Oadby \& Wigston}}

Resignation of Malcolm Brown (LD).

\noindent
\begin{tabular*}{\columnwidth}{@{\extracolsep{\fill}} p{0.545\columnwidth} >{\itshape}l r @{\extracolsep{\fill}}}
Bhupendra Dave & C & 454\\
Naveed Alam & LD & 360\\
Dan Price & UKIP & 151\\
\end{tabular*}



\section{Lincolnshire}

LincsInd = Lincolnshire Independents

\subsection*{County Council}\index{Lincolnshire}

\subsubsection*{Lincoln East \hspace*{\fill}\nolinebreak[1]%
\enspace\hspace*{\fill}
\finalhyphendemerits=0
[2nd August; Lab gain from C]}

\index{Lincoln East , Lincolnshire@Lincoln E., \emph{Lincs.}}

Resignation of Sara Cliff (C).

\noindent
\begin{tabular*}{\columnwidth}{@{\extracolsep{\fill}} p{0.52\columnwidth} >{\itshape}l r @{\extracolsep{\fill}}}
Robin Renshaw & Lab & 563\\
Simon Parr & C & 314\\
John Bishop & LD & 95\\
Pat Nurse & UKIP & 79\\
Nick Parker & TUSC & 79\\
Elliot Fountain & EDP & 24\\
\end{tabular*}

\subsubsection*{Nettleham and Saxilby \hspace*{\fill}\nolinebreak[1]%
\enspace\hspace*{\fill}
\finalhyphendemerits=0
[6th September; C gain from LD]}

\index{Nettleham and Saxilby , Lincolnshire@Nettleham \& Saxilby, \emph{Lincs.}}

Death of Ray Sellars (LD).

\noindent
\begin{tabular*}{\columnwidth}{@{\extracolsep{\fill}} p{0.49\columnwidth} >{\itshape}l r @{\extracolsep{\fill}}}
Jackie Brockway & C & 1026\\
Charles Shaw & LD & 600\\
\sloppyword{Howard Thompson} & UKIP & 266\\
Richard Coupland & Lab & 257\\
David Watson & LincsInd & 196\\
Elliot Fountain & EDP & 21\\
\end{tabular*}

\columnbreak

\council{Boston}

\subsubsection*{Frampton and Holme \hspace*{\fill}\nolinebreak[1]%
\enspace\hspace*{\fill}
\finalhyphendemerits=0
[18th October]}

\index{Frampton and Holme , Boston@Frampton \& Holme, \emph{Boston}}

Resignation of Brian Rush (Ind).

\noindent
\begin{tabular*}{\columnwidth}{@{\extracolsep{\fill}} p{0.52\columnwidth} >{\itshape}l r @{\extracolsep{\fill}}}
Stuart Ashton & Ind & 204\\
Maggie Peberdy & Ind & 139\\
Claire Rylott & C & 126\\
Sue Ransome & UKIP & 32\\
Mike Sheridan-Shinn & Lab & 19\\
\end{tabular*}



\council{East Lindsey}

SSF = Sutton on Sea First

\subsubsection*{Sutton on Sea North \hspace*{\fill}\nolinebreak[1]%
\enspace\hspace*{\fill}
\finalhyphendemerits=0
[11th October]}

\index{Sutton on Sea North , East Lindsey@Sutton on Sea N., \emph{E. Lindsey}}

Resignation of Andrew Ferryman (SSF).

\noindent
\begin{tabular*}{\columnwidth}{@{\extracolsep{\fill}} p{0.52\columnwidth} >{\itshape}l r @{\extracolsep{\fill}}}
Steve Palmer & SSF & 330\\
David Andrews & C & 118\\
Joyce Taylor & Lab & 90\\
Ian Wild & Ind & 63\\
\end{tabular*}



\council{North Kesteven}

\subsubsection*{Heckington Rural \hspace*{\fill}\nolinebreak[1]%
\enspace\hspace*{\fill}
\finalhyphendemerits=0
[29th March]}

\index{Heckington Rural , North Kesteven@Heckington Rural, \emph{N. Kesteven}}

Resignation of Andrew Key (C).

\noindent
\begin{tabular*}{\columnwidth}{@{\extracolsep{\fill}} p{0.52\columnwidth} >{\itshape}l r @{\extracolsep{\fill}}}
Sally Tarry & C & 578\\
Liz Peto & LincsInd & 510\\
\end{tabular*}



\council{North Lincolnshire}

\subsubsection*{Town \hspace*{\fill}\nolinebreak[1]%
\enspace\hspace*{\fill}
\finalhyphendemerits=0
[31st May]}

\index{Town , North Lincolnshire@Town, \emph{N. Lincs.}}

Death of Darrell Barkworth (Lab).

\noindent
\begin{tabular*}{\columnwidth}{@{\extracolsep{\fill}} p{0.52\columnwidth} >{\itshape}l r @{\extracolsep{\fill}}}
Haque Kataria & Lab & 1141\\
Abdul Wadud & C & 956\\
Douglas Ward & BNP & 143\\
\end{tabular*}



\council{South Holland}

\subsubsection*{Fleet \hspace*{\fill}\nolinebreak[1]%
\enspace\hspace*{\fill}
\finalhyphendemerits=0
[2nd August]}

\index{Fleet , South Holland@Fleet, \emph{S. Holland}}

Resignation of Shaun Keeble (C).

\noindent
\begin{tabular*}{\columnwidth}{@{\extracolsep{\fill}} p{0.52\columnwidth} >{\itshape}l r @{\extracolsep{\fill}}}
Peter Coupland & C & 285\\
Val Gemmell & Ind & 233\\
\end{tabular*}

\subsubsection*{Long Sutton \hspace*{\fill}\nolinebreak[1]%
\enspace\hspace*{\fill}
\finalhyphendemerits=0
[2nd August]}

\index{Long Sutton , South Holland@Long Sutton, \emph{S. Holland}}

Death of Dennis Tennant (Ind).

\noindent
\begin{tabular*}{\columnwidth}{@{\extracolsep{\fill}} p{0.52\columnwidth} >{\itshape}l r @{\extracolsep{\fill}}}
Andrew Tennant & Ind & 946\\
Jack Tyrrell & C & 757\\
\end{tabular*}



\council{West Lindsey}

\subsubsection*{Nettleham \hspace*{\fill}\nolinebreak[1]%
\enspace\hspace*{\fill}
\finalhyphendemerits=0
[6th September; C gain from LD]}

\index{Nettleham , West Lindsey@Nettleham, \emph{W. Lindsey}}

Death of Ray Sellars (LD).

\noindent
\begin{tabular*}{\columnwidth}{@{\extracolsep{\fill}} p{0.52\columnwidth} >{\itshape}l r @{\extracolsep{\fill}}}
Giles McNeill & C & 565\\
Guy Grainger & LD & 513\\
Howard Thompson & UKIP & 177\\
\end{tabular*}

\section{Norfolk}

\subsection*{County Council}\index{Norfolk}

\subsubsection*{\sloppyword{Clenchwarton and King's Lynn South} \hspace*{\fill}\nolinebreak[1]%
\enspace\hspace*{\fill}
\finalhyphendemerits=0
[27th September; Lab gain from C]}

\index{Clenchwarton and King's Lynn South , Norfolk@Clenchwarton \& King's Lynn S., \emph{Norfolk}}

Death of David Harwood (C).

\noindent
\begin{tabular*}{\columnwidth}{@{\extracolsep{\fill}} p{0.545\columnwidth} >{\itshape}l r @{\extracolsep{\fill}}}
Alexandra Kemp & Lab & 824\\
Paul Foster & C & 424\\
Kate Sayer & LD & 282\\
Michael Stone & UKIP & 271\\
\end{tabular*}



\council{Breckland}

\subsubsection*{Queen's \hspace*{\fill}\nolinebreak[1]%
\enspace\hspace*{\fill}
\finalhyphendemerits=0
[19th July]}

\index{Queen's , Breckland@Queen's, \emph{Breckland}}

Resignation of Simon Rogers (C).

\noindent
\begin{tabular*}{\columnwidth}{@{\extracolsep{\fill}} p{0.545\columnwidth} >{\itshape}l r @{\extracolsep{\fill}}}
Karen Pettitt & C & 489\\
John Williams & Lab & 454\\
\end{tabular*}

\subsubsection*{Harling and Heathlands \hspace*{\fill}\nolinebreak[1]%
\enspace\hspace*{\fill}
\finalhyphendemerits=0
[26th July]}

\index{Harling and Heathlands , Breckland@Harling \& Heathlands, \emph{Breckland}}

Resignation of Lady (Kay) Fisher (C).

\noindent
\begin{tabular*}{\columnwidth}{@{\extracolsep{\fill}} p{0.545\columnwidth} >{\itshape}l r @{\extracolsep{\fill}}}
\sloppyword{Marion Chapman-Allen} & C & 453\\
Denis Crawford & UKIP & 184\\
Stephen Green & Lab & 168\\
Steve Gordon & LD & 129\\
\end{tabular*}

\subsubsection*{Mid Forest \hspace*{\fill}\nolinebreak[1]%
\enspace\hspace*{\fill}
\finalhyphendemerits=0
[26th July]}

\index{Mid Forest , Breckland@Mid Forest, \emph{Breckland}}

Resignation of Bernard English (C).

\noindent
\begin{tabular*}{\columnwidth}{@{\extracolsep{\fill}} p{0.545\columnwidth} >{\itshape}l r @{\extracolsep{\fill}}}
Mike Nairn & C & 257\\
Alexander Vyse & Lab & 151\\
\end{tabular*}

\subsubsection*{Thetford-Abbey \hspace*{\fill}\nolinebreak[1]%
\enspace\hspace*{\fill}
\finalhyphendemerits=0
[13th September; Lab gain from Ind]}

\index{Thetford-Abbey , Breckland@Thetford-Abbey, \emph{Breckland}}

Resignation of Pauline Quadling (Ind).

\noindent
\begin{tabular*}{\columnwidth}{@{\extracolsep{\fill}} p{0.545\columnwidth} >{\itshape}l r @{\extracolsep{\fill}}}
Brenda Canham & Lab & 334\\
Roy Brame & C & 128\\
Denis Crawford & UKIP & 117\\
Mike Brindle & LD & 99\\
\end{tabular*}



\council{King's Lynn and West Norfolk}

\subsubsection*{Spellowfields \hspace*{\fill}\nolinebreak[1]%
\enspace\hspace*{\fill}
\finalhyphendemerits=0
[27th September]}

\index{Spellowfields , King's Lynn and West Norfolk@Spellowfields, \emph{King's Lynn \& W. Norfolk}}

Death of David Harwood (C).

\noindent
\begin{tabular*}{\columnwidth}{@{\extracolsep{\fill}} p{0.545\columnwidth} >{\itshape}l r @{\extracolsep{\fill}}}
Sheila Young & C & 348\\
Ken Hubbard & Lab & 243\\
Michael Stone & UKIP & 88\\
Rob Archer & Grn & 61\\
Ian Swinton & LD & 22\\
\end{tabular*}

\council{North Norfolk}

\subsubsection*{Waterside \hspace*{\fill}\nolinebreak[1]%
\enspace\hspace*{\fill}
\finalhyphendemerits=0
[26th April]}

\index{Waterside , North Norfolk@Waterside, \emph{N. Norfolk}}

Resignation of Simon Partridge (LD).

\noindent
\begin{tabular*}{\columnwidth}{@{\extracolsep{\fill}} p{0.545\columnwidth} >{\itshape}l r @{\extracolsep{\fill}}}
Paul Williams & LD & 494\\
Paul Rice & C & 420\\
Denise Burke & Lab & 246\\
Jeff Parkes & UKIP & 233\\
Anne Filgate & Grn & 73\\
Jean Partridge & Ind & 69\\
\end{tabular*}

\council{Norwich}

\subsubsection*{Catton Grove \hspace*{\fill}\nolinebreak[1]%
\enspace\hspace*{\fill}
\finalhyphendemerits=0
[3rd May]}

\index{Catton Grove , Norwich@Catton Grove, \emph{Norwich}}

Resignation of Julie Westmacott (Lab).

Combined with the 2012 ordinary election.
%; see page \pageref{CattonGroveNorwich} for the result.

\subsubsection*{Wensum \hspace*{\fill}\nolinebreak[1]%
\enspace\hspace*{\fill}
\finalhyphendemerits=0
[3rd May]}

\index{Wensum , Norwich@Wensum, \emph{Norwich}}

Resignation of Steven Altman (Grn).

Combined with the 2012 ordinary election.
%; see page \pageref{WensumNorwich} for the result.

\subsubsection*{Crome \hspace*{\fill}\nolinebreak[1]%
\enspace\hspace*{\fill}
\finalhyphendemerits=0
[Wednesday 19th December]}

\index{Crome , Norwich@Crome, \emph{Norwich}}

Resignation of Jenny Lay (Lab).

\noindent
\begin{tabular*}{\columnwidth}{@{\extracolsep{\fill}} p{0.545\columnwidth} >{\itshape}l r @{\extracolsep{\fill}}}
Marion Maxwell & Lab & 884\\
Evelyn Collishaw & C & 259\\
Glenn Tingle & UKIP & 232\\
Judith Ford & Grn & 73\\
Michael Sutton-Croft & LD & 42\\
\end{tabular*}

\subsubsection*{Nelson \hspace*{\fill}\nolinebreak[1]%
\enspace\hspace*{\fill}
\finalhyphendemerits=0
[Wednesday 19th December]}

\index{Nelson , Norwich@Nelson, \emph{Norwich}}

Resignation of David Rogers (Grn).

\noindent
\begin{tabular*}{\columnwidth}{@{\extracolsep{\fill}} p{0.545\columnwidth} >{\itshape}l r @{\extracolsep{\fill}}}
Andrew Boswell & Grn & 1121\\
Layla Dickerson & Lab & 599\\
Helen Whitworth & LD & 174\\
Alexandra Davies & C & 108\\
\end{tabular*}



\section{North Yorkshire}

\subsection*{County Council}\index{North Yorkshire}

\subsubsection*{Central Richmondshire \hspace*{\fill}\nolinebreak[1]%
\enspace\hspace*{\fill}
\finalhyphendemerits=0
[3rd May; Ind gain from C]}

\index{Central Richmondshire , North Yorkshire@Central Richmondshire, \emph{N. Yorks.}}

Resignation of Melva Steckles (C).

\noindent
\begin{tabular*}{\columnwidth}{@{\extracolsep{\fill}} p{0.545\columnwidth} >{\itshape}l r @{\extracolsep{\fill}}}
Helen Grant & Ind & 588\\
Steph Todd & C & 553\\
Eric Beechey & Lab & 205\\
\end{tabular*}

\council{Craven}

\subsubsection*{Hellifield and Long Preston \hspace*{\fill}\nolinebreak[1]%
\enspace\hspace*{\fill}
\finalhyphendemerits=0
[3rd May]}

\index{Hellifield and Long Preston , Craven@Hellifield \& Long Preston, \emph{Craven}}

Resignation of Helen Firth (C).

Combined with the 2012 ordinary election.
%; see page \pageref{HellifieldLongPrestonCraven} for the result.

\council{Harrogate}

\subsubsection*{Rossett \hspace*{\fill}\nolinebreak[1]%
\enspace\hspace*{\fill}
\finalhyphendemerits=0
[15th November; LD gain from C]}

\index{Rossett , Harrogate@Rossett, \emph{Harrogate}}

Resignation of Michelle Woolley (C).

\noindent
\begin{tabular*}{\columnwidth}{@{\extracolsep{\fill}} p{0.545\columnwidth} >{\itshape}l r @{\extracolsep{\fill}}}
David Siddans & LD & 807\\
Rebecca Burnett & C & 704\\
Salvina Bashforth & UKIP & 127\\
Patricia Foxall & Lab & 106\\
\end{tabular*}

\subsubsection*{Bilton \hspace*{\fill}\nolinebreak[1]%
\enspace\hspace*{\fill}
\finalhyphendemerits=0
[13th December; LD gain from C]}

\index{Bilton , Harrogate@Bilton, \emph{Harrogate}}

Resignation of Alec Brown (C).

\noindent
\begin{tabular*}{\columnwidth}{@{\extracolsep{\fill}} p{0.545\columnwidth} >{\itshape}l r @{\extracolsep{\fill}}}
Val Rodgers & LD & 623\\
Neil Bentley & C & 395\\
Andrew Gray & Lab & 208\\
David Simister & UKIP & 127\\
\end{tabular*}



\council{Middlesbrough}

\subsubsection*{North Ormesby and Brambles Farm \hspace*{\fill}\nolinebreak[1]%
\enspace\hspace*{\fill}
\finalhyphendemerits=0
[27th September]}

\index{North Ormesby and Brambles Farm , Middlesbrough@North Ormesby \& Brambles Farm, \emph{Middlesbrough}}

Resignation of Eleanor Lancaster (Lab).

\noindent
\begin{tabular*}{\columnwidth}{@{\extracolsep{\fill}} p{0.545\columnwidth} >{\itshape}l r @{\extracolsep{\fill}}}
Derek Loughborough & Lab & 471\\
Martin Brown & LD & 109\\
Stephen Riley & Ind & 71\\
Val Beadnall & C & 38\\
Daud Bashir & Ind & 6\\
\end{tabular*}



\council{Redcar and Cleveland}

\subsubsection*{Newcomen \hspace*{\fill}\nolinebreak[1]%
\enspace\hspace*{\fill}
\finalhyphendemerits=0
[19th January; Lab gain from LD]}

\index{Newcomen , Redcar and Cleveland@Newcomen, \emph{Redcar \& Cleveland}}

Death of Glynis Abbott (LD).

\noindent
\begin{tabular*}{\columnwidth}{@{\extracolsep{\fill}} p{0.545\columnwidth} >{\itshape}l r @{\extracolsep{\fill}}}
John Hannon & Lab & 539\\
Dave Stones & LD & 484\\
Matthew Bennett & C & 76\\
\end{tabular*}



\council{Richmondshire}

\subsubsection*{Hornby Castle \hspace*{\fill}\nolinebreak[1]%
\enspace\hspace*{\fill}
\finalhyphendemerits=0
[3rd May]}

\index{Hornby Castle , Richmondshire@Hornby Castle, \emph{Richmondshire}}

Resignation of Melva Steckles (C).

\noindent
\begin{tabular*}{\columnwidth}{@{\extracolsep{\fill}} p{0.545\columnwidth} >{\itshape}l r @{\extracolsep{\fill}}}
Lin Clarkson & C & 204\\
Helen Grant & Ind & 147\\
Colin Anderson & Lab & 45\\
\end{tabular*}



\council{Ryedale}

\subsubsection*{Norton West \hspace*{\fill}\nolinebreak[1]%
\enspace\hspace*{\fill}
\finalhyphendemerits=0
[15th November; LD gain from C]}

\index{Norton West , Ryedale@Norton W., \emph{Ryedale}}

Death of Judith Denniss (C).

\noindent
\begin{tabular*}{\columnwidth}{@{\extracolsep{\fill}} p{0.545\columnwidth} >{\itshape}l r @{\extracolsep{\fill}}}
Dinah Keal & LD & 381\\
Paul Farndale & C & 181\\
Karl Reveley & Lab & 75\\
\end{tabular*}

\council{Scarborough}

\subsubsection*{Hertford \hspace*{\fill}\nolinebreak[1]%
\enspace\hspace*{\fill}
\finalhyphendemerits=0
[15th March; C gain from Grn]}

\index{Hertford , Scarborough@Hertford, \emph{Scarborough}}

Resignation of Nick Harvey (Grn).

\noindent
\begin{tabular*}{\columnwidth}{@{\extracolsep{\fill}} p{0.545\columnwidth} >{\itshape}l r @{\extracolsep{\fill}}}
\sloppyword{Michelle Donohue-Moncrieff} & C & 663\\
Vanda Inman & Lab & 208\\
Michael James & UKIP & 126\\
Bob Jackman & LD & 99\\
\end{tabular*}

\subsubsection*{Esk Valley \hspace*{\fill}\nolinebreak[1]%
\enspace\hspace*{\fill}
\finalhyphendemerits=0
[20th September]}

\index{Esk Valley , Scarborough@Esk Valley, \emph{Scarborough}}

Resignation of James Preston (C).

\noindent
\begin{tabular*}{\columnwidth}{@{\extracolsep{\fill}} p{0.545\columnwidth} >{\itshape}l r @{\extracolsep{\fill}}}
Guy Coulson & C & 606\\
Mike Ward & Ind & 151\\
Simon Parkes & Lab & 87\\
Michael James & UKIP & 35\\
Ed Scott & EDP & 18\\
\end{tabular*}



\section{Northamptonshire}

\subsection*{County Council}\index{Northamptonshire}

\subsubsection*{Towcester \hspace*{\fill}\nolinebreak[1]%
\enspace\hspace*{\fill}
\finalhyphendemerits=0
[9th February; LD gain from C]}

\index{Towcester , Northamptonshire@Towcester, \emph{Northants.}}

Resignation of Rosemary Bromwich (C).

\noindent
\begin{tabular*}{\columnwidth}{@{\extracolsep{\fill}} p{0.545\columnwidth} >{\itshape}l r @{\extracolsep{\fill}}}
Chris Lofts & LD & 1279\\
Ian McCord & C & 638\\
Barry Mahoney & UKIP & 124\\
Mark Plowman & BNP & 66\\
\end{tabular*}

\council{Corby}

\subsubsection*{East \hspace*{\fill}\nolinebreak[1]%
\enspace\hspace*{\fill}
\finalhyphendemerits=0
[12th July]}

\index{East , Corby@East, \emph{Corby}}

Death of Pat Fawcett (Lab).

\noindent
\begin{tabular*}{\columnwidth}{@{\extracolsep{\fill}} p{0.545\columnwidth} >{\itshape}l r @{\extracolsep{\fill}}}
Seán Kettle & Lab & 1063\\
Kevin Watt & C & 252\\
Gordon Riddell & BNP & 141\\
Julie Grant & LD & 37\\
\end{tabular*}

\council{Daventry}

\subsubsection*{Brixworth \hspace*{\fill}\nolinebreak[1]%
\enspace\hspace*{\fill}
\finalhyphendemerits=0
[15th November]}

\index{Brixworth , Daventry@Brixworth, \emph{Daventry}}

Death of Frank Wiig (C).

\noindent
\begin{tabular*}{\columnwidth}{@{\extracolsep{\fill}} p{0.545\columnwidth} >{\itshape}l r @{\extracolsep{\fill}}}
Stephen Pointer & C & 857\\
Steve Whiffen & Grn & 484\\
\end{tabular*}

\council{East Northamptonshire}

\subsubsection*{Barnwell \hspace*{\fill}\nolinebreak[1]%
\enspace\hspace*{\fill}
\finalhyphendemerits=0
[14th June]}

\index{Barnwell , East Northamptonshire@Barnwell, \emph{E. Northants.}}

Death of Philip Hardcastle (C).

\noindent
\begin{tabular*}{\columnwidth}{@{\extracolsep{\fill}} p{0.545\columnwidth} >{\itshape}l r @{\extracolsep{\fill}}}
Derek Capp & C & 412\\
Karen Draycott & LD & 130\\
Phil Garnham & Lab & 101\\
\end{tabular*}

\subsubsection*{Oundle \hspace*{\fill}\nolinebreak[1]%
\enspace\hspace*{\fill}
\finalhyphendemerits=0
[15th November]}

\index{Oundle , East Northamptonshire@Oundle, \emph{E. Northants.}}

Resignation of David Bateman (C).

\noindent
\begin{tabular*}{\columnwidth}{@{\extracolsep{\fill}} p{0.545\columnwidth} >{\itshape}l r @{\extracolsep{\fill}}}
Jake Vowles & C & 1003\\
Paul King & Lab & 681\\
George Smid & LD & 230\\
\end{tabular*}

\council{South Northamptonshire}

\subsubsection*{Towcester Brook \hspace*{\fill}\nolinebreak[1]%
\enspace\hspace*{\fill}
\finalhyphendemerits=0
[9th February; LD gain from C]}

\index{Towcester Brook , South Northamptonshire@Towcester Brook, \emph{S. Northants.}}

Death of Diana Dallyn (C).

\noindent
\begin{tabular*}{\columnwidth}{@{\extracolsep{\fill}} p{0.545\columnwidth} >{\itshape}l r @{\extracolsep{\fill}}}
Lisa Samiotis & LD & 774\\
John Gasking & C & 401\\
Peter Conquest & UKIP & 129\\
\end{tabular*}

\subsubsection*{Grange Park \hspace*{\fill}\nolinebreak[1]%
\enspace\hspace*{\fill}
\finalhyphendemerits=0
[28th June]}

\index{Grange Park , South Northamptonshire@Grange Park, \emph{S. Northants.}}

Death of Paul Farrow (C).

\noindent
\begin{tabular*}{\columnwidth}{@{\extracolsep{\fill}} p{0.545\columnwidth} >{\itshape}l r @{\extracolsep{\fill}}}
Tharik Jainu-Deen & C & 313\\
Shaun Hope & LD & 98\\
\end{tabular*}

\columnbreak

\section{Nottinghamshire}

\subsection*{County Council}\index{Nottinghamshire}

NottsInd = Nottinghamshire Independents

\subsubsection*{Chilwell and Toton \hspace*{\fill}\nolinebreak[1]%
\enspace\hspace*{\fill}
\finalhyphendemerits=0
[15th March]}

\index{Chilwell and Toton , Nottinghamshire@Chilwell \& Toton, \emph{Notts.}}

Death of Tom Pettengell (C).

\noindent
\begin{tabular*}{\columnwidth}{@{\extracolsep{\fill}} p{0.545\columnwidth} >{\itshape}l r @{\extracolsep{\fill}}}
John Doddy & C & 1958\\
David Watts & LD & 1375\\
Lee Waters & UKIP & 682\\
\end{tabular*}

\subsubsection*{Rufford \hspace*{\fill}\nolinebreak[1]%
\enspace\hspace*{\fill}
\finalhyphendemerits=0
[20th September; Lab gain from Ind]}

\index{Rufford , Nottinghamshire@Rufford, \emph{Notts.}}

Death of Les Ward (Ind).

\noindent
\begin{tabular*}{\columnwidth}{@{\extracolsep{\fill}} p{0.48\columnwidth} >{\itshape}l r @{\extracolsep{\fill}}}
John Peck & Lab & 1557\\
Daniel Mottishaw & C & 660\\
Jim Gregson & NottsInd & 346\\
Carole Terzza & UKIP & 123\\
\end{tabular*}

\council{Broxtowe}

\subsubsection*{Toton and Chilwell Meadows \hspace*{\fill}\nolinebreak[1]%
\enspace\hspace*{\fill}
\finalhyphendemerits=0
[15th March]}

\index{Toton and Chilwell Meadows , Broxtowe@Toton \& Chilwell Meadows, \emph{Broxtowe}}

Resignation of Craig Cox (C).

\noindent
\begin{tabular*}{\columnwidth}{@{\extracolsep{\fill}} p{0.545\columnwidth} >{\itshape}l r @{\extracolsep{\fill}}}
Halimah Khaled & C & 831\\
Jane Marshall & Lab & 385\\
Barbara Carr & LD & 300\\
Keith Marriott & UKIP & 228\\
\end{tabular*}

\council{Newark and Sherwood}

\subsubsection*{Lowdham \hspace*{\fill}\nolinebreak[1]%
\enspace\hspace*{\fill}
\finalhyphendemerits=0
[12th April]}

\index{Lowdham , Newark and Sherwood@Lowdham, \emph{Newark \& Sherwood}}

Death of Keith Sheppard (C).

\noindent
\begin{tabular*}{\columnwidth}{@{\extracolsep{\fill}} p{0.545\columnwidth} >{\itshape}l r @{\extracolsep{\fill}}}
Tim Wendels & C & 787\\
William Davison & LD & 534\\
Daniel Hibberd & Lab & 117\\
Tim Cutler & Ind & 91\\
\end{tabular*}



\section{Oxfordshire}

\subsection*{County Council}\index{Oxfordshire}

\subsubsection*{Watlington \hspace*{\fill}\nolinebreak[1]%
\enspace\hspace*{\fill}
\finalhyphendemerits=0
[19th April]}

\index{Watlington , Oxfordshire@Watlington, \emph{Oxon.}}

Death of Roger Belson (C).

\noindent
\begin{tabular*}{\columnwidth}{@{\extracolsep{\fill}} p{0.545\columnwidth} >{\itshape}l r @{\extracolsep{\fill}}}
Caroline Newton & C & 865\\
Nicholas Hancock & LD & 259\\
James Merritt & Lab & 157\\
Jonathan Kent & UKIP & 110\\
\end{tabular*}

\subsubsection*{Cowley and Littlemore \hspace*{\fill}\nolinebreak[1]%
\enspace\hspace*{\fill}
\finalhyphendemerits=0
[12th July]}

\index{Cowley and Littlemore , Oxfordshire@Cowley \& Littlemore, \emph{Oxon.}}

Resignation of Sarah Hutchinson (Lab).

\noindent
\begin{tabular*}{\columnwidth}{@{\extracolsep{\fill}} p{0.545\columnwidth} >{\itshape}l r @{\extracolsep{\fill}}}
Gill Sanders & Lab & 1606\\
Judith Harley & C & 384\\
Paul Skinner & Grn & 330\\
Mike Tait & LD & 167\\
\end{tabular*}

\council{Cherwell}

\subsubsection*{Banbury Ruscote \hspace*{\fill}\nolinebreak[1]%
\enspace\hspace*{\fill}
\finalhyphendemerits=0
[15th November]}

\index{Banbury Ruscote , Cherwell@Banbury Ruscote, \emph{Cherwell}}

Resignation of George Parish (Lab).

\noindent
\begin{tabular*}{\columnwidth}{@{\extracolsep{\fill}} p{0.545\columnwidth} >{\itshape}l r @{\extracolsep{\fill}}}
Gordon Ross & Lab & 611\\
Pat Tompson & C & 349\\
David Burton & UKIP & 117\\
\end{tabular*}

\council{South Oxfordshire}

\subsubsection*{Chinnor \hspace*{\fill}\nolinebreak[1]%
\enspace\hspace*{\fill}
\finalhyphendemerits=0
[10th May]}

\index{Chinnor , South Oxfordshire@Chinnor, \emph{S. Oxon.}}

Resignation of Geoff Andrews (C).

\noindent
\begin{tabular*}{\columnwidth}{@{\extracolsep{\fill}} p{0.545\columnwidth} >{\itshape}l r @{\extracolsep{\fill}}}
Lynn Lloyd & C & 591\\
Martin Wright & Ind & 449\\
Simon Stone & Lab & 184\\
\end{tabular*}

\subsubsection*{Didcot All Saints \hspace*{\fill}\nolinebreak[1]%
\enspace\hspace*{\fill}
\finalhyphendemerits=0
[25th October]}

\index{Didcot All Saints , South Oxfordshire@Didcot All SS., \emph{S. Oxon.}}

Resignation of Terry Joslin (Lab).

\noindent
\begin{tabular*}{\columnwidth}{@{\extracolsep{\fill}} p{0.545\columnwidth} >{\itshape}l r @{\extracolsep{\fill}}}
Denise MacDonald & Lab & 436\\
Jane Murphy & C & 340\\
Andrew Jones & LD & 151\\
\end{tabular*}

\council{Vale of White Horse}

\subsubsection*{Sunningwell and Wootton (2) \hspace*{\fill}\nolinebreak[1]%
\enspace\hspace*{\fill}
\finalhyphendemerits=0
[6th December; 2 LD gains from C]}

\index{Sunningwell and Wootton , Vale of White Horse@Sunningwell \& Wootton, \emph{Vale of White Horse}}

Resignation of Jane Crossley and disqualification (non-attendance) of Tim Foggin (both C).

\noindent
\begin{tabular*}{\columnwidth}{@{\extracolsep{\fill}} p{0.545\columnwidth} >{\itshape}l r @{\extracolsep{\fill}}}
Valerie Shaw & LD & 577\\
Elizabeth Miles & LD & 549\\
Adam Hardiman & C & 346\\
Richard Treffler & C & 333\\
\end{tabular*}



\section{Rutland}

\subsubsection*{Uppingham \hspace*{\fill}\nolinebreak[1]%
\enspace\hspace*{\fill}
\finalhyphendemerits=0
[26th January; C gain from Ind]}

\index{Uppingham , Rutland@Uppingham, \emph{Rutland}}

Death of Colin Forsyth (Ind).

\noindent
\begin{tabular*}{\columnwidth}{@{\extracolsep{\fill}} p{0.545\columnwidth} >{\itshape}l r @{\extracolsep{\fill}}}
Carolyn Cartwright & C & 424\\
Julie Park & Lab & 211\\
Peter Golden & LD & 200\\
\end{tabular*}



\section{Shropshire}

\council{Shropshire}

\subsubsection*{Abbey \hspace*{\fill}\nolinebreak[1]%
\enspace\hspace*{\fill}
\finalhyphendemerits=0
[1st March; LD gain from C]}

\index{Abbey , Shropshire@Abbey, \emph{Shrops.}}

Resignation of Josephine Jones (C).

\noindent
\begin{tabular*}{\columnwidth}{@{\extracolsep{\fill}} p{0.5\columnwidth} >{\itshape}l r @{\extracolsep{\fill}}}
Hannah Fraser & LD & 550\\
Peter Wright & C & 542\\
John Brown & Grn & 122\\
\end{tabular*}

\subsubsection*{Church Stretton and Craven Arms \hspace*{\fill}\nolinebreak[1]%
\enspace\hspace*{\fill}
\finalhyphendemerits=0
[13th September]}

\index{Church Stretton and Craven Arms , Shropshire@Church Stretton \& Craven Arms, \emph{Shrops.}}

Death of James Gibson (C).

\noindent
\begin{tabular*}{\columnwidth}{@{\extracolsep{\fill}} p{0.5\columnwidth} >{\itshape}l r @{\extracolsep{\fill}}}
Lee Chapman & C & 1216\\
Bob Welch & LD & 969\\
Clive Leworthy & Lab & 529\\
\end{tabular*}



\section{Somerset}

\council{Bath and North East Somerset}

\subsubsection*{Chew Valley North \hspace*{\fill}\nolinebreak[1]%
\enspace\hspace*{\fill}
\finalhyphendemerits=0
[15th November]}

\index{Chew Valley North , Bath and North East Somerset@Chew Valley N., \emph{Bath \& N.E. Somerset}}

Resignation of Malcolm Hanney (C).

\noindent
\begin{tabular*}{\columnwidth}{@{\extracolsep{\fill}} p{0.5\columnwidth} >{\itshape}l r @{\extracolsep{\fill}}}
Liz Richardson & C & 417\\
Charles Fenn & LD & 271\\
Andrew Tanner & Ind & 106\\
Michael Jay & Grn & 26\\
\end{tabular*}



\section{Staffordshire}

Moorlnds = Moorlands Democratic Alliance

\subsection*{County Council}\index{Staffordshire}

\subsubsection*{Leek South \hspace*{\fill}\nolinebreak[1]%
\enspace\hspace*{\fill}
\finalhyphendemerits=0
[9th February; C gain from UKIP]}

\index{Leek South , Staffordshire@Leek S., \emph{Staffs.}}

Death of Steve Povey (UKIP).

\noindent
\begin{tabular*}{\columnwidth}{@{\extracolsep{\fill}} p{0.5\columnwidth} >{\itshape}l r @{\extracolsep{\fill}}}
Neal Podmore & C & 725\\
Alex Povey & UKIP & 556\\
Margaret Lovatt & Lab & 432\\
John Fisher & LD & 419\\
Pamela Wood & Moorlnds & 336\\
Bill Cawley & Ind & 192\\
\end{tabular*}



\council{Cannock Chase}

\subsubsection*{Hagley \hspace*{\fill}\nolinebreak[1]%
\enspace\hspace*{\fill}
\finalhyphendemerits=0
[3rd May]}

\index{Hagley , Cannock Chase@Hagley , \emph{Cannock Chase}}

Resignation of Gordon Brown (Lab).

Combined with the 2012 ordinary election.
%; see page \pageref{HagleyCannockChase} for the result.



\council{Newcastle-under-Lyme}

\subsubsection*{Madeley \hspace*{\fill}\nolinebreak[1]%
\enspace\hspace*{\fill}
\finalhyphendemerits=0
[2nd February; LD gain from Lab]}

\index{Madeley , Newcastle-under-Lyme@Madeley, \emph{Newcastle-under-Lyme}}

Death of Bill Sinnott (Lab).

\noindent
\begin{tabular*}{\columnwidth}{@{\extracolsep{\fill}} p{0.545\columnwidth} >{\itshape}l r @{\extracolsep{\fill}}}
Simon White & LD & 617\\
John Smart & Lab & 342\\
Howard Goodall & C & 294\\
Elaine Blake & UKIP & 41\\
\end{tabular*}

\subsubsection*{Kidsgrove \hspace*{\fill}\nolinebreak[1]%
\enspace\hspace*{\fill}
\finalhyphendemerits=0
[10th May]}

\index{Kidsgrove , Newcastle-under-Lyme@Kidsgrove, \emph{Newcastle-under-Lyme}}

Disqualification (sentenced to six months in prison, assault) of Kyle-Noel Taylor (Lab).

\noindent
\begin{tabular*}{\columnwidth}{@{\extracolsep{\fill}} p{0.545\columnwidth} >{\itshape}l r @{\extracolsep{\fill}}}
Terry Turner & Lab & 639\\
Linda Harold & C & 121\\
Phil Crank & LD & 105\\
Claire Vodrey & TUSC & 73\\
\end{tabular*}



\council{South Staffordshire}

\subsubsection*{\sloppyword{Wombourne North and Lower Penn} \hspace*{\fill}\nolinebreak[1]%
\enspace\hspace*{\fill}
\finalhyphendemerits=0
[6th December]}

\index{Wombourne North and Lower Penn , South Staffordshire@Wombourne N. \& Lower Penn, \emph{S. Staffs.}}

Death of Joan Williams (C).

\noindent
\begin{tabular*}{\columnwidth}{@{\extracolsep{\fill}} p{0.5\columnwidth} >{\itshape}l r @{\extracolsep{\fill}}}
Barry Bond & C & 404\\
Lyndon Jones & UKIP & 182\\
John Brindle & Lab & 95\\
\end{tabular*}



\council{Stafford}

\subsubsection*{Rowley \hspace*{\fill}\nolinebreak[1]%
\enspace\hspace*{\fill}
\finalhyphendemerits=0
[8th March]}

\index{Rowley , Stafford@Rowley, \emph{Stafford}}

Death of David Allan (C).

\noindent
\begin{tabular*}{\columnwidth}{@{\extracolsep{\fill}} p{0.5\columnwidth} >{\itshape}l r @{\extracolsep{\fill}}}
Anne Hobbs & Lab & 620\\
Violet Allan & C & 540\\
Kate Harding & Grn & 67\\
Malcolm Hurst & UKIP & 61\\
\end{tabular*}

\council{Staffordshire Moorlands}

\subsubsection*{Leek North \hspace*{\fill}\nolinebreak[1]%
\enspace\hspace*{\fill}
\finalhyphendemerits=0
[9th February; Lab gain from UKIP]}

\index{Leek North , Staffordshire Moorlands@Leek N., \emph{Staffs. Moorlands}}

Death of Steve Povey (UKIP).

\noindent
\begin{tabular*}{\columnwidth}{@{\extracolsep{\fill}} p{0.5\columnwidth} >{\itshape}l r @{\extracolsep{\fill}}}
Charlotte Atkins & Lab & 514\\
Alex Povey & UKIP & 188\\
Bob Bestwick & C & 178\\
Brian Pointon & Moorlnds & 33\\
Roy Gregg & LD & 24\\
\end{tabular*}

\council{Stoke-on-Trent}

CityInd = City Independents

DemNat = Democratic Nationalists

\subsubsection*{Springfields and Trent Vale \hspace*{\fill}\nolinebreak[1]%
\enspace\hspace*{\fill}
\finalhyphendemerits=0
[26th July]}

\index{Springfields and Trent Vale , Stoke-on-Trent@Springfields \& Trent Vale, \emph{Stoke-on-Trent}}

Resignation of Sarah Hill (Lab).

\noindent
\begin{tabular*}{\columnwidth}{@{\extracolsep{\fill}} p{0.5\columnwidth} >{\itshape}l r @{\extracolsep{\fill}}}
Jackie Barnes & CityInd & 370\\
Mubsira Aumir & Lab & 245\\
Les Porch & LD & 152\\
Harold Gregory & C & 109\\
Michael Bednarski & UKIP & 105\\
Gary Elsby & Ind & 36\\
Michael Coleman & BNP & 27\\
Matt Wright & TUSC & 14\\
Mark Leat & DemNat & 2\\
\end{tabular*}

\council{Tamworth}

At the May 2012 ordinary election there was an unfilled vacancy in Mercian ward due to the death of Sam Munn (C).
\index{Mercian , Tamworth@Mercian, \emph{Tamworth}}



\section{Suffolk}

\subsection*{County Council}\index{Suffolk}

\subsubsection*{Kesgrave and Rushmere St Andrew \hspace*{\fill}\nolinebreak[1]%
\enspace\hspace*{\fill}
\finalhyphendemerits=0
[9th February]}

\index{Kesgrave and Rushmere Saint Andrew , Suffolk@Kesgrave \& Rushmere St Andrew, \emph{Suffolk}}

Death of John Klaschka (C).

\noindent
\begin{tabular*}{\columnwidth}{@{\extracolsep{\fill}} p{0.545\columnwidth} >{\itshape}l r @{\extracolsep{\fill}}}
Christopher Hudson & C & 1302\\
Kevin Archer & Lab & 804\\
Derrick Fairbrother & LD & 514\\
\end{tabular*}

\subsubsection*{Bixley \hspace*{\fill}\nolinebreak[1]%
\enspace\hspace*{\fill}
\finalhyphendemerits=0
[3rd May]}

\index{Bixley , Suffolk@Bixley, \emph{Suffolk}}

Death of Russell Harsant (C).

\noindent
\begin{tabular*}{\columnwidth}{@{\extracolsep{\fill}} p{0.545\columnwidth} >{\itshape}l r @{\extracolsep{\fill}}}
Alan Murray & C & 866\\
John Cook & Lab & 566\\
Chris Streatfield & UKIP & 229\\
Barry Broom & Grn & 189\\
Peter Bagnall & LD & 127\\
\end{tabular*}

\council{Ipswich}

\subsubsection*{Bixley \hspace*{\fill}\nolinebreak[1]%
\enspace\hspace*{\fill}
\finalhyphendemerits=0
[3rd May]}

\index{Bixley , Ipswich@Bixley, \emph{Ipswich}}

Death of Russell Harsant (C).

Combined with the 2012 ordinary election.
%; see page \pageref{BixleyIpswich} for the result.

\subsection*{St Edmundsbury}\index{Saint Edmundsbury@St Edmundsbury}

\subsubsection*{Risbygate \hspace*{\fill}\nolinebreak[1]%
\enspace\hspace*{\fill}
\finalhyphendemerits=0
[15th November; Grn gain from C]}

\index{Risbygate , Saint Edmundsbury@Risbygate, \emph{St Edmundsbury}}

Resignation of Josh Hordern (C).

\noindent
\begin{tabular*}{\columnwidth}{@{\extracolsep{\fill}} p{0.545\columnwidth} >{\itshape}l r @{\extracolsep{\fill}}}
Julia Wakelam & Grn & 394\\
Susan Glossop & C & 239\\
Cliff Hind & Lab & 128\\
\end{tabular*}

\council{Suffolk Coastal}

\subsubsection*{Kesgrave East \hspace*{\fill}\nolinebreak[1]%
\enspace\hspace*{\fill}
\finalhyphendemerits=0
[9th February]}

\index{Kesgrave East , Suffolk Coastal@Kesgrave E., \emph{Suffolk Coastal}}

Death of John Klaschka (C).

\noindent
\begin{tabular*}{\columnwidth}{@{\extracolsep{\fill}} p{0.545\columnwidth} >{\itshape}l r @{\extracolsep{\fill}}}
Geoff Lynch & C & 531\\
David Isaacs & Lab & 241\\
Derrick Fairbrother & LD & 194\\
\end{tabular*}

\council{Waveney}

\subsubsection*{Beccles South \hspace*{\fill}\nolinebreak[1]%
\enspace\hspace*{\fill}
\finalhyphendemerits=0
[15th November; C gain from Lab]}

\index{Beccles South , Waveney@Beccles S., \emph{Waveney}}

Resignation of Alan Thwaites (Lab).

\noindent
\begin{tabular*}{\columnwidth}{@{\extracolsep{\fill}} p{0.545\columnwidth} >{\itshape}l r @{\extracolsep{\fill}}}
Graham Catchpole & C & 520\\
Nicky Elliott & Grn & 390\\
Alan Green & Lab & 369\\
Doug Farmer & LD & 35\\
\end{tabular*}



\section{Surrey}

\subsection*{County Council}\index{Surrey}

\subsubsection*{Worplesdon \hspace*{\fill}\nolinebreak[1]%
\enspace\hspace*{\fill}
\finalhyphendemerits=0
[3rd May]}

\index{Worplesdon , Surrey@Worplesdon, \emph{Surrey}}

Resignation of Nigel Sutcliffe (C).

\noindent
\begin{tabular*}{\columnwidth}{@{\extracolsep{\fill}} p{0.545\columnwidth} >{\itshape}l r @{\extracolsep{\fill}}}
Keith Witham & C & 2022\\
Paul Cragg & LD & 1236\\
Martin Phillips & Lab & 517\\
\end{tabular*}

\council{Elmbridge}

Esher = Esher Residents Association

\subsubsection*{Esher \hspace*{\fill}\nolinebreak[1]%
\enspace\hspace*{\fill}
\finalhyphendemerits=0
[21st June]}

\index{Esher , Elmbridge@Esher, \emph{Elmbridge}}

Ordinary election postponed from 3rd May: death of candidate Bruce King (Lab).

\noindent
\begin{tabular*}{\columnwidth}{@{\extracolsep{\fill}} p{0.545\columnwidth} >{\itshape}l r @{\extracolsep{\fill}}}
Tim Oliver & C & 711\\
Gary Lay & Esher & 665\\
Mick Moriarty & Lab & 91\\
Denis Hill & UKIP & 30\\
\end{tabular*}

%See page \pageref{EsherElmbridge} for the result.

\council{Runnymede}

\subsubsection*{Chertsey Meads \hspace*{\fill}\nolinebreak[1]%
\enspace\hspace*{\fill}
\finalhyphendemerits=0
[27th September]}

\index{Chertsey Meads , Runnymede@Chertsey Meads, \emph{Runnymede}}

Death of Diane Cotty (C).

\noindent
\begin{tabular*}{\columnwidth}{@{\extracolsep{\fill}} p{0.545\columnwidth} >{\itshape}l r @{\extracolsep{\fill}}}
Peter Boast & C & 450\\
Chris Browne & UKIP & 312\\
Doug Scott & Lab & 312\\
Andy Watson & LD & 34\\
Crazy Crab & Loony & 10\\
\end{tabular*}

\subsubsection*{New Haw \hspace*{\fill}\nolinebreak[1]%
\enspace\hspace*{\fill}
\finalhyphendemerits=0
[27th September]}

\index{New Haw , Runnymede@New Haw, \emph{Runnymede}}

Death of Chris Knight (C).

\noindent
\begin{tabular*}{\columnwidth}{@{\extracolsep{\fill}} p{0.545\columnwidth} >{\itshape}l r @{\extracolsep{\fill}}}
Mark Maddox & C & 346\\
David Bell & Lab & 148\\
Graham Wood & UKIP & 124\\
Jennifer Coulon & LD & 54\\
\end{tabular*}

\council{Spelthorne}

\subsubsection*{Sunbury Common \hspace*{\fill}\nolinebreak[1]%
\enspace\hspace*{\fill}
\finalhyphendemerits=0
[Wednesday 19th December]}

\index{Sunbury Common , Spelthorne@Sunbury Common, \emph{Spelthorne}}

\sloppyword{Resignation of Robbie Collison-Crawford (LD).}

\noindent
\begin{tabular*}{\columnwidth}{@{\extracolsep{\fill}} p{0.545\columnwidth} >{\itshape}l r @{\extracolsep{\fill}}}
Bernie Spoor & LD & 372\\
Bob Bromley & UKIP & 182\\
John Went & Lab & 129\\
Matthew Want & C & 115\\
\end{tabular*}



\section{Warwickshire}

\council{Rugby}

\subsubsection*{New Bilton \hspace*{\fill}\nolinebreak[1]%
\enspace\hspace*{\fill}
\finalhyphendemerits=0
[15th November]}

\index{New Bilton , Rugby@New Bilton, \emph{Rugby}}

Resignation of Robert McNally (Lab).

\noindent
\begin{tabular*}{\columnwidth}{@{\extracolsep{\fill}} p{0.545\columnwidth} >{\itshape}l r @{\extracolsep{\fill}}}
Steve Birkett & Lab & 496\\
Katie Ferrier & C & 192\\
Roy Sandison & Grn & 100\\
Roy Harvey & UKIP & 82\\
Patricia Wyatt & Ind & 56\\
David Merritt & LD & 41\\
Pete McLaren & TUSC & 39\\
\end{tabular*}

\council{Stratford-on-Avon}

\subsubsection*{Shipston \hspace*{\fill}\nolinebreak[1]%
\enspace\hspace*{\fill}
\finalhyphendemerits=0
[29th November; Lab gain from C]}

\index{Shipston , Stratford-on-Avon@Shipston, \emph{Stratford-on-Avon}}

Resignation of Jon Gullis (C).

\noindent
\begin{tabular*}{\columnwidth}{@{\extracolsep{\fill}} p{0.545\columnwidth} >{\itshape}l r @{\extracolsep{\fill}}}
Jeffrey Kenner & Lab & 613\\
Laura Nelson & LD & 575\\
Marion Lowe & C & 431\\
\end{tabular*}



\section{West Sussex}

\subsection*{County Council}\index{West Sussex}

\subsubsection*{Midhurst \hspace*{\fill}\nolinebreak[1]%
\enspace\hspace*{\fill}
\finalhyphendemerits=0
[15th November]}

\index{Midhurst , West Sussex@Midhurst, \emph{W. Sussex}}

Resignation of Nola Hendon (C).

\noindent
\begin{tabular*}{\columnwidth}{@{\extracolsep{\fill}} p{0.545\columnwidth} >{\itshape}l r @{\extracolsep{\fill}}}
John Cherry & C & 1410\\
Douglas Denny & UKIP & 392\\
\end{tabular*}

\council{Adur}

At the May 2012 ordinary election there was an unfilled vacancy in Eastbrook ward due to the disqualification (non-attendance) of Gavin Ayling (LD).
\index{Eastbrook , Adur@Eastbrook, \emph{Adur}}

\council{Chichester}

\subsubsection*{Plaistow \hspace*{\fill}\nolinebreak[1]%
\enspace\hspace*{\fill}
\finalhyphendemerits=0
[19th April]}

\index{Plaistow , Chichester@Plaistow, \emph{Chichester}}

Resignation of Linda Westmore (C).

\noindent
\begin{tabular*}{\columnwidth}{@{\extracolsep{\fill}} p{0.545\columnwidth} >{\itshape}l r @{\extracolsep{\fill}}}
Nick Thomas & C & 455\\
Ray Cooper & LD & 408\\
\end{tabular*}

\council{Horsham}

\subsubsection*{Itchingfield, Slinfold and Warnham \hspace*{\fill}\nolinebreak[1]%
\enspace\hspace*{\fill}
\finalhyphendemerits=0
[16th February]}

\index{Itchingfield, Slinfold and Warnham , Horsham@Itchingfield, Slinfold \& Warnham, \emph{Horsham}}

Resignation of Robert Nye (C).

\noindent
\begin{tabular*}{\columnwidth}{@{\extracolsep{\fill}} p{0.545\columnwidth} >{\itshape}l r @{\extracolsep{\fill}}}
Stuart Ritchie & C & 557\\
Ian Shepherd & LD & 320\\
George Tribe & UKIP & 173\\
Justin Pickard & Grn & 65\\
David Hide & Lab & 60\\
\end{tabular*}

\council{Worthing}

\subsubsection*{Offington \hspace*{\fill}\nolinebreak[1]%
\enspace\hspace*{\fill}
\finalhyphendemerits=0
[3rd May]}

\index{Offington , Worthing@Offington, \emph{Worthing}}

Death of Reg Green (C).

Combined with the 2012 ordinary election.
%; see page \pageref{OffingtonWorthing} for the result.



\section{Wiltshire}

\council{Swindon}

At the May 2012 ordinary election there was an unfilled vacancy in Moredon ward due to the death of Jenny Millin (Lab).
\index{Moredon , Swindon@Moredon, \emph{Swindon}}

\subsubsection*{Blunsdon and Highworth \hspace*{\fill}\nolinebreak[1]%
\enspace\hspace*{\fill}
\finalhyphendemerits=0
[15th November]}

\index{Blunsdon and Highworth , Swindon@Blunsdon \& Highworth, \emph{Swindon}}

Death of Doreen Dart (C).

\noindent
\begin{tabular*}{\columnwidth}{@{\extracolsep{\fill}} p{0.545\columnwidth} >{\itshape}l r @{\extracolsep{\fill}}}
Steve Weisinger & C & 1453\\
Phil Beaumont & Lab & 1075\\
John Lenton & UKIP & 195\\
Andrew Day & Grn & 111\\
Cath Smith & Ind & 23\\
\end{tabular*}



\section{Worcestershire}

\council{Malvern Hills}

\subsubsection*{West \hspace*{\fill}\nolinebreak[1]%
\enspace\hspace*{\fill}
\finalhyphendemerits=0
[5th July]}

\index{West , Malvern Hills@West, \emph{Malvern Hills}}

Resignation of Luna Deller (Grn).

\noindent
\begin{tabular*}{\columnwidth}{@{\extracolsep{\fill}} p{0.545\columnwidth} >{\itshape}l r @{\extracolsep{\fill}}}
Julian Roskams & Grn & 490\\
Jennie Kelly & C & 329\\
Sarah Rouse & Ind & 130\\
Mike Savage & UKIP & 91\\
Christopher Burrows & Lab & 68\\
\end{tabular*}

\council{Redditch}

\subsubsection*{Church Hill \hspace*{\fill}\nolinebreak[1]%
\enspace\hspace*{\fill}
\finalhyphendemerits=0
[3rd May]}

\index{Church Hill , Redditch@Church Hill, \emph{Redditch}}

Death of Robin King (Lab).

Combined with the 2012 ordinary election.
%; see page \pageref{ChurchHillRedditch} for the result.



\section{Glamorgan}

\council{Bridgend}

\subsubsection*{Bettws \hspace*{\fill}\nolinebreak[1]%
\enspace\hspace*{\fill}
\finalhyphendemerits=0
[30th August]}

\index{Bettws , Bridgend@Bettws, \emph{Bridgend}}

Death of Christopher Michaelides (Lab).

\noindent
\begin{tabular*}{\columnwidth}{@{\extracolsep{\fill}} p{0.545\columnwidth} >{\itshape}l r @{\extracolsep{\fill}}}
Martyn Jones & Lab & 494\\
Catherine Jones & Ind & 96\\
Gareth Harris & Grn & 16\\
Matthew Voisey & C & 12\\
\end{tabular*}

\council{Merthyr Tydfil}

Merthyr = Merthyr Independents

\subsubsection*{Cyfarthfa \hspace*{\fill}\nolinebreak[1]%
\enspace\hspace*{\fill}
\finalhyphendemerits=0
[15th November; Lab gain from Merthyr]}

\index{Cyfarthfa , Merthyr Tydfil@Cyfarthfa, \emph{Merthyr Tydfil}}

Resignation of Les Elliott (Merthyr).

\noindent
\begin{tabular*}{\columnwidth}{@{\extracolsep{\fill}} p{0.53\columnwidth} >{\itshape}l r @{\extracolsep{\fill}}}
Margaret Davies & Lab & 385\\
Malcolm Jenkins & Merthyr & 364\\
Malcolm Eddery & Ind & 289\\
Mark Evans & PC & 101\\
Jaswinder Beghal & C & 26\\
\end{tabular*}

\council{Neath Port Talbot}

\subsubsection*{Neath South \hspace*{\fill}\nolinebreak[1]%
\enspace\hspace*{\fill}
\finalhyphendemerits=0
[6th December]}

\index{Neath South , Neath Port Talbot@Neath S., \emph{Neath Port Talbot}}

Death of Mal Gunter (Lab).

\noindent
\begin{tabular*}{\columnwidth}{@{\extracolsep{\fill}} p{0.545\columnwidth} >{\itshape}l r @{\extracolsep{\fill}}}
Andrew Jenkins & Lab & 399\\
Charlotte Cross & LD & 130\\
\end{tabular*}

\council{Swansea}

At the May 2012 ordinary election there were unfilled vacancies in Castle division due to the death of Barbara Hynes (Lab), in Penyrheol division due to the resignation of Alan Jopling (Ind) and in Townhill division due to the death of Billy Jones (Lab).
\index{Castle , Swansea@Castle, \emph{Swansea}}
\index{Penyrheol , Swansea@Penyrheol, \emph{Swansea}}
\index{Townhill , Swansea@Townhill, \emph{Swansea}}

\council{Vale of Glamorgan}

\subsubsection*{Buttrills \hspace*{\fill}\nolinebreak[1]%
\enspace\hspace*{\fill}
\finalhyphendemerits=0
[2nd August; PC gain from Lab]}

\index{Buttrills , Vale of Glamorgan@Buttrills, \emph{Vale of Glamorgan}}

Death of Margaret Alexander (Lab).

\noindent
\begin{tabular*}{\columnwidth}{@{\extracolsep{\fill}} p{0.545\columnwidth} >{\itshape}l r @{\extracolsep{\fill}}}
Ian Johnson & PC & 541\\
Brian Morris & Lab & 503\\
Thomas Burley & C & 90\\
David Green & Ind & 82\\
\end{tabular*}

\section{Gwent}

\council{Caerphilly}

At the May 2012 ordinary election there was an unfilled vacancy in Argoed division due to the death of Allen Williams (Lab).
\index{Argoed , Caerphilly@Argoed, \emph{Caerphilly}}

\subsubsection*{New Tredegar \hspace*{\fill}\nolinebreak[1]%
\enspace\hspace*{\fill}
\finalhyphendemerits=0
[25th October]}

\index{New Tredegar , Caerphilly@New Tredegar, \emph{Caerphilly}}

Death of Les Rees (Lab).

\noindent
\begin{tabular*}{\columnwidth}{@{\extracolsep{\fill}} p{0.545\columnwidth} >{\itshape}l r @{\extracolsep{\fill}}}
Eluned Stenner & Lab & 694\\
Gillian Jones & PC & 94\\
Cameron Muir-Jones & C & 24\\
\end{tabular*}

\council{Newport}

At the May 2012 ordinary election there were unfilled vacancies in Allt-yr-yn and Alway divisions due to the deaths of Les Knight (C) and Ken Powell (Lab) respectively.
\index{Allt-yr-yn , Newport@Allt-yr-yn, \emph{Newport}}
\index{Alway , Newport@Alway, \emph{Newport}}



\section{Mid and West Wales}

\council{Carmarthenshire}

At the May 2012 ordinary election there was an unfilled vacancy in Tycroes division due to the death of Dewi Enoch (Lab).
\index{Tycroes , Carmarthenshire@Tycroes, \emph{Carmarthenshire}}

\council{Pembrokeshire}

At the May 2012 ordinary election there was an unfilled vacancy in Pembroke Dock Central division due to the resignation of Kate Becton (Lab).
\index{Pembroke Dock Central , Pembrokeshire@Pembroke Dock C., \emph{Pembrokeshire}}



\section{North Wales}

\council{Conwy}

\subsubsection*{Deganwy \hspace*{\fill}\nolinebreak[1]%
\enspace\hspace*{\fill}
\finalhyphendemerits=0
[15th November; C gain from Ind]}

\index{Deganwy , Conwy@Deganwy, \emph{Conwy}}

Resignation of Jason Weyman (Ind).

\noindent
\begin{tabular*}{\columnwidth}{@{\extracolsep{\fill}} p{0.545\columnwidth} >{\itshape}l r @{\extracolsep{\fill}}}
Julie Fallon & C & 437\\
Trystan Lewis & PC & 327\\
Bill Chapman & Lab & 142\\
Goronwy Edwards & Ind & 74\\
John Humberstone & UKIP & 57\\
Terry James & LD & 53\\
Malcolm Bullock & Ind & 49\\
\end{tabular*}



\council{Gwynedd}

LlG = Llais Gwynedd

\subsubsection*{Bryn-crug\slash Llanfihangel \hspace*{\fill}\nolinebreak[1]%
\enspace\hspace*{\fill}
\finalhyphendemerits=0
[14th June]}

\index{Bryn-crug\slash Llanfihangel , Gwynedd@Bryn-crug\slash Llanfihangel, \emph{Gwynedd}}

No candidates nominated in ordinary election on 3rd May.

\noindent
\begin{tabular*}{\columnwidth}{@{\extracolsep{\fill}} p{0.545\columnwidth} >{\itshape}l r @{\extracolsep{\fill}}}
Beth Lawton & Ind & 212\\
Alun Evans & PC & 136\\
John Pughe & Ind & 66\\
Nancy Clarke & Ind & 51\\
Gwyn Hughes & LlG & 35\\
\end{tabular*}

\council{Isle of Anglesey}

\subsubsection*{Gwyngyll \hspace*{\fill}\nolinebreak[1]%
\enspace\hspace*{\fill}
\finalhyphendemerits=0
[12th July; Ind gain from PC]}

\index{Gwyngyll , Isle of Anglesey@Gwyngyll, \emph{Isle of Anglesey}}

Resignation of John Penri Williams (PC).

\noindent
\begin{tabular*}{\columnwidth}{@{\extracolsep{\fill}} p{0.545\columnwidth} >{\itshape}l r @{\extracolsep{\fill}}}
Alun Mummery & Ind & 378\\
Elaine Gill & UKIP & 93\\
\end{tabular*}

\subsubsection*{Llanbedrgoch \hspace*{\fill}\nolinebreak[1]%
\enspace\hspace*{\fill}
\finalhyphendemerits=0
[12th July; PC gain from Ind]}

\index{Llanbedrgoch , Isle of Anglesey@Llanbedrgoch, \emph{Isle of Anglesey}}

Resignation of Barrie Durkin (Ind).

\noindent
\begin{tabular*}{\columnwidth}{@{\extracolsep{\fill}} p{0.545\columnwidth} >{\itshape}l r @{\extracolsep{\fill}}}
Vaughan Hughes & PC & 182\\
Jeffrey Cotterell & Ind & 92\\
Carys Kilkelly & Ind & 92\\
Nathan Gill & UKIP & 89\\
\end{tabular*}



\section{Ayrshire Councils}

\council{East Ayrshire}

At the May 2012 ordinary election there were unfilled vacancies in Ballochmyle and Doon Valley wards due to the death of Jimmy Kelly (Lab) and the disqualification (non-attendance) of Drew Filson (SNP) respectively.
\index{Ballochmyle , East Ayrshire@Ballochmyle, \emph{E. Ayrshire}}
\index{Doon Valley , East Ayrshire@Doon Valley, \emph{E. Ayrshire}}

\section[Border Councils]{\sloppyword{Border Councils}}

\council{Dumfries and Galloway}

\subsubsection*{Annandale North \hspace*{\fill}\nolinebreak[1]%
\enspace\hspace*{\fill}
\finalhyphendemerits=0
[15th November; C gain from Lab]}

\index{Annandale North , Dumfries and Galloway@Annandale N., \emph{Dumfries \& Galloway}}

Death of Ted Brown (Lab).

\noindent
\begin{tabular*}{\columnwidth}{@{\extracolsep{\fill}} p{0.545\columnwidth} >{\itshape}l r @{\extracolsep{\fill}}}
\emph{First preferences}\\
Graeme Tait & C & 1819\\
Peter Glanton & Lab & 1002\\
Alis Ballance & Grn & 464\\
Frank Macgregor & SNP & 371\\
Hugh Young & LD & 208\\
Bill Wright & UKIP & 89\\
\end{tabular*}

%\noindent
%\begin{tabular*}{\columnwidth}{@{\extracolsep{\fill}} p{0.545\columnwidth} >{\itshape}l r @{\extracolsep{\fill}}}
\emph{\sloppyword{Young and Wright eliminated}}: Tait 1922 Glanton 1053 Ballance 508 Macgregor 399
%Graeme Tait & C & 1922\\
%Peter Glanton & Lab & 1053\\
%Alis Ballance & Grn & 508\\
%Frank Macgregor & SNP & 399\\
%\end{tabular*}

\noindent
\begin{tabular*}{\columnwidth}{@{\extracolsep{\fill}} p{0.545\columnwidth} >{\itshape}l r @{\extracolsep{\fill}}}
\emph{Macgregor eliminated}\\
Graeme Tait & C & 1980\\
Peter Glanton & Lab & 1149\\
Alis Ballance & Grn & 617\\
\end{tabular*}



\section{Clyde Councils}

\council{East Dunbartonshire}

\subsubsection*{Campsie and Kirkintilloch North \hspace*{\fill}\nolinebreak[1]%
\enspace\hspace*{\fill}
\finalhyphendemerits=0
[13th September; Lab gain from East Dunbartonshire Independent Alliance]}

\index{Campsie and Kirkintilloch North , East Dunbartonshire@Campsie \& Kirkintilloch N., \emph{E. Duns.}}

Death of Charles Kennedy (East Dunbartonshire Independent Alliance).

\noindent
\begin{tabular*}{\columnwidth}{@{\extracolsep{\fill}} p{0.545\columnwidth} >{\itshape}l r @{\extracolsep{\fill}}}
\emph{First preferences}\\
Gemma Welsh & Lab & 851\\
Billy Hutchison & SNP & 743\\
Susan Murray & LD & 693\\
Brian Reid & Ind & 274\\
Alisdair Sinclair & C & 141\\
Ross Greer & Grn & 65\\
\end{tabular*}

%\noindent
%\begin{tabular*}{\columnwidth}{@{\extracolsep{\fill}} p{0.545\columnwidth} >{\itshape}l r @{\extracolsep{\fill}}}
\sloppyword{\emph{Reid, Sinclair and Greer eliminated}}: Welsh 939 Hutchison 845 Murray 788
%Gemma Welsh & Lab & 939\\
%Billy Hutchison & SNP & 845\\
%Susan Murray & LD & 788\\
%\end{tabular*}

\noindent
\begin{tabular*}{\columnwidth}{@{\extracolsep{\fill}} p{0.545\columnwidth} >{\itshape}l r @{\extracolsep{\fill}}}
\emph{Murray eliminated}\\
Gemma Welsh & Lab & 1146\\
Billy Hutchison & SNP & 1044\\
\end{tabular*}

\council{Renfrewshire}

At the May 2012 ordinary election there was an unfilled vacancy in Renfrew North ward due to the resignation of Derek Mackay (SNP).
\index{Renfrew North , Renfrewshire@Renfrew N., \emph{Renfs.}}

\council{West Dunbartonshire}

At the May 2012 ordinary election there was an unfilled vacancy in Dumbarton ward due to the resignation of Geoff Calvert (Lab).
\index{Dumbarton , West Dunbartonshire@Dumbarton, \emph{W. Duns.}}



\section{Highland Councils}

\council{Argyll and Bute}

At the May 2012 ordinary election there were unfilled vacancies in Helensburgh Central, and Oban South and the Isles wards due to the deaths of Al Reay (LD) and Donald McIntosh (SNP) respectively.
\index{Helensburgh Central , Argyll and Bute@Helensburgh C., \emph{Argyll \& Bute}}
\index{Oban South and the Isles , Argyll and Bute@Oban S. \& the Isles, \emph{Argyll \& Bute}}

\subsubsection*{Dunoon (3) \hspace*{\fill}\nolinebreak[1]%
\enspace\hspace*{\fill}
\finalhyphendemerits=0
[10th May]}

\index{Dunoon , Argyll and Bute@Dunoon, \emph{Argyll \& Bute}}

Ordinary election postponed from 3rd May: death of sitting councillor Alister MacAlister (SNP) who had been nominated for re-election.

%See page \pageref{DunoonArgyllBute} for the result.

\noindent
\begin{tabular*}{\columnwidth}{@{\extracolsep{\fill}} p{0.545\columnwidth} >{\itshape}l r @{\extracolsep{\fill}}}
	\emph{First preferences}\\
Dick Walsh & Ind & 777\\
Michael Breslin & SNP & 663\\
Jimmy McQueen & Ind & 432\\
Mick Rice & Lab & 296\\
Tony Miles & LD & 105\\
William Green & C & 94\\
\end{tabular*}

\emph{Quota = 592, Walsh and Breslin elected.  Walsh surplus transferred}: McQueen 527 Rice 315 Miles 128 Green 106

\emph{Breslin surplus transferred}: McQueen 544 Rice 327 Miles 137 Green 111

\noindent
\begin{tabular*}{\columnwidth}{@{\extracolsep{\fill}} p{0.545\columnwidth} >{\itshape}l r @{\extracolsep{\fill}}}
	\emph{Green eliminated}\\
	Dick Walsh & Ind & 592\\
	Michael Breslin & SNP & 592\\
	Jimmy McQueen & Ind & 590\\
	Mick Rice & Lab & 340\\
	Tony Miles & LD & 160\\
\end{tabular*}

\section{Island Councils}

\council{Eilean Siar}

\subsubsection*{Sgir' Uige agus Ceann a Tuath nan Loch \hspace*{\fill}\nolinebreak[1]%
\enspace\hspace*{\fill}
\finalhyphendemerits=0
[29th November; Ind gain from SNP]}

\index{Sgir' Uige agus Ceann a Tuath nan Loch , Eilean Siar@Sgir' Uige \& Ceann a T. nan Loch, \emph{Eilean Siar}}

Death of Bill Houston (SNP).

\noindent
\begin{tabular*}{\columnwidth}{@{\extracolsep{\fill}} p{0.545\columnwidth} >{\itshape}l r @{\extracolsep{\fill}}}
Angus Morrison & Ind & 745\\
John Macdonald & SNP & 195\\
Les Mac an Ultaigh & Ind & 73\\
\end{tabular*}

\council{Orkney}

At the May 2012 ordinary election there was an unfilled vacancy in Stromness and South Isles ward due to the death of John Eccles (Ind).
\index{Stromness and South Isles , Orkney@Stromness \& South Isles, \emph{Orkney}}

\end{resultsiii}

\part{2013}
\renewcommand\resultsyear{2013}

%\part{Referendums}

\chapter{Referendums in 2013}

\section{Middlesbrough mayoral referendum}
\index{Middlesbrough!Mayoral Retention Referendum}

A referendum was held in Middlesbrough on 26th September 2013 on whether to retain the mayoral system of government.

\noindent
\begin{tabular*}{\columnwidth}{@{\extracolsep{\fill}} p{0.545\columnwidth} >{\itshape}l r @{\extracolsep{\fill}}}
& Yes & 8674\\
& No & 6455\\
\end{tabular*}

%\part{By-elections}

\chapter{Parliamentary by-elections}

There were three parliamentary by-elections in 2013.

BeerBC = Beer, Baccy and Crumpet

Elvis = Elvis Loves Pets

NHA = National Health Action

Peace = Peace Party

Wessex = Wessex Regionalist

\section*{Eastleigh\hspace*{\fill}\nolinebreak[1]%
\enspace\hspace*{\fill}
\finalhyphendemerits=0
[28th February]}

\index{Eastleigh , House of Commons@Eastleigh, \emph{House of Commons}}

Resignation of Chris Huhne (LD).

\noindent
\begin{tabular*}{\columnwidth}{@{\extracolsep{\fill}} p{0.545\columnwidth} >{\itshape}l r @{\extracolsep{\fill}}}
Mike Thornton & LD & 13342\\
Diane James & UKIP & 11571\\
Maria Hutchings & C & 10559\\
John O'Farrell & Lab & 4088\\
Danny Stupple & Ind & 768\\
Iain Maclennan & NHA & 392\\
Ray Hall & BeerBC & 235\\
Kevin Milburn & Chr & 163\\
Howling Laud Hope & Loony & 136\\
Jim Duggan & Peace & 128\\
David Bishop & Elvis & 72\\
Michael Walters & EDP & 70\\
Daz Proctor & TUSC & 62\\
Colin Bex & Wessex & 30\\
\end{tabular*}

\vfill

\section*{Mid Ulster\hspace*{\fill}\nolinebreak[1]%
\enspace\hspace*{\fill}
\finalhyphendemerits=0
[7th March]}

\index{Mid Ulster , House of Commons@Mid Ulster, \emph{House of Commons}}

Resignation of Martin McGuinness (SF).

\noindent
\begin{tabular*}{\columnwidth}{@{\extracolsep{\fill}} p{0.545\columnwidth} >{\itshape}l r @{\extracolsep{\fill}}}
Francie Molloy & SF & 17462\\
Nigel Lutton & Ind & 12781\\
Patsy McGlone & SDLP & 6478\\
Eric Bullick & All & 487\\
\end{tabular*}

\eject

\section*{South Shields\hspace*{\fill}\nolinebreak[1]%
\enspace\hspace*{\fill}
\finalhyphendemerits=0
[2nd May]}

\index{South Shields , House of Commons@South Shields, \emph{House of Commons}}

Resignation of David Miliband (Lab).

\noindent
\begin{tabular*}{\columnwidth}{@{\extracolsep{\fill}} p{0.545\columnwidth} >{\itshape}l r @{\extracolsep{\fill}}}
Emma Lewell-Buck & Lab & 12493\\
Richard Elvin & UKIP & 5988\\
Karen Allen & C & 2857\\
Ahmed Khan & Ind & 1331\\
Phil Brown & IndSoc & 750\\
Lady Dorothy Brookes & BNP & 711\\
Hugh Annand & LD & 352\\
Howling Laud Hope & Loony & 197\\
Thomas Darwood & Ind & 57\\
\end{tabular*}

\chapter{By-elections to devolved assemblies, the European Parliament, and police and crime commissionerships}

\section{Greater London Authority}

There were no by-elections in 2013 to the Greater London Authority.

\section{National Assembly for Wales}

There was one by-election in 2013 to the National Assembly for Wales.

\subsection*{Ynys Môn \hspace*{\fill}\nolinebreak[1]%
\enspace\hspace*{\fill}
\finalhyphendemerits=0
[1st August]}

\index{Ynys Mon , Welsh Assembly@Ynys M\^on, \emph{Welsh Assembly}}

Resignation of Ieuan Wyn Jones (PC).

\noindent
\begin{tabular*}{\columnwidth}{@{\extracolsep{\fill}} p{0.545\columnwidth} >{\itshape}l r @{\extracolsep{\fill}}}
Rhun ap Iorwerth & PC & 12601\\
Tal Michael & Lab & 3435\\
Nathan Gill & UKIP & 3099\\
Neil Fairlamb & C & 1843\\
Kathrine Jones & SocLab & 348\\
Stephen Churchman & LD & 309\\
\end{tabular*}

\section{Scottish Parliament}

There were two by-elections in 2013 to the Scottish Parliament.

\subsection*{Aberdeen Donside\hspace*{\fill}\nolinebreak[1]%
\enspace\hspace*{\fill}
\finalhyphendemerits=0
[20th June]}

\index{Aberdeen Donside , Scottish Parliament@Aberdeen Donside, \emph{Scot. Parl.}}

Death of Brian Adam (SNP).

SDA = Scottish Democratic Alliance

\noindent
\begin{tabular*}{\columnwidth}{@{\extracolsep{\fill}} p{0.545\columnwidth} >{\itshape}l r @{\extracolsep{\fill}}}
Mark McDonald & SNP & 9814\\
Willie Young & Lab & 7789\\
Christine Jardine & LD & 1940\\
Ross Thomson & C & 1791\\
Otto Inglis & UKIP & 1128\\
Rhonda Reekie & Grn & 410\\
Dave MacDonald & NF & 249\\
Tom Morrow & Chr & 222\\
James Trolland & SDA & 35\\
\end{tabular*}

\subsection*{Dunfermline\hspace*{\fill}\nolinebreak[1]%
\enspace\hspace*{\fill}
\finalhyphendemerits=0
[24th October; Lab gain from SNP]}

\index{Dunfermline , Scottish Parliament@Dunfermline, \emph{Scot. Parl.}}

Resignation of Bill Walker (SNP).

\noindent
\begin{tabular*}{\columnwidth}{@{\extracolsep{\fill}} p{0.545\columnwidth} >{\itshape}l r @{\extracolsep{\fill}}}
Cara Hilton & Lab & 10279\\
Shirley-Anne Somerville & SNP & 7402\\
Susan Leslie & LD & 2852\\
James Reekie & C & 2009\\
Peter Adams & UKIP & 908\\
Zara Kitson & Grn & 593\\
John Black & Ind & 161\\
\end{tabular*}

\bigskip

Mark McDonald (SNP, North East Scotland) resigned on 14th May in order to contest the Aberdeen Donside by-election.  He was replaced by the only remaining person on the party's list from the 2011 election, Christian Allard.

David McLetchie (C, Lothians) died on 12th August.  He was replaced by Cameron Buchanan.

\section{Northern Ireland Assembly}

Vacancies in the Northern Ireland Assembly are filled by co-option.
%No co-options were made in 2013.
The following members were co-opted to the Assembly in 2014:
\begin{itemize}
\item Ian Milne (SF) replaced Francie Molloy following his resignation on 8th April (Mid Ulster).
\item Fearghal McKinney (SDLP) replaced Connall McDevitt following his resignation on 4th September (Belfast South).
\end{itemize}

\section{European Parliament}

UK vacancies in the European Parliament are filled by the next available person from the party list at the most recent election (which was held in 2009).
No replacements were made in 2013.

%The following replacement was made in 2010:
%\begin{itemize}
%\item Keith Taylor (Grn) replaced Caroline Lucas following her resignation on 17th May (South East).
%\end{itemize}

\section{Police and crime commissioners}

There were no by-elections in 2013 for vacant police and crime commissioner posts.

\chapter{Local by-elections and unfilled vacancies}

\begin{resultsiii}

\section{North London}

\council{City of London}

At the March 2013 ordinary election there was an unfilled vacancy in Dowgate ward due to the resignation of Robin Sherlock.
\index{Dowgate , City of London@Dowgate, \emph{City of London}}

\subsubsection*{Bread Street \hspace*{\fill}\nolinebreak[1]%
\enspace\hspace*{\fill}
\finalhyphendemerits=0
[11th March]}

\index{Bread Street , City of London@Bread St., \emph{City of London}}

Aldermanic election: Sir Michael Savory (Ind) retired.

\noindent
\begin{tabular*}{\columnwidth}{@{\extracolsep{\fill}} p{0.545\columnwidth} >{\itshape}l r @{\extracolsep{\fill}}}
William Russell & Ind & 85\\
Nicholas Bensted-Smith & Ind & 40\\
\end{tabular*}

\subsubsection*{Lime Street \hspace*{\fill}\nolinebreak[1]%
\enspace\hspace*{\fill}
\finalhyphendemerits=0
[7th May]}

\index{Lime Street , City of London@Lime St., \emph{City of London}}

Aldermanic election: Sir John Stuttard (Ind) retired.

\noindent
\begin{tabular*}{\columnwidth}{@{\extracolsep{\fill}} p{0.545\columnwidth} >{\itshape}l r @{\extracolsep{\fill}}}
Charles Bowman & Ind & \emph{unop.}\\
\end{tabular*}

\subsubsection*{Bassishaw \hspace*{\fill}\nolinebreak[1]%
\enspace\hspace*{\fill}
\finalhyphendemerits=0
[8th May]}

\index{Bassishaw , City of London@Bassishaw, \emph{City of London}}

Aldermanic election: Philip Remnant (Ind) retired.

\noindent
\begin{tabular*}{\columnwidth}{@{\extracolsep{\fill}} p{0.545\columnwidth} >{\itshape}l r @{\extracolsep{\fill}}}
Timothy Hailes & Ind & \emph{unop.}\\
\end{tabular*}

\subsubsection*{Farringdon Without \hspace*{\fill}\nolinebreak[1]%
\enspace\hspace*{\fill}
\finalhyphendemerits=0
[10th May]}

\index{Farringdon Without , City of London@Farringdon Wt., \emph{City of London}}

Aldermanic election: Robert Hall (Ind) retired.

\noindent
\begin{tabular*}{\columnwidth}{@{\extracolsep{\fill}} p{0.545\columnwidth} >{\itshape}l r @{\extracolsep{\fill}}}
Julian Malins & Ind & \emph{unop.}\\
\end{tabular*}

\subsubsection*{Broad Street \hspace*{\fill}\nolinebreak[1]%
\enspace\hspace*{\fill}
\finalhyphendemerits=0
[4th July]}

\index{Broad Street , City of London@Broad St., \emph{City of London}}

Aldermanic election: Sir David Lewis (Ind) retired.

\noindent
\begin{tabular*}{\columnwidth}{@{\extracolsep{\fill}} p{0.545\columnwidth} >{\itshape}l r @{\extracolsep{\fill}}}
Michael Mainelli & Ind & 62\\
Christopher Roebuck & Ind & 38\\
\end{tabular*}

\subsubsection*{Cheap \hspace*{\fill}\nolinebreak[1]%
\enspace\hspace*{\fill}
\finalhyphendemerits=0
[4th July]}

\index{Cheap , City of London@Cheap, \emph{City of London}}

Aldermanic election: Jeffrey Evans (Ind) sought re-election.

\noindent
\begin{tabular*}{\columnwidth}{@{\extracolsep{\fill}} p{0.545\columnwidth} >{\itshape}l r @{\extracolsep{\fill}}}
Jeffrey Evans & Ind & \emph{unop.}\\
\end{tabular*}

\subsubsection*{Farringdon Within \hspace*{\fill}\nolinebreak[1]%
\enspace\hspace*{\fill}
\finalhyphendemerits=0
[23rd October]}

\index{Farringdon Within , City of London@Farringdon Wn., \emph{City of London}}

Aldermanic election: Simon Walsh (Ind) resigned.

\noindent
\begin{tabular*}{\columnwidth}{@{\extracolsep{\fill}} p{0.545\columnwidth} >{\itshape}l r @{\extracolsep{\fill}}}
Vincent Keaveny & Ind & 169\\
Kenneth Olisa & Ind & 79\\
\end{tabular*}

\subsubsection*{Cheap \hspace*{\fill}\nolinebreak[1]%
\enspace\hspace*{\fill}
\finalhyphendemerits=0
[29th October]}

\index{Vintry , City of London@Vintry, \emph{City of London}}

Aldermanic election: Andrew Parmley (Ind) sought re-election.

\noindent
\begin{tabular*}{\columnwidth}{@{\extracolsep{\fill}} p{0.545\columnwidth} >{\itshape}l r @{\extracolsep{\fill}}}
Andrew Parmley & Ind & \emph{unop.}\\
\end{tabular*}

\subsubsection*{Coleman Street \hspace*{\fill}\nolinebreak[1]%
\enspace\hspace*{\fill}
\finalhyphendemerits=0
[29th November]}

\index{Coleman Street , City of London@Coleman St., \emph{City of London}}

Aldermanic election: Sir Robert Finch (Ind) retired.

\noindent
\begin{tabular*}{\columnwidth}{@{\extracolsep{\fill}} p{0.545\columnwidth} >{\itshape}l r @{\extracolsep{\fill}}}
Peter Estlin & Ind & 129\\
Andrew McMurtrie & Ind & 38\\
James Pollard & Ind & 29\\
\end{tabular*}

\council{Barking and Dagenham}

\subsubsection*{Longbridge \hspace*{\fill}\nolinebreak[1]%
\enspace\hspace*{\fill}
\finalhyphendemerits=0
[9th May]}

\index{Longbridge , Barking and Dagenham@Longbridge, \emph{Barking \& Dagenham}}

Death of Nirmal Singh Gill (Lab).

\noindent
\begin{tabular*}{\columnwidth}{@{\extracolsep{\fill}} p{0.545\columnwidth} >{\itshape}l r @{\extracolsep{\fill}}}
Syed Ahammad & Lab & 1555\\
Bert Bedwell & UKIP & 466\\
Paul Ayer & C & 284\\
Dave Croft & LD & 78\\
Giuseppe de Santis & BNP & 37\\
\end{tabular*}

\council{Camden}

\subsubsection*{Gospel Oak \hspace*{\fill}\nolinebreak[1]%
\enspace\hspace*{\fill}
\finalhyphendemerits=0
[14th March]}

\index{Gospel Oak , Camden@Gospel Oak, \emph{Camden}}

Resignation of Sean Birch (Lab).

\noindent
\begin{tabular*}{\columnwidth}{@{\extracolsep{\fill}} p{0.545\columnwidth} >{\itshape}l r @{\extracolsep{\fill}}}
Maeve McCormack & Lab & 1272\\
Leila Roy & C & 419\\
Constantine Buhayer & Grn & 134\\
Laura Noel & LD & 132\\
John Reid & TUSC & 109\\
Stephen Dornan & BNP & 57\\
\end{tabular*}

\council{Hammersmith and Fulham}

\subsubsection*{Wormholt and White City \hspace*{\fill}\nolinebreak[1]%
\enspace\hspace*{\fill}
\finalhyphendemerits=0
[7th February]}

\index{Wormholt and White City , Hammersmith and Fulham@\sloppyword{Wormholt \& White City, \emph{Hammersmith \& Fulham}}}

Death of Jean Campbell (Lab).

\noindent
\begin{tabular*}{\columnwidth}{@{\extracolsep{\fill}} p{0.545\columnwidth} >{\itshape}l r @{\extracolsep{\fill}}}
Max Schmid & Lab & 1419\\
Jamie McKittrick & C & 251\\
Chris Whittaker & LD & 209\\
Andrew Elston & UKIP & 122\\
Jeffrey Boateng & Ind & 75\\
Andrew Donald & BNP & 45\\
\end{tabular*}

\columnbreak

\council{Harrow}

Harrow = Harrow First

\subsubsection*{West Harrow \hspace*{\fill}\nolinebreak[1]%
\enspace\hspace*{\fill}
\finalhyphendemerits=0
[21st February]}

\index{West Harrow , Harrow@West Harrow, \emph{Harrow}}

Resignation of Brian Gate (Lab).

\noindent
\begin{tabular*}{\columnwidth}{@{\extracolsep{\fill}} p{0.545\columnwidth} >{\itshape}l r @{\extracolsep{\fill}}}
Christine Robson & Lab & 1042\\
Julia Merison & C & 761\\
Jeremy Zeid & UKIP & 171\\
Rowan Langley & Grn & 96\\
Pash Nandhra & LD & 68\\
Herbert Crossman & Ind & 53\\
\end{tabular*}

\subsubsection*{Harrow on the Hill \hspace*{\fill}\nolinebreak[1]%
\enspace\hspace*{\fill}
\finalhyphendemerits=0
[7th November]}

\index{Harrow on the Hill , Harrow@Harrow on the Hill, \emph{Harrow}}

Resignation of Ann Gate (Lab).

\noindent
\begin{tabular*}{\columnwidth}{@{\extracolsep{\fill}} p{0.545\columnwidth} >{\itshape}l r @{\extracolsep{\fill}}}
Glen Hearnden & Lab & 991\\
Stephen Lewis & C & 836\\
Eileen Kinnear & Ind & 308\\
Gajan Idaikkadar & Harrow & 173\\
Jeremy Zeid & UKIP & 168\\
Gaye Branch & LD & 70\\
\end{tabular*}

\council{Havering}

HWHPRA = Harold Wood Hill Park Residents Association

RAL = Residents Association of London

\subsubsection*{Gooshays \hspace*{\fill}\nolinebreak[1]%
\enspace\hspace*{\fill}
\finalhyphendemerits=0
[21st March; UKIP gain from C]}

\index{Gooshays , Havering@Gooshays, \emph{Havering}}

Death of Dennis Bull (C).

\noindent
\begin{tabular*}{\columnwidth}{@{\extracolsep{\fill}} p{0.466\columnwidth} >{\itshape}l r @{\extracolsep{\fill}}}
Lawrence Webb & UKIP & 831\\
Christine McGeary & Lab & 569\\
\sloppyword{Marcus Llewellyn-Rothschild} & C & 280\\
Darren Wise & HWHPRA & 227\\
Mick Braun & BNP & 202\\
Malvin Brown & RAL & 24\\
\end{tabular*}

\columnbreak

\council{Islington}

\subsubsection*{Junction \hspace*{\fill}\nolinebreak[1]%
\enspace\hspace*{\fill}
\finalhyphendemerits=0
[21st March; Lab gain from LD]}

\index{Junction , Islington@Junction, \emph{Islington}}

Resignation of Arthur Graves (Ind elected as LD).

\noindent
\begin{tabular*}{\columnwidth}{@{\extracolsep{\fill}} p{0.545\columnwidth} >{\itshape}l r @{\extracolsep{\fill}}}
Kaya Makarau-Schwartz & Lab & 1343\\
Mick Holloway & Grn & 381\\
Stefan Kasprzyk & LD & 276\\
Patricia Napier & C & 120\\
Gary Townsend & BNP & 31\\
Bill Martin & Soc & 18\\
\end{tabular*}

\subsubsection*{St George's \hspace*{\fill}\nolinebreak[1]%
\enspace\hspace*{\fill}
\finalhyphendemerits=0
[21st March]}

\index{Saint George's , Islington@St George's, \emph{Islington}}

Resignation of Jessica Asato (Lab).

\noindent
\begin{tabular*}{\columnwidth}{@{\extracolsep{\fill}} p{0.545\columnwidth} >{\itshape}l r @{\extracolsep{\fill}}}
Kat Fletcher & Lab & 1698\\
Julian Gregory & LD & 371\\
Jon Nott & Grn & 206\\
Evan Williams & C & 87\\
Walter Barfoot & BNP & 20\\
\end{tabular*}



\council{Westminster}

MbnRes = Marylebone Residents

\subsubsection*{Marylebone High Street \hspace*{\fill}\nolinebreak[1]%
\enspace\hspace*{\fill}
\finalhyphendemerits=0
[2nd May]}

\index{Marylebone High Street , Westminster@Marylebone High St., \emph{Westminster}}

Resignation of Harvey Marshall (C).

\noindent
\begin{tabular*}{\columnwidth}{@{\extracolsep{\fill}} p{0.53\columnwidth} >{\itshape}l r @{\extracolsep{\fill}}}
Iain Bott & C & 921\\
Nik Slingsby & Lab & 203\\
Yael Saunders & MbnRes & 184\\
Jeremy Hill & LD & 104\\
Paul Mercieca & UKIP & 96\\
Hugh Small & Grn & 50\\
\end{tabular*}

\section{South London}

\council{Kingston upon Thames}

\subsubsection*{Berrylands \hspace*{\fill}\nolinebreak[1]%
\enspace\hspace*{\fill}
\finalhyphendemerits=0
[28th February]}

\index{Berrylands , Kingston upon Thames@Berrylands, \emph{Kingston upon Thames}}

Death of Frances Moseley (LD).

\noindent
\begin{tabular*}{\columnwidth}{@{\extracolsep{\fill}} p{0.545\columnwidth} >{\itshape}l r @{\extracolsep{\fill}}}
Sushila Abraham & LD & 948\\
Mike Head & C & 761\\
Tony Banks & Lab & 455\\
Michael Watson & UKIP & 175\\
Ryan Coley & Grn & 112\\
\end{tabular*}

\subsubsection*{Beverley \hspace*{\fill}\nolinebreak[1]%
\enspace\hspace*{\fill}
\finalhyphendemerits=0
[25th July; C gain from LD]}

\index{Beverley , Kingston upon Thames@Beverley, \emph{Kingston upon Thames}}

Resignation of Derek Osbourne (LD).

\noindent
\begin{tabular*}{\columnwidth}{@{\extracolsep{\fill}} p{0.545\columnwidth} >{\itshape}l r @{\extracolsep{\fill}}}
Terence Paton & C & 1033\\
Lesley Heap & LD & 760\\
Marian Freedman & Lab & 717\\
Michael Watson & UKIP & 223\\
Chris Walker & Grn & 207\\
\end{tabular*}

\council{Lambeth}

\subsubsection*{Brixton Hill \hspace*{\fill}\nolinebreak[1]%
\enspace\hspace*{\fill}
\finalhyphendemerits=0
[17th January]}

\index{Brixton Hill , Lambeth@Brixton Hill, \emph{Lambeth}}

Resignation of Steve Reed MP (Lab).

\noindent
\begin{tabular*}{\columnwidth}{@{\extracolsep{\fill}} p{0.545\columnwidth} >{\itshape}l r @{\extracolsep{\fill}}}
Martin Tiedemann & Lab & 1593\\
Andrew Child & Grn & 344\\
Liz Maffei & LD & 274\\
Timothy Briggs & C & 164\\
Steve Nally & TUSC & 72\\
Elizabeth Jones & UKIP & 63\\
Daniel Lambert & Soc & 34\\
\end{tabular*}

\subsubsection*{Tulse Hill \hspace*{\fill}\nolinebreak[1]%
\enspace\hspace*{\fill}
\finalhyphendemerits=0
[25th July]}

\index{Tulse Hill , Lambeth@Tulse Hill, \emph{Lambeth}}

Death of Ruth Ling (Lab).

\noindent
\begin{tabular*}{\columnwidth}{@{\extracolsep{\fill}} p{0.545\columnwidth} >{\itshape}l r @{\extracolsep{\fill}}}
Mary Atkins & Lab & 1575\\
Amna Ahmad & LD & 277\\
Bernard Atwell & Grn & 177\\
Steve Nally & TUSC & 76\\
Timothy Briggs & C & 74\\
Elizabeth Jones & UKIP & 64\\
Valentine Walker & Ind & 20\\
Adam Buick & Soc & 11\\
\end{tabular*}

\subsubsection*{Vassall \hspace*{\fill}\nolinebreak[1]%
\enspace\hspace*{\fill}
\finalhyphendemerits=0
[28th November]}

\index{Vassall , Lambeth@Vassall, \emph{Lambeth}}

Resignation of Kingsley Abrams (Lab).

\noindent
\begin{tabular*}{\columnwidth}{@{\extracolsep{\fill}} p{0.545\columnwidth} >{\itshape}l r @{\extracolsep{\fill}}}
Paul Gadsby & Lab & 1319\\
Colette Thomas & LD & 468\\
Kelly Ben-Maimon & C & 153\\
Rachel Laurence & Grn & 113\\
Elizabeth Jones & UKIP & 87\\
Steven Nally & TUSC & 44\\
Danny Lambert & Soc & 22\\
\end{tabular*}

\council{Lewisham}

PBP = People Before Profit

\subsubsection*{Evelyn \hspace*{\fill}\nolinebreak[1]%
\enspace\hspace*{\fill}
\finalhyphendemerits=0
[28th March]}

\index{Evelyn , Lewisham@Evelyn, \emph{Lewisham}}

Resignation of Joseph Folorunso (Lab).

\noindent
\begin{tabular*}{\columnwidth}{@{\extracolsep{\fill}} p{0.545\columnwidth} >{\itshape}l r @{\extracolsep{\fill}}}
Olufunke Abidoye & Lab & 918\\
Barbara Raymond & PBP & 404\\
Bill Town & LD & 131\\
Simon Nundy & C & 119\\
Paul Oakley & UKIP & 119\\
\end{tabular*}

\council{Merton}

\subsubsection*{Colliers Wood \hspace*{\fill}\nolinebreak[1]%
\enspace\hspace*{\fill}
\finalhyphendemerits=0
[8th August]}

\index{Colliers Wood , Merton@Colliers Wood, \emph{Merton}}

Death of Gam Gurung (Lab).

\noindent
\begin{tabular*}{\columnwidth}{@{\extracolsep{\fill}} p{0.545\columnwidth} >{\itshape}l r @{\extracolsep{\fill}}}
\sloppyword{Caroline Cooper-Marbiah} & Lab & 1685\\
Peter Lord & C & 441\\
Shafqat Janjua & UKIP & 157\\
Phil Ling & LD & 52\\
\end{tabular*}

\columnbreak

\section{Greater Manchester}

\council{Bolton}

\subsubsection*{Harper Green \hspace*{\fill}\nolinebreak[1]%
\enspace\hspace*{\fill}
\finalhyphendemerits=0
[19th December]}

\index{Harper Green , Bolton@Harper Green, \emph{Bolton}}

Death of Margaret Clare (Lab).

\noindent
\begin{tabular*}{\columnwidth}{@{\extracolsep{\fill}} p{0.545\columnwidth} >{\itshape}l r @{\extracolsep{\fill}}}
Asha Ali Ismail & Lab & 744\\
Robert Tyler & C & 325\\
Peter McGeehan & UKIP & 252\\
Kathy Sykes & Grn & 60\\
Wendy Connor & LD & 53\\
\end{tabular*}



\council{Manchester}

CommLg = Communist League

Pirate = Pirate Party

\subsubsection*{Ancoats and Clayton \hspace*{\fill}\nolinebreak[1]%
\enspace\hspace*{\fill}
\finalhyphendemerits=0
[10th October]}

\index{Ancoats and Clayton , Manchester@Ancoats \& Clayton, \emph{Manchester}}

Resignation of Jim Battle (Lab).

\noindent
\begin{tabular*}{\columnwidth}{@{\extracolsep{\fill}} p{0.545\columnwidth} >{\itshape}l r @{\extracolsep{\fill}}}
Donna Ludford & Lab & 1239\\
Adrienne Shaw & UKIP & 166\\
Pete Birkinshaw & Grn & 89\\
Nicholas Savage & C & 82\\
Loz Kaye & Pirate & 79\\
Gareth Black & BNP & 58\\
John Bridges & LD & 44\\
\end{tabular*}

\subsubsection*{Ancoats and Clayton \hspace*{\fill}\nolinebreak[1]%
\enspace\hspace*{\fill}
\finalhyphendemerits=0
[5th December]}

\index{Ancoats and Clayton , Manchester@Ancoats \& Clayton, \emph{Manchester}}

Death of Mike Carmody (Lab).

\noindent
\begin{tabular*}{\columnwidth}{@{\extracolsep{\fill}} p{0.5\columnwidth} >{\itshape}l r @{\extracolsep{\fill}}}
Ollie Manco & Lab & 965\\
Ken Dobson & Lib & 219\\
Martin Power & UKIP & 138\\
Pete Birkinshaw & Grn & 106\\
David Semple & C & 75\\
Loz Kaye & Pirate & 72\\
Gareth Black & BNP & 46\\
Claude Nsumbu & LD & 31\\
Alex Davidson & TUSC & 17\\
Caroline Bellamy & CommLg & 9\\
\end{tabular*}

\council{Oldham}

\subsubsection*{Royton South \hspace*{\fill}\nolinebreak[1]%
\enspace\hspace*{\fill}
\finalhyphendemerits=0
[14th March]}

\index{Royton South , Oldham@Royton S., \emph{Oldham}}

Death of Phil Harrison (Lab).

\noindent
\begin{tabular*}{\columnwidth}{@{\extracolsep{\fill}} p{0.545\columnwidth} >{\itshape}l r @{\extracolsep{\fill}}}
Marie Bashforth & Lab & 938\\
Allan Fish & C & 244\\
Stephen Barrow & LD & 221\\
Roger Pakeman & Grn & 70\\
\end{tabular*}

\subsubsection*{Alexandra \hspace*{\fill}\nolinebreak[1]%
\enspace\hspace*{\fill}
\finalhyphendemerits=0
[9th May]}

\index{Alexandra , Oldham@Alexandra, \emph{Oldham}}

Death of Dilys Fletcher (Lab).

\noindent
\begin{tabular*}{\columnwidth}{@{\extracolsep{\fill}} p{0.545\columnwidth} >{\itshape}l r @{\extracolsep{\fill}}}
Zahid Chauhan & Lab & 1553\\
Derek Fletcher & UKIP & 412\\
Kevin Dawson & LD & 96\\
Neil Allsopp & C & 80\\
Miranda Meadowcroft & Grn & 55\\
\end{tabular*}

\council{Rochdale}

\subsubsection*{Norden \hspace*{\fill}\nolinebreak[1]%
\enspace\hspace*{\fill}
\finalhyphendemerits=0
[25th April]}

\index{Norden , Rochdale@Norden, \emph{Rochdale}}

Death of Ann Metcalfe (C).

\noindent
\begin{tabular*}{\columnwidth}{@{\extracolsep{\fill}} p{0.545\columnwidth} >{\itshape}l r @{\extracolsep{\fill}}}
Peter Winkler & C & 1081\\
Anthony Bennett & Lab & 627\\
Patricia Colclough & LD & 246\\
Peter Greenwood & NF & 156\\
\end{tabular*}

\council{Salford}

\subsubsection*{Weaste and Seedley \hspace*{\fill}\nolinebreak[1]%
\enspace\hspace*{\fill}
\finalhyphendemerits=0
[20th June]}

\index{Weaste and Seedley , Salford@Weaste \& Seedley, \emph{Salford}}

Death of Jan Rochford (Lab).

\noindent
\begin{tabular*}{\columnwidth}{@{\extracolsep{\fill}} p{0.545\columnwidth} >{\itshape}l r @{\extracolsep{\fill}}}
Paul Wilson & Lab & 785\\
Glyn Wright & UKIP & 401\\
Adam Kennaugh & C & 260\\
Mary Ferrer & Grn & 80\\
Kay Pollitt & BNP & 74\\
Matt Simpson & Ind & 64\\
Valerie Kelly & LD & 58\\
Terry Simmons & TUSC & 30\\
Alan Valentine & Ind & 15\\
\end{tabular*}

\subsubsection*{Weaste and Seedley \hspace*{\fill}\nolinebreak[1]%
\enspace\hspace*{\fill}
\finalhyphendemerits=0
[10th October]}

\index{Weaste and Seedley , Salford@Weaste \& Seedley, \emph{Salford}}

Resignation of Thomas Murphy (Lab).

\noindent
\begin{tabular*}{\columnwidth}{@{\extracolsep{\fill}} p{0.545\columnwidth} >{\itshape}l r @{\extracolsep{\fill}}}
Stephen Hesling & Lab & 803\\
Glyn Wright & UKIP & 280\\
Adam Kennaugh & C & 240\\
Matt Simpson & Ind & 96\\
Andrew Olsen & Grn & 42\\
Kay Pollitt & BNP & 29\\
Terry Simmons & TUSC & 21\\
\end{tabular*}

\council{Wigan}

CA = Community Action

\subsubsection*{Pemberton \hspace*{\fill}\nolinebreak[1]%
\enspace\hspace*{\fill}
\finalhyphendemerits=0
[4th April]}

\index{Pemberton , Wigan@Pemberton, \emph{Wigan}}

Death of Barbara Bourne (Lab).

\noindent
\begin{tabular*}{\columnwidth}{@{\extracolsep{\fill}} p{0.545\columnwidth} >{\itshape}l r @{\extracolsep{\fill}}}
Sam Murphy & Lab & 1084\\
Alan Freeman & UKIP & 451\\
Peter Franzen & CA & 203\\
Jonathan Cartwright & C & 89\\
Dennis Shambley & BNP & 63\\
\end{tabular*}

\subsubsection*{Winstanley \hspace*{\fill}\nolinebreak[1]%
\enspace\hspace*{\fill}
\finalhyphendemerits=0
[17th October]}

\index{Winstanley , Wigan@Winstanley, \emph{Wigan}}

Death of Rona Winkworth (Lab).

\noindent
\begin{tabular*}{\columnwidth}{@{\extracolsep{\fill}} p{0.545\columnwidth} >{\itshape}l r @{\extracolsep{\fill}}}
Maria Morgan & Lab & 746\\
Andrew Collinson & UKIP & 421\\
Stan Barnes & CA & 326\\
Michael Winstanley & C & 180\\
Steven Heyes & Grn & 55\\
David Bowker & Ind & 27\\
John Skipworth & LD & 19\\
\end{tabular*}

\section{Merseyside}

\council{Knowsley}

\subsubsection*{Prescot East \hspace*{\fill}\nolinebreak[1]%
\enspace\hspace*{\fill}
\finalhyphendemerits=0
[4th April]}

\index{Prescot East , Knowsley@Prescot E., \emph{Knowsley}}

Death of Derek McEgan (Lab).

\noindent
\begin{tabular*}{\columnwidth}{@{\extracolsep{\fill}} p{0.545\columnwidth} >{\itshape}l r @{\extracolsep{\fill}}}
Steff O'Keeffe & Lab & 515\\
Carl Cashman & LD & 328\\
Adam Butler & C & 47\\
\end{tabular*}

\subsubsection*{Prescot West \hspace*{\fill}\nolinebreak[1]%
\enspace\hspace*{\fill}
\finalhyphendemerits=0
[4th April]}

\index{Prescot West , Knowsley@Prescot W., \emph{Knowsley}}

Death of Bob Whiley (Lab).

\noindent
\begin{tabular*}{\columnwidth}{@{\extracolsep{\fill}} p{0.545\columnwidth} >{\itshape}l r @{\extracolsep{\fill}}}
Lynn O'Keeffe & Lab & 441\\
Ian Smith & LD & 403\\
Stephen Whatham & TUSC & 86\\
Robert Avery & C & 62\\
Robert Mbanu & Grn & 14\\
\end{tabular*}

\subsubsection*{St Michaels \hspace*{\fill}\nolinebreak[1]%
\enspace\hspace*{\fill}
\finalhyphendemerits=0
[4th April]}

\index{Saint Michaels , Knowsley@St Michaels, \emph{Knowsley}}

Death of Ken Keith (Lab).

\noindent
\begin{tabular*}{\columnwidth}{@{\extracolsep{\fill}} p{0.545\columnwidth} >{\itshape}l r @{\extracolsep{\fill}}}
Vickie Lamb & Lab & 676\\
Mike Currie & LD & 69\\
David Dunne & C & 44\\
\end{tabular*}

\council{Liverpool}

\subsubsection*{Riverside \hspace*{\fill}\nolinebreak[1]%
\enspace\hspace*{\fill}
\finalhyphendemerits=0
[5th December]}

\index{Riverside , Liverpool@Riverside, \emph{Liverpool}}

Resignation of Paul Brant (Lab).

\noindent
\begin{tabular*}{\columnwidth}{@{\extracolsep{\fill}} p{0.55\columnwidth} >{\itshape}l r @{\extracolsep{\fill}}}
Michelle Corrigan & Lab & 1055\\
Martin Dobson & Grn & 144\\
Adam Heatherington & UKIP & 119\\
Kevin White & LD & 64\\
John Marston & TUSC & 49\\
Chris Hall & C & 39\\
Steven McEllenborough & EDP & 9\\
Peter Cooney & Ind & 7\\
Alison Goudie & Ind & 1\\
\end{tabular*}

\subsection{St Helens}
\index{Saint Helens@St Helens}

\subsubsection*{Windle \hspace*{\fill}\nolinebreak[1]%
\enspace\hspace*{\fill}
\finalhyphendemerits=0
[2nd May]}

\index{Windle , Saint Helens@Windle, \emph{St Helens}}

Death of Pat Martinez-Williams (Lab).

\noindent
\begin{tabular*}{\columnwidth}{@{\extracolsep{\fill}} p{0.545\columnwidth} >{\itshape}l r @{\extracolsep{\fill}}}
David Baines & Lab & 1329\\
Robert Reynolds & C & 612\\
Francis Williams & Grn & 345\\
\end{tabular*}

\subsubsection*{Billinge and Seneley Green \hspace*{\fill}\nolinebreak[1]%
\enspace\hspace*{\fill}
\finalhyphendemerits=0
[28th November]}

\index{Billinge and Seneley Green , Saint Helens@Billinge \& Seneley Green, \emph{St Helens}}

Resignation of Alison Bacon (Lab).

\noindent
\begin{tabular*}{\columnwidth}{@{\extracolsep{\fill}} p{0.545\columnwidth} >{\itshape}l r @{\extracolsep{\fill}}}
Dennis McDonnell & Lab & 936\\
Laurence Allen & UKIP & 442\\
John Cunliffe & C & 248\\
Sue Rahman & Grn & 94\\
Alan Brindle & BNP & 73\\
Noreen Knowles & LD & 52\\
\end{tabular*}

\council{Sefton}

\subsubsection*{Derby \hspace*{\fill}\nolinebreak[1]%
\enspace\hspace*{\fill}
\finalhyphendemerits=0
[7th November]}

\index{Derby , Sefton@Derby, \emph{Sefton}}

Death of Carol Gustafson (Lab).

\noindent
\begin{tabular*}{\columnwidth}{@{\extracolsep{\fill}} p{0.545\columnwidth} >{\itshape}l r @{\extracolsep{\fill}}}
Anne Thompson & Lab & 903\\
Jack Colbert & UKIP & 293\\
Juliet Edgar & Ind & 97\\
Graham Woodhouse & TUSC & 48\\
Janice Blanchard & Ind & 29\\
Laurence Rankin & Grn & 25\\
\end{tabular*}

\council{Wirral}

\subsubsection*{Heswall \hspace*{\fill}\nolinebreak[1]%
\enspace\hspace*{\fill}
\finalhyphendemerits=0
[17th January]}

\index{Heswall , Wirral@Heswall, \emph{Wirral}}

Death of Peter Johnson (C).

\noindent
\begin{tabular*}{\columnwidth}{@{\extracolsep{\fill}} p{0.545\columnwidth} >{\itshape}l r @{\extracolsep{\fill}}}
Kathryn Hodson & C & 1254\\
David Scott & UKIP & 460\\
Michael Holliday & Lab & 289\\
Barbara Burton & Grn & 110\\
Gregory North & TUSC & 19\\
\end{tabular*}

\subsubsection*{Leasowe and Moreton East \hspace*{\fill}\nolinebreak[1]%
\enspace\hspace*{\fill}
\finalhyphendemerits=0
[17th January; C gain from Lab]}

\index{Leasowe and Moreton East , Wirral@Leasowe \& Moreton E., \emph{Wirral}}

Death of Anne McArdle (Lab).

\noindent
\begin{tabular*}{\columnwidth}{@{\extracolsep{\fill}} p{0.545\columnwidth} >{\itshape}l r @{\extracolsep{\fill}}}
Ian Lewis & C & 1620\\
Pauline Daniels & Lab & 1355\\
Susan Whitham & UKIP & 148\\
Mark Halligan & TUSC & 31\\
Daniel Clein & LD & 28\\
James McGinley & Grn & 28\\
\end{tabular*}

\subsubsection*{Pensby and Thingwall \hspace*{\fill}\nolinebreak[1]%
\enspace\hspace*{\fill}
\finalhyphendemerits=0
[28th February; Lab gain from C]}

\index{Pensby and Thingwall , Wirral@Pensby \& Thingwall, \emph{Wirral}}

Resignation of Don McCubbin (C).

\noindent
\begin{tabular*}{\columnwidth}{@{\extracolsep{\fill}} p{0.545\columnwidth} >{\itshape}l r @{\extracolsep{\fill}}}
Phillip Brightmore & Lab & 1411\\
Sheila Clarke & C & 868\\
Damien Cummins & LD & 834\\
Jan Davison & UKIP & 426\\
Allen Burton & Grn & 74\\
Neil Kenny & EDP & 53\\
\end{tabular*}

\subsubsection*{Upton \hspace*{\fill}\nolinebreak[1]%
\enspace\hspace*{\fill}
\finalhyphendemerits=0
[24th October]}

\index{Upton , Wirral@Upton, \emph{Wirral}}

Resignation of Sylvia Hodrien (Lab).

\noindent
\begin{tabular*}{\columnwidth}{@{\extracolsep{\fill}} p{0.545\columnwidth} >{\itshape}l r @{\extracolsep{\fill}}}
Matthew Patrick & Lab & 1954\\
Geoffrey Gubb & C & 762\\
Jim McGinley & Grn & 143\\
Alan Davies & LD & 130\\
\end{tabular*}

\section{South Yorkshire}

\council{Barnsley}

\subsubsection*{Wombwell \hspace*{\fill}\nolinebreak[1]%
\enspace\hspace*{\fill}
\finalhyphendemerits=0
[26th September]}

\index{Wombwell , Barnsley@Wombwell, \emph{Barnsley}}

Death of Denise Wilde (Lab).

\noindent
\begin{tabular*}{\columnwidth}{@{\extracolsep{\fill}} p{0.545\columnwidth} >{\itshape}l r @{\extracolsep{\fill}}}
Robert Frost & Lab & 1240\\
Neil Robinson & UKIP & 457\\
Clive Watkinson & C & 81\\
Kevin Riddiough & EDP & 78\\
\end{tabular*}

\subsubsection*{Royston \hspace*{\fill}\nolinebreak[1]%
\enspace\hspace*{\fill}
\finalhyphendemerits=0
[10th October]}

\index{Royston , Barnsley@Royston, \emph{Barnsley}}

Resignation of Graham Kyte (Lab).

\noindent
\begin{tabular*}{\columnwidth}{@{\extracolsep{\fill}} p{0.545\columnwidth} >{\itshape}l r @{\extracolsep{\fill}}}
Caroline Makinson & Lab & 1143\\
James Johnson & UKIP & 393\\
Paul Buckley & C & 100\\
Justin Saxton & EDP & 32\\
Mark Baker & BNP & 28\\
\end{tabular*}

\council{Doncaster}

\subsubsection*{Askern Spa \hspace*{\fill}\nolinebreak[1]%
\enspace\hspace*{\fill}
\finalhyphendemerits=0
[22nd August]}

\index{Askern Spa , Doncaster@Askern Spa, \emph{Doncaster}}

Election of Ros Jones (Lab) as Mayor of Doncaster.

\noindent
\begin{tabular*}{\columnwidth}{@{\extracolsep{\fill}} p{0.545\columnwidth} >{\itshape}l r @{\extracolsep{\fill}}}
Iris Beech & Lab & 1165\\
Adrian McLeay & LD & 261\\
Frank Calladine & UKIP & 231\\
Martin Greenhalgh & C & 225\\
Martyn Bev & Ind & 106\\
David Allen & EDP & 98\\
Mary Jackson & TUSC & 72\\
\end{tabular*}

\council{Rotherham}

\subsubsection*{Rawmarsh \hspace*{\fill}\nolinebreak[1]%
\enspace\hspace*{\fill}
\finalhyphendemerits=0
[16th May; UKIP gain from Lab]}

\index{Rawmarsh , Rotherham@Rawmarsh, \emph{Rotherham}}

Resignation of Shaun Wright (Lab).

\noindent
\begin{tabular*}{\columnwidth}{@{\extracolsep{\fill}} p{0.545\columnwidth} >{\itshape}l r @{\extracolsep{\fill}}}
Caven Vines & UKIP & 1143\\
Lisa Wright & Lab & 1039\\
Martyn Parker & C & 107\\
George Baldwin & BNP & 80\\
Andrew Gray & TUSC & 61\\
Mohammed Meharban & LD & 28\\
\end{tabular*}

\council{Sheffield}

\subsubsection*{Fulwood \hspace*{\fill}\nolinebreak[1]%
\enspace\hspace*{\fill}
\finalhyphendemerits=0
[2nd May]}

\index{Fulwood , Sheffield@Fulwood, \emph{Sheffield}}

Death of Janice Sidebottom (LD).

\noindent
\begin{tabular*}{\columnwidth}{@{\extracolsep{\fill}} p{0.545\columnwidth} >{\itshape}l r @{\extracolsep{\fill}}}
Cliff Woodcraft & LD & 2563\\
Olivia Blake & Lab & 1035\\
Vonny Watts & C & 826\\
John Greenfield & UKIP & 501\\
Brian Webster & Grn & 379\\
\end{tabular*}

\section{Tyne and Wear}

\council{Newcastle upon Tyne}

NuTCFP = Newcastle upon Tyne Community First Party

\subsubsection*{Castle \hspace*{\fill}\nolinebreak[1]%
\enspace\hspace*{\fill}
\finalhyphendemerits=0
[25th April]}

\index{Castle , Newcastle upon Tyne@Castle, \emph{Newcastle upon Tyne}}

Resignation of Ian Laverick (LD).

\noindent
\begin{tabular*}{\columnwidth}{@{\extracolsep{\fill}} p{0.48\columnwidth} >{\itshape}l r @{\extracolsep{\fill}}}
Philip Lower & LD & 1165\\
Ben Riley & Lab & 1043\\
John Gordon & NuTCFP & 215\\
Jennifer Nixon & C & 194\\
Rory Jobe & TUSC & 47\\
\end{tabular*}

\columnbreak

\subsubsection*{South Heaton \hspace*{\fill}\nolinebreak[1]%
\enspace\hspace*{\fill}
\finalhyphendemerits=0
[25th April]}

\index{South Heaton , Newcastle upon Tyne@South Heaton, \emph{Newcastle upon Tyne}}

Resignation of Henri Murison (Lab).

\noindent
\begin{tabular*}{\columnwidth}{@{\extracolsep{\fill}} p{0.5\columnwidth} >{\itshape}l r @{\extracolsep{\fill}}}
Denise Jones & Lab & 798\\
Andrew Gray & Grn & 205\\
Rachel Auld & LD & 114\\
Paul Phillips & TUSC & 69\\
Katie Bennett & C & 52\\
Timothy Gilks & NuTCFP & 44\\
Reg Sibley & Ind & 22\\
\end{tabular*}

\subsubsection*{Walkergate \hspace*{\fill}\nolinebreak[1]%
\enspace\hspace*{\fill}
\finalhyphendemerits=0
[6th June]}

\index{Walkergate , Newcastle upon Tyne@Walkergate, \emph{Newcastle upon Tyne}}

Resignation of Tania Armstrong (Lab).

\noindent
\begin{tabular*}{\columnwidth}{@{\extracolsep{\fill}} p{0.48\columnwidth} >{\itshape}l r @{\extracolsep{\fill}}}
Stephen Wood & Lab & 1080\\
Lorraine Smith & UKIP & 668\\
Kevin Brown & LD & 460\\
Davy Hicks & Ind & 64\\
Olga Shorton & NuTCFP & 61\\
Marian McWilliams & C & 54\\
Martin Collins & Grn & 30\\
Bobbie Cranney & TUSC & 24\\
Reg Sibley & Ind & 12\\
\end{tabular*}

\council{North Tyneside}

\subsubsection*{Riverside \hspace*{\fill}\nolinebreak[1]%
\enspace\hspace*{\fill}
\finalhyphendemerits=0
[4th July]}

\index{Riverside , North Tyneside@Riverside, \emph{N. Tyneside}}

Election of Norma Redfearn (Lab) as Mayor.

\noindent
\begin{tabular*}{\columnwidth}{@{\extracolsep{\fill}} p{0.48\columnwidth} >{\itshape}l r @{\extracolsep{\fill}}}
Wendy Lott & Lab & 1067\\
Barbara Stevens & C & 179\\
\end{tabular*}

\council{South Tyneside}

\subsubsection*{Cleadon and East Boldon \hspace*{\fill}\nolinebreak[1]%
\enspace\hspace*{\fill}
\finalhyphendemerits=0
[27th June; Lab gain from C]}

\index{Cleadon and East Boldon , South Tyneside@Cleadon \& E. Boldon, \emph{S. Tyneside}}

Death of David Potts (UKIP elected as C).

\noindent
\begin{tabular*}{\columnwidth}{@{\extracolsep{\fill}} p{0.48\columnwidth} >{\itshape}l r @{\extracolsep{\fill}}}
Margaret Meling & Lab & 991\\
Fiona Milburn & C & 899\\
Colin Campbell & UKIP & 666\\
\end{tabular*}

\subsubsection*{Primrose \hspace*{\fill}\nolinebreak[1]%
\enspace\hspace*{\fill}
\finalhyphendemerits=0
[27th June]}

\index{Primrose , South Tyneside@Primrose, \emph{S. Tyneside}}

\sloppyword{Resignation of Emma Lewell-Buck MP (Lab).}

\noindent
\begin{tabular*}{\columnwidth}{@{\extracolsep{\fill}} p{0.48\columnwidth} >{\itshape}l r @{\extracolsep{\fill}}}
Moira Smith & Lab & 755\\
John Clarke & UKIP & 520\\
Martin Vaughan & BNP & 146\\
John Coe & C & 80\\
\end{tabular*}

\council{Sunderland}

\subsubsection*{Houghton \hspace*{\fill}\nolinebreak[1]%
\enspace\hspace*{\fill}
\finalhyphendemerits=0
[2nd May]}

\index{Houghton , Sunderland@Houghton, \emph{Sunderland}}

Death of Kath Rolph (Lab).

\noindent
\begin{tabular*}{\columnwidth}{@{\extracolsep{\fill}} p{0.545\columnwidth} >{\itshape}l r @{\extracolsep{\fill}}}
Gemma Taylor & Lab & 1418\\
John Ellis & Ind & 1124\\
\sloppyword{Edward Coleman-Hughes} & UKIP & 302\\
George Brown & C & 111\\
Sue Sterling & LD & 55\\
\end{tabular*}

\section{West Midlands}

\council{Dudley}

\subsubsection*{\sloppyword{Wollaston and Stourbridge Town} \hspace*{\fill}\nolinebreak[1]%
\enspace\hspace*{\fill}
\finalhyphendemerits=0
[31st January; Lab gain from C]}

\index{Wollaston and Stourbridge Town , Dudley@Wollaston \& Stourbridge Town, \emph{Dudley}}

Death of Malcolm Knowles (C).

\noindent
\begin{tabular*}{\columnwidth}{@{\extracolsep{\fill}} p{0.545\columnwidth} >{\itshape}l r @{\extracolsep{\fill}}}
Barbara Sykes & Lab & 847\\
Matt Rogers & C & 787\\
Barbara Deeley & UKIP & 249\\
Russell Eden & Ind & 211\\
Christopher Bramall & LD & 169\\
Ken Griffiths & BNP & 96\\
Ben Sweeney & Grn & 7\\
\end{tabular*}

\subsubsection*{Coseley East \hspace*{\fill}\nolinebreak[1]%
\enspace\hspace*{\fill}
\finalhyphendemerits=0
[19th September]}

\index{Coseley East , Dudley@Coseley E., \emph{Dudley}}

Death of George Davies (Lab).

\noindent
\begin{tabular*}{\columnwidth}{@{\extracolsep{\fill}} p{0.545\columnwidth} >{\itshape}l r @{\extracolsep{\fill}}}
Clem Baugh & Lab & 1053\\
Star Etheridge & UKIP & 478\\
Julian Ryder & C & 190\\
Ken Griffiths & BNP & 120\\
Becky Blatchford & Grn & 33\\
Kevin Inman & NF & 16\\
\end{tabular*}

\council{Walsall}

\subsubsection*{Aldridge Central and South \hspace*{\fill}\nolinebreak[1]%
\enspace\hspace*{\fill}
\finalhyphendemerits=0
[15th August]}

\index{Aldridge Central and South , Walsall@Aldridge C. \& S., \emph{Walsall}}

Death of Tom Ansell (C).

\noindent
\begin{tabular*}{\columnwidth}{@{\extracolsep{\fill}} p{0.545\columnwidth} >{\itshape}l r @{\extracolsep{\fill}}}
Timothy Wilson & C & 1254\\
Liz Hazell & UKIP & 615\\
Bob Grainger & Lab & 470\\
Roy Sheward & LD & 114\\
Chris Newey & EDP & 72\\
\end{tabular*}

\council{Wolverhampton}

\subsubsection*{Blakenhall \hspace*{\fill}\nolinebreak[1]%
\enspace\hspace*{\fill}
\finalhyphendemerits=0
[2nd May]}

\index{Blakenhall , Wolverhampton@Blakenhall, \emph{Wolverhampton}}

Resignation of Bob Jones (Lab).

\noindent
\begin{tabular*}{\columnwidth}{@{\extracolsep{\fill}} p{0.545\columnwidth} >{\itshape}l r @{\extracolsep{\fill}}}
Harbans Singh Bagri & Lab & 1934\\
David Mackintosh & UKIP & 263\\
Stephen Dion & C & 242\\
Eileen Birch & LD & 89\\
\end{tabular*}

\section{West Yorkshire}

\council{Kirklees}

\subsubsection*{Liversedge and Gomersal \hspace*{\fill}\nolinebreak[1]%
\enspace\hspace*{\fill}
\finalhyphendemerits=0
[13th June; Lab gain from C]}

\index{Liversedge and Gomersal , Kirklees@Liversedge \& Gomersal, \emph{Kirklees}}

Resignation of Margaret Bates (C).

\noindent
\begin{tabular*}{\columnwidth}{@{\extracolsep{\fill}} p{0.545\columnwidth} >{\itshape}l r @{\extracolsep{\fill}}}
Simon Alvy & Lab & 1517\\
Sharon Light & C & 1378\\
Richard Farnhill & LD & 599\\
\end{tabular*}

\subsubsection*{Golcar \hspace*{\fill}\nolinebreak[1]%
\enspace\hspace*{\fill}
\finalhyphendemerits=0
[21st November; LD gain from Lab]}

\index{Golcar , Kirklees@Golcar, \emph{Kirklees}}

Resignation of Paul Salveson (Lab).

\noindent
\begin{tabular*}{\columnwidth}{@{\extracolsep{\fill}} p{0.545\columnwidth} >{\itshape}l r @{\extracolsep{\fill}}}
Christine Iredale & LD & 1591\\
Stephan Jungnitz & Lab & 901\\
Gregory Broome & UKIP & 450\\
Daniel Greenwood & Grn & 210\\
Clinton Simpson & C & 189\\
\end{tabular*}

\council{Leeds}

\subsubsection*{Cross Gates and Whinmoor \hspace*{\fill}\nolinebreak[1]%
\enspace\hspace*{\fill}
\finalhyphendemerits=0
[2nd May]}

\index{Cross Gates and Whinmoor , Leeds@Cross Gates \& Whinmoor, \emph{Leeds}}

Death of Suzi Armitage (Lab).

\noindent
\begin{tabular*}{\columnwidth}{@{\extracolsep{\fill}} p{0.545\columnwidth} >{\itshape}l r @{\extracolsep{\fill}}}
Debra Coupar & Lab & 2481\\
Darren Oddy & UKIP & 1582\\
William Flynn & C & 587\\
Martin Hemingway & Grn & 229\\
Keith Norman & LD & 145\\
\end{tabular*}

\council{Wakefield}

\subsubsection*{\sloppyword{Castleford Central and Glasshoughton} \hspace*{\fill}\nolinebreak[1]%
\enspace\hspace*{\fill}
\finalhyphendemerits=0
[21st February]}

\index{Castleford Central and Glasshoughton , Wakefield@Castleford C. \& Glasshoughton, \emph{Wakefield}}

\sloppyword{Resignation of Mark Burns-Williamson (Lab).}

\noindent
\begin{tabular*}{\columnwidth}{@{\extracolsep{\fill}} p{0.545\columnwidth} >{\itshape}l r @{\extracolsep{\fill}}}
Richard Foster & Lab & 1567\\
Nathan Garbutt & UKIP & 349\\
Annemarie Glover & C & 95\\
Mark Goodair & LD & 33\\
\end{tabular*}

\subsubsection*{Horbury and South Ossett \hspace*{\fill}\nolinebreak[1]%
\enspace\hspace*{\fill}
\finalhyphendemerits=0
[28th November]}

\index{Horbury and South Ossett , Wakefield@Horbury \& South Ossett, \emph{Wakefield}}

Death of Brian Holmes (Lab).

\noindent
\begin{tabular*}{\columnwidth}{@{\extracolsep{\fill}} p{0.545\columnwidth} >{\itshape}l r @{\extracolsep{\fill}}}
Rory Bickerton & Lab & 1041\\
Graham Jesty & UKIP & 856\\
Angela Holwell & C & 504\\
Mark Goodair & LD & 212\\
\end{tabular*}

\section{Bedfordshire}

\council{Central Bedfordshire}

\subsubsection*{Dunstable -- Northfields \hspace*{\fill}\nolinebreak[1]%
\enspace\hspace*{\fill}
\finalhyphendemerits=0
[12th September; Ind gain from C]}

\index{Dunstable -- Northfields , Central Bedfordshire@Dunstable -- Northfields, \emph{C. Beds.}}

Death of Denise Green (C).

\noindent
\begin{tabular*}{\columnwidth}{@{\extracolsep{\fill}} p{0.545\columnwidth} >{\itshape}l r @{\extracolsep{\fill}}}
Bev Coleman & Ind & 434\\
Jeanette Freeman & C & 305\\
Duncan Ross & Lab & 297\\
Garry Lelliott & UKIP & 227\\
Lynda Walmsley & LD & 35\\
\end{tabular*}

\subsubsection*{Caddington \hspace*{\fill}\nolinebreak[1]%
\enspace\hspace*{\fill}
\finalhyphendemerits=0
[28th November]}

\index{Caddington , Central Bedfordshire@Caddington, \emph{C. Beds.}}

Resignation of Ruth Gammons (C).

\noindent
\begin{tabular*}{\columnwidth}{@{\extracolsep{\fill}} p{0.545\columnwidth} >{\itshape}l r @{\extracolsep{\fill}}}
Kevin Collins & C & 738\\
Christine Smith & Ind & 560\\
Steven Wildman & UKIP & 334\\
Ian Lowery & Lab & 209\\
Alan Winter & LD & 24\\
\end{tabular*}

\council{Luton}

\subsubsection*{Wigmore \hspace*{\fill}\nolinebreak[1]%
\enspace\hspace*{\fill}
\finalhyphendemerits=0
[Wednesday 10th April]}

\index{Wigmore , Luton@Wigmore, \emph{Luton}}

Death of Roy Davies (LD).

\noindent
\begin{tabular*}{\columnwidth}{@{\extracolsep{\fill}} p{0.545\columnwidth} >{\itshape}l r @{\extracolsep{\fill}}}
Alan Skepelhorn & LD & 982\\
James Taylor & Lab & 517\\
John Young & C & 281\\
Lance Richardson & UKIP & 230\\
Marc Scheimann & Grn & 82\\
John Magill & Ind & 27\\
\end{tabular*}

\subsubsection*{Barnfield \hspace*{\fill}\nolinebreak[1]%
\enspace\hspace*{\fill}
\finalhyphendemerits=0
[17th October]}

\index{Barnfield , Luton@Barnfield, \emph{Luton}}

Resignation of Martin Pantling (LD).

\noindent
\begin{tabular*}{\columnwidth}{@{\extracolsep{\fill}} p{0.545\columnwidth} >{\itshape}l r @{\extracolsep{\fill}}}
Clive Mead & LD & 674\\
Francis Steer & Lab & 635\\
Geoff Simons & C & 397\\
Simon Hall & Grn & 63\\
\end{tabular*}

\section{Berkshire}

\council{Bracknell Forest}

\subsubsection*{Winkfield and Cranbourne \hspace*{\fill}\nolinebreak[1]%
\enspace\hspace*{\fill}
\finalhyphendemerits=0
[28th November]}

\index{Winkfield and Cranbourne , Bracknell Forest@Winkfield \& Cranbourne, \emph{Bracknell Forest}}

Death of Mary Ballin (C).

\noindent
\begin{tabular*}{\columnwidth}{@{\extracolsep{\fill}} p{0.545\columnwidth} >{\itshape}l r @{\extracolsep{\fill}}}
Susie Phillips & C & 582\\
Ken la Garde & UKIP & 318\\
Janet Keene & Lab & 139\\
Paul Birchall & LD & 69\\
\end{tabular*}

\council{West Berkshire}

UPP = United People's Party

\subsubsection*{Hungerford \hspace*{\fill}\nolinebreak[1]%
\enspace\hspace*{\fill}
\finalhyphendemerits=0
[15th August]}

\index{Hungerford , West Berkshire@Hungerford, \emph{W. Berks.}}

Death of David Holtby (C).

\noindent
\begin{tabular*}{\columnwidth}{@{\extracolsep{\fill}} p{0.545\columnwidth} >{\itshape}l r @{\extracolsep{\fill}}}
James Podger & C & 810\\
Denise Gaines & LD & 751\\
Gary Puffett & Lab & 86\\
Andrew Stott & UPP & 28\\
\end{tabular*}

\section{Buckinghamshire}

\council{Aylesbury Vale}

\subsubsection*{Oakfield \hspace*{\fill}\nolinebreak[1]%
\enspace\hspace*{\fill}
\finalhyphendemerits=0
[3rd October]}

\index{Oakfield , Aylesbury Vale@Oakfield, \emph{Aylesbury Vale}}

Death of Steve Patrick (LD).

\noindent
\begin{tabular*}{\columnwidth}{@{\extracolsep{\fill}} p{0.545\columnwidth} >{\itshape}l r @{\extracolsep{\fill}}}
Allison Harrison & LD & 406\\
Phil Gomm & UKIP & 325\\
Edward Sims & C & 173\\
Roy McNickle & Lab & 145\\
Patrick Martin & Ind & 118\\
\end{tabular*}

\council{Milton Keynes}

\subsubsection*{Bletchley and Fenny Stratford \hspace*{\fill}\nolinebreak[1]%
\enspace\hspace*{\fill}
\finalhyphendemerits=0
[20th June]}

\index{Bletchley and Fenny Stratford , Milton Keynes@Bletchley \& Fenny Stratford, \emph{Milton Keynes}}

Death of Rita Venn (Lab).

\noindent
\begin{tabular*}{\columnwidth}{@{\extracolsep{\fill}} p{0.545\columnwidth} >{\itshape}l r @{\extracolsep{\fill}}}
Mohammed Khan & Lab & 1356\\
Vince Peddle & UKIP & 855\\
John Bailey & C & 779\\
Keith Allen & Grn & 277\\
Rosemary Snell & LD & 128\\
\end{tabular*}

\council{South Bucks}

\subsubsection*{Iver Village and Richings Park \hspace*{\fill}\nolinebreak[1]%
\enspace\hspace*{\fill}
\finalhyphendemerits=0
[12th December; C gain from LD]}

\index{Iver Village and Richings Park , South Bucks@Iver Village \& Richings Park, \emph{S. Bucks}}

Resignation of Alan Oxley (LD).

\noindent
\begin{tabular*}{\columnwidth}{@{\extracolsep{\fill}} p{0.545\columnwidth} >{\itshape}l r @{\extracolsep{\fill}}}
Paul Griffin & C & 422\\
Ken Wight & UKIP & 377\\
Peter Chapman & LD & 101\\
\end{tabular*}

\council{Wycombe}

\subsubsection*{Disraeli \hspace*{\fill}\nolinebreak[1]%
\enspace\hspace*{\fill}
\finalhyphendemerits=0
[2nd May]}

\index{Disraeli , Wycombe@Disraeli, \emph{Wycombe}}

Resignation of Victoria Groulef (Lab).

\noindent
\begin{tabular*}{\columnwidth}{@{\extracolsep{\fill}} p{0.545\columnwidth} >{\itshape}l r @{\extracolsep{\fill}}}
Maz Hussain & C & 553\\
Mohammed Rafiq & Lab & 466\\
Vijay Singh Srao & UKIP & 234\\
Andrew Stevens & LD & 129\\
\end{tabular*}

\subsubsection*{Hambleden Valley \hspace*{\fill}\nolinebreak[1]%
\enspace\hspace*{\fill}
\finalhyphendemerits=0
[5th September]}

\index{Hambleden Valley , Wycombe@Hambleden Valley, \emph{Wycombe}}

Death of Roger Emmett (C).

\noindent
\begin{tabular*}{\columnwidth}{@{\extracolsep{\fill}} p{0.545\columnwidth} >{\itshape}l r @{\extracolsep{\fill}}}
Roger Metcalfe & C & 379\\
Brian Mapletoft & UKIP & 97\\
Julian Grigg & Lab & 63\\
\end{tabular*}

\section{Cambridgeshire}

\council{Cambridge}

\subsubsection*{Abbey \hspace*{\fill}\nolinebreak[1]%
\enspace\hspace*{\fill}
\finalhyphendemerits=0
[2nd May; Lab gain from Grn]}

\index{Abbey , Cambridge@Abbey, \emph{Cambridge}}

\sloppyword{Resignation of Adam Pogonowski (Lab elected as Grn).}

\noindent
\begin{tabular*}{\columnwidth}{@{\extracolsep{\fill}} p{0.545\columnwidth} >{\itshape}l r @{\extracolsep{\fill}}}
Peter Roberts & Lab & 878\\
Oliver Perkins & Grn & 336\\
Eric Barrett-Payton & C & 284\\
Marcus Streets & LD & 209\\
\end{tabular*}

\council{East Cambridgeshire}

\subsubsection*{Ely East \hspace*{\fill}\nolinebreak[1]%
\enspace\hspace*{\fill}
\finalhyphendemerits=0
[5th September]}

\index{Ely East , East Cambridgeshire@Ely E., \emph{E. Cambs.}}

Resignation of Will Burton (C).

\noindent
\begin{tabular*}{\columnwidth}{@{\extracolsep{\fill}} p{0.545\columnwidth} >{\itshape}l r @{\extracolsep{\fill}}}
Lis Every & C & 418\\
Dian Warman & LD & 322\\
Jeremy Tyrrell & UKIP & 145\\
Jane Frances & Lab & 138\\
John Borland & Ind & 93\\
\end{tabular*}

\council{Fenland}

\subsubsection*{Hill \hspace*{\fill}\nolinebreak[1]%
\enspace\hspace*{\fill}
\finalhyphendemerits=0
[17th January]}

\index{Hill , Fenland@Hill, \emph{Fenland}}

Death of Bruce Wegg (C).

\noindent
\begin{tabular*}{\columnwidth}{@{\extracolsep{\fill}} p{0.545\columnwidth} >{\itshape}l r @{\extracolsep{\fill}}}
Samantha Hoy & C & 319\\
Dean Reeves & Lab & 110\\
John White & Ind & 105\\
\end{tabular*}

\subsubsection*{Parson Drove and Wisbech St Mary \hspace*{\fill}\nolinebreak[1]%
\enspace\hspace*{\fill}
\finalhyphendemerits=0
[28th March]}

\index{Parson Drove and Wisbech Saint Mary , Fenland@Parson Drove \& Wisbech St Mary, \emph{Fenland}}

Death of Robert Scrimshaw (C).

\noindent
\begin{tabular*}{\columnwidth}{@{\extracolsep{\fill}} p{0.545\columnwidth} >{\itshape}l r @{\extracolsep{\fill}}}
David Broker & C & 384\\
Mary Lane & LD & 240\\
Alan Lay & UKIP & 214\\
Maria Goldspink & EDP & 33\\
\end{tabular*}

\subsubsection*{Elm and Christchurch \hspace*{\fill}\nolinebreak[1]%
\enspace\hspace*{\fill}
\finalhyphendemerits=0
[19th December]}

\index{Elm and Christchurch , Fenland@Elm \& Christchurch, \emph{Fenland}}

Death of Mac Cotterell (C).

\noindent
\begin{tabular*}{\columnwidth}{@{\extracolsep{\fill}} p{0.545\columnwidth} >{\itshape}l r @{\extracolsep{\fill}}}
Michelle Tanfield & C & 301\\
Alan Burbridge & UKIP & 234\\
Phil Webb & Ind & 73\\
Dean Reeves & Lab & 51\\
Trevor Brookman & LD & 27\\
\end{tabular*}

\council{Huntingdonshire}

\subsubsection*{Brampton \hspace*{\fill}\nolinebreak[1]%
\enspace\hspace*{\fill}
\finalhyphendemerits=0
[2nd May]}

\index{Brampton , Huntingdonshire@Brampton, \emph{Hunts.}}

Resignation of Peter Downes (LD).

\noindent
\begin{tabular*}{\columnwidth}{@{\extracolsep{\fill}} p{0.545\columnwidth} >{\itshape}l r @{\extracolsep{\fill}}}
John Morris & LD & 856\\
Jane King & C & 506\\
Adrian Arnett & UKIP & 481\\
Mark Johnson & Lab & 90\\
\end{tabular*}

\council{South Cambridgeshire}

\subsubsection*{Balsham \hspace*{\fill}\nolinebreak[1]%
\enspace\hspace*{\fill}
\finalhyphendemerits=0
[2nd May; C gain from LD]}

\index{Balsham , South Cambridgeshire@Balsham, \emph{S. Cambs.}}

Resignation of Pauline Jarvis (LD).

\noindent
\begin{tabular*}{\columnwidth}{@{\extracolsep{\fill}} p{0.545\columnwidth} >{\itshape}l r @{\extracolsep{\fill}}}
Andrew Fraser & C & 731\\
John Batchelor & LD & 722\\
\end{tabular*}

\subsubsection*{Meldreth \hspace*{\fill}\nolinebreak[1]%
\enspace\hspace*{\fill}
\finalhyphendemerits=0
[2nd May]}

\index{Meldreth , South Cambridgeshire@Meldreth, \emph{S. Cambs.}}

Resignation of Surinder Soond (Ind elected as LD).

\noindent
\begin{tabular*}{\columnwidth}{@{\extracolsep{\fill}} p{0.545\columnwidth} >{\itshape}l r @{\extracolsep{\fill}}}
Susan van de Ven & LD & 607\\
David Kendrick & UKIP & 112\\
Duncan Bullivant & C & 101\\
Turlough Stone & Lab & 89\\
\end{tabular*}

\subsubsection*{Orwell and Barrington \hspace*{\fill}\nolinebreak[1]%
\enspace\hspace*{\fill}
\finalhyphendemerits=0
[2nd May; LD gain from C]}

\index{Orwell and Barrington , South Cambridgeshire@Orwell \& Barrington, \emph{S. Cambs.}}

Resignation of Ted Ridgway-Watt (C).

\noindent
\begin{tabular*}{\columnwidth}{@{\extracolsep{\fill}} p{0.545\columnwidth} >{\itshape}l r @{\extracolsep{\fill}}}
Aiden van de Weyer & LD & 428\\
Steven Sparkes & C & 326\\
\end{tabular*}

\subsubsection*{Sawston \hspace*{\fill}\nolinebreak[1]%
\enspace\hspace*{\fill}
\finalhyphendemerits=0
[18th July; C gain from Ind]}

\index{Sawston , South Cambridgeshire@Sawston, \emph{S. Cambs.}}

Resignation of Sally Hatton (Ind).

\noindent
\begin{tabular*}{\columnwidth}{@{\extracolsep{\fill}} p{0.545\columnwidth} >{\itshape}l r @{\extracolsep{\fill}}}
Kevin Cuffley & C & 477\\
Elizabeth Smith & UKIP & 233\\
Mike Nettleton & Lab & 199\\
Michael Kilpatrick & LD & 110\\
\end{tabular*}

\subsubsection*{Comberton \hspace*{\fill}\nolinebreak[1]%
\enspace\hspace*{\fill}
\finalhyphendemerits=0
[21st November; C gain from LD]}

\index{Comberton , South Cambridgeshire@Comberton, \emph{S. Cambs.}}

Resignation of Stephen Harangozo (LD).

\noindent
\begin{tabular*}{\columnwidth}{@{\extracolsep{\fill}} p{0.545\columnwidth} >{\itshape}l r @{\extracolsep{\fill}}}
Tim Scott & C & 378\\
Fay Boissieux & LD & 96\\
Helen Haugh & Lab & 74\\
Elizabeth Smith & UKIP & 48\\
\end{tabular*}

\section{Cheshire}

\council{Cheshire East}

\subsubsection*{Macclesfield Hurdsfield \hspace*{\fill}\nolinebreak[1]%
\enspace\hspace*{\fill}
\finalhyphendemerits=0
[25th April]}

\index{Macclesfield Hurdsfield , Cheshire East@Macclesfield Hurdsfield, \emph{Ches. E.}}

Resignation of Gill Boston (Lab).

\noindent
\begin{tabular*}{\columnwidth}{@{\extracolsep{\fill}} p{0.545\columnwidth} >{\itshape}l r @{\extracolsep{\fill}}}
Steve Carter & Lab & 341\\
Stephen Broadhead & LD & 239\\
Alastair Kennedy & C & 168\\
David Lonsdale & UKIP & 132\\
John Knight & Grn & 48\\
\end{tabular*}

\section{Cornwall}
\index{Cornwall}

At the May 2013 ordinary election there were unfilled vacancies in Ludgvan, Rame and Wadebridge East divisions due to the resignations of Irene Bailey (Ind), George Trubody (C) and Colin Brewer (Ind) respectively.

\index{Wadebridge East , Cornwall@Wadebridge E., \emph{Cornwall}}
\index{Ludgvan , Cornwall@Ludgvan, \emph{Cornwall}}
\index{Rame , Cornwall@Rame, \emph{Cornwall}}

\subsubsection*{Wadebridge East \hspace*{\fill}\nolinebreak[1]%
\enspace\hspace*{\fill}
\finalhyphendemerits=0
[5th September; LD gain from Ind]}

\index{Wadebridge East , Cornwall@Wadebridge E., \emph{Cornwall}}

Resignation of Colin Brewer (Ind).

\noindent
\begin{tabular*}{\columnwidth}{@{\extracolsep{\fill}} p{0.545\columnwidth} >{\itshape}l r @{\extracolsep{\fill}}}
Steve Knightley & LD & 408\\
Tony Rush & Ind & 399\\
Stephen Rushworth & C & 217\\
Rod Harrison & UKIP & 202\\
Adrian Jones & Lab & 58\\
\end{tabular*}

\section{Cumbria}

\subsection*{County Council}\index{Cumbria}

At the May 2013 ordinary election there was an unfilled vacancy in Distington and Moresby division due to the death of Cam Ross (Lab).

\index{Distington and Moresby , Cumbria@Distington \& Moresby, \emph{Cumbria}}

\subsubsection*{Seaton \hspace*{\fill}\nolinebreak[1]%
\enspace\hspace*{\fill}
\finalhyphendemerits=0
[14th November; Lab gain from Ind]}

\index{Seaton , Cumbria@Seaton, \emph{Cumbria}}

Death of Trevor Fee (Ind).

\noindent
\begin{tabular*}{\columnwidth}{@{\extracolsep{\fill}} p{0.545\columnwidth} >{\itshape}l r @{\extracolsep{\fill}}}
Celia Tibble & Lab & 628\\
Robert Hardon & UKIP & 483\\
Mike Davidson & C & 107\\
Tony North & Ind & 98\\
Frank Hollowell & LD & 26\\
\end{tabular*}

\council{Allerdale}

\subsubsection*{Derwent Valley \hspace*{\fill}\nolinebreak[1]%
\enspace\hspace*{\fill}
\finalhyphendemerits=0
[2nd May]}

\index{Derwent Valley , Allerdale@Derwent Valley, \emph{Allerdale}}

Resignation of Tim Heslop (C).

\noindent
\begin{tabular*}{\columnwidth}{@{\extracolsep{\fill}} p{0.545\columnwidth} >{\itshape}l r @{\extracolsep{\fill}}}
Hilary Hope & C & 225\\
Mark Jenkinson & UKIP & 92\\
Geoff Smith & Grn & 82\\
Andrew Lawson & Lab & 57\\
\end{tabular*}

\subsubsection*{Wampool \hspace*{\fill}\nolinebreak[1]%
\enspace\hspace*{\fill}
\finalhyphendemerits=0
[2nd May]}

\index{Wampool , Allerdale@Wampool, \emph{Allerdale}}

Resignation of Stuart Moffat (C).

\noindent
\begin{tabular*}{\columnwidth}{@{\extracolsep{\fill}} p{0.545\columnwidth} >{\itshape}l r @{\extracolsep{\fill}}}
Patricia Macdonald & C & 195\\
Robert Hardon & UKIP & 102\\
Marion Fitzgerald & Ind & 98\\
Jill Perry & Grn & 40\\
Charles Miles & LD & 39\\
\end{tabular*}

\subsubsection*{Boltons \hspace*{\fill}\nolinebreak[1]%
\enspace\hspace*{\fill}
\finalhyphendemerits=0
[20th June]}

\index{Boltons , Allerdale@Boltons, \emph{Allerdale}}

Death of Joe Mumberson (Ind).

\noindent
\begin{tabular*}{\columnwidth}{@{\extracolsep{\fill}} p{0.545\columnwidth} >{\itshape}l r @{\extracolsep{\fill}}}
Marion Fitzgerald & Ind & 175\\
Colin Smithson & C & 157\\
Mary Mumberson & Ind & 103\\
Dianne Standen & Grn & 55\\
\end{tabular*}

\subsubsection*{Seaton \hspace*{\fill}\nolinebreak[1]%
\enspace\hspace*{\fill}
\finalhyphendemerits=0
[14th November; Lab gain from Ind]}

\index{Seaton , Allerdale@Seaton, \emph{Allerdale}}

Death of Trevor Fee (Ind).

\noindent
\begin{tabular*}{\columnwidth}{@{\extracolsep{\fill}} p{0.545\columnwidth} >{\itshape}l r @{\extracolsep{\fill}}}
Andrew Lawson & Lab & 464\\
Mark Jenkinson & UKIP & 426\\
Mike Davidson & C & 133\\
Alistair Grey & Grn & 108\\
Chris Snowden & LD & 30\\
\end{tabular*}

\council{Carlisle}

\subsubsection*{Yewdale \hspace*{\fill}\nolinebreak[1]%
\enspace\hspace*{\fill}
\finalhyphendemerits=0
[5th September]}

\index{Yewdale , Carlisle@Yewdale, \emph{Carlisle}}

Death of Joe Hendry (Lab).

\noindent
\begin{tabular*}{\columnwidth}{@{\extracolsep{\fill}} p{0.545\columnwidth} >{\itshape}l r @{\extracolsep{\fill}}}
Tom Dodd & Lab & 716\\
Christine Finlayson & C & 453\\
Mike Owen & UKIP & 257\\
Terence Jones & LD & 31\\
Charmian McCutcheon & Grn & 14\\
\end{tabular*}

\subsubsection*{Dalston \hspace*{\fill}\nolinebreak[1]%
\enspace\hspace*{\fill}
\finalhyphendemerits=0
[17th October; LD gain from C]}

\index{Dalston , Carlisle@Dalston, \emph{Carlisle}}

Resignation of Nicola Clarke (C).

\noindent
\begin{tabular*}{\columnwidth}{@{\extracolsep{\fill}} p{0.545\columnwidth} >{\itshape}l r @{\extracolsep{\fill}}}
Michael Gee & LD & 506\\
Michael Randall & C & 476\\
Ruth Alcroft & Lab & 186\\
Robert Dickinson & UKIP & 167\\
James Tucker & Grn & 27\\
\end{tabular*}

\council{Copeland}

\subsubsection*{Hensingham \hspace*{\fill}\nolinebreak[1]%
\enspace\hspace*{\fill}
\finalhyphendemerits=0
[2nd May]}

\index{Hensingham , Copeland@Hensingham, \emph{Copeland}}

Resignation of Norman Williams (Lab).

\noindent
\begin{tabular*}{\columnwidth}{@{\extracolsep{\fill}} p{0.545\columnwidth} >{\itshape}l r @{\extracolsep{\fill}}}
Allan Forster & Lab & 670\\
Genna Martin & C & 223\\
\end{tabular*}

\council{South Lakeland}

\subsubsection*{Windermere Bowness North \hspace*{\fill}\nolinebreak[1]%
\enspace\hspace*{\fill}
\finalhyphendemerits=0
[29th August]}

\index{Windermere Bowness North , South Lakeland@Windermere Bowness N., \emph{S. Lakeland}}

Resignation of Hilary Stephenson (LD).

\noindent
\begin{tabular*}{\columnwidth}{@{\extracolsep{\fill}} p{0.545\columnwidth} >{\itshape}l r @{\extracolsep{\fill}}}
Colin Jones & LD & 431\\
Ian Keeling & C & 248\\
Rae Cross & Lab & 29\\
\end{tabular*}

\subsubsection*{Levens \hspace*{\fill}\nolinebreak[1]%
\enspace\hspace*{\fill}
\finalhyphendemerits=0
[17th October]}

\index{Levens , South Lakeland@Levens, \emph{S. Lakeland}}

Resignation of Mary Orr (LD).

\noindent
\begin{tabular*}{\columnwidth}{@{\extracolsep{\fill}} p{0.545\columnwidth} >{\itshape}l r @{\extracolsep{\fill}}}
Annie Rawlinson & LD & 569\\
Brian Rendell & C & 430\\
\end{tabular*}

\section{Derbyshire}

\subsection*{County Council}\index{Derbyshire}

At the May 2013 ordinary election there was an unfilled vacancy in Buxton West division due to the disqualification (suspended prison sentence, theft of subsistence allowances) of Robin Baldry (C).

\index{Buxton West , Derbyshire@Buxton W., \emph{Derbys.}}

\council{Amber Valley}

\subsubsection*{Codnor and Waingroves \hspace*{\fill}\nolinebreak[1]%
\enspace\hspace*{\fill}
\finalhyphendemerits=0
[1st August]}

\index{Codnor and Waingroves , Amber Valley@Codnor \& Waingroves, \emph{Amber Valley}}

Death of George Parkes (Lab).

\noindent
\begin{tabular*}{\columnwidth}{@{\extracolsep{\fill}} p{0.5\columnwidth} >{\itshape}l r @{\extracolsep{\fill}}}
Isobel Harry & Lab & 557\\
Garry Smith & UKIP & 250\\
Ron Ashton & C & 219\\
Keith Falconbridge & LD & 39\\
\end{tabular*}

\council{Bolsover}

Whitwell = Whitwell Residents Association

\subsubsection*{Whitwell \hspace*{\fill}\nolinebreak[1]%
\enspace\hspace*{\fill}
\finalhyphendemerits=0
[23rd May; Whitwell gain from Grn]}

\index{Whitwell , Bolsover@Whitwell, \emph{Bolsover}}

Resignation of Duncan Kerr (Grn).

\noindent
\begin{tabular*}{\columnwidth}{@{\extracolsep{\fill}} p{0.5\columnwidth} >{\itshape}l r @{\extracolsep{\fill}}}
Viv Mills & Whitwell & 347\\
Frank Raspin & Lab & 256\\
\end{tabular*}

\council{Chesterfield}

\subsubsection*{St Helen's \hspace*{\fill}\nolinebreak[1]%
\enspace\hspace*{\fill}
\finalhyphendemerits=0
[2nd May]}

\index{Saint Helen's , Chesterfield@St Helen's, \emph{Chesterfield}}

Resignation of Angela Reynolds (Lab).

\noindent
\begin{tabular*}{\columnwidth}{@{\extracolsep{\fill}} p{0.545\columnwidth} >{\itshape}l r @{\extracolsep{\fill}}}
Helen Bagley & Lab & 602\\
Keith Lomas & UKIP & 246\\
Daniel Marshall & LD & 135\\
Michael Bagshaw & Ind & 46\\
Shelley Dale & C & 35\\
\end{tabular*}

\subsubsection*{St Leonard's \hspace*{\fill}\nolinebreak[1]%
\enspace\hspace*{\fill}
\finalhyphendemerits=0
[2nd May]}

\index{Saint Leonard's , Chesterfield@St Leonard's, \emph{Chesterfield}}

Resignation of Nick Stringer (Lab).

\noindent
\begin{tabular*}{\columnwidth}{@{\extracolsep{\fill}} p{0.545\columnwidth} >{\itshape}l r @{\extracolsep{\fill}}}
Linda Clarke & Lab & 904\\
Stuart Yeowart & UKIP & 301\\
Adrian Mather & Ind & 294\\
Margaret Cannon & LD & 164\\
Simon Temperton & C & 120\\
\end{tabular*}

\section{Devon}

\council{East Devon}

\subsubsection*{Feniton and Buckerell \hspace*{\fill}\nolinebreak[1]%
\enspace\hspace*{\fill}
\finalhyphendemerits=0
[2nd May; Ind gain from C]}

\index{Feniton and Buckerell , East Devon@Feniton \& Buckerell, \emph{E. Devon}}

Resignation of Graham Brown (C).

\noindent
\begin{tabular*}{\columnwidth}{@{\extracolsep{\fill}} p{0.545\columnwidth} >{\itshape}l r @{\extracolsep{\fill}}}
Susan Bond & Ind & 772\\
Andrew Dinnis & C & 113\\
\end{tabular*}

\council{Mid Devon}

\subsubsection*{Way \hspace*{\fill}\nolinebreak[1]%
\enspace\hspace*{\fill}
\finalhyphendemerits=0
[26th September]}

\index{Way , Mid Devon@Way, \emph{Mid Devon}}

Resignation of Sarah Fox (C).

\noindent
\begin{tabular*}{\columnwidth}{@{\extracolsep{\fill}} p{0.545\columnwidth} >{\itshape}l r @{\extracolsep{\fill}}}
Cathryn Heal & C & 189\\
Judi Binks & LD & 130\\
Bob Edwards & UKIP & 60\\
John Jordan & Ind & 15\\
\end{tabular*}

\council{Plymouth}

\subsubsection*{Southway \hspace*{\fill}\nolinebreak[1]%
\enspace\hspace*{\fill}
\finalhyphendemerits=0
[27th June; Lab gain from C]}

\index{Southway , Plymouth@Southway, \emph{Plymouth}}

Disqualification (non-attendance) of Tom Browne (C).

\noindent
\begin{tabular*}{\columnwidth}{@{\extracolsep{\fill}} p{0.545\columnwidth} >{\itshape}l r @{\extracolsep{\fill}}}
Jonny Morris & Lab & 1247\\
Peter Berrow & UKIP & 764\\
David Downie & C & 487\\
Dennis Silverwood & Ind & 290\\
Justin Stafford & LD & 82\\
Ryan Aldred & TUSC & 22\\
\end{tabular*}

\council{Teignbridge}

\subsubsection*{Bovey \hspace*{\fill}\nolinebreak[1]%
\enspace\hspace*{\fill}
\finalhyphendemerits=0
[24th October]}

\index{Bovey , Teignbridge@Bovey, \emph{Teignbridge}}

Resignation of Fernley Holmes (Ind elected as C).

\noindent
\begin{tabular*}{\columnwidth}{@{\extracolsep{\fill}} p{0.545\columnwidth} >{\itshape}l r @{\extracolsep{\fill}}}
Avril Kerswell & C & 933\\
Charlie West & LD & 472\\
Bruce Meechan & UKIP & 253\\
Lisa Robillard Webb & Lab & 196\\
\end{tabular*}

\council{Torridge}

\subsubsection*{Shebbear and Langtree \hspace*{\fill}\nolinebreak[1]%
\enspace\hspace*{\fill}
\finalhyphendemerits=0
[15th August]}

\index{Shebbear and Langtree , Torridge@Shebbear \& Langtree, \emph{Torridge}}

Resignation of John Lewis (C).

\noindent
\begin{tabular*}{\columnwidth}{@{\extracolsep{\fill}} p{0.545\columnwidth} >{\itshape}l r @{\extracolsep{\fill}}}
David Hurley & C & 240\\
Penny Mills & UKIP & 217\\
Colin Jones & Grn & 41\\
Bob Wootton & Ind & 10\\
\end{tabular*}

\subsubsection*{Torrington \hspace*{\fill}\nolinebreak[1]%
\enspace\hspace*{\fill}
\finalhyphendemerits=0
[5th September; Grn gain from LD]}

\index{Torrington , Torridge@Torrington, \emph{Torridge}}

Resignation of Geoff Lee (LD).

\noindent
\begin{tabular*}{\columnwidth}{@{\extracolsep{\fill}} p{0.545\columnwidth} >{\itshape}l r @{\extracolsep{\fill}}}
Cathrine Simmons & Grn & 292\\
Robin Julian & UKIP & 181\\
David Cox & Ind & 160\\
Adrian Freeland & Ind & 106\\
Phil Pester & C & 88\\
\end{tabular*}

\section{Dorset}

\council{Bournemouth}

\subsubsection*{Winton East \hspace*{\fill}\nolinebreak[1]%
\enspace\hspace*{\fill}
\finalhyphendemerits=0
[14th November]}

\index{Winton East , Bournemouth@Winton E., \emph{Bournemouth}}

Resignation of Anniina Davie (C).

\noindent
\begin{tabular*}{\columnwidth}{@{\extracolsep{\fill}} p{0.545\columnwidth} >{\itshape}l r @{\extracolsep{\fill}}}
Pat Oakley & C & 503\\
Mike Goff & Lab & 215\\
Laurence Fear & UKIP & 212\\
Matthew Gillett & LD & 191\\
Sandra Hale & Grn & 48\\
Kathleen Mortimer & Ind & 34\\
\end{tabular*}

\council{North Dorset}

\subsubsection*{Lodbourne \hspace*{\fill}\nolinebreak[1]%
\enspace\hspace*{\fill}
\finalhyphendemerits=0
[21st March; LD gain from C]}

\index{Lodbourne , North Dorset@Lodbourne, \emph{N. Dorset}}

Resignation of Helen Webb (C).

\noindent
\begin{tabular*}{\columnwidth}{@{\extracolsep{\fill}} p{0.545\columnwidth} >{\itshape}l r @{\extracolsep{\fill}}}
Richard Arnold & LD & 187\\
Mike Gould & C & 134\\
Bob Messer & Lab & 69\\
\end{tabular*}

\subsubsection*{The Stours \hspace*{\fill}\nolinebreak[1]%
\enspace\hspace*{\fill}
\finalhyphendemerits=0
[21st March]}

\index{Stours , North Dorset@The Stours, \emph{N. Dorset}}

Resignation of Peter Webb (C).

\noindent
\begin{tabular*}{\columnwidth}{@{\extracolsep{\fill}} p{0.545\columnwidth} >{\itshape}l r @{\extracolsep{\fill}}}
Traci Handford & C & 207\\
Joseph Pestell & Lab & 51\\
\end{tabular*}

\council{West Dorset}

\subsubsection*{Cam Vale \hspace*{\fill}\nolinebreak[1]%
\enspace\hspace*{\fill}
\finalhyphendemerits=0
[2nd May]}

\index{Cam Vale , West Dorset@Cam Vale, \emph{W. Dorset}}

Death of Richard Jungius (C).

\noindent
\begin{tabular*}{\columnwidth}{@{\extracolsep{\fill}} p{0.545\columnwidth} >{\itshape}l r @{\extracolsep{\fill}}}
Christopher Loder & C & 335\\
Michael Sandy & LD & 252\\
Peter Jenkins & UKIP & 192\\
\end{tabular*}

\council{Weymouth and Portland}

\subsubsection*{Melcombe Regis \hspace*{\fill}\nolinebreak[1]%
\enspace\hspace*{\fill}
\finalhyphendemerits=0
[16th May; Lab gain from C]}

\index{Melcombe Regis , Weymouth and Portland@Melcombe Regis, \emph{Weymouth \& Portland}}

Death of Peter Farrell (C).

\noindent
\begin{tabular*}{\columnwidth}{@{\extracolsep{\fill}} p{0.545\columnwidth} >{\itshape}l r @{\extracolsep{\fill}}}
Stewart Pearson & Lab & 279\\
Andrew Manvell & C & 258\\
Jim Williamson & Ind & 204\\
Steph Taylor & LD & 170\\
Jon Orrell & Grn & 143\\
\end{tabular*}



\section{Durham}

\council{Darlington}

\subsubsection*{Lascelles \hspace*{\fill}\nolinebreak[1]%
\enspace\hspace*{\fill}
\finalhyphendemerits=0
[11th April]}

\index{Lascelles , Darlington@Lascelles, \emph{Darlington}}

Death of Jacqui Maddison (Lab).

\noindent
\begin{tabular*}{\columnwidth}{@{\extracolsep{\fill}} p{0.545\columnwidth} >{\itshape}l r @{\extracolsep{\fill}}}
Helen Crumbie & Lab & 426\\
Howard Jones & LD & 129\\
Lewis Cairns & C & 117\\
\end{tabular*}

\council{Durham}

At the May 2013 ordinary election there was an unfilled vacancy in Aycliffe West division due to the death of Enid Paylor (Lab).

\sloppyword{WVIG = Wear Valley Independent Group}

\index{Aycliffe West , Durham@Aycliffe W., \emph{Durham}}

\subsubsection*{Crook \hspace*{\fill}\nolinebreak[1]%
\enspace\hspace*{\fill}
\finalhyphendemerits=0
[7th November]}

\index{Crook , Durham@Crook, \emph{Durham}}

Death of Geoff Mowbray (Lab).

\noindent
\begin{tabular*}{\columnwidth}{@{\extracolsep{\fill}} p{0.545\columnwidth} >{\itshape}l r @{\extracolsep{\fill}}}
Andrea Patterson & Lab & 741\\
Ian Hirst & Ind & 496\\
John Bailey & WVIG & 360\\
David English & LD & 145\\
Beaty Bainbridge & C & 54\\
Joanne Yelland & Grn & 40\\
\end{tabular*}

\council{Hartlepool}

PHF = Putting Hartlepool First

\subsubsection*{Manor House \hspace*{\fill}\nolinebreak[1]%
\enspace\hspace*{\fill}
\finalhyphendemerits=0
[15th August]}

\index{Manor House , Hartlepool@Manor House, \emph{Hartlepool}}

Resignation of Angie Wilcox (Lab).

\noindent
\begin{tabular*}{\columnwidth}{@{\extracolsep{\fill}} p{0.545\columnwidth} >{\itshape}l r @{\extracolsep{\fill}}}
Allan Barclay & Lab & 639\\
Tom Hind & UKIP & 226\\
Mick Stevens & PHF & 194\\
Mandy Loynes & C & 74\\
\end{tabular*}

\council{Stockton-on-Tees}

TIA = Thornaby Independent Association

\subsubsection*{Village \hspace*{\fill}\nolinebreak[1]%
\enspace\hspace*{\fill}
\finalhyphendemerits=0
[7th February]}

\index{Village , Stockton-on-Tees@Village, \emph{Stockton-on-Tees}}

Death of Mick Eddy (TIA).

\noindent
\begin{tabular*}{\columnwidth}{@{\extracolsep{\fill}} p{0.545\columnwidth} >{\itshape}l r @{\extracolsep{\fill}}}
Mick Moore & TIA & 800\\
Leslie Hodge & Lab & 270\\
Ted Strike & UKIP & 135\\
John Chapman & C & 85\\
Isabel Willis & LD & 18\\
\end{tabular*}

\section{East Sussex}

\council{Brighton and Hove}

\subsubsection*{Hanover and Elm Grove \hspace*{\fill}\nolinebreak[1]%
\enspace\hspace*{\fill}
\finalhyphendemerits=0
[11th July; Lab gain from Grn]}

\index{Hanover and Elm Grove , Brighton and Hove@Hanover \& Elm Grove, \emph{Brighton \& Hove}}

Resignation of Matt Follett (Grn).

\noindent
\begin{tabular*}{\columnwidth}{@{\extracolsep{\fill}} p{0.545\columnwidth} >{\itshape}l r @{\extracolsep{\fill}}}
Emma Daniel & Lab & 1396\\
David Gibson & Grn & 1358\\
Robert Knight & C & 275\\
Patricia Mountain & UKIP & 250\\
Phil Clarke & TUSC & 172\\
Lev Eakins & LD & 56\\
\end{tabular*}

\section{East Yorkshire}

\council{East Riding}

\subsubsection*{Howdenshire \hspace*{\fill}\nolinebreak[1]%
\enspace\hspace*{\fill}
\finalhyphendemerits=0
[2nd May]}

\index{Howdenshire , East Riding@Howdenshire, \emph{E. Riding}}

Resignation of Paul Robinson (C).

\noindent
\begin{tabular*}{\columnwidth}{@{\extracolsep{\fill}} p{0.545\columnwidth} >{\itshape}l r @{\extracolsep{\fill}}}
Victoria Aitken & C & 1573\\
Clive Waddington & UKIP & 1260\\
Mike Whitley & Lab & 563\\
Alan Luckraft & LD & 212\\
\end{tabular*}

\subsubsection*{Mid Holderness \hspace*{\fill}\nolinebreak[1]%
\enspace\hspace*{\fill}
\finalhyphendemerits=0
[2nd May]}

\index{Mid Holderness , East Riding@Mid Holderness, \emph{E. Riding}}

Resignation of Matthew Grove (C).

\noindent
\begin{tabular*}{\columnwidth}{@{\extracolsep{\fill}} p{0.545\columnwidth} >{\itshape}l r @{\extracolsep{\fill}}}
John Holtby & C & 1353\\
Gary Shores & UKIP & 1269\\
George McManus & Lab & 852\\
\end{tabular*}

\section{Essex}

\subsection*{County Council}\index{Essex}

At the May 2013 ordinary election there was an unfilled vacancy in Rochford South division due to the death of Roy Pearson (C).

\index{Rochford South , Essex@Rochford S., \emph{Essex}}

\council{Basildon}

\subsubsection*{Wickford Castledon \hspace*{\fill}\nolinebreak[1]%
\enspace\hspace*{\fill}
\finalhyphendemerits=0
[2nd May; UKIP gain from C]}

\index{Wickford Castledon , Basildon@Wickford Castledon, \emph{Basildon}}

Death of Sylvia Buckley (C).

\noindent
\begin{tabular*}{\columnwidth}{@{\extracolsep{\fill}} p{0.545\columnwidth} >{\itshape}l r @{\extracolsep{\fill}}}
Nigel le Gresley & UKIP & 637\\
Jennie Jackman & C & 589\\
Alan Ball & Ind & 288\\
Albert Ede & Lab & 179\\
Philip Jenkins & LD & 32\\
Philip Howell & BNP & 24\\
Thomas Beaney & NF & 7\\
\end{tabular*}

\subsubsection*{Billericay East \hspace*{\fill}\nolinebreak[1]%
\enspace\hspace*{\fill}
\finalhyphendemerits=0
[27th June]}

\index{Billericay East , Basildon@Billericay E., \emph{Basildon}}

Death of Tony Archer (C).

\noindent
\begin{tabular*}{\columnwidth}{@{\extracolsep{\fill}} p{0.545\columnwidth} >{\itshape}l r @{\extracolsep{\fill}}}
Andrew Schrader & C & 790\\
Terry Gandy & UKIP & 464\\
Lauren Brown & Lab & 170\\
Nigel Horn & LD & 128\\
Thomas Beaney & NF & 3\\
\end{tabular*}

\council{Braintree}

\subsubsection*{Braintree East \hspace*{\fill}\nolinebreak[1]%
\enspace\hspace*{\fill}
\finalhyphendemerits=0
[25th July]}

\index{Braintree East , Braintree@Braintree E., \emph{Braintree}}

Death of Eric Lynch (Lab).

\noindent
\begin{tabular*}{\columnwidth}{@{\extracolsep{\fill}} p{0.545\columnwidth} >{\itshape}l r @{\extracolsep{\fill}}}
Celia Burne & Lab & 461\\
Jennifer Smith & C & 267\\
Philip Palij & UKIP & 194\\
John Malam & Grn & 67\\
\end{tabular*}



\council{Chelmsford}

SWFInd = South Woodham Ferrers Independents

\subsubsection*{Broomfield \& The Walthams \hspace*{\fill}\nolinebreak[1]%
\enspace\hspace*{\fill}
\finalhyphendemerits=0
[Tuesday 19th February; LD gain from Ind]}

\index{Broomfield and Walthams , Chelmsford@Broomfield \& The Walthams, \emph{Chelmsford}}

Death of Delmas Ashford (Ind).

\noindent
\begin{tabular*}{\columnwidth}{@{\extracolsep{\fill}} p{0.545\columnwidth} >{\itshape}l r @{\extracolsep{\fill}}}
Graham Pooley & LD & 543\\
William Wetton & C & 423\\
Ian Nicholls & UKIP & 280\\
Sinead Jein & Lab & 129\\
Reza Hossain & Grn & 47\\
\end{tabular*}

\subsubsection*{South Woodham - Elmwood and Woodville  \hspace*{\fill}\nolinebreak[1]%
\enspace\hspace*{\fill}
\finalhyphendemerits=0
[5th December; SWFInd gain from C]}

\index{South Woodham Elmwood and Woodville , Chelmsford@South Woodham - Elmwood \& Woodville, \emph{Chelmsford}}

Disqualification (non-attendance) of Maureen Moulds (C).

\noindent
\begin{tabular*}{\columnwidth}{@{\extracolsep{\fill}} p{0.53\columnwidth} >{\itshape}l r @{\extracolsep{\fill}}}
Ian Roberts & SWFInd & 281\\
Linda Denston & C & 275\\
Ian Nicholls & UKIP & 249\\
Derek Barnett & Lab & 65\\
Jeni Goldfinch & LD & 24\\
\end{tabular*}

\council{Colchester}

\subsubsection*{West Mersea \hspace*{\fill}\nolinebreak[1]%
\enspace\hspace*{\fill}
\finalhyphendemerits=0
[2nd May]}

\index{West Mersea , Colchester@West Mersea, \emph{Colchester}}

Resignation of Glenn Granger (C).

\noindent
\begin{tabular*}{\columnwidth}{@{\extracolsep{\fill}} p{0.545\columnwidth} >{\itshape}l r @{\extracolsep{\fill}}}
Peter Sheane & C & 1208\\
Bernard Ready & Lab & 288\\
Sue Bailey & Grn & 208\\
Owen Bartholomew & LD & 100\\
\end{tabular*}

\council{Epping Forest}

\subsubsection*{Waltham Abbey Honey Lane \hspace*{\fill}\nolinebreak[1]%
\enspace\hspace*{\fill}
\finalhyphendemerits=0
[2nd May; UKIP gain from C]}

\index{Waltham Abbey Honey Lane , Epping Forest@Waltham Abbey Honey Lane, \emph{Epping Forest}}

Resignation of David Johnson (C).

\noindent
\begin{tabular*}{\columnwidth}{@{\extracolsep{\fill}} p{0.545\columnwidth} >{\itshape}l r @{\extracolsep{\fill}}}
Rod Butler & UKIP & 465\\
Christine Ball & C & 455\\
\end{tabular*}

\council{Rochford}

\subsubsection*{Hawkwell North \hspace*{\fill}\nolinebreak[1]%
\enspace\hspace*{\fill}
\finalhyphendemerits=0
[2nd May]}

\index{Hawkwell North , Rochford@Hawkwell N., \emph{Rochford}}

Resignation of Robert Pointer (C).

\noindent
\begin{tabular*}{\columnwidth}{@{\extracolsep{\fill}} p{0.545\columnwidth} >{\itshape}l r @{\extracolsep{\fill}}}
Lesley Butcher & C & 372\\
Keith Gibbs & UKIP & 311\\
Arthur Williams & Ind & 173\\
John Jefferies & Lab & 133\\
\end{tabular*}

\subsubsection*{Whitehouse \hspace*{\fill}\nolinebreak[1]%
\enspace\hspace*{\fill}
\finalhyphendemerits=0
[2nd May]}

\index{Whitehouse , Rochford@Whitehouse, \emph{Rochford}}

Resignation of Peter Webster (C).

\noindent
\begin{tabular*}{\columnwidth}{@{\extracolsep{\fill}} p{0.545\columnwidth} >{\itshape}l r @{\extracolsep{\fill}}}
Robin Dray & C & 408\\
Linda Kendall & UKIP & 401\\
David Bodimeade & Lab & 137\\
\end{tabular*}



\council{Tendring}

CRP = Community Representatives Party

Tendring = Tendring First

\subsubsection*{Harwich West \hspace*{\fill}\nolinebreak[1]%
\enspace\hspace*{\fill}
\finalhyphendemerits=0
[28th March]}

\index{Harwich West , Tendring@Harwich W., \emph{Tendring}}

Death of Les Double (Lab).

\noindent
\begin{tabular*}{\columnwidth}{@{\extracolsep{\fill}} p{0.545\columnwidth} >{\itshape}l r @{\extracolsep{\fill}}}
John Hawkins & Lab & 282\\
Hugh Thompson & C & 220\\
Steven Henderson & CRP & 163\\
Simon Banks & LD & 143\\
\end{tabular*}

\subsubsection*{St James \hspace*{\fill}\nolinebreak[1]%
\enspace\hspace*{\fill}
\finalhyphendemerits=0
[26th September]}

\index{Saint James , Tendring@St James, \emph{Tendring}}

Death of Gill Downing (C).

\noindent
\begin{tabular*}{\columnwidth}{@{\extracolsep{\fill}} p{0.51\columnwidth} >{\itshape}l r @{\extracolsep{\fill}}}
Andy Wood & C & 445\\
Susan Shearing & UKIP & 196\\
Dave Bolton & Lab & 135\\
Mark Stephenson & Tendring & 82\\
Amanda Peters & LD & 35\\
\end{tabular*}

\council{Thurrock}

\subsubsection*{Stifford Clays \hspace*{\fill}\nolinebreak[1]%
\enspace\hspace*{\fill}
\finalhyphendemerits=0
[17th October]}

\index{Stifford Clays , Thurrock@Stifford Clays, \emph{Thurrock}}

Death of Diana Hale (Lab).

\noindent
\begin{tabular*}{\columnwidth}{@{\extracolsep{\fill}} p{0.51\columnwidth} >{\itshape}l r @{\extracolsep{\fill}}}
Susan Shinnick & Lab & 646\\
Danny Nicklen & C & 570\\
Clive Broad & UKIP & 504\\
Kevin Mulroue & LD & 35\\
\end{tabular*}

\council{Uttlesford}

\subsubsection*{Newport \hspace*{\fill}\nolinebreak[1]%
\enspace\hspace*{\fill}
\finalhyphendemerits=0
[2nd May; Ind gain from LD]}

\index{Newport , Uttlesford@Newport, \emph{Uttlesford}}

Resignation of Peter Wilcock (LD).

\noindent
\begin{tabular*}{\columnwidth}{@{\extracolsep{\fill}} p{0.545\columnwidth} >{\itshape}l r @{\extracolsep{\fill}}}
Joanna Parry & Ind & 428\\
Peter Arscott & C & 380\\
Howard Bowman & LD & 240\\
Denis Mongon & Lab & 125\\
\end{tabular*}

\subsubsection*{Felsted \hspace*{\fill}\nolinebreak[1]%
\enspace\hspace*{\fill}
\finalhyphendemerits=0
[25th July]}

\index{Felsted , Uttlesford@Felsted, \emph{Uttlesford}}

Disqualification (non-attendance) of David Crome (C).

\noindent
\begin{tabular*}{\columnwidth}{@{\extracolsep{\fill}} p{0.545\columnwidth} >{\itshape}l r @{\extracolsep{\fill}}}
Marie Felton & C & 557\\
Antoinette Wattebot & LD & 253\\
Alan Stannard & UKIP & 181\\
Yad Zanganah & Lab & 38\\
\end{tabular*}

\section{Gloucestershire}

\council{Cheltenham}

\subsubsection*{Warden Hill \hspace*{\fill}\nolinebreak[1]%
\enspace\hspace*{\fill}
\finalhyphendemerits=0
[2nd May]}

\index{Warden Hill , Cheltenham@Warden Hill, \emph{Cheltenham}}

Resignation of Josephine Tinkle (C).

\noindent
\begin{tabular*}{\columnwidth}{@{\extracolsep{\fill}} p{0.545\columnwidth} >{\itshape}l r @{\extracolsep{\fill}}}
Chris Ryder & C & 852\\
Tony Oliver & LD & 735\\
\end{tabular*}

\council{Forest of Dean}

\subsubsection*{Berry Hill \hspace*{\fill}\nolinebreak[1]%
\enspace\hspace*{\fill}
\finalhyphendemerits=0
[7th February]}

\index{Berry Hill , Forest of Dean@Berry Hill, \emph{Forest of Dean}}

Death of Helen Stewart (Lab).

\noindent
\begin{tabular*}{\columnwidth}{@{\extracolsep{\fill}} p{0.545\columnwidth} >{\itshape}l r @{\extracolsep{\fill}}}
Timothy Gwilliam & Lab & 276\\
Nigel Bluett & C & 101\\
John McOwan & UKIP & 85\\
\end{tabular*}

\subsubsection*{Bromesberrow and Dymock \hspace*{\fill}\nolinebreak[1]%
\enspace\hspace*{\fill}
\finalhyphendemerits=0
[2nd May; UKIP gain from C]}

\index{Bromesberrow and Dymock , Forest of Dean@Bromesberrow \& Dymock, \emph{Forest of Dean}}

Resignation of Jim Connell (C).

\noindent
\begin{tabular*}{\columnwidth}{@{\extracolsep{\fill}} p{0.545\columnwidth} >{\itshape}l r @{\extracolsep{\fill}}}
Simon Roberts & UKIP & 308\\
Mike Rees & C & 259\\
\end{tabular*}

\subsubsection*{Coleford East \hspace*{\fill}\nolinebreak[1]%
\enspace\hspace*{\fill}
\finalhyphendemerits=0
[26th September]}

\index{Coleford East , Forest of Dean@Coleford E., \emph{Forest of Dean}}

Death of Frank Raynham (Lab).

\noindent
\begin{tabular*}{\columnwidth}{@{\extracolsep{\fill}} p{0.545\columnwidth} >{\itshape}l r @{\extracolsep{\fill}}}
Tanya Palmer & Lab & 289\\
Alan Grant & UKIP & 227\\
Harry Ives & C & 104\\
Heather Lusty & LD & 80\\
Keith Aburrow & Ind & 76\\
\end{tabular*}

\subsubsection*{Redmarley \hspace*{\fill}\nolinebreak[1]%
\enspace\hspace*{\fill}
\finalhyphendemerits=0
[26th September]}

\index{Redmarley , Forest of Dean@Redmarley, \emph{Forest of Dean}}

Death of Peter Ede (C).

\noindent
\begin{tabular*}{\columnwidth}{@{\extracolsep{\fill}} p{0.545\columnwidth} >{\itshape}l r @{\extracolsep{\fill}}}
Clayton Williams & C & 332\\
Alex Tritton & UKIP & 119\\
Andy Hewlett & Lab & 56\\
\end{tabular*}

\council{Stroud}

\subsubsection*{Cam East \hspace*{\fill}\nolinebreak[1]%
\enspace\hspace*{\fill}
\finalhyphendemerits=0
[2nd May; Lab gain from C]}

\index{Cam East , Stroud@Cam E., \emph{Stroud}}

Resignation of John Hudson (Ind elected as C).

\noindent
\begin{tabular*}{\columnwidth}{@{\extracolsep{\fill}} p{0.545\columnwidth} >{\itshape}l r @{\extracolsep{\fill}}}
Miranda Clifton & Lab & 654\\
Loraine Patrick & C & 575\\
\end{tabular*}

\section{Hampshire}

\council{East Hampshire}

\subsubsection*{Liss \hspace*{\fill}\nolinebreak[1]%
\enspace\hspace*{\fill}
\finalhyphendemerits=0
[2nd May]}

\index{Liss , East Hampshire@Liss, \emph{E. Hants.}}

Resignation of Gina Logan (C).

\noindent
\begin{tabular*}{\columnwidth}{@{\extracolsep{\fill}} p{0.545\columnwidth} >{\itshape}l r @{\extracolsep{\fill}}}
Richard Harris & C & 593\\
Donald Jerrard & UKIP & 350\\
Roger Mullenger & LD & 224\\
Keith Budden & Lab & 193\\
\end{tabular*}

\subsubsection*{Four Marks and Medstead \hspace*{\fill}\nolinebreak[1]%
\enspace\hspace*{\fill}
\finalhyphendemerits=0
[19th September]}

\index{Four Marks and Medstead , East Hampshire@Four Marks \& Medstead, \emph{E. Hants.}}

Resignation of Pat Seward (C).

\noindent
\begin{tabular*}{\columnwidth}{@{\extracolsep{\fill}} p{0.545\columnwidth} >{\itshape}l r @{\extracolsep{\fill}}}
Ingrid Thomas & C & 749\\
Ruth Duffin & UKIP & 348\\
Janice Treacher & Lab & 119\\
Marjorie Pooley & Grn & 73\\
\end{tabular*}

\council{Havant}

\subsubsection*{Bedhampton \hspace*{\fill}\nolinebreak[1]%
\enspace\hspace*{\fill}
\finalhyphendemerits=0
[2nd May]}

\index{Bedhampton , Havant@Bedhampton, \emph{Havant}}

Resignation of Jenny Wride (C).

\noindent
\begin{tabular*}{\columnwidth}{@{\extracolsep{\fill}} p{0.545\columnwidth} >{\itshape}l r @{\extracolsep{\fill}}}
David Smith & C & 766\\
Ann Brown & LD & 652\\
Stephen Harris & UKIP & 584\\
Anthony Berry & Lab & 191\\
Terry Mitchell & Grn & 124\\
\end{tabular*}

\subsubsection*{Emsworth \hspace*{\fill}\nolinebreak[1]%
\enspace\hspace*{\fill}
\finalhyphendemerits=0
[2nd May]}

\index{Emsworth , Havant@Emsworth, \emph{Havant}}

Resignation of David Gillett (C).

\noindent
\begin{tabular*}{\columnwidth}{@{\extracolsep{\fill}} p{0.545\columnwidth} >{\itshape}l r @{\extracolsep{\fill}}}
Colin Mackey & C & 1296\\
Ian Reddoch & UKIP & 604\\
Susan Kelly & Grn & 353\\
Christine Armitage & Lab & 350\\
Roisin Miller & LD & 234\\
\end{tabular*}

\subsubsection*{Waterloo \hspace*{\fill}\nolinebreak[1]%
\enspace\hspace*{\fill}
\finalhyphendemerits=0
[24th October]}

\index{Waterloo , Havant@Waterloo, \emph{Havant}}

Resignation of John Hunt (C).

\noindent
\begin{tabular*}{\columnwidth}{@{\extracolsep{\fill}} p{0.545\columnwidth} >{\itshape}l r @{\extracolsep{\fill}}}
Peter Wade & C & 693\\
David Crichton & LD & 446\\
Gary Kerrin & UKIP & 296\\
Anthony Berry & Lab & 129\\
\end{tabular*}

\council{Southampton}

\subsubsection*{Woolston \hspace*{\fill}\nolinebreak[1]%
\enspace\hspace*{\fill}
\finalhyphendemerits=0
[13th June]}

\index{Woolston , Southampton@Woolston, \emph{Southampton}}

Resignation of Richard Williams (Lab).

\noindent
\begin{tabular*}{\columnwidth}{@{\extracolsep{\fill}} p{0.545\columnwidth} >{\itshape}l r @{\extracolsep{\fill}}}
Christopher Hammond & Lab & 864\\
John Sharp & UKIP & 741\\
Alex Houghton & C & 704\\
Sue Atkins & TUSC & 136\\
Adrian Ford & LD & 120\\
Christopher Bluemel & Grn & 107\\
\end{tabular*}

\section{Herefordshire}
\index{Herefordshire}

IOCH = It's Our County (Herefordshire)

\subsubsection*{Ross-on-Wye West \hspace*{\fill}\nolinebreak[1]%
\enspace\hspace*{\fill}
\finalhyphendemerits=0
[24th January]}

\index{Ross-on-Wye West , Herefordshire@Ross-on-Wye W., \emph{Herefs.}}

Death of Gordon Lucas (C).

\noindent
\begin{tabular*}{\columnwidth}{@{\extracolsep{\fill}} p{0.545\columnwidth} >{\itshape}l r @{\extracolsep{\fill}}}
Richard Mayo & C & 695\\
David Ravenscroft & Ind & 312\\
Caroline Bennett & LD & 270\\
\end{tabular*}

\subsubsection*{Tupsley \hspace*{\fill}\nolinebreak[1]%
\enspace\hspace*{\fill}
\finalhyphendemerits=0
[7th November]}

\index{Tupsley , Herefordshire@Tupsley, \emph{Herefs.}}

\sloppyword{Resignation of Alex Hempton-Smith (IOCH).}

\noindent
\begin{tabular*}{\columnwidth}{@{\extracolsep{\fill}} p{0.545\columnwidth} >{\itshape}l r @{\extracolsep{\fill}}}
Cath North & IOCH & 987\\
Jason Kay & C & 347\\
Duncan Fraser & LD & 277\\
\end{tabular*}

\subsubsection*{Pontrilas \hspace*{\fill}\nolinebreak[1]%
\enspace\hspace*{\fill}
\finalhyphendemerits=0
[21st November; IOCH gain from C]}

\index{Pontrilas , Herefordshire@Pontrilas, \emph{Herefs.}}

Resignation of Russell Hamilton (C).

\noindent
\begin{tabular*}{\columnwidth}{@{\extracolsep{\fill}} p{0.545\columnwidth} >{\itshape}l r @{\extracolsep{\fill}}}
Jon Norris & IOCH & 429\\
Elaine Godding & Ind & 261\\
Elissa Swinglehurst & C & 229\\
\end{tabular*}

\section{Hertfordshire}

\subsection*{County Council}\index{Hertfordshire}

\subsubsection*{Hitchin North \hspace*{\fill}\nolinebreak[1]%
\enspace\hspace*{\fill}
\finalhyphendemerits=0
[12th September]}

\index{Hitchin North , Hertfordshire@Hitchin N., \emph{Herts.}}

Death of David Billing (Lab).

\noindent
\begin{tabular*}{\columnwidth}{@{\extracolsep{\fill}} p{0.545\columnwidth} >{\itshape}l r @{\extracolsep{\fill}}}
Judi Billing & Lab & 1250\\
Alan Millard & C & 673\\
Lisa Courts & LD & 246\\
John Barry & UKIP & 235\\
Gavin Nicholson & Grn & 212\\
\end{tabular*}

\council{Dacorum}

\subsubsection*{Adeyfield West \hspace*{\fill}\nolinebreak[1]%
\enspace\hspace*{\fill}
\finalhyphendemerits=0
[21st March; LD gain from Lab]}

\index{Adeyfield West , Dacorum@Adeyfield W., \emph{Dacorum}}

Resignation of Keith White (Lab).

\noindent
\begin{tabular*}{\columnwidth}{@{\extracolsep{\fill}} p{0.545\columnwidth} >{\itshape}l r @{\extracolsep{\fill}}}
Ron Tindall & LD & 363\\
Mike Moore & Lab & 278\\
Barry Newton & C & 229\\
Noel Swinford & UKIP & 193\\
Simon Deacon & EDP & 51\\
\end{tabular*}

\subsubsection*{Watling \hspace*{\fill}\nolinebreak[1]%
\enspace\hspace*{\fill}
\finalhyphendemerits=0
[2nd May]}

\index{Watling , Dacorum@Watling, \emph{Dacorum}}

Resignation of David Lloyd (C).

\noindent
\begin{tabular*}{\columnwidth}{@{\extracolsep{\fill}} p{0.545\columnwidth} >{\itshape}l r @{\extracolsep{\fill}}}
Hilary Killen & C & 676\\
Mark Anderson & UKIP & 261\\
Margaret Coxage & Lab & 183\\
Sally Symington & LD & 108\\
\end{tabular*}

\council{East Hertfordshire}

\subsubsection*{Bishop's Stortford Meads \hspace*{\fill}\nolinebreak[1]%
\enspace\hspace*{\fill}
\finalhyphendemerits=0
[2nd May]}

\index{Bishop's Stortford Meads , East Hertfordshire@Bishop's Stortford Meads, \emph{E. Herts.}}

Resignation of Jill Demonti (C).

\noindent
\begin{tabular*}{\columnwidth}{@{\extracolsep{\fill}} p{0.545\columnwidth} >{\itshape}l r @{\extracolsep{\fill}}}
Keith Warnell & C & 713\\
Mione Goldspink & LD & 452\\
Val Cooke & Lab & 263\\
\end{tabular*}

\subsubsection*{Buntingford \hspace*{\fill}\nolinebreak[1]%
\enspace\hspace*{\fill}
\finalhyphendemerits=0
[2nd May]}

\index{Buntingford , East Hertfordshire@Buntingford, \emph{E. Herts.}}

Resignation of Surjit Singh Basra (C).

\noindent
\begin{tabular*}{\columnwidth}{@{\extracolsep{\fill}} p{0.545\columnwidth} >{\itshape}l r @{\extracolsep{\fill}}}
Jeff Jones & C & 626\\
Debbie Lemay & Ind & 404\\
Anthony Martin & Lab & 233\\
\end{tabular*}

\subsubsection*{Hertford Castle \hspace*{\fill}\nolinebreak[1]%
\enspace\hspace*{\fill}
\finalhyphendemerits=0
[2nd May; Ind gain from C]}

\index{Hertford Castle , East Hertfordshire@Hertford Castle, \emph{E. Herts.}}

Disqualification (non-attendance) of Russell Radford (C).

\noindent
\begin{tabular*}{\columnwidth}{@{\extracolsep{\fill}} p{0.545\columnwidth} >{\itshape}l r @{\extracolsep{\fill}}}
Jim Thornton & Ind & 961\\
Barbara Martin & C & 769\\
\end{tabular*}

\subsubsection*{Watton-at-Stone \hspace*{\fill}\nolinebreak[1]%
\enspace\hspace*{\fill}
\finalhyphendemerits=0
[2nd May]}

\index{Watton-at-Stone , East Hertfordshire@Watton-at-Stone, \emph{E. Herts.}}

Resignation of Nigel Poulton (C).

\noindent
\begin{tabular*}{\columnwidth}{@{\extracolsep{\fill}} p{0.545\columnwidth} >{\itshape}l r @{\extracolsep{\fill}}}
Rik Sharma & C & 755\\
Steve Buckingham & Lab & 82\\
\end{tabular*}

\council{Hertsmere}

\subsubsection*{Borehamwood Hillside \hspace*{\fill}\nolinebreak[1]%
\enspace\hspace*{\fill}
\finalhyphendemerits=0
[2nd May]}

\index{Borehamwood Hillside , Hertsmere@Borehamwood Hillside, \emph{Hertsmere}}

Resignation of Hannah David (C).

\noindent
\begin{tabular*}{\columnwidth}{@{\extracolsep{\fill}} p{0.545\columnwidth} >{\itshape}l r @{\extracolsep{\fill}}}
Farida Turner & C & 703\\
Tony Breslin & Lab & 518\\
David Rutter & UKIP & 397\\
Anita Gamble & LD & 62\\
\end{tabular*}

\council{North Hertfordshire}

\subsubsection*{Hitchin Oughton \hspace*{\fill}\nolinebreak[1]%
\enspace\hspace*{\fill}
\finalhyphendemerits=0
[12th September]}

\index{Hitchin Oughton , North Hertfordshire@Hitchin Oughton, \emph{N. Herts.}}

Death of David Billing (Lab).

\noindent
\begin{tabular*}{\columnwidth}{@{\extracolsep{\fill}} p{0.545\columnwidth} >{\itshape}l r @{\extracolsep{\fill}}}
Frank Radcliffe & Lab & 361\\
Mara MacSeoinin & C & 180\\
Peter Croft & UKIP & 148\\
Jacqueline Greatorex & Grn & 32\\
Clare Body & LD & 31\\
\end{tabular*}

\council{Watford}

\subsubsection*{Woodside \hspace*{\fill}\nolinebreak[1]%
\enspace\hspace*{\fill}
\finalhyphendemerits=0
[2nd May]}

\index{Woodside , Watford@Woodside, \emph{Watford}}

Resignation of Alan Burtenshaw (LD).

\noindent
\begin{tabular*}{\columnwidth}{@{\extracolsep{\fill}} p{0.545\columnwidth} >{\itshape}l r @{\extracolsep{\fill}}}
Glen Saffery & LD & 487\\
Tony Rogers & C & 292\\
Philip Cox & UKIP & 277\\
Omar Ashraf & Lab & 212\\
Alison Wiesner & Grn & 42\\
\end{tabular*}

\council{Welwyn Hatfield}

\subsubsection*{Haldens \hspace*{\fill}\nolinebreak[1]%
\enspace\hspace*{\fill}
\finalhyphendemerits=0
[2nd May]}

\index{Haldens , Welwyn Hatfield@Haldens, \emph{Welwyn Hatfield}}

Resignation of Ben Yetts (Lab).

\noindent
\begin{tabular*}{\columnwidth}{@{\extracolsep{\fill}} p{0.545\columnwidth} >{\itshape}l r @{\extracolsep{\fill}}}
Tony Crump & Lab & 580\\
Madeleine Sawle & C & 458\\
Kevin Daley & UKIP & 277\\
Susan Groom & Grn & 104\\
Frank Marsh & LD & 66\\
\end{tabular*}

\section{Kent}

\subsection*{County Council}\index{Kent}

At the May 2013 ordinary election there was an unfilled vacancy in Gravesham East division due to the resignation of Bryan Sweetland (C) who had won a by-election in Gravesham Rural division in December 2012.

\index{Gravesham East , Kent@Gravesham E., \emph{Kent}}

\columnbreak

\council{Ashford}

Ashford = Ashford Independent

\subsubsection*{Beaver \hspace*{\fill}\nolinebreak[1]%
\enspace\hspace*{\fill}
\finalhyphendemerits=0
[28th February]}

\index{Beaver , Ashford@Beaver, \emph{Ashford}}

Resignation of Rebecca Rutter (Lab).

\noindent
\begin{tabular*}{\columnwidth}{@{\extracolsep{\fill}} p{0.545\columnwidth} >{\itshape}l r @{\extracolsep{\fill}}}
Jill Britcher & Lab & 296\\
Jane Martin & C & 158\\
Angharad Yeo & UKIP & 155\\
Palma Laughton & Ashford & 85\\
Jack Cowen & LD & 79\\
John Holland & Ind & 34\\
Mark Reed & Grn & 19\\
\end{tabular*}

\subsubsection*{Saxon Shore \hspace*{\fill}\nolinebreak[1]%
\enspace\hspace*{\fill}
\finalhyphendemerits=0
[2nd May]}

\index{Saxon Shore , Ashford@Saxon Shore, \emph{Ashford}}

Death of Peter Wood (C).

\noindent
\begin{tabular*}{\columnwidth}{@{\extracolsep{\fill}} p{0.545\columnwidth} >{\itshape}l r @{\extracolsep{\fill}}}
Jane Martin & C & 759\\
Christopher Cooper & UKIP & 410\\
Thomas Reed & Lab & 174\\
Geoffrey Meaden & Grn & 129\\
\end{tabular*}

\council{Canterbury}

\subsubsection*{Seasalter \hspace*{\fill}\nolinebreak[1]%
\enspace\hspace*{\fill}
\finalhyphendemerits=0
[2nd May]}

\index{Seasalter , Canterbury@Seasalter, \emph{Canterbury}}

Resignation of Cyril Windsor (C).

\noindent
\begin{tabular*}{\columnwidth}{@{\extracolsep{\fill}} p{0.545\columnwidth} >{\itshape}l r @{\extracolsep{\fill}}}
Louise Morgan & C & 789\\
Howard Farmer & UKIP & 706\\
Rachel Goodwin & Lab & 427\\
Debra Enever & LD & 93\\
Eriks Puce & TUSC & 41\\
\end{tabular*}

\subsubsection*{Seasalter \hspace*{\fill}\nolinebreak[1]%
\enspace\hspace*{\fill}
\finalhyphendemerits=0
[19th September; UKIP gain from C]}

\index{Seasalter , Canterbury@Seasalter, \emph{Canterbury}}

Death of Mike Sharp (C).

\noindent
\begin{tabular*}{\columnwidth}{@{\extracolsep{\fill}} p{0.545\columnwidth} >{\itshape}l r @{\extracolsep{\fill}}}
Mike Bull & UKIP & 644\\
Annette Stein & C & 522\\
Rachel Goodwin & Lab & 307\\
Keith Hooker & LD & 147\\
Russell Page & Grn & 54\\
\end{tabular*}

\council{Dartford}

SGRA = Swanscome and Greenhithe Residents Association

\subsubsection*{Newtown \hspace*{\fill}\nolinebreak[1]%
\enspace\hspace*{\fill}
\finalhyphendemerits=0
[27th June; Lab gain from C]}

\index{Newtown , Dartford@Newtown, \emph{Dartford}}

Resignation of Gary Reynolds (C).

\noindent
\begin{tabular*}{\columnwidth}{@{\extracolsep{\fill}} p{0.545\columnwidth} >{\itshape}l r @{\extracolsep{\fill}}}
David Baker & Lab & 536\\
Rosanna Currans & C & 376\\
Ivan Burch & UKIP & 268\\
\end{tabular*}

\subsubsection*{Swanscombe \hspace*{\fill}\nolinebreak[1]%
\enspace\hspace*{\fill}
\finalhyphendemerits=0
[5th December; Lab gain from SGRA]}

\index{Swanscombe , Dartford@Swanscombe, \emph{Dartford}}

Death of Leslie Bobby (SGRA).

\noindent
\begin{tabular*}{\columnwidth}{@{\extracolsep{\fill}} p{0.545\columnwidth} >{\itshape}l r @{\extracolsep{\fill}}}
Steve Doran & Lab & 274\\
Vic Openshaw & SGRA & 273\\
Stephen Wilders & UKIP & 200\\
Richard Lees & Ind & 138\\
Steven Jarnell & C & 38\\
\end{tabular*}

\council{Dover}

\subsubsection*{Maxton, Elms Vale and Priory \hspace*{\fill}\nolinebreak[1]%
\enspace\hspace*{\fill}
\finalhyphendemerits=0
[2nd May]}

\index{Maxton, Elms Vale and Priory , Dover@Maxton, Elms Vale \& Priory, \emph{Dover}}

Death of Diane Smallwood (Lab).

\noindent
\begin{tabular*}{\columnwidth}{@{\extracolsep{\fill}} p{0.545\columnwidth} >{\itshape}l r @{\extracolsep{\fill}}}
Peter Wallace & Lab & 744\\
Deborah Boulares & C & 642\\
\end{tabular*}

\council{Gravesham}

\subsubsection*{Painters Ash \hspace*{\fill}\nolinebreak[1]%
\enspace\hspace*{\fill}
\finalhyphendemerits=0
[2nd May]}

\index{Painters Ash , Gravesham@Painters Ash, \emph{Gravesham}}

Resignation of Colin Dennis (Lab).

\noindent
\begin{tabular*}{\columnwidth}{@{\extracolsep{\fill}} p{0.545\columnwidth} >{\itshape}l r @{\extracolsep{\fill}}}
Lenny Rolles & Lab & 646\\
Michael Dixon & UKIP & 382\\
Alan Ridgers & C & 367\\
\end{tabular*}

\council{Sevenoaks}

\subsubsection*{Crockenhill and Well Hill \hspace*{\fill}\nolinebreak[1]%
\enspace\hspace*{\fill}
\finalhyphendemerits=0
[26th September; UKIP gain from Lab]}

\index{Crockenhill and Well Hill , Sevenoaks@Crockenhill \& Well Hill, \emph{Sevenoaks}}

Death of Jenny Dibsdall (Lab).

\noindent
\begin{tabular*}{\columnwidth}{@{\extracolsep{\fill}} p{0.545\columnwidth} >{\itshape}l r @{\extracolsep{\fill}}}
Steve Lindsay & UKIP & 216\\
Rachel Waterton & Lab & 188\\
Allrik Birch & C & 139\\
Philip Hobson & LD & 62\\
\end{tabular*}

\council{Thanet}

\subsubsection*{Cliftonville East \hspace*{\fill}\nolinebreak[1]%
\enspace\hspace*{\fill}
\finalhyphendemerits=0
[9th May; UKIP gain from C]}

\index{Cliftonville East , Thanet@Cliftonville E., \emph{Thanet}}

Disqualification (sentenced to 18 months' imprisonment, misconduct in public office) of Sandy Ezekiel (C).

\noindent
\begin{tabular*}{\columnwidth}{@{\extracolsep{\fill}} p{0.545\columnwidth} >{\itshape}l r @{\extracolsep{\fill}}}
Rozanne Duncan & UKIP & 699\\
Wendy Chaplin & C & 526\\
Alan Currie & Lab & 352\\
Louise Oldfield & Ind & 112\\
Seth Proctor & LD & 32\\
\end{tabular*}

\section{Lancashire}

\council{Blackpool}

\subsubsection*{Highfield \hspace*{\fill}\nolinebreak[1]%
\enspace\hspace*{\fill}
\finalhyphendemerits=0
[26th September]}

\index{Highfield , Blackpool@Highfield, \emph{Blackpool}}

Resignation of Chris Maughan (Lab).

\noindent
\begin{tabular*}{\columnwidth}{@{\extracolsep{\fill}} p{0.545\columnwidth} >{\itshape}l r @{\extracolsep{\fill}}}
Peter Hunter & Lab & 518\\
Sue Ridyard & C & 388\\
Stephen Flanigan & UKIP & 324\\
Rob Mottershead & Ind & 90\\
Bill Greene & LD & 59\\
Shereen Reedman & Grn & 37\\
Chris Maher & Ind & 8\\
\end{tabular*}

\council{Lancaster}

\subsubsection*{Bulk \hspace*{\fill}\nolinebreak[1]%
\enspace\hspace*{\fill}
\finalhyphendemerits=0
[2nd May]}

\index{Bulk , Lancaster@Bulk, \emph{Lancaster}}

Resignation of Ceri Mumford (Grn).

\noindent
\begin{tabular*}{\columnwidth}{@{\extracolsep{\fill}} p{0.545\columnwidth} >{\itshape}l r @{\extracolsep{\fill}}}
Caroline Jackson & Grn & 769\\
Bob Clark & Lab & 753\\
Kevan Walton & C & 142\\
\end{tabular*}

\council{Pendle}

\subsubsection*{Coates \hspace*{\fill}\nolinebreak[1]%
\enspace\hspace*{\fill}
\finalhyphendemerits=0
[2nd May]}

\index{Coates , Pendle@Coates, \emph{Pendle}}

Resignation of Janine Throupe (LD).

\noindent
\begin{tabular*}{\columnwidth}{@{\extracolsep{\fill}} p{0.6\columnwidth} >{\itshape}l r @{\extracolsep{\fill}}}
Claire Teall & LD & 623\\
Michael Thompson & C & 427\\
Christopher McKimm & Lab & 221\\
\end{tabular*}

\council{Ribble Valley}

\subsubsection*{Littlemoor \hspace*{\fill}\nolinebreak[1]%
\enspace\hspace*{\fill}
\finalhyphendemerits=0
[1st August; LD gain from C]}

\index{Littlemoor , Ribble Valley@Littlemoor, \emph{Ribble Valley}}

Resignation of Christine Connor (C).

\noindent
\begin{tabular*}{\columnwidth}{@{\extracolsep{\fill}} p{0.6\columnwidth} >{\itshape}l r @{\extracolsep{\fill}}}
James Shervey & LD & 361\\
Steve Rush & Ind & 249\\
Jean Forshaw & C & 109\\
Liz Webbe & Lab & 85\\
\end{tabular*}

\council{South Ribble}

\subsubsection*{Howick and Priory \hspace*{\fill}\nolinebreak[1]%
\enspace\hspace*{\fill}
\finalhyphendemerits=0
[2nd May; LD gain from C]}

\index{Howick and Priory , South Ribble@Howick \& Priory, \emph{S. Ribble}}

Resignation of Mary Robinson (C).

\noindent
\begin{tabular*}{\columnwidth}{@{\extracolsep{\fill}} p{0.545\columnwidth} >{\itshape}l r @{\extracolsep{\fill}}}
David Howarth & LD & 469\\
Angela Turner & C & 390\\
Robert Taylor & Lab & 185\\
David Duxbury & UKIP & 159\\
\end{tabular*}

\subsubsection*{Leyland St Ambrose \hspace*{\fill}\nolinebreak[1]%
\enspace\hspace*{\fill}
\finalhyphendemerits=0
[2nd May]}

\index{Leyland Saint Ambrose , South Ribble@Leyland St Ambrose, \emph{S. Ribble}}

Resignation of Sarah Tomlinson (Lab).

\noindent
\begin{tabular*}{\columnwidth}{@{\extracolsep{\fill}} p{0.545\columnwidth} >{\itshape}l r @{\extracolsep{\fill}}}
Ken Jones & Lab & 534\\
Paul Wharton & C & 440\\
Gareth Armstrong & LD & 129\\
\end{tabular*}

\subsubsection*{Leyland St Mary's \hspace*{\fill}\nolinebreak[1]%
\enspace\hspace*{\fill}
\finalhyphendemerits=0
[2nd May]}

\index{Leyland Saint Mary's , South Ribble@Leyland St Mary's, \emph{S. Ribble}}

Resignation of Michael McNulty (C).

\noindent
\begin{tabular*}{\columnwidth}{@{\extracolsep{\fill}} p{0.545\columnwidth} >{\itshape}l r @{\extracolsep{\fill}}}
Alan Ogilvie & C & 744\\
Carole Titherington & Lab & 401\\
Charlotte Ashworth & LD & 95\\
\end{tabular*}

\council{West Lancashire}

\subsubsection*{Digmoor \hspace*{\fill}\nolinebreak[1]%
\enspace\hspace*{\fill}
\finalhyphendemerits=0
[2nd May]}

\index{Digmoor , West Lancashie@Digmoor, \emph{W. Lancs.}}

Resignation of Jackie Coyle (Lab).

\noindent
\begin{tabular*}{\columnwidth}{@{\extracolsep{\fill}} p{0.545\columnwidth} >{\itshape}l r @{\extracolsep{\fill}}}
Chris Wynn & Lab & 685\\
Peter Cranie & Grn & 57\\
Edward McCarthy & C & 47\\
\end{tabular*}

\subsubsection*{Parbold \hspace*{\fill}\nolinebreak[1]%
\enspace\hspace*{\fill}
\finalhyphendemerits=0
[10th October]}

\index{Parbold , West Lancashie@Parbold, \emph{W. Lancs.}}

Death of Barbara Kean (C).

\noindent
\begin{tabular*}{\columnwidth}{@{\extracolsep{\fill}} p{0.545\columnwidth} >{\itshape}l r @{\extracolsep{\fill}}}
David Whittington & C & 554\\
Clare Gillard & Lab & 461\\
Damon Noone & UKIP & 103\\
\end{tabular*}

\council{Wyre}

\subsubsection*{Park \hspace*{\fill}\nolinebreak[1]%
\enspace\hspace*{\fill}
\finalhyphendemerits=0
[2nd May]}

\index{Park , Wyre@Park, \emph{Wyre}}

Resignation of Julie Grunshaw (Lab).

\noindent
\begin{tabular*}{\columnwidth}{@{\extracolsep{\fill}} p{0.545\columnwidth} >{\itshape}l r @{\extracolsep{\fill}}}
Brian Stephenson & Lab & 502\\
James McConnachie & C & 276\\
\end{tabular*}

\subsubsection*{Pharos \hspace*{\fill}\nolinebreak[1]%
\enspace\hspace*{\fill}
\finalhyphendemerits=0
[2nd May]}

\index{Pharos , Wyre@Pharos, \emph{Wyre}}

Resignation of Clive Grunshaw (Lab).

\noindent
\begin{tabular*}{\columnwidth}{@{\extracolsep{\fill}} p{0.545\columnwidth} >{\itshape}l r @{\extracolsep{\fill}}}
Evelyn Stephenson & Lab & 618\\
Dave Shaw & C & 339\\
\end{tabular*}

\subsubsection*{Staina \hspace*{\fill}\nolinebreak[1]%
\enspace\hspace*{\fill}
\finalhyphendemerits=0
[2nd May]}

\index{Staina , Wyre@Staina, \emph{Wyre}}

Death of Ian Perkin (C).

\noindent
\begin{tabular*}{\columnwidth}{@{\extracolsep{\fill}} p{0.545\columnwidth} >{\itshape}l r @{\extracolsep{\fill}}}
Kerry Jones & C & 974\\
Eddie Rawlings & Lab & 528\\
\end{tabular*}



\section{Leicestershire}

BritDem = British Democratic Party

\council{Charnwood}

\subsubsection*{Loughborough Ashby \hspace*{\fill}\nolinebreak[1]%
\enspace\hspace*{\fill}
\finalhyphendemerits=0
[5th September]}

\index{Loughborough Ashby , Charnwood@Loughborough Ashby, \emph{Charnwood}}

Resignation of Chris Carter (Lab).

\noindent
\begin{tabular*}{\columnwidth}{@{\extracolsep{\fill}} p{0.545\columnwidth} >{\itshape}l r @{\extracolsep{\fill}}}
Mary Draycott & Lab & 375\\
Andy McWilliam & UKIP & 118\\
Kirti Asmal & C & 29\\
\end{tabular*}

\subsubsection*{Wreake Villages \hspace*{\fill}\nolinebreak[1]%
\enspace\hspace*{\fill}
\finalhyphendemerits=0
[12th September]}

\index{Wreake Villages , Charnwood@Wreake Villages, \emph{Charnwood}}

Resignation of Matthew Blain (C).

\noindent
\begin{tabular*}{\columnwidth}{@{\extracolsep{\fill}} p{0.545\columnwidth} >{\itshape}l r @{\extracolsep{\fill}}}
James Poland & C & 396\\
Steve Brown & Lab & 87\\
\end{tabular*}

\subsubsection*{Loughborough Hastings \hspace*{\fill}\nolinebreak[1]%
\enspace\hspace*{\fill}
\finalhyphendemerits=0
[24th October]}

\index{Loughborough Hastings , Charnwood@Loughborough Hastings, \emph{Charnwood}}

Resignation of Litu Choudhury (Lab).

\noindent
\begin{tabular*}{\columnwidth}{@{\extracolsep{\fill}} p{0.52\columnwidth} >{\itshape}l r @{\extracolsep{\fill}}}
Sarah Maynard Smith & Lab & 554\\
Judith Spence & C & 127\\
Andy McWilliam & UKIP & 111\\
Kevan Stafford & BritDem & 85\\
Simon Atkins & LD & 26\\
\end{tabular*}

\subsubsection*{Shepshed West \hspace*{\fill}\nolinebreak[1]%
\enspace\hspace*{\fill}
\finalhyphendemerits=0
[24th October; Lab gain from C]}

\index{Shepshed West , Charnwood@Shepshed W., \emph{Charnwood}}

Death of Bernard Burr (C).

\noindent
\begin{tabular*}{\columnwidth}{@{\extracolsep{\fill}} p{0.545\columnwidth} >{\itshape}l r @{\extracolsep{\fill}}}
Jane Lennie & Lab & 683\\
Joan Tassell & C & 560\\
Diane Horn & LD & 178\\
\end{tabular*}

\council{Harborough}

\subsubsection*{Thurnby and Houghton \hspace*{\fill}\nolinebreak[1]%
\enspace\hspace*{\fill}
\finalhyphendemerits=0
[2nd May]}

\index{Thurnby and Houghton , Harborough@Thurnby \& Houghton, \emph{Harborough}}

Death of Jan Tooley (LD).

\noindent
\begin{tabular*}{\columnwidth}{@{\extracolsep{\fill}} p{0.545\columnwidth} >{\itshape}l r @{\extracolsep{\fill}}}
Peter Elliott & LD & 1341\\
Simon Whelband & C & 785\\
\end{tabular*}

\subsubsection*{Bosworth \hspace*{\fill}\nolinebreak[1]%
\enspace\hspace*{\fill}
\finalhyphendemerits=0
[7th November]}

\index{Bosworth , Harborough@Bosworth, \emph{Harborough}}

Resignation of Brian Smith (C).

\noindent
\begin{tabular*}{\columnwidth}{@{\extracolsep{\fill}} p{0.545\columnwidth} >{\itshape}l r @{\extracolsep{\fill}}}
Lesley Bowles & C & 259\\
Annette Deacon & LD & 114\\
Bill Piper & UKIP & 105\\
\end{tabular*}

\council{Leicester}

\subsubsection*{Abbey \hspace*{\fill}\nolinebreak[1]%
\enspace\hspace*{\fill}
\finalhyphendemerits=0
[9th May]}

\index{Abbey , Leicester@Abbey, \emph{Leicester}}

Resignation of Colin Marriott (Lab).

\noindent
\begin{tabular*}{\columnwidth}{@{\extracolsep{\fill}} p{0.545\columnwidth} >{\itshape}l r @{\extracolsep{\fill}}}
Vijay Singh Riyait & Lab & 1190\\
Dipak Joshi & C & 562\\
Terry McGreal & Ind & 352\\
John Taylor & LD & 212\\
Tessa Warrington & TUSC & 165\\
\end{tabular*}

\section{Lincolnshire}

\subsection*{County Council}\index{Lincolnshire}

LincsInd = Lincolnshire Independent

At the May 2013 ordinary election there was an unfilled vacancy in Lincoln Bracebridge division due to the resignation of Rachel Hubbard (C).
\index{Lincoln Bracebridge , Lincolnshire@Lincoln Bracebridge, \emph{Lincs.}}

\subsubsection*{Scotter Rural \hspace*{\fill}\nolinebreak[1]%
\enspace\hspace*{\fill}
\finalhyphendemerits=0
[19th December; LD gain from C]}

\index{Scotter Rural , Lincolnshire@Scotter Rural, \emph{Lincs.}}

Death of Chris Underwood-Frost (C).

\noindent
\begin{tabular*}{\columnwidth}{@{\extracolsep{\fill}} p{0.52\columnwidth} >{\itshape}l r @{\extracolsep{\fill}}}
Lesley Rollings & LD & 726\\
Richard Butroid & C & 348\\
Nick Smith & UKIP & 264\\
Chris Darcel & LincsInd & 137\\
\end{tabular*}

\council{Boston}

\subsubsection*{Staniland South \hspace*{\fill}\nolinebreak[1]%
\enspace\hspace*{\fill}
\finalhyphendemerits=0
[2nd May; UKIP gain from C]}

\index{Staniland South , Boston@Staniland S., \emph{Boston}}

Death of Paul Mould (C).

\noindent
\begin{tabular*}{\columnwidth}{@{\extracolsep{\fill}} p{0.545\columnwidth} >{\itshape}l r @{\extracolsep{\fill}}}
Bob McAuley & UKIP & 376\\
Pam Kenny & Lab & 202\\
Robert Lauberts & Ind & 145\\
Carl Richmond & C & 124\\
\end{tabular*}

\subsubsection*{Fenside \hspace*{\fill}\nolinebreak[1]%
\enspace\hspace*{\fill}
\finalhyphendemerits=0
[5th September; UKIP gain from EDP]}

\index{Fenside , Boston@Fenside, \emph{Boston}}

Disqualification (non-attendance) of Elliott Fountain (EDP).

\noindent
\begin{tabular*}{\columnwidth}{@{\extracolsep{\fill}} p{0.57\columnwidth} >{\itshape}l r @{\extracolsep{\fill}}}
\sloppyword{Tiggs Keywood-Wainwright} & UKIP & 162\\
Dan Elkington & C & 87\\
Alan Taylor & LD & 87\\
Ben Cook & Lab & 75\\
\end{tabular*}

\council{East Lindsey}

\subsubsection*{Coningsby and Tattershall \hspace*{\fill}\nolinebreak[1]%
\enspace\hspace*{\fill}
\finalhyphendemerits=0
[2nd May; UKIP gain from LD]}

\index{Coningsby and Tattershall , East Lindsey@Coningsby \& Tattershall, \emph{E. Lindsey}}

Death of Ray Curtis (LD).

\noindent
\begin{tabular*}{\columnwidth}{@{\extracolsep{\fill}} p{0.545\columnwidth} >{\itshape}l r @{\extracolsep{\fill}}}
Julia Pears & UKIP & 710\\
Richard Avison & C & 578\\
\end{tabular*}

\subsubsection*{Frithville \hspace*{\fill}\nolinebreak[1]%
\enspace\hspace*{\fill}
\finalhyphendemerits=0
[12th September; C gain from Ind]}

\index{Frithville , East Lindsey@Frithville, \emph{E. Lindsey}}

Resignation of Steve Doughty (Ind).

\noindent
\begin{tabular*}{\columnwidth}{@{\extracolsep{\fill}} p{0.545\columnwidth} >{\itshape}l r @{\extracolsep{\fill}}}
Neil Jones & C & 221\\
Colin Mair & UKIP & 163\\
\end{tabular*}

\subsubsection*{Chapel St Leonards \hspace*{\fill}\nolinebreak[1]%
\enspace\hspace*{\fill}
\finalhyphendemerits=0
[3rd October; Lab gain from Ind]}

\index{Chapel Saint Leonards , East Lindsey@Chapel St Leonards, \emph{E. Lindsey}}

Resignation of Philip Leivers (Ind).

\noindent
\begin{tabular*}{\columnwidth}{@{\extracolsep{\fill}} p{0.545\columnwidth} >{\itshape}l r @{\extracolsep{\fill}}}
Fiona Brown & Lab & 382\\
Giles Crust & UKIP & 228\\
Richard Enderby & Ind & 206\\
Mel Turton-Leivers & Ind & 175\\
Kevin Sharpe & C & 149\\
\end{tabular*}

\council{Lincoln}

\subsubsection*{Bracebridge \hspace*{\fill}\nolinebreak[1]%
\enspace\hspace*{\fill}
\finalhyphendemerits=0
[22nd August; Lab gain from C]}

\index{Bracebridge , Lincoln@Bracebridge, \emph{Lincoln}}

Disqualification (non-attendance) of Darren Grice (Ind elected as C).

\noindent
\begin{tabular*}{\columnwidth}{@{\extracolsep{\fill}} p{0.545\columnwidth} >{\itshape}l r @{\extracolsep{\fill}}}
Katie Vause & Lab & 577\\
David Denman & C & 480\\
Elaine Warde & UKIP & 345\\
Ross Pepper & LD & 75\\
Karen Williams & TUSC & 14\\
\end{tabular*}

\council{North East Lincolnshire}

\subsubsection*{Humberston and New Waltham \hspace*{\fill}\nolinebreak[1]%
\enspace\hspace*{\fill}
\finalhyphendemerits=0
[4th April; UKIP gain from C]}

\index{Humberston and New Waltham , North East Lincolnshire@Humberston \& New Waltham, \emph{N.E. Lincs.}}

Death of John Colebrook (C).

\noindent
\begin{tabular*}{\columnwidth}{@{\extracolsep{\fill}} p{0.545\columnwidth} >{\itshape}l r @{\extracolsep{\fill}}}
Stephen Harness & UKIP & 1098\\
Harry Hall & C & 738\\
Ashley Smith & Lab & 470\\
Stephen Stead & LD & 311\\
\end{tabular*}

\council{North Kesteven}

\subsubsection*{Sleaford Holdingham \hspace*{\fill}\nolinebreak[1]%
\enspace\hspace*{\fill}
\finalhyphendemerits=0
[11th July]}

\index{Sleaford Holdingham , North Kesteven@Sleaford Holdingham, \emph{N. Kesteven}}

Resignation of Peter Haysum (Ind).

\noindent
\begin{tabular*}{\columnwidth}{@{\extracolsep{\fill}} p{0.545\columnwidth} >{\itshape}l r @{\extracolsep{\fill}}}
Grenville Jackson & Ind & 140\\
Mike Benthall & Lab & 79\\
Ken Fernandes & Ind & 57\\
John Dilks & UKIP & 48\\
Lucille Hagues & C & 48\\
\end{tabular*}

\council{West Lindsey}

\subsubsection*{Gainsborough East \hspace*{\fill}\nolinebreak[1]%
\enspace\hspace*{\fill}
\finalhyphendemerits=0
[14th February]}

\index{Gainsborough East , West Lindsey@Gainsborough E., \emph{W. Lindsey}}

Death of Mel Starkey (LD).

\noindent
\begin{tabular*}{\columnwidth}{@{\extracolsep{\fill}} p{0.545\columnwidth} >{\itshape}l r @{\extracolsep{\fill}}}
Mark Binns & LD & 169\\
Mick Devine & Lab & 149\\
Howard Thompson & UKIP & 143\\
Richard Butroid & C & 129\\
\end{tabular*}

\subsubsection*{Scotter \hspace*{\fill}\nolinebreak[1]%
\enspace\hspace*{\fill}
\finalhyphendemerits=0
[19th December; Ind gain from C]}

\index{Scotter , West Lindsey@Scotter, \emph{W. Lindsey}}

Death of Chris Underwood-Frost (C).

\noindent
\begin{tabular*}{\columnwidth}{@{\extracolsep{\fill}} p{0.545\columnwidth} >{\itshape}l r @{\extracolsep{\fill}}}
Chris Day & Ind & 529\\
John Cox & C & 219\\
Barry Coward & LD & 148\\
Howard Thompson & UKIP & 138\\
\end{tabular*}

\section{Norfolk}

\subsection*{County Council}\index{Norfolk}

At the May 2013 ordinary election there was an unfilled vacancy in North Coast division due to the resignation of Stephen Bett (Ind elected as C).

\index{North Coast , Norfolk@North Coast, \emph{Norfolk}}

\subsubsection*{Thetford West \hspace*{\fill}\nolinebreak[1]%
\enspace\hspace*{\fill}
\finalhyphendemerits=0
[1st August; Lab gain from UKIP]}

\index{Thetford West , Norfolk@Thetford W., \emph{Norfolk}}

Resignation of Peter Georgiou (UKIP).

\noindent
\begin{tabular*}{\columnwidth}{@{\extracolsep{\fill}} p{0.545\columnwidth} >{\itshape}l r @{\extracolsep{\fill}}}
Terry Jermy & Lab & 1071\\
John Newton & UKIP & 900\\
Tristan Ashby & C & 282\\
Danny Jeffrey & Ind & 78\\
Sandra Walmsley & Grn & 40\\
\end{tabular*}

\subsubsection*{North Walsham East \hspace*{\fill}\nolinebreak[1]%
\enspace\hspace*{\fill}
\finalhyphendemerits=0
[24th October]}

\index{North Walsham East , Norfolk@North Walsham E., \emph{Norfolk}}

Resignation of Ed Foss (LD).

\noindent
\begin{tabular*}{\columnwidth}{@{\extracolsep{\fill}} p{0.545\columnwidth} >{\itshape}l r @{\extracolsep{\fill}}}
Eric Seward & LD & 1044\\
Lynette Comber & UKIP & 565\\
Stephen Burke & Lab & 442\\
Rhodri Oliver & C & 359\\
Paul Oakes & Grn & 80\\
Graham Jones & Ind & 61\\
\end{tabular*}

\council{Broadland}

\subsubsection*{Blofield with South Walsham \hspace*{\fill}\nolinebreak[1]%
\enspace\hspace*{\fill}
\finalhyphendemerits=0
[2nd May]}

\index{Blofield with South Walsham , Broadland@Blofield with South Walsham, \emph{Broadland}}

Resignation of James Carswell (C).

\noindent
\begin{tabular*}{\columnwidth}{@{\extracolsep{\fill}} p{0.545\columnwidth} >{\itshape}l r @{\extracolsep{\fill}}}
Susan Lawn & C & 561\\
George Debbage & Ind & 522\\
Stephen Heard & Ind & 232\\
Brenda Jones & Lab & 222\\
Lauren Bean & LD & 105\\
\end{tabular*}

\subsubsection*{Aylsham \hspace*{\fill}\nolinebreak[1]%
\enspace\hspace*{\fill}
\finalhyphendemerits=0
[4th July; LD gain from C]}

\index{Aylsham , Broadland@Aylsham, \emph{Broadland}}

Resignation of Jo Cottingham (C).

\noindent
\begin{tabular*}{\columnwidth}{@{\extracolsep{\fill}} p{0.545\columnwidth} >{\itshape}l r @{\extracolsep{\fill}}}
Steve Riley & LD & 688\\
Judi Meredith & C & 501\\
Malcolm Kemp & Lab & 181\\
\end{tabular*}

\council{King's Lynn and West Norfolk}

\subsubsection*{Snettisham \hspace*{\fill}\nolinebreak[1]%
\enspace\hspace*{\fill}
\finalhyphendemerits=0
[2nd May]}

\index{Snettisham , King's Lynn and West Norfolk@Snettisham, \emph{King's Lynn \& W. Norfolk}}

Death of David Johnson (C).

\noindent
\begin{tabular*}{\columnwidth}{@{\extracolsep{\fill}} p{0.545\columnwidth} >{\itshape}l r @{\extracolsep{\fill}}}
Avril Wright & C & 593\\
Michael Stone & UKIP & 361\\
Richard Pennington & Lab & 263\\
\end{tabular*}

\subsubsection*{Watlington \hspace*{\fill}\nolinebreak[1]%
\enspace\hspace*{\fill}
\finalhyphendemerits=0
[13th June; UKIP gain from LD]}

\index{Watlington , King's Lynn and West Norfolk@Watlington, \emph{King's Lynn \& W. Norfolk}}

Resignation of Ian Mack (LD).

\noindent
\begin{tabular*}{\columnwidth}{@{\extracolsep{\fill}} p{0.545\columnwidth} >{\itshape}l r @{\extracolsep{\fill}}}
Ashley Collins & UKIP & 179\\
John Dobson & C & 115\\
Emilia Rust & Lab & 99\\
\end{tabular*}

\council{North Norfolk}

\subsubsection*{Cromer Town \hspace*{\fill}\nolinebreak[1]%
\enspace\hspace*{\fill}
\finalhyphendemerits=0
[21st February; LD gain from C]}

\index{Cromer Town , North Norfolk@Cromer Town, \emph{N. Norfolk}}

Death of Keith Johnson (C).

\noindent
\begin{tabular*}{\columnwidth}{@{\extracolsep{\fill}} p{0.545\columnwidth} >{\itshape}l r @{\extracolsep{\fill}}}
Andreas Yiasimi & LD & 558\\
Jen Emery & Lab & 240\\
David Ramsbotham & UKIP & 218\\
Tony Nash & C & 181\\
\end{tabular*}

\council{Norwich}

\subsubsection*{Mancroft \hspace*{\fill}\nolinebreak[1]%
\enspace\hspace*{\fill}
\finalhyphendemerits=0
[2nd May]}

\index{Mancroft , Norwich@Mancroft, \emph{Norwich}}

Resignation of Graeme Gee (Grn).

\noindent
\begin{tabular*}{\columnwidth}{@{\extracolsep{\fill}} p{0.545\columnwidth} >{\itshape}l r @{\extracolsep{\fill}}}
Simeon Jackson & Grn & 1152\\
Tony Waring & Lab & 861\\
Samuel Stringer & C & 308\\
Jeremy Hooke & LD & 177\\
\end{tabular*}

\council{South Norfolk}

\subsubsection*{New Costessey \hspace*{\fill}\nolinebreak[1]%
\enspace\hspace*{\fill}
\finalhyphendemerits=0
[2nd May]}

\index{New Costessey , South Norfolk@New Costessey, \emph{S. Norfolk}}

Resignation of Jan Hardinge (LD).

\noindent
\begin{tabular*}{\columnwidth}{@{\extracolsep{\fill}} p{0.545\columnwidth} >{\itshape}l r @{\extracolsep{\fill}}}
Katy Smith & LD & 659\\
Cyril Gibbs & Lab & 312\\
Ian Boreham & Grn & 277\\
\end{tabular*}

\section{North Yorkshire}

\subsection*{County Council}\index{North Yorkshire}

\subsubsection*{South Selby \hspace*{\fill}\nolinebreak[1]%
\enspace\hspace*{\fill}
\finalhyphendemerits=0
[10th October]}

\index{South Selby , North Yorkshire@South Selby, \emph{N. Yorks.}}

Death of Margaret Hulme (C).

\noindent
\begin{tabular*}{\columnwidth}{@{\extracolsep{\fill}} p{0.545\columnwidth} >{\itshape}l r @{\extracolsep{\fill}}}
Mike Jordan & C & 592\\
Rod Price & Lab & 525\\
Colin Heath & UKIP & 282\\
David McSherry & Ind & 201\\
\end{tabular*}

\council{Hambleton}

\subsubsection*{Whitestonecliffe \hspace*{\fill}\nolinebreak[1]%
\enspace\hspace*{\fill}
\finalhyphendemerits=0
[2nd May]}

\index{Whitestonecliffe , Hambleton@Whitestonecliffe, \emph{Hambleton}}

Resignation of Nigel Clack (C).

\noindent
\begin{tabular*}{\columnwidth}{@{\extracolsep{\fill}} p{0.545\columnwidth} >{\itshape}l r @{\extracolsep{\fill}}}
Janet Watson & C & 507\\
Abbie O'Neill & Lab & 120\\
\end{tabular*}

\subsubsection*{Topcliffe \hspace*{\fill}\nolinebreak[1]%
\enspace\hspace*{\fill}
\finalhyphendemerits=0
[12th December]}

\index{Topcliffe , Hambleton@Topcliffe, \emph{Hambleton}}

Resignation of Neville Huxtable (C).

\noindent
\begin{tabular*}{\columnwidth}{@{\extracolsep{\fill}} p{0.545\columnwidth} >{\itshape}l r @{\extracolsep{\fill}}}
Garry Key & C & \emph{unop.}\\
\end{tabular*}

\council{Middlesbrough}

\subsubsection*{Pallister \hspace*{\fill}\nolinebreak[1]%
\enspace\hspace*{\fill}
\finalhyphendemerits=0
[2nd May]}

\index{Pallister , Middlesbrough@Pallister, \emph{Middlesbrough}}

Resignation of Barry Coppinger (Lab).

\noindent
\begin{tabular*}{\columnwidth}{@{\extracolsep{\fill}} p{0.545\columnwidth} >{\itshape}l r @{\extracolsep{\fill}}}
Michael Thompson & Lab & 608\\
David Cottrell & UKIP & 262\\
Zoe Greaves & C & 29\\
Gary McArthur & LD & 27\\
\end{tabular*}

\council{Redcar and Cleveland}

\subsubsection*{Skelton \hspace*{\fill}\nolinebreak[1]%
\enspace\hspace*{\fill}
\finalhyphendemerits=0
[8th August]}

\index{Skelton , Redcar and Cleveland@Skelton, \emph{Redcar \& Cleveland}}

Resignation of Dave McLuckie (Lab).

\noindent
\begin{tabular*}{\columnwidth}{@{\extracolsep{\fill}} p{0.545\columnwidth} >{\itshape}l r @{\extracolsep{\fill}}}
David Walsh & Lab & 745\\
Stuart Todd & UKIP & 485\\
Anne Watts & C & 176\\
James Carrolle & Ind & 170\\
Rod Waite & LD & 40\\
\end{tabular*}

\council{Richmondshire}

\subsubsection*{Hornby Castle \hspace*{\fill}\nolinebreak[1]%
\enspace\hspace*{\fill}
\finalhyphendemerits=0
[12th December]}

\index{Hornby Castle , Richmondshire@Hornby Castle, \emph{Richmondshire}}

Resignation of Lin Clarkson (C).

\noindent
\begin{tabular*}{\columnwidth}{@{\extracolsep{\fill}} p{0.545\columnwidth} >{\itshape}l r @{\extracolsep{\fill}}}
Robin Scott & C & 127\\
Helen Grant & Ind & 98\\
Jacqueline Brakenberry & UKIP & 50\\
\end{tabular*}

\council{Ryedale}

\subsubsection*{Pickering East \hspace*{\fill}\nolinebreak[1]%
\enspace\hspace*{\fill}
\finalhyphendemerits=0
[2nd May; Lib gain from C]}

\index{Pickering East , Ryedale@Pickering E., \emph{Ryedale}}

Resignation of Vivienne Knaggs (C).

\noindent
\begin{tabular*}{\columnwidth}{@{\extracolsep{\fill}} p{0.545\columnwidth} >{\itshape}l r @{\extracolsep{\fill}}}
Joy Andrews & Lib & 373\\
Matthew Wilkinson & C & 330\\
William Oxley & Ind & 273\\
\end{tabular*}

\subsubsection*{Ryedale South West \hspace*{\fill}\nolinebreak[1]%
\enspace\hspace*{\fill}
\finalhyphendemerits=0
[2nd May]}

\index{Ryedale South West , Ryedale@Ryedale S.W., \emph{Ryedale}}

Resignation of Keith Knaggs (C).

\noindent
\begin{tabular*}{\columnwidth}{@{\extracolsep{\fill}} p{0.545\columnwidth} >{\itshape}l r @{\extracolsep{\fill}}}
Shane Collinson & C & 327\\
Denys Townsend & Ind & 196\\
\end{tabular*}

\council{Scarborough}

\subsubsection*{Esk Valley \hspace*{\fill}\nolinebreak[1]%
\enspace\hspace*{\fill}
\finalhyphendemerits=0
[2nd May; Ind gain from C]}

\index{Esk Valley , Scarborough@Esk Valley, \emph{Scarborough}}

Resignation of Tim Lawn (C).

\noindent
\begin{tabular*}{\columnwidth}{@{\extracolsep{\fill}} p{0.545\columnwidth} >{\itshape}l r @{\extracolsep{\fill}}}
Mike Ward & Ind & 698\\
Colin Barnes & Lab & 178\\
Bob Jackman & LD & 164\\
\end{tabular*}

\subsubsection*{Streonshalh \hspace*{\fill}\nolinebreak[1]%
\enspace\hspace*{\fill}
\finalhyphendemerits=0
[2nd May; Lab gain from C]}

\index{Streonshalh , Scarborough@Streonshalh, \emph{Scarborough}}

Resignation of Sandra Turner (C).

\noindent
\begin{tabular*}{\columnwidth}{@{\extracolsep{\fill}} p{0.545\columnwidth} >{\itshape}l r @{\extracolsep{\fill}}}
Tina Davy & Lab & 229\\
Derek Robinson & Ind & 209\\
John Thistle & UKIP & 209\\
Janet MacDonald & C & 125\\
\end{tabular*}

\subsubsection*{Newby \hspace*{\fill}\nolinebreak[1]%
\enspace\hspace*{\fill}
\finalhyphendemerits=0
[22nd August]}

\index{Newby , Scarborough@Newby, \emph{Scarborough}}

Resignation of Mick Cooper (Ind).

\noindent
\begin{tabular*}{\columnwidth}{@{\extracolsep{\fill}} p{0.545\columnwidth} >{\itshape}l r @{\extracolsep{\fill}}}
Sue Backhouse & C & 380\\
Andy Smith & UKIP & 285\\
Carl Maw & Lab & 197\\
Bonnie Purchon & Ind & 143\\
Helen Kindness & Grn & 76\\
\end{tabular*}

\subsubsection*{Ramshill \hspace*{\fill}\nolinebreak[1]%
\enspace\hspace*{\fill}
\finalhyphendemerits=0
[22nd August; Lab gain from C]}

\index{Ramshill , Scarborough@Ramshill, \emph{Scarborough}}

Resignation of Nick Brown (C).

\noindent
\begin{tabular*}{\columnwidth}{@{\extracolsep{\fill}} p{0.545\columnwidth} >{\itshape}l r @{\extracolsep{\fill}}}
Steve Siddons & Lab & 190\\
Michael James & UKIP & 149\\
Peter Southward & C & 122\\
Mark Vesey & Grn & 67\\
Lana Rodgers & LD & 56\\
\end{tabular*}

\subsubsection*{Eastfield \hspace*{\fill}\nolinebreak[1]%
\enspace\hspace*{\fill}
\finalhyphendemerits=0
[21st November; Lab gain from LD]}

\index{Eastfield , Scarborough@Eastfield, \emph{Scarborough}}

Resignation of Geoff Evans (LD).

\noindent
\begin{tabular*}{\columnwidth}{@{\extracolsep{\fill}} p{0.545\columnwidth} >{\itshape}l r @{\extracolsep{\fill}}}
Tony Randerson & Lab & 310\\
Jonathan Dodds & UKIP & 175\\
Carole Gerarda & Ind & 97\\
William Barnes & C & 32\\
Annette Hudspeth & Grn & 11\\
Dawn Maxwell & Ind & 10\\
\end{tabular*}

\section{Northamptonshire}

\subsection*{County Council}\index{Northamptonshire}

\subsubsection*{Middleton Cheney \hspace*{\fill}\nolinebreak[1]%
\enspace\hspace*{\fill}
\finalhyphendemerits=0
[5th September]}

\index{Middleton Cheney , Northamptonshire@Middleton Cheney, \emph{Northants.}}

Death of Ken Melling (C).

\noindent
\begin{tabular*}{\columnwidth}{@{\extracolsep{\fill}} p{0.545\columnwidth} >{\itshape}l r @{\extracolsep{\fill}}}
Ron Sawbridge & C & 1081\\
Barry Mahoney & UKIP & 604\\
Christopher Lee & Lab & 221\\
Scott Collins & LD & 141\\
\end{tabular*}

\council{Corby}

\subsubsection*{Kingswood \hspace*{\fill}\nolinebreak[1]%
\enspace\hspace*{\fill}
\finalhyphendemerits=0
[7th November]}

\index{Kingswood , Corby@Kingswood, \emph{Corby}}

Resignation of Maureen Forshaw (Lab).

\noindent
\begin{tabular*}{\columnwidth}{@{\extracolsep{\fill}} p{0.545\columnwidth} >{\itshape}l r @{\extracolsep{\fill}}}
Elise Elliston & Lab & 722\\
Peter McGowan & UKIP & 246\\
Phil Ewers & C & 154\\
Julie Grant & LD & 18\\
\end{tabular*}

\council{Daventry}

\subsubsection*{Brixworth \hspace*{\fill}\nolinebreak[1]%
\enspace\hspace*{\fill}
\finalhyphendemerits=0
[2nd May]}

\index{Brixworth , Daventry@Brixworth, \emph{Daventry}}

Resignation of Nick Bunting (C).

\noindent
\begin{tabular*}{\columnwidth}{@{\extracolsep{\fill}} p{0.545\columnwidth} >{\itshape}l r @{\extracolsep{\fill}}}
Ian Barratt & C & 1082\\
Robert McNally & Lab & 307\\
Steve Whiffen & Grn & 258\\
Neil Farmer & LD & 89\\
\end{tabular*}

\subsubsection*{Welford \hspace*{\fill}\nolinebreak[1]%
\enspace\hspace*{\fill}
\finalhyphendemerits=0
[2nd May]}

\index{Welford , Daventry@Welford, \emph{Daventry}}

Resignation of Kay Driver (C).

\noindent
\begin{tabular*}{\columnwidth}{@{\extracolsep{\fill}} p{0.545\columnwidth} >{\itshape}l r @{\extracolsep{\fill}}}
Luke Major & C & 824\\
Katharine Wicksteed & Grn & 242\\
Sue Myers & Lab & 178\\
\end{tabular*}

\subsubsection*{Ravensthorpe \hspace*{\fill}\nolinebreak[1]%
\enspace\hspace*{\fill}
\finalhyphendemerits=0
[5th September]}

\index{Ravensthorpe , Daventry@Ravensthorpe, \emph{Daventry}}

Death of Ken Melling (C).

\noindent
\begin{tabular*}{\columnwidth}{@{\extracolsep{\fill}} p{0.545\columnwidth} >{\itshape}l r @{\extracolsep{\fill}}}
Bryn Aldridge & C & 285\\
Eric MacAnndrais & UKIP & 212\\
Sue Myers & Lab & 93\\
Neil Farmer & LD & 23\\
\end{tabular*}

\council{East Northamptonshire}

\subsubsection*{Thrapston Market \hspace*{\fill}\nolinebreak[1]%
\enspace\hspace*{\fill}
\finalhyphendemerits=0
[25th July]}

\index{Thrapston Market , East Northamptonshire@Thrapston Market, \emph{E. Northants.}}

Resignation of Michael Finch (C).

\noindent
\begin{tabular*}{\columnwidth}{@{\extracolsep{\fill}} p{0.545\columnwidth} >{\itshape}l r @{\extracolsep{\fill}}}
Alex Smith & C & 396\\
Valerie Carter & Ind & 210\\
Alex Izycky & Lab & 166\\
Joseph Garner & UKIP & 146\\
\end{tabular*}

\council{Kettering}

\subsubsection*{Avondale Grange \hspace*{\fill}\nolinebreak[1]%
\enspace\hspace*{\fill}
\finalhyphendemerits=0
[2nd May]}

\index{Avondale Grange , Kettering@Avondale Grange, \emph{Kettering}}

Resignation of Paul Corrazo (Lab).

\noindent
\begin{tabular*}{\columnwidth}{@{\extracolsep{\fill}} p{0.545\columnwidth} >{\itshape}l r @{\extracolsep{\fill}}}
Eileen Hales & Lab & 403\\
John Raffill & UKIP & 325\\
Christina Smith-Haynes & C & 140\\
Derek Hilling & EDP & 52\\
David Tate & LD & 37\\
\end{tabular*}

\subsubsection*{Welland \hspace*{\fill}\nolinebreak[1]%
\enspace\hspace*{\fill}
\finalhyphendemerits=0
[2nd May]}

\index{Welland , Kettering@Welland, \emph{Kettering}}

Resignation of Alison Wiley (C).

\noindent
\begin{tabular*}{\columnwidth}{@{\extracolsep{\fill}} p{0.545\columnwidth} >{\itshape}l r @{\extracolsep{\fill}}}
David Howes & C & 429\\
Paul Oakden & UKIP & 295\\
John Padwick & Lab & 199\\
Stanley Freeman & LD & 55\\
Kevin Sills & EDP & 14\\
\end{tabular*}

\council{Wellingborough}

\subsubsection*{Earls Barton \hspace*{\fill}\nolinebreak[1]%
\enspace\hspace*{\fill}
\finalhyphendemerits=0
[2nd May; C gain from Lab]}

\index{Earls Barton , Wellingborough@Earls Barton, \emph{Wellingborough}}

Death of Peter Wright (Lab).

\noindent
\begin{tabular*}{\columnwidth}{@{\extracolsep{\fill}} p{0.545\columnwidth} >{\itshape}l r @{\extracolsep{\fill}}}
Robert Gough & C & 626\\
Kevin Watts & Lab & 417\\
Debra Elderton & UKIP & 396\\
Daniel Jones & LD & 69\\
\end{tabular*}

\subsubsection*{Redwell West \hspace*{\fill}\nolinebreak[1]%
\enspace\hspace*{\fill}
\finalhyphendemerits=0
[2nd May]}

\index{Redwell West , Wellingborough@Redwell W., \emph{Wellingborough}}

Resignation of John Raymond (C).

\noindent
\begin{tabular*}{\columnwidth}{@{\extracolsep{\fill}} p{0.545\columnwidth} >{\itshape}l r @{\extracolsep{\fill}}}
Veronica Waters & C & 396\\
Allan Shipham & UKIP & 244\\
Elfred Brown & Lab & 193\\
Marshall Walker & EDP & 35\\
\end{tabular*}

\section{Nottinghamshire}

\council{Ashfield}

\subsubsection*{Sutton in Ashfield Central \hspace*{\fill}\nolinebreak[1]%
\enspace\hspace*{\fill}
\finalhyphendemerits=0
[2nd May]}

\index{Sutton in Ashfield Central , Ashfield@Sutton in Ashfield C., \emph{Ashfield}}

Resignation of Mick Coppin (Lab).

\noindent
\begin{tabular*}{\columnwidth}{@{\extracolsep{\fill}} p{0.545\columnwidth} >{\itshape}l r @{\extracolsep{\fill}}}
Jim Aspinall & Lab & 1250\\
Tony Wallis & Ind & 458\\
Shaun Hartley & C & 317\\
Anthony Brewer & LD & 184\\
\end{tabular*}

\council{Gedling}

\subsubsection*{Kingswell \hspace*{\fill}\nolinebreak[1]%
\enspace\hspace*{\fill}
\finalhyphendemerits=0
[2nd May]}

\index{Kingswell , Gedling@Kingswell, \emph{Gedling}}

Resignation of Paul Key (Lab).

\noindent
\begin{tabular*}{\columnwidth}{@{\extracolsep{\fill}} p{0.57\columnwidth} >{\itshape}l r @{\extracolsep{\fill}}}
David Ellis & Lab & 547\\
Michael Adams & C & 483\\
Lee Waters & UKIP & 397\\
Rhiann Stansfield-Coyne & LD & 80\\
\end{tabular*}

\council{Newark and Sherwood}

\subsubsection*{Collingham and Meering \hspace*{\fill}\nolinebreak[1]%
\enspace\hspace*{\fill}
\finalhyphendemerits=0
[2nd May]}

\index{Collingham and Meering , Newark and Sherwood@\sloppyword{Collingham \& Meering, \emph{Newark \& Sherwood}}}

Resignation of Kevin Rontree (C).

\noindent
\begin{tabular*}{\columnwidth}{@{\extracolsep{\fill}} p{0.545\columnwidth} >{\itshape}l r @{\extracolsep{\fill}}}
Derek Evans & C & 1112\\
Daniel Hibberd & Lab & 370\\
\end{tabular*}

\subsubsection*{Farnsfield and Bilsthorpe \hspace*{\fill}\nolinebreak[1]%
\enspace\hspace*{\fill}
\finalhyphendemerits=0
[27th June]}

\index{Farnsfield and Bilsthorpe , Newark and Sherwood@Farnsfield \& Bilsthorpe, \emph{Newark \& Sherwood}}

Death of Nora Armstrong (C).

\noindent
\begin{tabular*}{\columnwidth}{@{\extracolsep{\fill}} p{0.545\columnwidth} >{\itshape}l r @{\extracolsep{\fill}}}
Frank Taylor & C & 1174\\
Glenn Bardill & Lab & 682\\
\end{tabular*}

\council{Nottingham}

\subsubsection*{Bilborough \hspace*{\fill}\nolinebreak[1]%
\enspace\hspace*{\fill}
\finalhyphendemerits=0
[4th April]}

\index{Bilborough , Nottingham@Bilborough, \emph{Nottingham}}

Death of Derek Cresswell (Lab).

Elvis = Elvis Loves Pets Party

\noindent
\begin{tabular*}{\columnwidth}{@{\extracolsep{\fill}} p{0.545\columnwidth} >{\itshape}l r @{\extracolsep{\fill}}}
Wendy Smith & Lab & 1542\\
Irenea Marriott & UKIP & 347\\
Ian Culley & C & 176\\
Katharina Boettge & Grn & 103\\
John Calvert & LD & 96\\
David Bishop & Elvis & 31\\
\end{tabular*}

\subsubsection*{Wollaton East and Lenton Abbey \hspace*{\fill}\nolinebreak[1]%
\enspace\hspace*{\fill}
\finalhyphendemerits=0
[4th April]}

\index{Wollaton East and Lenton Abbey , Nottingham@Wollaton E. \& Lenton Abbey, \emph{Nottingham}}

Resignation of Stuart Fox (Lab).

\noindent
\begin{tabular*}{\columnwidth}{@{\extracolsep{\fill}} p{0.545\columnwidth} >{\itshape}l r @{\extracolsep{\fill}}}
Sam Webster & Lab & 627\\
Tony Sutton & LD & 368\\
Jeanna Parton & C & 116\\
Andrew Taylor & UKIP & 75\\
\end{tabular*}

\subsubsection*{Wollaton West \hspace*{\fill}\nolinebreak[1]%
\enspace\hspace*{\fill}
\finalhyphendemerits=0
[6th June; Lab gain from C]}

\index{Wollaton West , Nottingham@Wollaton W., \emph{Nottingham}}

Resignation of Stuart Fox (Lab).

Elvis = Militant Elvis Anti HS2

\noindent
\begin{tabular*}{\columnwidth}{@{\extracolsep{\fill}} p{0.545\columnwidth} >{\itshape}l r @{\extracolsep{\fill}}}
Steve Battlemuch & Lab & 2211\\
James Spencer & C & 1594\\
Chris Clarke & UKIP & 565\\
Barbara Pearce & LD & 216\\
Katharina Boettge & Grn & 106\\
David Bishop & Elvis & 28\\
\end{tabular*}

\subsubsection*{Dales \hspace*{\fill}\nolinebreak[1]%
\enspace\hspace*{\fill}
\finalhyphendemerits=0
[7th November]}

\index{Dales , Nottingham@Dales, \emph{Nottingham}}

Resignation of Ken Williams (Lab).

\noindent
\begin{tabular*}{\columnwidth}{@{\extracolsep{\fill}} p{0.5\columnwidth} >{\itshape}l r @{\extracolsep{\fill}}}
Neghat Khan & Lab & 1644\\
Irenea Marriott & UKIP & 364\\
\sloppyword{Neale Mittenshaw-Hodge} & C & 220\\
Adam McGregor & Grn & 99\\
Tad Jones & LD & 78\\
Cathy Meadows & TUSC & 72\\
\end{tabular*}

\subsubsection*{Radford and Park \hspace*{\fill}\nolinebreak[1]%
\enspace\hspace*{\fill}
\finalhyphendemerits=0
[7th November]}

\index{Radford and Park , Nottingham@Radford \& Park, \emph{Nottingham}}

Resignation of Steph Williams (Lab).

Elvis = Elvis Loves Pets Party

\noindent
\begin{tabular*}{\columnwidth}{@{\extracolsep{\fill}} p{0.545\columnwidth} >{\itshape}l r @{\extracolsep{\fill}}}
Anne Peach & Lab & 1146\\
Nicholas Packham & C & 355\\
Francesco Lari & UKIP & 123\\
Katharine Boettge & Grn & 80\\
David Bishop & Elvis & 31\\
Geraint Thomas & TUSC & 22\\
\end{tabular*}

\council{Rushcliffe}

\subsubsection*{Leake \hspace*{\fill}\nolinebreak[1]%
\enspace\hspace*{\fill}
\finalhyphendemerits=0
[2nd May]}

\index{Leake , Rushcliffe@Leake, \emph{Rushcliffe}}

Resignation of Brian Dale (C).

\noindent
\begin{tabular*}{\columnwidth}{@{\extracolsep{\fill}} p{0.545\columnwidth} >{\itshape}l r @{\extracolsep{\fill}}}
John Thurman & C & 793\\
Carys Thomas & Ind & 498\\
Steve Collins & Lab & 345\\
Matthew Faithfull & UKIP & 296\\
\end{tabular*}

\section{Oxfordshire}

\council{Cherwell}

\subsubsection*{Hook Norton \hspace*{\fill}\nolinebreak[1]%
\enspace\hspace*{\fill}
\finalhyphendemerits=0
[2nd May]}

\index{Hook Norton , Cherwell@Hook Norton, \emph{Cherwell}}

Resignation of Victoria Irvine (C).

\noindent
\begin{tabular*}{\columnwidth}{@{\extracolsep{\fill}} p{0.545\columnwidth} >{\itshape}l r @{\extracolsep{\fill}}}
Ray Jelf & C & 511\\
Perran Moon & Lab & 155\\
Colin Clark & Grn & 101\\
\end{tabular*}

\subsubsection*{Banbury Ruscote \hspace*{\fill}\nolinebreak[1]%
\enspace\hspace*{\fill}
\finalhyphendemerits=0
[26th September]}

\index{Banbury Ruscote , Cherwell@Banbury Ruscote, \emph{Cherwell}}

Resignation of Pat Cartledge (Lab).

\noindent
\begin{tabular*}{\columnwidth}{@{\extracolsep{\fill}} p{0.545\columnwidth} >{\itshape}l r @{\extracolsep{\fill}}}
Mark Cherry & Lab & 758\\
Pat Tompson & C & 323\\
Christian Miller & UKIP & 206\\
\end{tabular*}

\council{Oxford}

\subsubsection*{Churchill \hspace*{\fill}\nolinebreak[1]%
\enspace\hspace*{\fill}
\finalhyphendemerits=0
[2nd May]}

\index{Churchill , Oxford@Churchill, \emph{Oxford}}

Resignation of Joe McManners (Lab).

\noindent
\begin{tabular*}{\columnwidth}{@{\extracolsep{\fill}} p{0.545\columnwidth} >{\itshape}l r @{\extracolsep{\fill}}}
Susan Brown & Lab & 555\\
Julian Faultless & Grn & 163\\
Gary Dixon & C & 138\\
Nicholas Dewey & LD & 63\\
\end{tabular*}

\subsubsection*{Rose Hill and Iffley \hspace*{\fill}\nolinebreak[1]%
\enspace\hspace*{\fill}
\finalhyphendemerits=0
[2nd May]}

\index{Rose Hill and Iffley , Oxford@Rose Hill \& Iffley, \emph{Oxford}}

Resignation of Antonia Bance (Lab).

\noindent
\begin{tabular*}{\columnwidth}{@{\extracolsep{\fill}} p{0.545\columnwidth} >{\itshape}l r @{\extracolsep{\fill}}}
Michele Paule & Lab & 735\\
Timothy Patmore & C & 211\\
Manishta Sunnia & Grn & 200\\
Michael Tait & LD & 87\\
\end{tabular*}

\subsubsection*{North \hspace*{\fill}\nolinebreak[1]%
\enspace\hspace*{\fill}
\finalhyphendemerits=0
[19th September; Lab gain from LD]}

\index{North , Oxford@North, \emph{Oxford}}

Resignation of Alan Armitage (LD).

\noindent
\begin{tabular*}{\columnwidth}{@{\extracolsep{\fill}} p{0.545\columnwidth} >{\itshape}l r @{\extracolsep{\fill}}}
Louise Upton & Lab & 367\\
Tim Bearder & LD & 330\\
Sushila Dhall & Grn & 236\\
John Walsh & C & 100\\
\end{tabular*}

\council{Vale of White Horse}

\subsubsection*{Abingdon Peachcroft \hspace*{\fill}\nolinebreak[1]%
\enspace\hspace*{\fill}
\finalhyphendemerits=0
[2nd May; LD gain from C]}

\index{Abingdon Peachcroft , Vale of White Horse@Abingdon Peachcroft, \emph{Vale of White Horse}}

Disqualification (non-attendance) of Peter Jones (C).

\noindent
\begin{tabular*}{\columnwidth}{@{\extracolsep{\fill}} p{0.545\columnwidth} >{\itshape}l r @{\extracolsep{\fill}}}
Andrew Skinner & LD & 683\\
Peter Wiblin & C & 518\\
George Ryall & Lab & 133\\
\end{tabular*}

\subsubsection*{Greendown \hspace*{\fill}\nolinebreak[1]%
\enspace\hspace*{\fill}
\finalhyphendemerits=0
[2nd May; C gain from LD]}

\index{Greendown , Vale of White Horse@Greendown, \emph{Vale of White Horse}}

Resignation of Andrew Crawford (LD).

\noindent
\begin{tabular*}{\columnwidth}{@{\extracolsep{\fill}} p{0.545\columnwidth} >{\itshape}l r @{\extracolsep{\fill}}}
St John Dickson & C & 351\\
Brian Sadler & LD & 199\\
Jason Kent & UKIP & 115\\
\end{tabular*}

\subsubsection*{Marcham and Shippon \hspace*{\fill}\nolinebreak[1]%
\enspace\hspace*{\fill}
\finalhyphendemerits=0
[2nd May]}

\index{Marcham and Shippon , Vale of White Horse@Marcham \& Shippon, \emph{Vale of White Horse}}

Resignation of Jane Hanna (LD).

\noindent
\begin{tabular*}{\columnwidth}{@{\extracolsep{\fill}} p{0.545\columnwidth} >{\itshape}l r @{\extracolsep{\fill}}}
Catherine Webber & LD & 308\\
Jackie Gibb & C & 260\\
Christopher Parkes & UKIP & 140\\
\end{tabular*}

\subsubsection*{Abingdon Fitzharris \hspace*{\fill}\nolinebreak[1]%
\enspace\hspace*{\fill}
\finalhyphendemerits=0
[4th July; LD gain from C]}

\index{Abingdon Fitzharris , Vale of White Horse@\sloppyword{Abingdon Fitzharris, \emph{Vale of White Horse}}}

Resignation of Holly Holman (C).

\noindent
\begin{tabular*}{\columnwidth}{@{\extracolsep{\fill}} p{0.545\columnwidth} >{\itshape}l r @{\extracolsep{\fill}}}
Jeanette Halliday & LD & 479\\
Monica Lovatt & C & 378\\
George Ryall & Lab & 96\\
\end{tabular*}

\council{West Oxfordshire}

\subsubsection*{Carterton South \hspace*{\fill}\nolinebreak[1]%
\enspace\hspace*{\fill}
\finalhyphendemerits=0
[2nd May]}

\index{Carterton South , West Oxfordshire@Carterton S., \emph{W. Oxon.}}

Resignation of Joe Walcott (C).

\noindent
\begin{tabular*}{\columnwidth}{@{\extracolsep{\fill}} p{0.545\columnwidth} >{\itshape}l r @{\extracolsep{\fill}}}
Lynn Little & C & 477\\
Dave Wesson & Lab & 108\\
Alma Tumilowicz & Grn & 78\\
Amanda Epps & LD & 44\\
\end{tabular*}

\subsubsection*{Witney East \hspace*{\fill}\nolinebreak[1]%
\enspace\hspace*{\fill}
\finalhyphendemerits=0
[2nd May]}

\index{Witney East , West Oxfordshire@Witney E., \emph{W. Oxon.}}

Resignation of Sian Davies (C).

\noindent
\begin{tabular*}{\columnwidth}{@{\extracolsep{\fill}} p{0.545\columnwidth} >{\itshape}l r @{\extracolsep{\fill}}}
Jeanette Baker & C & 794\\
Alfred Fullah & Lab & 642\\
Kate Griffin & Grn & 270\\
\end{tabular*}

\subsubsection*{Chipping Norton \hspace*{\fill}\nolinebreak[1]%
\enspace\hspace*{\fill}
\finalhyphendemerits=0
[7th November]}

\index{Chipping Norton , West Oxfordshire@Chipping Norton, \emph{W. Oxon.}}

Death of Rob Evans (Lab).

\noindent
\begin{tabular*}{\columnwidth}{@{\extracolsep{\fill}} p{0.545\columnwidth} >{\itshape}l r @{\extracolsep{\fill}}}
Geoff Saul & Lab & 810\\
Joe Johnson & C & 500\\
Matthew Clayton & Grn & 58\\
Andrew Crick & LD & 53\\
\end{tabular*}

\section{Rutland}\index{Rutland}

\subsubsection*{Ketton \hspace*{\fill}\nolinebreak[1]%
\enspace\hspace*{\fill}
\finalhyphendemerits=0
[27th June]}

\index{Ketton , Rutland@Ketton, \emph{Rutland}}

Resignation of Barrie Roper (C).

\noindent
\begin{tabular*}{\columnwidth}{@{\extracolsep{\fill}} p{0.545\columnwidth} >{\itshape}l r @{\extracolsep{\fill}}}
Gary Conde & C & 330\\
Andrew McGilvray & Ind & 260\\
Liam Powell & UKIP & 130\\
Martin Brookes & Ind & 24\\
\end{tabular*}

\section{Shropshire}

\council{Telford and Wrekin}

\subsubsection*{Dawley Magna \hspace*{\fill}\nolinebreak[1]%
\enspace\hspace*{\fill}
\finalhyphendemerits=0
[14th February]}

\index{Dawley Magna , Telford and Wrekin@Dawley Magna, \emph{Telford \& Wrekin}}

Death of Brian Duce (Lab).

\noindent
\begin{tabular*}{\columnwidth}{@{\extracolsep{\fill}} p{0.545\columnwidth} >{\itshape}l r @{\extracolsep{\fill}}}
Jane Pinter & Lab & 957\\
Alan Scott & C & 379\\
Ryan Laing & UKIP & 312\\
Angela Jordan & Ind & 126\\
\end{tabular*}

\section{Somerset}

\subsection*{County Council}\index{Somerset}

\subsubsection*{Coker \hspace*{\fill}\nolinebreak[1]%
\enspace\hspace*{\fill}
\finalhyphendemerits=0
[16th May]}

\index{Coker , Somerset@Coker, \emph{Somerset}}

Ordinary election postponed from 2nd May: death of candidate Audrey Spencer (UKIP).

%See page \pageref{CokerSomerset} for the result.

\noindent
\begin{tabular*}{\columnwidth}{@{\extracolsep{\fill}} p{0.545\columnwidth} >{\itshape}l r @{\extracolsep{\fill}}}
Marcus Fish & C & 1303\\
Ian Stephen & LD & 1079\\
Vhie Boxall & UKIP & 702\\
Murry Shepstone & Lab & 195\\
Peter Bysouth & Grn & 165\\
\end{tabular*}

\council{Mendip}

\subsubsection*{\sloppyword{Ashwick, Chilcompton and Stratton} \hspace*{\fill}\nolinebreak[1]%
\enspace\hspace*{\fill}
\finalhyphendemerits=0
[2nd May]}

\index{Ashwick, Chilcompton and Stratton , Mendip@\sloppyword{Ashwick, Chilcompton \& Stratton, \emph{Mendip}}}

Resignation of Wayne Closier (C).

\noindent
\begin{tabular*}{\columnwidth}{@{\extracolsep{\fill}} p{0.545\columnwidth} >{\itshape}l r @{\extracolsep{\fill}}}
Rachel Carter & C & 634\\
Roger Anderson & Lab & 262\\
Christine Cockcroft & LD & 175\\
\end{tabular*}

\subsubsection*{Rode and Norton St Philip \hspace*{\fill}\nolinebreak[1]%
\enspace\hspace*{\fill}
\finalhyphendemerits=0
[2nd May]}

\index{Rode and Norton Saint Philip , Mendip@Rode \& Norton St Philip, \emph{Mendip}}

Resignation of Matthew Ellis (C).

\noindent
\begin{tabular*}{\columnwidth}{@{\extracolsep{\fill}} p{0.6\columnwidth} >{\itshape}l r @{\extracolsep{\fill}}}
Linda Oliver & C & 452\\
David Brown & LD & 306\\
Catherine Richardson & Lab & 72\\
\end{tabular*}

\subsubsection*{Shepton West \hspace*{\fill}\nolinebreak[1]%
\enspace\hspace*{\fill}
\finalhyphendemerits=0
[2nd May; Lab gain from LD]}

\index{Shepton West , Mendip@Shepton W., \emph{Mendip}}

Resignation of Sue Cook (LD).

\noindent
\begin{tabular*}{\columnwidth}{@{\extracolsep{\fill}} p{0.545\columnwidth} >{\itshape}l r @{\extracolsep{\fill}}}
Chris Inchley & Lab & 437\\
Simon Davies & C & 380\\
Richard Champion & LD & 335\\
Chris Briton & Grn & 106\\
\end{tabular*}

\council{North Somerset}

\subsubsection*{Weston-super-Mare South \hspace*{\fill}\nolinebreak[1]%
\enspace\hspace*{\fill}
\finalhyphendemerits=0
[2nd May]}

\index{Weston-super-Mare South , North Somerset@Weston-super-Mare S., \emph{N. Somerset}}

Resignation of Deborah Stone (Lab).

\noindent
\begin{tabular*}{\columnwidth}{@{\extracolsep{\fill}} p{0.545\columnwidth} >{\itshape}l r @{\extracolsep{\fill}}}
James Clayton & Lab & 1044\\
Louis Rostill & UKIP & 449\\
Rachel Ling & LD & 214\\
John Butler & C & 192\\
\end{tabular*}

\subsubsection*{\sloppyword{Weston-super-Mare North Worle} \hspace*{\fill}\nolinebreak[1]%
\enspace\hspace*{\fill}
\finalhyphendemerits=0
[25th July; Ind gain from C]}

\index{Weston-super-Mare North Worle , North Somerset@\sloppyword{Weston-super-Mare North Worle, \emph{N. Somerset}}}

Resignation of Philip Judd (C).

\noindent
\begin{tabular*}{\columnwidth}{@{\extracolsep{\fill}} p{0.59\columnwidth} >{\itshape}l r @{\extracolsep{\fill}}}
Derek Mead & Ind & 531\\
Richard Nightingale & C & 471\\
Denise Hunt & Lab & 445\\
Edward Keating & LD & 321\\
Steven Pearse-Danker & UKIP & 220\\
Rachel Ling & Ind & 93\\
\end{tabular*}

\council{Sedgemoor}

\subsubsection*{Highbridge and Burnham Marine \hspace*{\fill}\nolinebreak[1]%
\enspace\hspace*{\fill}
\finalhyphendemerits=0
[24th January]}

\index{Highbridge and Burnham Marine , Sedgemoor@Highbridge \& Burnham Marine, \emph{Sedgemoor}}

Resignation of Joe Leach (LD).

\noindent
\begin{tabular*}{\columnwidth}{@{\extracolsep{\fill}} p{0.545\columnwidth} >{\itshape}l r @{\extracolsep{\fill}}}
Helen Groves & LD & 379\\
Bill Hancock & C & 288\\
Sally Williams & Ind & 253\\
Ricky Holcombe & Lab & 249\\
Purple Watkins & Ind & 68\\
\end{tabular*}

\council{Taunton Deane}

\subsubsection*{Taunton Halcon \hspace*{\fill}\nolinebreak[1]%
\enspace\hspace*{\fill}
\finalhyphendemerits=0
[2nd May]}

\index{Taunton Halcon , Taunton Deane@Taunton Halcon, \emph{Taunton Deane}}

Resignation of Melvyn Mullins (LD).

\noindent
\begin{tabular*}{\columnwidth}{@{\extracolsep{\fill}} p{0.545\columnwidth} >{\itshape}l r @{\extracolsep{\fill}}}
Judith Gaden & LD & 457\\
Laura Bailhache & UKIP & 295\\
Aaron Miller & C & 179\\
Martin Jevon & Lab & 159\\
Alan Debenham & Grn & 78\\
\end{tabular*}

\subsubsection*{Taunton Halcon \hspace*{\fill}\nolinebreak[1]%
\enspace\hspace*{\fill}
\finalhyphendemerits=0
[3rd October]}

\index{Taunton Halcon , Taunton Deane@Taunton Halcon, \emph{Taunton Deane}}

Resignation of Steve Brooks (LD).

\noindent
\begin{tabular*}{\columnwidth}{@{\extracolsep{\fill}} p{0.545\columnwidth} >{\itshape}l r @{\extracolsep{\fill}}}
Federica Smith & LD & 282\\
Dorothy Baker & UKIP & 172\\
Marcus Palmer & C & 165\\
Anna Lynch & Lab & 146\\
\end{tabular*}

\section{Staffordshire}

\subsection*{County Council}\index{Staffordshire}

At the May 2013 ordinary election there were unfilled vacancies in Perton and Stonydelph divisions due to the deaths of David Billson and Brian Beale (both C) respectively.
\index{Perton , Staffordshire@Perton, \emph{Staffs.}}
\index{Stonydelph , Staffordshire@Stonydelph, \emph{Staffs.}}

\council{Lichfield}

\subsubsection*{Fazeley \hspace*{\fill}\nolinebreak[1]%
\enspace\hspace*{\fill}
\finalhyphendemerits=0
[30th May]}

\index{Fazeley , Lichfield@Fazeley, \emph{Lichfield}}

Resignation of Ian Lewin (C).

\noindent
\begin{tabular*}{\columnwidth}{@{\extracolsep{\fill}} p{0.545\columnwidth} >{\itshape}l r @{\extracolsep{\fill}}}
Doug Pullen & C & 423\\
Dave Whatton & Lab & 375\\
\end{tabular*}

\council{Newcastle-under-Lyme}

\subsubsection*{Silverdale and Parksite \hspace*{\fill}\nolinebreak[1]%
\enspace\hspace*{\fill}
\finalhyphendemerits=0
[4th July]}

\index{Silverdale and Parksite , Newcastle-under-Lyme@Silverdale \& Parksite, \emph{Newcastle-under-Lyme}}

Resignation of Thomas Lawton (Lab).

\noindent
\begin{tabular*}{\columnwidth}{@{\extracolsep{\fill}} p{0.545\columnwidth} >{\itshape}l r @{\extracolsep{\fill}}}
Amelia Rout & Lab & 387\\
Elaine Blake & UKIP & 254\\
James Vernon & C & 58\\
Richard Steele & TUSC & 14\\
\end{tabular*}

\council{South Staffordshire}

\subsubsection*{Perton East \hspace*{\fill}\nolinebreak[1]%
\enspace\hspace*{\fill}
\finalhyphendemerits=0
[14th February; Ind gain from C]}

\index{Perton East , South Staffordshire@Perton E., \emph{S. Staffs.}}

Death of David Billson (C).

\noindent
\begin{tabular*}{\columnwidth}{@{\extracolsep{\fill}} p{0.545\columnwidth} >{\itshape}l r @{\extracolsep{\fill}}}
Anthony Bourke & Ind & 314\\
Nigel Caine & C & 168\\
\end{tabular*}

\subsubsection*{Brewood and Coven \hspace*{\fill}\nolinebreak[1]%
\enspace\hspace*{\fill}
\finalhyphendemerits=0
[10th October]}

\index{Brewood and Coven , South Staffordshire@Brewood \& Coven, \emph{S. Staffs.}}

Resignation of Ivor Clay (C).

\noindent
\begin{tabular*}{\columnwidth}{@{\extracolsep{\fill}} p{0.545\columnwidth} >{\itshape}l r @{\extracolsep{\fill}}}
Wendy Sutton & C & 459\\
Lorna Jones & Lab & 352\\
Christopher Lenton & UKIP & 225\\
Moira Alden-Court & Ind & 89\\
\end{tabular*}

\council{Stafford}

\subsubsection*{Coton \hspace*{\fill}\nolinebreak[1]%
\enspace\hspace*{\fill}
\finalhyphendemerits=0
[2nd May]}

\index{Coton , Stafford@Coton, \emph{Stafford}}

Death of Tony Welch (Lab).

\noindent
\begin{tabular*}{\columnwidth}{@{\extracolsep{\fill}} p{0.545\columnwidth} >{\itshape}l r @{\extracolsep{\fill}}}
Sharon Hollinshead & Lab & 448\\
Jonathan Price & C & 251\\
\end{tabular*}

\council{Staffordshire Moorlands}

Moorlnds = Moorlands Democratic Alliance

\subsubsection*{Leek North \hspace*{\fill}\nolinebreak[1]%
\enspace\hspace*{\fill}
\finalhyphendemerits=0
[2nd May]}

\index{Leek North , Staffordshire Moorlands@Leek N., \emph{Staffs. Moorlands}}

Resignation of Sandra Cooper (Lab).

\noindent
\begin{tabular*}{\columnwidth}{@{\extracolsep{\fill}} p{0.5\columnwidth} >{\itshape}l r @{\extracolsep{\fill}}}
Darren Price & Lab & 397\\
Matthew Cooper & UKIP & 347\\
Bob Bestwick & C & 260\\
Maureen Motum & Moorlnds & 203\\
Roy Gregg & LD & 23\\
\end{tabular*}

\council{Stoke-on-Trent}

CityInd = City Independents

\subsubsection*{Baddeley, Milton and Norton \hspace*{\fill}\nolinebreak[1]%
\enspace\hspace*{\fill}
\finalhyphendemerits=0
[14th November; CityInd gain from Lab]}

\index{Baddeley, Milton and Norton , Stoke-on-Trent@Baddeley, Milton \& Norton, \emph{Stoke-on-Trent}}

Resignation of Andy Lilley (Ind elected as Lab).

\noindent
\begin{tabular*}{\columnwidth}{@{\extracolsep{\fill}} p{0.5\columnwidth} >{\itshape}l r @{\extracolsep{\fill}}}
Anthony Munday & CityInd & 861\\
Sam Richardson & C & 504\\
Candi Chetwynd & Lab & 444\\
Mick Harold & UKIP & 333\\
Gary Elsby & Ind & 313\\
Michael White & BNP & 79\\
Adam Colclough & Grn & 50\\
Tom Grocock & LD & 32\\
John Davis & Ind & 27\\
Liat Norris & TUSC & 25\\
\end{tabular*}

\council{Tamworth}

\subsubsection*{Wilnecote \hspace*{\fill}\nolinebreak[1]%
\enspace\hspace*{\fill}
\finalhyphendemerits=0
[2nd May; Lab gain from C]}

\index{Wilnecote , Tamworth@Wilnecote, \emph{Tamworth}}

Death of Brian Beale (C).

\noindent
\begin{tabular*}{\columnwidth}{@{\extracolsep{\fill}} p{0.545\columnwidth} >{\itshape}l r @{\extracolsep{\fill}}}
Joan Jenkins & Lab & 890\\
Alex Farrell & C & 873\\
Roger Jones & LD & 87\\
\end{tabular*}

\section{Suffolk}

\council{Babergh}

\subsubsection*{Bures St Mary \hspace*{\fill}\nolinebreak[1]%
\enspace\hspace*{\fill}
\finalhyphendemerits=0
[2nd May]}

\index{Bures Saint Mary , Babergh@Bures St Mary, \emph{Babergh}}

Resignation of Peter Holbrook (C).

\noindent
\begin{tabular*}{\columnwidth}{@{\extracolsep{\fill}} p{0.545\columnwidth} >{\itshape}l r @{\extracolsep{\fill}}}
James Cartlidge & C & 305\\
Laura Smith & Grn & 136\\
Hadley French Gerrard & Lab & 71\\
\end{tabular*}

\council{Forest Heath}

\subsubsection*{Exning \hspace*{\fill}\nolinebreak[1]%
\enspace\hspace*{\fill}
\finalhyphendemerits=0
[11th July; Ind gain from LD]}

\index{Exning , Forest Heath@Exning, \emph{Forest Heath}}

Resignation of Neil Williams (LD).

\noindent
\begin{tabular*}{\columnwidth}{@{\extracolsep{\fill}} p{0.6\columnwidth} >{\itshape}l r @{\extracolsep{\fill}}}
Simon Cole & Ind & 263\\
Marion Fairman-Smith & C & 147\\
\end{tabular*}

\subsubsection*{Market \hspace*{\fill}\nolinebreak[1]%
\enspace\hspace*{\fill}
\finalhyphendemerits=0
[19th December]}

\index{Market , Forest Heath@Market, \emph{Forest Heath}}

Death of Malcolm Smith (C).

\noindent
\begin{tabular*}{\columnwidth}{@{\extracolsep{\fill}} p{0.545\columnwidth} >{\itshape}l r @{\extracolsep{\fill}}}
John Bloodworth & C & 266\\
David Chandler & UKIP & 263\\
\end{tabular*}

\council{Ipswich}

\subsubsection*{Alexandra \hspace*{\fill}\nolinebreak[1]%
\enspace\hspace*{\fill}
\finalhyphendemerits=0
[2nd May; Lab gain from LD]}

\index{Alexandra , Ipswich@Alexandra, \emph{Ipswich}}

Resignation of Ken Bates (LD).

\noindent
\begin{tabular*}{\columnwidth}{@{\extracolsep{\fill}} p{0.545\columnwidth} >{\itshape}l r @{\extracolsep{\fill}}}
John Cook & Lab & 772\\
Alan Cotterell & UKIP & 279\\
Edward Phillips & C & 274\\
Tom Wilmot & Grn & 193\\
Ken Toye & LD & 126\\
\end{tabular*}

\subsubsection*{Whitehouse \hspace*{\fill}\nolinebreak[1]%
\enspace\hspace*{\fill}
\finalhyphendemerits=0
[2nd May]}

\index{Whitehouse , Ipswich@Whitehouse, \emph{Ipswich}}

Resignation of David Ball (Lab).

\noindent
\begin{tabular*}{\columnwidth}{@{\extracolsep{\fill}} p{0.545\columnwidth} >{\itshape}l r @{\extracolsep{\fill}}}
Colin Wright & Lab & 630\\
James Crossley & UKIP & 537\\
Paul West & C & 238\\
Geoff Reynish & Grn & 55\\
Moira Kleissner & LD & 52\\
\end{tabular*}

\subsection*{St Edmundsbury}\index{Saint Edmundsbury@St Edmundsbury}

\subsubsection*{Abbeygate \hspace*{\fill}\nolinebreak[1]%
\enspace\hspace*{\fill}
\finalhyphendemerits=0
[2nd May]}

\index{Abbeygate , Saint Edmundsbury@Abbeygate, \emph{St Edmundsbury}}

Resignation of Richard Rout (C).

\noindent
\begin{tabular*}{\columnwidth}{@{\extracolsep{\fill}} p{0.545\columnwidth} >{\itshape}l r @{\extracolsep{\fill}}}
Charlotte Howard & C & 562\\
Philippa Judd & Grn & 399\\
Quentin Cornish & Lab & 154\\
Judith Broadway & LD & 149\\
\end{tabular*}

\subsubsection*{Bardwell\hspace*{\fill}\nolinebreak[1]%
\enspace\hspace*{\fill}
\finalhyphendemerits=0
[5th September]}

\index{Bardwell , Saint Edmundsbury@Bardwell, \emph{St Edmundsbury}}

Death of John Hale (C).

\noindent
\begin{tabular*}{\columnwidth}{@{\extracolsep{\fill}} p{0.545\columnwidth} >{\itshape}l r @{\extracolsep{\fill}}}
Paula Wade & C & 419\\
James Lumley & UKIP & 150\\
Thomas Stebbing & Lab & 65\\
\end{tabular*}

\subsubsection*{Abbeygate \hspace*{\fill}\nolinebreak[1]%
\enspace\hspace*{\fill}
\finalhyphendemerits=0
[3rd October]}

\index{Abbeygate , Saint Edmundsbury@Abbeygate, \emph{St Edmundsbury}}

Death of Charlotte Howard (C).

\noindent
\begin{tabular*}{\columnwidth}{@{\extracolsep{\fill}} p{0.545\columnwidth} >{\itshape}l r @{\extracolsep{\fill}}}
Joanna Rayner & C & 359\\
Mark Ereira-Guyer & Grn & 236\\
Clive Reason & UKIP & 85\\
Chris Lale & LD & 83\\
Quentin Cornish & Lab & 76\\
\end{tabular*}

\council{Waveney}

\subsubsection*{Harbour \hspace*{\fill}\nolinebreak[1]%
\enspace\hspace*{\fill}
\finalhyphendemerits=0
[2nd May]}

\index{Harbour , Waveney@Harbour, \emph{Waveney}}

Resignation of Tess Gandy (Lab).

\noindent
\begin{tabular*}{\columnwidth}{@{\extracolsep{\fill}} p{0.545\columnwidth} >{\itshape}l r @{\extracolsep{\fill}}}
Janet Craig & Lab & 647\\
Bertie Poole & UKIP & 358\\
Anthony Taylor & C & 217\\
George Langley & Grn & 85\\
Christopher Thomas & LD & 36\\
\end{tabular*}

\subsubsection*{Oulton \hspace*{\fill}\nolinebreak[1]%
\enspace\hspace*{\fill}
\finalhyphendemerits=0
[8th August]}

\index{Oulton , Waveney@Oulton, \emph{Waveney}}

Death of Mike Ives-Keeler (Lab).

\noindent
\begin{tabular*}{\columnwidth}{@{\extracolsep{\fill}} p{0.545\columnwidth} >{\itshape}l r @{\extracolsep{\fill}}}
Len Jacklin & Lab & 449\\
Deanna Law & C & 329\\
Bert Poole & UKIP & 269\\
Maxine Narburgh & Grn & 23\\
Chris Thomas & LD & 21\\
\end{tabular*}

\section{Surrey}

\subsection*{County Council}\index{Surrey}

At the May 2013 ordinary election there was an unfilled vacancy in Earlswood and Reigate South division due to the death of Frances King (C).

\index{Earlswood and Reigate South , Surrey@Earlswood \& Reigate S., \emph{Surrey}}

\council{Elmbridge}

\subsubsection*{Claygate \hspace*{\fill}\nolinebreak[1]%
\enspace\hspace*{\fill}
\finalhyphendemerits=0
[2nd May]}

\index{Claygate , Elmbridge@Claygate, \emph{Elmbridge}}

Resignation of Jimmy Cartwright (LD).

\noindent
\begin{tabular*}{\columnwidth}{@{\extracolsep{\fill}} p{0.545\columnwidth} >{\itshape}l r @{\extracolsep{\fill}}}
Mary Marshall & LD & 1046\\
Mark Sugden & C & 723\\
Bernard Collignon & UKIP & 239\\
\end{tabular*}

\subsubsection*{Weybridge South \hspace*{\fill}\nolinebreak[1]%
\enspace\hspace*{\fill}
\finalhyphendemerits=0
[25th July]}

\index{Weybridge South , Elmbridge@Weybridge S., \emph{Elmbridge}}

Resignation of Simon Dodsworth (C).

\noindent
\begin{tabular*}{\columnwidth}{@{\extracolsep{\fill}} p{0.545\columnwidth} >{\itshape}l r @{\extracolsep{\fill}}}
Richard Knight & C & 274\\
Gillian Solway & LD & 150\\
Ian Lake & UKIP & 140\\
\end{tabular*}

\council{Guildford}

\subsubsection*{Ash South and Tongham \hspace*{\fill}\nolinebreak[1]%
\enspace\hspace*{\fill}
\finalhyphendemerits=0
[2nd May]}

\index{Ash South and Tongham , Guildford@Ash S. \& Tongham, \emph{Guildford}}

Resignation of Doug Richards (C).

\noindent
\begin{tabular*}{\columnwidth}{@{\extracolsep{\fill}} p{0.545\columnwidth} >{\itshape}l r @{\extracolsep{\fill}}}
Paul Spooner & C & \emph{unop.}\\
\end{tabular*}

\subsubsection*{Ash Wharf \hspace*{\fill}\nolinebreak[1]%
\enspace\hspace*{\fill}
\finalhyphendemerits=0
[2nd May]}

\index{Ash Wharf , Guildford@Ash Wharf, \emph{Guildford}}

Resignation of John Randall (C).

\noindent
\begin{tabular*}{\columnwidth}{@{\extracolsep{\fill}} p{0.545\columnwidth} >{\itshape}l r @{\extracolsep{\fill}}}
Murray Grubb & C & 687\\
Alan Hilliar & LD & 420\\
Joan O'Byrne & Lab & 225\\
\end{tabular*}

\subsubsection*{Stoughton \hspace*{\fill}\nolinebreak[1]%
\enspace\hspace*{\fill}
\finalhyphendemerits=0
[2nd May]}

\index{Stoughton , Guildford@Stoughton, \emph{Guildford}}

Resignation of Wendy May (LD).

\noindent
\begin{tabular*}{\columnwidth}{@{\extracolsep{\fill}} p{0.545\columnwidth} >{\itshape}l r @{\extracolsep{\fill}}}
Elizabeth Griffiths & LD & 999\\
Sharon Stokes & C & 768\\
William Cooper & Lab & 312\\
\end{tabular*}

\council{Mole Valley}

\subsubsection*{\sloppyword{Mickleham, Westhumble and Pixham} \hspace*{\fill}\nolinebreak[1]%
\enspace\hspace*{\fill}
\finalhyphendemerits=0
[26th September]}

\index{Mickleham, Westhumble and Pixham , Mole Valley@\sloppyword{Mickleham, Westhumble \& Pixham, \emph{Mole Valley}}}

Resignation of Rebecca McCheyne (LD).

\noindent
\begin{tabular*}{\columnwidth}{@{\extracolsep{\fill}} p{0.545\columnwidth} >{\itshape}l r @{\extracolsep{\fill}}}
Roger Hurst & LD & 423\\
Duncan Irvine & C & 236\\
Adrian Daniels & UKIP & 101\\
\end{tabular*}

\council{Reigate and Banstead}

\subsubsection*{South Park and Woodhatch \hspace*{\fill}\nolinebreak[1]%
\enspace\hspace*{\fill}
\finalhyphendemerits=0
[2nd May]}

\index{South Park and Woodhatch , Reigate and Banstead@South Park \& Woodhatch, \emph{Reigate \& Banstead}}

Death of Frances King (C).

\noindent
\begin{tabular*}{\columnwidth}{@{\extracolsep{\fill}} p{0.545\columnwidth} >{\itshape}l r @{\extracolsep{\fill}}}
Simon Rickman & C & 546\\
Joseph Fox & UKIP & 518\\
Helen Young & Lab & 327\\
Lynne Burnham & Grn & 149\\
Anthony Lambell & LD & 115\\
\end{tabular*}

\council{Runnymede}

\subsubsection*{Foxhills \hspace*{\fill}\nolinebreak[1]%
\enspace\hspace*{\fill}
\finalhyphendemerits=0
[14th March; UKIP gain from C]}

\index{Foxhills , Runnymede@Foxhills, \emph{Runnymede}}

Death of Frances Barden (C).

\noindent
\begin{tabular*}{\columnwidth}{@{\extracolsep{\fill}} p{0.545\columnwidth} >{\itshape}l r @{\extracolsep{\fill}}}
Chris Browne & UKIP & 336\\
Barry Pitt & C & 318\\
John Gurney & Lab & 181\\
\end{tabular*}

\subsubsection*{Chertsey South and Row Town \hspace*{\fill}\nolinebreak[1]%
\enspace\hspace*{\fill}
\finalhyphendemerits=0
[2nd May; Ind gain from C]}

\index{Chertsey South and Row Town , Runnymede@Chertsey S. \& Row Town, \emph{Runnymede}}

Resignation of Stewart Mackay (C).

\noindent
\begin{tabular*}{\columnwidth}{@{\extracolsep{\fill}} p{0.545\columnwidth} >{\itshape}l r @{\extracolsep{\fill}}}
Gillian Ellis & Ind & 525\\
Mark Nuti & C & 476\\
Paul Stevens & UKIP & 361\\
Arran Neathey & Lab & 143\\
\end{tabular*}

\council{Spelthorne}

\subsubsection*{Sunbury East \hspace*{\fill}\nolinebreak[1]%
\enspace\hspace*{\fill}
\finalhyphendemerits=0
[2nd May; C gain from LD]}

\index{Sunbury East , Spelthorne@Sunbury E., \emph{Spelthorne}}

Resignation of Caroline Nichols (LD).

\noindent
\begin{tabular*}{\columnwidth}{@{\extracolsep{\fill}} p{0.545\columnwidth} >{\itshape}l r @{\extracolsep{\fill}}}
Ian Harvey & C & 635\\
Kathy Grant & LD & 583\\
Robert Bromley & UKIP & 468\\
\end{tabular*}

\subsubsection*{Riverside and Laleham \hspace*{\fill}\nolinebreak[1]%
\enspace\hspace*{\fill}
\finalhyphendemerits=0
[7th November]}

\index{Riverside and Laleham , Spelthorne@Riverside \& Laleham, \emph{Spelthorne}}

Death of Isobel Napper (C).

\noindent
\begin{tabular*}{\columnwidth}{@{\extracolsep{\fill}} p{0.59\columnwidth} >{\itshape}l r @{\extracolsep{\fill}}}
Denise Saliagopoulos & C & 895\\
Michael Fuller & UKIP & 441\\
John Johnston & Lab & 227\\
Susan Vincent & LD & 56\\
\end{tabular*}

\council{Tandridge}

\subsubsection*{Burstow, Horne and Outwood \hspace*{\fill}\nolinebreak[1]%
\enspace\hspace*{\fill}
\finalhyphendemerits=0
[2nd May]}

\index{Burstow, Horne and Outwood , Tandridge@Burstow, Horne \& Outwood, \emph{Tandridge}}

Resignation of Michael Keenan (C).

\noindent
\begin{tabular*}{\columnwidth}{@{\extracolsep{\fill}} p{0.545\columnwidth} >{\itshape}l r @{\extracolsep{\fill}}}
Christopher Byrne & C & 699\\
Graham Bailey & UKIP & 548\\
Judy Wilkinson & LD & 188\\
\end{tabular*}

\council{Waverley}

\subsubsection*{Farnham Wrecclesham and Rowledge \hspace*{\fill}\nolinebreak[1]%
\enspace\hspace*{\fill}
\finalhyphendemerits=0
[2nd May]}

\index{Farnham Wrecclesham and Rowledge , Waverley@\sloppyword{Farnham Wrecclesham \& Rowledge, \emph{Waverley}}}

Resignation of Nerissa Warner-O'Neill (C).

\noindent
\begin{tabular*}{\columnwidth}{@{\extracolsep{\fill}} p{0.545\columnwidth} >{\itshape}l r @{\extracolsep{\fill}}}
Wyatt Ramsdale & C & 545\\
David Beaman & Ind & 294\\
Richard Dancy & UKIP & 252\\
Andrew Jones & Lab & 91\\
\end{tabular*}

\council{Woking}

\subsubsection*{Maybury and Sheerwater \hspace*{\fill}\nolinebreak[1]%
\enspace\hspace*{\fill}
\finalhyphendemerits=0
[Tuesday 17th September; C gain from LD]}

\index{Maybury and Sheerwater , Woking@Maybury \& Sheerwater, \emph{Woking}}

Void election (registration fraud) of Mohammed Bashir (LD).

\noindent
\begin{tabular*}{\columnwidth}{@{\extracolsep{\fill}} p{0.545\columnwidth} >{\itshape}l r @{\extracolsep{\fill}}}
Rashid Mohammed & C & 1057\\
Stephen Tudhope & Lab & 833\\
Neil Willetts & UKIP & 255\\
Norman Johns & LD & 252\\
\end{tabular*}

\section{Warwickshire}

\subsection*{County Council}\index{Warwickshire}

At the May 2013 ordinary election there were unfilled vacancies in Bedworth West and Leamington Brunswick divisions due to the disqualifications (non-attendance) of Frank McCarney and Penny Bould (both Lab) respectively.

\index{Bedworth West , Warwickshire@Bedworth W., \emph{Warks.}}
\index{Leamington Brunswick , Warwickshire@Leamington Brunswick, \emph{Warks.}}

\subsubsection*{Bedworth West \hspace*{\fill}\nolinebreak[1]%
\enspace\hspace*{\fill}
\finalhyphendemerits=0
[12th December]}

\index{Bedworth West , Warwickshire@Bedworth W., \emph{Warks.}}

Death of Keith Richardson (Lab).

\noindent
\begin{tabular*}{\columnwidth}{@{\extracolsep{\fill}} p{0.545\columnwidth} >{\itshape}l r @{\extracolsep{\fill}}}
Brian Hawkes & Lab & 904\\
Janet Batterbee & C & 353\\
Andrew Hutchings & UKIP & 142\\
Natara Hunter & TUSC & 46\\
\end{tabular*}

\council{Nuneaton and Bedworth}

\subsubsection*{Arbury \hspace*{\fill}\nolinebreak[1]%
\enspace\hspace*{\fill}
\finalhyphendemerits=0
[5th December; C gain from Lab]}

\index{Arbury , Nuneaton and Bedworth@Arbury, \emph{Nuneaton \& Bedworth}}

Disqualification (non-attendance) of Kevin Young (Lab).

\noindent
\begin{tabular*}{\columnwidth}{@{\extracolsep{\fill}} p{0.545\columnwidth} >{\itshape}l r @{\extracolsep{\fill}}}
Jeff Morgan & C & 395\\
Tricia Elliott & Lab & 369\\
Trevor Beard & UKIP & 109\\
Mike Wright & Grn & 56\\
Alwyn Deacon & BNP & 35\\
Aidan O'Toole & TUSC & 8\\
Stephen Paxton & EDP & 6\\
\end{tabular*}

\council{Rugby}

\subsubsection*{Bilton \hspace*{\fill}\nolinebreak[1]%
\enspace\hspace*{\fill}
\finalhyphendemerits=0
[2nd May]}

\index{Bilton , Rugby@Bilton, \emph{Rugby}}

Resignation of David Wright (C).

\noindent
\begin{tabular*}{\columnwidth}{@{\extracolsep{\fill}} p{0.545\columnwidth} >{\itshape}l r @{\extracolsep{\fill}}}
Martin Walton & C & 851\\
Roy Harvey & UKIP & 440\\
Owen Richards & Lab & 345\\
Bill Lewis & LD & 308\\
Peter Reynolds & Grn & 57\\
Steve Roberts & TUSC & 37\\
\end{tabular*}

\subsubsection*{Hillmorton \hspace*{\fill}\nolinebreak[1]%
\enspace\hspace*{\fill}
\finalhyphendemerits=0
[21st November]}

\index{Hillmorton , Rugby@Hillmorton, \emph{Rugby}}

Resignation of Bill Sewell (C).

\noindent
\begin{tabular*}{\columnwidth}{@{\extracolsep{\fill}} p{0.545\columnwidth} >{\itshape}l r @{\extracolsep{\fill}}}
Jim Buckley & C & 400\\
Barbara Brown & Lab & 339\\
Roy Harvey & UKIP & 231\\
Tim Douglas & LD & 221\\
Peter Burrows & Grn & 21\\
\end{tabular*}

\section{West Sussex}

\subsection*{County Council}\index{West Sussex}

\subsubsection*{Storrington \hspace*{\fill}\nolinebreak[1]%
\enspace\hspace*{\fill}
\finalhyphendemerits=0
[26th September]}

\index{Storrington , West Sussex@Storrington, \emph{W. Sussex}}

Death of Frank Wilkinson (C).

\noindent
\begin{tabular*}{\columnwidth}{@{\extracolsep{\fill}} p{0.545\columnwidth} >{\itshape}l r @{\extracolsep{\fill}}}
Philip Circus & C & 1037\\
John Wallace & UKIP & 729\\
Nick Hopkinson & LD & 364\\
James Doyle & Grn & 131\\
\end{tabular*}

\subsubsection*{Warnham and Rusper \hspace*{\fill}\nolinebreak[1]%
\enspace\hspace*{\fill}
\finalhyphendemerits=0
[24th October]}

\index{Warnham and Rusper , West Sussex@Warnham \& Rusper, \emph{W. Sussex}}

Death of Mick Hodgson (C).

\noindent
\begin{tabular*}{\columnwidth}{@{\extracolsep{\fill}} p{0.545\columnwidth} >{\itshape}l r @{\extracolsep{\fill}}}
Liz Kitchen & C & 868\\
Geoff Stevens & UKIP & 335\\
Darrin Green & Grn & 119\\
Tony Millson & LD & 103\\
Carol Hayton & Lab & 65\\
\end{tabular*}

\subsubsection*{Haywards Heath East \hspace*{\fill}\nolinebreak[1]%
\enspace\hspace*{\fill}
\finalhyphendemerits=0
[19th December]}

\index{Haywards Heath East , West Sussex@Haywards Heath E., \emph{W. Sussex}}

Death of John de Mierre (C).

\noindent
\begin{tabular*}{\columnwidth}{@{\extracolsep{\fill}} p{0.545\columnwidth} >{\itshape}l r @{\extracolsep{\fill}}}
Stephen Hillier & C & 649\\
Charles Burrell & UKIP & 576\\
Richard Goddard & Lab & 346\\
Anne Hall & LD & 201\\
Paul Brown & Grn & 55\\
\end{tabular*}

\council{Adur}

\subsubsection*{Southlands \hspace*{\fill}\nolinebreak[1]%
\enspace\hspace*{\fill}
\finalhyphendemerits=0
[2nd May; UKIP gain from C]}

\index{Southlands , Adur@Southlands, \emph{Adur}}

Resignation of Paul Graysmark (C) to seek re-election.

\noindent
\begin{tabular*}{\columnwidth}{@{\extracolsep{\fill}} p{0.545\columnwidth} >{\itshape}l r @{\extracolsep{\fill}}}
Paul Graysmark & UKIP & 354\\
Andy Bray & Lab & 254\\
Vicky Parkin & C & 228\\
Cyril Cannings & LD & 51\\
\end{tabular*}

\council{Arun}

\subsubsection*{Aldwick East \hspace*{\fill}\nolinebreak[1]%
\enspace\hspace*{\fill}
\finalhyphendemerits=0
[14th March; LD gain from C]}

\index{Aldwick East , Arun@Aldwick E., \emph{Arun}}

Death of Robin Brown (C).

\noindent
\begin{tabular*}{\columnwidth}{@{\extracolsep{\fill}} p{0.545\columnwidth} >{\itshape}l r @{\extracolsep{\fill}}}
Paul Wells & LD & 383\\
Bill Smee & C & 357\\
Janet Taylor & UKIP & 339\\
Richard Dawson & Lab & 61\\
\end{tabular*}

\subsubsection*{Angmering \hspace*{\fill}\nolinebreak[1]%
\enspace\hspace*{\fill}
\finalhyphendemerits=0
[18th April]}

\index{Angmering , Arun@Angmering, \emph{Arun}}

Death of Julie Hazlehurst (C).

\noindent
\begin{tabular*}{\columnwidth}{@{\extracolsep{\fill}} p{0.545\columnwidth} >{\itshape}l r @{\extracolsep{\fill}}}
Andy Cooper & C & 878\\
Carly Godwin & Lab & 268\\
Jamie Bennett & LD & 228\\
\end{tabular*}

\council{Chichester}

\subsubsection*{Westbourne \hspace*{\fill}\nolinebreak[1]%
\enspace\hspace*{\fill}
\finalhyphendemerits=0
[17th October]}

\index{Westbourne , Chichester@Westbourne, \emph{Chichester}}

Resignation of Maureen Elliott (C).

\noindent
\begin{tabular*}{\columnwidth}{@{\extracolsep{\fill}} p{0.545\columnwidth} >{\itshape}l r @{\extracolsep{\fill}}}
Mark Dunn & C & 184\\
Alicia Denne & UKIP & 106\\
Thomas French & Grn & 85\\
Philip MacDougall & LD & 68\\
Andrew Emerson & Patria & 3\\
\end{tabular*}

\council{Horsham}

\subsubsection*{Chantry \hspace*{\fill}\nolinebreak[1]%
\enspace\hspace*{\fill}
\finalhyphendemerits=0
[2nd May]}

\index{Chantry , Horsham@Chantry, \emph{Horsham}}

Resignation of Chris Mason (C).

\noindent
\begin{tabular*}{\columnwidth}{@{\extracolsep{\fill}} p{0.545\columnwidth} >{\itshape}l r @{\extracolsep{\fill}}}
Diana van der Klugt & C & 1181\\
Peter Westrip & UKIP & 1101\\
Rosalyn Deedman & LD & 426\\
\end{tabular*}

\subsubsection*{\sloppyword{Cowfold, Shermanbury and West Grinstead} \hspace*{\fill}\nolinebreak[1]%
\enspace\hspace*{\fill}
\finalhyphendemerits=0
[2nd May]}

\index{Cowfold, Shermanbury and West Grinstead , Horsham@Cowfold, Shermanbury \& West Grinstead, \emph{Horsham}}

Resignation of Andrew Dunlop (C).

\noindent
\begin{tabular*}{\columnwidth}{@{\extracolsep{\fill}} p{0.545\columnwidth} >{\itshape}l r @{\extracolsep{\fill}}}
Robert Clarke & C & 873\\
Andrew Purches & LD & 243\\
\end{tabular*}

\council{Mid Sussex}

\subsubsection*{Cuckfield \hspace*{\fill}\nolinebreak[1]%
\enspace\hspace*{\fill}
\finalhyphendemerits=0
[2nd May]}

\index{Cuckfield , Mid Sussex@Cuckfield, \emph{Mid Sussex}}

Resignation of Katy Bourne (C).

\noindent
\begin{tabular*}{\columnwidth}{@{\extracolsep{\fill}} p{0.545\columnwidth} >{\itshape}l r @{\extracolsep{\fill}}}
Pete Bradbury & C & 770\\
Stephen Blanch & LD & 429\\
Stuart Gilboy & Lab & 134\\
\end{tabular*}

\subsubsection*{Haywards Heath Franklands \hspace*{\fill}\nolinebreak[1]%
\enspace\hspace*{\fill}
\finalhyphendemerits=0
[19th December]}

\index{Haywards Heath Franklands , Mid Sussex@Haywards Heath Franklands, \emph{Mid Sussex}}

Death of John de Mierre (C).

\noindent
\begin{tabular*}{\columnwidth}{@{\extracolsep{\fill}} p{0.545\columnwidth} >{\itshape}l r @{\extracolsep{\fill}}}
Roderick Clarke & C & 414\\
Howard Burrell & UKIP & 269\\
Gregory Mountain & Lab & 103\\
Anne-Marie Lucraft & LD & 91\\
Miranda Diboll & Grn & 31\\
\end{tabular*}

\council{Worthing}

\subsubsection*{Goring \hspace*{\fill}\nolinebreak[1]%
\enspace\hspace*{\fill}
\finalhyphendemerits=0
[2nd May]}

\index{Goring , Worthing@Goring, \emph{Worthing}}

Resignation of Steve Waight (C).

\noindent
\begin{tabular*}{\columnwidth}{@{\extracolsep{\fill}} p{0.545\columnwidth} >{\itshape}l r @{\extracolsep{\fill}}}
Mark Nolan & C & 997\\
Adrian Price & UKIP & 769\\
Janet Haden & Lab & 195\\
David Aherne & Grn & 171\\
Neil Campbell & LD & 149\\
\end{tabular*}

\section{Wiltshire}

\council{Swindon}

\subsubsection*{Haydon Wick \hspace*{\fill}\nolinebreak[1]%
\enspace\hspace*{\fill}
\finalhyphendemerits=0
[8th August]}

\index{Haydon Wick , Swindon@Haydon Wick, \emph{Swindon}}

Death of Rex Barnett (C).

\noindent
\begin{tabular*}{\columnwidth}{@{\extracolsep{\fill}} p{0.545\columnwidth} >{\itshape}l r @{\extracolsep{\fill}}}
Oliver Donachie & C & 1376\\
Maura Clarke & Lab & 887\\
Ed Gerrard & UKIP & 426\\
Sean Davey & LD & 83\\
\end{tabular*}

\section{Worcestershire}

\subsection*{County Council}\index{Worcestershire}

\sloppyword{ICHC = Independent Community and Health Concern}

At the May 2013 ordinary election there was an unfilled vacancy in Croome division due to the death of Bob Bullock (C).

\index{Croome , Worcestershire@Croome, \emph{Worcs.}}

\subsubsection*{Stourport-on-Severn \hspace*{\fill}\nolinebreak[1]%
\enspace\hspace*{\fill}
\finalhyphendemerits=0
[27th June; ICHC gain from UKIP]}

\index{Stourport-on-Severn , Worcestershire@Stourport-on-Severn, \emph{Worcs.}}

Resignation of Eric Kitson (UKIP).

\noindent
\begin{tabular*}{\columnwidth}{@{\extracolsep{\fill}} p{0.545\columnwidth} >{\itshape}l r @{\extracolsep{\fill}}}
John Thomas & ICHC & 1055\\
John Holden & UKIP & 892\\
Chris Rogers & C & 753\\
Carol Warren & Lab & 607\\
Angela Hartwich & Grn & 77\\
Carl Mason & BNP & 39\\
Paul Preston & LD & 30\\
\end{tabular*}

\subsubsection*{St Marys \hspace*{\fill}\nolinebreak[1]%
\enspace\hspace*{\fill}
\finalhyphendemerits=0
[1st August; C gain from UKIP]}

\index{Saint Marys , Worcestershire@St Marys, \emph{Worcs.}}

Death of Tony Baker (UKIP).

\noindent
\begin{tabular*}{\columnwidth}{@{\extracolsep{\fill}} p{0.545\columnwidth} >{\itshape}l r @{\extracolsep{\fill}}}
Nathan Desmond & C & 504\\
Peter Willoughby & UKIP & 442\\
Mumshad Ahmed & Lab & 338\\
Graham Ballinger & ICHC & 321\\
Helen Dyke & Ind & 195\\
\end{tabular*}

\council{Malvern Hills}

\subsubsection*{Chase \hspace*{\fill}\nolinebreak[1]%
\enspace\hspace*{\fill}
\finalhyphendemerits=0
[2nd May]}

\index{Chase , Malvern Hills@Chase, \emph{Malvern Hills}}

Resignation of Steve Brown (C).

\noindent
\begin{tabular*}{\columnwidth}{@{\extracolsep{\fill}} p{0.545\columnwidth} >{\itshape}l r @{\extracolsep{\fill}}}
Melanie Baker & C & 595\\
Robert Tilley & LD & 513\\
Jeanette Sheen & UKIP & 392\\
Jill Smith & Lab & 262\\
\end{tabular*}

\section{Glamorgan}

\council{Bridgend}

\subsubsection*{Bryncoch \hspace*{\fill}\nolinebreak[1]%
\enspace\hspace*{\fill}
\finalhyphendemerits=0
[31st January]}

\index{Bryncoch , Bridgend@Bryncoch, \emph{Bridgend}}

Death of Pat Penpraze (Lab).

\noindent
\begin{tabular*}{\columnwidth}{@{\extracolsep{\fill}} p{0.545\columnwidth} >{\itshape}l r @{\extracolsep{\fill}}}
Janice Lewis & Lab & 222\\
Tim Thomas & PC & 54\\
Shawn Cullen & Ind & 41\\
Anita Davies & LD & 24\\
Michael Quick & Ind & 14\\
\end{tabular*}

\subsubsection*{Llangewydd and Brynhyfryd \hspace*{\fill}\nolinebreak[1]%
\enspace\hspace*{\fill}
\finalhyphendemerits=0
[2nd May]}

\index{Llangewydd and Brynhyfryd , Bridgend@Llangewydd \& Brynhyfryd, \emph{Bridgend}}

Death of Mal Francis (Lab).

\noindent
\begin{tabular*}{\columnwidth}{@{\extracolsep{\fill}} p{0.545\columnwidth} >{\itshape}l r @{\extracolsep{\fill}}}
Charles Smith & Lab & 281\\
Tony Berrow & LD & 143\\
Eric Hughes & Ind & 66\\
Clare Lewis & C & 44\\
Sian Davies & PC & 34\\
\end{tabular*}

\council{Cardiff}

\subsubsection*{Riverside \hspace*{\fill}\nolinebreak[1]%
\enspace\hspace*{\fill}
\finalhyphendemerits=0
[5th December]}

\index{Riverside , Cardiff@Riverside, \emph{Cardiff}}

Resignation of Phil Hawkins (Lab).

\noindent
\begin{tabular*}{\columnwidth}{@{\extracolsep{\fill}} p{0.545\columnwidth} >{\itshape}l r @{\extracolsep{\fill}}}
Darren Williams & Lab & 1120\\
Elizabeth Gould & PC & 773\\
Aled Crow & C & 107\\
Simon Zeigler & UKIP & 97\\
Joel Beer & TUSC & 70\\
Sian Donne & LD & 58\\
\end{tabular*}

\subsubsection*{Splott \hspace*{\fill}\nolinebreak[1]%
\enspace\hspace*{\fill}
\finalhyphendemerits=0
[5th December]}

\index{Splott , Cardiff@Splott, \emph{Cardiff}}

Resignation of Luke Holland (Lab).

\noindent
\begin{tabular*}{\columnwidth}{@{\extracolsep{\fill}} p{0.545\columnwidth} >{\itshape}l r @{\extracolsep{\fill}}}
Edward Stubbs & Lab & 706\\
Jamie Matthews & LD & 604\\
Brian Morris & UKIP & 209\\
Elys John & Ind & 94\\
Daniel Mason & C & 86\\
Katrine Williams & TUSC & 80\\
\end{tabular*}

\council{Neath Port Talbot}

NPTRA = Neath Port Talbot Residents Association

\subsubsection*{Sandfields East \hspace*{\fill}\nolinebreak[1]%
\enspace\hspace*{\fill}
\finalhyphendemerits=0
[17th October]}

\index{Sandfields East , Neath Port Talbot@Sandfields E., \emph{Neath Port Talbot}}

Death of Collin Crowley (Lab).

\noindent
\begin{tabular*}{\columnwidth}{@{\extracolsep{\fill}} p{0.54\columnwidth} >{\itshape}l r @{\extracolsep{\fill}}}
Mike Davies & Lab & 718\\
Barry Kirk & NPTRA & 222\\
Keith Suter & UKIP & 154\\
Daniel Thomas & PC & 69\\
Richard Minshull & C & 40\\
\end{tabular*}

\council{Swansea}

\subsubsection*{Llansamlet \hspace*{\fill}\nolinebreak[1]%
\enspace\hspace*{\fill}
\finalhyphendemerits=0
[4th July]}

\index{Llansamlet , Swansea@Llansamlet, \emph{Swansea}}

Death of Dennis Jones (Lab).

\noindent
\begin{tabular*}{\columnwidth}{@{\extracolsep{\fill}} p{0.545\columnwidth} >{\itshape}l r @{\extracolsep{\fill}}}
Robert Clay & Lab & 1368\\
James Hatton & C & 236\\
Samuel Rees & LD & 113\\
Claire Thomas & NF & 108\\
\end{tabular*}

\section{Gwent}

\council{Caerphilly}

\subsubsection*{Risca East \hspace*{\fill}\nolinebreak[1]%
\enspace\hspace*{\fill}
\finalhyphendemerits=0
[2nd May]}

\index{Risca East , Caerphilly@Risca E., \emph{Caerphilly}}

Death of Stan Jenkins (Lab).

\noindent
\begin{tabular*}{\columnwidth}{@{\extracolsep{\fill}} p{0.545\columnwidth} >{\itshape}l r @{\extracolsep{\fill}}}
Philippa Leonard & Lab & 529\\
Bob Owen & Ind & 299\\
Cliff Edwards & Ind & 209\\
Matthew Farrell & PC & 119\\
Cameron Muir-Jones & C & 36\\
\end{tabular*}

\subsubsection*{Penyrheol \hspace*{\fill}\nolinebreak[1]%
\enspace\hspace*{\fill}
\finalhyphendemerits=0
[1st August]}

\index{Penyrheol , Caerphilly@Penyrheol, \emph{Caerphilly}}

Death of Anne Collins (PC).

\noindent
\begin{tabular*}{\columnwidth}{@{\extracolsep{\fill}} p{0.545\columnwidth} >{\itshape}l r @{\extracolsep{\fill}}}
Steve Skivens & PC & 929\\
Gareth Pratt & Lab & 554\\
Jaime Davies & TUSC & 173\\
Cameron Muir-Jones & C & 135\\
\end{tabular*}

\council{Newport}

\subsubsection*{Pillgwenlly \hspace*{\fill}\nolinebreak[1]%
\enspace\hspace*{\fill}
\finalhyphendemerits=0
[31st October]}

\index{Pillgwenlly , Newport@Pillgwenlly, \emph{Newport}}

Death of Ron Jones (Lab).

\noindent
\begin{tabular*}{\columnwidth}{@{\extracolsep{\fill}} p{0.545\columnwidth} >{\itshape}l r @{\extracolsep{\fill}}}
Omar Ali & Lab & 500\\
Paul Haliday & LD & 233\\
Khalilur Rahman & PC & 167\\
Tony Ismail & C & 155\\
\end{tabular*}

\council{Torfaen}

\subsubsection*{Croesyceiliog North \hspace*{\fill}\nolinebreak[1]%
\enspace\hspace*{\fill}
\finalhyphendemerits=0
[19th December]}

\index{Croesyceiliog North , Torfaen@Croesyceiliog N., \emph{Torfaen}}

Death of Cynthia Beynon (Lab).

\noindent
\begin{tabular*}{\columnwidth}{@{\extracolsep{\fill}} p{0.59\columnwidth} >{\itshape}l r @{\extracolsep{\fill}}}
Nigel Davies & Lab & 227\\
Aneurin Preece & UKIP & 122\\
Terry Irons & Ind & 79\\
Nick Jones & C & 55\\
Darren Hackley-Green & PC & 13\\
\end{tabular*}

\section{Mid and West Wales}

\council{Pembrokeshire}

\subsubsection*{Burton \hspace*{\fill}\nolinebreak[1]%
\enspace\hspace*{\fill}
\finalhyphendemerits=0
[11th April]}

\index{Burton , Pembrokeshire@Burton, \emph{Pembrokeshire}}

Resignation of David Wildman (Ind).

\noindent
\begin{tabular*}{\columnwidth}{@{\extracolsep{\fill}} p{0.545\columnwidth} >{\itshape}l r @{\extracolsep{\fill}}}
Rob Summons & Ind & 291\\
Robin Wilson & C & 166\\
Robin Howells & Lab & 162\\
Jon Harvey & Ind & 46\\
\end{tabular*}

\section{North Wales}

\council{Conwy}

\subsubsection*{Caerhun \hspace*{\fill}\nolinebreak[1]%
\enspace\hspace*{\fill}
\finalhyphendemerits=0
[11th July; Ind gain from C]}

\index{Caerhun , Conwy@Caerhun, \emph{Conwy}}

Resignation of Paul Roberts (C).

\noindent
\begin{tabular*}{\columnwidth}{@{\extracolsep{\fill}} p{0.545\columnwidth} >{\itshape}l r @{\extracolsep{\fill}}}
Goronwy Edwards & Ind & 321\\
Neil Bradshaw & C & 170\\
Pered Morris & PC & 162\\
Siân Peake-Jones & Lab & 109\\
\end{tabular*}

\council{Flintshire}

\subsubsection*{Connah's Quay Golftyn \hspace*{\fill}\nolinebreak[1]%
\enspace\hspace*{\fill}
\finalhyphendemerits=0
[18th July]}

\index{Connah's Quay Golftyn , Flintshire@Connah's Quay Golftyn, \emph{Flintshire}}

Death of Peter Macfarlane (Lab).

\noindent
\begin{tabular*}{\columnwidth}{@{\extracolsep{\fill}} p{0.6\columnwidth} >{\itshape}l r @{\extracolsep{\fill}}}
Andy Dunbobbin & Lab & 386\\
Eric Owen & Ind & 285\\
David Chamberlain-Jones & C & 34\\
\end{tabular*}

\columnbreak

\section{Border Councils}

\council{Scottish Borders}

BP = Borders Party

\subsubsection*{Leaderdale and Melrose \hspace*{\fill}\nolinebreak[1]%
\enspace\hspace*{\fill}
\finalhyphendemerits=0
[2nd May]}

\index{Leaderdale and Melrose , Scottish Borders@Leaderdale \& Melrose, \emph{Scottish Borders}}

Resignation of Nicholas Watson (BP).

\noindent
\begin{tabular*}{\columnwidth}{@{\extracolsep{\fill}} p{0.545\columnwidth} >{\itshape}l r @{\extracolsep{\fill}}}
\emph{First preferences}\\
Rachael Hamilton & C & 956\\
Iain Gillespie & BP & 814\\
John Paton Day & LD & 744\\
Harry Cummings & SNP & 613\\
Robin Tatler & Lab & 235\\
Sherry Fowler & UKIP & 105\\
\end{tabular*}

\emph{Tatler and Fowler eliminated}: Hamilton 982 Gillespie 900 Paton Day 816 Cummings 666

\emph{Cummings eliminated}: Gillespie 1118 Hamilton 1083 Paton Day 983

\noindent
\begin{tabular*}{\columnwidth}{@{\extracolsep{\fill}} p{0.545\columnwidth} >{\itshape}l r @{\extracolsep{\fill}}}
\emph{Paton Day eliminated}\\
Iain Gillespie & BP & 1444\\
Rachael Hamilton & C & 1283\\
\end{tabular*}

\subsubsection*{Tweeddale West \hspace*{\fill}\nolinebreak[1]%
\enspace\hspace*{\fill}
\finalhyphendemerits=0
[10th October]}

\index{Tweeddale West , Scottish Borders@Tweeddale W., \emph{Scottish Borders}}

Resignation of Nathaniel Buckingham (C).

\noindent
\begin{tabular*}{\columnwidth}{@{\extracolsep{\fill}} p{0.545\columnwidth} >{\itshape}l r @{\extracolsep{\fill}}}
\emph{First preferences}\\
Keith Cockburn & C & 1155\\
Nancy Norman & LD & 677\\
Morag Kerr & SNP & 359\\
David Pye & BP & 228\\
Veronica McTernan & Lab & 203\\
David Cox & Ind & 43\\
Mars Goodman & UKIP & 43\\
\end{tabular*}

\emph{Cox and Goodman eliminated}: Cockburn 1176 Norman 690 Kerr 373 Pye 245 McTernan 207

\emph{McTernan eliminated}: Cockburn 1193 Norman 759 Kerr 409 Pye 272

\emph{Pye eliminated}: Cockburn 1255 Norman 858 Kerr 444

\noindent
\begin{tabular*}{\columnwidth}{@{\extracolsep{\fill}} p{0.545\columnwidth} >{\itshape}l r @{\extracolsep{\fill}}}
\emph{Kerr eliminated}\\
Keith Cockburn & C & 1288\\
Nancy Norman & LD & 1034\\
\end{tabular*}

\columnbreak

\section{Clyde Councils}

\council{Glasgow}

Britnca = Britannica

NoBedTax = No Bedroom Tax

SDA = Scottish Democratic Alliance

Sol = Solidarity

\subsubsection*{Govan \hspace*{\fill}\nolinebreak[1]%
\enspace\hspace*{\fill}
\finalhyphendemerits=0
[10th October; Lab gain from SNP]}

\index{Govan , Glasgow@Govan, \emph{Glasgow}}

Death of Alison Hunter (SNP).

\noindent
\begin{tabular*}{\columnwidth}{@{\extracolsep{\fill}} p{0.455\columnwidth} >{\itshape}l r @{\extracolsep{\fill}}}
\emph{First preferences}\\
John Kane & Lab & 2055\\
Helen Walker & SNP & 1424\\
John Flanagan & NoBedTax & 446\\
Richard Sullivan & C & 215\\
Janice MacKay & UKIP & 113\\
Moira Crawford & Grn & 112\\
George Laird & Ind & 103\\
Ewan Hoyle & LD & 73\\
John Cormack & Chr & 60\\
\sloppyword{Thomas Rannachan} & Ind & 52\\
Ryan Boyle & Comm & 35\\
Joyce Drummond & Sol & 28\\
Charles Baillie & Britnca & 19\\
James Trolland & SDA & 1\\
\end{tabular*}

\emph{Baillie and Trolland eliminated}: Kane 2056 Walker 1426 Flanagan 449 Sullivan 216 MacKay 116 Crawford 113 Laird 103 Hoyle 74 Cormack 61 Rannachan 52 Boyle 35 Drummond 29

\sloppyword{\emph{Drummond eliminated}: Kane 2058 Walker 1431 Flanagan 455 Sullivan 216 MacKay 116 Crawford 116 Laird 105 Hoyle 75 Cormack 62 Rannachan 52 Boyle 37}

\emph{Boyle eliminated}: Kane 2062 Walker 1435 Flanagan 466 Sullivan 216 Crawford 125 MacKay 116 Laird 105 Hoyle 76 Cormack 65 Rannachan 53

\sloppyword{\emph{Rannachan eliminated}: Kane 2075 Walker 1449 Flanagan 479 Sullivan 218 Crawford 129 MacKay 116 Laird 106 Hoyle 76 Cormack 65}

\emph{Cormack eliminated}: Kane 2082 Walker 1456 Flanagan 483 Sullivan 229 Crawford 131 MacKay 119 Laird 114 Hoyle 79

\emph{Hoyle eliminated}: Kane 2108 Walker 1466 Flanagan 487 Sullivan 237 Crawford 142 MacKay 120 Laird 118

\emph{Laird eliminated}: Kane 2137 Walker 1485 Flanagan 503 Sullivan 243 Crawford 150 MacKay 126

\emph{MacKay eliminated}: Kane 2149 Walker 1497 Flanagan 517 Sullivan 269 Crawford 158

\emph{Crawford eliminated}: Kane 2180 Walker 1541 Flanagan 541 Sullivan 275

\noindent
\begin{tabular*}{\columnwidth}{@{\extracolsep{\fill}} p{0.455\columnwidth} >{\itshape}l r @{\extracolsep{\fill}}}
\emph{Sullivan eliminated}\\
John Kane & Lab & 2216\\
Helen Walker & SNP & 1575\\
John Flanagan & NoBedTax & 564\\
\end{tabular*}



\subsubsection*{Shettleston \hspace*{\fill}\nolinebreak[1]%
\enspace\hspace*{\fill}
\finalhyphendemerits=0
[5th December]}

\index{Shettleston , Glasgow@Shettleston, \emph{Glasgow}}

Death of George Ryan (Lab).

\noindent
\begin{tabular*}{\columnwidth}{@{\extracolsep{\fill}} p{0.459\columnwidth} >{\itshape}l r @{\extracolsep{\fill}}}
Martin Neill & Lab & 2025\\
Laura Doherty & SNP & 1086\\
Raymond McCrae & C & 224\\
Arthur Thackeray & UKIP & 129\\
Jamie Cocozza & TUSC & 68\\
James Speirs & LD & 53\\
John Flanagan & NoBedTax & 50\\
Alasdair Duke & Grn & 41\\
Tommy Ball & SSP & 35\\
Victor Murphy & Chr & 34\\
Charles Baillie & Britnca & 31\\
James Trolland & SDA & 6\\
\end{tabular*}

\council{North Lanarkshire}

\subsubsection*{Coatbridge West \hspace*{\fill}\nolinebreak[1]%
\enspace\hspace*{\fill}
\finalhyphendemerits=0
[28th February]}

\index{Coatbridge West , North Lanarkshire@Coatbridge W., \emph{N. Lanarks.}}

Death of Tom Maginnis (Lab).

\noindent
\begin{tabular*}{\columnwidth}{@{\extracolsep{\fill}} p{0.545\columnwidth} >{\itshape}l r @{\extracolsep{\fill}}}
Kevin Docherty & Lab & 2145\\
Patrick Rolink & SNP & 452\\
Ashley Baird & C & 71\\
Billy Mitchell & UKIP & 34\\
John Love & LD & 19\\
\end{tabular*}

\council{South Lanarkshire}

\subsubsection*{Rutherglen South \hspace*{\fill}\nolinebreak[1]%
\enspace\hspace*{\fill}
\finalhyphendemerits=0
[14th February; Lab gain from SNP]}

\index{Rutherglen South , South Lanarkshire@Rutherglen S., \emph{S. Lanarks.}}

Death of Anne Higgins (SNP).

\noindent
\begin{tabular*}{\columnwidth}{@{\extracolsep{\fill}} p{0.545\columnwidth} >{\itshape}l r @{\extracolsep{\fill}}}
\emph{First preferences}\\
Gerard Killen & Lab & 1352\\
David Baillie & LD & 999\\
Margaret Ferrier & SNP & 712\\
Aric Gilinsky & C & 128\\
Donald MacKay & UKIP & 111\\
Susan Martin & Grn & 59\\
Craig Smith & Ind & 31\\
\end{tabular*}

\emph{Gilinsky, MacKay, Martin and Smith eliminated}: Killen 1396 Baillie 1104 Ferrier 755

\noindent
\begin{tabular*}{\columnwidth}{@{\extracolsep{\fill}} p{0.545\columnwidth} >{\itshape}l r @{\extracolsep{\fill}}}
\emph{Ferrier eliminated}\\
Gerard Killen & Lab & 1616\\
David Baillie & LD & 1278\\
\end{tabular*}

\subsubsection*{Hamilton South \hspace*{\fill}\nolinebreak[1]%
\enspace\hspace*{\fill}
\finalhyphendemerits=0
[24th October; Lab gain from SNP]}

\index{Hamilton South , South Lanarkshire@Hamilton S., \emph{S. Lanarks.}}

Death of Bobby Lawson (SNP).

\noindent
\begin{tabular*}{\columnwidth}{@{\extracolsep{\fill}} p{0.545\columnwidth} >{\itshape}l r @{\extracolsep{\fill}}}
Stuart Gallacher & Lab & 1781\\
Josh Wilson & SNP & 1120\\
Lynne Nailon & C & 322\\
Craig Smith & Chr & 133\\
Josh Richardson & UKIP & 86\\
\end{tabular*}

\columnbreak

\section{Forth Councils}

\council{Edinburgh}

Pirate = Pirate Party

\subsubsection*{Liberton\slash Gilmerton \hspace*{\fill}\nolinebreak[1]%
\enspace\hspace*{\fill}
\finalhyphendemerits=0
[20th June; Lab gain from SNP]}

\index{Liberton/Gilmerton , Edinburgh@Liberton\slash Gilmerton, \emph{Edinburgh}}

Death of Tom Buchanan (SNP).

\noindent
\begin{tabular*}{\columnwidth}{@{\extracolsep{\fill}} p{0.545\columnwidth} >{\itshape}l r @{\extracolsep{\fill}}}
\emph{First preferences}\\
Keith Robson & Lab & 2892\\
Derek Howie & SNP & 2249\\
Stephanie Murray & C & 823\\
John Knox & LD & 605\\
Alys Mumford & Grn & 412\\
Jonathan Stanley & UKIP & 235\\
John Scott & Ind & 64\\
Phil Hunt & Pirate & 47\\
\end{tabular*}

\sloppyword{\emph{Stanley, Scott and Hunt eliminated}: Robson 2941 Howie 2287 Murray 903 Knox 625 Mumford 466}

\emph{Mumford eliminated}: Robson 3070 Howie 2403 Murray 934 Knox 708

\emph{Knox eliminated}: Robson 3255 Howie 2523 Murray 1098

\noindent
\begin{tabular*}{\columnwidth}{@{\extracolsep{\fill}} p{0.545\columnwidth} >{\itshape}l r @{\extracolsep{\fill}}}
\emph{Murray eliminated}\\
Keith Robson & Lab & 3448\\
Derek Howie & SNP & 2633\\
\end{tabular*}

\council{Fife}

\subsubsection*{\sloppyword{Glenrothes North, Leslie and Markinch} \hspace*{\fill}\nolinebreak[1]%
\enspace\hspace*{\fill}
\finalhyphendemerits=0
[20th June]}

\index{Glenrothes North, Leslie and Markinch , Fife@Glenrothes N., Leslie \& Markinch, \emph{Fife}}

Death of Bill Kay (Lab).

\noindent
\begin{tabular*}{\columnwidth}{@{\extracolsep{\fill}} p{0.545\columnwidth} >{\itshape}l r @{\extracolsep{\fill}}}
\emph{First preferences}\\
John Wincott & Lab & 1896\\
Keith Grieve & SNP & 1711\\
Allan Smith & C & 272\\
Peter Taggerty & UKIP & 176\\
Harry Wills & LD & 83\\
\end{tabular*}

\emph{Taggerty and Wills eliminated}: Wincott 1950 Grieve 1761 Smith 335

\noindent
\begin{tabular*}{\columnwidth}{@{\extracolsep{\fill}} p{0.545\columnwidth} >{\itshape}l r @{\extracolsep{\fill}}}
\emph{Smith eliminated}\\
John Wincott & Lab & 2079\\
Keith Grieve & SNP & 1814\\
\end{tabular*}

\subsubsection*{Dunfermline South \hspace*{\fill}\nolinebreak[1]%
\enspace\hspace*{\fill}
\finalhyphendemerits=0
[24th October]}

\index{Dunfermline South , Fife@Dunfermline S., \emph{Fife}}

Death of Mike Rumney (Lab).

\noindent
\begin{tabular*}{\columnwidth}{@{\extracolsep{\fill}} p{0.545\columnwidth} >{\itshape}l r @{\extracolsep{\fill}}}
\emph{First preferences}\\
Billy Pollock & Lab & 2552\\
Helen Cannon-Todd & SNP & 2057\\
Robin Munro & LD & 1009\\
David Ross & C & 450\\
Angela Dixon & Grn & 183\\
Judith Rideout & UKIP & 183\\
\end{tabular*}

\emph{Ross, Dickson and Rideout eliminated}: Pollock 2697 Cannon-Todd 2142 Munro 1257

\noindent
\begin{tabular*}{\columnwidth}{@{\extracolsep{\fill}} p{0.545\columnwidth} >{\itshape}l r @{\extracolsep{\fill}}}
\emph{Munro eliminated}\\
Billy Pollock & Lab & 3170\\
Helen Cannon-Todd & SNP & 2358\\
\end{tabular*}

\section{Highland Councils}

\council{Highland}

\subsubsection*{Landward Caithness \hspace*{\fill}\nolinebreak[1]%
\enspace\hspace*{\fill}
\finalhyphendemerits=0
[2nd May]}

\index{Landward Caithness , Highland@Landward Caithness, \emph{Highland}}

Resignation of Robert Coghill (Ind).

\noindent
\begin{tabular*}{\columnwidth}{@{\extracolsep{\fill}} p{0.545\columnwidth} >{\itshape}l r @{\extracolsep{\fill}}}
Gillian Coghill & Ind & 1317\\
Hanna Miedema & SNP & 525\\
David Baron & Lab & 417\\
Barbara Watson & C & 203\\
\end{tabular*}

\subsubsection*{Landward Caithness \hspace*{\fill}\nolinebreak[1]%
\enspace\hspace*{\fill}
\finalhyphendemerits=0
[28th November; Ind gain from SNP]}

\index{Landward Caithness , Highland@Landward Caithness, \emph{Highland}}

Resignation of Alex MacLeod (SNP).

\noindent
\begin{tabular*}{\columnwidth}{@{\extracolsep{\fill}} p{0.545\columnwidth} >{\itshape}l r @{\extracolsep{\fill}}}
\emph{First preferences}\\
Matthew Reiss & Ind & 1150\\
Winifred Sutherland & Ind & 593\\
Ed Boyter & SNP & 546\\
Kerensa Carr & C & 171\\
Tina Irving & Ind & 128\\
\end{tabular*}

\noindent
\begin{tabular*}{\columnwidth}{@{\extracolsep{\fill}} p{0.545\columnwidth} >{\itshape}l r @{\extracolsep{\fill}}}
\multicolumn{3}{@{\extracolsep{\fill}}l}{\emph{Carr and Irving eliminated}}\\
Matthew Reiss & Ind & 1273\\
Winifred Sutherland & Ind & 667\\
Ed Boyter & SNP & 568\\
\end{tabular*}

\subsubsection*{Black Isle \hspace*{\fill}\nolinebreak[1]%
\enspace\hspace*{\fill}
\finalhyphendemerits=0
[19th December]}

\index{Black Isle , Highland@Black Isle, \emph{Highland}}

Death of Billy Barclay (Ind).

\noindent
\begin{tabular*}{\columnwidth}{@{\extracolsep{\fill}} p{0.545\columnwidth} >{\itshape}l r @{\extracolsep{\fill}}}
\emph{First preferences}\\
Jennifer Barclay & Ind & 1003\\
Jackie Hendry & SNP & 439\\
Bill Fraser & Ind & 382\\
George Normington & LD & 334\\
Gwyn Phillips & Ind & 275\\
Myra Carus & Grn & 269\\
Sean Finlayson & Lab & 184\\
Douglas MacLean & C & 175\\
\end{tabular*}

\sloppyword{\emph{MacLean eliminated}: Barclay 1031 Hendry 439 Fraser 413 Normington 373 Phillips 289 Carus 280 Finlayson 194}

\sloppyword{\emph{Finlayson eliminated}: Barclay 1061 Hendry 453 Fraser 431 Normington 408 Carus 319 Phillips 304}

\emph{Phillips eliminated}: Barclay 1157 Fraser 485 Hendry 470 Normington 434 Carus 362

\emph{Carus eliminated}: Barclay 1210 Fraser 556 Hendry 530 Normington 489

\noindent
\begin{tabular*}{\columnwidth}{@{\extracolsep{\fill}} p{0.545\columnwidth} >{\itshape}l r @{\extracolsep{\fill}}}
\emph{Normington eliminated}\\
Jennifer Barclay & Ind & 1343\\
Bill Fraser & Ind & 663\\
Jackie Hendry & SNP & 594\\
\end{tabular*}

\council{Moray}

\subsubsection*{Heldon and Laich \hspace*{\fill}\nolinebreak[1]%
\enspace\hspace*{\fill}
\finalhyphendemerits=0
[7th March; Ind gain from SNP]}

\index{Heldon and Laich , Moray@Heldon \& Laich, \emph{Moray}}

Resignation of Carolle Ralph (SNP).

\noindent
\begin{tabular*}{\columnwidth}{@{\extracolsep{\fill}} p{0.6\columnwidth} >{\itshape}l r @{\extracolsep{\fill}}}
\emph{First preferences}\\
John Cowe & Ind & 972\\
Stuart Crowther & SNP & 833\\
Pete Bloomfield & C & 473\\
Nick Traynor & Ind & 418\\
James MacKessack-Leitch & Grn & 228\\
Jeff Hamilton & Ind & 175\\
\end{tabular*}

\emph{MacKessack-Leitch and Hamilton eliminated}: Cowe 1058 Crowther 891 Bloomfield 516 Traynor 496

\emph{Traynor eliminated}: Cowe 1273 Crowther 960 Bloomfield 558

\noindent
\begin{tabular*}{\columnwidth}{@{\extracolsep{\fill}} p{0.545\columnwidth} >{\itshape}l r @{\extracolsep{\fill}}}
\emph{Bloomfield eliminated}\\
John Cowe & Ind & 1507\\
Stuart Crowther & SNP & 1005\\
\end{tabular*}

\end{resultsiii}

\part{2014}
\renewcommand\resultsyear{2014}

\chapter{Parliamentary by-elections}

There were five parliamentary by-elections in 2014.

Brit1st = Britain First

CGood = Common Good

Elvis = Bus-Pass Elvis

PatSoc = Patriotic Socialist

PBP = People Before Profit

\vfill

\section*{Wythenshawe and Sale East \hspace*{\fill}\nolinebreak[1]%
\enspace\hspace*{\fill}
\finalhyphendemerits=0
[13th February]}

\index{Wythenshawe and Sale East , House of Commons@Wythenshawe \& Sale E., \emph{House of Commons}}

Death of Paul Goggins (Lab).

\noindent
\begin{tabular*}{\columnwidth}{@{\extracolsep{\fill}} p{0.545\columnwidth} >{\itshape}l r @{\extracolsep{\fill}}}
Mike Kane & Lab & 13261\\
John Bickley & UKIP & 4301\\
Rev Daniel Critchlow & C & 3479\\
Mary di Mauro & LD & 1176\\
Nigel Woodcock & Grn & 748\\
Eddy O'Sullivan & BNP & 708\\
Captain Chaplington-Smythe & Loony & 288\\
\end{tabular*}

\vfill

\section*{Newark \hspace*{\fill}\nolinebreak[1]%
\enspace\hspace*{\fill}
\finalhyphendemerits=0
[5th June]}

\index{Newark , House of Commons@Newark, \emph{House of Commons}} 

Resignation of Patrick Mercer (C).

\noindent
\begin{tabular*}{\columnwidth}{@{\extracolsep{\fill}} p{0.545\columnwidth} >{\itshape}l r @{\extracolsep{\fill}}}
Robert Jenrick & C & 17431\\
Roger Helmer & UKIP & 10028\\
Michael Payne & Lab & 6842\\
Paul Baggaley & Ind & 1891\\
David Kirwan & Grn & 1057\\
David Watts & LD & 1004\\
Nick the Flying Brick & Loony & 168\\
Andy Hayes & Ind & 117\\
David Bishop & Elvis & 87\\
Dick Rodgers & CGood & 64\\
Lee Woods & PatSoc & 18\\
\end{tabular*}

\vfill\eject

\section*{Clacton \hspace*{\fill}\nolinebreak[1]%
\enspace\hspace*{\fill}
\finalhyphendemerits=0
[9th October; UKIP gain from C]}

\index{Clacton , House of Commons@Clacton, \emph{House of Commons}} 

Resignation of Douglas Carswell (UKIP elected as C) to seek re-election.

\noindent
\begin{tabular*}{\columnwidth}{@{\extracolsep{\fill}} p{0.545\columnwidth} >{\itshape}l r @{\extracolsep{\fill}}}
Douglas Carswell & UKIP & 21113\\
Giles Watling & C & 8709\\
Tim Young & Lab & 3957\\
Chris Southall & Grn & 688\\
Andy Graham & LD & 483\\
Bruce Sizer & Ind & 205\\
Howling Laud Hope & Loony & 127\\
Charlotte Rose & Ind & 56\\
\end{tabular*}

\section*{Heywood and Middleton \hspace*{\fill}\nolinebreak[1]%
\enspace\hspace*{\fill}
\finalhyphendemerits=0
[9th October]}

\index{Heywood and Middleton , House of Commons@Heywood \& Middleton, \emph{House of Commons}} 

Death of Jim Dobbin (Lab).

\noindent
\begin{tabular*}{\columnwidth}{@{\extracolsep{\fill}} p{0.545\columnwidth} >{\itshape}l r @{\extracolsep{\fill}}}
Liz McInnes & Lab & 11633\\
John Bickley & UKIP & 11016\\
Iain Gartside & C & 3496\\
Anthony Smith & LD & 1457\\
Abi Jackson & Grn & 870\\
\end{tabular*}

\section*{Rochester and Strood \hspace*{\fill}\nolinebreak[1]%
\enspace\hspace*{\fill}
\finalhyphendemerits=0
[20th November; UKIP gain from C]}

\index{Rochester and Strood , Houes of Commons@Rochester \& Strood, \emph{House of Commons}} 

Resignation of Mark Reckless (UKIP elected as C) to seek re-election.

\noindent
\begin{tabular*}{\columnwidth}{@{\extracolsep{\fill}} p{0.545\columnwidth} >{\itshape}l r @{\extracolsep{\fill}}}
Mark Reckless & UKIP & 16867\\
Kelly Tolhurst & C & 13947\\
Naushabah Khan & Lab & 6713\\
Clive Gregory & Grn & 1692\\
Hairy Knorm Davidson & Loony & 151\\
Stephen Goldsbrough & Ind & 69\\
Nick Long & PBP & 69\\
Jayda Fransen & Brit1st & 56\\
Mike Baker & Ind & 54\\
Charlotte Rose & Ind & 43\\
Dave Osborn & PatSoc & 33\\
Christopher Challis & Ind & 22\\
\end{tabular*}

\chapter{By-elections to devolved assemblies, the European Parliament, and police and crime commissionerships}

\section{Greater London Authority}

There were no by-elections in 2014 to the Greater London Authority.

\section{National Assembly for Wales}

There were no by-elections in 2014 to the National Assembly for Wales.

\section{Scottish Parliament}

There was one by-election in 2014 to the Scottish Parliament.

\subsection*{Cowdenbeath \hspace*{\fill}\nolinebreak[1]%
\enspace\hspace*{\fill}
\finalhyphendemerits=0
[23rd January]}

\index{Cowdenbeath , Scottish Parliament@Cowdenbeath, \emph{Scot. Parl.}}

Death of Helen Eadie (Lab).

SDA = Scottish Democratic Alliance; VFR = Victims Final Right

\noindent
\begin{tabular*}{\columnwidth}{@{\extracolsep{\fill}} p{0.545\columnwidth} >{\itshape}l r @{\extracolsep{\fill}}}
Alex Rowley & Lab & 11192\\
Natalie McGarry & SNP & 5704\\
Dave Dempsey & C & 1893\\
Denise Baykal & UKIP & 610\\
Jade Holden & LD & 425\\
Stuart Graham & VFR & 187\\
James Trolland & SDA & 51\\
\end{tabular*}

\medskip

Margo MacDonald (Ind, Lothian) died on 4th April.  Her seat will remain vacant until the end of the parliamentary term.

\section{Northern Ireland Assembly}

Vacancies in the Northern Ireland Assembly are filled by co-option. The following members were co-opted to the Assembly in 2014:
\begin{itemize}
\item Claire Sugden (Ind) replaced David McClarty following his death on 18th April (East Londonderry).
\end{itemize}

\section{European Parliament}

UK vacancies in the European Parliament are filled by the next available person from the party list at the most recent election. 
%No replacements were made in 2014.
The following replacement was made in 2014:
\begin{itemize}
\item Dan Dalton (C) replaced Philip Bradbourn following his death on 19th December (West Midlands).
\end{itemize}

\section{Police and crime commissioners}

There were two by-elections in 2014 for vacant police and crime commissioner posts.

\subsection*{West Midlands \hspace*{\fill}\nolinebreak[1]%
\enspace\hspace*{\fill}
\finalhyphendemerits=0
[21st August]}

\index{West Midlands Police and Crime Commissioner}

Death of Bob Jones (Lab).

\noindent
\begin{tabular*}{\columnwidth}{@{\extracolsep{\fill}} p{0.545\columnwidth} >{\itshape}l r @{\extracolsep{\fill}}}
David Jamieson & Lab & 102561\\
Les Jones & C & 54091\\
Keith Rowe & UKIP & 32187\\
Ayoub Khan & LD & 12950\\
\end{tabular*}

\subsection*{South Yorkshire \hspace*{\fill}\nolinebreak[1]%
\enspace\hspace*{\fill}
\finalhyphendemerits=0
[30th October]}

\index{South Yorkshire Police and Crime Commissioner}

Resignation of Shaun Wright (Lab).

\noindent
\begin{tabular*}{\columnwidth}{@{\extracolsep{\fill}} p{0.545\columnwidth} >{\itshape}l r @{\extracolsep{\fill}}}
Alan Billings & Lab & 74060\\
Jack Clarkson & UKIP & 46883\\
Ian Walker & C & 18536\\
David Allen & EDP & 8583\\
\end{tabular*}

\chapter{Local by-elections and unfilled vacancies}

\section{North London}

\begin{resultsiii}

\subsection*{City of London Corporation}

\subsubsection*{Castle Baynard \hspace*{\fill}\nolinebreak[1]%
\enspace\hspace*{\fill}
\finalhyphendemerits=0
[Monday 10th February]}

\index{Castle Baynard , City of London@Castle Baynard, \emph{City of London}}

Resignation of Ray Catt (Ind).

\noindent
\begin{tabular*}{\columnwidth}{@{\extracolsep{\fill}} p{0.545\columnwidth} >{\itshape}l r @{\extracolsep{\fill}}}
Emma Edhem & Ind & 161\\
John Petrie & Ind & 40\\
\end{tabular*}

\subsubsection*{Portsoken \hspace*{\fill}\nolinebreak[1]%
\enspace\hspace*{\fill}
\finalhyphendemerits=0
[20th March; Lab gain from Ind]}

\index{Portsoken , City of London@Portsoken, \emph{City of London}}

Resignation of Shadique Gani (Ind).

\noindent
\begin{tabular*}{\columnwidth}{@{\extracolsep{\fill}} p{0.6\columnwidth} >{\itshape}l r @{\extracolsep{\fill}}}
William Campbell-Taylor & Lab & 137\\
Marie Brockington & Ind & 98\\
Evan Millner & Ind & 47\\
Syed Mahmood & Ind & 44\\
Roger Jones & Ind & 26\\
Andr\'e Walker & Ind & 11\\
Muhammad al-Hussaini & Ind & 9\\
\end{tabular*}

\subsubsection*{Cheap \hspace*{\fill}\nolinebreak[1]%
\enspace\hspace*{\fill}
\finalhyphendemerits=0
[Tuesday 25th March]}

\index{Cheap , City of London@Cheap, \emph{City of London}}

Death of Robin Eve (Ind).

\noindent
\begin{tabular*}{\columnwidth}{@{\extracolsep{\fill}} p{0.545\columnwidth} >{\itshape}l r @{\extracolsep{\fill}}}
Nicholas Bensted-Smith & Ind & 66\\
Robert Pitcher & Ind & 49\\
Colin Gregory & Ind & 43\\
John Spanner & Ind & 25\\
Peter Hardwick & Ind & 4\\
\end{tabular*}

\subsection*{Barking and Dagenham}

At the May 2014 ordinary election there was an unfilled vacancy in Whalebone ward due to the death of Tony Perry (Lab).
\index{Whalebone , Barking and Dagenham@Whalebone, \emph{Barking \& Dagenham}}

\columnbreak

\subsection*{Barnet}

\subsubsection*{Colindale (3) \hspace*{\fill}\nolinebreak[1]%
\enspace\hspace*{\fill}
\finalhyphendemerits=0
[26th June]}

\index{Colindale , Barnet@Colindale, \emph{Barnet}}

Ordinary election postponed from 22nd May; death of candidate Jessica Yorke (Grn).

\noindent
\begin{tabular*}{\columnwidth}{@{\extracolsep{\fill}} p{0.545\columnwidth} >{\itshape}l r @{\extracolsep{\fill}}}
Nagus Narenthira & Lab & 2119\\
Gill Sargeant & Lab & 2088\\
Zakia Zubairi & Lab & 2015\\
Nneka Akwaeze & C & 501\\
William Nicholson & C & 466\\
Golnar Bokaei & C & 420\\
John Baskin & UKIP & 347\\
Barry Ryan & UKIP & 309\\
Khalid Khan & UKIP & 268\\
Daniel Estermann & LD & 133\\
Maggie Curati & Grn & 130\\
Andrew Newby & Grn & 114\\
Francesco Marasco & Grn & 108\\
Victor Corney & LD & 90\\
Sabriye Warsame & LD & 87\\
\end{tabular*}

\subsection*{Brent}

At the May 2014 ordinary election there was an unfilled vacancy in Dudden Hill ward due to the resignation of Rev David Clues (LD).
\index{Dudden Hill , Brent@Dudden Hill, \emph{Brent}}

\subsection*{Ealing}

At the May 2014 ordinary election there was an unfilled vacancy in Hobbayne ward due to the resignation of Wendy Langham (Lab).
\index{Hobbayne , Ealing@Hobbayne, \emph{Ealing}}

\subsection*{Hackney}

At the May 2014 ordinary election there were unfilled vacancies in Dalston and Lordship wards due to the resignations of Angus Mulready-Jones and of Daniel Stevens (Lab) respectively.
\index{Dalston , Hackney@Dalston, \emph{Hackney}}
\index{Lordship , Hackney@Lordship, \emph{Hackney}}

\columnbreak

\subsection*{Haringey}

\subsubsection*{Woodside \hspace*{\fill}\nolinebreak[1]%
\enspace\hspace*{\fill}
\finalhyphendemerits=0
[2nd October]}

\index{Woodside , Haringey@Woodside, \emph{Haringey}}

Death of Pat Egan (Lab).

\noindent
\begin{tabular*}{\columnwidth}{@{\extracolsep{\fill}} p{0.545\columnwidth} >{\itshape}l r @{\extracolsep{\fill}}}
Charles Wright & Lab & 1331\\
Dawn Barnes & LD & 482\\
Tom Davidson & Grn & 191\\
Andrew Price & UKIP & 161\\
Scott Green & C & 140\\
Vivek Lehal & TUSC & 35\\
Pauline Gibson & Ind & 23\\
\end{tabular*}

\subsection*{Havering}

At the May 2014 ordinary election there was an unfilled vacancy in Rainham and Wennington ward due to the disqualification (non-attendancw) of Mark Logan (Rainham and Wennington Residents).
\index{Rainham and Wennington , Havering@Rainham \& Wennington, \emph{Havering}}

\subsection*{Hillingdon}

\subsubsection*{Charville \hspace*{\fill}\nolinebreak[1]%
\enspace\hspace*{\fill}
\finalhyphendemerits=0
[27th November]}

\index{Charville , Hillingdon@Charville, \emph{Hillingdon}}

Resignation of David Horne (Lab).

\noindent
\begin{tabular*}{\columnwidth}{@{\extracolsep{\fill}} p{0.545\columnwidth} >{\itshape}l r @{\extracolsep{\fill}}}
John Oswell & Lab & 950\\
Mary O'Connor & C & 929\\
Cliff Dixon & UKIP & 468\\
Wally Kennedy & TUSC & 40\\
Paul McKeown & LD & 37\\
\end{tabular*}

\subsection*{Newham}

\subsubsection*{Beckton \hspace*{\fill}\nolinebreak[1]%
\enspace\hspace*{\fill}
\finalhyphendemerits=0
[11th September]}

\index{Beckton , Newham@Beckton, \emph{Newham}}

Death of Alec Kellaway (Lab).

\noindent
\begin{tabular*}{\columnwidth}{@{\extracolsep{\fill}} p{0.545\columnwidth} >{\itshape}l r @{\extracolsep{\fill}}}
Toni Wilson & Lab & 1006\\
Syed Ahmed & C & 584\\
David Mears & UKIP & 215\\
Jane Lithgow & Grn & 70\\
David Thorpe & LD & 43\\
Kayode Shedowo & CPA & 33\\
Mark Dunne & TUSC & 21\\
\end{tabular*}

\columnbreak

\subsection*{Tower Hamlets}

THF = Tower Hamlets First

\subsubsection*{Blackwall and Cubitt Town (3) \hspace*{\fill}\nolinebreak[1]%
\enspace\hspace*{\fill}
\finalhyphendemerits=0
[3rd July]}

\index{Blackwall and Cubitt Town , Tower Hamlets@Blackwall \& Cubitt Town, \emph{Tower Hamlets}}

Ordinary election postponed from 22nd May; death of candidate Hifzur Rahman (THF).

\noindent
\begin{tabular*}{\columnwidth}{@{\extracolsep{\fill}} p{0.58\columnwidth} >{\itshape}l r @{\extracolsep{\fill}}}
Dave Chesterton & Lab & 956\\
Christopher Chapman & C & 877\\
Candida Ronald & Lab & 875\\
Anisur Rahman & Lab & 872\\
Gloria Thienel & C & 815\\
Geeta Kasanga & C & 762\\
Faruk Khan & THF & 744\\
Kabir Ahmed & THF & 726\\
\sloppyword{Mohammed Aktaruzzaman} & THF & 713\\
Diana Lochner & UKIP & 240\\
Paul Shea & UKIP & 190\\
Anthony Registe & UKIP & 188\\
Katy Guttmann & Grn & 110\\
Mark Lomas & Grn & 98\\
Chris Smith & Grn & 74\\
Elaine Bagshaw & LD & 71\\
Richard Flowers & LD & 68\\
Stephen Clarke & LD & 58\\
Ellen Kenyon Peers & TUSC & 11\\
John Peers & TUSC & 11\\
Mohammed Rahman & Ind & 11\\
\end{tabular*}

\subsection*{Waltham Forest}

At the May 2014 ordinary election there was an unfilled vacancy in Higham Hill ward due to the resignation of Geoff Hammond (Lab).
\index{Higham Hill , Waltham Forest@Higham Hill, \emph{Waltham Forest}}

\section{South London}

\subsection*{Bexley}

At the May 2014 ordinary election there was an unfilled vacancy in Sidcup ward due to the resignation of Jackie Evans (C).
\index{Sidcup , Bexley@Sidcup, \emph{Bexley}}

\subsection*{Kingston upon Thames}

\subsubsection*{Tudor \hspace*{\fill}\nolinebreak[1]%
\enspace\hspace*{\fill}
\finalhyphendemerits=0
[16th October]}

\index{Tudor , Kingston upon Thames@Tudor, \emph{Kingston u. Thames}}

Resignation of Frank Thompson (C).

\noindent
\begin{tabular*}{\columnwidth}{@{\extracolsep{\fill}} p{0.545\columnwidth} >{\itshape}l r @{\extracolsep{\fill}}}
Maria Netley & C & 1062\\
Marilyn Mason & LD & 725\\
Chris Priest & Lab & 314\\
Ben Roberts & UKIP & 269\\
Ryan Coley & Grn & 219\\
\end{tabular*}

\subsubsection*{St James \hspace*{\fill}\nolinebreak[1]%
\enspace\hspace*{\fill}
\finalhyphendemerits=0
[18th December]}

\index{Saint James , Kingston upon Thames@St James, \emph{Kingston upon Thames}}

Death of Howard Jones (C).

\noindent
\begin{tabular*}{\columnwidth}{@{\extracolsep{\fill}} p{0.545\columnwidth} >{\itshape}l r @{\extracolsep{\fill}}}
Jack Cheetham & C & 1123\\
Annette Wookey & LD & 865\\
Stephen Dunkling & Lab & 355\\
Ben Roberts & UKIP & 206\\
Alex Nelson & Grn & 71\\
\end{tabular*}

\subsection*{Lambeth}

At the May 2014 ordinary election there were unfilled vacancies in Streatham South and Vassall wards due to the death of Mark Bennett (Lab) and the resignation of Steve Bradley (LD) respectively.
\index{Streatham South , Lambeth@Streatham S., \emph{Lambeth}}
\index{Vassall , Lambeth@Vassall, \emph{Lambeth}}

\subsubsection*{Knight's Hill \hspace*{\fill}\nolinebreak[1]%
\enspace\hspace*{\fill}
\finalhyphendemerits=0
[14th August]}

\index{Knight's Hill , Lambeth@Knight's Hill, \emph{Lambeth}}

Disqualification (employed by the concil) of Sonia Winifred (Lab).

\noindent
\begin{tabular*}{\columnwidth}{@{\extracolsep{\fill}} p{0.545\columnwidth} >{\itshape}l r @{\extracolsep{\fill}}}
Sonia Winifred & Lab & 1265\\
Heidi Nicholson & C & 248\\
Christopher Hocknell & Grn & 230\\
Robin Lambert & UKIP & 99\\
Robert Hardware & LD & 94\\
Nelly Amos & Ind & 51\\
\end{tabular*}

\subsection*{Lewisham}

At the May 2014 ordinary election there was an unfilled vacancy in Evelyn ward due to the death of Sam Owalabi-Oluyole (Lab).
\index{Evelyn , Lewisham@Evelyn, \emph{Lewisham}}

\section{Greater Manchester}

\subsection*{Bolton}

\subsubsection*{Horwich and Blackrod \hspace*{\fill}\nolinebreak[1]%
\enspace\hspace*{\fill}
\finalhyphendemerits=0
[22nd May]}

\index{Horwich and Blackrod , Bolton@Horwich and Blackrod, \emph{Bolton}}

Resignation of Lindsey Kell (Lab).

Combined with the 2014 ordinary election.

\subsubsection*{Harper Green \hspace*{\fill}\nolinebreak[1]%
\enspace\hspace*{\fill}
\finalhyphendemerits=0
[16th October]}

\index{Harper Green , Bolton@Harper Green, \emph{Bolton}}

Death of Asha Ali Ismail (Lab).

\noindent
\begin{tabular*}{\columnwidth}{@{\extracolsep{\fill}} p{0.545\columnwidth} >{\itshape}l r @{\extracolsep{\fill}}}
Susan Haworth & Lab & 1176\\
Jeff Armstrong & UKIP & 777\\
Robert Tyler & C & 282\\
James Tomkinson & Grn & 38\\
Rebekah Fairhurst & LD & 28\\
Joseph Holt & Ind & 19\\
\end{tabular*}

\subsection*{Bury}

\subsubsection*{Ramsbottom \hspace*{\fill}\nolinebreak[1]%
\enspace\hspace*{\fill}
\finalhyphendemerits=0
[6th March; C gain from Lab]}

\index{Ramsbottom , Bury@Ramsbottom, \emph{Bury}}

Resignation of Joanne Columbine (Lab).

\noindent
\begin{tabular*}{\columnwidth}{@{\extracolsep{\fill}} p{0.545\columnwidth} >{\itshape}l r @{\extracolsep{\fill}}}
Robert Hodkinson & C & 1398\\
Sarah Southworth & Lab & 1033\\
Dave Barker & UKIP & 351\\
Glyn Heath & Grn & 157\\
David Foss & LD & 38\\
\end{tabular*}

\subsection*{Manchester}

\subsubsection*{Sharston \hspace*{\fill}\nolinebreak[1]%
\enspace\hspace*{\fill}
\finalhyphendemerits=0
[22nd May]}

\index{Sharston , Manchester@Sharston, \emph{Manchester}}

Resignation of Joyce Keller (Lab).

Combined with the 2014 ordinary election.

\subsection*{Oldham}

\subsubsection*{St James \hspace*{\fill}\nolinebreak[1]%
\enspace\hspace*{\fill}
\finalhyphendemerits=0
[22nd May]}

\index{Saint James , Oldham@St James, \emph{Oldham}}

Resignation of Nigel Newton (Lab).

Combined with the 2014 ordinary election.

\subsection*{Salford}

\subsubsection*{Swinton South \hspace*{\fill}\nolinebreak[1]%
\enspace\hspace*{\fill}
\finalhyphendemerits=0
[Tuesday 7th January]}

\index{Swinton South , Salford@Swinton S., \emph{Salford}}

Death of Norbert Potter (Lab).

\noindent
\begin{tabular*}{\columnwidth}{@{\extracolsep{\fill}} p{0.545\columnwidth} >{\itshape}l r @{\extracolsep{\fill}}}
Neil Blower & Lab & 661\\
Anne Broomhead & C & 298\\
Robert Wakefield & UKIP & 215\\
Joe O'Neill & Grn & 196\\
Paul Whitelegg & EDP & 54\\
Steve Cullen & TUSC & 43\\
\end{tabular*}

\subsection*{Stockport}

\subsubsection*{Bramhall South and Woodford \hspace*{\fill}\nolinebreak[1]%
\enspace\hspace*{\fill}
\finalhyphendemerits=0
[20th November]}

\index{Bramhall South and Woodford , Stockport@Bramhall S. \& Woodford, \emph{Stockport}}

Resignation of Anita Johnson (C).

\noindent
\begin{tabular*}{\columnwidth}{@{\extracolsep{\fill}} p{0.545\columnwidth} >{\itshape}l r @{\extracolsep{\fill}}}
John McGahan & C & 2080\\
Jeremy Meal & LD & 1502\\
David McDonough & Grn & 197\\
Kathryn Priestley & Lab & 132\\
\end{tabular*}

\subsection*{Trafford}

\subsubsection*{Broadheath \hspace*{\fill}\nolinebreak[1]%
\enspace\hspace*{\fill}
\finalhyphendemerits=0
[16th January; Lab gain from C]}

\index{Broadheath , Trafford@Broadheath, \emph{Trafford}}

Death of Ken Weston (C).

\noindent
\begin{tabular*}{\columnwidth}{@{\extracolsep{\fill}} p{0.545\columnwidth} >{\itshape}l r @{\extracolsep{\fill}}}
Helen Boyle & Lab & 1377\\
Brenda Houraghan & C & 1258\\
Ron George & UKIP & 234\\
Will Jones & LD & 150\\
Joe Ryan & Grn & 67\\
\end{tabular*}

\subsubsection*{Altrincham \hspace*{\fill}\nolinebreak[1]%
\enspace\hspace*{\fill}
\finalhyphendemerits=0
[22nd May]}

\index{Altrincham , Trafford@Altrincham, \emph{Trafford}}

Resignation of Matt Colledge (C).

Combined with the 2014 ordinary election.

\subsection*{Wigan}

CA = Community Action

\subsubsection*{Douglas \hspace*{\fill}\nolinebreak[1]%
\enspace\hspace*{\fill}
\finalhyphendemerits=0
[13th November]}

\index{Douglas , Wigan@Douglas, \emph{Wigan}}

Resignation of Joy Birch (Lab).

\noindent
\begin{tabular*}{\columnwidth}{@{\extracolsep{\fill}} p{0.545\columnwidth} >{\itshape}l r @{\extracolsep{\fill}}}
Maggie Skilling & Lab & 874\\
Derek Wilkes & UKIP & 452\\
Margaret Atherton & C & 80\\
Damien Hendry & Grn & 37\\
Michael Moulding & CA & 29\\
\end{tabular*}

\section{Merseyside}

\subsection*{Knowsley}

\subsubsection*{Longview \hspace*{\fill}\nolinebreak[1]%
\enspace\hspace*{\fill}
\finalhyphendemerits=0
[13th March]}

\index{Longview , Knowsley@Longview, \emph{Knowsley}}

Resignation of Diane Reid (Lab).

\noindent
\begin{tabular*}{\columnwidth}{@{\extracolsep{\fill}} p{0.545\columnwidth} >{\itshape}l r @{\extracolsep{\fill}}}
Margi O'Mara & Lab & 670\\
Paul Woods & Ind & 327\\
Adam Butler & C & 37\\
\end{tabular*}

\subsection*{Wirral}

\subsubsection*{Greasby, Frankby and Irby \hspace*{\fill}\nolinebreak[1]%
\enspace\hspace*{\fill}
\finalhyphendemerits=0
[22nd May]}

\index{Greasby, Frankby and Irby , Wirral@Greasby, Frankby \& Irby, \emph{Wirral}}

Resignation of Tony Cox (C).

Combined with the 2014 ordinary election.

\section{South Yorkshire}

\subsection*{Barnsley}

\subsubsection*{Penistone West \hspace*{\fill}\nolinebreak[1]%
\enspace\hspace*{\fill}
\finalhyphendemerits=0
[10th July]}

\index{Penistone West , Barnsley@Penistone W., \emph{Barnsley}}

Death of Peter Starling (Lab).

\noindent
\begin{tabular*}{\columnwidth}{@{\extracolsep{\fill}} p{0.545\columnwidth} >{\itshape}l r @{\extracolsep{\fill}}}
David Griffin & Lab & 772\\
Andrew Millner & C & 719\\
David Wood & UKIP & 622\\
Steve Webber & Ind & 348\\
\end{tabular*}

\subsection*{Doncaster}

\subsubsection*{\sloppyword{Edenthorpe, Kirk Sandall and Barnby Dun} \hspace*{\fill}\nolinebreak[1]%
\enspace\hspace*{\fill}
\finalhyphendemerits=0
[24th July; UKIP gain from Lab]}

\index{Edenthorpe, Kirk Sandall and Barnby Dun , Doncaster@Edenthorpe, Kirk Sandall \& Barnby Dun, \emph{Doncaster}}

Resignation of Pat Hall (Lab).

\noindent
\begin{tabular*}{\columnwidth}{@{\extracolsep{\fill}} p{0.545\columnwidth} >{\itshape}l r @{\extracolsep{\fill}}}
Paul Bissett & UKIP & 1203\\
David Nevett & Lab & 1109\\
Nick Allen & C & 479\\
Pete Kennedy & Grn & 160\\
\end{tabular*}

\subsection*{Sheffield}

\subsubsection*{Arbourthorne \hspace*{\fill}\nolinebreak[1]%
\enspace\hspace*{\fill}
\finalhyphendemerits=0
[6th February]}

\index{Arbourthorne , Sheffield@Arbourthorne, \emph{Sheffield}}

Death of John Robson (Lab).

\noindent
\begin{tabular*}{\columnwidth}{@{\extracolsep{\fill}} p{0.545\columnwidth} >{\itshape}l r @{\extracolsep{\fill}}}
Mike Drabble & Lab & 1398\\
Bob Sheridan & UKIP & 482\\
Peter Smith & C & 213\\
Alan Munro & TUSC & 204\\
Susan Ross & LD & 161\\
Jennyfer Barnard & Grn & 143\\
David Wildgoose & EDP & 75\\
\end{tabular*}

\subsubsection*{Walkley \hspace*{\fill}\nolinebreak[1]%
\enspace\hspace*{\fill}
\finalhyphendemerits=0
[22nd May]}

\index{Walkley , Sheffield@Walkley, \emph{Sheffield}}

Resignation of Nikki Sharpe (Lab).

Combined with the 2014 ordinary election.

\section{Tyne and Wear}

\subsection*{Newcastle upon Tyne}

\subsubsection*{Woolsington \hspace*{\fill}\nolinebreak[1]%
\enspace\hspace*{\fill}
\finalhyphendemerits=0
[22nd May]}

\index{Woolsington , Newcastle upon Tyne@Woolsington, \emph{Newcastle upon Tyne}}

Resignation of Kevin Graham (Lab).

Combined with the 2014 ordinary election.

\subsubsection*{North Jesmond \hspace*{\fill}\nolinebreak[1]%
\enspace\hspace*{\fill}
\finalhyphendemerits=0
[28th August]}

\index{North Jesmond , Newcastle upon Tyne@North Jesmond, \emph{Newcastle upon Tyne}}

Resignation of Peter Andras (LD).

\noindent
\begin{tabular*}{\columnwidth}{@{\extracolsep{\fill}} p{0.545\columnwidth} >{\itshape}l r @{\extracolsep{\fill}}}
Gerard Keating & LD & 711\\
Peter Smith & Lab & 320\\
Duncan Crute & C & 117\\
Daniel Thompson & UKIP & 112\\
Shehla Naqvi & Grn & 94\\
\end{tabular*}

\subsection*{South Tyneside}

At the May 2014 ordinary election there was an unfilled vacancy in West Park ward due to the death of Bob Watters (Lab).
\index{West Park , South Tyneside@West Park, \emph{S. Tyneside}}

\subsubsection*{Westoe \hspace*{\fill}\nolinebreak[1]%
\enspace\hspace*{\fill}
\finalhyphendemerits=0
[2nd October; UKIP gain from Ind]}

\index{Westoe , South Tyneside@Westoe, \emph{S. Tyneside}}

Death of Jane McBride (Ind).

\noindent
\begin{tabular*}{\columnwidth}{@{\extracolsep{\fill}} p{0.545\columnwidth} >{\itshape}l r @{\extracolsep{\fill}}}
Norman Dennis & UKIP & 676\\
Katharine Maxwell & Lab & 625\\
Edward Russell & C & 219\\
Tony Bengtssom & Grn & 90\\
Carole Troupe & LD & 41\\
\end{tabular*}

\subsection*{Sunderland}

\subsubsection*{St Anne's \hspace*{\fill}\nolinebreak[1]%
\enspace\hspace*{\fill}
\finalhyphendemerits=0
[27th March]}

\index{Saint Anne's , Sunderland@St Anne's, \emph{Sunderland}}

Resignation of Lisa Smiles (Lab).

\noindent
\begin{tabular*}{\columnwidth}{@{\extracolsep{\fill}} p{0.545\columnwidth} >{\itshape}l r @{\extracolsep{\fill}}}
Jacqui Gallagher & Lab & 945\\
Aileen Casey & UKIP & 555\\
Tony Morrissey & C & 345\\
Emily Blyth & Grn & 120\\
\end{tabular*}

\subsubsection*{Washington East \hspace*{\fill}\nolinebreak[1]%
\enspace\hspace*{\fill}
\finalhyphendemerits=0
[11th December]}

\index{Washington East , Sunderland@Washington E., \emph{Sunderland}}

Resignation of Neville Padgett (Lab).

\noindent
\begin{tabular*}{\columnwidth}{@{\extracolsep{\fill}} p{0.545\columnwidth} >{\itshape}l r @{\extracolsep{\fill}}}
Tony Taylor & Lab & 775\\
Hilary Johnson & C & 595\\
Alistair Baxter & UKIP & 506\\
Tony Murphy & Grn & 93\\
Stephen O'Brien & LD & 52\\
\end{tabular*}

\section{West Midlands}

\subsection*{Birmingham}

\subsubsection*{Kingstanding \hspace*{\fill}\nolinebreak[1]%
\enspace\hspace*{\fill}
\finalhyphendemerits=0
[13th February; C gain from Lab]}

\index{Kingstanding , Birmingham@Kingstanding, \emph{Birmingham}}

Resignation of Cath Grundy (Lab).

\noindent
\begin{tabular*}{\columnwidth}{@{\extracolsep{\fill}} p{0.545\columnwidth} >{\itshape}l r @{\extracolsep{\fill}}}
Gary Sambrook & C & 1571\\
Lorraine Owen & Lab & 1433\\
Roger Tempest & UKIP & 266\\
Graham Lippiatt & LD & 43\\
Terry Williams & Ind & 33\\
\end{tabular*}

\subsubsection*{Sutton Trinity \hspace*{\fill}\nolinebreak[1]%
\enspace\hspace*{\fill}
\finalhyphendemerits=0
[22nd May]}

\index{Sutton Trinity , Birmingham@Sutton Trinity, \emph{Birmingham}}

Resignation of Philip Parkin (C).

Combined with the 2014 ordinary election.

\subsection*{Coventry}

\subsubsection*{Cheylesmore \hspace*{\fill}\nolinebreak[1]%
\enspace\hspace*{\fill}
\finalhyphendemerits=0
[22nd May]}

\index{Cheylesmore , Coventry@Cheylesmore, \emph{Coventry}}

Resignation of Kevin Foster (C).

Combined with the 2014 ordinary election.

\subsection*{Walsall}

\subsubsection*{Birchills Leamore \hspace*{\fill}\nolinebreak[1]%
\enspace\hspace*{\fill}
\finalhyphendemerits=0
[24th July]}

\index{Birchills Leamore , Walsall@Birchills-Leamore, \emph{Walsall}}

Death of Tim Oliver (Lab).

\noindent
\begin{tabular*}{\columnwidth}{@{\extracolsep{\fill}} p{0.545\columnwidth} >{\itshape}l r @{\extracolsep{\fill}}}
Chris Jones & Lab & 1075\\
Gazanfer Ali & C & 710\\
Paul White & UKIP & 445\\
Chris Newey & EDP & 20\\
\end{tabular*}

\subsection*{Wolverhampton}

At the May 2014 ordinary election there was an unfilled vacancy in Wednesfield North ward due to the death of Neil Clarke (C).
\index{Wednesfield North , Wolverhampton@Wednesfield N., \emph{Wolverhampton}}

\section{West Yorkshire}

\subsection*{Bradford}

\subsubsection*{Craven \hspace*{\fill}\nolinebreak[1]%
\enspace\hspace*{\fill}
\finalhyphendemerits=0
[22nd May]}

\index{Craven , Bradford@Craven, \emph{Bradford}}

Death of Michael Kelly (C).

Combined with the 2014 ordinary election.

\subsection*{Leeds}

At the May 2014 ordinary election there was an unfilled vacancy in Adel and Wharfedale ward due to the death of Clive Fox (C).
\index{Adel and Wharfedale , Leeds@Adel \& Wharfedale, \emph{Leeds}}

\section{Bedfordshire}

\subsection*{Bedford}

\subsubsection*{Putnoe \hspace*{\fill}\nolinebreak[1]%
\enspace\hspace*{\fill}
\finalhyphendemerits=0
[22nd May]}

\index{Putnoe , Bedford@Putnoe, \emph{Bedford}}

Resignation of Sallyanne Smith (LD).

\noindent
\begin{tabular*}{\columnwidth}{@{\extracolsep{\fill}} p{0.545\columnwidth} >{\itshape}l r @{\extracolsep{\fill}}}
Rosemary Bootiman & LD & 1364\\
Susan Spratt & C & 707\\
Adrian Haynes & UKIP & 412\\
Graham Tranquada & Lab & 367\\
Ben Fitch & Grn & 94\\
\end{tabular*}

\subsection*{Luton}

\subsubsection*{Farley \hspace*{\fill}\nolinebreak[1]%
\enspace\hspace*{\fill}
\finalhyphendemerits=0
[13th March]}

\index{Farley , Luton@Farley, \emph{Luton}}

Resignation of Robin Harris (Lab).

\noindent
\begin{tabular*}{\columnwidth}{@{\extracolsep{\fill}} p{0.545\columnwidth} >{\itshape}l r @{\extracolsep{\fill}}}
Paul Castleman & Lab & 1232\\
Charles Lawman & UKIP & 226\\
David Coulter & C & 154\\
Anne Mead & LD & 46\\
Marc Scheimann & Grn & 41\\
\end{tabular*}

\section{Berkshire}

\subsection*{Reading}

\subsubsection*{Park \hspace*{\fill}\nolinebreak[1]%
\enspace\hspace*{\fill}
\finalhyphendemerits=0
[22nd May]}

\index{Park , Reading@Park, \emph{Reading}}

Resignation of Melanie Eastwood (Grn).

Combined with the 2014 ordinary election.

\subsubsection*{Southcote \hspace*{\fill}\nolinebreak[1]%
\enspace\hspace*{\fill}
\finalhyphendemerits=0
[24th July]}

\index{Southcote , Reading@Southcote, \emph{Reading}}

Death of Pete Ruhemann (Lab).

\noindent
\begin{tabular*}{\columnwidth}{@{\extracolsep{\fill}} p{0.545\columnwidth} >{\itshape}l r @{\extracolsep{\fill}}}
Matthew Lawrence & Lab & 1019\\
Ellis Wiggins & C & 340\\
Ann Zebedee & UKIP & 226\\
Alan Lockey & Grn & 69\\
Margaret McNeill & LD & 49\\
\end{tabular*}

\subsection*{Windsor and Maidenhead}

\subsubsection*{Clewer North \hspace*{\fill}\nolinebreak[1]%
\enspace\hspace*{\fill}
\finalhyphendemerits=0
[24th July]}

\index{Clewer North , Windsor and Maidenhead@Clewer N., \emph{Windsor \& Maidenhead}}

Death of Cynthia Endacott (Ind).

\noindent
\begin{tabular*}{\columnwidth}{@{\extracolsep{\fill}} p{0.545\columnwidth} >{\itshape}l r @{\extracolsep{\fill}}}
Wisdom da Costa & Ind & 878\\
John Collins & C & 486\\
Peter Shearman & Lab & 158\\
\end{tabular*}

\subsubsection*{Cox Green \hspace*{\fill}\nolinebreak[1]%
\enspace\hspace*{\fill}
\finalhyphendemerits=0
[11th December]}

\index{Cox Green , Windsor and Maidenhead@Cox Green, \emph{Windsor \& Maidenhead}}

Resignation of Alan Mellins (C).

\noindent
\begin{tabular*}{\columnwidth}{@{\extracolsep{\fill}} p{0.545\columnwidth} >{\itshape}l r @{\extracolsep{\fill}}}
Ross McWilliams & C & 738\\
Gareth Jones & LD & 315\\
Lance Carter & UKIP & 278\\
Robert Horner & Lab & 124\\
\end{tabular*}

\subsection*{Wokingham}

\subsubsection*{Bulmershe and Whitegates \hspace*{\fill}\nolinebreak[1]%
\enspace\hspace*{\fill}
\finalhyphendemerits=0
[13th November; C gain from LD]}

\index{Bulmershe and Whitegates , Wokingham@Bulmershe \& Whitegates, \emph{Wokingham}}

Disqualification (sentenced to four months in prison, suspended, benefit fraud) of Lesley Hayward (LD).

\noindent
\begin{tabular*}{\columnwidth}{@{\extracolsep{\fill}} p{0.545\columnwidth} >{\itshape}l r @{\extracolsep{\fill}}}
Alison Swaddle & C & 726\\
Greg Bello & Lab & 498\\
Munir Ahmed & LD & 448\\
Peter Jackson & UKIP & 275\\
Adrian Windisch & Grn & 105\\
\end{tabular*}

\section{Bristol}

\subsubsection*{St George West \hspace*{\fill}\nolinebreak[1]%
\enspace\hspace*{\fill}
\finalhyphendemerits=0
[22nd May]}

\index{Saint George West , Bristol@St George W., \emph{Bristol}}

Resignation of Peter Hammond (Lab).

Combined with the 2014 ordinary election.

\columnbreak

\section{Buckinghamshire}

\subsection*{Aylesbury Vale}

\subsubsection*{Gatehouse \hspace*{\fill}\nolinebreak[1]%
\enspace\hspace*{\fill}
\finalhyphendemerits=0
[11th December]}

\index{Gatehouse , Aylesbury Vale@Gatehouse, \emph{Aylesbury Vale}}

Resignation of Stuart Jarvis (LD).

\noindent
\begin{tabular*}{\columnwidth}{@{\extracolsep{\fill}} p{0.545\columnwidth} >{\itshape}l r @{\extracolsep{\fill}}}
Anders Christensen & LD & 295\\
Graham Cadle & UKIP & 267\\
Samantha North & C & 113\\
Lucio Tangi & Lab & 113\\
Mary Hunt & Grn & 28\\
George Entecott & Ind & 12\\
\end{tabular*}

\subsubsection*{Southcourt \hspace*{\fill}\nolinebreak[1]%
\enspace\hspace*{\fill}
\finalhyphendemerits=0
[11th December; LD gain from Lab]}

\index{Southcourt , Aylesbury Vale@Southcourt, \emph{Aylesbury Vale}}

Resignation of Michael Beall (Lab).

\noindent
\begin{tabular*}{\columnwidth}{@{\extracolsep{\fill}} p{0.545\columnwidth} >{\itshape}l r @{\extracolsep{\fill}}}
Peter Agoro & LD & 429\\
Brian Adams & UKIP & 266\\
Mark Bateman & Lab & 175\\
Sarah Sproat & C & 112\\
Andrew Kulig & Grn & 33\\
\end{tabular*}

\section{Cambridgeshire}

\subsection*{County Council}

\subsubsection*{Willingham \hspace*{\fill}\nolinebreak[1]%
\enspace\hspace*{\fill}
\finalhyphendemerits=0
[22nd May]}

\index{Willingham , Cambridgeshire@Willingham, \emph{Cambs.}}

Resignation of Ray Manning (C).

\noindent
\begin{tabular*}{\columnwidth}{@{\extracolsep{\fill}} p{0.545\columnwidth} >{\itshape}l r @{\extracolsep{\fill}}}
Peter Hudson & C & 1252\\
Martin Hale & UKIP & 642\\
Benjamin Monks & Lab & 471\\
Susan Gymer & LD & 338\\
Helen Stocks-Morgan & Grn & 295\\
\end{tabular*}

\subsection*{Cambridge}

\subsubsection*{Petersfield \hspace*{\fill}\nolinebreak[1]%
\enspace\hspace*{\fill}
\finalhyphendemerits=0
[22nd May]}

\index{Petersfield , Cambridge@Petersfield, \emph{Cambridge}}

Resignation of Gail Marchant-Daisley (Lab).

Combined with the 2014 ordinary election.

\subsubsection*{Queen Edith's \hspace*{\fill}\nolinebreak[1]%
\enspace\hspace*{\fill}
\finalhyphendemerits=0
[13th November; LD gain from Lab]}

\index{Queen Edith's , Cambridge@Queen Edith's, \emph{Cambridge}}

Resignation of Sue Birtles (Lab).

\noindent
\begin{tabular*}{\columnwidth}{@{\extracolsep{\fill}} p{0.545\columnwidth} >{\itshape}l r @{\extracolsep{\fill}}}
Viki Sanders & LD & 933\\
Rahima Ahammed & Lab & 790\\
Andrew Bower & C & 614\\
Joel Chalfen & Grn & 222\\
\end{tabular*}

\subsection*{East Cambridgeshire}

\subsubsection*{Sutton \hspace*{\fill}\nolinebreak[1]%
\enspace\hspace*{\fill}
\finalhyphendemerits=0
[24th April; LD gain from C]}

\index{Sutton , East Cambridgeshire@Sutton, \emph{E. Cambs.}}

Resignation of Peter Moakes (C).

\noindent
\begin{tabular*}{\columnwidth}{@{\extracolsep{\fill}} p{0.545\columnwidth} >{\itshape}l r @{\extracolsep{\fill}}}
Lorna Dupre & LD & 523\\
Neil Hitchin & C & 280\\
Daniel Divine & UKIP & 162\\
Jane Frances & Lab & 63\\
\end{tabular*}

\subsubsection*{Soham South \hspace*{\fill}\nolinebreak[1]%
\enspace\hspace*{\fill}
\finalhyphendemerits=0
[19th June; C gain from Ind]}

\index{Soham South , East Cambridgeshire@Soham S., \emph{E. Cambs.}}

Death of John Palmer (Ind).

\noindent
\begin{tabular*}{\columnwidth}{@{\extracolsep{\fill}} p{0.545\columnwidth} >{\itshape}l r @{\extracolsep{\fill}}}
Hamish Ross & C & 363\\
Daniel Divine & UKIP & 201\\
Charles Warner & LD & 191\\
Geoffrey Woollard & Ind & 148\\
Gerard Hobbs & Ind & 80\\
Fiona Ross & Lab & 71\\
\end{tabular*}

\subsection*{Fenland}

\subsubsection*{Roman Bank \hspace*{\fill}\nolinebreak[1]%
\enspace\hspace*{\fill}
\finalhyphendemerits=0
[8th May]}

\index{Roman Bank , Fenland@Roman Bank, \emph{Fenland}}

Resignation of Phil Hatton (C).

\noindent
\begin{tabular*}{\columnwidth}{@{\extracolsep{\fill}} p{0.545\columnwidth} >{\itshape}l r @{\extracolsep{\fill}}}
Samantha Clark & C & 763\\
Alan Lay & UKIP & 537\\
Barry Diggle & Lab & 193\\
Erbie Murat & Ind & 70\\
Stephen Court & LD & 24\\
\end{tabular*}

\subsubsection*{Medworth \hspace*{\fill}\nolinebreak[1]%
\enspace\hspace*{\fill}
\finalhyphendemerits=0
[16th October]}

\index{Medworth , Fenland@Medworth, \emph{Fenland}}

Disqualification (sentenced to 21 months' imprisonment, suspended, illegal possession of a firearm) of Jonathan Farmer (C).

\noindent
\begin{tabular*}{\columnwidth}{@{\extracolsep{\fill}} p{0.545\columnwidth} >{\itshape}l r @{\extracolsep{\fill}}}
Steve Tierney & C & 257\\
Andrew Hunt & UKIP & 201\\
Kathy Dougall & Lab & 79\\
Josephine Ratcliffe & LD & 24\\
Erbie Murat & Ind & 15\\
\end{tabular*}

\subsection*{Huntingdonshire}

\subsubsection*{Warboys and Bury \hspace*{\fill}\nolinebreak[1]%
\enspace\hspace*{\fill}
\finalhyphendemerits=0
[7th August]}

\index{Warboys and Bury , Huntingdonshire@Warboys \& Bury, \emph{Hunts.}}

Death of Flt/Lt John Pethard (C).

\noindent
\begin{tabular*}{\columnwidth}{@{\extracolsep{\fill}} p{0.545\columnwidth} >{\itshape}l r @{\extracolsep{\fill}}}
Angie Curtis & C & 619\\
Michael Tew & UKIP & 560\\
Christine Wills & LD & 78\\
Mary Howell & Lab & 72\\
\end{tabular*}

\subsubsection*{St Neots Priory Park \hspace*{\fill}\nolinebreak[1]%
\enspace\hspace*{\fill}
\finalhyphendemerits=0
[27th November]}

\index{Saint Neots Priory Park , Huntingdonshire@St Neots Priory Park, \emph{Huntingdonshire}}

Death of Paula Longford (C).

\noindent
\begin{tabular*}{\columnwidth}{@{\extracolsep{\fill}} p{0.545\columnwidth} >{\itshape}l r @{\extracolsep{\fill}}}
Ian Gardener & C & 448\\
Carol Gamby & UKIP & 337\\
Angela Hogan & Lab & 199\\
\end{tabular*}

\subsection*{Peterborough}

\subsubsection*{Eye and Thorney \hspace*{\fill}\nolinebreak[1]%
\enspace\hspace*{\fill}
\finalhyphendemerits=0
[22nd May]}

\index{Eye and Thorney , Peterborough@Eye \& Thorney, \emph{Peterborough}}

Resignation of Dale McKean (C).

Combined with the 2014 ordinary election.

\subsubsection*{West \hspace*{\fill}\nolinebreak[1]%
\enspace\hspace*{\fill}
\finalhyphendemerits=0
[22nd May]}

\index{West , Peterborough@West, \emph{Peterborough}}

Resignation of Matthew Dalton (C).

Combined with the 2014 ordinary election.

\subsection*{South Cambridgeshire}

At the May 2014 ordinary election there was an unfilled vacancy in Bourn ward due to the resignation of Clayton Hudson (Ind elected as C).
\index{Bourn , South Cambridgeshire@Bourn, \emph{S. Cambs.}}

\section{Cheshire}

\subsection*{Cheshire East}

\subsubsection*{Crewe West \hspace*{\fill}\nolinebreak[1]%
\enspace\hspace*{\fill}
\finalhyphendemerits=0
[13th March]}

\index{Crewe West , Cheshire East@Crewe W., \emph{Cheshire E.}}

Death of Peter Nurse (Lab).

\noindent
\begin{tabular*}{\columnwidth}{@{\extracolsep{\fill}} p{0.545\columnwidth} >{\itshape}l r @{\extracolsep{\fill}}}
Kevin Hickson & Lab & 720\\
Richard Lee & UKIP & 387\\
Chris Curran & Ind & 159\\
Chris Waling & C & 122\\
Robert Icke & LD & 55\\
\end{tabular*}

\subsection*{Cheshire West and Chester}

\subsubsection*{Boughton \hspace*{\fill}\nolinebreak[1]%
\enspace\hspace*{\fill}
\finalhyphendemerits=0
[10th July]}

\index{Boughton , Cheshire West and Chester@Boughton, \emph{Cheshire W. \& Chester}}

Death of David Robinson (Lab).

\noindent
\begin{tabular*}{\columnwidth}{@{\extracolsep{\fill}} p{0.545\columnwidth} >{\itshape}l r @{\extracolsep{\fill}}}
Martyn Delaney & Lab & 614\\
Kate Vaughan & C & 469\\
Charles Dodman & UKIP & 131\\
John McNamara & Grn & 86\\
Mark Gant & LD & 70\\
\end{tabular*}

\subsubsection*{Winnington and Castle \hspace*{\fill}\nolinebreak[1]%
\enspace\hspace*{\fill}
\finalhyphendemerits=0
[10th July]}

\index{Winnington and Castle , Cheshire West and Chester@Winnington \& Castle, \emph{Cheshire W. \& Chester}}

Resignation of Amy Mercer-Bailey (Lab).

\noindent
\begin{tabular*}{\columnwidth}{@{\extracolsep{\fill}} p{0.545\columnwidth} >{\itshape}l r @{\extracolsep{\fill}}}
Sam Naylor & Lab & 525\\
Jim Sinar & C & 418\\
Amos Wright & UKIP & 307\\
Alice Chapman & LD & 80\\
\end{tabular*}

\subsection*{Halton}

\subsubsection*{Kingsway \hspace*{\fill}\nolinebreak[1]%
\enspace\hspace*{\fill}
\finalhyphendemerits=0
[11th December]}

\index{Kingsway , Halton@Kingsway, \emph{Halton}}

Death of Frank Fraser (Lab).

\noindent
\begin{tabular*}{\columnwidth}{@{\extracolsep{\fill}} p{0.545\columnwidth} >{\itshape}l r @{\extracolsep{\fill}}}
Andrea Wall & Lab & 537\\
Brad Bradshaw & UKIP & 164\\
Duncan Harper & C & 22\\
Paul Meara & LD & 11\\
\end{tabular*}

\section{Cornwall}

\subsubsection*{Illogan \hspace*{\fill}\nolinebreak[1]%
\enspace\hspace*{\fill}
\finalhyphendemerits=0
[10th July; LD gain from C]}

\index{Illogan , Cornwall@Illogan, \emph{Cornwall}}

Resignation of Terry Wilkins (C).

\noindent
\begin{tabular*}{\columnwidth}{@{\extracolsep{\fill}} p{0.545\columnwidth} >{\itshape}l r @{\extracolsep{\fill}}}
David Ekinsmyth & LD & 277\\
Stephen Richardson & MK & 217\\
Adam Desmonde & C & 215\\
Clive Polkinghorne & UKIP & 156\\
Trevor Chalker & Lab & 129\\
Paul Holmes & Lib & 121\\
Jacqueline Merrick & Grn & 50\\
\end{tabular*}

\subsubsection*{Mabe, Perranarworthal and St Gluvias \hspace*{\fill}\nolinebreak[1]%
\enspace\hspace*{\fill}
\finalhyphendemerits=0
[17th July; C gain from UKIP]}

\index{Mabe, Perranarworthal and Saint Gluvias , Cornwall@Mabe, Perranarworthal \& St Gluvias, \emph{Cornwall}}

Resignation of Michael Keogh (UKIP).

\noindent
\begin{tabular*}{\columnwidth}{@{\extracolsep{\fill}} p{0.545\columnwidth} >{\itshape}l r @{\extracolsep{\fill}}}
Peter Williams & C & 406\\
John Ault & LD & 405\\
Pete Tisdale & UKIP & 271\\
Linda Hitchcox & Lab & 107\\
Karen Sumser-Lupson & MK & 58\\
\end{tabular*}

\subsubsection*{Mevagissey \hspace*{\fill}\nolinebreak[1]%
\enspace\hspace*{\fill}
\finalhyphendemerits=0
[6th November; C gain from Lab]}

\index{Mevagissey , Cornwall@Mevagissey, \emph{Cornwall}}

Resignation of Michael Bunney (Lab).

\noindent
\begin{tabular*}{\columnwidth}{@{\extracolsep{\fill}} p{0.545\columnwidth} >{\itshape}l r @{\extracolsep{\fill}}}
James Mustoe & C & 348\\
Michael Williams & UKIP & 281\\
Charmain Nicholas & Lab & 204\\
Christopher Maynard & LD & 197\\
Katherine Moseley & Grn & 50\\
\end{tabular*}

\section{Cumbria}

\subsection*{County Council}

\subsubsection*{Belle Vue \hspace*{\fill}\nolinebreak[1]%
\enspace\hspace*{\fill}
\finalhyphendemerits=0
[10th April]}

\index{Belle Vue , Cumbria@Belle Vue, \emph{Cumbria}}

Death of Ian Stockdale (Lab).

\noindent
\begin{tabular*}{\columnwidth}{@{\extracolsep{\fill}} p{0.545\columnwidth} >{\itshape}l r @{\extracolsep{\fill}}}
Christine Bowditch & Lab & 565\\
Nigel Christian & C & 435\\
John Stanyer & UKIP & 234\\
\end{tabular*}

\subsubsection*{Castle \hspace*{\fill}\nolinebreak[1]%
\enspace\hspace*{\fill}
\finalhyphendemerits=0
[11th September]}

\index{Castle , Cumbria@Castle, \emph{Cumbria}}

Death of Willie Whalen (Lab).

\noindent
\begin{tabular*}{\columnwidth}{@{\extracolsep{\fill}} p{0.545\columnwidth} >{\itshape}l r @{\extracolsep{\fill}}}
Alan McGuckin & Lab & 389\\
James Bainbridge & C & 245\\
John Stanyer & UKIP & 235\\
Lawrence Jennings & LD & 112\\
Neil Boothman & Grn & 51\\
\end{tabular*}

\subsubsection*{Windermere \hspace*{\fill}\nolinebreak[1]%
\enspace\hspace*{\fill}
\finalhyphendemerits=0
[2nd October]}

\index{Windermere , Cumbria@Windermere, \emph{Cumbria}}

Death of Jo Stephenson (LD).

\noindent
\begin{tabular*}{\columnwidth}{@{\extracolsep{\fill}} p{0.545\columnwidth} >{\itshape}l r @{\extracolsep{\fill}}}
Colin Jones & LD & 1061\\
Ben Berry & C & 810\\
Robert Judson & Ind & 123\\
Gwen Harrison & Grn & 61\\
\end{tabular*}

\subsection*{Carlisle}

\subsubsection*{Castle \hspace*{\fill}\nolinebreak[1]%
\enspace\hspace*{\fill}
\finalhyphendemerits=0
[11th September]}

\index{Castle , Carlisle@Castle, \emph{Carlisle}}

Death of Willie Whalen (Lab).

\noindent
\begin{tabular*}{\columnwidth}{@{\extracolsep{\fill}} p{0.545\columnwidth} >{\itshape}l r @{\extracolsep{\fill}}}
Alan Taylor & Lab & 364\\
Robert Currie & C & 212\\
Fiona Mills & UKIP & 208\\
Lawrence Jennings & LD & 121\\
Richard Hunt & Grn & 42\\
\end{tabular*}

\subsection*{South Lakeland}

\subsubsection*{Windermere Town \hspace*{\fill}\nolinebreak[1]%
\enspace\hspace*{\fill}
\finalhyphendemerits=0
[2nd October]}

\index{Windermere Town , South Lakeland@Windermere Town, \emph{S. Lakeland}}

Death of Jo Stephenson (LD).

\noindent
\begin{tabular*}{\columnwidth}{@{\extracolsep{\fill}} p{0.545\columnwidth} >{\itshape}l r @{\extracolsep{\fill}}}
Dyan Jones & LD & 416\\
Sandra Lilley & C & 184\\
Gwen Harrison & Grn & 50\\
\end{tabular*}

\section{Derbyshire}

\subsection*{County Council}

\subsubsection*{Alport and Derwent \hspace*{\fill}\nolinebreak[1]%
\enspace\hspace*{\fill}
\finalhyphendemerits=0
[13th November]}

\index{Alport and Derwent , Derbyshire@Alport \& Derwent, \emph{Derbys.}}

Resignation of Martin Tomlinson (C).

\noindent
\begin{tabular*}{\columnwidth}{@{\extracolsep{\fill}} p{0.545\columnwidth} >{\itshape}l r @{\extracolsep{\fill}}}
David Taylor & C & 1118\\
David Fisher & UKIP & 715\\
Mike Ratcliffe & Lab & 656\\
\end{tabular*}

\subsection*{Amber Valley}

\subsubsection*{Heanor East \hspace*{\fill}\nolinebreak[1]%
\enspace\hspace*{\fill}
\finalhyphendemerits=0
[30th January]}

\index{Heanor East , Amber Valley@Heanor E., \emph{Amber Valley}}

Resignation of Phil Hill (Lab).

\noindent
\begin{tabular*}{\columnwidth}{@{\extracolsep{\fill}} p{0.545\columnwidth} >{\itshape}l r @{\extracolsep{\fill}}}
Sheila Oakes & Lab & 548\\
Steven Grainger & C & 350\\
Kate Smith & LD & 41\\
\end{tabular*}

\subsubsection*{Heanor West \hspace*{\fill}\nolinebreak[1]%
\enspace\hspace*{\fill}
\finalhyphendemerits=0
[13th March]}

\index{Heanor West , Amber Valley@Heanor W., \emph{Amber Valley}}

Death of Bob Janes (Lab).

\noindent
\begin{tabular*}{\columnwidth}{@{\extracolsep{\fill}} p{0.545\columnwidth} >{\itshape}l r @{\extracolsep{\fill}}}
Celia Cox & Lab & 595\\
Philip Rose & UKIP & 259\\
Mark Burrell & C & 229\\
Kate Smith & LD & 41\\
\end{tabular*}

\subsubsection*{Heage and Ambergate \hspace*{\fill}\nolinebreak[1]%
\enspace\hspace*{\fill}
\finalhyphendemerits=0
[22nd May]}

\index{Heage and Ambergate , Amber Valley@Heage \& Ambergate, \emph{Amber Valley}}

Resignation of Maurice Gent (Lab).

Combined with the 2014 ordinary election.

\subsubsection*{Swanwick \hspace*{\fill}\nolinebreak[1]%
\enspace\hspace*{\fill}
\finalhyphendemerits=0
[26th June]}

\index{Swanwick , Amber Valley@Swanwick, \emph{Amber Valley}}

Ordinary election postponed from 22nd May; death of candidate Patricia Watson (C).

\noindent
\begin{tabular*}{\columnwidth}{@{\extracolsep{\fill}} p{0.545\columnwidth} >{\itshape}l r @{\extracolsep{\fill}}}
David Wilson & C & 474\\
Antony Tester & Lab & 298\\
George Soudah & Ind & 252\\
Allen King & UKIP & 245\\
Joel Hunt & LD & 32\\
\end{tabular*}

\subsection*{Bolsover}

\subsubsection*{South Normanton East \hspace*{\fill}\nolinebreak[1]%
\enspace\hspace*{\fill}
\finalhyphendemerits=0
[14th August]}

\index{South Normanton East , Bolsover@South Normanton E., \emph{Bolsover}}

Death of Terry Cook (Lab).

\noindent
\begin{tabular*}{\columnwidth}{@{\extracolsep{\fill}} p{0.545\columnwidth} >{\itshape}l r @{\extracolsep{\fill}}}
Tracey Cannon & Lab & 293\\
Robert Sainsbury & C & 120\\
\end{tabular*}

\subsection*{Derby}

\subsubsection*{Littleover \hspace*{\fill}\nolinebreak[1]%
\enspace\hspace*{\fill}
\finalhyphendemerits=0
[22nd May]}

\index{Littleover , Derby@Littleover, \emph{Derby}}

Resignation of Les Allen (LD).

Combined with the 2014 ordinary election.

\subsection*{North East Derbyshire}

\subsubsection*{Coal Aston \hspace*{\fill}\nolinebreak[1]%
\enspace\hspace*{\fill}
\finalhyphendemerits=0
[10th April]}

\index{Coal Aston , North East Derbyshire@Coal Aston, \emph{N.E. Derbys.}}

Death of Mike Emmens (C).

\noindent
\begin{tabular*}{\columnwidth}{@{\extracolsep{\fill}} p{0.545\columnwidth} >{\itshape}l r @{\extracolsep{\fill}}}
John Allsop & C & 518\\
David Cheetham & Lab & 409\\
Charles Watson & UKIP & 193\\
\end{tabular*}

\section{Devon}

\subsection*{County Council}

\subsubsection*{Yealmpton \hspace*{\fill}\nolinebreak[1]%
\enspace\hspace*{\fill}
\finalhyphendemerits=0
[22nd May]}

\index{Yealmpton , Devon@Yealmpton, \emph{Devon}}

Resignation of Will Mumford (C).

\noindent
\begin{tabular*}{\columnwidth}{@{\extracolsep{\fill}} p{0.545\columnwidth} >{\itshape}l r @{\extracolsep{\fill}}}
Richard Hosking & C & 2493\\
Brian Blake & LD & 1028\\
David Trigger & Lab & 706\\
\end{tabular*}

\subsection*{South Hams}

\subsubsection*{Salcombe and Malborough \hspace*{\fill}\nolinebreak[1]%
\enspace\hspace*{\fill}
\finalhyphendemerits=0
[22nd May]}

\index{Salcombe and Malborough , South Hams@Salcombe \& Malborough, \emph{S. Hams}}

Death of John Carter (C).

\noindent
\begin{tabular*}{\columnwidth}{@{\extracolsep{\fill}} p{0.545\columnwidth} >{\itshape}l r @{\extracolsep{\fill}}}
Judy Pearce & C & 726\\
Dennis Hobday & UKIP & 411\\
\end{tabular*}

\subsection*{Torridge}

\subsubsection*{Bideford East \hspace*{\fill}\nolinebreak[1]%
\enspace\hspace*{\fill}
\finalhyphendemerits=0
[20th March]}

\index{Bideford East , Torridge@Bideford E., \emph{Torridge}}

Death of Steve Clarke (Ind).

\noindent
\begin{tabular*}{\columnwidth}{@{\extracolsep{\fill}} p{0.545\columnwidth} >{\itshape}l r @{\extracolsep{\fill}}}
Sam Robinson & Ind & 295\\
Dermot McGeough & C & 150\\
James Craigie & Lab & 140\\
David Ratcliff & Ind & 106\\
Bob Wootton & LD & 39\\
Alan Smith & Ind & 17\\
\end{tabular*}

\subsubsection*{Kenwith \hspace*{\fill}\nolinebreak[1]%
\enspace\hspace*{\fill}
\finalhyphendemerits=0
[10th July]}

\index{Kenwith , Torridge@Kenwith, \emph{Torridge}}

Resignation of Kathy Murdoch (Ind elected as C).

\noindent
\begin{tabular*}{\columnwidth}{@{\extracolsep{\fill}} p{0.545\columnwidth} >{\itshape}l r @{\extracolsep{\fill}}}
Alison Boyle & C & 136\\
Derek Sargent & UKIP & 99\\
David Gale & Ind & 98\\
Hugh Bone & Ind & 69\\
Simon Mathers & Grn & 28\\
Geoff Hastings & Lab & 26\\
\end{tabular*}

\section{Dorset}

\subsection*{East Dorset}

\subsubsection*{Colehill East \hspace*{\fill}\nolinebreak[1]%
\enspace\hspace*{\fill}
\finalhyphendemerits=0
[17th July]}

\index{Colehill East , East Dorset@Colehill E., \emph{E. Dorset}}

Death of Don Wallace (LD).

\noindent
\begin{tabular*}{\columnwidth}{@{\extracolsep{\fill}} p{0.545\columnwidth} >{\itshape}l r @{\extracolsep{\fill}}}
Barry Roberts & LD & 741\\
Graeme Smith & C & 326\\
David Mattocks & UKIP & 184\\
\end{tabular*}

\section{Durham}

\subsection*{Darlington}

\subsubsection*{Mowden \hspace*{\fill}\nolinebreak[1]%
\enspace\hspace*{\fill}
\finalhyphendemerits=0
[22nd May]}

\index{Mowden , Darlington@Mowden, \emph{Darlington}}

Death of Ron Lewis (C).

\noindent
\begin{tabular*}{\columnwidth}{@{\extracolsep{\fill}} p{0.545\columnwidth} >{\itshape}l r @{\extracolsep{\fill}}}
Pauline Culley & C & 647\\
Jackie Saint & Lab & 614\\
Charlotte Bull & UKIP & 235\\
Hilary Allen & LD & 93\\
\end{tabular*}

\subsection*{Durham}

DInd = Derwentside Independent

\subsubsection*{Crook \hspace*{\fill}\nolinebreak[1]%
\enspace\hspace*{\fill}
\finalhyphendemerits=0
[18th September; Lab gain from Ind]}

\index{Crook , Durham@Crook, \emph{Durham}}

Death of Eddie Murphy (Ind).

\noindent
\begin{tabular*}{\columnwidth}{@{\extracolsep{\fill}} p{0.545\columnwidth} >{\itshape}l r @{\extracolsep{\fill}}}
Maureen Stanton & Lab & 753\\
Betty Hopson & UKIP & 339\\
David English & LD & 233\\
Tony Simpson & Ind & 193\\
Alan Booth & C & 90\\
\end{tabular*}

\subsubsection*{Burnopfield and Dipton \hspace*{\fill}\nolinebreak[1]%
\enspace\hspace*{\fill}
\finalhyphendemerits=0
[23rd October; Lab gain from DInd]}

\index{Burnopfield and Dipton , Durham@Burnopfield and Dipton, \emph{Durham}}

Resignation of Bob Alderson (DInd).

\noindent
\begin{tabular*}{\columnwidth}{@{\extracolsep{\fill}} p{0.545\columnwidth} >{\itshape}l r @{\extracolsep{\fill}}}
Joanne Carr & Lab & 656\\
Gill Burnett & DInd & 655\\
Alan Booth & C & 83\\
Melanie Howd & Grn & 68\\
\end{tabular*}

\subsubsection*{Evenwood \hspace*{\fill}\nolinebreak[1]%
\enspace\hspace*{\fill}
\finalhyphendemerits=0
[23rd October]}

\index{Evenwood , Durham@Evenwood, \emph{Durham}}

Death of Pauline Charlton (Lab).

\noindent
\begin{tabular*}{\columnwidth}{@{\extracolsep{\fill}} p{0.545\columnwidth} >{\itshape}l r @{\extracolsep{\fill}}}
Heather Smith & Lab & 546\\
Stephen Hugill & C & 396\\
Ben Casey & UKIP & 309\\
Lee Carnighan & Ind & 108\\
Greg Robinson & Grn & 72\\
\end{tabular*}

\subsection*{Hartlepool}

At the May 2014 ordinary election there was an unfilled vacancy in Seaton ward, caused by the death of Cath Hill (Ind).
\index{Seaton , Hartlepool@Seaton, \emph{Hartlepool}}

\section{East Sussex}

\subsection*{Rother}

\subsubsection*{Collington \hspace*{\fill}\nolinebreak[1]%
\enspace\hspace*{\fill}
\finalhyphendemerits=0
[12th June]}

\index{Collington , Rother@Collington, \emph{Rother}}

Resignation of John Lee (Ind).

\noindent
\begin{tabular*}{\columnwidth}{@{\extracolsep{\fill}} p{0.545\columnwidth} >{\itshape}l r @{\extracolsep{\fill}}}
Douglas Oliver & Ind & 570\\
Gillian Wheeler & C & 378\\
Philip Moore & UKIP & 311\\
Alan Bearne & Lab & 102\\
\end{tabular*}

\subsubsection*{Darwell \hspace*{\fill}\nolinebreak[1]%
\enspace\hspace*{\fill}
\finalhyphendemerits=0
[31st July]}

\index{Darwell , Rother@Darwell, \emph{Rother}}

Resignation of Bob White (C).

\noindent
\begin{tabular*}{\columnwidth}{@{\extracolsep{\fill}} p{0.545\columnwidth} >{\itshape}l r @{\extracolsep{\fill}}}
Eleanor Kirby-Green & C & 361\\
Edward Smith & UKIP & 182\\
Andrew Wedmore & Grn & 154\\
Suz Evasdaughter & Lab & 84\\
Tracy Dixon & LD & 65\\
\end{tabular*}

\section{East Yorkshire}

\subsection*{East Riding}

\subsubsection*{Bridlington Central and Old Town \hspace*{\fill}\nolinebreak[1]%
\enspace\hspace*{\fill}
\finalhyphendemerits=0
[27th November; UKIP gain from SDP]}

\index{Bridlington Central and Old Town , East Riding@Bridlington C. \& Old Town, \emph{E. Riding}}

Death of Ray Allerston (SDP).

\noindent
\begin{tabular*}{\columnwidth}{@{\extracolsep{\fill}} p{0.545\columnwidth} >{\itshape}l r @{\extracolsep{\fill}}}
Malcolm Milns & UKIP & 401\\
John Copsey & C & 352\\
Liam Dealtry & Ind & 217\\
Terry Dixon & Ind & 214\\
Neil Tate & Ind & 116\\
\end{tabular*}

\subsubsection*{Howdenshire \hspace*{\fill}\nolinebreak[1]%
\enspace\hspace*{\fill}
\finalhyphendemerits=0
[27th November]}

\index{Howdenshire , East Riding@Howdenshire, \emph{E. Riding}}

Death of Doreen Engall (C).

\noindent
\begin{tabular*}{\columnwidth}{@{\extracolsep{\fill}} p{0.545\columnwidth} >{\itshape}l r @{\extracolsep{\fill}}}
Nigel Wilkinson & C & 1020\\
Clive Waddington & UKIP & 891\\
Danny Marten & Lab & 298\\
\end{tabular*}

\subsubsection*{Willerby and Kirk Ella \hspace*{\fill}\nolinebreak[1]%
\enspace\hspace*{\fill}
\finalhyphendemerits=0
[27th November]}

\index{Willerby and Kirk Ella , East Riding@Willerby \& Kirk Ella, \emph{E. Riding}}

Death of Angela Ibson (C).

\noindent
\begin{tabular*}{\columnwidth}{@{\extracolsep{\fill}} p{0.545\columnwidth} >{\itshape}l r @{\extracolsep{\fill}}}
Mike Burchill & C & 1522\\
Robert Skinner & UKIP & 699\\
Daniel Palmer & Lab & 515\\
\end{tabular*}

\subsection*{Kingston upon Hull}

\subsubsection*{Marfleet \hspace*{\fill}\nolinebreak[1]%
\enspace\hspace*{\fill}
\finalhyphendemerits=0
[22nd May]}

\index{Marfleet , Kingston upon Hull@Marfleet, \emph{Kingston u. Hull}}

Death of Sheila Waudby (Lab).

Combined with the 2014 ordinary election.

\section{Essex}

\subsection*{County Council}

\subsubsection*{Brightlingsea \hspace*{\fill}\nolinebreak[1]%
\enspace\hspace*{\fill}
\finalhyphendemerits=0
[9th October; C gain from UKIP]}

\index{Brightlingsea , Essex@Brightlingsea, \emph{Essex}}

Resignation of Roger Lord (UKIP).

\noindent
\begin{tabular*}{\columnwidth}{@{\extracolsep{\fill}} p{0.545\columnwidth} >{\itshape}l r @{\extracolsep{\fill}}}
Alan Goggin & C & 1809\\
Anne Poonian & UKIP & 1642\\
Gary Scott & LD & 1199\\
Carol Carlsson Browne & Lab & 524\\
Beverley Maltby & Grn & 200\\
\end{tabular*}

\subsection*{Basildon}

\subsubsection*{Lee Chapel North \hspace*{\fill}\nolinebreak[1]%
\enspace\hspace*{\fill}
\finalhyphendemerits=0
[22nd May]}

\index{Lee Chapel North , Basildon@Lee Chapel N., \emph{Basildon}}

Resignation of Lynda Gordon (Ind elected as Lab).

Combined with the 2014 ordinary election.

\subsection*{Castle Point}

CIIP = Canvey Island Independent Party

\subsubsection*{Canvey Island East \hspace*{\fill}\nolinebreak[1]%
\enspace\hspace*{\fill}
\finalhyphendemerits=0
[23rd October; Ind gain from CIIP]}

\index{Canvey Island East , Castle Point@Canvey Island E., \emph{Castle Point}}

Disqualification of Gail Barton (CIIP) for non-attendance.

\noindent
\begin{tabular*}{\columnwidth}{@{\extracolsep{\fill}} p{0.545\columnwidth} >{\itshape}l r @{\extracolsep{\fill}}}
Colin Letchford & Ind & 389\\
John Payne & CIIP & 323\\
Chas Mumford & C & 208\\
Jackie Reilly & Lab & 76\\
\end{tabular*}

\subsection*{Chelmsford}

\subsubsection*{Bicknacre and East and West Hanningfield \hspace*{\fill}\nolinebreak[1]%
\enspace\hspace*{\fill}
\finalhyphendemerits=0
[9th October]}

\index{Bicknacre and East and West Hanningfield , Chelmsford@Bicknacre and E. \& W. Hanningfield, \emph{Chelmsford}}

Death of Michael Harris (C).

\noindent
\begin{tabular*}{\columnwidth}{@{\extracolsep{\fill}} p{0.545\columnwidth} >{\itshape}l r @{\extracolsep{\fill}}}
Matt Flack & C & 649\\
David Kirkwood & UKIP & 359\\
Tony Lees & Lab & 80\\
Reza Hossain & Grn & 35\\
Andy Robson & LD & 34\\
\end{tabular*}

\columnbreak

\subsection*{Colchester}

PatSoc = Patriotic Socialist

\subsubsection*{Wivenhoe Quay \hspace*{\fill}\nolinebreak[1]%
\enspace\hspace*{\fill}
\finalhyphendemerits=0
[3rd July]}

\index{Wivenhoe Quay , Colchester@Wivenhoe Quay, \emph{Colchester}}

Death of Steve Ford (Lab).

\noindent
\begin{tabular*}{\columnwidth}{@{\extracolsep{\fill}} p{0.545\columnwidth} >{\itshape}l r @{\extracolsep{\fill}}}
Rosalind Scott &Lab&857\\
Peter Hill &C&629\\
John Pitts &UKIP&129\\
Shaun Broughton &LD&127\\
Tim Glover &Grn&90\\
Dave Osborn &PatSoc&2\\
\end{tabular*}

\subsection*{Epping Forest}

At the May 2014 ordinary election there was an unfilled vacancy in Loughton Forest ward, caused by the death of Colin Finn (Loughton Residents' Association).

\subsubsection*{Broadley Common, Epping Upland and Nazeing \hspace*{\fill}\nolinebreak[1]%
\enspace\hspace*{\fill}
\finalhyphendemerits=0
[7th August]}

\index{Broadley Common, Epping Upland and Nazeing , Epping Forest@Broadley Common, Epping Upland \& Nazeing, \emph{Epping Forest}}

Death of Penny Smith (C).

\noindent
\begin{tabular*}{\columnwidth}{@{\extracolsep{\fill}} p{0.545\columnwidth} >{\itshape}l r @{\extracolsep{\fill}}}
Robert Glozier & C & 155\\
Ron McEvoy & UKIP & 122\\
William Hartington & Grn & 23\\
Arnold Verrall & LD & 7\\
\end{tabular*}

\subsubsection*{Epping Hemnall \hspace*{\fill}\nolinebreak[1]%
\enspace\hspace*{\fill}
\finalhyphendemerits=0
[25th September; LD gain from C]}

\index{Epping Hemnall , Epping Forest@Epping Hemnall, \emph{Epping Forest}}

Death of Kenneth Avey (C).

\noindent
\begin{tabular*}{\columnwidth}{@{\extracolsep{\fill}} p{0.545\columnwidth} >{\itshape}l r @{\extracolsep{\fill}}}
Kim Adams & LD & 607\\
Nigel Avey & C & 386\\
Andrew Smith & UKIP & 339\\
Anna Widdup & Grn & 69\\
\end{tabular*}

\subsection*{Harlow}

\subsubsection*{Mark Hall \hspace*{\fill}\nolinebreak[1]%
\enspace\hspace*{\fill}
\finalhyphendemerits=0
[22nd May]}

\index{Mark Hall , Harlow@Mark Hall, \emph{Harlow}}

Resignation of Paul Schroder (Lab).

Combined with the 2014 ordinary election.

\subsection*{Southend-on-Sea}

\subsubsection*{West Leigh \hspace*{\fill}\nolinebreak[1]%
\enspace\hspace*{\fill}
\finalhyphendemerits=0
[23rd January]}

\index{West Leigh , Southend-on-Sea@West Leigh, \emph{Southend-on-Sea}}

Resignation of Gwen Horrigan (C).

\noindent
\begin{tabular*}{\columnwidth}{@{\extracolsep{\fill}} p{0.545\columnwidth} >{\itshape}l r @{\extracolsep{\fill}}}
Georgina Phillips & C & 743\\
Christopher Bailey & LD & 688\\
Tino Callaghan & UKIP & 418\\
Jane Norman & Lab & 149\\
\end{tabular*}

\subsubsection*{Westborough \hspace*{\fill}\nolinebreak[1]%
\enspace\hspace*{\fill}
\finalhyphendemerits=0
[22nd May]}

\index{Westborough , Southend-on-Sea@Westborough, \emph{Southend-on-Sea}}

Resignation of Martin Terry (Ind).

Combined with the 2014 ordinary election.

\subsection*{Tendring}

\subsubsection*{Peter Bruff \hspace*{\fill}\nolinebreak[1]%
\enspace\hspace*{\fill}
\finalhyphendemerits=0
[6th February]}

\index{Peter Bruff , Tendring@Peter Bruff, \emph{Tendring}}

Death of Mitch Mitchell (C).

\noindent
\begin{tabular*}{\columnwidth}{@{\extracolsep{\fill}} p{0.545\columnwidth} >{\itshape}l r @{\extracolsep{\fill}}}
Sara Richardson & C & 271\\
Jon Salisbury & Lab & 180\\
Sue Shearing & UKIP & 153\\
John Candler & LD & 108\\
\end{tabular*}

\subsubsection*{St Johns \hspace*{\fill}\nolinebreak[1]%
\enspace\hspace*{\fill}
\finalhyphendemerits=0
[6th February]}

\index{Saint Johns , Tendring@St Johns, \emph{Tendring}}

Resignation of Peter Halliday (C).

\noindent
\begin{tabular*}{\columnwidth}{@{\extracolsep{\fill}} p{0.545\columnwidth} >{\itshape}l r @{\extracolsep{\fill}}}
Mick Skeels & C & 423\\
Norman Jacobs & Lab & 272\\
Laurie Gray & UKIP & 208\\
Clive Upton & Ind & 49\\
\end{tabular*}

\subsubsection*{Manningtree, Mistley, Little Bentley and Tendring \hspace*{\fill}\nolinebreak[1]%
\enspace\hspace*{\fill}
\finalhyphendemerits=0
[3rd July]}

\index{Manningtree, Mistley, Little Bentley and Tendring , Tendring@Manningtree, Mistley, Little Bentley and Tendring, \emph{Tendring}\looseness=-1}

Death of Sarah Candy (C).

\noindent
\begin{tabular*}{\columnwidth}{@{\extracolsep{\fill}} p{0.545\columnwidth} >{\itshape}l r @{\extracolsep{\fill}}}
Alan Coley &C&376\\
Rosemary Smith &LD&159\\
Mark Cole &UKIP&154\\
Jo Richardson &Lab&129\\
\end{tabular*}

\subsection*{Thurrock}

\subsubsection*{West Thurrock and South Stifford \hspace*{\fill}\nolinebreak[1]%
\enspace\hspace*{\fill}
\finalhyphendemerits=0
[16th October]}

\index{West Thurrock and South Stifford , Thurrock@W. Thurrock \& S. Stifford, \emph{Thurrock}}

Death of Andrew Smith (Lab).

\noindent
\begin{tabular*}{\columnwidth}{@{\extracolsep{\fill}} p{0.545\columnwidth} >{\itshape}l r @{\extracolsep{\fill}}}
Terry Brookes & Lab & 903\\
Russell Cherry & UKIP & 621\\
John Rowles & C & 270\\
\end{tabular*}

\subsubsection*{Aveley and Uplands \hspace*{\fill}\nolinebreak[1]%
\enspace\hspace*{\fill}
\finalhyphendemerits=0
[4th December]}

\index{Aveley and Uplands , Thurrock@Aveley \& Uplands, \emph{Thurrock}}

Death of Maggie O'Keeffe-Ray (UKIP).

\noindent
\begin{tabular*}{\columnwidth}{@{\extracolsep{\fill}} p{0.545\columnwidth} >{\itshape}l r @{\extracolsep{\fill}}}
Tim Aker & UKIP & 747\\
Teresa Webster & C & 520\\
John O'Regan & Lab & 338\\
Eddie Stringer & Ind & 217\\
\end{tabular*}

\section{Gloucestershire}

\subsection*{County Council}

\subsubsection*{Mitcheldean \hspace*{\fill}\nolinebreak[1]%
\enspace\hspace*{\fill}
\finalhyphendemerits=0
[23rd October; C gain from Ind]}

\index{Mitcheldean , Gloucestershire@Mitcheldean, \emph{Glos.}}

Death of Norman Stephens (Ind).

\noindent
\begin{tabular*}{\columnwidth}{@{\extracolsep{\fill}} p{0.545\columnwidth} >{\itshape}l r @{\extracolsep{\fill}}}
Brian Robinson & C & 959\\
Malcolm Berry & UKIP & 550\\
Ian Whitburn & Ind & 455\\
Jackie Fraser & Lab & 278\\
Sue Henchley & LD & 150\\
Ken Power & Grn & 106\\
\end{tabular*}

\subsection*{Cheltenham}

\subsubsection*{Charlton Park \hspace*{\fill}\nolinebreak[1]%
\enspace\hspace*{\fill}
\finalhyphendemerits=0
[3rd July; LD gain from C]}

\index{Charlton Park , Cheltenham@Charlton Park, \emph{Cheltenham}}

Ordinary election postponed from 22nd May; death of candidate Mark Daniel (Ind).

\noindent
\begin{tabular*}{\columnwidth}{@{\extracolsep{\fill}} p{0.545\columnwidth} >{\itshape}l r @{\extracolsep{\fill}}}
Paul Baker & LD & 861\\
Penny Hall & C & 767\\
Justin Dunne & UKIP & 154\\
John Bride & Lab & 46\\
Wayne Spiller & Grn & 46\\
\end{tabular*}

\subsection*{Forest of Dean}

\subsubsection*{Newnham and Westbury \hspace*{\fill}\nolinebreak[1]%
\enspace\hspace*{\fill}
\finalhyphendemerits=0
[23rd October]}

\index{Newnham and Westbury , Forest of Dean@Newnham \& Westbury, \emph{Forest of Dean}}

Death of Norman Stephens (Ind).

\noindent
\begin{tabular*}{\columnwidth}{@{\extracolsep{\fill}} p{0.545\columnwidth} >{\itshape}l r @{\extracolsep{\fill}}}
Simon Phelps & Ind & 321\\
Richard Boyles & C & 216\\
Peter Foster & UKIP & 102\\
Jenny Shaw & Lab & 100\\
Sid Phelps & Grn & 70\\
Ian King & LD & 25\\
\end{tabular*}

\subsection*{Stroud}

\subsubsection*{Central \hspace*{\fill}\nolinebreak[1]%
\enspace\hspace*{\fill}
\finalhyphendemerits=0
[22nd May]}

\index{Central , Stroud@Central, \emph{Stroud}}

Resignation of Andy Read (Ind).

Combined with the 2014 ordinary election.

\subsubsection*{Valley \hspace*{\fill}\nolinebreak[1]%
\enspace\hspace*{\fill}
\finalhyphendemerits=0
[7th August]}

\index{Valley , Stroud@Valley, \emph{Stroud}}

Resignation of Molly Scott Cato (Grn).

\noindent
\begin{tabular*}{\columnwidth}{@{\extracolsep{\fill}} p{0.545\columnwidth} >{\itshape}l r @{\extracolsep{\fill}}}
Martin Baxendale & Grn & 291\\
James Heslop & Lab & 230\\
Stuart Love & UKIP & 76\\
Stephen Davies & C & 67\\
Lucy Roberts & TUSC & 16\\
\end{tabular*}

\subsection*{Tewkesbury}

\subsubsection*{Brockworth \hspace*{\fill}\nolinebreak[1]%
\enspace\hspace*{\fill}
\finalhyphendemerits=0
[22nd May; C gain from LD]}

\index{Brockworth , Tewkesbury@Brockworth, \emph{Tewkesbury}}

Resignation of Maureen Rowcliffe-Quarry (LD).

\noindent
\begin{tabular*}{\columnwidth}{@{\extracolsep{\fill}} p{0.545\columnwidth} >{\itshape}l r @{\extracolsep{\fill}}}
Harry Turbyfield & C & 685\\
Edward Buxton & Lab & 455\\
Phillip Quarry & LD & 409\\
Robert Rendell & Grn & 281\\
\end{tabular*}

\section{Hampshire}

\subsection*{County Council}

\subsubsection*{Petersfield Butser \hspace*{\fill}\nolinebreak[1]%
\enspace\hspace*{\fill}
\finalhyphendemerits=0
[Wednesday 12th March]}

\index{Petersfield Butser , Hampshire@Petersfield Butser, \emph{Hants.}}

Death of John West (C).

\noindent
\begin{tabular*}{\columnwidth}{@{\extracolsep{\fill}} p{0.545\columnwidth} >{\itshape}l r @{\extracolsep{\fill}}}
Ken Moon & C & 1156\\
David Alexander & UKIP & 720\\
Richard Robinson & LD & 685\\
Bill Organ & Lab & 322\\
Adam Harper & Grn & 220\\
\end{tabular*}

\subsection*{Basingstoke and Deane}

\subsubsection*{Baughurst and Tadley North \hspace*{\fill}\nolinebreak[1]%
\enspace\hspace*{\fill}
\finalhyphendemerits=0
[22nd May]}

\index{Baughurst and Tadley North , Basingstoke and Deane@Baughurst \& Tadley N., \emph{Basingstoke \& Deane}}

Resignation of Graham Round (C).

Combined with the 2014 ordinary election.

\subsubsection*{Brighton Hill South \hspace*{\fill}\nolinebreak[1]%
\enspace\hspace*{\fill}
\finalhyphendemerits=0
[22nd May]}

\index{Brighton Hill South , Basingstoke and Deane@Brighton Hill S., \emph{Basingstoke \& Deane}}

Resignation of David Eyre (Lab).

Combined with the 2014 ordinary election.

\subsection*{East Hampshire}

\subsubsection*{Petersfield Bell Hill \hspace*{\fill}\nolinebreak[1]%
\enspace\hspace*{\fill}
\finalhyphendemerits=0
[Wednesday 12th March]}

\index{Petersfield Bell Hill , East Hampshire@Petersfield Bell Hill, \emph{E. Hants.}}

Death of John West (C).

\noindent
\begin{tabular*}{\columnwidth}{@{\extracolsep{\fill}} p{0.545\columnwidth} >{\itshape}l r @{\extracolsep{\fill}}}
Peter Marshall & C & 190\\
Peter Dimond & UKIP & 110\\
Colin Brazier & Lab & 75\\
Roger Mullenger & LD & 74\\
\end{tabular*}

\subsection*{New Forest}

\subsubsection*{Bransgore and Burley \hspace*{\fill}\nolinebreak[1]%
\enspace\hspace*{\fill}
\finalhyphendemerits=0
[11th December]}

\index{Bransgore and Burley , New Forest@Bransgore \& Burley, \emph{New Forest}}

Death of Ann Hickman (C).

\noindent
\begin{tabular*}{\columnwidth}{@{\extracolsep{\fill}} p{0.545\columnwidth} >{\itshape}l r @{\extracolsep{\fill}}}
Richard Frampton & C & 834\\
Roz Mills & UKIP & 171\\
Brian Curwain & Lab & 74\\
\end{tabular*}

\subsection*{Rushmoor}

\subsubsection*{West Heath \hspace*{\fill}\nolinebreak[1]%
\enspace\hspace*{\fill}
\finalhyphendemerits=0
[9th October]}

\index{West Heath , Rushmoor@West Heath, \emph{Rushmoor}}

Resignation of Malcolm Small (UKIP).

\noindent
\begin{tabular*}{\columnwidth}{@{\extracolsep{\fill}} p{0.545\columnwidth} >{\itshape}l r @{\extracolsep{\fill}}}
Dave Bell & UKIP & 662\\
Brian Parker & C & 312\\
Sue Gadsby & Lab & 196\\
Charlie Fraser-Fleming & LD & 132\\
\end{tabular*}

\subsection*{Southampton}

\subsubsection*{Millbrook \hspace*{\fill}\nolinebreak[1]%
\enspace\hspace*{\fill}
\finalhyphendemerits=0
[22nd May]}

\index{Millbrook , Southampton@Millbrook, \emph{Southampton}}

Resignation of Georgina Laming (Lab).

Combined with the 2014 ordinary election.

\section{Herefordshire}

IOCH = It's OUR County (Herefordshire)

\subsubsection*{Ledbury \hspace*{\fill}\nolinebreak[1]%
\enspace\hspace*{\fill}
\finalhyphendemerits=0
[17th July; IOCH gain from C]}

\index{Ledbury , Herefordshire@Ledbury, \emph{Herefs.}}

Death of Peter Watts (C).

\noindent
\begin{tabular*}{\columnwidth}{@{\extracolsep{\fill}} p{0.545\columnwidth} >{\itshape}l r @{\extracolsep{\fill}}}
Terry Widdows & IOCH & 835\\
Allen Conway & C & 618\\
Paul Stanford & UKIP & 166\\
\end{tabular*}

\subsubsection*{Leominster South \hspace*{\fill}\nolinebreak[1]%
\enspace\hspace*{\fill}
\finalhyphendemerits=0
[17th July; Grn gain from C]}

\index{Leominster South , Herefordshire@Leominster S., \emph{Herefs.}}

Death of Roger Hunt (C).

\noindent
\begin{tabular*}{\columnwidth}{@{\extracolsep{\fill}} p{0.545\columnwidth} >{\itshape}l r @{\extracolsep{\fill}}}
Jennifer Bartlett & Grn & 384\\
Wayne Rosser & C & 222\\
Angela Pendleton & Ind & 198\\
Liz Portman-Lewis & UKIP & 111\\
Emma Pardoe & Lab & 99\\
\end{tabular*}

\section{Hertfordshire}

\subsection*{Broxbourne}

\subsubsection*{Wormley and Turnford \hspace*{\fill}\nolinebreak[1]%
\enspace\hspace*{\fill}
\finalhyphendemerits=0
[22nd May]}

\index{Wormley and Turnford , Broxbourne@Wormley \& Turnford, \emph{Broxbourne}}

Resignation of Bob Bick (C).

Combined with the 2014 ordinary election.

\subsection*{East Hertfordshire}

\subsubsection*{Bishops Stortford Central \hspace*{\fill}\nolinebreak[1]%
\enspace\hspace*{\fill}
\finalhyphendemerits=0
[22nd May]}

\index{Bishops Stortford Central , East Hertfordshire@Bishops Stortford C., \emph{E. Herts.}}

Resignation of Peter Gray (C).

\noindent
\begin{tabular*}{\columnwidth}{@{\extracolsep{\fill}} p{0.545\columnwidth} >{\itshape}l r @{\extracolsep{\fill}}}
George Cutting & C & 779\\
Deborah Rennie & UKIP & 734\\
Natalie Russell & Lab & 509\\
Madeline Goldspink & LD & 483\\
\end{tabular*}

\subsection*{North Hertfordshire}

\subsubsection*{Hitchwood, Offa and Hoo \hspace*{\fill}\nolinebreak[1]%
\enspace\hspace*{\fill}
\finalhyphendemerits=0
[10th July]}

\index{Hitchwood, Offa and Hoo , North Hertfordshire@Hitchwood, Offa \& Hoo, \emph{N. Herts.}}

Ordinary election postponed from 22nd May; death of candidate.

\noindent
\begin{tabular*}{\columnwidth}{@{\extracolsep{\fill}} p{0.545\columnwidth} >{\itshape}l r @{\extracolsep{\fill}}}
Faye Barnard & C & 734\\
Colin Rafferty & UKIP & 203\\
Simon Watson & Lab & 116\\
Orla Nicholls & Grn & 74\\
Peter Johnson & LD & 57\\
\end{tabular*}

\subsection*{St Albans}

\subsubsection*{St Stephen \hspace*{\fill}\nolinebreak[1]%
\enspace\hspace*{\fill}
\finalhyphendemerits=0
[22nd May]}

\index{Saint Stephen , Saint Albans@St Stephen, \emph{St Albans}}

Death of Gordon Myland (C).

Combined with the 2014 ordinary election.

\subsubsection*{Wheathampstead \hspace*{\fill}\nolinebreak[1]%
\enspace\hspace*{\fill}
\finalhyphendemerits=0
[22nd May]}

\index{Wheathampstead , Saint Albans@Wheathampstead, \emph{St Albans}}

Resignation of Nigel Huddleston (C).

Combined with the 2014 ordinary election.

\subsection*{Stevenage}

At the May 2014 ordinary election there was an unfilled vacancy in Longmeadow ward due to the disqualification (non-attendance) of Christine Hurst (C).
\index{Longmeadow, Stevenage@Longmeadow, \emph{Stevenage}}

\subsubsection*{Shephall \hspace*{\fill}\nolinebreak[1]%
\enspace\hspace*{\fill}
\finalhyphendemerits=0
[22nd May]}

\index{Shephall , Stevenage@Shephall, \emph{Stevenage}}

Resignation of Jack Pickersgill (Lab).

Combined with the 2014 ordinary election.

\subsection*{Watford}

\subsubsection*{Callowland \hspace*{\fill}\nolinebreak[1]%
\enspace\hspace*{\fill}
\finalhyphendemerits=0
[22nd May]}

\index{Callowland , Watford@Callowland, \emph{Watford}}

Resignation of Steve Rackett (Grn).

Combined with the 2014 ordinary election.

\subsection*{Welwyn Hatfield}

\subsubsection*{Howlands \hspace*{\fill}\nolinebreak[1]%
\enspace\hspace*{\fill}
\finalhyphendemerits=0
[22nd May]}

\index{Howlands , Welwyn Hatfield@Howlands, \emph{Welwyn Hatfield}}

Resignation of David Hughes (C).

Combined with the 2014 ordinary election.

\subsubsection*{Northaw and Cuffley \hspace*{\fill}\nolinebreak[1]%
\enspace\hspace*{\fill}
\finalhyphendemerits=0
[22nd May]}

\index{Northaw and Cuffley , Welwyn Hatfield@Northaw \& Cuffley, \emph{Welwyn Hatfield}}

Resignation of Colin Crouch (C).

Combined with the 2014 ordinary election.

\section{Kent}

\subsection*{Ashford}

Ashford = Ashford Independent

\subsubsection*{Wye \hspace*{\fill}\nolinebreak[1]%
\enspace\hspace*{\fill}
\finalhyphendemerits=0
[6th March; Ashford gain from C]}

\index{Wye , Ashford@Wye, \emph{Ashford}}

Resignation of Steve Wright (C).

\noindent
\begin{tabular*}{\columnwidth}{@{\extracolsep{\fill}} p{0.545\columnwidth} >{\itshape}l r @{\extracolsep{\fill}}}
Noel Ovenden & Ashford & 323\\
Ian Cooling & C & 240\\
Elaine Evans & UKIP & 97\\
Geoff Meaden & Grn & 55\\
Dylan Jones & Lab & 22\\
Ken Blanshard & LD & 13\\
\end{tabular*}

\subsection*{Canterbury}

\subsubsection*{Barham Downs \hspace*{\fill}\nolinebreak[1]%
\enspace\hspace*{\fill}
\finalhyphendemerits=0
[13th March; LD gain from C]}

\index{Barham Downs , Canterbury@Barham Downs, \emph{Canterbury}}

Death of Bill Oakey (C).

\noindent
\begin{tabular*}{\columnwidth}{@{\extracolsep{\fill}} p{0.545\columnwidth} >{\itshape}l r @{\extracolsep{\fill}}}
Michael Sole & LD & 337\\
Stuart Walker & C & 285\\
Dave de Boick & UKIP & 164\\
David Wilson & Lab & 78\\
Pat Marsh & Grn & 40\\
\end{tabular*}

\subsection*{Dartford}

\subsubsection*{Stone \hspace*{\fill}\nolinebreak[1]%
\enspace\hspace*{\fill}
\finalhyphendemerits=0
[27th March]}

\index{Stone , Dartford@Stone, \emph{Dartford}}

Death of John Adams (Lab).

\noindent
\begin{tabular*}{\columnwidth}{@{\extracolsep{\fill}} p{0.545\columnwidth} >{\itshape}l r @{\extracolsep{\fill}}}
Catherine Stafford & Lab & 426\\
Stephanie Thredgle & C & 397\\
Jim Moore & UKIP & 307\\
\end{tabular*}

\subsubsection*{Brent \hspace*{\fill}\nolinebreak[1]%
\enspace\hspace*{\fill}
\finalhyphendemerits=0
[13th November]}

\index{Brent , Dartford@Brent, \emph{Dartford}}

Death of Nancy Wightman (C).

\noindent
\begin{tabular*}{\columnwidth}{@{\extracolsep{\fill}} p{0.545\columnwidth} >{\itshape}l r @{\extracolsep{\fill}}}
Rosanna Currans & C & 579\\
Mark Maddison & Lab & 402\\
Shan-e-din Choycha & UKIP & 316\\
\end{tabular*}

\subsubsection*{Littlebrook \hspace*{\fill}\nolinebreak[1]%
\enspace\hspace*{\fill}
\finalhyphendemerits=0
[13th November]}

\index{Littlebrook , Dartford@Littlebrook, \emph{Dartford}}

Death of John Muckle (Lab).

\noindent
\begin{tabular*}{\columnwidth}{@{\extracolsep{\fill}} p{0.545\columnwidth} >{\itshape}l r @{\extracolsep{\fill}}}
Daisy Page & Lab & 358\\
Sonia Keane & UKIP & 220\\
Calvin McLean & C & 172\\
\end{tabular*}

\subsection*{Gravesham}

\subsubsection*{Coldharbour \hspace*{\fill}\nolinebreak[1]%
\enspace\hspace*{\fill}
\finalhyphendemerits=0
[22nd May]}

\index{Coldharbour , Gravesham@Coldharbour, \emph{Gravesham}}

Resignation of Rosemary Leadley (Lab).

\noindent
\begin{tabular*}{\columnwidth}{@{\extracolsep{\fill}} p{0.545\columnwidth} >{\itshape}l r @{\extracolsep{\fill}}}
Shane Mochrie-Cox & Lab & 520\\
Sarinder Duroch & UKIP & 486\\
Bronwen McGarrity & C & 289\\
Ian Stevenson & LD & 35\\
John Michael & TUSC & 22\\
\end{tabular*}

\subsection*{Maidstone}

\subsubsection*{Harrietsham and Lenham \hspace*{\fill}\nolinebreak[1]%
\enspace\hspace*{\fill}
\finalhyphendemerits=0
[22nd May]}

\index{Harrietsham and Lenham , Maidstone@Harrietsham \& Lenham, \emph{Maidstone}}

Resignation of Tom Sams (Ind).

Combined with the 2014 ordinary election.

\subsubsection*{North \hspace*{\fill}\nolinebreak[1]%
\enspace\hspace*{\fill}
\finalhyphendemerits=0
[22nd May]}

\index{North , Maidstone@North, \emph{Maidstone}}

Resignation of Mervyn Warner (LD).

Combined with the 2014 ordinary election.

\subsubsection*{Staplehurst \hspace*{\fill}\nolinebreak[1]%
\enspace\hspace*{\fill}
\finalhyphendemerits=0
[24th July; LD gain from C]}

\index{Staplehurst , Maidstone@Staplehurst, \emph{Maidstone}}

Resignation of Richard Lusty (C).

\noindent
\begin{tabular*}{\columnwidth}{@{\extracolsep{\fill}} p{0.545\columnwidth} >{\itshape}l r @{\extracolsep{\fill}}}
Paulina Watson & LD & 609\\
Louise Brice & C & 603\\
Jamie Kalmar & UKIP & 311\\
John Randall & Lab & 117\\
David George & Grn & 41\\
\end{tabular*}

\subsection*{Medway}

\subsubsection*{Peninsula \hspace*{\fill}\nolinebreak[1]%
\enspace\hspace*{\fill}
\finalhyphendemerits=0
[20th November; UKIP gain from C]}

\index{Peninsula , Medway@Peninsula, \emph{Medway}}

Resignation of Chris Irvine (UKIP elected as C) to seek re-election.

\noindent
\begin{tabular*}{\columnwidth}{@{\extracolsep{\fill}} p{0.545\columnwidth} >{\itshape}l r @{\extracolsep{\fill}}}
Christopher Irvine & UKIP & 2850\\
Ron Sands & C & 1965\\
Pete Tungate & Lab & 716\\
Clive Gregory & Grn & 314\\
Christopher Sams & LD & 60\\
\end{tabular*}

\subsection*{Shepway}

\subsubsection*{Folkestone Harvey Central \hspace*{\fill}\nolinebreak[1]%
\enspace\hspace*{\fill}
\finalhyphendemerits=0
[2nd September; UKIP gain from C]}

\index{Folkestone Harvey Central , Shepway@Folkestone Harvey C., \emph{Shepway}}

Resignation of David Johnson (C).

\noindent
\begin{tabular*}{\columnwidth}{@{\extracolsep{\fill}} p{0.545\columnwidth} >{\itshape}l r @{\extracolsep{\fill}}}
David Callahan & UKIP & 287\\
Rodica Wheeler & C & 224\\
Tom McNeice & LD & 198\\
Wendy Mitchell & Lab & 196\\
David Horton & Grn & 96\\
Seth Cruse & TUSC & 29\\
\end{tabular*}

\subsubsection*{Folkestone Harvey West \hspace*{\fill}\nolinebreak[1]%
\enspace\hspace*{\fill}
\finalhyphendemerits=0
[23rd October]}

\index{Folkestone Harvey West , Shepway@Folkestone Harvey W., \emph{Shepway}}

Death of George Bunting (C).

\noindent
\begin{tabular*}{\columnwidth}{@{\extracolsep{\fill}} p{0.6\columnwidth} >{\itshape}l r @{\extracolsep{\fill}}}
Helen Barker & C & 385\\
Stephen Jardine & UKIP & 293\\
Hugh Robertson-Ritchie & LD & 262\\
Jasmine Heywood & Grn & 61\\
Nicola Keen & Lab & 57\\
\end{tabular*}

\subsection*{Swale}

\subsubsection*{Sheppey Central \hspace*{\fill}\nolinebreak[1]%
\enspace\hspace*{\fill}
\finalhyphendemerits=0
[16th October; UKIP gain from C]}

\index{Sheppey Central , Swale@Sheppey C., \emph{Swale}}

Death of John Morris (C).

\noindent
\begin{tabular*}{\columnwidth}{@{\extracolsep{\fill}} p{0.545\columnwidth} >{\itshape}l r @{\extracolsep{\fill}}}
David Jones & UKIP & 831\\
Tina Booth & C & 324\\
Alan Henley & Lab & 240\\
Mad Mike Young & Loony & 27\\
\end{tabular*}

\subsection*{Tonbridge and Malling}

\subsubsection*{Borough Green and Long Mill \hspace*{\fill}\nolinebreak[1]%
\enspace\hspace*{\fill}
\finalhyphendemerits=0
[9th January; Ind gain from C]}

\index{Borough Green and Long Mill , Tonbridge and Malling@Borough Green \& Long Mill, \emph{Tonbridge \& Malling}}

Resignation of David Evans (C).

\noindent
\begin{tabular*}{\columnwidth}{@{\extracolsep{\fill}} p{0.545\columnwidth} >{\itshape}l r @{\extracolsep{\fill}}}
Mike Taylor & Ind & 692\\
Stuart Murray & C & 588\\
David Waller & UKIP & 349\\
Victoria Hayman & Lab & 84\\
Howard Porter & Grn & 68\\
\end{tabular*}

\subsection*{Tunbridge Wells}

At the May 2014 ordinary election there was an unfilled vacancy in Southborough and High Brooms ward due to the death of Colin Bothwell (C).
\index{Southborough and High Brooms , Tunbridge Wells@Southborough \& High Brooms, \emph{Tunbridge Wells}}

\section{Lancashire}

\subsection*{Blackburn with Darwen}

\subsubsection*{Higher Croft \hspace*{\fill}\nolinebreak[1]%
\enspace\hspace*{\fill}
\finalhyphendemerits=0
[22nd May]}

\index{Higher Croft , Blackburn with Darwen@Higher Croft, \emph{Blackburn with Darwen}}

Death of Dorothy Walsh (Lab).

Combined with the 2014 ordinary election.

\subsection*{Blackpool}

\subsubsection*{Hawes Side \hspace*{\fill}\nolinebreak[1]%
\enspace\hspace*{\fill}
\finalhyphendemerits=0
[22nd May]}

\index{Hawes Side , Blackpool@Hawes Side, \emph{Blackpool}}

Death of Norman Hardy (Lab).

\noindent
\begin{tabular*}{\columnwidth}{@{\extracolsep{\fill}} p{0.545\columnwidth} >{\itshape}l r @{\extracolsep{\fill}}}
Pamela Jackson & Lab & 713\\
John Braithwaite & UKIP & 467\\
Stewart Forshaw & C & 329\\
Julie Daniels & Grn & 65\\
Paul Baron & LD & 46\\
Jonathan Baugh & TUSC & 18\\
\end{tabular*}

\subsubsection*{Layton \hspace*{\fill}\nolinebreak[1]%
\enspace\hspace*{\fill}
\finalhyphendemerits=0
[22nd May]}

\index{Layton , Blackpool@Layton, \emph{Blackpool}}

Resignation of John Boughton (Lab).

\noindent
\begin{tabular*}{\columnwidth}{@{\extracolsep{\fill}} p{0.545\columnwidth} >{\itshape}l r @{\extracolsep{\fill}}}
Kathryn Benson & Lab & 769\\
Sue Ridyard & C & 521\\
Stephen Werry & UKIP & 362\\
Jonathan Knowles & Grn & 60\\
Susan Close & LD & 48\\
Philip Watt & TUSC & 17\\
\end{tabular*}

\subsubsection*{Talbot \hspace*{\fill}\nolinebreak[1]%
\enspace\hspace*{\fill}
\finalhyphendemerits=0
[22nd May]}

\index{Talbot , Blackpool@Talbot, \emph{Blackpool}}

Resignation of Sarah Riding (Lab).

\noindent
\begin{tabular*}{\columnwidth}{@{\extracolsep{\fill}} p{0.545\columnwidth} >{\itshape}l r @{\extracolsep{\fill}}}
Ian Coleman & Lab & 583\\
Charlie Docherty & C & 431\\
Warwick Howlett & UKIP & 336\\
Philip Mitchell & Grn & 89\\
Paul Hindley & LD & 44\\
Stephen Troy & TUSC & 13\\
\end{tabular*}

\subsubsection*{Clifton \hspace*{\fill}\nolinebreak[1]%
\enspace\hspace*{\fill}
\finalhyphendemerits=0
[24th July]}

\index{Clifton , Blackpool@Clifton, \emph{Blackpool}}

Death of Joan Greenhalgh (Lab).

\noindent
\begin{tabular*}{\columnwidth}{@{\extracolsep{\fill}} p{0.545\columnwidth} >{\itshape}l r @{\extracolsep{\fill}}}
Luke Taylor & Lab & 501\\
Shackleton Spencer & UKIP & 362\\
Bruce Allen & C & 283\\
Gita Gordon & LD & 33\\
Tina Rothery & Grn & 25\\
Philip Watt & TUSC & 10\\
\end{tabular*}

\subsubsection*{Waterloo \hspace*{\fill}\nolinebreak[1]%
\enspace\hspace*{\fill}
\finalhyphendemerits=0
[9th October]}

\index{Waterloo , Blackpool@Waterloo, \emph{Blackpool}}

Death of Tony Lee (C).

\noindent
\begin{tabular*}{\columnwidth}{@{\extracolsep{\fill}} p{0.545\columnwidth} >{\itshape}l r @{\extracolsep{\fill}}}
Derek Robertson & C & 406\\
John Braithwaite & UKIP & 372\\
Kathy Ellis & Lab & 347\\
Mike Hodkinson & LD & 34\\
Jack Renshaw & BNP & 17\\
\end{tabular*}

\subsection*{Fylde}

FyldeRA = Fylde Ratepayers

\subsubsection*{St John's \hspace*{\fill}\nolinebreak[1]%
\enspace\hspace*{\fill}
\finalhyphendemerits=0
[27th March; FyldeRA gain from Ind]}

\index{Saint John's , Fylde@St John's, \emph{Fylde}}

Death of Kath Harper (Ind).

\noindent
\begin{tabular*}{\columnwidth}{@{\extracolsep{\fill}} p{0.53\columnwidth} >{\itshape}l r @{\extracolsep{\fill}}}
Mark Bamforth & FyldeRA & 804\\
Brenda Blackshaw & C & 205\\
Timothy Wood & UKIP & 100\\
Carol Gilligan & LD & 62\\
Bob Dennett & Grn & 53\\
\end{tabular*}

\subsection*{Lancaster}

\subsubsection*{Scotforth West \hspace*{\fill}\nolinebreak[1]%
\enspace\hspace*{\fill}
\finalhyphendemerits=0
[22nd May; Grn gain from Lab]}

\index{Scotforth West , Lancaster@Scotford W., \emph{Lancaster}}

Resignation of Josh Bancroft (Lab).

\noindent
\begin{tabular*}{\columnwidth}{@{\extracolsep{\fill}} p{0.545\columnwidth} >{\itshape}l r @{\extracolsep{\fill}}}
Abi Mills & Grn & 823\\
Colin Hartley & Lab & 802\\
Janet Walton & C & 517\\
Phil Dunster & LD & 80\\
Steve Metcalfe & TUSC & 49\\
\end{tabular*}

\subsubsection*{University \hspace*{\fill}\nolinebreak[1]%
\enspace\hspace*{\fill}
\finalhyphendemerits=0
[22nd May; Grn gain from Lab]}

\index{University , Lancaster@University, \emph{Lancaster}}

Resignation of Paul Aitchison (Lab).

\noindent
\begin{tabular*}{\columnwidth}{@{\extracolsep{\fill}} p{0.545\columnwidth} >{\itshape}l r @{\extracolsep{\fill}}}
Jack Filmore & Grn & 273\\
James Leyshon & Lab & 237\\
Daniel Aldred & C & 128\\
Oliver Mountjoy & LD & 33\\
Stuart Langhorn & Ind & 24\\
\end{tabular*}

\subsection*{Pendle}

Blue = Blue Party

\subsubsection*{Blacko and Higherford \hspace*{\fill}\nolinebreak[1]%
\enspace\hspace*{\fill}
\finalhyphendemerits=0
[3rd April]}

\index{Blacko and Higherford , Pendle@Blacko \& Higherford, \emph{Pendle}}

Disqualification of Shelagh Derwent (C) for non-attenance.

\noindent
\begin{tabular*}{\columnwidth}{@{\extracolsep{\fill}} p{0.545\columnwidth} >{\itshape}l r @{\extracolsep{\fill}}}
Noel McEvoy & C & 370\\
Mick Waddington & UKIP & 86\\
Robert Oliver & Lab & 65\\
Darren Reynolds & LD & 34\\
\end{tabular*}

\subsubsection*{Old Laund Booth \hspace*{\fill}\nolinebreak[1]%
\enspace\hspace*{\fill}
\finalhyphendemerits=0
[3rd July]}

\index{Old Laund Booth , Pendle@Old Laund Booth, \emph{Pendle}}

Resignation of John David (LD).

\noindent
\begin{tabular*}{\columnwidth}{@{\extracolsep{\fill}} p{0.545\columnwidth} >{\itshape}l r @{\extracolsep{\fill}}}
Brian Newman & LD & 427\\
Jill Hartley & C & 266\\
Michael Waddington & UKIP & 27\\
Kieron Hartley & Blue & 13\\
\end{tabular*}

\subsection*{Preston}

At the May 2014 ordinary election there was an unfilled vacancy in Fishwick ward due to the death of Tom Burns (Lab).
\index{Fishwick , Preston@Fishwick, \emph{Preston}}

\subsubsection*{Town Centre \hspace*{\fill}\nolinebreak[1]%
\enspace\hspace*{\fill}
\finalhyphendemerits=0
[22nd May]}

\index{Town Centre , Preston@Town Centre, \emph{Preston}}

Resignation of Michael Lavalette (Ind).

Combined with the 2014 ordinary election.

\subsection*{Rossendale}

\subsubsection*{Helmshore \hspace*{\fill}\nolinebreak[1]%
\enspace\hspace*{\fill}
\finalhyphendemerits=0
[16th October]}

\index{Helmshore , Rossendale@Helmshore, \emph{Rossendale}}

Resignation of Amanda Milling (C).

\noindent
\begin{tabular*}{\columnwidth}{@{\extracolsep{\fill}} p{0.545\columnwidth} >{\itshape}l r @{\extracolsep{\fill}}}
Tony Haworth & C & 771\\
Emma Harding & Lab & 444\\
Granville Barker & UKIP & 364\\
\end{tabular*}

\subsubsection*{Longholme \hspace*{\fill}\nolinebreak[1]%
\enspace\hspace*{\fill}
\finalhyphendemerits=0
[4th December]}

\index{Longholme , Rossendale@Longholme, \emph{Rossendale}}

Resignation of Liz McInnes (Lab).

\noindent
\begin{tabular*}{\columnwidth}{@{\extracolsep{\fill}} p{0.6\columnwidth} >{\itshape}l r @{\extracolsep{\fill}}}
Annie McMahon & Lab & 505\\
Mischa Charlton-Mockett & C & 390\\
Gary Barnes & UKIP & 258\\
\end{tabular*}

\subsection*{West Lancashire}

\subsubsection*{Tarleton \hspace*{\fill}\nolinebreak[1]%
\enspace\hspace*{\fill}
\finalhyphendemerits=0
[22nd May]}

\index{Tarleton , West Lancashire@Tarleton, \emph{W. Lancs.}}

Resignation of Andrew Cheetham (C).

Combined with the 2014 ordinary election.

\subsubsection*{Skelmersdale North \hspace*{\fill}\nolinebreak[1]%
\enspace\hspace*{\fill}
\finalhyphendemerits=0
[11th December]}

\index{Skelmersdale North , West Lancashire@Skelmersdale N., \emph{W. Lancs.}}

Death of Barry Nolan (Lab).

\noindent
\begin{tabular*}{\columnwidth}{@{\extracolsep{\fill}} p{0.545\columnwidth} >{\itshape}l r @{\extracolsep{\fill}}}
Jennifer Patterson & Lab & 591\\
David Meadows & C & 81\\
\end{tabular*}

\subsection*{Wyre}

\subsubsection*{Hardhorn \hspace*{\fill}\nolinebreak[1]%
\enspace\hspace*{\fill}
\finalhyphendemerits=0
[22nd May]}

\index{Hardhorn , Wyre@Hardhorn, \emph{Wyre}}

Death of Graeme Cocker (C).

\noindent
\begin{tabular*}{\columnwidth}{@{\extracolsep{\fill}} p{0.545\columnwidth} >{\itshape}l r @{\extracolsep{\fill}}}
Colette Birch & C & 828\\
Christopher Frist & Lab & 390\\
\end{tabular*}

\section{Leicestershire}

\subsection*{Charnwood}

BritDem = British Democrats

\subsubsection*{Birstall Wanlip \hspace*{\fill}\nolinebreak[1]%
\enspace\hspace*{\fill}
\finalhyphendemerits=0
[20th February; LD gain from C]}

\index{Birstall Wanlip , Charnwood@Birstall Wanlip, \emph{Charnwood}}

Death of Stuart Jones (C).

\noindent
\begin{tabular*}{\columnwidth}{@{\extracolsep{\fill}} p{0.545\columnwidth} >{\itshape}l r @{\extracolsep{\fill}}}
Simon Sansome & LD & 508\\
Mary Allen & C & 419\\
Marilyn Cowles & Lab & 355\\
\end{tabular*}

\subsubsection*{Thurmaston \hspace*{\fill}\nolinebreak[1]%
\enspace\hspace*{\fill}
\finalhyphendemerits=0
[31st July; Lab gain from C]}

\index{Thurmaston , Charnwood@Thurmaston, \emph{Charnwood}}

Resignation of Paul Harley (C).

\noindent
\begin{tabular*}{\columnwidth}{@{\extracolsep{\fill}} p{0.53\columnwidth} >{\itshape}l r @{\extracolsep{\fill}}}
Ralph Raven & Lab & 783\\
Tom Prior & UKIP & 496\\
Hanif Asmal & C & 404\\
Chris Canham & BritDem & 95\\
Stephen Denham & BNP & 58\\
\end{tabular*}

\subsection*{Melton}

\subsubsection*{Asfordby \hspace*{\fill}\nolinebreak[1]%
\enspace\hspace*{\fill}
\finalhyphendemerits=0
[27th November; C gain from Lab]}

\index{Asfordby , Melton@Asfordby, \emph{Melton}}

Death of Trevor Moncrieff (Lab).

\noindent
\begin{tabular*}{\columnwidth}{@{\extracolsep{\fill}} p{0.545\columnwidth} >{\itshape}l r @{\extracolsep{\fill}}}
Ronnie de Burle & C & 265\\
Michael Blase & Lab & 129\\
Sacha Barnes & UKIP & 94\\
\end{tabular*}

\columnbreak

\section{Lincolnshire}

LincsInd = Lincolnshire Independent

\subsection*{County Council}

\subsubsection*{Stamford North \hspace*{\fill}\nolinebreak[1]%
\enspace\hspace*{\fill}
\finalhyphendemerits=0
[11th December; UKIP gain from Ind]}

\index{Stamford North , Lincolnshire@Stamford N., \emph{Lincs.}}

Death of John Hicks (Ind).

\noindent
\begin{tabular*}{\columnwidth}{@{\extracolsep{\fill}} p{0.53\columnwidth} >{\itshape}l r @{\extracolsep{\fill}}}
Robert Foulkes & UKIP & 400\\
Mark Ashberry & Lab & 268\\
Matthew Lee & C & 261\\
Max Sawyer & LincsInd & 199\\
Harrish Bisnauthsing & LD & 142\\
\end{tabular*}

\subsection*{East Lindsey}

\subsubsection*{Horncastle \hspace*{\fill}\nolinebreak[1]%
\enspace\hspace*{\fill}
\finalhyphendemerits=0
[24th April]}

\index{Horncastle , East Lindsey@Horncastle, \emph{E. Lindsey}}

Death of Steve Newton (C).

\noindent
\begin{tabular*}{\columnwidth}{@{\extracolsep{\fill}} p{0.545\columnwidth} >{\itshape}l r @{\extracolsep{\fill}}}
Richard Avison & C & 432\\
David Roark & Ind & 353\\
Mike Beecham & UKIP & 339\\
\end{tabular*}

\subsection*{North Kesteven}

\subsubsection*{Osbournby \hspace*{\fill}\nolinebreak[1]%
\enspace\hspace*{\fill}
\finalhyphendemerits=0
[24th April]}

\index{Osbournby , North Kesteven@Osbournby, \emph{N. Kesteven}}

Death of Jim Cook (Ind).

\noindent
\begin{tabular*}{\columnwidth}{@{\extracolsep{\fill}} p{0.53\columnwidth} >{\itshape}l r @{\extracolsep{\fill}}}
Kate Cook & C & 312\\
Fay Cooper & LincsInd & 269\\
Robert Greetham & Lab & 38\\
Tony Richardson & LD & 9\\
\end{tabular*}

\subsubsection*{\sloppyword{Sleaford Quarrington and Mareham} \hspace*{\fill}\nolinebreak[1]%
\enspace\hspace*{\fill}
\finalhyphendemerits=0
[12th June; LincsInd gain from Ind]}

\index{Sleaford Quarrington and Mareham , North Kesteven@Sleaford Quarrington and Mareham, \emph{N. Kesteven}}

Resignation of Ian Dolby (Ind).

\noindent
\begin{tabular*}{\columnwidth}{@{\extracolsep{\fill}} p{0.53\columnwidth} >{\itshape}l r @{\extracolsep{\fill}}}
Mark Suffield & LincsInd & 527\\
Gary Titmus & C & 477\\
Nigel Gresham & Ind & 178\\
\end{tabular*}

\subsubsection*{Sleaford Westholme \hspace*{\fill}\nolinebreak[1]%
\enspace\hspace*{\fill}
\finalhyphendemerits=0
[12th June; LincsInd gain from Ind]}

\index{Sleaford Westholme , North Kesteven@Sleaford Westholme, \emph{N. Kesteven}}

Resignation of Brian Watson (Ind).

\noindent
\begin{tabular*}{\columnwidth}{@{\extracolsep{\fill}} p{0.53\columnwidth} >{\itshape}l r @{\extracolsep{\fill}}}
Steve Fields & LincsInd & 342\\
Andrew Rayner & C & 119\\
Robert Greetham & Lab & 38\\
\end{tabular*}

\subsection*{South Kesteven}

\subsubsection*{Aveland \hspace*{\fill}\nolinebreak[1]%
\enspace\hspace*{\fill}
\finalhyphendemerits=0
[13th March]}

\index{Aveland , South Kesteven@Aveland, \emph{S. Kesteven}}

Resignation of Debbie Wren (C).

\noindent
\begin{tabular*}{\columnwidth}{@{\extracolsep{\fill}} p{0.545\columnwidth} >{\itshape}l r @{\extracolsep{\fill}}}
Peter Moseley & C & 359\\
John Morgan & Lab & 116\\
\end{tabular*}

\subsection*{West Lindsey}

\subsubsection*{Scotter \hspace*{\fill}\nolinebreak[1]%
\enspace\hspace*{\fill}
\finalhyphendemerits=0
[27th February]}

\index{Scotter , West Lindsey@Scotter, \emph{W. Lindsey}}

Death of William Parry (C).

\noindent
\begin{tabular*}{\columnwidth}{@{\extracolsep{\fill}} p{0.545\columnwidth} >{\itshape}l r @{\extracolsep{\fill}}}
Pat Mewis & C & 577\\
Keith Panter & LD & 301\\
\end{tabular*}

\section{Norfolk}

\subsection*{Broadland}

\subsubsection*{Wroxham \hspace*{\fill}\nolinebreak[1]%
\enspace\hspace*{\fill}
\finalhyphendemerits=0
[20th March]}

\index{Wroxham , Broadland@Wroxham, \emph{Broadland}}

Resignation of Ben McGilvray (LD).

\noindent
\begin{tabular*}{\columnwidth}{@{\extracolsep{\fill}} p{0.545\columnwidth} >{\itshape}l r @{\extracolsep{\fill}}}
Alex Cassam & LD & 482\\
Fran Whymark & C & 341\\
David Moreland & UKIP & 112\\
Malcolm Kemp & Lab & 63\\
\end{tabular*}

\subsubsection*{Wroxham \hspace*{\fill}\nolinebreak[1]%
\enspace\hspace*{\fill}
\finalhyphendemerits=0
[21st August; C gain from LD]}

\index{Wroxham , Broadland@Wroxham, \emph{Broadland}}

Resignation of Steve Buckle (LD).

\noindent
\begin{tabular*}{\columnwidth}{@{\extracolsep{\fill}} p{0.545\columnwidth} >{\itshape}l r @{\extracolsep{\fill}}}
Fran Whymark & C & 400\\
Malcolm Springall & LD & 386\\
Malcolm Kemp & Lab & 103\\
\end{tabular*}

\subsection*{King's Lynn and West Norfolk}

\subsubsection*{Burnham \hspace*{\fill}\nolinebreak[1]%
\enspace\hspace*{\fill}
\finalhyphendemerits=0
[6th March]}

\index{Burnham , King's Lynn and West Norfolk@Burnham, \emph{King's Lynn \& W. Norfolk}}

Death of Garry Sandell (C).

\noindent
\begin{tabular*}{\columnwidth}{@{\extracolsep{\fill}} p{0.545\columnwidth} >{\itshape}l r @{\extracolsep{\fill}}}
Sam Sandell & C & 374\\
Jean Smith & UKIP & 103\\
\end{tabular*}

\subsubsection*{Airfield \hspace*{\fill}\nolinebreak[1]%
\enspace\hspace*{\fill}
\finalhyphendemerits=0
[17th July; C gain from Grn]}

\index{Airfield , King's Lynn and West Norfolk@Airfield, \emph{King's Lynn \& W. Norfolk}}

Resignation of Lori Allen (Grn).

\noindent
\begin{tabular*}{\columnwidth}{@{\extracolsep{\fill}} p{0.545\columnwidth} >{\itshape}l r @{\extracolsep{\fill}}}
Geoff Hipperson & C & 305\\
Bob Scully & UKIP & 233\\
Jonathan Burr & Grn & 72\\
Sebastian Polhill & Lab & 57\\
\end{tabular*}

\subsection*{Norwich}

\subsubsection*{University \hspace*{\fill}\nolinebreak[1]%
\enspace\hspace*{\fill}
\finalhyphendemerits=0
[22nd May]}

\index{University , Norwich@University, \emph{Norwich}}

Resignation of Sarah Grenville (Lab).

Combined with the 2014 ordinary election.

\section{North Yorkshire}

\subsection*{County Council}

\subsubsection*{Skipton West \hspace*{\fill}\nolinebreak[1]%
\enspace\hspace*{\fill}
\finalhyphendemerits=0
[Wednesday 2nd July; Ind gain from LD]}

\index{Skipton West , North Yorkshire@Skipton W., \emph{N. Yorks.}}

Death of Polly English (LD).

\noindent
\begin{tabular*}{\columnwidth}{@{\extracolsep{\fill}} p{0.545\columnwidth} >{\itshape}l r @{\extracolsep{\fill}}}
Andy Solloway &Ind&391\\
Paul Whitaker &C&355\\
Paul English &LD&309\\
Roger Baxandall &UKIP&238\\
Claire Nash &Grn&194\\
Andy Rankine &Lab&181\\
\end{tabular*}

\subsection*{Craven}

\subsubsection*{Skipton West \hspace*{\fill}\nolinebreak[1]%
\enspace\hspace*{\fill}
\finalhyphendemerits=0
[Wednesday 2nd July; Lab gain from LD]}

\index{Skipton West , Craven@Skipton W., \emph{Craven}}

Death of Polly English (LD).

\noindent
\begin{tabular*}{\columnwidth}{@{\extracolsep{\fill}} p{0.545\columnwidth} >{\itshape}l r @{\extracolsep{\fill}}}
Peter Madeley & Lab & 185\\
Edward Walker & LD & 143\\
Tim Hudson-Brunt & C & 131\\
Roger Baxandall & UKIP & 126\\
Bernard Clarke & Ind & 120\\
John Launder & Grn & 67\\
\end{tabular*}

\subsection*{Harrogate}

\subsubsection*{Hookstone \hspace*{\fill}\nolinebreak[1]%
\enspace\hspace*{\fill}
\finalhyphendemerits=0
[17th July]}

\index{Hookstone , Harrogate@Hookstone, \emph{Harrogate}}

Death of Reg Marsh (LD).

\noindent
\begin{tabular*}{\columnwidth}{@{\extracolsep{\fill}} p{0.545\columnwidth} >{\itshape}l r @{\extracolsep{\fill}}}
Clare Skardon & LD & 886\\
Phil Headford & C & 551\\
Alan Henderson & UKIP & 206\\
Pat Foxall & Lab & 71\\
\end{tabular*}

\subsection*{Redcar and Cleveland}

\subsubsection*{Dormanstown \hspace*{\fill}\nolinebreak[1]%
\enspace\hspace*{\fill}
\finalhyphendemerits=0
[22nd May]}

\index{Dormanstown , Redcar and Cleveland@Dormanstown, \emph{Redcar \& Cleveland}}

Resignation of John Earl (LD).

\noindent
\begin{tabular*}{\columnwidth}{@{\extracolsep{\fill}} p{0.545\columnwidth} >{\itshape}l r @{\extracolsep{\fill}}}
Sabrina Thompson & LD & 753\\
Neil Bendelow & Lab & 741\\
Andrea Turner & UKIP & 523\\
\end{tabular*}

\subsection*{Richmondshire}

\subsubsection*{Reeth and Arkengarthdale \hspace*{\fill}\nolinebreak[1]%
\enspace\hspace*{\fill}
\finalhyphendemerits=0
[13th February]}

\index{Reeth and Arkengarthdale , Richmondshire@Reeth \& Arkengarthdale, \emph{Richmondshire}}

Death of Bob Gale (Ind).

\noindent
\begin{tabular*}{\columnwidth}{@{\extracolsep{\fill}} p{0.545\columnwidth} >{\itshape}l r @{\extracolsep{\fill}}}
Richard Beal & Ind & 273\\
Dave Morton & C & 83\\
\end{tabular*}

\subsection*{Scarborough}

\subsubsection*{Derwent Valley \hspace*{\fill}\nolinebreak[1]%
\enspace\hspace*{\fill}
\finalhyphendemerits=0
[22nd May; C gain from Ind]}

\index{Derwent Valley , Scarborough@Derwent Valley, \emph{Scarborough}}

Death of Mick Jay-Hanmer (Ind).

\noindent
\begin{tabular*}{\columnwidth}{@{\extracolsep{\fill}} p{0.545\columnwidth} >{\itshape}l r @{\extracolsep{\fill}}}
Heather Phillips & C & 632\\
Michael James & UKIP & 383\\
Marcus Missen & Lab & 183\\
Robert Lockwood & LD & 109\\
Michael Cutler & Grn & 99\\
David Wright & Ind & 96\\
\end{tabular*}

\subsection*{York}

\subsubsection*{Westfield \hspace*{\fill}\nolinebreak[1]%
\enspace\hspace*{\fill}
\finalhyphendemerits=0
[16th October; LD gain from Lab]}

\index{Westfield , York@Westfield, \emph{York}}

Death of Lynn Jeffries (LD elected as Lab).

\noindent
\begin{tabular*}{\columnwidth}{@{\extracolsep{\fill}} p{0.545\columnwidth} >{\itshape}l r @{\extracolsep{\fill}}}
Andrew Waller & LD & 1804\\
Louise Corson & Lab & 588\\
Judith Morris & UKIP & 398\\
Jason Brown & C & 113\\
Alison Webb & Grn & 87\\
Sam Kelly & EDP & 5\\
\end{tabular*}

\section{Northamptonshire}

\subsection*{County Council}

\subsubsection*{Braunston and Crick \hspace*{\fill}\nolinebreak[1]%
\enspace\hspace*{\fill}
\finalhyphendemerits=0
[3rd July]}

\index{Braunston and Crick , Northamptonshire@Braunston \& Crick, \emph{North\-ants.}}

Resignation of Steven Slatter (C).

\noindent
\begin{tabular*}{\columnwidth}{@{\extracolsep{\fill}} p{0.545\columnwidth} >{\itshape}l r @{\extracolsep{\fill}}}
Malcolm Longley & C & 1019\\
Abigail Campbell & Lab & 989\\
Eric MacAnndrais & UKIP & 506\\
\end{tabular*}

\subsubsection*{Brixworth \hspace*{\fill}\nolinebreak[1]%
\enspace\hspace*{\fill}
\finalhyphendemerits=0
[3rd July]}

\index{Brixworth , Northamptonshire@Brixworth, \emph{Northants.}}

Resignation of Catherine Boardman (C).

\noindent
\begin{tabular*}{\columnwidth}{@{\extracolsep{\fill}} p{0.545\columnwidth} >{\itshape}l r @{\extracolsep{\fill}}}
Cecile Irving-Swift & C & 1297\\
Stephen Pointer & UKIP & 500\\
Robert McNally & Lab & 248\\
Stephen Whiffen & Grn & 228\\
Daniel Jones & LD & 69\\
\end{tabular*}

\subsection*{Northampton}

\subsubsection*{St James \hspace*{\fill}\nolinebreak[1]%
\enspace\hspace*{\fill}
\finalhyphendemerits=0
[3rd July]}

\index{Saint James , Northampton@St James, \emph{Northampton}}

Death of Terry Wire (Lab).

\noindent
\begin{tabular*}{\columnwidth}{@{\extracolsep{\fill}} p{0.545\columnwidth} >{\itshape}l r @{\extracolsep{\fill}}}
Rufia Ashraf & Lab & 307\\
Jill Hope & LD & 262\\
John Howsam & UKIP & 201\\
Andrew Kilbride & C & 198\\
\end{tabular*}

\subsection*{South Northamptonshire}

\subsubsection*{Grange Park (2) \hspace*{\fill}\nolinebreak[1]%
\enspace\hspace*{\fill}
\finalhyphendemerits=0
[2nd October]}

\index{Grange Park , South Northamptonshire@Grange Park, \emph{S. Northants.}}

Resignations of Mark Davidson and Tharik Jainu-Deen (C).

\noindent
\begin{tabular*}{\columnwidth}{@{\extracolsep{\fill}} p{0.545\columnwidth} >{\itshape}l r @{\extracolsep{\fill}}}
Simon Clifford & C & 433\\
Adil Sadygov & C & 313\\
Ian Grant & Lab & 151\\
Peter Conquest & UKIP & 100\\
Katie Chick & UKIP & 84\\
\end{tabular*}

\section{Northumberland}

\subsubsection*{Longhoughton \hspace*{\fill}\nolinebreak[1]%
\enspace\hspace*{\fill}
\finalhyphendemerits=0
[24th July; LD gain from Ind]}

\index{Longhoughton , Northumberland@Longhoughton, \emph{Northd.}}

Death of John Taylor (Ind).

\noindent
\begin{tabular*}{\columnwidth}{@{\extracolsep{\fill}} p{0.545\columnwidth} >{\itshape}l r @{\extracolsep{\fill}}}
Kate Cairns & LD & 742\\
John Hope & C & 352\\
Wendy Pattison & Ind & 208\\
Michael Weatheritt & UKIP & 146\\
Nicola Morrison & Lab & 48\\
\end{tabular*}

\section{Nottinghamshire}

\subsection*{County Council}

\subsubsection*{Ollerton \hspace*{\fill}\nolinebreak[1]%
\enspace\hspace*{\fill}
\finalhyphendemerits=0
[18th December]}

\index{Ollerton , Nottinghamshire@Ollerton, \emph{Notts.}}

Death of Stella Smedley (Lab).

\noindent
\begin{tabular*}{\columnwidth}{@{\extracolsep{\fill}} p{0.545\columnwidth} >{\itshape}l r @{\extracolsep{\fill}}}
Michael Pringle & Lab & 1171\\
Ben Bradley & C & 533\\
Colin Hart & UKIP & 347\\
Marylyn Rayner & LD & 24\\
\end{tabular*}

\subsection*{Broxtowe}

\subsubsection*{Toton and Chilwell Meadows \hspace*{\fill}\nolinebreak[1]%
\enspace\hspace*{\fill}
\finalhyphendemerits=0
[11th December]}

\index{Toton and Chilwell Meadows , Broxtowe@Toton \& Chilwell Meadows, \emph{Broxtowe}}

Death of Marilyn Hegyi (C).

\noindent
\begin{tabular*}{\columnwidth}{@{\extracolsep{\fill}} p{0.545\columnwidth} >{\itshape}l r @{\extracolsep{\fill}}}
Mia Kee & C & 952\\
David Patrick & Lab & 454\\
Darryl Paxford & UKIP & 340\\
\end{tabular*}

\subsection*{Gedling}

\subsubsection*{Gedling \hspace*{\fill}\nolinebreak[1]%
\enspace\hspace*{\fill}
\finalhyphendemerits=0
[27th March; Lab gain from LD]}

\index{Gedling , Gedling@Gedling, \emph{Gedling}}

Resignation of Gordon Tunnicliffe (LD).

\noindent
\begin{tabular*}{\columnwidth}{@{\extracolsep{\fill}} p{0.545\columnwidth} >{\itshape}l r @{\extracolsep{\fill}}}
Lynda Pearson & Lab & 482\\
Maggie Dunkin & LD & 428\\
Claude-Francois Loi & UKIP & 337\\
James Faulconbridge & C & 233\\
\end{tabular*}

\subsection*{Mansfield}

MIF = Mansfield Independent Forum

\subsubsection*{Netherfield \hspace*{\fill}\nolinebreak[1]%
\enspace\hspace*{\fill}
\finalhyphendemerits=0
[4th December; Lab gain from MIF]}

\index{Netherfield , Mansfield@Netherfield, \emph{Mansfield}}

Death of Derek Evans (MIF).

\noindent
\begin{tabular*}{\columnwidth}{@{\extracolsep{\fill}} p{0.545\columnwidth} >{\itshape}l r @{\extracolsep{\fill}}}
Lesley Wright & Lab & 347\\
Sid Walker & UKIP & 225\\
Karen Seymour & TUSC & 29\\
\end{tabular*}

\subsection*{Newark and Sherwood}

\subsubsection*{Collingham and Meering \hspace*{\fill}\nolinebreak[1]%
\enspace\hspace*{\fill}
\finalhyphendemerits=0
[11th September]}

\index{Collingham and Meering , Newark and Sherwood@Collingham and Meering, \emph{Newark \& Sherwood}}

Death of Derek Evans (C).

\noindent
\begin{tabular*}{\columnwidth}{@{\extracolsep{\fill}} p{0.545\columnwidth} >{\itshape}l r @{\extracolsep{\fill}}}
Richard Shillito & C & 568\\
David Clarke & Ind & 476\\
Sara Chadd & UKIP & 218\\
Kieran Owen & Lab & 118\\
\end{tabular*}

\subsubsection*{Ollerton \hspace*{\fill}\nolinebreak[1]%
\enspace\hspace*{\fill}
\finalhyphendemerits=0
[11th September]}

\index{Ollerton , Newark and Sherwood@Ollerton, \emph{Newark \& Sherwood}}

Death of Stan Crawford (Lab).

\noindent
\begin{tabular*}{\columnwidth}{@{\extracolsep{\fill}} p{0.545\columnwidth} >{\itshape}l r @{\extracolsep{\fill}}}
Michael Pringle & Lab & 837\\
Mary Brown & C & 323\\
Moritz Dawkins & UKIP & 280\\
\end{tabular*}

\subsection*{Nottingham}

Elvis = Bus-Pass Elvis Party

\subsubsection*{Clifton North \hspace*{\fill}\nolinebreak[1]%
\enspace\hspace*{\fill}
\finalhyphendemerits=0
[6th March]}

\index{Clifton North , Nottingham@Clifton N., \emph{Nottingham}}

Disqualification of Lee Jeffery (Lab) for non-attendance.

\noindent
\begin{tabular*}{\columnwidth}{@{\extracolsep{\fill}} p{0.545\columnwidth} >{\itshape}l r @{\extracolsep{\fill}}}
Patricia Ferguson & Lab & 1179\\
Andrew Rule & C & 1025\\
Kevin Clarke & UKIP & 536\\
David Bishop & Elvis & 67\\
Tony Marshall & LD & 56\\
\end{tabular*}

\subsection*{Rushcliffe}

\subsubsection*{Gamston \hspace*{\fill}\nolinebreak[1]%
\enspace\hspace*{\fill}
\finalhyphendemerits=0
[20th March]}

\index{Gamston , Rushcliffe@Gamston, \emph{Rushcliffe}}

Death of Mike Hemsley (C).

\noindent
\begin{tabular*}{\columnwidth}{@{\extracolsep{\fill}} p{0.545\columnwidth} >{\itshape}l r @{\extracolsep{\fill}}}
Jonathan Wheeler & C & 444\\
Alan Hardwick & Lab & 218\\
Matthew Faithfull & UKIP & 173\\
Davinder Virdi & LD & 170\\
\end{tabular*}

\section{Oxfordshire}

\subsection*{County Council}

\subsubsection*{Chalgrove and Watlington \hspace*{\fill}\nolinebreak[1]%
\enspace\hspace*{\fill}
\finalhyphendemerits=0
[27th March]}

\index{Chalgrove and Watlington , Oxfordshire@Chalgrove \& Watlington, \emph{Oxon.}}

Resignation of Caroline Newton (C).

\noindent
\begin{tabular*}{\columnwidth}{@{\extracolsep{\fill}} p{0.545\columnwidth} >{\itshape}l r @{\extracolsep{\fill}}}
Stephen Harrod & C & 871\\
Susan Cooper & LD & 629\\
Craig Laird & UKIP & 311\\
Paul Collins & Lab & 159\\
Colin Tudge & Grn & 116\\
\end{tabular*}

\subsubsection*{Leys \hspace*{\fill}\nolinebreak[1]%
\enspace\hspace*{\fill}
\finalhyphendemerits=0
[27th November]}

\index{Leys , Oxfordshire@Leys, \emph{Oxon.}}

Resignation of Val Smith (Lab).

\noindent
\begin{tabular*}{\columnwidth}{@{\extracolsep{\fill}} p{0.545\columnwidth} >{\itshape}l r @{\extracolsep{\fill}}}
Steve Curran & Lab & 879\\
Dave Slater & UKIP & 168\\
Samuel Burgess & C & 88\\
Ann Duncan & Grn & 57\\
Lesley Mallinder & LD & 30\\
James Morbin & TUSC & 27\\
\end{tabular*}

\subsection*{Oxford}

\subsubsection*{Summertown \hspace*{\fill}\nolinebreak[1]%
\enspace\hspace*{\fill}
\finalhyphendemerits=0
[22nd May]}

\index{Summertown , Oxford@Summertown, \emph{Oxford}}

Resignation of Stuart McCready (LD).

Combined with the 2014 ordinary election.

\subsubsection*{Cowley \hspace*{\fill}\nolinebreak[1]%
\enspace\hspace*{\fill}
\finalhyphendemerits=0
[17th July]}

\index{Cowley , Oxford@Cowley, \emph{Oxford}}

Resignation of Helen O'Hara (Lab).

\noindent
\begin{tabular*}{\columnwidth}{@{\extracolsep{\fill}} p{0.545\columnwidth} >{\itshape}l r @{\extracolsep{\fill}}}
David Henwood & Lab & 512\\
Hazel Dawe & Grn & 269\\
Artwell & Ind & 257\\
Katharine Harborne & C & 152\\
Ian Macdonald & UKIP & 72\\
Prakash Sharma & LD & 39\\
\end{tabular*}

\subsubsection*{Carfax \hspace*{\fill}\nolinebreak[1]%
\enspace\hspace*{\fill}
\finalhyphendemerits=0
[4th September]}

\index{Carfax , Oxford@Carfax, \emph{Oxford}}

Resignation of Anne-Marie Canning (Lab).

\noindent
\begin{tabular*}{\columnwidth}{@{\extracolsep{\fill}} p{0.545\columnwidth} >{\itshape}l r @{\extracolsep{\fill}}}
Alex Hollingsworth & Lab & 168\\
Tony Brett & LD & 101\\
Richard Scrase & Grn & 63\\
Maryam Ahmed & C & 24\\
Kenrick Bird & UKIP & 24\\
\end{tabular*}

\subsubsection*{Quarry and Risinghurst \hspace*{\fill}\nolinebreak[1]%
\enspace\hspace*{\fill}
\finalhyphendemerits=0
[18th September]}

\index{Quarry and Risinghurst , Oxford@Quarry \& Risinghurst, \emph{Oxford}}

Resignation of Laurence Baxter (Lab).

\noindent
\begin{tabular*}{\columnwidth}{@{\extracolsep{\fill}} p{0.545\columnwidth} >{\itshape}l r @{\extracolsep{\fill}}}
Chewe Munkonge & Lab & 782\\
Roz Smith & LD & 615\\
Katharine Harborne & C & 222\\
Liz Taylor & Grn & 186\\
Julia Gasper & EDP & 43\\
\end{tabular*}

\subsubsection*{Blackbird Leys \hspace*{\fill}\nolinebreak[1]%
\enspace\hspace*{\fill}
\finalhyphendemerits=0
[27th November]}

\index{Blackbird Leys , Oxford@Blackbird Leys, \emph{Oxford}}

Resignation of Val Smith (Lab).

\noindent
\begin{tabular*}{\columnwidth}{@{\extracolsep{\fill}} p{0.545\columnwidth} >{\itshape}l r @{\extracolsep{\fill}}}
Linda Smith & Lab & 509\\
Dave Slater & UKIP & 91\\
Berk Bektas & C & 27\\
Elizabeth McHale & Grn & 21\\
Stella Collier & TUSC & 13\\
Lesley Mallinder & LD & 11\\
\end{tabular*}

\subsubsection*{Northfield Brook \hspace*{\fill}\nolinebreak[1]%
\enspace\hspace*{\fill}
\finalhyphendemerits=0
[27th November]}

\index{Northfield Brook , Oxford@Northfield Brook, \emph{Oxford}}

Resignation of Steve Curran (Lab).

\noindent
\begin{tabular*}{\columnwidth}{@{\extracolsep{\fill}} p{0.545\columnwidth} >{\itshape}l r @{\extracolsep{\fill}}}
Sian Taylor & Lab & 401\\
Gary Dixon & C & 65\\
Ann Duncan & Grn & 50\\
James Morbin & TUSC & 34\\
Michael Tait & LD & 18\\
\end{tabular*}

\subsection*{Vale of White Horse}

\subsubsection*{Wantage Charlton \hspace*{\fill}\nolinebreak[1]%
\enspace\hspace*{\fill}
\finalhyphendemerits=0
[10th April]}

\index{Wantage Charlton , Vale of White Horse@Wantage Charlton, \emph{Vale of White Horse}}

Disqualification (sentenced to five years' imprisonment, theft) of John Morgan (C).

\noindent
\begin{tabular*}{\columnwidth}{@{\extracolsep{\fill}} p{0.545\columnwidth} >{\itshape}l r @{\extracolsep{\fill}}}
Julia Reynolds & C & 591\\
Jim Sibbald & LD & 542\\
Nathan Sparks & Lab & 155\\
Kevin Harris & Grn & 124\\
\end{tabular*}

\subsubsection*{Abingdon Dunmore \hspace*{\fill}\nolinebreak[1]%
\enspace\hspace*{\fill}
\finalhyphendemerits=0
[11th September]}

\index{Abingdon Dunmore , Vale of White Horse@Abingdon Dunmore, \emph{Vale of White Horse}}

Resignation of Julia Bricknell (LD).

\noindent
\begin{tabular*}{\columnwidth}{@{\extracolsep{\fill}} p{0.545\columnwidth} >{\itshape}l r @{\extracolsep{\fill}}}
Margaret Crick &LD&745\\
Andrew Todd&C&501\\
Christopher Parkes &UKIP&90\\
Mike Gould &Lab&87\\
\end{tabular*}

\subsection*{West Oxfordshire}

At the May 2014 ordinary election there was an unfilled vacancy in Ducklington ward due to the resignation of Steve Hayward (Ind elected as C).
\index{Ducklington , West Oxfordshire@Ducklington, \emph{W. Oxon.}}

\subsubsection*{Stonesfield and Tackley \hspace*{\fill}\nolinebreak[1]%
\enspace\hspace*{\fill}
\finalhyphendemerits=0
[22nd May]}

\index{Stonesfield and Tackley , West Oxfordshire@Stonesfield \& Tackley, \emph{W. Oxon.}}

Resignation of Charles Cottrell-Dormer (C).

Combined with the 2014 ordinary election.

\section{Rutland}

\subsubsection*{Oakham South West \hspace*{\fill}\nolinebreak[1]%
\enspace\hspace*{\fill}
\finalhyphendemerits=0
[16th October]}

\index{Oakham South West , Rutland@Oakham S.W., \emph{Rutland}}

Resignation of Joanne Figgis (C).

\noindent
\begin{tabular*}{\columnwidth}{@{\extracolsep{\fill}} p{0.545\columnwidth} >{\itshape}l r @{\extracolsep{\fill}}}
Richard Clifton & C & 240\\
Ben Callaghan & Ind & 177\\
Richard Swift & LD & 43\\
\end{tabular*}

\subsubsection*{Whissendine \hspace*{\fill}\nolinebreak[1]%
\enspace\hspace*{\fill}
\finalhyphendemerits=0
[16th October; LD gain from C]}

\index{Whissendine , Rutland@Whissendine, \emph{Rutland}}

Death of Brian Montgomery (Ind).

\noindent
\begin{tabular*}{\columnwidth}{@{\extracolsep{\fill}} p{0.545\columnwidth} >{\itshape}l r @{\extracolsep{\fill}}}
Sam Asplin & LD & 192\\
Jonny Baker & C & 179\\
\end{tabular*}

\section{Shropshire}

\subsection*{Shropshire}

\subsubsection*{Ludlow North \hspace*{\fill}\nolinebreak[1]%
\enspace\hspace*{\fill}
\finalhyphendemerits=0
[13th March; LD gain from C]}

\index{Ludlow North , Shropshire@Ludlow N., \emph{Salop.}}

Resignation of Rosanna Taylor-Smith (C).

\noindent
\begin{tabular*}{\columnwidth}{@{\extracolsep{\fill}} p{0.545\columnwidth} >{\itshape}l r @{\extracolsep{\fill}}}
Andy Boddington & LD & 579\\
Anthony Bevington & C & 382\\
Graeme Perks & Ind & 223\\
Danny Sweeney & Lab & 94\\
\end{tabular*}

\subsection*{Telford and Wrekin}

\subsubsection*{Ironbridge Gorge \hspace*{\fill}\nolinebreak[1]%
\enspace\hspace*{\fill}
\finalhyphendemerits=0
[30th October]}

\index{Ironbridge Gorge , Telford and Wrekin@Ironbridge Gorge, \emph{Telford \& Wrekin}}

Death of Dave Davies (Lab).

\noindent
\begin{tabular*}{\columnwidth}{@{\extracolsep{\fill}} p{0.545\columnwidth} >{\itshape}l r @{\extracolsep{\fill}}}
Ken Stringer & Lab & 325\\
Elizabeth Mollett & C & 276\\
Richard Soame & UKIP & 136\\
\end{tabular*}

\subsubsection*{Newport West \hspace*{\fill}\nolinebreak[1]%
\enspace\hspace*{\fill}
\finalhyphendemerits=0
[30th October; Ind gain from C]}

\index{Newport West , Telford and Wrekin@Newport W., \emph{Telford and Wrekin}}

Resignation of Adam Stanton (C).

\noindent
\begin{tabular*}{\columnwidth}{@{\extracolsep{\fill}} p{0.545\columnwidth} >{\itshape}l r @{\extracolsep{\fill}}}
Peter Scott & Ind & 264\\
Rodney Pitt & C & 179\\
Warwick McKenzie & UKIP & 157\\
Phil Norton & Lab & 63\\
\end{tabular*}

\section{Somerset}

\subsection*{County Council}

\subsubsection*{Frome North \hspace*{\fill}\nolinebreak[1]%
\enspace\hspace*{\fill}
\finalhyphendemerits=0
[25th September; C gain from LD]}

\index{Frome North , Somerset@Frome N., \emph{Somerset}}

Resignation of Sam Phripp (LD).

\noindent
\begin{tabular*}{\columnwidth}{@{\extracolsep{\fill}} p{0.545\columnwidth} >{\itshape}l r @{\extracolsep{\fill}}}
Linda Oliver & C & 1163\\
Damon Hooton & LD & 836\\
Catherine Richardson & Lab & 163\\
Adrian Dobinson & Ind & 139\\
Les Spalding & Grn & 139\\
\end{tabular*}

\subsection*{Bath and North East Somerset}

\subsubsection*{Bathavon North \hspace*{\fill}\nolinebreak[1]%
\enspace\hspace*{\fill}
\finalhyphendemerits=0
[22nd May]}

\index{Bathavon North , Bath and North East Somerset@Bathavon N., \emph{Bath \& N.E. Somerset}}

Death of Gabriel Batt (C).

\noindent
\begin{tabular*}{\columnwidth}{@{\extracolsep{\fill}} p{0.545\columnwidth} >{\itshape}l r @{\extracolsep{\fill}}}
Terry Gazzard & C & 1047\\
Dorian Baker & LD & 712\\
George Aylett & Lab & 353\\
Justin Temblett-Wood & Grn & 353\\
Hugo Jenks & UKIP & 316\\
\end{tabular*}

\section{Staffordshire}

\subsection*{Lichfield}

\subsubsection*{Chadsmead \hspace*{\fill}\nolinebreak[1]%
\enspace\hspace*{\fill}
\finalhyphendemerits=0
[30th January; LD gain from C]}

\index{Chadsmead , Lichfield@Chadsmead, \emph{Lichfield}}

Disqualification (non-attendance) of Michael Fryers (C).

\noindent
\begin{tabular*}{\columnwidth}{@{\extracolsep{\fill}} p{0.545\columnwidth} >{\itshape}l r @{\extracolsep{\fill}}}
Marion Bland & LD & 206\\
Caroline Wood & Lab & 157\\
Bob Green & UKIP & 108\\
Jon O'Hagan & C & 102\\
\end{tabular*}

\subsection*{Newcastle-under-Lyme}

\subsubsection*{Town \hspace*{\fill}\nolinebreak[1]%
\enspace\hspace*{\fill}
\finalhyphendemerits=0
[22nd May]}

\index{Town , Newcastle-under-Lyme@Town, \emph{Newcastle-u.-Lyme}}

Resignation of Matt Taylor (Lab).

Combined with the 2014 ordinary election.

\subsection*{Staffordshire Moorlands}

StaffsIG = Staffordshire Independent Group

\subsubsection*{Cellarhead \hspace*{\fill}\nolinebreak[1]%
\enspace\hspace*{\fill}
\finalhyphendemerits=0
[20th March; C gain from Ind]}

\index{Cellarhead , Staffordshire Moorlands@Cellarhead, \emph{Staffs. Moorlands}}

Death of May Day (Ind).

\noindent
\begin{tabular*}{\columnwidth}{@{\extracolsep{\fill}} p{0.545\columnwidth} >{\itshape}l r @{\extracolsep{\fill}}}
Barbara Hughes & C & 178\\
Jocelyn Morrison & Lab & 132\\
Jean Hodgetts & StaffsIG & 119\\
Alex Povey & UKIP & 105\\
Phil Routledge & LD & 13\\
\end{tabular*}

\section{Suffolk}

\subsection*{Babergh}

\subsubsection*{South Cosford \hspace*{\fill}\nolinebreak[1]%
\enspace\hspace*{\fill}
\finalhyphendemerits=0
[22nd May; Grn gain from C]}

\index{South Cosford , Babergh@South Cosford, \emph{Babergh}}

Resignation of Dawn Kendall (C).

\noindent
\begin{tabular*}{\columnwidth}{@{\extracolsep{\fill}} p{0.545\columnwidth} >{\itshape}l r @{\extracolsep{\fill}}}
Robert Lindsay & Grn & 346\\
David Talbot Clarke & C & 330\\
Stephen Laing & UKIP & 219\\
Angela Wiltshire & Lab & 72\\
\end{tabular*}

\subsection*{Ipswich}

\subsubsection*{Alexandra \hspace*{\fill}\nolinebreak[1]%
\enspace\hspace*{\fill}
\finalhyphendemerits=0
[22nd May]}

\index{Alexandra , Ipswich@Alexandra, \emph{Ipswich}}

Resignation of Harvey Crane (Lab).

Combined with the 2014 ordinary election.

\subsection*{Mid Suffolk}

\subsubsection*{Stowmarket North \hspace*{\fill}\nolinebreak[1]%
\enspace\hspace*{\fill}
\finalhyphendemerits=0
[22nd May]}

\index{Stowmarket North , Mid Suffolk@Stowmarket N., \emph{Mid Suffolk}}

Resignation of Frank Whittle (UKIP elected as C).

\noindent
\begin{tabular*}{\columnwidth}{@{\extracolsep{\fill}} p{0.545\columnwidth} >{\itshape}l r @{\extracolsep{\fill}}}
Barry Humphreys & C & 707\\
Stephen Searle & UKIP & 616\\
Nigel Rozier & Grn & 444\\
Anthony Elliott & Lab & 315\\
Nichola Willshere & LD & 144\\
\end{tabular*}

\subsection*{St Edmundsbury}

\subsubsection*{Haverhill East \hspace*{\fill}\nolinebreak[1]%
\enspace\hspace*{\fill}
\finalhyphendemerits=0
[9th January; UKIP gain from C]}

\index{Haverhill East , Saint Edmundsbury@Haverhill E., \emph{St Edmundsbury}}

Death of Les Ager (C).

\noindent
\begin{tabular*}{\columnwidth}{@{\extracolsep{\fill}} p{0.545\columnwidth} >{\itshape}l r @{\extracolsep{\fill}}}
Tony Brown & UKIP & 529\\
Pat Hanlon & Lab & 240\\
David Roach & C & 157\\
Ken Rolph & LD & 54\\
\end{tabular*}

\subsection*{Waveney}

\subsubsection*{Halesworth \hspace*{\fill}\nolinebreak[1]%
\enspace\hspace*{\fill}
\finalhyphendemerits=0
[22nd May]}

\index{Halesworth , Waveney@Halesworth, \emph{Waveney}}

Resignation of Patricia Flegg (C).

\noindent
\begin{tabular*}{\columnwidth}{@{\extracolsep{\fill}} p{0.545\columnwidth} >{\itshape}l r @{\extracolsep{\fill}}}
Letitia Smith & C & 726\\
Tobias Walton & Lab & 535\\
Jennifer Berry & Grn & 245\\
Jack Tyler & Ind & 213\\
\end{tabular*}

\section{Surrey}

\subsection*{Guildford}

\subsubsection*{Lovelace \hspace*{\fill}\nolinebreak[1]%
\enspace\hspace*{\fill}
\finalhyphendemerits=0
[25th September; LD gain from C]}

\index{Lovelace , Guildford@Lovelace, \emph{Guildford}}

Death of John Garrett (C).

\noindent
\begin{tabular*}{\columnwidth}{@{\extracolsep{\fill}} p{0.545\columnwidth} >{\itshape}l r @{\extracolsep{\fill}}}
Colin Cross & LD & 555\\
Ben Paton & C & 255\\
David Sheppard & UKIP & 63\\
Robin Woof & Lab & 32\\
\end{tabular*}

\subsection*{Mole Valley}

At the May 2014 ordinary election there was an unfilled vacancy in Ashtead Park ward due to the resignation of Richard Brooke (Ashtead Inds).
\index{Ashtead Park , Mole Valley@Ashtead Park, \emph{Mole Valley}}

\subsection*{Runnymede}

\subsubsection*{Chertsey Meads \hspace*{\fill}\nolinebreak[1]%
\enspace\hspace*{\fill}
\finalhyphendemerits=0
[13th March]}

\index{Chertsey Meads , Runnymede@Chertsey Meads, \emph{Runnymede}}

Death of Peter Boast (C).

\noindent
\begin{tabular*}{\columnwidth}{@{\extracolsep{\fill}} p{0.545\columnwidth} >{\itshape}l r @{\extracolsep{\fill}}}
Mark Nuti & C & 489\\
David Bell & Lab & 329\\
Grahame Leon-Smith & UKIP & 327\\
Crazy Crab & Loony & 15\\
\end{tabular*}

\subsection*{Surrey Heath}

\subsubsection*{Old Dean \hspace*{\fill}\nolinebreak[1]%
\enspace\hspace*{\fill}
\finalhyphendemerits=0
[4th September]}

\index{Old Dean , Surrey Heath@Old Dean, \emph{Surrey Heath}}

Death of Margaret Moher (Lab).

\noindent
\begin{tabular*}{\columnwidth}{@{\extracolsep{\fill}} p{0.545\columnwidth} >{\itshape}l r @{\extracolsep{\fill}}}
Heather Gerred & Lab & 290\\
Max Nelson & C & 196\\
Eddie Hill & UKIP & 171\\
\end{tabular*}

\section{Warwickshire}

\subsection*{County Council}

\subsubsection*{Hartshill \hspace*{\fill}\nolinebreak[1]%
\enspace\hspace*{\fill}
\finalhyphendemerits=0
[22nd May]}

\index{Hartshill , Warwickshire@Hartshill, \emph{Warks.}}

Resignation of Ann McLauchlan (Lab).

\noindent
\begin{tabular*}{\columnwidth}{@{\extracolsep{\fill}} p{0.545\columnwidth} >{\itshape}l r @{\extracolsep{\fill}}}
Chris Clark & Lab & 738\\
Bella Wayte & UKIP & 699\\
David Wright & 616\\
Carol Fox & Ind & 190\\
\end{tabular*}

\columnbreak

\subsection*{Rugby}

\subsubsection*{Bilton \hspace*{\fill}\nolinebreak[1]%
\enspace\hspace*{\fill}
\finalhyphendemerits=0
[6th November]}

\index{Bilton , Rugby@Bilton, \emph{Rugby}}

Resignation of Craig Humphrey (C).

\noindent
\begin{tabular*}{\columnwidth}{@{\extracolsep{\fill}} p{0.545\columnwidth} >{\itshape}l r @{\extracolsep{\fill}}}
Julie A'Barrow & C & 668\\
Gordon Davies & UKIP & 325\\
Lesley George & LD & 280\\
John Wells & Lab & 212\\
John Herman & Ind & 60\\
Kate Crowley & Grn & 37\\
Pete McLaren & TUSC & 10\\
\end{tabular*}

\subsection*{Stratford-on-Avon}

At the May 2014 ordinary election there was an unfilled vacancy in Long Itchington ward due to the resignation of Chris Spencer (C).
\index{Long Itchington , Stratford-on-Avon@Long Itchington, \emph{Stratford-on-Avon}}

\subsubsection*{Southam \hspace*{\fill}\nolinebreak[1]%
\enspace\hspace*{\fill}
\finalhyphendemerits=0
[19th June]}

\index{Southam , Stratford-on-Avon@Southam, \emph{Stratford-on-Avon}}

Resignation of David Wise (Ind elected as C).

\noindent
\begin{tabular*}{\columnwidth}{@{\extracolsep{\fill}} p{0.545\columnwidth} >{\itshape}l r @{\extracolsep{\fill}}}
Tony Bromwich & C & 493\\
Bransby Thomas & Lab & 398\\
Emily Bleloch & UKIP & 259\\
\end{tabular*}

\section{West Sussex}

\subsection*{Adur}

\subsubsection*{Mash Barn \hspace*{\fill}\nolinebreak[1]%
\enspace\hspace*{\fill}
\finalhyphendemerits=0
[22nd May]}

\index{Mash Barn , Adur@Mash Barn, \emph{Adur}}

Resignation of Andrew Barnes (C).

Combined with the 2014 ordinary election.

\subsubsection*{St Mary's \hspace*{\fill}\nolinebreak[1]%
\enspace\hspace*{\fill}
\finalhyphendemerits=0
[Friday 5th December]}

\index{Saint Mary's , Adur@St Mary's, \emph{Adur}}

Resignation of Mike Mendoza (C).

\noindent
\begin{tabular*}{\columnwidth}{@{\extracolsep{\fill}} p{0.545\columnwidth} >{\itshape}l r @{\extracolsep{\fill}}}
Stephen Chipp & C & 340\\
Irene Reed & Lab & 223\\
Jenny Greig & UKIP & 216\\
Jennie Tindall & Grn & 106\\
\end{tabular*}

\subsection*{Chichester}

\subsubsection*{Rogate \hspace*{\fill}\nolinebreak[1]%
\enspace\hspace*{\fill}
\finalhyphendemerits=0
[23rd October]}

\index{Rogate , Chichester@Rogate, \emph{Chichester}}

Death of John Kingston (C).

\noindent
\begin{tabular*}{\columnwidth}{@{\extracolsep{\fill}} p{0.545\columnwidth} >{\itshape}l r @{\extracolsep{\fill}}}
Gillian Keegan & C & 342\\
Elena McCloskey & UKIP & 138\\
\end{tabular*}

\subsection*{Crawley}

Justice = Justice Party

\subsubsection*{Pound Hill South and Worth \hspace*{\fill}\nolinebreak[1]%
\enspace\hspace*{\fill}
\finalhyphendemerits=0
[22nd May]}

\index{Pound Hill South and Worth , Crawley@Pound Hill S. \& Worth, \emph{Crawley}}

Resignation of Claire Denman (C).

Combined with the 2014 ordinary election.

\subsubsection*{Southgate \hspace*{\fill}\nolinebreak[1]%
\enspace\hspace*{\fill}
\finalhyphendemerits=0
[9th October; Lab gain from C]}

\index{Southgate , Crawley@Southgate, \emph{Crawley}}

Resignation of Karl Southgate (UKIP elected as C).

\noindent
\begin{tabular*}{\columnwidth}{@{\extracolsep{\fill}} p{0.545\columnwidth} >{\itshape}l r @{\extracolsep{\fill}}}
Michael Pickett & Lab & 733\\
Jan Tarrant & C & 642\\
Simon Darroch & UKIP & 277\\
Arshad Khan & Justice & 10\\
\end{tabular*}

\subsection*{Mid Sussex}

\subsubsection*{Haywards Heath Lucastes \hspace*{\fill}\nolinebreak[1]%
\enspace\hspace*{\fill}
\finalhyphendemerits=0
[23rd October]}

\index{Haywards Heath Lucastes , Mid Sussex@Haywards Heath Lucastes, \emph{M. Sussex}}

Death of Tim Farmer (C).

\noindent
\begin{tabular*}{\columnwidth}{@{\extracolsep{\fill}} p{0.545\columnwidth} >{\itshape}l r @{\extracolsep{\fill}}}
Geoffrey Rawlinson & C & 524\\
Marc Montgomery & UKIP & 203\\
Nicholas Chapman & LD & 112\\
Henry Fowler & Lab & 90\\
\end{tabular*}

\subsubsection*{Bolney \hspace*{\fill}\nolinebreak[1]%
\enspace\hspace*{\fill}
\finalhyphendemerits=0
[13th November]}

\index{Bolney , Mid Sussex@Bolney, \emph{M. Sussex}}

Resignation of Sue Seward (C).

\noindent
\begin{tabular*}{\columnwidth}{@{\extracolsep{\fill}} p{0.545\columnwidth} >{\itshape}l r @{\extracolsep{\fill}}}
John Allen & C & 261\\
Anthony Williams & UKIP & 187\\
Simon Hicks & LD & 161\\
\end{tabular*}

\subsection*{Worthing}

At the May 2014 ordinary election there was an unfilled vacancy in Marine ward due to the resignation of Tom Wye (C).
\index{Marine , Worthing@Marine, \emph{Worthing}}

\subsubsection*{Castle \hspace*{\fill}\nolinebreak[1]%
\enspace\hspace*{\fill}
\finalhyphendemerits=0
[7th August; UKIP gain from LD]}

\index{Castle , Worthing@Castle, \emph{Worthing}}

Disqualification (non-attendance) of David Potter (LD).

\noindent
\begin{tabular*}{\columnwidth}{@{\extracolsep{\fill}} p{0.545\columnwidth} >{\itshape}l r @{\extracolsep{\fill}}}
Charles James & UKIP & 568\\
Alex Harman & C & 485\\
Nicholas Wiltshire & LD & 242\\
Jim Deen & Lab & 197\\
Stefan Sykes & Grn & 49\\
\end{tabular*}

\section{Wiltshire}

\subsection*{Swindon}

\subsubsection*{St Andrews \hspace*{\fill}\nolinebreak[1]%
\enspace\hspace*{\fill}
\finalhyphendemerits=0
[22nd May]}

\index{Saint Andrews , Swindon@St Andrews, \emph{Swindon}}

Resignation of Peter Heaton-Jones (C).

Combined with the 2014 ordinary election.

\subsection*{Wiltshire}

\subsubsection*{Ethandune \hspace*{\fill}\nolinebreak[1]%
\enspace\hspace*{\fill}
\finalhyphendemerits=0
[6th March]}

\index{Ethandune , Wiltshire@Ethandune, \emph{Wilts.}}

Death of Linda Conley (C).

\noindent
\begin{tabular*}{\columnwidth}{@{\extracolsep{\fill}} p{0.545\columnwidth} >{\itshape}l r @{\extracolsep{\fill}}}
Jerry Wickham & C & 480\\
Carole King & LD & 372\\
Rod Eaton & UKIP & 236\\
Francis Morland & Ind & 192\\
Shaun Henley & Lab & 69\\
\end{tabular*}

\section{Worcestershire}

\subsection*{County Council}

\subsubsection*{Arrow Valley East \hspace*{\fill}\nolinebreak[1]%
\enspace\hspace*{\fill}
\finalhyphendemerits=0
[22nd May]}

\index{Arrow Valley East , Worcestershire@Arrow Valley E., \emph{Worcs.}}

Resignation of Martin Jenkins (UKIP).

\noindent
\begin{tabular*}{\columnwidth}{@{\extracolsep{\fill}} p{0.545\columnwidth} >{\itshape}l r @{\extracolsep{\fill}}}
Peter Bridle & UKIP & 2017\\
Phil Mould & Lab & 1601\\
Juliet Brunner & C & 1448\\
Simon Oliver & LD & 286\\
Emma Bradley & Grn & 241\\
Isabel Armstrong & Ind & 133\\
\end{tabular*}

\subsection*{Malvern Hills}

\subsubsection*{Lindridge \hspace*{\fill}\nolinebreak[1]%
\enspace\hspace*{\fill}
\finalhyphendemerits=0
[22nd May]}

\index{Lindridge , Malvern Hills@Lindridge, \emph{Malvern Hills}}

Resignation of William Redman (C).

\noindent
\begin{tabular*}{\columnwidth}{@{\extracolsep{\fill}} p{0.545\columnwidth} >{\itshape}l r @{\extracolsep{\fill}}}
Chris Dell & C & 437\\
Andrew Dolan & UKIP & 229\\
Simon Gill & LD & 128\\
\end{tabular*}

\subsubsection*{Wells \hspace*{\fill}\nolinebreak[1]%
\enspace\hspace*{\fill}
\finalhyphendemerits=0
[7th August]}

\index{Wells , Malvern Hills@Wells, \emph{Malvern Hills}}

Death of Chris Cheeseman (C).

\noindent
\begin{tabular*}{\columnwidth}{@{\extracolsep{\fill}} p{0.545\columnwidth} >{\itshape}l r @{\extracolsep{\fill}}}
Chris O'Donnell & C & 317\\
Simon Gill & LD & 227\\
Richard Spencer & UKIP & 158\\
Louise Gibson & Ind & 76\\
Christopher Burrows & Lab & 71\\
\end{tabular*}

\subsection*{Redditch}

\subsubsection*{Batchley and Brockhill \hspace*{\fill}\nolinebreak[1]%
\enspace\hspace*{\fill}
\finalhyphendemerits=0
[22nd May]}

\index{Batchley and Brockhill , Redditch@Batchley \& Brockhill, \emph{Redditch}}

Resignation of Luke Stephens (Lab).

Combined with the 2014 ordinary election.

\subsubsection*{Church Hill \hspace*{\fill}\nolinebreak[1]%
\enspace\hspace*{\fill}
\finalhyphendemerits=0
[17th July; Lab gain from UKIP]}

\index{Church Hill , Redditch@Church Hill, \emph{Redditch}}

Resignation of Dave Small (UKIP).

\noindent
\begin{tabular*}{\columnwidth}{@{\extracolsep{\fill}} p{0.545\columnwidth} >{\itshape}l r @{\extracolsep{\fill}}}
Nina Wood-Ford & Lab & 600\\
Kathy Haslam & C & 339\\
Len Harris & UKIP & 332\\
David Gee & LD & 40\\
Lee Bradley & Grn & 34\\
Isabel Armstrong & Ind & 13\\
Agnieszka Wiecek & Ind & 9\\
\end{tabular*}

\subsection*{Wychavon}

\subsubsection*{Fladbury \hspace*{\fill}\nolinebreak[1]%
\enspace\hspace*{\fill}
\finalhyphendemerits=0
[22nd May]}

\index{Fladbury , Wychavon@Fladbury, \emph{Wychavon}}

Death of Tom McDonald (C).

\noindent
\begin{tabular*}{\columnwidth}{@{\extracolsep{\fill}} p{0.545\columnwidth} >{\itshape}l r @{\extracolsep{\fill}}}
Bradley Thomas & C & 456\\
Diana Brown & LD & 319\\
Neil Whelan & UKIP & 234\\
\end{tabular*}

\columnbreak

\section{Glamorgan}

\subsection*{Cardiff}

LlNInds = Llandaff North Independents

\subsubsection*{Canton \hspace*{\fill}\nolinebreak[1]%
\enspace\hspace*{\fill}
\finalhyphendemerits=0
[20th February]}

\index{Canton , Cardiff@Canton, \emph{Cardiff}}

Resignation of Cerys Furlong (Lab).

\noindent
\begin{tabular*}{\columnwidth}{@{\extracolsep{\fill}} p{0.545\columnwidth} >{\itshape}l r @{\extracolsep{\fill}}}
Susan Elsmore & Lab & 1201\\
Elin Tudur & PC & 972\\
Pam Richards & C & 381\\
Jake Griffiths & Grn & 148\\
Steffan Bateman & TUSC & 101\\
Matt Hemsley & LD & 80\\
\end{tabular*}

\subsubsection*{Llandaff North \hspace*{\fill}\nolinebreak[1]%
\enspace\hspace*{\fill}
\finalhyphendemerits=0
[2nd October]}

\index{Llandaff North , Cardiff@Llandaff N., \emph{Cardiff}}

Resignation of Siobhan Corria (Lab).

\noindent
\begin{tabular*}{\columnwidth}{@{\extracolsep{\fill}} p{0.54\columnwidth} >{\itshape}l r @{\extracolsep{\fill}}}
Susan White & Lab & 898\\
David Coggins Cogan & LlNInds & 419\\
Simon Zeigler & UKIP & 204\\
Peter Hudson & C & 136\\
Ann Rowland-James & LD & 134\\
\end{tabular*}

\subsection*{Merthyr Tydfil}

\subsubsection*{Penydarren \hspace*{\fill}\nolinebreak[1]%
\enspace\hspace*{\fill}
\finalhyphendemerits=0
[31st July; Lab gain from UKIP]}

\index{Penydarren , Merthyr Tydfil@Penydarren, \emph{Merthyr Tydfil}}

Death of Neil Greer (UKIP).

\noindent
\begin{tabular*}{\columnwidth}{@{\extracolsep{\fill}} p{0.545\columnwidth} >{\itshape}l r @{\extracolsep{\fill}}}
John McCarthy & Lab & 257\\
Kerry Baker Thomas & Ind & 235\\
Clive Barsi & Ind & 228\\
Robert Griffin & LD & 62\\
Kimberley Murphy & C & 40\\
\end{tabular*}

\subsection*{Neath Port Talbot}

\subsubsection*{Sandfields East \hspace*{\fill}\nolinebreak[1]%
\enspace\hspace*{\fill}
\finalhyphendemerits=0
[30th October]}

\index{Sandfields East , Neath Port Talbot@Sandfields E., \emph{Neath Port Talbot}}

Death of Mike Davies (Lab).

\noindent
\begin{tabular*}{\columnwidth}{@{\extracolsep{\fill}} p{0.545\columnwidth} >{\itshape}l r @{\extracolsep{\fill}}}
Matthew Crowley & Lab & 641\\
Keith Suter & UKIP & 361\\
Richard Minshull & C & 47\\
\end{tabular*}

\subsection*{Rhondda Cynon Taf}

\subsubsection*{Aberaman North \hspace*{\fill}\nolinebreak[1]%
\enspace\hspace*{\fill}
\finalhyphendemerits=0
[24th July]}

\index{Aberaman North , Rhondda Cynon Taf@Aberaman North, \emph{Rhondda Cynon Taf}}

Death of Anthony Christopher (Lab).

\noindent
\begin{tabular*}{\columnwidth}{@{\extracolsep{\fill}} p{0.545\columnwidth} >{\itshape}l r @{\extracolsep{\fill}}}
Sheryl Evans & Lab & 356\\
Andrew Thomas & Ind & 276\\
Julie Williams & PC & 228\\
Mia Hollsing & TUSC & 23\\
Lewis Israel & C & 20\\
\end{tabular*}

\columnbreak

\subsection*{Swansea}

I@S = Independents @ Swansea

\subsubsection*{Uplands \hspace*{\fill}\nolinebreak[1]%
\enspace\hspace*{\fill}
\finalhyphendemerits=0
[20th November; Ind gain from Lab]}

\index{Uplands , Swansea@Uplands, \emph{Swansea}}

Resignation of Pearleen Sangha (Lab).

\noindent
\begin{tabular*}{\columnwidth}{@{\extracolsep{\fill}} p{0.545\columnwidth} >{\itshape}l r @{\extracolsep{\fill}}}
Peter May & Ind & 671\\
Fran Griffiths & Lab & 533\\
Janet Thomas & LD & 215\\
Ashley Wakeling & Grn & 179\\
Pat Dwan & I@S & 158\\
Josh Allard & C & 154\\
Rhydian Fitter & PC & 104\\
Ronnie Job & TUSC & 31\\
\end{tabular*}

\section{Gwent}

\subsection*{Caerphilly}

\subsubsection*{Blackwood \hspace*{\fill}\nolinebreak[1]%
\enspace\hspace*{\fill}
\finalhyphendemerits=0
[24th April]}

\index{Blackwood , Caerphilly@Blackwood, \emph{Caerphilly}}

Resignation of Diana Ellis (Lab).

\noindent
\begin{tabular*}{\columnwidth}{@{\extracolsep{\fill}} p{0.545\columnwidth} >{\itshape}l r @{\extracolsep{\fill}}}
Allan Rees & Lab & 620\\
Keith Smallman & Ind & 477\\
Andrew Farina-Childs & PC & 349\\
Cameron Muir-Jones & C & 86\\
\end{tabular*}

\section{Mid and West Wales}

\subsection*{Carmarthenshire}

\subsubsection*{Trelech \hspace*{\fill}\nolinebreak[1]%
\enspace\hspace*{\fill}
\finalhyphendemerits=0
[11th December; PC gain from Ind]}

\index{Trelech , Carmarthenshire@Trelech, \emph{Carms.}}

Resignation of Dai Thomas (Ind).

\noindent
\begin{tabular*}{\columnwidth}{@{\extracolsep{\fill}} p{0.545\columnwidth} >{\itshape}l r @{\extracolsep{\fill}}}
Jean Lewis & PC & 598\\
Hugh Phillips & Ind & 181\\
Selwyn Runnett & LD & 96\\
\end{tabular*}

\section{North Wales}

\subsection*{Conwy}

\subsubsection*{Betws yn Rhos \hspace*{\fill}\nolinebreak[1]%
\enspace\hspace*{\fill}
\finalhyphendemerits=0
[6th February]}

\index{Betws yn Rhos , Conwy@Betws yn Rhos, \emph{Conwy}}

Resignation of Ahmed Jamil (Ind).

\noindent
\begin{tabular*}{\columnwidth}{@{\extracolsep{\fill}} p{0.545\columnwidth} >{\itshape}l r @{\extracolsep{\fill}}}
Ifor Lloyd & Ind & 347\\
Clwyd Roberts & PC & 197\\
Caroline Evans & Ind & 127\\
Bryn Jones & C & 83\\
\end{tabular*}

\subsubsection*{Abergele Pensarn \hspace*{\fill}\nolinebreak[1]%
\enspace\hspace*{\fill}
\finalhyphendemerits=0
[18th September]}

\index{Abergele Pensarn , Conwy@Abergele Pensarn, \emph{Conwy}}

Resignation of Jean Stubbs (Lab).

\noindent
\begin{tabular*}{\columnwidth}{@{\extracolsep{\fill}} p{0.545\columnwidth} >{\itshape}l r @{\extracolsep{\fill}}}
Rick Stubbs & Lab & 160\\
Michael Smith & Ind & 134\\
Sarah Wardlaw & UKIP & 129\\
Ken Sudlow & Ind & 74\\
Barry Griffiths & Ind & 56\\
John Pitt & C & 54\\
Val Parker & Ind & 10\\
\end{tabular*}

\subsubsection*{Towyn \hspace*{\fill}\nolinebreak[1]%
\enspace\hspace*{\fill}
\finalhyphendemerits=0
[16th October]}

\index{Towyn , Conwy@Towyn, \emph{Conwy}}

Death of William Knightly (C).

\noindent
\begin{tabular*}{\columnwidth}{@{\extracolsep{\fill}} p{0.545\columnwidth} >{\itshape}l r @{\extracolsep{\fill}}}
Laura Knightly & C & 143\\
Michael Smith & Ind & 116\\
David Johnson & Ind & 104\\
Beverley Pickard-Jones & Lab & 98\\
Barry Griffiths & Ind & 69\\
Geoff Corry & Ind & 43\\
\end{tabular*}

\subsection*{Flintshire}

\subsubsection*{Flint Trelawny \hspace*{\fill}\nolinebreak[1]%
\enspace\hspace*{\fill}
\finalhyphendemerits=0
[10th April]}

\index{Flint Trelawny , Flintshire@Flint Trelawny, \emph{Flints.}}

Death of Ted Evans (Lab).

\noindent
\begin{tabular*}{\columnwidth}{@{\extracolsep{\fill}} p{0.545\columnwidth} >{\itshape}l r @{\extracolsep{\fill}}}
Paul Cunningham & Lab & 350\\
Nigel Williams & UKIP & 261\\
John Yorke & Ind & 242\\
Swapna Das & C & 54\\
\end{tabular*}

\subsubsection*{Flint Trelawny \hspace*{\fill}\nolinebreak[1]%
\enspace\hspace*{\fill}
\finalhyphendemerits=0
[22nd May]}

\index{Flint Trelawny , Flintshire@Flint Trelawny, \emph{Flints.}}

Resignation of Trefor Howarth (Lab).

\noindent
\begin{tabular*}{\columnwidth}{@{\extracolsep{\fill}} p{0.545\columnwidth} >{\itshape}l r @{\extracolsep{\fill}}}
Vicky Perfect & Lab & 577\\
Nigel Williams & UKIP & 434\\
Swapna Das & C & 90\\
\end{tabular*}

\subsubsection*{Mostyn \hspace*{\fill}\nolinebreak[1]%
\enspace\hspace*{\fill}
\finalhyphendemerits=0
[31st July]}

\index{Mostyn , Flintshire@Mostyn, \emph{Flints.}}

Disqualification of Patrick Heeson (Ind) by the Adjudication Panel for Wales.

\noindent
\begin{tabular*}{\columnwidth}{@{\extracolsep{\fill}} p{0.545\columnwidth} >{\itshape}l r @{\extracolsep{\fill}}}
David Roney & Ind & 205\\
Pam Banks & Lab & 191\\
Liz Soutter & UKIP & 90\\
Richard Pendlebury & C & 27\\
\end{tabular*}

\subsection*{Gwynedd}

\subsubsection*{Bowydd and Rhiw \hspace*{\fill}\nolinebreak[1]%
\enspace\hspace*{\fill}
\finalhyphendemerits=0
[18th December]}

\index{Bowydd and Rhiw , Gwynedd@Bowydd \& Rhiw, \emph{Gwynedd}}

Resignation of Paul Thomas (PC).

\noindent
\begin{tabular*}{\columnwidth}{@{\extracolsep{\fill}} p{0.545\columnwidth} >{\itshape}l r @{\extracolsep{\fill}}}
Annwen Daniels & PC & \emph{unop.}\\
\end{tabular*}

\section{Aberdeen City and Shire}

\subsection*{Aberdeenshire}

\subsubsection*{Troup \hspace*{\fill}\nolinebreak[1]%
\enspace\hspace*{\fill}
\finalhyphendemerits=0
[27th November; SNP gain from C]}

\index{Troup , Aberdeenshire@Troup, \emph{Aberdeenshire}}

Death of John Duncan (C).

\noindent
\begin{tabular*}{\columnwidth}{@{\extracolsep{\fill}} p{0.545\columnwidth} >{\itshape}l r @{\extracolsep{\fill}}}
\emph{First preferences}\\
Ross Cassie & SNP & 1159\\
Iain Taylor & C & 574\\
Alan Still & Ind & 391\\
Ann Bell & LD & 141\\
Alan Duffill & Lab & 140\\
Darren Duncan & Grn & 68\\
Philip Mitchell & Ind & 43\\
\end{tabular*}

\emph{Duncan and Mitchell eliminated}: Cassie 1183 Taylor 588 Still 415 Bell 149 Duffill 148

\emph{Duffill eliminated}: Cassie 1205 Taylor 604 Still 423 Bell 180

\noindent
\begin{tabular*}{\columnwidth}{@{\extracolsep{\fill}} p{0.545\columnwidth} >{\itshape}l r @{\extracolsep{\fill}}}
\emph{Bell eliminated}\\
Ross Cassie & SNP & 1244\\
Iain Taylor & C & 645\\
Alan Still & Ind & 446\\
\end{tabular*}

\section{Ayrshire Councils}

\subsection*{East Ayrshire}

\subsubsection*{Kilmarnock North \hspace*{\fill}\nolinebreak[1]%
\enspace\hspace*{\fill}
\finalhyphendemerits=0
[27th March]}

\index{Kilmarnock North , East Ayrshire@Kilmarnock N., \emph{E. Ayrs.}}

Death of Andrew Hershaw (SNP).

\noindent
\begin{tabular*}{\columnwidth}{@{\extracolsep{\fill}} p{0.545\columnwidth} >{\itshape}l r @{\extracolsep{\fill}}}
\emph{First preferences}\\
Elaine Cowan & SNP & 1334\\
Scott Thomson & Lab & 1130\\
Ian Grant & C & 493\\
Jen Broadhurst & Grn & 61\\
\end{tabular*}

\noindent
\begin{tabular*}{\columnwidth}{@{\extracolsep{\fill}} p{0.545\columnwidth} >{\itshape}l r @{\extracolsep{\fill}}}
\multicolumn{3}{@{\extracolsep{\fill}}l}{\emph{Grant and Broadhurst eliminated}}\\
Elaine Cowan & SNP & 1473\\
Scott Thomson & Lab & 1320\\
\end{tabular*}

\subsection*{North Ayrshire}

\subsubsection*{North Coast and Cumbraes \hspace*{\fill}\nolinebreak[1]%
\enspace\hspace*{\fill}
\finalhyphendemerits=0
[30th October]}

\index{North Coast and Cumbraes , North Ayrshire@North Coast \& Cumbraes, \emph{N. Ayrs.}}

Death of Alex McLean (SNP).

\noindent
\begin{tabular*}{\columnwidth}{@{\extracolsep{\fill}} p{0.545\columnwidth} >{\itshape}l r @{\extracolsep{\fill}}}
\emph{First preferences}\\
Grace McLean & SNP & 2021\\
Drew Cochrane & Ind & 1190\\
Toni Dawson & C & 1125\\
Valerie Reid & Lab & 691\\
Meilan Henderson & UKIP & 192\\
\end{tabular*}

\sloppyword{\emph{Reid and Henderson eliminated}: McLean 2156 Cochrane 1411 Dawson 1296}

\noindent
\begin{tabular*}{\columnwidth}{@{\extracolsep{\fill}} p{0.545\columnwidth} >{\itshape}l r @{\extracolsep{\fill}}}
\emph{Exclude Dawson}\\
Grace McLean & SNP & 2279\\
Drew Cochrane & Ind & 2000\\
\end{tabular*}

\section{Border Councils}

\subsection*{Scottish Borders}

\subsubsection*{Hawick and Denholm \hspace*{\fill}\nolinebreak[1]%
\enspace\hspace*{\fill}
\finalhyphendemerits=0
[22nd May; Ind gain from C]}

\index{Hawick and Denholm , Scottish Borders@Hawick \& Denholm, \emph{Scot. Borders}}

Death of Zandra Elliott (C).

\noindent
\begin{tabular*}{\columnwidth}{@{\extracolsep{\fill}} p{0.545\columnwidth} >{\itshape}l r @{\extracolsep{\fill}}}
\emph{First preferences}\\
Watson McAteer & Ind & 732\\
Trevor Adams & C & 622\\
Ian Turnbull & LD & 450\\
Harry Stoddart & SNP & 419\\
Marion Short & Ind & 326\\
Davie Paterson & Ind & 250\\
Craig Bryson & Ind & 74\\
\end{tabular*}

\emph{Paterson and Bryson eliminated}: McAteer 803 Adams 638 Turnbull 485 Stoddart 461 Short 385

\emph{Short eliminated}: McAteer 900 Adams 665 Turnbull 545 Stoddart 505

\emph{Stoddard eliminated}: McAteer 1010 Adams 692 Turnbull 637

\noindent
\begin{tabular*}{\columnwidth}{@{\extracolsep{\fill}} p{0.545\columnwidth} >{\itshape}l r @{\extracolsep{\fill}}}
\emph{Turnbull eliminated}\\
Watson McAteer & Ind & 1222\\
Trevor Adams & C & 798\\
\end{tabular*}

\columnbreak

\section{Clyde Councils}

\subsection*{North Lanarkshire}

\subsubsection*{Motherwell North \hspace*{\fill}\nolinebreak[1]%
\enspace\hspace*{\fill}
\finalhyphendemerits=0
[23rd January]}

\index{Motherwell North , North Lanarkshire@Motherwell N., \emph{N. Lanarks.}}

Death of Annita McAuley (Lab).

\noindent
\begin{tabular*}{\columnwidth}{@{\extracolsep{\fill}} p{0.545\columnwidth} >{\itshape}l r @{\extracolsep{\fill}}}
Pat O'Rourke & Lab & 1719\\
Jordan Linden & SNP & 520\\
Bob Burgess & C & 173\\
Neil Wilson & UKIP & 107\\
\end{tabular*}

\subsection*{South Lanarkshire}

\subsubsection*{Clydesdale South \hspace*{\fill}\nolinebreak[1]%
\enspace\hspace*{\fill}
\finalhyphendemerits=0
[5th June; Lab gain from SNP]}

\index{Clydesdale South , South Lanarkshire@Clydesdale S., \emph{S. Lanarks.}}

Resignation of Archie Manson (SNP).

\noindent
\begin{tabular*}{\columnwidth}{@{\extracolsep{\fill}} p{0.545\columnwidth} >{\itshape}l r @{\extracolsep{\fill}}}
\emph{First preferences}\\
Gordon Muir & Lab & 1492\\
George Sneddon & SNP & 1170\\
Donna Hood & C & 659\\
Donald MacKay & UKIP & 233\\
Ruth Thomas & Grn & 104\\
\end{tabular*}

\noindent
\begin{tabular*}{\columnwidth}{@{\extracolsep{\fill}} p{0.545\columnwidth} >{\itshape}l r @{\extracolsep{\fill}}}
\multicolumn{3}{@{\extracolsep{\fill}}l}{\emph{Hood, MacKay and Thomas eliminated}}\\
Gordon Muir & Lab & 1819\\
George Sneddon & SNP & 1356\\
\end{tabular*}

\section{Forth Councils}

\subsection*{Fife}

\subsubsection*{Cowdenbeath \hspace*{\fill}\nolinebreak[1]%
\enspace\hspace*{\fill}
\finalhyphendemerits=0
[22nd May]}

\index{Cowdenbeath , Fife@Cowdenbeath, \emph{Fife}}

Resignation of Jayne Baxter (Lab).

\noindent
\begin{tabular*}{\columnwidth}{@{\extracolsep{\fill}} p{0.545\columnwidth} >{\itshape}l r @{\extracolsep{\fill}}}
Gary Guichan & Lab & 2039\\
Connor Watt & SNP & 834\\
Judith Rideout & UKIP & 277\\
John Wheatley & C & 164\\
\end{tabular*}

\subsubsection*{The Lochs \hspace*{\fill}\nolinebreak[1]%
\enspace\hspace*{\fill}
\finalhyphendemerits=0
[22nd May]}

\index{Lochs , Fife@The Lochs, \emph{Fife}}

Resignation of Alex Rowley (Lab).

\noindent
\begin{tabular*}{\columnwidth}{@{\extracolsep{\fill}} p{0.545\columnwidth} >{\itshape}l r @{\extracolsep{\fill}}}
Alex Campbell & Lab & 2042\\
Lesley Backhouse & SNP & 753\\
Martin Green & UKIP & 162\\
Jonathan Gray & C & 141\\
\end{tabular*}

\subsection*{Midlothian}

\subsubsection*{Midlothian East \hspace*{\fill}\nolinebreak[1]%
\enspace\hspace*{\fill}
\finalhyphendemerits=0
[27th November]}

\index{Midlothian East , Midlothian@Midlothian E., \emph{Midlothian}}

Resignation of Peter Boyes (Ind elected as Lab).

\noindent
\begin{tabular*}{\columnwidth}{@{\extracolsep{\fill}} p{0.545\columnwidth} >{\itshape}l r @{\extracolsep{\fill}}}
\emph{First preferences}\\
Kenny Young & Lab & 1294\\
Colin Cassidy & SNP & 1260\\
Robert Hogg & Ind & 780\\
Andrew Hardie & C & 331\\
Bill Kerr-Smith & Grn & 197\\
Euan Davidson & LD & 68\\
\end{tabular*}
\noindent

\noindent
\begin{tabular*}{\columnwidth}{@{\extracolsep{\fill}} p{0.545\columnwidth} >{\itshape}l r @{\extracolsep{\fill}}}
\multicolumn{3}{@{\extracolsep{\fill}}l}{\emph{Four candidates eliminated}}\\
Kenny Young & Lab & 1682\\
Colin Cassidy & SNP & 1613\\
\end{tabular*}

\section{Highland Councils}

\subsection*{Argyll and Bute}

\subsubsection*{Oban South and the Isles \hspace*{\fill}\nolinebreak[1]%
\enspace\hspace*{\fill}
\finalhyphendemerits=0
[22nd May; Lab gain from SNP]}

\index{Oban South and the Isles , Argyll and Bute@Oban S. \& the Isles, \emph{Argyll \& Bute}}

Resignation of Fred Hall (Ind elected as SNP).

\noindent
\begin{tabular*}{\columnwidth}{@{\extracolsep{\fill}} p{0.545\columnwidth} >{\itshape}l r @{\extracolsep{\fill}}}
\emph{First preferences}\\
Iain MacLean & SNP & 980\\
Neil MacIntyre & Lab & 918\\
David Pollard & LD & 635\\
John MacGregor & Ind & 616\\
Andrew Vennard & C & 345\\
\end{tabular*}

\emph{Vennard eliminated}: MacLean 988 MacIntyre 946 Pollard 742 MacGregor 688

\emph{MacGregor eliminated}: MacIntyre 1134 MacLean 1102 Pollard 865

\noindent
\begin{tabular*}{\columnwidth}{@{\extracolsep{\fill}} p{0.545\columnwidth} >{\itshape}l r @{\extracolsep{\fill}}}
\emph{Pollard eliminated}\\
Neil MacIntyre & Lab & 1413\\
Iain MacLean & SNP & 1260\\
\end{tabular*}

\subsubsection*{Oban North and Lorn \hspace*{\fill}\nolinebreak[1]%
\enspace\hspace*{\fill}
\finalhyphendemerits=0
[17th July; Ind gain from SNP]}

\index{Oban North and Lorn , Argyll and Bute@Oban N. \& Lorn, \emph{Argyll \& Bute}}

Resignation of Louise Glen-Lee (SNP).

\noindent
\begin{tabular*}{\columnwidth}{@{\extracolsep{\fill}} p{0.545\columnwidth} >{\itshape}l r @{\extracolsep{\fill}}}
\emph{First preferences}\\
Gerry Fisher & SNP & 595\\
John MacGregor & Ind & 548\\
Kieron Green & Lab & 526\\
Andrew Vennard & C & 445\\
Marri Malloy & Ind & 301\\
\end{tabular*}

\emph{Malloy eliminated}: Fisher 635 MacGregor 614 Green 572 Vennard 493

\emph{Vennard eliminated}: MacGregor 771 Green 681 Fisher 658

\noindent
\begin{tabular*}{\columnwidth}{@{\extracolsep{\fill}} p{0.545\columnwidth} >{\itshape}l r @{\extracolsep{\fill}}}
\emph{Fisher eliminated}\\
John MacGregor & Ind & 920\\
Kieron Green & Lab & 874\\
\end{tabular*}

\subsubsection*{Oban North and Lorn \hspace*{\fill}\nolinebreak[1]%
\enspace\hspace*{\fill}
\finalhyphendemerits=0
[23rd October; SNP gain from Ind]}

\index{Oban North and Lorn , Argyll and Bute@Oban N. \& Lorn, \emph{Argyll \& Bute}}

Death of John MacGregor (Ind).

\noindent
\begin{tabular*}{\columnwidth}{@{\extracolsep{\fill}} p{0.545\columnwidth} >{\itshape}l r @{\extracolsep{\fill}}}
\emph{First preferences}\\
Iain MacLean & SNP & 1090\\
Stephanie Irvine & Ind & 629\\
Kieron Green & Lab & 530\\
Andrew Vennard & C & 415\\
\end{tabular*}

\emph{Vennard eliminated}: MacLean 1102 Irvine 812 Green 621

\noindent
\begin{tabular*}{\columnwidth}{@{\extracolsep{\fill}} p{0.545\columnwidth} >{\itshape}l r @{\extracolsep{\fill}}}
\emph{Green eliminated}\\
Iain MacLean & SNP & 1199\\
Stephanie Irvine & Ind & 1080\\
\end{tabular*}

\subsubsection*{South Kintyre \hspace*{\fill}\nolinebreak[1]%
\enspace\hspace*{\fill}
\finalhyphendemerits=0
[11th December]}

\index{South Kintyre , Argyll and Bute@South Kintyre, \emph{Argyll \& Bute}}

Resignation of John Semple (SNP).

\noindent
\begin{tabular*}{\columnwidth}{@{\extracolsep{\fill}} p{0.545\columnwidth} >{\itshape}l r @{\extracolsep{\fill}}}
John Armour & SNP & 942\\
Joyce Oxborrow & LD & 214\\
Charlotte Hanbury & C & 203\\
Michael Kelly & Lab & 156\\
\end{tabular*}

\subsection*{Highland}

\subsubsection*{Caol and Mallaig \hspace*{\fill}\nolinebreak[1]%
\enspace\hspace*{\fill}
\finalhyphendemerits=0
[1st May]}

\index{Caol and Mallaig , Highland@Caol and Mallaig, \emph{Highland}}

Resignation of Eddie Hunter (Ind).

\noindent
\begin{tabular*}{\columnwidth}{@{\extracolsep{\fill}} p{0.545\columnwidth} >{\itshape}l r @{\extracolsep{\fill}}}
\emph{First preferences}\\
Ben Thompson & Ind & 932\\
William Macdonald & SNP & 724\\
Sandy Watson & Ind & 537\\
Liam Simmonds & UKIP & 133\\
Susan Wallace & Chr & 63\\
\end{tabular*}

\sloppyword{\emph{Simmonds and Wallace eliminated}: Thompson 1109 Macdonald 756 Watson 551}

\noindent
\begin{tabular*}{\columnwidth}{@{\extracolsep{\fill}} p{0.545\columnwidth} >{\itshape}l r @{\extracolsep{\fill}}}
\emph{Watson eliminated}\\
Ben Thompson & Ind & 1176\\
William Macdonald & SNP & 881\\
\end{tabular*}

\subsection*{Moray}

\subsubsection*{Buckie \hspace*{\fill}\nolinebreak[1]%
\enspace\hspace*{\fill}
\finalhyphendemerits=0
[30th January]}

\index{Buckie , Moray@Buckie, \emph{Moray}}

Resignation of Anne McKay (Ind).

\noindent
\begin{tabular*}{\columnwidth}{@{\extracolsep{\fill}} p{0.545\columnwidth} >{\itshape}l r @{\extracolsep{\fill}}}
\emph{First preferences}\\
Gordon Cowie & Ind & 830\\
Linda McDonald & SNP & 670\\
Marc MacRae & Ind & 220\\
Margaret Gambles & C & 143\\
\end{tabular*}

\emph{Gambles eliminated}: Cowie 907 McDonald 679 MacRae 248

\noindent
\begin{tabular*}{\columnwidth}{@{\extracolsep{\fill}} p{0.545\columnwidth} >{\itshape}l r @{\extracolsep{\fill}}}
\emph{MacRae eliminated}\\
Gordon Cowie & Ind & 1034\\
Linda McDonald & SNP & 710\\
\end{tabular*}

\subsubsection*{Elgin City North \hspace*{\fill}\nolinebreak[1]%
\enspace\hspace*{\fill}
\finalhyphendemerits=0
[11th December]}

\index{Elgin City North , Moray@Elgin City N., \emph{Moray}}

Resignation of Barry Jarvis (Lab).

\noindent
\begin{tabular*}{\columnwidth}{@{\extracolsep{\fill}} p{0.545\columnwidth} >{\itshape}l r @{\extracolsep{\fill}}}
\emph{First preferences}\\
Kirsty Reid & SNP & 728\\
Sandy Cooper & Ind & 472\\
Craig Graham & Lab & 287\\
Alex Griffiths & C & 273\\
Ramsay Urquhart & UKIP & 81\\
Morvern Forrest & Grn & 77\\
\end{tabular*}

\emph{Urquhart and Forrest eliminated}: Reid 764 Cooper 499 Graham 316 Griffiths 299

\emph{Griffiths eliminated}: Reid 773 Cooper 595 Graham 368

\noindent
\begin{tabular*}{\columnwidth}{@{\extracolsep{\fill}} p{0.545\columnwidth} >{\itshape}l r @{\extracolsep{\fill}}}
\emph{Graham eliminated}\\
Kirsty Reid & SNP & 850\\
Sandy Cooper & Ind & 693\\
\end{tabular*}

\section{Island Councils}

\subsection*{Orkney}

\subsubsection*{Kirkwall West and Orphir \hspace*{\fill}\nolinebreak[1]%
\enspace\hspace*{\fill}
\finalhyphendemerits=0
[27th November]}

\index{Kirkwall West and Orphir , Orkney@Kirkwall W. and Orphir, \emph{Orkney}}

Resignation of Jack Moodie (Ind).

Note:--- Laurence Leonard died before polling; as he was an independent candidate the election continued as scheduled.

\noindent
\begin{tabular*}{\columnwidth}{@{\extracolsep{\fill}} p{0.545\columnwidth} >{\itshape}l r @{\extracolsep{\fill}}}
Leslie Manson & Ind & 647\\
Gillian Skuse & Ind & 281\\
Lorraine McBrearty & Ind & 142\\
\dag{}Laurence Leonard & Ind & 55\\
\end{tabular*}

\end{resultsiii}

\part{2015}
\renewcommand\resultsyear{2015}

%\part{By-elections}

\chapter{Parliamentary by-elections}

There was one parliamentary by-election in 2015.

\section*{Oldham West and Royton \hspace*{\fill}\nolinebreak[1]%
\enspace\hspace*{\fill}
\finalhyphendemerits=0
[3rd December]}

\index{Oldham West and Royton , House of Commons@Oldham W. \& Royton, \emph{House of Commons}}

Death of Michael Meacher (Lab).

\noindent
\begin{tabular*}{\columnwidth}{@{\extracolsep{\fill}} p{0.545\columnwidth} >{\itshape}l r @{\extracolsep{\fill}}}
Jim McMahon & Lab & 17209\\
John Bickley & UKIP & 6487\\
James Daly & C & 2596\\
Jane Brophy & LD & 1024\\
Simeon Hart & Grn & 249\\
Sir Oink A-Lot & Loony & 141\\
\end{tabular*}

\chapter{By-elections to devolved assemblies, the European Parliament, and police and crime commissionerships}

\section{Greater London Authority}

There were no by-elections in 2015 to the Greater London Authority.

Victoria Borwick (C, London Member) resigned on 16th September 2015.  Her seat was filled by Kemi Badenoch.

\section{National Assembly for Wales}

There were no by-elections in 2015 to the National Assembly for Wales.

Antionette Sandbach (C, North Wales) resigned on 10th May 2015.  Her seat was filled by Janet Howarth.

Byron Davies (C, South Wales West) resigned on 11th May 2015.  His seat was filled by Altaf Hussein.

\section{Scottish Parliament}

There were no by-elections in 2015 to the Scottish Parliament.

Margo MacDonald (Ind, Lothian) died on 4th April 2014.  Her seat will remain vacant until the end of the parliamentary term.

\section{Northern Ireland Assembly}

Vacancies in the Northern Ireland Assembly are filled by co-option.  No co-options were made in 2015.

%The following members were co-opted to the Assembly in 2014:
%\begin{itemize}
%\item Claire Sugden (Ind) replaced David McClarty following his death on 18th April (East Londonderry).
%\end{itemize}

\section{European Parliament}

UK vacancies in the European Parliament are filled by the next available person from the party list at the most recent election (which was held in 2014). 
No replacements were made in 2015.
%The following replacement was made in 2010:
%\begin{itemize}
%\item Keith Taylor (Grn) replaced Caroline Lucas following her resignation on 17th May (South East).
%\end{itemize}

\section{Police and crime commissioners}

There were no by-elections in 2015 for vacant police and crime commissioner posts.

\chapter{Local by-elections and unfilled vacancies}

\begin{resultsiii}

\section{North London}

\council{City of London}

\subsubsection*{Candlewick \hspace*{\fill}\nolinebreak[1]%
\enspace\hspace*{\fill}
\finalhyphendemerits=0
[Wednesday 17th June]}

\index{Candlewick , City of London@Candlewick, \emph{City of London}}

Resignation of Stanley Knowles.

\noindent
\begin{tabular*}{\columnwidth}{@{\extracolsep{\fill}} p{0.545\columnwidth} >{\itshape}l r @{\extracolsep{\fill}}}
Havilland de Sausmarez & Ind & 35\\
Richard Evans & Ind & 33\\
Leslie Jonas & Ind & 28\\
Gillian Kaile & Ind & 15\\
\end{tabular*}

\subsubsection*{Bassishaw \hspace*{\fill}\nolinebreak[1]%
\enspace\hspace*{\fill}
\finalhyphendemerits=0
[18th June]}

\index{Bassishaw , City of London@Bassishaw, \emph{City of London}}

Resignation of Kenneth Ayers.

\noindent
\begin{tabular*}{\columnwidth}{@{\extracolsep{\fill}} p{0.65\columnwidth} >{\itshape}l r @{\extracolsep{\fill}}}
Graeme Harrower & Ind & 62\\
Richard Hopkinson-Woolley & Ind & 49\\
Ross Cowling & Ind & 39\\
Tristan Feunteun & Ind & 2\\
\end{tabular*}

\subsubsection*{Bridge and Bridge Without \hspace*{\fill}\nolinebreak[1]%
\enspace\hspace*{\fill}
\finalhyphendemerits=0
[Friday 3rd July]}

\index{Bridge and Bridge Without , City of London@Bridge \& Bridge Wt., \emph{City of London}}

Resignation of John Owen-Ward.

\noindent
\begin{tabular*}{\columnwidth}{@{\extracolsep{\fill}} p{0.545\columnwidth} >{\itshape}l r @{\extracolsep{\fill}}}
Keith Bottomley & Ind & 75\\
Jason Groves & Ind & 20\\
Colin Gregory & Ind & 17\\
Stephen Evans & Ind & 1\\
\end{tabular*}

\subsubsection*{Cornhill \hspace*{\fill}\nolinebreak[1]%
\enspace\hspace*{\fill}
\finalhyphendemerits=0
[Tuesday 13th October]}

\index{Cornhill , City of London@Cornhill, \emph{City of London}}

Aldermanic election: resignation of Sir David Howard, Bt.\ (Ind).

\noindent
\begin{tabular*}{\columnwidth}{@{\extracolsep{\fill}} p{0.545\columnwidth} >{\itshape}l r @{\extracolsep{\fill}}}
Robert Howard & Ind & 137\\
Richard Hills & Ind & 46\\
\end{tabular*}

\subsubsection*{Walbrook \hspace*{\fill}\nolinebreak[1]%
\enspace\hspace*{\fill}
\finalhyphendemerits=0
[Wednesday 2nd December]}

\index{Walbrook , City of London@Walbrook, \emph{City of London}}

Aldermanic election: John Garbutt (Ind) sought re-election.

\noindent
\begin{tabular*}{\columnwidth}{@{\extracolsep{\fill}} p{0.545\columnwidth} >{\itshape}l r @{\extracolsep{\fill}}}
John Garbutt & Ind & 86\\
Robert Waddingham & Ind & 61\\
Clive Bannister & Ind & 25\\
\end{tabular*}

\council{Barnet}

\subsubsection*{Garden Suburb \hspace*{\fill}\nolinebreak[1]%
\enspace\hspace*{\fill}
\finalhyphendemerits=0
[7th May]}

\index{Garden Suburb , Barnet@Garden Suburb, \emph{Barnet}}

Resignation of Danny Seal (C).

\noindent
\begin{tabular*}{\columnwidth}{@{\extracolsep{\fill}} p{0.545\columnwidth} >{\itshape}l r @{\extracolsep{\fill}}}
Rohit Grover & C & 5015\\
Janet Solomons & Lab & 1667\\
Altan Akbikyik & LD & 423\\
Adele Ward & Grn & 418\\
Barry Ryan & UKIP & 194\\
\end{tabular*}

\council{Brent}

\subsubsection*{Kenton \hspace*{\fill}\nolinebreak[1]%
\enspace\hspace*{\fill}
\finalhyphendemerits=0
[5th March]}

\index{Kenton , Brent@Kenton, \emph{Brent}}

Death of Bhiku Patel (C).

\noindent
\begin{tabular*}{\columnwidth}{@{\extracolsep{\fill}} p{0.545\columnwidth} >{\itshape}l r @{\extracolsep{\fill}}}
Michael Maurice & C & 1097\\
Vincent Lo & Lab & 839\\
Michaela Lichten & Grn & 121\\
Bob Wharton & LD & 70\\
\end{tabular*}

\subsubsection*{Kensal Green \hspace*{\fill}\nolinebreak[1]%
\enspace\hspace*{\fill}
\finalhyphendemerits=0
[17th December]}

\index{Kensal Green , Brent@Kensal Green, \emph{Brent}}

Death of Dan Filson (Lab).

\noindent
\begin{tabular*}{\columnwidth}{@{\extracolsep{\fill}} p{0.545\columnwidth} >{\itshape}l r @{\extracolsep{\fill}}}
Jumbo Chan & Lab & 931\\
Sarah Dickson & LD & 417\\
Chris Alley & C & 255\\
Jafar Hussain & Grn & 102\\
Juliette Nibbs & UKIP & 38\\
\end{tabular*}

\council{Camden}

\subsubsection*{St Pancras and Somers Town \hspace*{\fill}\nolinebreak[1]%
\enspace\hspace*{\fill}
\finalhyphendemerits=0
[5th March]}

\index{Saint Pancras and Somers Town , Camden@St Pancras \& Somers Town, \emph{Camden}}

Death of Peter Brayshaw (Lab).

\noindent
\begin{tabular*}{\columnwidth}{@{\extracolsep{\fill}} p{0.545\columnwidth} >{\itshape}l r @{\extracolsep{\fill}}}
Paul Tomlinson & Lab & 1481\\
Shahin Ahmed & C & 243\\
Tina Swasey & Grn & 213\\
Zack Polanski & LD & 96\\
\end{tabular*}

\subsubsection*{Hampstead Town \hspace*{\fill}\nolinebreak[1]%
\enspace\hspace*{\fill}
\finalhyphendemerits=0
[7th May]}

\index{Hampstead Town , Camden@Hampstead Town, \emph{Camden}}

Resignation of Simon Marcus (C).

\noindent
\begin{tabular*}{\columnwidth}{@{\extracolsep{\fill}} p{0.545\columnwidth} >{\itshape}l r @{\extracolsep{\fill}}}
Oliver Cooper & C & 2693\\
Maddy Raman & Lab & 1381\\
Sophie Dix & Grn & 597\\
Yannick Bultingaire & LD & 543\\
Nigel Rumble &Ind & 73\\
\end{tabular*}

\council{Ealing}

\subsubsection*{Northfield \hspace*{\fill}\nolinebreak[1]%
\enspace\hspace*{\fill}
\finalhyphendemerits=0
[7th May]}

\index{Northfield , Ealing@Northfield, \emph{Ealing}}

Resignation of Mark Reen (C).

\noindent
\begin{tabular*}{\columnwidth}{@{\extracolsep{\fill}} p{0.545\columnwidth} >{\itshape}l r @{\extracolsep{\fill}}}
Fabio Conti & C & 2750\\
Anita Macdonald & Lab & 2630\\
Bruni de la Motte & Grn & 751\\
Joanna Dugdale & LD & 570\\
Bob Little & UKIP & 262\\
\end{tabular*}

\council{Enfield}

\subsubsection*{Jubilee \hspace*{\fill}\nolinebreak[1]%
\enspace\hspace*{\fill}
\finalhyphendemerits=0
[7th May]}

\index{Jubilee , Enfield@Jubilee, \emph{Enfield}}

Resignation of Rohini Simbodyal (Lab).

\noindent
\begin{tabular*}{\columnwidth}{@{\extracolsep{\fill}} p{0.545\columnwidth} >{\itshape}l r @{\extracolsep{\fill}}}
Nesil Cazimoglu & Lab & 3313\\
Nazim Celebi & C & 1339\\
Sharon Downer & UKIP & 602\\
Benjamin Maydon & Grn & 229\\
Matt McLaren & LD & 108\\
\end{tabular*}

\council{Haringey}

\subsubsection*{Noel Park \hspace*{\fill}\nolinebreak[1]%
\enspace\hspace*{\fill}
\finalhyphendemerits=0
[17th September]}

\index{Noel Park , Haringey@Noel Park, \emph{Haringey}}

Resignation of Denise Marshall (Lab).

\noindent
\begin{tabular*}{\columnwidth}{@{\extracolsep{\fill}} p{0.545\columnwidth} >{\itshape}l r @{\extracolsep{\fill}}}
Stephen Mann & Lab & 1005\\
Derin Adebiyi & LD & 247\\
Mike Burgess & C & 178\\
Mike McGowan & Grn & 124\\
Neville Watson & UKIP & 48\\
Paul Burnham & TUSC & 38\\
\end{tabular*}

\subsubsection*{Woodside \hspace*{\fill}\nolinebreak[1]%
\enspace\hspace*{\fill}
\finalhyphendemerits=0
[17th September]}

\index{Woodside , Haringey@Woodside, \emph{Haringey}}

Death of George Meehan (Lab).

\noindent
\begin{tabular*}{\columnwidth}{@{\extracolsep{\fill}} p{0.545\columnwidth} >{\itshape}l r @{\extracolsep{\fill}}}
Peter Mitchell & Lab & 1279\\
Jenni Hollis & LD & 435\\
Robert Broadhurst & C & 141\\
Annette Baker & Grn & 122\\
Andrew Price & UKIP & 95\\
\end{tabular*}

\council{Hounslow}

\subsubsection*{Brentford \hspace*{\fill}\nolinebreak[1]%
\enspace\hspace*{\fill}
\finalhyphendemerits=0
[9th July]}

\index{Brentford , Hounslow@Brentford, \emph{Hounslow}}

Resignation of Ruth Cadbury (Lab).

\noindent
\begin{tabular*}{\columnwidth}{@{\extracolsep{\fill}} p{0.545\columnwidth} >{\itshape}l r @{\extracolsep{\fill}}}
Guy Lambert & Lab & 1292\\
Patrick Barr & C & 664\\
Diane Scott & Grn & 209\\
Joe Bourke & LD & 116\\
George Radulski & UKIP & 113\\
\end{tabular*}

\council{Kensington and Chelsea}

\subsubsection*{Stanley \hspace*{\fill}\nolinebreak[1]%
\enspace\hspace*{\fill}
\finalhyphendemerits=0
[7th May]}

\index{Stanley , Kensington and Chelsea@Stanley, \emph{Kensington \& Chelsea}}

Resignation of Sir Merrick Cockell (C).

\noindent
\begin{tabular*}{\columnwidth}{@{\extracolsep{\fill}} p{0.545\columnwidth} >{\itshape}l r @{\extracolsep{\fill}}}
Kim Taylor-Smith & C & 2349\\
Isabel Atkinson & Lab & 693\\
Ian Henderson & Ind & 343\\
\end{tabular*}

\council{Newham}

\subsubsection*{Stratford and New Town \hspace*{\fill}\nolinebreak[1]%
\enspace\hspace*{\fill}
\finalhyphendemerits=0
[7th May]}

\index{Stratford and New Town , Newham@Stratford \& New Town, \emph{Newham}}

Disqualification of Charlene McLean (Lab) for non-attendance.

\noindent
\begin{tabular*}{\columnwidth}{@{\extracolsep{\fill}} p{0.545\columnwidth} >{\itshape}l r @{\extracolsep{\fill}}}
Charlene McLean & Lab & 4607\\
Matthew Gass & C & 1778\\
Isabelle Anderson & Grn & 1170\\
Jamie McKenzie & UKIP & 403\\
Joe Mettle & CPA & 99\\
Bob Severn & TUSC & 70\\
\end{tabular*}

\subsubsection*{Boleyn \hspace*{\fill}\nolinebreak[1]%
\enspace\hspace*{\fill}
\finalhyphendemerits=0
[3rd December]}

\index{Boleyn , Newham@Boleyn, \emph{Newham}}

Death of Charity Fiberesima (Lab).

\noindent
\begin{tabular*}{\columnwidth}{@{\extracolsep{\fill}} p{0.545\columnwidth} >{\itshape}l r @{\extracolsep{\fill}}}
Veronica Oakeshott & Lab & 1440\\
Sheree Miller & LD & 181\\
\sloppyword{Emmanuel Finndoro-Obasi} & C & 171\\
Frankie-Rose Taylor & Grn & 117\\
David Mears & UKIP & 78\\
Diana Ofori & Ind & 10\\
\end{tabular*}

\council{Tower Hamlets}

AWP = Animal Welfare Party

RFAC = Red Flag Anti-Corruption

SthgNew = Something New

THF = Tower Hamlets First

\subsubsection*{Mayor of Tower Hamlets \hspace*{\fill}\nolinebreak[1]%
\enspace\hspace*{\fill}
\finalhyphendemerits=0
[11th June; Lab gain from THF]}

\index{Elected Mayors!Tower Hamlets}

Void election of Lutfur Rahman (THF): reported personally guilty and guilty by his agents of corrupt and illegal practices (personation; postal vote fraud; registration fraud; making false statements about a candidate; paying canvassers; bribery; undue spiritual influence) during the 2014 election of the Mayor of Tower Hamlets.

\noindent
\begin{tabular*}{\columnwidth}{@{\extracolsep{\fill}} p{0.53\columnwidth} >{\itshape}l r @{\extracolsep{\fill}}}
\emph{First preferences}\\
John Biggs & Lab & 27255\\
Rabina Khan & Ind & 25763\\
Peter Golds & C & 5940\\
John Foster & Grn & 2678\\
Elaine Bagshaw & LD & 2152\\
Andy Erlam & RFAC & 1768\\
Nicholas McQueen & UKIP & 1669\\
Hafiz Kadir & Ind & 316\\
Vanessa Hudson & AWP & 305\\
\sloppyword{Mohammed Rahman Nanu} & Ind & 292\\
\end{tabular*}

\noindent
\begin{tabular*}{\columnwidth}{@{\extracolsep{\fill}} p{0.545\columnwidth} >{\itshape}l r @{\extracolsep{\fill}}}
\emph{Runoff}\\
John Biggs & Lab & 32754\\
Rabina Khan & Ind & 26384\\
\end{tabular*}

\columnbreak

\subsubsection*{Stepney Green \hspace*{\fill}\nolinebreak[1]%
\enspace\hspace*{\fill}
\finalhyphendemerits=0
[11th June; Lab gain from THF]}

\index{Stepney Green , Tower Hamlets@Stepney Green, \emph{Tower Hamlets}}

Disqualification of Alibor Choudhury (THF): reported guilty of corrupt and illegal practices (bribery; making false statements about a candidate; paying canvassers) during the 2014 election of the Mayor of Tower Hamlets.

\noindent
\begin{tabular*}{\columnwidth}{@{\extracolsep{\fill}} p{0.5\columnwidth} >{\itshape}l r @{\extracolsep{\fill}}}
Sabina Akhtar & Lab & 1643\\
Abu Chowdhury & Ind & 1472\\
Kirsty Chestnutt & Grn & 272\\
Paul Shea & UKIP & 203\\
Safiul Azam & C & 158\\
Will Dyer & LD & 114\\
Jessie MacNeil-Brown & SthgNew & 40\\
\end{tabular*}

\council{Westminster}

BakerSt = Baker Street: No Two Ways

\subsubsection*{Warwick \hspace*{\fill}\nolinebreak[1]%
\enspace\hspace*{\fill}
\finalhyphendemerits=0
[7th May]}

\index{Warwick , Westminster@Warwick, \emph{Westminster}}

Resignation of Ed Argar (C).

\noindent
\begin{tabular*}{\columnwidth}{@{\extracolsep{\fill}} p{0.545\columnwidth} >{\itshape}l r @{\extracolsep{\fill}}}
Jacqui Wilkinson & C & 2397\\
Sophia Eglin & Lab & 1216\\
Mohammad Bhatti & UKIP & 207\\
\end{tabular*}

\subsubsection*{Harrow Road \hspace*{\fill}\nolinebreak[1]%
\enspace\hspace*{\fill}
\finalhyphendemerits=0
[23rd July]}

\index{Harrow Road , Westminster@Harrow Rd., \emph{Westminster}}

Resignation of Nilavra Mukerji (Lab).

\noindent
\begin{tabular*}{\columnwidth}{@{\extracolsep{\fill}} p{0.545\columnwidth} >{\itshape}l r @{\extracolsep{\fill}}}
Tim Roca & Lab & 1139\\
Wilford Augustus & C & 334\\
Robert Stephenson & UKIP & 38\\
\end{tabular*}

\subsubsection*{Bryanston and Dorset Square \hspace*{\fill}\nolinebreak[1]%
\enspace\hspace*{\fill}
\finalhyphendemerits=0
[22nd October]}

\index{Bryanston and Dorset Square , Westminster@Bryanston \& Dorset Square, \emph{Westminster}}

Death of Audrey Lewis (C).

\noindent
\begin{tabular*}{\columnwidth}{@{\extracolsep{\fill}} p{0.545\columnwidth} >{\itshape}l r @{\extracolsep{\fill}}}
Julia Alexander & C & 582\\
Steve Dollond & BakerSt & 218\\
Ananthi Paskaralingam & Lab & 167\\
Hugh Small & Grn & 116\\
Martin Thompson & LD & 46\\
Jill de Quincey & UKIP & 42\\
\end{tabular*}

\section{South London}

\council{Croydon}

\subsubsection*{Selhurst \hspace*{\fill}\nolinebreak[1]%
\enspace\hspace*{\fill}
\finalhyphendemerits=0
[5th March]}

\index{Selhurst , Croydon@Selhurst, \emph{Croydon}}

Death of Gerry Ryan (Lab).

\noindent
\begin{tabular*}{\columnwidth}{@{\extracolsep{\fill}} p{0.545\columnwidth} >{\itshape}l r @{\extracolsep{\fill}}}
David Wood & Lab & 1517\\
Tirena Gunter & C & 246\\
Tracey Hague & Grn & 148\\
Annette Reid & UKIP & 147\\
Geoff Morley & LD & 65\\
\end{tabular*}

\council{Greenwich}

\subsubsection*{Greenwich West \hspace*{\fill}\nolinebreak[1]%
\enspace\hspace*{\fill}
\finalhyphendemerits=0
[7th May]}

\index{Greenwich West , Greenwich@Greenwich W., \emph{Greenwich}}

Resignation of Matt Pennycook (Lab).

\noindent
\begin{tabular*}{\columnwidth}{@{\extracolsep{\fill}} p{0.545\columnwidth} >{\itshape}l r @{\extracolsep{\fill}}}
Mehboob Khan & Lab & 3430\\
Thomas Turrell & C & 2466\\
Robin Stott & Grn & 1452\\
Sonia Dunlop & LD & 756\\
Paul Butler & UKIP & 422\\
Christina Charles & Ind & 138\\
Sara Kasab & TUSC & 80\\
\end{tabular*}

\council{Kingston upon Thames}

\subsubsection*{Grove \hspace*{\fill}\nolinebreak[1]%
\enspace\hspace*{\fill}
\finalhyphendemerits=0
[7th May]}

\index{Grove , Kingston upon Thames@Grove, \emph{Kingston upon Thames}}

Resignation of Stephen Brister (LD).

\noindent
\begin{tabular*}{\columnwidth}{@{\extracolsep{\fill}} p{0.545\columnwidth} >{\itshape}l r @{\extracolsep{\fill}}}
Rebekah Moll & LD & 1634\\
Jason Hughes & C & 1616\\
Laurie South & Lab & 853\\
Tim Corbett & Grn & 458\\
John Anderson & UKIP & 241\\
John Tolley & Ind & 238\\
Gabrielle Thorpe & TUSC & 44\\
\end{tabular*}

\subsubsection*{Tolworth and Hook Rise \hspace*{\fill}\nolinebreak[1]%
\enspace\hspace*{\fill}
\finalhyphendemerits=0
[7th May]}

\index{Tolworth and Hook Rise , Kingston upon Thames@Tolworth \& Hook Rise, \emph{Kingston upon Thames}}

Resignation of Vicki Harris (LD).

\noindent
\begin{tabular*}{\columnwidth}{@{\extracolsep{\fill}} p{0.545\columnwidth} >{\itshape}l r @{\extracolsep{\fill}}}
Tom Davies & LD & 1729\\
Ronak Pandya & C & 1579\\
Tony Banks & Lab & 898\\
Vic Bellamy & UKIP & 514\\
Nik Way & Grn & 206\\
Mike Briggs & Ind & 120\\
Dan Celardi & TUSC & 29\\
\end{tabular*}

\subsubsection*{Grove \hspace*{\fill}\nolinebreak[1]%
\enspace\hspace*{\fill}
\finalhyphendemerits=0
[16th July]}

\index{Grove , Kingston upon Thames@Grove, \emph{Kingston upon Thames}}

Death of Chrissie Hitchcock (LD).

\noindent
\begin{tabular*}{\columnwidth}{@{\extracolsep{\fill}} p{0.545\columnwidth} >{\itshape}l r @{\extracolsep{\fill}}}
Jon Tolley & LD & 1577\\
Jenny Lewington & C & 688\\
Laurie South & Lab & 223\\
Clare Keogh & Grn & 88\\
John Anderson & UKIP & 58\\
\end{tabular*}

\council{Lambeth}

LeftU = Left Unity

\subsubsection*{Prince's \hspace*{\fill}\nolinebreak[1]%
\enspace\hspace*{\fill}
\finalhyphendemerits=0
[7th May]}

\index{Prince's , Lambeth@Prince's, \emph{Lambeth}}

Resignation of Chris Marsh (Lab).

\noindent
\begin{tabular*}{\columnwidth}{@{\extracolsep{\fill}} p{0.57\columnwidth} >{\itshape}l r @{\extracolsep{\fill}}}
Vaila McClure & Lab & 3452\\
Adrian Hyyryläinen-Trett & LD & 1748\\
Gareth Wallace & C & 1518\\
Marie James & Grn & 901\\
Kingsley Abrams & LeftU & 99\\
Danny Lambert & Soc & 42\\
\end{tabular*}

\council{Richmond upon Thames}

\subsubsection*{Hampton Wick \hspace*{\fill}\nolinebreak[1]%
\enspace\hspace*{\fill}
\finalhyphendemerits=0
[2nd July; LD gain from C]}

\index{Hampton Wick , Richmond upon Thames@\sloppyword{Hampton Wick, \emph{Richmond upon Thames}}}

Resignation of Tania Mathias (C).

\noindent
\begin{tabular*}{\columnwidth}{@{\extracolsep{\fill}} p{0.545\columnwidth} >{\itshape}l r @{\extracolsep{\fill}}}
Geraldine Locke & LD & 1189\\
Jon Hollis & C & 1081\\
Anthony Breslin & Grn & 234\\
Paul Tanto & Lab & 185\\
Sam Naz & UKIP & 69\\
Michael Lloyd & Ind & 7\\
\end{tabular*}

\columnbreak

\council{Southwark}

APP = All People's Party

\subsubsection*{Chaucer \hspace*{\fill}\nolinebreak[1]%
\enspace\hspace*{\fill}
\finalhyphendemerits=0
[7th May]}

\index{Chaucer , Southwark@Chaucer, \emph{Southwark}}

Resignation of Claire Maugham (Lab).

\noindent
\begin{tabular*}{\columnwidth}{@{\extracolsep{\fill}} p{0.545\columnwidth} >{\itshape}l r @{\extracolsep{\fill}}}
Helen Dennis & Lab & 2951\\
William Houngbo & LD & 1532\\
Michael Dowsett & C & 994\\
Gareth Rees & Grn & 564\\
Dean Conway & UKIP & 240\\
Piers Corbyn & Ind & 67\\
Ade Lasaki & APP & 25\\
\end{tabular*}

\subsubsection*{South Camberwell \hspace*{\fill}\nolinebreak[1]%
\enspace\hspace*{\fill}
\finalhyphendemerits=0
[15th October]}

\index{South Camberwell , Southwark@South Camberwell, \emph{Southwark}}

Resignation of Chris Gonde (Lab).

\noindent
\begin{tabular*}{\columnwidth}{@{\extracolsep{\fill}} p{0.545\columnwidth} >{\itshape}l r @{\extracolsep{\fill}}}
Octavia Lamb & Lab & 1243\\
Eleanor Margolies & Grn & 441\\
Benjamin Maitland & LD & 223\\
Christopher Mottau & C & 200\\
Stephen Govier & APP & 39\\
\end{tabular*}

\council{Sutton}

\subsubsection*{Wallington South \hspace*{\fill}\nolinebreak[1]%
\enspace\hspace*{\fill}
\finalhyphendemerits=0
[11th June]}

\index{Wallington South , Sutton@Wallington S., \emph{Sutton}}

Death of Colin Hall (LD).

\noindent
\begin{tabular*}{\columnwidth}{@{\extracolsep{\fill}} p{0.545\columnwidth} >{\itshape}l r @{\extracolsep{\fill}}}
Steve Cook & LD & 1251\\
Jim Simms & C & 936\\
Sarah Gwynn & Lab & 181\\
Duncan Mattey & Ind & 180\\
Andy Beadle & UKIP & 164\\
Rosa Rajendran & Grn & 122\\
\end{tabular*}

\council{Wandsworth}

\subsubsection*{Earlsfield \hspace*{\fill}\nolinebreak[1]%
\enspace\hspace*{\fill}
\finalhyphendemerits=0
[7th May]}

\index{Earlsfield , Wandsworth@Earlsfield, \emph{Wandsworth}}

Death of Adrian Knowles (C).

\noindent
\begin{tabular*}{\columnwidth}{@{\extracolsep{\fill}} p{0.545\columnwidth} >{\itshape}l r @{\extracolsep{\fill}}}
Angela Graham & C & 4218\\
Paul White & Lab & 3079\\
Liam Morgan & Grn & 514\\
Oliver Bailey & LD & 367\\
Thomas Blackwell & UKIP & 274\\
\end{tabular*}

\section{Greater Manchester}

\council{Bolton}

\subsubsection*{Heaton and Lostock \hspace*{\fill}\nolinebreak[1]%
\enspace\hspace*{\fill}
\finalhyphendemerits=0
[7th May]}

\index{Heaton and Lostock , Bolton@Heaton \& Lostock, \emph{Bolton}}

Resignation of Alan Rushton (C).

Combined with the 2015 ordinary election.
%; see page \pageref{HeatonLostockBolton} for the result.

\subsubsection*{\sloppyword{Westhoughton North and Chew Moor} \hspace*{\fill}\nolinebreak[1]%
\enspace\hspace*{\fill}
\finalhyphendemerits=0
[7th May]}

\index{Westhoughton North and Chew Moor , Bolton@Westhoughton N. \& Chew Moor, \emph{Bolton}}

Resignation of Sean Harkin (Lab).

Combined with the 2015 ordinary election.
%; see page \pageref{WesthoughtonNorthChewMoorBolton} for the result.

\council{Bury}

\subsubsection*{Tottington \hspace*{\fill}\nolinebreak[1]%
\enspace\hspace*{\fill}
\finalhyphendemerits=0
[22nd October; C gain from Lab]}

\index{Tottington , Bury@Tottington, \emph{Bury}}

Resignation of Simon Carter (Lab).

\noindent
\begin{tabular*}{\columnwidth}{@{\extracolsep{\fill}} p{0.545\columnwidth} >{\itshape}l r @{\extracolsep{\fill}}}
Greg Keeley & C & 1046\\
Martin Hayes & Lab & 619\\
Ian Henderson & UKIP & 198\\
David Foss & LD & 87\\
John Southworth & Grn & 54\\
\end{tabular*}

\council{Stockport}

At the May 2015 ordinary election there was an unfilled vacancy in Bredbury and Woodley ward due to the death of Mike Wilson (LD).
\index{Bredbury and Woodley , Stockport@Bredbury \& Woodley, \emph{Stockport}}

\section{Merseyside}

\council{Knowsley}

\subsubsection*{Halewood West \hspace*{\fill}\nolinebreak[1]%
\enspace\hspace*{\fill}
\finalhyphendemerits=0
[7th May]}

\index{Halewood West , Knowsley@Halewood W., \emph{Knowsley}}

Resignation of Tom Fearns (Lab).

Combined with the 2015 ordinary election.
%; see page \pageref{HalewoodWestKnowsley} for the result.

\council{Liverpool}

\subsubsection*{Fazakerley \hspace*{\fill}\nolinebreak[1]%
\enspace\hspace*{\fill}
\finalhyphendemerits=0
[7th May]}

\index{Fazakerley , Liverpool@Fazakerley, \emph{Liverpool}}

Resignation of Louise Ashton-Armstrong (Lab).

Combined with the 2015 ordinary election.
%; see page \pageref{FazakerleyLiverpool} for the result.

\subsection{St Helens}
\index{Saint Helens@St Helens}

\subsubsection*{Haydock \hspace*{\fill}\nolinebreak[1]%
\enspace\hspace*{\fill}
\finalhyphendemerits=0
[7th May]}

\index{Haydock , Saint Helens@Haydock, \emph{St Helens}}

Death of Bill Anderton (Lab).

Combined with the 2015 ordinary election.
%; see page \pageref{HaydockStHelens} for the result.

\council{Sefton}

\subsubsection*{Ravenmeols \hspace*{\fill}\nolinebreak[1]%
\enspace\hspace*{\fill}
\finalhyphendemerits=0
[7th May]}

\index{Ravenmeols , Sefton@Ravenmeols, \emph{Sefton}}

Resignation of Peter Maguire (Lab).

Combined with the 2015 ordinary election.
%; see page \pageref{RavenmeolsSefton} for the result.

\section{South Yorkshire}

\council{Barnsley}

Yorks1st = Yorkshire First

\subsubsection*{Dearne North \hspace*{\fill}\nolinebreak[1]%
\enspace\hspace*{\fill}
\finalhyphendemerits=0
[27th August]}

\index{Dearne North , Barnsley@Dearne N., \emph{Barnsley}}

Death of Dave Sim (Lab).

\noindent
\begin{tabular*}{\columnwidth}{@{\extracolsep{\fill}} p{0.53\columnwidth} >{\itshape}l r @{\extracolsep{\fill}}}
Annette Gollick & Lab & 817\\
Jim Johnson & UKIP & 140\\
Tony Devoy & Yorks1st & 115\\
Karen Fletcher & TUSC & 55\\
Lee Ogden & C & 43\\
\end{tabular*}

\council{Rotherham}

At the May 2015 ordinary election there were unfilled vacancies in Silverwood and Valley wards due to the resignations of Roger Stone and Paul Lakin (both Lab) respectively.
\index{Silverwood , Rotherham@Silverwood, \emph{Rotherham}}
\index{Valley , Rotherham@Valley, \emph{Rotherham}}

\subsubsection*{Boston Castle \hspace*{\fill}\nolinebreak[1]%
\enspace\hspace*{\fill}
\finalhyphendemerits=0
[7th May]}

\index{Boston Castle , Rotherham@Boston Castle, \emph{Rotherham}}

Resignation of Mahroof Hussain (Lab).

Combined with the 2015 ordinary election.
%; see page \pageref{BostonCastleRotherham} for the result.

\section{Tyne and Wear}

\council{Newcastle upon Tyne}

\subsubsection*{Denton \hspace*{\fill}\nolinebreak[1]%
\enspace\hspace*{\fill}
\finalhyphendemerits=0
[7th May]}

\index{Denton , Newcastle upon Tyne@Denton, \emph{Newcastle upon Tyne}}

Resignation of Anna Round (Lab).

Combined with the 2015 ordinary election.
%; see page \pageref{DentonNewcastleTyne} for the result.

\subsubsection*{Lemington \hspace*{\fill}\nolinebreak[1]%
\enspace\hspace*{\fill}
\finalhyphendemerits=0
[7th May]}

\index{Lemington , Newcastle upon Tyne@Lemington, \emph{Newcastle upon Tyne}}

Resignation of Louise Sutcliffe (Lab).

Combined with the 2015 ordinary election.
%; see page \pageref{LemingtonNewcastleTyne} for the result.

\council{Sunderland}

At the May 2015 ordinary election there were unfilled vacancies in Houghton and Southwick wards due to the resignation of Gemma Taylor and the death of Christine Shattock (both Lab) respectively.
\index{Houghton , Sunderland@Houghton, \emph{Sunderland}}
\index{Southwick , Sunderland@Southwick, \emph{Sunderland}}

\section{West Midlands}

\council{Birmingham}

\subsubsection*{Sutton New Hall \hspace*{\fill}\nolinebreak[1]%
\enspace\hspace*{\fill}
\finalhyphendemerits=0
[7th May]}

\index{Sutton New Hall , Birmingham@Sutton New Hall, \emph{Birmingham}}

Resignation of Guy Roberts (C).

Combined with the 2015 ordinary election.
%; see page \pageref{SuttonNewHallBirmingham} for the result.

\council{Sandwell}

\subsubsection*{Newton \hspace*{\fill}\nolinebreak[1]%
\enspace\hspace*{\fill}
\finalhyphendemerits=0
[9th July]}

\index{Newton , Sandwell@Newton, \emph{Sandwell}}

Resignation of Tony Underhill (Lab).

\noindent
\begin{tabular*}{\columnwidth}{@{\extracolsep{\fill}} p{0.545\columnwidth} >{\itshape}l r @{\extracolsep{\fill}}}
Keith Allcock & Lab & 1152\\
Tony Ward & C & 452\\
Steve Latham & UKIP & 310\\
Murray Abbott & Grn & 36\\
\end{tabular*}

\subsubsection*{Blackheath \hspace*{\fill}\nolinebreak[1]%
\enspace\hspace*{\fill}
\finalhyphendemerits=0
[10th September]}

\index{Blackheath , Sandwell@Blackheath, \emph{Sandwell}}

Death of Malcolm Bridges (Lab).

\noindent
\begin{tabular*}{\columnwidth}{@{\extracolsep{\fill}} p{0.545\columnwidth} >{\itshape}l r @{\extracolsep{\fill}}}
Danny Millard & Lab & 915\\
Shirley Ching & C & 544\\
Ian Keeling & UKIP & 287\\
Ben Groom & Grn & 35\\
\end{tabular*}

\section{West Yorkshire}

\council{Kirklees}

\subsubsection*{Greenhead \hspace*{\fill}\nolinebreak[1]%
\enspace\hspace*{\fill}
\finalhyphendemerits=0
[7th May]}

\index{Greenhead , Kirklees@Greenhead, \emph{Kirklees}}

Resignation of Mahroof Hussain (Lab).

Combined with the 2015 ordinary election.
%; see page \pageref{GreenheadKirklees} for the result.

\council{Wakefield}

Yorks1st = Yorkshire First

\subsubsection*{Pontefract North \hspace*{\fill}\nolinebreak[1]%
\enspace\hspace*{\fill}
\finalhyphendemerits=0
[24th September]}

\index{Pontefract North , Wakefield@Pontefract N., \emph{Wakefield}}

Resignation of Paula Sherriff (Lab).

\noindent
\begin{tabular*}{\columnwidth}{@{\extracolsep{\fill}} p{0.53\columnwidth} >{\itshape}l r @{\extracolsep{\fill}}}
Lorna Malkin & Lab & 909\\
Nathan Garbutt & UKIP & 453\\
Anthony Hill & C & 199\\
Lucy Brown & Yorks1st & 124\\
Daniel Woodlock & LD & 86\\
Daniel Dearden & TUSC & 24\\
\end{tabular*}

\section{Bedfordshire}

\council{Bedford}

At the May 2015 ordinary election there was an unfilled vacancy in Great Barford ward due to the death of Carole Ellis (C).
\index{Great Barford , Bedford@Great Barford, \emph{Bedford}}

\section{Berkshire}

\council{Bracknell Forest}

At the May 2015 ordinary election there was an unfilled vacancy in Winkfield and Cranbourne ward due to the death of Alan Kendall (C).
\index{Winkfield and Cranbourne , Bracknell Forest@Winkfield \& Cranbourne, \emph{Bracknell Forest}}

\council{West Berkshire}

\subsubsection*{Purley on Thames \hspace*{\fill}\nolinebreak[1]%
\enspace\hspace*{\fill}
\finalhyphendemerits=0
[Wednesday 28th January]}

\index{Purley on Thames , West Berkshire@Purley on Thames, \emph{W. Berks.}}

Death of David Betts (C).

\noindent
\begin{tabular*}{\columnwidth}{@{\extracolsep{\fill}} p{0.545\columnwidth} >{\itshape}l r @{\extracolsep{\fill}}}
Rick Jones & C & 936\\
Ian Stevens & Lab & 172\\
Catherine Anderson & UKIP & 163\\
Steve Bown & LD & 104\\
\end{tabular*}

\section{Buckinghamshire}

\council{Aylesbury Vale}

\subsubsection*{Grendon Underwood and Brill \hspace*{\fill}\nolinebreak[1]%
\enspace\hspace*{\fill}
\finalhyphendemerits=0
[Wednesday 23rd December]}

\index{Grendon Underwood and Brill , Aylesbury Vale@Grendon Underwood \& Brill, \emph{Aylesbury Vale}}

Death of John Cartwright (C).

\noindent
\begin{tabular*}{\columnwidth}{@{\extracolsep{\fill}} p{0.545\columnwidth} >{\itshape}l r @{\extracolsep{\fill}}}
Cameron Branston & C & 326\\
Julian Newman & LD & 275\\
Gary Good & UKIP & 148\\
\end{tabular*}

\council{Milton Keynes}

At the May 2015 ordinary election there was an unfilled vacancy in Monkston ward due to the resignation of Subhan Shafiq (LD).
\index{Monkston , Milton Keynes@Monkston, \emph{Milton Keynes}}

\section{Cambridgeshire}

\subsection*{County Council}\index{Cambridgeshire}

\subsubsection*{Bar Hill \hspace*{\fill}\nolinebreak[1]%
\enspace\hspace*{\fill}
\finalhyphendemerits=0
[12th February]}

\index{Bar Hill , Cambridgeshire@Bar Hill, \emph{Cambs.}}

Death of John Reynolds (C).

\noindent
\begin{tabular*}{\columnwidth}{@{\extracolsep{\fill}} p{0.545\columnwidth} >{\itshape}l r @{\extracolsep{\fill}}}
Lynda Harford & C & 787\\
Martin Hale & UKIP & 251\\
Fiona Whelan & LD & 238\\
Alex Smith & Lab & 235\\
Claudia Roland & Grn & 200\\
\end{tabular*}

\subsubsection*{Whittlesey North \hspace*{\fill}\nolinebreak[1]%
\enspace\hspace*{\fill}
\finalhyphendemerits=0
[7th May]}

\index{Whittlesey North , Cambridgeshire@Whittlesey N., \emph{Cambs.}}

Resignation of Martin Curtis (C).

\noindent
\begin{tabular*}{\columnwidth}{@{\extracolsep{\fill}} p{0.545\columnwidth} >{\itshape}l r @{\extracolsep{\fill}}}
Chris Boden & C & 2237\\
Paul Edwards & UKIP & 1131\\
David Chapman & LD & 615\\
\end{tabular*}

\subsubsection*{Wisbech South \hspace*{\fill}\nolinebreak[1]%
\enspace\hspace*{\fill}
\finalhyphendemerits=0
[4th June; C gain from UKIP]}

\index{Wisbech South , Cambridgeshire@Wisbech S., \emph{Cambs.}}

Resignation of Peter Lagoda (Ind elected as UKIP).

\noindent
\begin{tabular*}{\columnwidth}{@{\extracolsep{\fill}} p{0.545\columnwidth} >{\itshape}l r @{\extracolsep{\fill}}}
Samantha Hoy & C & 1020\\
Susan Carson & UKIP & 298\\
Dean Reeves & Lab & 219\\
Josephine Ratcliffe & LD & 61\\
\end{tabular*}

\subsubsection*{Romsey \hspace*{\fill}\nolinebreak[1]%
\enspace\hspace*{\fill}
\finalhyphendemerits=0
[25th June; Lab gain from LD]}

\index{Romsey , Cambridgeshire@Romsey, \emph{Cambs.}}

Resignation of Killian Bourke (LD).

\noindent
\begin{tabular*}{\columnwidth}{@{\extracolsep{\fill}} p{0.545\columnwidth} >{\itshape}l r @{\extracolsep{\fill}}}
Zoe Moghadas & Lab & 829\\
Nichola Martin & LD & 782\\
Debbie Aitchison & Grn & 467\\
Rahatul Raja & C & 100\\
Richard Jeffs & UKIP & 46\\
\end{tabular*}

\subsubsection*{Chatteris \hspace*{\fill}\nolinebreak[1]%
\enspace\hspace*{\fill}
\finalhyphendemerits=0
[15th October]}

\index{Chatteris , Cambridgeshire@Chatteris, \emph{Cambs.}}

Death of Sandra Rylance (UKIP).

\noindent
\begin{tabular*}{\columnwidth}{@{\extracolsep{\fill}} p{0.545\columnwidth} >{\itshape}l r @{\extracolsep{\fill}}}
Richard Mandley & UKIP & 600\\
Alan Melton & C & 590\\
John Freeman & LD & 274\\
\end{tabular*}

\council{Fenland}

At the May 2015 ordinary election there was an unfilled vacancy in Manea ward due to the disqualification (non-attendance) of Mark Archer (Ind).
\index{Manea , Fenland@Manea, \emph{Fenland}}

\council{Huntingdonshire}

\subsubsection*{Huntingdon East \hspace*{\fill}\nolinebreak[1]%
\enspace\hspace*{\fill}
\finalhyphendemerits=0
[10th December; LD gain from UKIP]}

\index{Huntingdon East , Huntingdonshire@Huntingdon E., \emph{Hunts.}}

Resignation of Andrew Hardy (UKIP).

\noindent
\begin{tabular*}{\columnwidth}{@{\extracolsep{\fill}} p{0.545\columnwidth} >{\itshape}l r @{\extracolsep{\fill}}}
Ste Greenall & LD & 844\\
Tom Fletcher & C & 596\\
Martin Cohen & UKIP & 293\\
Duncan Williams & Lab & 141\\
\end{tabular*}

\council{Peterborough}

\subsubsection*{Stanground Central \hspace*{\fill}\nolinebreak[1]%
\enspace\hspace*{\fill}
\finalhyphendemerits=0
[7th May]}

\index{Stanground Central , Peterborough@Stanground C., \emph{Peterborough}}

Resignation of Irene Walsh (C).

Combined with the 2015 ordinary election.
%; see page \pageref{StangroundCentralPeterborough} for the result.

\subsubsection*{West \hspace*{\fill}\nolinebreak[1]%
\enspace\hspace*{\fill}
\finalhyphendemerits=0
[29th October]}

\index{West , Peterborough@West, \emph{Peterborough}}

Resignation of Nick Arculus (C).

\noindent
\begin{tabular*}{\columnwidth}{@{\extracolsep{\fill}} p{0.545\columnwidth} >{\itshape}l r @{\extracolsep{\fill}}}
Lynne Ayres & C & 1174\\
Mohammed Sabir & Lab & 742\\
John Myles & UKIP & 415\\
Malcolm Pollack & LD & 103\\
Alex Airey & Grn & 94\\
\end{tabular*}

\council{South Cambridgeshire}

\subsubsection*{Bourn \hspace*{\fill}\nolinebreak[1]%
\enspace\hspace*{\fill}
\finalhyphendemerits=0
[17th September]}

\index{Bourn , South Cambridgeshire@Bourn, \emph{S. Cambs.}}

Resignation of David Morgan (C).

\noindent
\begin{tabular*}{\columnwidth}{@{\extracolsep{\fill}} p{0.545\columnwidth} >{\itshape}l r @{\extracolsep{\fill}}}
Simon Crocker & C & 579\\
Jeni Sawford & LD & 247\\
Gavin Clayton & Lab & 235\\
Helene Green & UKIP & 121\\
Marcus Pitcaithly & Grn & 64\\
\end{tabular*}

\section{Cheshire}

\council{Cheshire East}

At the May 2015 ordinary election there was an unfilled vacancy in Poynton West and Adlington ward due to the death of Phil Hoyland (C).
\index{Poynton West and Adlington , Cheshire East@Poynton W. \& Adlington, \emph{Ches. E.}}

\subsubsection*{Congleton East \hspace*{\fill}\nolinebreak[1]%
\enspace\hspace*{\fill}
\finalhyphendemerits=0
[29th October]}

\index{Congleton East , Cheshire East@Congleton E., \emph{Ches. E.}}

Death of Peter Mason (C).

\noindent
\begin{tabular*}{\columnwidth}{@{\extracolsep{\fill}} p{0.545\columnwidth} >{\itshape}l r @{\extracolsep{\fill}}}
Geoff Baggott & C & 700\\
Denis Murphy & LD & 542\\
Robert Boston & Lab & 409\\
Dawn Allen & UKIP & 266\\
\end{tabular*}

\section{Cornwall}

\council{Cornwall}

MK = Mebyon Kernow

\subsubsection*{Camborne Treswithian \hspace*{\fill}\nolinebreak[1]%
\enspace\hspace*{\fill}
\finalhyphendemerits=0
[7th May; Lab gain from UKIP]}

\index{Camborne Treswithian , Cornwall@Camborne Treswithian, \emph{Cornwall}}

Resignation of Viv Lewis (UKIP).

\noindent
\begin{tabular*}{\columnwidth}{@{\extracolsep{\fill}} p{0.545\columnwidth} >{\itshape}l r @{\extracolsep{\fill}}}
Jude Robinson & Lab & 538\\
Dave Biggs & C & 530\\
Peter Tisdale & UKIP & 321\\
Anna Pascoe & LD & 268\\
Zoe Fox & MK & 180\\
\end{tabular*}

\subsubsection*{Constantine, Mawnan and Budock \hspace*{\fill}\nolinebreak[1]%
\enspace\hspace*{\fill}
\finalhyphendemerits=0
[7th May]}

\index{Constantine, Mawnan and Budock , Cornwall@Constantine, Mawnan \& Budock, \emph{Cornwall}}

Resignation of Neil Hatton (C).

\noindent
\begin{tabular*}{\columnwidth}{@{\extracolsep{\fill}} p{0.545\columnwidth} >{\itshape}l r @{\extracolsep{\fill}}}
John Bastin & C & 1431\\
Rowland Abram & LD & 434\\
Chris Kinder & UKIP & 416\\
Adam Crickett & Lab & 384\\
Charlotte Evans & MK & 340\\
\end{tabular*}

\subsubsection*{Camborne Pendarves \hspace*{\fill}\nolinebreak[1]%
\enspace\hspace*{\fill}
\finalhyphendemerits=0
[20th August; C gain from UKIP]}

\index{Camborne Pendarves , Cornwall@Camborne Pendarves, \emph{Cornwall}}

Resignation of Harry Blakely (UKIP).

\noindent
\begin{tabular*}{\columnwidth}{@{\extracolsep{\fill}} p{0.545\columnwidth} >{\itshape}l r @{\extracolsep{\fill}}}
John Herd & C & 325\\
Nathan Billings & LD & 311\\
Val Dalley & Lab & 220\\
Michael Pascoe & UKIP & 89\\
Deborah Fox & MK & 85\\
Jacqueline Merrick & Grn & 31\\
Peter Channon & Ind & 13\\
\end{tabular*}

\section{Cumbria}

\subsection*{County Council}\index{Cumbria}

\subsubsection*{Walney South \hspace*{\fill}\nolinebreak[1]%
\enspace\hspace*{\fill}
\finalhyphendemerits=0
[16th April]}

\index{Walney South , Cumbria@Walney S., \emph{Cumbria}}

Resignation of Mandy Telford (Lab).

\noindent
\begin{tabular*}{\columnwidth}{@{\extracolsep{\fill}} p{0.545\columnwidth} >{\itshape}l r @{\extracolsep{\fill}}}
Frank Cassidy & Lab & 727\\
Graham Pritchard & UKIP & 239\\
Greg Peers & C & 181\\
\end{tabular*}

\subsubsection*{Cartmel \hspace*{\fill}\nolinebreak[1]%
\enspace\hspace*{\fill}
\finalhyphendemerits=0
[7th May]}

\index{Cartmel , Cumbria@Cartmel, \emph{Cumbria}}

Resignation of Roderick Wilson (LD).

\noindent
\begin{tabular*}{\columnwidth}{@{\extracolsep{\fill}} p{0.545\columnwidth} >{\itshape}l r @{\extracolsep{\fill}}}
Sue Sanderson & LD & 1394\\
Tom Harvey & C & 1355\\
Gwen Harrison & Grn & 276\\
\end{tabular*}

\subsubsection*{Greystoke and Hesket \hspace*{\fill}\nolinebreak[1]%
\enspace\hspace*{\fill}
\finalhyphendemerits=0
[16th July]}

\index{Greystoke and Hesket , Cumbria@Greystoke \& Hesket, \emph{Cumbria}}

Resignation of Bert Richardson (C).

\noindent
\begin{tabular*}{\columnwidth}{@{\extracolsep{\fill}} p{0.6\columnwidth} >{\itshape}l r @{\extracolsep{\fill}}}
Thomas Wentworth Waites & C & 635\\
Judith Derbyshire & LD & 518\\
\end{tabular*}

\subsubsection*{Howgate \hspace*{\fill}\nolinebreak[1]%
\enspace\hspace*{\fill}
\finalhyphendemerits=0
[15th October]}

\index{Howgate , Cumbria@Howgate, \emph{Cumbria}}

Resignation of Sue Hayman (Lab).

\noindent
\begin{tabular*}{\columnwidth}{@{\extracolsep{\fill}} p{0.545\columnwidth} >{\itshape}l r @{\extracolsep{\fill}}}
Gillian Troughton & Lab & 435\\
Andrew Wonnacott & C & 304\\
Eric Atkinson & UKIP & 176\\
\end{tabular*}

\council{Barrow-in-Furness}

\subsubsection*{Risedale \hspace*{\fill}\nolinebreak[1]%
\enspace\hspace*{\fill}
\finalhyphendemerits=0
[29th October]}

\index{Risedale , Barrow-in-Furness@Risedale, \emph{Barrow-in-Furness}}

Resignation of Lesley Graham (Lab).

\noindent
\begin{tabular*}{\columnwidth}{@{\extracolsep{\fill}} p{0.545\columnwidth} >{\itshape}l r @{\extracolsep{\fill}}}
Michael Cassells & Lab & 428\\
Colin Rudd & UKIP & 193\\
Carole Friend & C & 187\\
\end{tabular*}

\section{Derbyshire}

\subsection*{County Council}\index{Derbyshire}

\subsubsection*{Brimington \hspace*{\fill}\nolinebreak[1]%
\enspace\hspace*{\fill}
\finalhyphendemerits=0
[5th February]}

\index{Brimington , Derbyshire@Brimington, \emph{Derbys.}}

Death of Walter Burrows (Lab).

\noindent
\begin{tabular*}{\columnwidth}{@{\extracolsep{\fill}} p{0.545\columnwidth} >{\itshape}l r @{\extracolsep{\fill}}}
Tricia Gilby & Lab & 1293\\
Paul Stone & UKIP & 380\\
Mick Bagshaw & Ind & 157\\
John Ahern & LD & 135\\
Lewis Preston & C & 120\\
\end{tabular*}

\subsubsection*{Ashbourne \hspace*{\fill}\nolinebreak[1]%
\enspace\hspace*{\fill}
\finalhyphendemerits=0
[7th May]}

\index{Ashbourne , Derbyshire@Ashbourne, \emph{Derbys.}}

Resignation of Andrew Lewer (C).

\noindent
\begin{tabular*}{\columnwidth}{@{\extracolsep{\fill}} p{0.545\columnwidth} >{\itshape}l r @{\extracolsep{\fill}}}
Stephen Bull & C & 4715\\
Simon Meredith & Lab & 965\\
Andrew White & Grn & 647\\
David Rowe & LD & 543\\
\end{tabular*}

\subsubsection*{Derwent Valley \hspace*{\fill}\nolinebreak[1]%
\enspace\hspace*{\fill}
\finalhyphendemerits=0
[24th September]}

\index{Derwent Valley , Derbyshire@Derwent Valley, \emph{Derbys.}}

Death of Mike Longden (C).

\noindent
\begin{tabular*}{\columnwidth}{@{\extracolsep{\fill}} p{0.545\columnwidth} >{\itshape}l r @{\extracolsep{\fill}}}
Jo Wild & C & 1107\\
Martin Rutter & Lab & 466\\
Michael Crapper & LD & 314\\
Mike Dawson & UKIP & 285\\
\end{tabular*}

\council{Bolsover}

\subsubsection*{Bolsover North West \hspace*{\fill}\nolinebreak[1]%
\enspace\hspace*{\fill}
\finalhyphendemerits=0
[8th January]}

\index{Bolsover North West , Bolsover@Bolsover N.W., \emph{Bolsover}}

Death of Thomas Rodda (Lab).

\noindent
\begin{tabular*}{\columnwidth}{@{\extracolsep{\fill}} p{0.545\columnwidth} >{\itshape}l r @{\extracolsep{\fill}}}
Sue Statter & Lab & 174\\
John Bagshaw & UKIP & 153\\
Maxine Hunt & C & 60\\
\end{tabular*}

\subsubsection*{Bolsover South \hspace*{\fill}\nolinebreak[1]%
\enspace\hspace*{\fill}
\finalhyphendemerits=0
[8th October]}

\index{Bolsover South , Bolsover@Bolsover S., \emph{Bolsover}}

Death of James Hall (Lab).

\noindent
\begin{tabular*}{\columnwidth}{@{\extracolsep{\fill}} p{0.545\columnwidth} >{\itshape}l r @{\extracolsep{\fill}}}
Pat Cooper & Lab & 232\\
John Bagshaw & UKIP & 127\\
Juliet Armstrong & C & 109\\
Jon Dale & TUSC & 78\\
\end{tabular*}

\council{High Peak}

At the May 2015 ordinary election there was an unfilled vacancy in Limestone Peak ward due to the death of Derek Udale (C).
\index{Limestone Peak , High Peak@Limestone Peak, \emph{High Peak}}

\council{North East Derbyshire}

\subsubsection*{Coal Aston \hspace*{\fill}\nolinebreak[1]%
\enspace\hspace*{\fill}
\finalhyphendemerits=0
[10th December]}

\index{Coal Aston , North East Derbyshire@Coal Aston, \emph{N.E. Derbys.}}

Resignation of Gary Griffin-Chappel (C).

\noindent
\begin{tabular*}{\columnwidth}{@{\extracolsep{\fill}} p{0.545\columnwidth} >{\itshape}l r @{\extracolsep{\fill}}}
Gareth Hopkinson & C & 606\\
David Cheetham & Lab & 304\\
\end{tabular*}

\subsubsection*{Holmewood and Heath \hspace*{\fill}\nolinebreak[1]%
\enspace\hspace*{\fill}
\finalhyphendemerits=0
[10th December]}

\index{Holmewood and Heath , North East Derbyshire@Holmewood \& Heath, \emph{N.E. Derbys.}}

Death of Tricia Williams (Lab).

\noindent
\begin{tabular*}{\columnwidth}{@{\extracolsep{\fill}} p{0.545\columnwidth} >{\itshape}l r @{\extracolsep{\fill}}}
Suzy Cornwell & Lab & 282\\
Caroline Ellis & C & 75\\
\end{tabular*}

\council{South Derbyshire}

At the May 2015 ordinary election there was an unfilled vacancy in Hilton ward due to the death of Mick Bale (C).
\index{Hilton , South Derbyshire@Hilton, \emph{S. Derbys.}}

\section{Devon}

\council{Exeter}

\subsubsection*{Pinhoe \hspace*{\fill}\nolinebreak[1]%
\enspace\hspace*{\fill}
\finalhyphendemerits=0
[13th August; C gain from Lab]}

\index{Pinhoe , Exeter@Pinhoe, \emph{Exeter}}

Resignation of Simon Bowkett (Lab).

\noindent
\begin{tabular*}{\columnwidth}{@{\extracolsep{\fill}} p{0.545\columnwidth} >{\itshape}l r @{\extracolsep{\fill}}}
Cynthia Thompson & C & 755\\
David Harvey & Lab & 749\\
Alison Sheridan & UKIP & 143\\
Michael Payne & LD & 63\\
John Moreman & Grn & 62\\
David Smith & Ind & 11\\
\end{tabular*}

\council{South Hams}

\subsubsection*{Totnes \hspace*{\fill}\nolinebreak[1]%
\enspace\hspace*{\fill}
\finalhyphendemerits=0
[8th October]}

\index{Totnes , South Hams@Totnes, \emph{S. Hams}}

Resignation of Barrie Wood (Grn).

\noindent
\begin{tabular*}{\columnwidth}{@{\extracolsep{\fill}} p{0.545\columnwidth} >{\itshape}l r @{\extracolsep{\fill}}}
John Green & Grn & 570\\
John Birch & LD & 558\\
Eleanor Cohen & Lab & 432\\
Ralph Clark & C & 268\\
Peter Pirnie & Ind & 63\\
\end{tabular*}

\council{Torbay}

At the May 2015 ordinary election there was an unfilled vacancy in Blatchcombe ward due to the death of Jeanette Richards (C).
\index{Blatchcombe , Torbay@Blatchcombe, \emph{Torbay}}

\subsubsection*{Clifton-with-Maidenway \hspace*{\fill}\nolinebreak[1]%
\enspace\hspace*{\fill}
\finalhyphendemerits=0
[5th November]}

\index{Clifton-with-Maidenway , Torbay@Clifton-with-Maidenway, \emph{Torbay}}

Death of Ruth Pentney (LD).

\noindent
\begin{tabular*}{\columnwidth}{@{\extracolsep{\fill}} p{0.545\columnwidth} >{\itshape}l r @{\extracolsep{\fill}}}
Adrian Sanders & LD & 1096\\
Richard Barnby & C & 234\\
Anthony Rayner & UKIP & 158\\
Eddie Harris & Lab & 53\\
Stephen Pocock & Grn & 43\\
\end{tabular*}

\council{West Devon}

At the May 2015 ordinary election there was an unfilled vacancy in Thrushel ward due to the death of Donald Horn (C).

\section{Dorset}

\subsection*{County Council}\index{Dorset}

\subsubsection*{Rodwell \hspace*{\fill}\nolinebreak[1]%
\enspace\hspace*{\fill}
\finalhyphendemerits=0
[12th November; Grn gain from Lab]}

\index{Rodwell , Dorset@Rodwell, \emph{Dorset}}

Resignation of Dan Brember (Lab).

\noindent
\begin{tabular*}{\columnwidth}{@{\extracolsep{\fill}} p{0.545\columnwidth} >{\itshape}l r @{\extracolsep{\fill}}}
Clare Sutton & Grn & 663\\
Richard Nickinson & C & 561\\
Hazel Priest & Lab & 417\\
Francis Drake & UKIP & 174\\
Graham Winter & LD & 87\\
\end{tabular*}

\columnbreak

\council{Bournemouth}

BrnmthIA = Bournemouth Independent Alliance

\subsubsection*{Kinson South (2) \hspace*{\fill}\nolinebreak[1]%
\enspace\hspace*{\fill}
\finalhyphendemerits=0
[10th December]}

\index{Kinson South , Bournemouth@Kinson S., \emph{Bournemouth}}

Void election of Norman Decent and Roger Marley (both C): voters given ballot papers for Kinson North ward in error.

\noindent
\begin{tabular*}{\columnwidth}{@{\extracolsep{\fill}} p{0.48\columnwidth} >{\itshape}l r @{\extracolsep{\fill}}}
Norman Decent & C & 520\\
Roger Marley & C & 509\\
Beryl Baxter & Lab & 471\\
Mel Semple & Lab & 371\\
Duane Farr & UKIP & 313\\
Philip Davenport & BrnmthIA & 168\\
Roger West & BrnmthIA & 116\\
Carla Gregory-May & Grn & 63\\
Gill Pardy & LD & 61\\
Phil Dunn & LD & 60\\
Geoffrey Darnton & Grn & 54\\
Kevin Dixon & Patria & 8\\
\end{tabular*}

\council{Christchurch}

\subsubsection*{North Highcliffe and Walkford \hspace*{\fill}\nolinebreak[1]%
\enspace\hspace*{\fill}
\finalhyphendemerits=0
[18th June]}

\index{North Highcliffe and Walkford , Christchurch@North Highcliffe \& Walkford, \emph{Christchurch}}

Ordinary election postponed from 7th May: death of candidate Richard Walls (Lab).

\noindent
\begin{tabular*}{\columnwidth}{@{\extracolsep{\fill}} p{0.545\columnwidth} >{\itshape}l r @{\extracolsep{\fill}}}
Sally Derham Wilkes & C & 793\\
Nick Geary & C & 775\\
Robin Grey & UKIP & 315\\
Janet Hatton & UKIP & 288\\
Donald Barr & Lab & 143\\
Gareth Walls & Lab & 132\\
\end{tabular*}

\council{North Dorset}

At the May 2015 ordinary election there was an unfilled vacancy in Blandford Hilltop ward due to the death of Tony Harrocks (C).
\index{Blandford Hilltop , North Dorset@Blandford Hilltop, \emph{N. Dorset}}

\section{Durham}

\council{Darlington}

At the May 2015 ordinary election there was an unfilled vacancy in Park East ward due to the death of Joe Lyonette (Lab).
\index{Park East , Darlington@Park E., \emph{Darlington}}

\council{Durham}

NEast = North East Party

\subsubsection*{Barnard Castle West \hspace*{\fill}\nolinebreak[1]%
\enspace\hspace*{\fill}
\finalhyphendemerits=0
[7th May]}

\index{Barnard Castle West , Durham@Barnard Castle W., \emph{Durham}}

Resignation of Barbara Harrison (C).

\noindent
\begin{tabular*}{\columnwidth}{@{\extracolsep{\fill}} p{0.545\columnwidth} >{\itshape}l r @{\extracolsep{\fill}}}
Edward Henderson & C & 2518\\
Thomas Robinson & Grn & 972\\
Philip Hunt & Lab & 878\\
\end{tabular*}

\subsubsection*{Ferryhill \hspace*{\fill}\nolinebreak[1]%
\enspace\hspace*{\fill}
\finalhyphendemerits=0
[7th May]}

\index{Ferryhill , Durham@Ferryhill, \emph{Durham}}

Resignation of Pat McCourt (Lab).

\noindent
\begin{tabular*}{\columnwidth}{@{\extracolsep{\fill}} p{0.545\columnwidth} >{\itshape}l r @{\extracolsep{\fill}}}
John Lindsay & Lab & 2266\\
Joseph Makepeace & Ind & 1969\\
William Lawrence & Grn & 316\\
\end{tabular*}

\subsubsection*{Sherburn \hspace*{\fill}\nolinebreak[1]%
\enspace\hspace*{\fill}
\finalhyphendemerits=0
[7th May]}

\index{Sherburn , Durham@Sherburn, \emph{Durham}}

Resignation of Stephen Guy (Lab).

\noindent
\begin{tabular*}{\columnwidth}{@{\extracolsep{\fill}} p{0.545\columnwidth} >{\itshape}l r @{\extracolsep{\fill}}}
William Kellett & Lab & 2218\\
Michael Fishwick & C & 965\\
Andrew Tibbs & LD & 531\\
Joanna Smith & Grn & 508\\
\end{tabular*}

\subsubsection*{Willington and Hunwick \hspace*{\fill}\nolinebreak[1]%
\enspace\hspace*{\fill}
\finalhyphendemerits=0
[7th May]}

\index{Willington and Hunwick , Durham@Willington \& Hunwick, \emph{Durham}}

Resignation of Joe Buckham (Lab).

\noindent
\begin{tabular*}{\columnwidth}{@{\extracolsep{\fill}} p{0.545\columnwidth} >{\itshape}l r @{\extracolsep{\fill}}}
Fraser Tinsley & Lab & 2169\\
Matthew Todd & Ind & 1512\\
Mark Quinn & Grn & 301\\
\end{tabular*}

\subsubsection*{Shotton and South Hetton \hspace*{\fill}\nolinebreak[1]%
\enspace\hspace*{\fill}
\finalhyphendemerits=0
[20th August]}

\index{Shotton and South Hetton , Durham@Shotton \& South Hetton, \emph{Durham}}

Death of Robin Todd (Lab).

\noindent
\begin{tabular*}{\columnwidth}{@{\extracolsep{\fill}} p{0.545\columnwidth} >{\itshape}l r @{\extracolsep{\fill}}}
Alan Liversidge & Lab & 595\\
Ted Hall & NEast & 214\\
Lee-James Harris & UKIP & 131\\
Michael Anderson & LD & 107\\
Beaty Bainbridge & C & 67\\
Martie Warin & Grn & 19\\
\end{tabular*}

\section{East Sussex}

\subsection*{County Council}\index{East Sussex}

\subsubsection*{Old Hastings and Tressell \hspace*{\fill}\nolinebreak[1]%
\enspace\hspace*{\fill}
\finalhyphendemerits=0
[9th July]}

\index{Old Hastings and Tressell , East Sussex@Old Hastings \& Tressell, \emph{E. Sussex}}

Death of Jeremy Birch (Lab).

\noindent
\begin{tabular*}{\columnwidth}{@{\extracolsep{\fill}} p{0.545\columnwidth} >{\itshape}l r @{\extracolsep{\fill}}}
Tania Charman & Lab & 961\\
Robert Cooke & C & 368\\
Sebastian Norton & UKIP & 174\\
Andrea Needham & Grn & 149\\
Stewart Rayment & LD & 48\\
\end{tabular*}

\council{Hastings}

\subsubsection*{Central St Leonards \hspace*{\fill}\nolinebreak[1]%
\enspace\hspace*{\fill}
\finalhyphendemerits=0
[9th July]}

\index{Central Saint Leonards , Hastings@Central St Leonards, \emph{Hastings}}

Death of Jeremy Birch (Lab).

\noindent
\begin{tabular*}{\columnwidth}{@{\extracolsep{\fill}} p{0.545\columnwidth} >{\itshape}l r @{\extracolsep{\fill}}}
Terry Dowling & Lab & 481\\
John Rankin & C & 259\\
Clive Gross & Ind & 184\\
Kevin Hill & UKIP & 77\\
Alan Dixon & Grn & 75\\
Susan Tait & LD & 17\\
\end{tabular*}

\subsubsection*{St Helens \hspace*{\fill}\nolinebreak[1]%
\enspace\hspace*{\fill}
\finalhyphendemerits=0
[9th July]}

\index{Saint Helens , Hastings@St Helens, \emph{Hastings}}

Resignation of Matthew Lock (C).

\noindent
\begin{tabular*}{\columnwidth}{@{\extracolsep{\fill}} p{0.545\columnwidth} >{\itshape}l r @{\extracolsep{\fill}}}
Martin Clarke & C & 663\\
Graham Crane & Lab & 557\\
Gary Spencer-Holmes & LD & 136\\
Kenneth Pankhurst & UKIP & 120\\
Christopher Petts & Grn & 48\\
\end{tabular*}

\council{Rother}

\subsubsection*{Battle Town \hspace*{\fill}\nolinebreak[1]%
\enspace\hspace*{\fill}
\finalhyphendemerits=0
[16th July; LD gain from C]}

\index{Battle Town , Rother@Battle Town, \emph{Rother}}

Resignation of Martin Noakes (C).

\noindent
\begin{tabular*}{\columnwidth}{@{\extracolsep{\fill}} p{0.545\columnwidth} >{\itshape}l r @{\extracolsep{\fill}}}
Kevin Dixon & LD & 751\\
Hazel Sharman & C & 342\\
Tony Smith & UKIP & 107\\
Timothy MacPherson & Lab & 100\\
\end{tabular*}

\council{Wealden}

At the May 2015 ordinary election there was an unfilled vacancy in Buxted and Maresfield ward due to the resignation of Norman Buck (C).
\index{Buxted and Maresfield , Wealden@Buxted \& Maresfield, \emph{Wealden}}

\subsubsection*{Crowborough West \hspace*{\fill}\nolinebreak[1]%
\enspace\hspace*{\fill}
\finalhyphendemerits=0
[22nd January]}

\index{Crowborough West , Wealden@Crowborough W., \emph{Wealden}}

Death of Maj Anthony Quin (C).

\noindent
\begin{tabular*}{\columnwidth}{@{\extracolsep{\fill}} p{0.545\columnwidth} >{\itshape}l r @{\extracolsep{\fill}}}
Jeannette Towey & C & 465\\
Simon Staveley & UKIP & 327\\
\end{tabular*}

\subsubsection*{Hellingly \hspace*{\fill}\nolinebreak[1]%
\enspace\hspace*{\fill}
\finalhyphendemerits=0
[29th October; LD gain from C]}

\index{Hellingly , Wealden@Hellingly, \emph{Wealden}}

Resignation of Paul Soane (C).

\noindent
\begin{tabular*}{\columnwidth}{@{\extracolsep{\fill}} p{0.545\columnwidth} >{\itshape}l r @{\extracolsep{\fill}}}
David White & LD & 875\\
Alex Willis & C & 222\\
Paul Soane & Ind & 154\\
\end{tabular*}

\section{East Yorkshire}

\council{East Riding}

At the May 2015 ordinary election there were unfilled vacancies in Dale, and Willerby and Kirk Ella wards due to the resignations of Rita Hudson (C) and Mike Whitehead (UKIP elected as C) respectively.
\index{Dale , East Riding@Dale, \emph{E. Riding}}
\index{Willerby and Kirk Ella , East Riding@Willerby \& Kirk Ella, \emph{E. Riding}}

\council{Kingston upon Hull}

At the May 2015 ordinary election there was an unfilled vacancy in Drypool ward due to the resignation of Gary Wareing (Lab).
\index{Drypool , Kingston upon Hull@Drypool, \emph{Kingston upon Hull}}

\section{Essex}

\subsection*{County Council}\index{Essex}

\subsubsection*{Bocking \hspace*{\fill}\nolinebreak[1]%
\enspace\hspace*{\fill}
\finalhyphendemerits=0
[5th March; C gain from UKIP]}

\index{Bocking , Essex@Bocking, \emph{Essex}}

Death of Gordon Helm (UKIP).

\noindent
\begin{tabular*}{\columnwidth}{@{\extracolsep{\fill}} p{0.545\columnwidth} >{\itshape}l r @{\extracolsep{\fill}}}
Stephen Canning & C & 1071\\
Lynn Watson & Lab & 974\\
Michael Ford & UKIP & 855\\
John Malam & Grn & 165\\
Peter Sale & Ind & 58\\
\end{tabular*}

\council{Brentwood}

\subsubsection*{Shenfield \hspace*{\fill}\nolinebreak[1]%
\enspace\hspace*{\fill}
\finalhyphendemerits=0
[29th October; C gain from LD]}

\index{Shenfield , Brentwood@Shenfield, \emph{Brentwood}}

Resignation of Liz Cohen (LD).

\noindent
\begin{tabular*}{\columnwidth}{@{\extracolsep{\fill}} p{0.545\columnwidth} >{\itshape}l r @{\extracolsep{\fill}}}
Jan Pound & C & 852\\
Alison Fulcher & LD & 483\\
Peter Sceats & UKIP & 85\\
Cameron Ball & Lab & 49\\
John Hamilton & Grn & 16\\
\end{tabular*}

\council{Chelmsford}

At the May 2015 ordinary election there was an unfilled vacancy in The Lawns ward due to the death of Robert Burgoyne (C).
\index{Lawns , Chelmsford@The Lawns, \emph{Chelmsford}}

\council{Colchester}

\subsubsection*{Dedham and Langham \hspace*{\fill}\nolinebreak[1]%
\enspace\hspace*{\fill}
\finalhyphendemerits=0
[22nd October]}

\index{Dedham and Langham , Colchester@Dedham \& Langham, \emph{Colchester}}

Disqualification of Mark Cable (C) for non-attendance.

\noindent
\begin{tabular*}{\columnwidth}{@{\extracolsep{\fill}} p{0.545\columnwidth} >{\itshape}l r @{\extracolsep{\fill}}}
Anne Brown & C & 545\\
Bill Faram & UKIP & 60\\
George Penny & LD & 57\\
John Spademan & Lab & 38\\
\end{tabular*}

\council{Harlow}

\subsubsection*{Mark Hall \hspace*{\fill}\nolinebreak[1]%
\enspace\hspace*{\fill}
\finalhyphendemerits=0
[12th February; Lab gain from UKIP]}

\index{Mark Hall , Harlow@Mark Hall, \emph{Harlow}}

Resignation of Jerry Crawford (UKIP).

\noindent
\begin{tabular*}{\columnwidth}{@{\extracolsep{\fill}} p{0.545\columnwidth} >{\itshape}l r @{\extracolsep{\fill}}}
Danny Purton & Lab & 586\\
Mark Gough & UKIP & 353\\
Jane Steer & C & 334\\
Murray Sackwild & Grn & 55\\
Lesley Rideout & LD & 47\\
\end{tabular*}

\subsubsection*{Great Parndon \hspace*{\fill}\nolinebreak[1]%
\enspace\hspace*{\fill}
\finalhyphendemerits=0
[7th May]}

\index{Great Parndon , Harlow@Great Parndon, \emph{Harlow}}

Resignation of Terry Spooner (UKIP).

Combined with the 2015 ordinary election.
%; see page \pageref{GreatParndonHarlow} for the result.

\council{Rochford}

\subsubsection*{Rochford \hspace*{\fill}\nolinebreak[1]%
\enspace\hspace*{\fill}
\finalhyphendemerits=0
[26th November; Lab gain from C]}

\index{Rochford , Rochford@Rochford, \emph{Rochford}}

Death of Gillian Lucas-Gill (C).

\noindent
\begin{tabular*}{\columnwidth}{@{\extracolsep{\fill}} p{0.545\columnwidth} >{\itshape}l r @{\extracolsep{\fill}}}
Matthew Softly & Lab & 332\\
Michael Lucas-Gill & C & 328\\
Nicholas Cooper & UKIP & 250\\
Daniel Irlam & LD & 114\\
\end{tabular*}

\council{Southend-on-Sea}

\subsubsection*{St Laurence \hspace*{\fill}\nolinebreak[1]%
\enspace\hspace*{\fill}
\finalhyphendemerits=0
[7th May]}

\index{Saint Laurence , Southend-on-Sea@St Laurence, \emph{Southend-on-Sea}}

Resignation of Lee Burling (UKIP).

Combined with the 2015 ordinary election.
%; see page \pageref{StLaurenceSouthendonSea} for the result.

\subsubsection*{West Shoebury \hspace*{\fill}\nolinebreak[1]%
\enspace\hspace*{\fill}
\finalhyphendemerits=0
[7th May]}

\index{West Shoebury , Southend-on-Sea@West Shoebury, \emph{Southend-on-Sea}}

Resignation of Liz Day (C).

Combined with the 2015 ordinary election.
%; see page \pageref{WestShoeburySouthendonSea} for the result.

\council{Tendring}

\subsubsection*{Rush Green \hspace*{\fill}\nolinebreak[1]%
\enspace\hspace*{\fill}
\finalhyphendemerits=0
[16th July]}

\index{Rush Green , Tendring@Rush Green, \emph{Tendring}}

Resignation of Len Sibbald (UKIP).

\noindent
\begin{tabular*}{\columnwidth}{@{\extracolsep{\fill}} p{0.545\columnwidth} >{\itshape}l r @{\extracolsep{\fill}}}
Richard Everett & UKIP & 338\\
Danny Mayzes & C & 290\\
Samantha Atkinson & Lab & 213\\
William Hones & Ind & 36\\
\end{tabular*}

\council{Thurrock}

\subsubsection*{West Thurrock and South Stifford \hspace*{\fill}\nolinebreak[1]%
\enspace\hspace*{\fill}
\finalhyphendemerits=0
[10th September]}

\index{West Thurrock and South Stifford , Thurrock@West Thurrock \& South Stifford, \emph{Thurrock}}

Resignation of Terry Brookes (Lab).

\noindent
\begin{tabular*}{\columnwidth}{@{\extracolsep{\fill}} p{0.545\columnwidth} >{\itshape}l r @{\extracolsep{\fill}}}
Cliff Holloway & Lab & 629\\
Helen Adams & UKIP & 450\\
Tony Coughlin & C & 384\\
\end{tabular*}

\section{Gloucestershire}

\council{Cheltenham}

\subsubsection*{Battledown \hspace*{\fill}\nolinebreak[1]%
\enspace\hspace*{\fill}
\finalhyphendemerits=0
[7th May]}

\index{Battledown , Cheltenham@Battledown, \emph{Cheltenham}}

Resignation of Andrew Wall (C).

\noindent
\begin{tabular*}{\columnwidth}{@{\extracolsep{\fill}} p{0.545\columnwidth} >{\itshape}l r @{\extracolsep{\fill}}}
Louis Savage & C & 1477\\
Paul McCloskey & LD & 1037\\
Roberta Smart & Grn & 243\\
Helen Pemberton & Lab & 200\\
Elizabeth Roberts & UKIP & 181\\
\end{tabular*}

\council{Forest of Dean}

At the May 2015 ordinary election there was an unfilled vacancy in Coleford East ward due to the disqualification of Tanya Palmer (Lab) for non-attendance.
\index{Coleford East , Forest of Dean@Coleford E., \emph{Forest of Dean}}

\council{Stroud}

At the May 2015 ordinary election there was an unfilled vacancy in Nailsworth ward due to the death of Paul Carter (C).
\index{Nailsworth , Stroud@Nailsworth, \emph{Stroud}}

\subsubsection*{Eastington and Standish \hspace*{\fill}\nolinebreak[1]%
\enspace\hspace*{\fill}
\finalhyphendemerits=0
[7th May]}

\index{Eastington and Standish , Stroud@Eastington \& Standish, \emph{Stroud}}

Resignation of Ken Stephens (Lab).

Combined with the 2015 ordinary election.
%; see page \pageref{EastingtonStandishStroud} for the result.

\columnbreak

\section{Hampshire}

\subsection*{County Council}\index{Hampshire}

\subsubsection*{Andover West \hspace*{\fill}\nolinebreak[1]%
\enspace\hspace*{\fill}
\finalhyphendemerits=0
[7th May]}

\index{Andover West , Hampshire@Andover W., \emph{Hants.}}

Resignation of Pat West (C).

\noindent
\begin{tabular*}{\columnwidth}{@{\extracolsep{\fill}} p{0.545\columnwidth} >{\itshape}l r @{\extracolsep{\fill}}}
Zilliah Brooks & C & 5208\\
Christine Forrester & UKIP & 1846\\
Michael Mumford & Lab & 1304\\
Dean Marriner & Grn & 698\\
\end{tabular*}

\subsubsection*{Chandler's Ford \hspace*{\fill}\nolinebreak[1]%
\enspace\hspace*{\fill}
\finalhyphendemerits=0
[22nd October]}

\index{Chandler's Ford , Hampshire@Chandler's Ford, \emph{Hants.}}

Resignation of Colin Davidovitz (C).

\noindent
\begin{tabular*}{\columnwidth}{@{\extracolsep{\fill}} p{0.545\columnwidth} >{\itshape}l r @{\extracolsep{\fill}}}
Judith Grajewski & C & 2074\\
James Duguid & LD & 1493\\
John Edwards & UKIP & 358\\
Sarah Smith & Lab & 285\\
\end{tabular*}

\council{Basingstoke and Deane}

\subsubsection*{Rooksdown \hspace*{\fill}\nolinebreak[1]%
\enspace\hspace*{\fill}
\finalhyphendemerits=0
[7th May]}

\index{Rooksdown , Basingstoke and Deane@Rooksdown, \emph{Basingstoke \& Deane}}

Resignation of Karen Cherrett (C).

Combined with the 2015 ordinary election.
%; see page \pageref{RooksdownBasingstokeDeane} for the result.

\council{Eastleigh}

At the May 2015 ordinary election there was an unfilled vacancy in Hedge End Wildern ward due to the resignation of Jenny Hughes (LD).
\index{Hedge End Wildern , Eastleigh@Hedge End Wildern, \emph{Eastleigh}}

\council{Hart}

\subsubsection*{Hartley Wintney \hspace*{\fill}\nolinebreak[1]%
\enspace\hspace*{\fill}
\finalhyphendemerits=0
[7th May]}

\index{Hartley Wintney , Hart@Hartley Wintney, \emph{Hart}}

Resignation of Sara Kinnell (C).

Combined with the 2015 ordinary election.
%; see page \pageref{HartleyWintneyHart} for the result.

\council{Rushmoor}

At the May 2015 ordinary election there was an unfilled vacancy in Aldershot Park ward due to the resignation of Don Cappleman (Lab).
\index{Aldershot Park , Rushmoor@Aldershot Park, \emph{Rushmoor}}

\subsubsection*{West Heath \hspace*{\fill}\nolinebreak[1]%
\enspace\hspace*{\fill}
\finalhyphendemerits=0
[7th May]}

\index{West Heath , Rushmoor@West Heath, \emph{Rushmoor}}

Resignation of Barbara Donaghue (UKIP).

Combined with the 2015 ordinary election.
%; see page \pageref{WestHeathRushmoor} for the result.

\council{Winchester}

\subsubsection*{Cheriton and Bishops Sutton \hspace*{\fill}\nolinebreak[1]%
\enspace\hspace*{\fill}
\finalhyphendemerits=0
[7th May]}

\index{Cheriton and Bishops Sutton , Winchester@Cheriton \& Bishops Sutton, \emph{Winchester}}

Resignation of Harry Verney (C).

Combined with the 2015 ordinary election.
%; see page \pageref{CheritonBishopsSuttonWinchester} for the result.

\section{Herefordshire}
\index{Herefordshire}

At the May 2015 ordinary election there was an unfilled vacancy in Mortimer ward due to the death of Olwyn Barnett (C).
\index{Mortimer , Herefordshire@Mortimer, \emph{Herefs.}}

\section{Hertfordshire}

\council{East Hertfordshire}

\subsubsection*{Hertford Heath \hspace*{\fill}\nolinebreak[1]%
\enspace\hspace*{\fill}
\finalhyphendemerits=0
[17th December]}

\index{Hertford Heath , East Hertfordshire@Hertford Heath, \emph{E. Herts.}}

Resignation of Adrian McNeece (C).

Note:--- Snowdon appeared as an independent candidate on the ballot paper due to errors on her nomination papers.  She was in fact the official Conservative candidate.

\noindent
\begin{tabular*}{\columnwidth}{@{\extracolsep{\fill}} p{0.545\columnwidth} >{\itshape}l r @{\extracolsep{\fill}}}
Charlotte Snowdon & C & 269\\
Rob Lambie & LD & 101\\
Sheila Pettman & UKIP & 70\\
Graham Nickson & Lab & 56\\
Hilary Cullen & Grn & 21\\
\end{tabular*}

\council{Hertsmere}

At the May 2015 ordinary election there was an unfilled vacancy in Aldenham West ward due to the resignation of Dan Griffin (C).
\index{Aldenham West , Hertsmere@Aldenham W., \emph{Hertsmere}}

\subsection*{St Albans}\index{Saint Albans@St Albans}

\subsubsection*{Marshalswick South (2) \hspace*{\fill}\nolinebreak[1]%
\enspace\hspace*{\fill}
\finalhyphendemerits=0
[29th January]}

\index{Marshalswick South , Saint Albans@Marshalswick S., \emph{St Albans}}

Resignations of Heidi Allen and Seema Kennedy (both C).

\noindent
\begin{tabular*}{\columnwidth}{@{\extracolsep{\fill}} p{0.545\columnwidth} >{\itshape}l r @{\extracolsep{\fill}}}
Steve McKeown & C & 667\\
Richard Curthoys & C & 647\\
Mark Pedroz & LD & 495\\
Elizabeth Needham & LD & 488\\
Jill Mills & Grn & 450\\
Richard Harris & Lab & 406\\
Vivienne Windle & Lab & 312\\
Tim Robinson & Grn & 166\\
David Dickson & UKIP & 148\\
Michael Hollins & UKIP & 147\\
\end{tabular*}

\council{Watford}

\subsubsection*{Callowland \hspace*{\fill}\nolinebreak[1]%
\enspace\hspace*{\fill}
\finalhyphendemerits=0
[7th May]}

\index{Callowland , Watford@Callowland, \emph{Watford}}

Resignation of Ian Brandon (Grn).

Combined with the 2015 ordinary election.
%; see page \pageref{CallowlandWatford} for the result.

\section{Kent}

\subsection*{County Council}\index{Kent}

\subsubsection*{Romney Marsh \hspace*{\fill}\nolinebreak[1]%
\enspace\hspace*{\fill}
\finalhyphendemerits=0
[7th May; C gain from UKIP]}

\index{Romney Marsh , Kent@Romney Marsh, \emph{Kent}}

Resignation of David Baker (UKIP).

\noindent
\begin{tabular*}{\columnwidth}{@{\extracolsep{\fill}} p{0.545\columnwidth} >{\itshape}l r @{\extracolsep{\fill}}}
Carole Waters & C & 4913\\
Susanna Govett & UKIP & 3903\\
Arran Harvey & Lab & 1342\\
Valerie Loseby & LD & 626\\
Andrew South & Grn & 435\\
\end{tabular*}

\columnbreak

\council{Ashford}

Ashford = Ashford Independent

\subsubsection*{Aylesford Green \hspace*{\fill}\nolinebreak[1]%
\enspace\hspace*{\fill}
\finalhyphendemerits=0
[19th November; C gain from Lab]}

\index{Aylesford Green , Ashford@Aylesford Green, \emph{Ashford}}

Resignation of Kate Hooker (Lab).

\noindent
\begin{tabular*}{\columnwidth}{@{\extracolsep{\fill}} p{0.545\columnwidth} >{\itshape}l r @{\extracolsep{\fill}}}
Alex Howard & C & 110\\
Harriet Yeo & UKIP & 109\\
Gordon Miller & Lab & 106\\
\sloppyword{Christine Kathawick-Smith} & Ashford & 92\\
Adrian Gee-Turner & LD & 42\\
Thom Pizzey & Grn & 10\\
\end{tabular*}

\council{Canterbury}

At the May 2015 ordinary election there was an unfilled vacancy in Heron ward due to the death of Ron Flaherty (LD).
\index{Heron , Canterbury@Heron, \emph{Canterbury}}

\council{Dartford}

At the May 2015 ordinary election there was an unfilled vacancy in Brent ward due to the resignation of Peter Cannon (C).
\index{Brent , Dartford@Brent, \emph{Dartford}}

\council{Maidstone}

At the May 2015 ordinary election there was an unfilled vacancy in Leeds ward due to the death of Peter Parvin (C).
\index{Leeds , Maidstone@Leeds, \emph{Maidstone}}

\subsubsection*{Park Wood \hspace*{\fill}\nolinebreak[1]%
\enspace\hspace*{\fill}
\finalhyphendemerits=0
[7th May]}

\index{Park Wood , Maidstone@Park Wood, \emph{Maidstone}}

\sloppyword{Resignation of Christine Edwards-Daem (UKIP).}

Combined with the 2015 ordinary election.
%; see page \pageref{ParkWoodMaidstone} for the result.

\subsubsection*{Fant \hspace*{\fill}\nolinebreak[1]%
\enspace\hspace*{\fill}
\finalhyphendemerits=0
[10th September]}

\index{Fant , Maidstone@Fant, \emph{Maidstone}}

Death of Alistair Black (C).

\noindent
\begin{tabular*}{\columnwidth}{@{\extracolsep{\fill}} p{0.545\columnwidth} >{\itshape}l r @{\extracolsep{\fill}}}
Matt Boughton & C & 477\\
Rosaline Janko & LD & 424\\
Keith Adkinson & Lab & 352\\
Stuart Jeffery & Grn & 249\\
Colin Taylor & UKIP & 180\\
Mike Hogg & Ind & 75\\
\end{tabular*}

\council{Sevenoaks}

At the May 2015 ordinary election there was an unfilled vacancy in Cowden and Hever ward due to the resignation of Christopher Neal (UKIP elected as C).
\index{Cowden and Hever , Sevenoaks@Cowden \& Hever, \emph{Sevenoaks}}

\council{Shepway}

At the May 2015 ordinary election there was an unfilled vacancy in Hythe East ward due to the death of Keren Belcourt (C).
\index{Hythe East, Shepway@Hythe E, \emph{Shepway}}

\council{Tunbridge Wells}

\subsubsection*{Rusthall \hspace*{\fill}\nolinebreak[1]%
\enspace\hspace*{\fill}
\finalhyphendemerits=0
[7th May]}

\index{Rusthall , Tunbridge Wells@Rusthall, \emph{Tunbridge Wells}}

Resignation of Piers Wauchope (UKIP).

Combined with the 2015 ordinary election.
%; see page \pageref{RusthallTunbridgeWells} for the result.

\subsubsection*{Southborough North \hspace*{\fill}\nolinebreak[1]%
\enspace\hspace*{\fill}
\finalhyphendemerits=0
[10th September]}

\index{Southborough North , Tunbridge Wells@Southborough N., \emph{Tunbridge Wells}}

Death of Mike Rusbridge (C).

\noindent
\begin{tabular*}{\columnwidth}{@{\extracolsep{\fill}} p{0.545\columnwidth} >{\itshape}l r @{\extracolsep{\fill}}}
Joe Simmons & C & 483\\
Trevor Poile & LD & 434\\
William O'Shea & UKIP & 188\\
\end{tabular*}

\section{Lancashire}

\council{Blackburn with Darwen}

\subsubsection*{Mill Hill \hspace*{\fill}\nolinebreak[1]%
\enspace\hspace*{\fill}
\finalhyphendemerits=0
[23rd July]}

\index{Mill Hill , Blackburn with Darwen@Mill Hill, \emph{Blackburn with Darwen}}

Resignation of Carol Walsh (Lab).

\noindent
\begin{tabular*}{\columnwidth}{@{\extracolsep{\fill}} p{0.545\columnwidth} >{\itshape}l r @{\extracolsep{\fill}}}
Carl Nuttall & Lab & 505\\
Michael Longbottom & UKIP & 179\\
Helen Tolley & C & 106\\
Alan Dean & LD & 69\\
\end{tabular*}

\council{Burnley}

\subsubsection*{Hapton with Park \hspace*{\fill}\nolinebreak[1]%
\enspace\hspace*{\fill}
\finalhyphendemerits=0
[7th May]}

\index{Hapton with Park , Burnley@Hapton with Park, \emph{Burnley}}

Resignation of Jonathan Barker (Lab).

Combined with the 2015 ordinary election.
%; see page \pageref{HaptonParkBurnley} for the result.

\council{Chorley}

\subsubsection*{Euxton North \hspace*{\fill}\nolinebreak[1]%
\enspace\hspace*{\fill}
\finalhyphendemerits=0
[29th October]}

\index{Euxton North , Chorley@Euxton N., \emph{Chorley}}

Disqualification (non-attendance) of Mike Handley (Lab).

\noindent
\begin{tabular*}{\columnwidth}{@{\extracolsep{\fill}} p{0.545\columnwidth} >{\itshape}l r @{\extracolsep{\fill}}}
Tommy Gray & Lab & 697\\
Alan Platt & C & 443\\
Christopher Suart & UKIP & 76\\
\end{tabular*}

\council{Fylde}

\subsubsection*{Clifton \hspace*{\fill}\nolinebreak[1]%
\enspace\hspace*{\fill}
\finalhyphendemerits=0
[10th December]}

\index{Clifton , Fylde@Clifton, \emph{Fylde}}

Resignation of Len Davies (C).

\noindent
\begin{tabular*}{\columnwidth}{@{\extracolsep{\fill}} p{0.545\columnwidth} >{\itshape}l r @{\extracolsep{\fill}}}
Peter Anthony & C & 576\\
Tim Wood & UKIP & 128\\
Noreen Griffiths & Ind & 122\\
Jed Sullivan & Lab & 84\\
Luke Gibbon & LD & 53\\
\end{tabular*}

\council{Hyndburn}

\subsubsection*{Spring Hill \hspace*{\fill}\nolinebreak[1]%
\enspace\hspace*{\fill}
\finalhyphendemerits=0
[9th July]}

\index{Spring Hill , Hyndburn@Spring Hill, \emph{Hyndburn}}

Death of Pam Barton (Lab).

\noindent
\begin{tabular*}{\columnwidth}{@{\extracolsep{\fill}} p{0.545\columnwidth} >{\itshape}l r @{\extracolsep{\fill}}}
Diane Fielding & Lab & 778\\
Mohammad Safdar & C & 475\\
Ken Smith & UKIP & 137\\
Kerry Gormley & Grn & 17\\
\end{tabular*}

\council{Lancaster}

\subsubsection*{Carnforth and Millhead \hspace*{\fill}\nolinebreak[1]%
\enspace\hspace*{\fill}
\finalhyphendemerits=0
[26th November]}

\index{Carnforth and Millhead , Lancaster@Carnforth \& Millhead, \emph{Lancaster}}

Death of Christopher Leadbetter (C).

\noindent
\begin{tabular*}{\columnwidth}{@{\extracolsep{\fill}} p{0.545\columnwidth} >{\itshape}l r @{\extracolsep{\fill}}}
George Askew & C & 545\\
Paul Gardner & Lab & 320\\
Christopher Coates & Grn & 52\\
Philip Dunster & LD & 38\\
Michelle Ogden & UKIP & 37\\
\end{tabular*}

\council{Pendle}

At the May 2015 ordinary election there was an unfilled vacancy in Clover Hill ward due to the resignation of Richard Smith (Lab).
\index{Clover Hill , Pendle@Clover Hill, \emph{Pendle}}

\subsubsection*{Barrowford \hspace*{\fill}\nolinebreak[1]%
\enspace\hspace*{\fill}
\finalhyphendemerits=0
[7th May]}

\index{Barrowford , Pendle@Barrowford, \emph{Pendle}}

Resignation of Anthony Beckett (C).

Combined with the 2015 ordinary election.
%; see page \pageref{BarrowfordPendle} for the result.

\council{Rossendale}

At the May 2015 ordinary election there was an unfilled vacancy in Longholme ward due to the resignation of Peter Roberts (Lab).
\index{Longholme , Rossendale@Longholme, \emph{Rossendale}}

\subsubsection*{Irwell \hspace*{\fill}\nolinebreak[1]%
\enspace\hspace*{\fill}
\finalhyphendemerits=0
[7th May]}

\index{Irwell , Rossendale@Irwell, \emph{Rossendale}}

Resignation of Helen Jackson (Lab).

Combined with the 2015 ordinary election.
%; see page \pageref{IrwellRossendale} for the result.

\subsubsection*{Whitewell \hspace*{\fill}\nolinebreak[1]%
\enspace\hspace*{\fill}
\finalhyphendemerits=0
[7th May]}

\index{Whitewell , Rossendale@Whitewell, \emph{Rossendale}}

Resignation of Karen Creaser (Lab).

Combined with the 2015 ordinary election.
%; see page \pageref{WhitewellRossendale} for the result.

\council{South Ribble}

At the May 2015 ordinary election there was an unfilled vacancy in Bamber Bridge West ward due to the death of Tom Hanson (Lab).
\index{Bamber Bridge West , South Ribble@Bamber Bridge W., \emph{S. Ribble}}

\section{Leicestershire}

\subsection*{County Council}\index{Leicestershire}

\subsubsection*{Narborough and Whetstone \hspace*{\fill}\nolinebreak[1]%
\enspace\hspace*{\fill}
\finalhyphendemerits=0
[7th May]}

\index{Narborough and Whetstone , Leicestershire@Narborough \& Whetstone, \emph{Leics.}}

Resignation of Karl Coles (C).

\noindent
\begin{tabular*}{\columnwidth}{@{\extracolsep{\fill}} p{0.545\columnwidth} >{\itshape}l r @{\extracolsep{\fill}}}
Terence Richardson & C & 4011\\
Michael Bounds & Lab & 1658\\
Carolyn Brennan & UKIP & 1491\\
\end{tabular*}

\council{Charnwood}

At the May 2015 ordinary election there was an unfilled vacancy in Thurmaston ward due to the death of Ralph Raven (Lab).
\index{Thurmaston , Charnwood@Thurmaston, \emph{Charnwood}}

\council{Harborough}

\subsubsection*{Market Harborough---Logan \hspace*{\fill}\nolinebreak[1]%
\enspace\hspace*{\fill}
\finalhyphendemerits=0
[10th December]}

\index{Market Harborough Logan , Harborough@Market Harborough---Logan, \emph{Harborough}}

Death of Peter Callis (LD).

\noindent
\begin{tabular*}{\columnwidth}{@{\extracolsep{\fill}} p{0.545\columnwidth} >{\itshape}l r @{\extracolsep{\fill}}}
Barbara Johnson & LD & 402\\
Paul Bremner & C & 303\\
Anne Pridmore & Lab & 82\\
Darren Woodiwiss & Grn & 56\\
Robert Davison & UKIP & 47\\
\end{tabular*}

\council{Melton}

At the May 2015 ordinary election there was an unfilled vacancy in Melton Warwick ward due to the resignation of Norman Slater (Ind elected as C).
\index{Melton Warwick , Melton@Melton Warwick, \emph{Melton}}

\council{North West Leicestershire}

At the May 2015 ordinary election there was an unfilled vacancy in Ibstock and Heather ward due to the death of Dave de Lacy (Lab).
\index{Ibstock and Heather , North West Leicestershire@Ibstock \& Heather, \emph{N.W. Leics.}}

\section{Lincolnshire}

LincsInd = Lincolnshire Independent

\subsection*{County Council}\index{Lincolnshire}

\subsubsection*{Grantham Barrowby \hspace*{\fill}\nolinebreak[1]%
\enspace\hspace*{\fill}
\finalhyphendemerits=0
[2nd July]}

\index{Grantham Barrowby , Lincolnshire@Grantham Barrowby, \emph{Lincs.}}

Resignation of Jo Churchill (C).

\noindent
\begin{tabular*}{\columnwidth}{@{\extracolsep{\fill}} p{0.52\columnwidth} >{\itshape}l r @{\extracolsep{\fill}}}
Mark Whittington & C & 579\\
Rob Shorrock & Lab & 257\\
Maureen Simon & UKIP & 179\\
Mike Williams & LincsInd & 155\\
\end{tabular*}

\council{Boston}

At the May 2015 ordinary election there was an unfilled vacancy in Coastal ward due to the resignation of Raymond Singleton-McGuire (Ind elected as C).
\index{Coastal , Boston@Coastal, \emph{Boston}}

\council{East Lindsey}

At the May 2015 ordinary election there was an unfilled vacancy in Sibsey ward due to the death of Rick Harvey (East Lindsey Ind Group).
\index{Sibsey, East Lindsey@Sibsey, \emph{E. Lindsey}}

\council{North East Lincolnshire}

\subsubsection*{East Marsh \hspace*{\fill}\nolinebreak[1]%
\enspace\hspace*{\fill}
\finalhyphendemerits=0
[7th May]}

\index{East Marsh , North East Lincolnshire@East Marsh, \emph{N.E. Lincs.}}

Resignation of Jon-Paul Howarth (Ind elected as Lab).

Combined with the 2015 ordinary election.
%; see page \pageref{EastMarshNELincs} for the result.

\columnbreak

\subsubsection*{Croft Baker \hspace*{\fill}\nolinebreak[1]%
\enspace\hspace*{\fill}
\finalhyphendemerits=0
[23rd July]}

\index{Croft Baker , North East Lincolnshire@Croft Baker, \emph{N.E. Lincs.}}

Death of Mick Burnett (Lab).

\noindent
\begin{tabular*}{\columnwidth}{@{\extracolsep{\fill}} p{0.545\columnwidth} >{\itshape}l r @{\extracolsep{\fill}}}
Annie Darby & Lab & 768\\
Hayden Dawkins & C & 513\\
Roy Horrobin & LD & 323\\
Graham Critchley & UKIP & 318\\
Dave Mitchell & TUSC & 85\\
James Barker & Grn & 66\\
\end{tabular*}

\council{North Kesteven}

HykhmInd = Hykeham Independents

\subsubsection*{North Hykeham Mill \hspace*{\fill}\nolinebreak[1]%
\enspace\hspace*{\fill}
\finalhyphendemerits=0
[30th July; C gain from LincsInd]}

\index{North Hykeham Mill , North Kesteven@North Hykeham Mill, \emph{N. Kesteven}}

Resignation of Jill Wilson (LincsInd).

\noindent
\begin{tabular*}{\columnwidth}{@{\extracolsep{\fill}} p{0.47\columnwidth} >{\itshape}l r @{\extracolsep{\fill}}}
Mike Clarke & C & 286\\
John Bishop & HykhmInd & 180\\
Terence Dooley & Lab & 161\\
\sloppyword{Elizabeth Bathory-Porter} & Grn & 64\\
Diana Catton & LD & 22\\
\end{tabular*}

\council{North Lincolnshire}

At the May 2015 ordinary election there was an unfilled vacancy in Crosby and Park ward due to the disqualification (sentenced to 3\textonehalf{} years' imprisonment, forgery) of Jawaid Ishaq (Lab).
\index{Crosby and Park , North Lincolnshire@Crosby \& Park, \emph{N. Lincs.}}

\council{South Holland}

At the May 2015 ordinary election there was an unfilled vacancy in Donington, Quadring and Gosberton ward due to the death of Amanda Puttick (C).
\index{Donington, Quadring and Gosberton , South Holland@Donington, Quadring \& Gosberton, \emph{S. Holland}}

\council{South Kesteven}

At the May 2015 ordinary election there was an unfilled vacancy in Bourne West ward due to the death of John Smith (C).
\index{Bourne West , South Kesteven@Bourne W., \emph{S. Kesteven}}

\subsubsection*{Market and West Deeping (3) \hspace*{\fill}\nolinebreak[1]%
\enspace\hspace*{\fill}
\finalhyphendemerits=0
[25th June]}

\index{Market and West Deeping , South Kesteven@Market \& W. Deeping, \emph{S. Kesteven}}

Ordinary election postponed from 7th May: death of outgoing councillor Reg Howard (Ind) who had been nominated for re-election.

\noindent
\begin{tabular*}{\columnwidth}{@{\extracolsep{\fill}} p{0.545\columnwidth} >{\itshape}l r @{\extracolsep{\fill}}}
Bob Broughton & Ind & 612\\
Ashley Baxter & Ind & 609\\
Nick Neilson & C & 605\\
David Shelton & Ind & 426\\
Adam Brookes & LD & 229\\
Robert O'Farrell & UKIP & 224\\
William Learoyd & UKIP & 129\\
Roger Woodbridge & UKIP & 113\\
\end{tabular*}

\subsubsection*{Belvoir \hspace*{\fill}\nolinebreak[1]%
\enspace\hspace*{\fill}
\finalhyphendemerits=0
[3rd December]}

\index{Belvoir , South Kesteven@Belvoir, \emph{S. Kesteven}}

Resignation of Tom Webster (C).

\noindent
\begin{tabular*}{\columnwidth}{@{\extracolsep{\fill}} p{0.545\columnwidth} >{\itshape}l r @{\extracolsep{\fill}}}
Hannah Westropp & C & 603\\
Laura King & Ind & 212\\
Louise Clack & Lab & 175\\
Mike Taylor & UKIP & 159\\
\end{tabular*}

\section{Norfolk}

\subsection*{County Council}\index{Norfolk}

\subsubsection*{Loddon \hspace*{\fill}\nolinebreak[1]%
\enspace\hspace*{\fill}
\finalhyphendemerits=0
[7th May]}

\index{Loddon , Norfolk@Loddon, \emph{Norfolk}}

Resignation of Adrian Gunson (C).

\noindent
\begin{tabular*}{\columnwidth}{@{\extracolsep{\fill}} p{0.545\columnwidth} >{\itshape}l r @{\extracolsep{\fill}}}
Derek Blake & C & 3002\\
David Bissonnet & Lab & 1018\\
Alan Baugh & UKIP & 923\\
\sloppyword{Kieran Campbell-Johnston} & Grn & 555\\
Christopher Brown & LD & 496\\
\end{tabular*}

\subsubsection*{Gorleston St Andrews \hspace*{\fill}\nolinebreak[1]%
\enspace\hspace*{\fill}
\finalhyphendemerits=0
[16th July; C gain from UKIP]}

\index{Gorleston Saint Andrews , Norfolk@Gorleston St Andrews, \emph{Norfolk}}

Disqualification of Matthew Smith (UKIP): found guilty of a corrupt practice (false signatures on nomination papers) during the 2013 Norfolk county council election.

\noindent
\begin{tabular*}{\columnwidth}{@{\extracolsep{\fill}} p{0.545\columnwidth} >{\itshape}l r @{\extracolsep{\fill}}}
Graham Plant & C & 876\\
Tony Wright & Lab & 773\\
Adrian Myers & UKIP & 285\\
Tony Harris & LD & 66\\
Harry Webb & Grn & 51\\
\end{tabular*}

\subsubsection*{Mile Cross \hspace*{\fill}\nolinebreak[1]%
\enspace\hspace*{\fill}
\finalhyphendemerits=0
[16th July]}

\index{Mile Cross , Norfolk@Mile Cross, \emph{Norfolk}}

Resignation of Deborah Gihawi (Lab).

\noindent
\begin{tabular*}{\columnwidth}{@{\extracolsep{\fill}} p{0.545\columnwidth} >{\itshape}l r @{\extracolsep{\fill}}}
Chrissie Rumsby & Lab & 749\\
Chelsea Bales & C & 279\\
Richard Edwards & Grn & 209\\
Michelle Ho & UKIP & 148\\
Tom Dymoke & LD & 62\\
\end{tabular*}

\subsubsection*{Loddon \hspace*{\fill}\nolinebreak[1]%
\enspace\hspace*{\fill}
\finalhyphendemerits=0
[24th September]}

\index{Loddon , Norfolk@Loddon, \emph{Norfolk}}

Death of Derek Blake (C).

\noindent
\begin{tabular*}{\columnwidth}{@{\extracolsep{\fill}} p{0.545\columnwidth} >{\itshape}l r @{\extracolsep{\fill}}}
Barry Stone & C & 1094\\
David Bissonnet & Lab & 357\\
Christopher Brown & LD & 235\\
Alan Baugh & UKIP & 233\\
\end{tabular*}

\subsubsection*{South Smallburgh \hspace*{\fill}\nolinebreak[1]%
\enspace\hspace*{\fill}
\finalhyphendemerits=0
[19th November]}

\index{South Smallburgh , Norfolk@South Smallburgh, \emph{Norfolk}}

Resignation of David Thomas (LD).

\noindent
\begin{tabular*}{\columnwidth}{@{\extracolsep{\fill}} p{0.545\columnwidth} >{\itshape}l r @{\extracolsep{\fill}}}
Allison Bradnock & LD & 1383\\
Paul Rice & C & 697\\
Barry Whitehouse & UKIP & 219\\
David Spencer & Lab & 103\\
Anne Filgate & Grn & 52\\
\end{tabular*}

\subsubsection*{Watton \hspace*{\fill}\nolinebreak[1]%
\enspace\hspace*{\fill}
\finalhyphendemerits=0
[19th November; C gain from UKIP]}

\index{Watton , Norfolk@Watton, \emph{Norfolk}}

Resignation of Stan Hebborn (UKIP).

\noindent
\begin{tabular*}{\columnwidth}{@{\extracolsep{\fill}} p{0.545\columnwidth} >{\itshape}l r @{\extracolsep{\fill}}}
Claire Bowes & C & 822\\
Keith Gilbert & Ind & 793\\
Joe Sisto & Lab & 105\\
Timothy Birt & Grn & 81\\
\end{tabular*}

\council{Norwich}

\subsubsection*{Mile Cross \hspace*{\fill}\nolinebreak[1]%
\enspace\hspace*{\fill}
\finalhyphendemerits=0
[7th May]}

\index{Mile Cross , Norwich@Mile Cross, \emph{Norwich}}

Resignation of Deborah Gihawi (Ind elected as Lab).

Combined with the 2015 ordinary election.
%; see page \pageref{MileCrossNorwich} for the result.

\subsubsection*{Sewell \hspace*{\fill}\nolinebreak[1]%
\enspace\hspace*{\fill}
\finalhyphendemerits=0
[7th May]}

\index{Sewell , Norwich@Sewell, \emph{Norwich}}

Resignation of Kevin Barker (Lab).

Combined with the 2015 ordinary election.
%; see page \pageref{SewellNorwich} for the result.

\council{South Norfolk}

\subsubsection*{Chedgrave and Thurton \hspace*{\fill}\nolinebreak[1]%
\enspace\hspace*{\fill}
\finalhyphendemerits=0
[24th September]}

\index{Chedgrave and Thurton , South Norfolk@Chedgrave \& Thurton, \emph{S. Norfolk}}

Death of Derek Blake (C).

\noindent
\begin{tabular*}{\columnwidth}{@{\extracolsep{\fill}} p{0.545\columnwidth} >{\itshape}l r @{\extracolsep{\fill}}}
Jaan Larner & C & 260\\
Sarah Langton & Lab & 93\\
Ernest Green & LD & 69\\
Ron Murphy & UKIP & 64\\
\end{tabular*}

\section{North Yorkshire}

\council{Craven}

\subsubsection*{Barden Fell \hspace*{\fill}\nolinebreak[1]%
\enspace\hspace*{\fill}
\finalhyphendemerits=0
[7th May]}

\index{Barden Fell , Craven@Barden Fell, \emph{Craven}}

Resignation of Chris Knowles-Fitton (C).

Combined with the 2015 ordinary election.
%; see page \pageref{BardenFellCraven} for the result.

\subsubsection*{Settle and Ribblebanks \hspace*{\fill}\nolinebreak[1]%
\enspace\hspace*{\fill}
\finalhyphendemerits=0
[7th May]}

\index{Settle and Ribblebanks , Craven@Settle \& Ribblebanks, \emph{Craven}}

Death of Donny Whaites (C).

Combined with the 2015 ordinary election.
%; see page \pageref{SettleRibblebanksCraven} for the result.

\subsubsection*{Upper Wharfedale \hspace*{\fill}\nolinebreak[1]%
\enspace\hspace*{\fill}
\finalhyphendemerits=0
[7th May]}

\index{Upper Wharfedale , Craven@Upper Wharfedale, \emph{Craven}}

Death of John Roberts (C).

Combined with the 2015 ordinary election.
%; see page \pageref{UpperWharfedaleCraven} for the result.

\council{Middlesbrough}

At the May 2015 ordinary election there was an unfilled vacancy in Stainton and Thornton ward due to the death of Maelor Williams (LD).
\index{Stainton and Thornton , Middlesbrough@Stainton \& Thornton, \emph{Middlesbrough}}

\council{Richmondshire}

RIG = Richmondshire Independent Group

At the May 2015 ordinary election there was an unfilled vacancy in Catterick ward due to the death of Rob Johnson (C).
\index{Catterick , Richmondshire@Catterick, \emph{Richmondshire}}

\subsubsection*{Richmond East \hspace*{\fill}\nolinebreak[1]%
\enspace\hspace*{\fill}
\finalhyphendemerits=0
[17th September]}

\index{Richmond East , Richmondshire@Richmond E., \emph{Richmondshire}}

Resignation of Karen Kirby (C).

\noindent
\begin{tabular*}{\columnwidth}{@{\extracolsep{\fill}} p{0.545\columnwidth} >{\itshape}l r @{\extracolsep{\fill}}}
Louise Dickens & C & 307\\
Lorraine Hodgson & RIG & 289\\
Philip Knowles & LD & 136\\
Tina McKay & Lab & 41\\
\end{tabular*}

\columnbreak

\council{Ryedale}

Yorks1st = Yorkshire First

\subsubsection*{Derwent \hspace*{\fill}\nolinebreak[1]%
\enspace\hspace*{\fill}
\finalhyphendemerits=0
[17th December; Lib gain from C]}

\index{Derwent , Ryedale@Derwent, \emph{Ryedale}}

Resignation of Phil Evans (C).

Note:--- Allanson was the official Liberal Democrat candidate but appeared on the ballot paper as an independent due to issues with his nomination papers.

\noindent
\begin{tabular*}{\columnwidth}{@{\extracolsep{\fill}} p{0.53\columnwidth} >{\itshape}l r @{\extracolsep{\fill}}}
Mike Potter & Lib & 283\\
Kerry Ennis & C & 278\\
Stephen Shaw & Ind & 124\\
Darren Allanson & LD & 81\\
Tony Barran & Yorks1st & 32\\
\end{tabular*}

\section{Northamptonshire}

\council{Kettering}

\subsubsection*{Rothwell (3) \hspace*{\fill}\nolinebreak[1]%
\enspace\hspace*{\fill}
\finalhyphendemerits=0
[4th June]}

\index{Rothwell , Kettering@Rothwell, \emph{Kettering}}

Ordinary election postponed from 7th May: death of candidate Alan Pote (UKIP).

\noindent
\begin{tabular*}{\columnwidth}{@{\extracolsep{\fill}} p{0.545\columnwidth} >{\itshape}l r @{\extracolsep{\fill}}}
Alan Mills & Lab & 951\\
Karl Sumpter & C & 853\\
Margaret Talbot & C & 777\\
Ian Jelley & C & 771\\
Malcolm Jones & Lab & 623\\
Kathleen Harris & Lab & 614\\
Sally Hogston & UKIP & 370\\
Stephen Jones & Grn & 119\\
Robert Reeves & Grn & 89\\
Alan Heath & Grn & 82\\
\end{tabular*}

\section{Northumberland}

\subsubsection*{College \hspace*{\fill}\nolinebreak[1]%
\enspace\hspace*{\fill}
\finalhyphendemerits=0
[30th July]}

\index{College , Northumberland@College, \emph{Northd.}}

Death of Jimmy Sawyer (Lab).

\noindent
\begin{tabular*}{\columnwidth}{@{\extracolsep{\fill}} p{0.545\columnwidth} >{\itshape}l r @{\extracolsep{\fill}}}
Mark Purvis & Lab & 508\\
Peter Curtis & UKIP & 102\\
Andy McGregor & LD & 82\\
Chris Galley & C & 39\\
\end{tabular*}

\section{Nottinghamshire}

Selston = Selston Parish Independent

\subsection*{County Council}\index{Nottinghamshire}

\subsubsection*{Selston \hspace*{\fill}\nolinebreak[1]%
\enspace\hspace*{\fill}
\finalhyphendemerits=0
[26th November]}

\index{Selston , Nottinghamshire@Selston, \emph{Notts.}}

Resignation of Gail Turner (Selston).

\noindent
\begin{tabular*}{\columnwidth}{@{\extracolsep{\fill}} p{0.53\columnwidth} >{\itshape}l r @{\extracolsep{\fill}}}
David Martin & Selston & 2054\\
Sam Wilson & Ind & 794\\
Mike Hollis & Lab & 355\\
Ray Young & UKIP & 161\\
Paul Saxelby & C & 103\\
\end{tabular*}

\council{Ashfield}

\subsubsection*{Selston \hspace*{\fill}\nolinebreak[1]%
\enspace\hspace*{\fill}
\finalhyphendemerits=0
[26th November]}

\index{Selston , Ashfield@Selston, \emph{Ashfield}}

Resignation of Gail Turner (Selston).

\noindent
\begin{tabular*}{\columnwidth}{@{\extracolsep{\fill}} p{0.545\columnwidth} >{\itshape}l r @{\extracolsep{\fill}}}
Christine Quinn-Wilcox & Selston & 1180\\
Anna Wilson & Ind & 294\\
Donna Gilbert & Lab & 172\\
Ray Young & UKIP & 77\\
Michelle Sims & C & 52\\
\end{tabular*}

\council{Bassetlaw}

At the May 2015 ordinary election there was an unfilled vacancy in East Retford North ward due to the resignation of Adele Mumby (Lab).
\index{East Retford North , Bassetlaw@East Retford N., \emph{Bassetlaw}}

\council{Newark and Sherwood}

At the May 2015 ordinary election there was an unfilled vacancy in Beacon ward due to the death of Marika Tribe (C).
\index{Beacon , Newark and Sherwood@Beacon, \emph{Newark \& Sherwood}}

\council{Rushcliffe}

At the May 2015 ordinary election there were unfilled vacancies in Manvers and Thoroton wards due to the deaths of David Smith and John Cranswick (both C) respectively.
\index{Manvers , Rushcliffe@Manvers, \emph{Rushcliffe}}
\index{Thoroton , Rushcliffe@Thoroton, \emph{Rushcliffe}}

\section{Oxfordshire}

\subsection*{County Council}\index{Oxfordshire}

\subsubsection*{Witney West and Bampton \hspace*{\fill}\nolinebreak[1]%
\enspace\hspace*{\fill}
\finalhyphendemerits=0
[7th May]}

\index{Witney West and Bampton , Oxfordshire@Witney W. \& Bampton, \emph{Oxon}}

Resignation of Simon Hoare (C).

\noindent
\begin{tabular*}{\columnwidth}{@{\extracolsep{\fill}} p{0.545\columnwidth} >{\itshape}l r @{\extracolsep{\fill}}}
James Mills & C & 3465\\
Calvert McGibbon & Lab & 800\\
Jim Stanley & UKIP & 678\\
Liz Leffman & LD & 472\\
Nick Owen & Grn & 462\\
\end{tabular*}

\council{Cherwell}

\subsubsection*{Caversfield \hspace*{\fill}\nolinebreak[1]%
\enspace\hspace*{\fill}
\finalhyphendemerits=0
[7th May]}

\index{Caversfield , Cherwell@Caversfield, \emph{Cherwell}}

Resignation of Jon O'Neill (C).

Combined with the 2015 ordinary election.
%; see page \pageref{CaversfieldCherwell} for the result.

\subsubsection*{Banbury Grimsbury and Castle \hspace*{\fill}\nolinebreak[1]%
\enspace\hspace*{\fill}
\finalhyphendemerits=0
[1st October; Lab gain from C]}

\index{Banbury Grimsbury and Castle , Cherwell@Banbury Grimsbury \& Castle, \emph{Cherwell}}

Death of Ann Bonner (C).

\noindent
\begin{tabular*}{\columnwidth}{@{\extracolsep{\fill}} p{0.545\columnwidth} >{\itshape}l r @{\extracolsep{\fill}}}
Shaida Hussain & Lab & 781\\
Tony Mepham & C & 661\\
Linda Wren & UKIP & 150\\
Kenneth Ashworth & LD & 73\\
Christopher Manley & Grn & 72\\
\end{tabular*}

\council{Oxford}

\subsubsection*{Northfield Brook \hspace*{\fill}\nolinebreak[1]%
\enspace\hspace*{\fill}
\finalhyphendemerits=0
[22nd October]}

\index{Northfield Brook , Oxford@Northfield Brook, \emph{Oxford}}

Resignation of Scott Seamans (Lab).

\noindent
\begin{tabular*}{\columnwidth}{@{\extracolsep{\fill}} p{0.545\columnwidth} >{\itshape}l r @{\extracolsep{\fill}}}
Jennifer Pegg & Lab & 509\\
Joe Lawes & UKIP & 60\\
Gary Dixon & C & 47\\
Lucy Ayrton & Grn & 28\\
James Morbin & TUSC & 9\\
\end{tabular*}

\council{South Oxfordshire}

At the May 2015 ordinary election there were unfilled vacancies in Goring and Thame North wards due to the deaths of Ann Ducker and Michael Welply (both C) respectively.
\index{Goring , South Oxfordshire@Goring, \emph{S. Oxon.}}
\index{Thame North , South Oxfordshire@Thame N., \emph{S. Oxon.}}

\subsubsection*{Sandford and The Wittenhams \hspace*{\fill}\nolinebreak[1]%
\enspace\hspace*{\fill}
\finalhyphendemerits=0
[8th October]}

\index{Sandford and Wittenhams , South Oxfordshire@Sandford \& The Wittenhams, \emph{S. Oxon.}}

Resignation of Jon Woodley-Shead (C).

\noindent
\begin{tabular*}{\columnwidth}{@{\extracolsep{\fill}} p{0.545\columnwidth} >{\itshape}l r @{\extracolsep{\fill}}}
Sue Lawson & C & 290\\
Simon Thompson & LD & 249\\
Jim Merritt & Lab & 89\\
Sam Casey-Rerhaye & Grn & 50\\
\end{tabular*}

\subsubsection*{Sonning Common \hspace*{\fill}\nolinebreak[1]%
\enspace\hspace*{\fill}
\finalhyphendemerits=0
[22nd October]}

\index{Sonning Common , South Oxfordshire@Sonning Common, \emph{S. Oxon.}}

Resignation of Martin Akehurst (C).

\noindent
\begin{tabular*}{\columnwidth}{@{\extracolsep{\fill}} p{0.545\columnwidth} >{\itshape}l r @{\extracolsep{\fill}}}
William Hall & C & 635\\
David Winchester & Lab & 200\\
Sue Cooper & LD & 127\\
\end{tabular*}

\council{West Oxfordshire}

\subsubsection*{Hailey, Minster Lovell and Leafield \hspace*{\fill}\nolinebreak[1]%
\enspace\hspace*{\fill}
\finalhyphendemerits=0
[7th May]}

\index{Hailey, Minster Lovell and Leafield , West Oxfordshire@Hailey, Minster Lovell \& Leafield, \emph{W. Oxon.}}

Resignation of Simon Hoare (C).

Combined with the 2015 ordinary election.
%; see page \pageref{HaileyMinsterLovellLeafieldWOxon} for the result.

\subsubsection*{Witney North \hspace*{\fill}\nolinebreak[1]%
\enspace\hspace*{\fill}
\finalhyphendemerits=0
[20th August]}

\index{Witney North , West Oxfordshire@Witney N., \emph{W. Oxon}}

Resignation of David Snow (Ind elected as C).

\noindent
\begin{tabular*}{\columnwidth}{@{\extracolsep{\fill}} p{0.545\columnwidth} >{\itshape}l r @{\extracolsep{\fill}}}
Carol Reynolds & C & 264\\
Diane West & LD & 201\\
Brigitte Hickman & Grn & 136\\
Trevor License & Lab & 114\\
James Robertshaw & UKIP & 64\\
\end{tabular*}

\section{Shropshire}

\council{Shropshire}

\subsubsection*{Oswestry East \hspace*{\fill}\nolinebreak[1]%
\enspace\hspace*{\fill}
\finalhyphendemerits=0
[12th February]}

\index{Oswestry East , Shropshire@Oswestry E., \emph{Shrops.}}

Resignation of Martin Bennett (C).

\noindent
\begin{tabular*}{\columnwidth}{@{\extracolsep{\fill}} p{0.545\columnwidth} >{\itshape}l r @{\extracolsep{\fill}}}
John Price & C & 629\\
Claire Norris & Lab & 247\\
Duncan Kerr & Grn & 231\\
Amanda Woof & LD & 218\\
\end{tabular*}

\subsubsection*{Belle Vue \hspace*{\fill}\nolinebreak[1]%
\enspace\hspace*{\fill}
\finalhyphendemerits=0
[12th November]}

\index{Belle Vue , Shropshire@Belle Vue, \emph{Shrops.}}

Resignation of Mansel Williams (Lab).

\noindent
\begin{tabular*}{\columnwidth}{@{\extracolsep{\fill}} p{0.545\columnwidth} >{\itshape}l r @{\extracolsep{\fill}}}
Amy Liebich & Lab & 546\\
Andrew Wagner & C & 282\\
Beverley Baker & LD & 240\\
Sam Taylor & Grn & 75\\
\end{tabular*}

\subsubsection*{Meole \hspace*{\fill}\nolinebreak[1]%
\enspace\hspace*{\fill}
\finalhyphendemerits=0
[3rd December]}

\index{Meole , Shropshire@Meole, \emph{Shrops.}}

Resignation of Mike Owen (C).

\noindent
\begin{tabular*}{\columnwidth}{@{\extracolsep{\fill}} p{0.545\columnwidth} >{\itshape}l r @{\extracolsep{\fill}}}
Nic Laurens & C & 490\\
John Lewis & Lab & 303\\
Nat Green & LD & 223\\
David Morgan & UKIP & 64\\
John Newnham & Grn & 56\\
\end{tabular*}

\section{Somerset}

\subsection*{County Council}\index{Somerset}

\subsubsection*{Taunton North \hspace*{\fill}\nolinebreak[1]%
\enspace\hspace*{\fill}
\finalhyphendemerits=0
[7th May; C gain from LD]}

\index{Taunton North , Somerset@Taunton N., \emph{Somerset}}

Resignation of Claire Gordon (LD).

\noindent
\begin{tabular*}{\columnwidth}{@{\extracolsep{\fill}} p{0.545\columnwidth} >{\itshape}l r @{\extracolsep{\fill}}}
Michael Adkins & C & 1298\\
Barrie Hall & LD & 976\\
Libby Lisgo & Lab & 927\\
Robert Bainbridge & UKIP & 814\\
Alan Debenham & Grn & 326\\
\end{tabular*}

\section{Staffordshire}

\council{East Staffordshire}

At the May 2015 ordinary election there was an unfilled vacancy in Yoxall ward due to the death of Beryl Behague (C).
\index{Yoxall , East Staffordshire@Yoxall, \emph{E. Staffs.}}

\council{Lichfield}

At the May 2015 ordinary election there was an unfilled vacancy in St John's ward due to the death of John Wilks (C).
\index{Saint John's , Lichfield@St John's, \emph{Lichfield}}

\council{Newcastle-under-Lyme}

\subsubsection*{Clayton \hspace*{\fill}\nolinebreak[1]%
\enspace\hspace*{\fill}
\finalhyphendemerits=0
[7th May]}

\index{Clayton , Newcastle-under-Lyme@Clayton, \emph{Newcastle-under-Lyme}}

Death of Ann Heames (C).

Combined with the 2015 ordinary election.
%; see page \pageref{ClaytonNewcastleLyme} for the result.

\council{Stoke-on-Trent}

At the May 2015 ordinary election there was an unfilled vacancy in Birches Head and Central Forest Park ward due to the death of Paul Breeze (City Independent).
\index{Birches Head and Central Forest Park , Stoke-on-Trent@Birches Head \& Central Forest Park, \emph{Stoke-on-Trent}}

\columnbreak

\section{Suffolk}

\subsection*{County Council}\index{Suffolk}

\subsubsection*{Haverhill Cangle \hspace*{\fill}\nolinebreak[1]%
\enspace\hspace*{\fill}
\finalhyphendemerits=0	
[7th May]}

\index{Haverhill Cangle , Suffolk@Haverhill Cangle, \emph{Suffolk}}

Resignation of Anne Gower (C).

\noindent
\begin{tabular*}{\columnwidth}{@{\extracolsep{\fill}} p{0.545\columnwidth} >{\itshape}l r @{\extracolsep{\fill}}}
Tim Marks & C & 3001\\
John Burns & UKIP & 2313\\
Maureen Byrne & Lab & 2004\\
Ken Rolph & LD & 404\\
\end{tabular*}

\council{Babergh}

At the May 2015 ordinary election there was an unfilled vacancy in Great Cornard North ward due to the disqualification (non-attendance) of Neil MacMaster (Lab).
\index{Great Cornard North , Babergh@Great Cornard N., \emph{Babergh}}

\council{Ipswich}

\subsubsection*{St John's \hspace*{\fill}\nolinebreak[1]%
\enspace\hspace*{\fill}
\finalhyphendemerits=0
[7th May]}

\index{Saint John's , Ipswich@St John's, \emph{Ipswich}}

Resignation of Jen Stimson (Lab).

Combined with the 2015 ordinary election.
%; see page \pageref{StJohnsIpswich} for the result.

\subsubsection*{Sprites \hspace*{\fill}\nolinebreak[1]%
\enspace\hspace*{\fill}
\finalhyphendemerits=0
[7th May]}

\index{Sprites , Ipswich@Sprites, \emph{Ipswich}}

Resignation of Richard Kirby (Lab).

Combined with the 2015 ordinary election.
%; see page \pageref{SpritesIpswich} for the result.

\subsection*{St Edmundsbury}\index{Saint Edmundsbury@St Edmundsbury}

At the May 2015 ordinary election there was an unfilled vacancy in Haverhill East ward due to the death of Gordon Cox (UKIP elected as C).
\index{Haverhill East , Saint Edmundsbury@Haverhill E., \emph{St Edmundsbury}}

\council{Waveney}

At the May 2015 ordinary election there was an unfilled vacancy in St Margaret's ward due to the death of Roger Bellham (Lab).
\index{Saint Margaret's , Waveney@St Margaret's, \emph{Waveney}}

\section{Surrey}

\subsection*{County Council}\index{Surrey}

RAEE = Residents Associations of Epsom and Ewell

WeybInd = Weybridge Independents

\subsubsection*{Weybridge \hspace*{\fill}\nolinebreak[1]%
\enspace\hspace*{\fill}
\finalhyphendemerits=0
[7th May]}

\index{Weybridge , Surrey@Weybridge, \emph{Surrey}}

Resignation of Christian Mahne (C).

\noindent
\begin{tabular*}{\columnwidth}{@{\extracolsep{\fill}} p{0.485\columnwidth} >{\itshape}l r @{\extracolsep{\fill}}}
Ramon Gray & C & 4190\\
Peter Harman & WeybInd & 1899\\
Elinor Jones & Lab & 967\\
Joe Branco & UKIP & 622\\
\end{tabular*}

\subsubsection*{Epsom West \hspace*{\fill}\nolinebreak[1]%
\enspace\hspace*{\fill}
\finalhyphendemerits=0
[19th November; C gain from LD]}

\index{Epsom West , Surrey@Epsom W., \emph{Surrey}}

Resignation of Stella Lallement (LD).

\noindent
\begin{tabular*}{\columnwidth}{@{\extracolsep{\fill}} p{0.485\columnwidth} >{\itshape}l r @{\extracolsep{\fill}}}
Karan Persand & C & 612\\
Neil Dallen & RAEE & 591\\
Julie Morris & LD & 588\\
Kate Chinn & Lab & 578\\
Robert Leach & UKIP & 168\\
Chris Crook & Grn & 58\\
\end{tabular*}

\council{Elmbridge}

\subsubsection*{Long Ditton \hspace*{\fill}\nolinebreak[1]%
\enspace\hspace*{\fill}
\finalhyphendemerits=0
[23rd July]}

\index{Long Ditton , Elmbridge@Long Ditton, \emph{Elmbridge}}

Resignation of Toni Izard (LD).

\noindent
\begin{tabular*}{\columnwidth}{@{\extracolsep{\fill}} p{0.545\columnwidth} >{\itshape}l r @{\extracolsep{\fill}}}
Neil Houston & LD & 770\\
Hugh Evans & C & 611\\
Laura Harmour & Grn & 79\\
Susannah Cunningham & UKIP & 61\\
\end{tabular*}

\council{Guildford}

GGBG = Guildford Greenbelt Group

\subsubsection*{Ash South and Tongham \hspace*{\fill}\nolinebreak[1]%
\enspace\hspace*{\fill}
\finalhyphendemerits=0
[3rd December]}

\index{Ash South and Tongham , Guildford@Ash S. \& Tongham, \emph{Guildford}}

Resignation of Stephen Mansbridge (C).

\noindent
\begin{tabular*}{\columnwidth}{@{\extracolsep{\fill}} p{0.545\columnwidth} >{\itshape}l r @{\extracolsep{\fill}}}
Andrew Gomm & C & 540\\
Alan Hilliar & LD & 286\\
Kyle Greaves & UKIP & 153\\
Ramsey Nagaty & GGBG & 145\\
George Dokimakis & Lab & 125\\
\end{tabular*}

\council{Mole Valley}

\subsubsection*{Holmwoods (2) \hspace*{\fill}\nolinebreak[1]%
\enspace\hspace*{\fill}
\finalhyphendemerits=0
[18th June; 1 LD gain from UKIP]}

\index{Holmwoods , Mole Valley@Holmwoods, \emph{Mole Valley}}

Resignation of Stephen Musgrove (UKIP).

This by-election was combined with the 2015 ordinary election and scheduled for 7th May, but then postponed due to the death of outgoing councillor Mick Longhurst (LD) who had been nominated for re-election.  
%See page \pageref{HolmwoodsMoleValley} for the result.

\noindent
\begin{tabular*}{\columnwidth}{@{\extracolsep{\fill}} p{0.545\columnwidth} >{\itshape}l r @{\extracolsep{\fill}}}
Claire Malcomson & LD & 804\\
Clayton Wellman & LD & 768\\
James Baird & C & 492\\
Emma Whittinger & C & 458\\
Michael Foulston & UKIP & 201\\
Stephen Morgan & UKIP & 180\\
Jeff Zie & Grn & 105\\
Eugene Suggett & Grn & 78\\
\end{tabular*}

\council{Reigate and Banstead}

\subsubsection*{Reigate Hill \hspace*{\fill}\nolinebreak[1]%
\enspace\hspace*{\fill}
\finalhyphendemerits=0
[7th May]}

\index{Reigate Hill , Reigate and Banstead@Reigate Hill, \emph{Reigate \& Banstead}}

Resignation of Lisa Brunt (C).

Combined with the 2015 ordinary election.
%; see page \pageref{ReigateHillReigateBanstead} for the result.

\council{Runnymede}

\subsubsection*{Addlestone North \hspace*{\fill}\nolinebreak[1]%
\enspace\hspace*{\fill}
\finalhyphendemerits=0
[7th May]}

\index{Addlestone North , Runnymede@Addlestone N., \emph{Runnymede}}

Resignation of Nick Rogers (C).

Combined with the 2015 ordinary election.
%; see page \pageref{AddlestoneNorthRunnymede} for the result.

\council{Spelthorne}

\subsubsection*{Shepperton Town \hspace*{\fill}\nolinebreak[1]%
\enspace\hspace*{\fill}
\finalhyphendemerits=0
[Tuesday 15th December]}

\index{Shepperton Town , Spelthorne@Shepperton Town, \emph{Spelthorne}}

Death of Robert Watts (C).

\noindent
\begin{tabular*}{\columnwidth}{@{\extracolsep{\fill}} p{0.545\columnwidth} >{\itshape}l r @{\extracolsep{\fill}}}
Colin Barnard & C & 858\\
Brian Catt & UKIP & 180\\
Richard Dunn & LD & 154\\
Jonathan Button & Lab & 123\\
Nigel Scott & Grn & 68\\
\end{tabular*}

\council{Tandridge}

\subsubsection*{Whyteleafe \hspace*{\fill}\nolinebreak[1]%
\enspace\hspace*{\fill}
\finalhyphendemerits=0
[Wednesday 1st April; LD gain from C]}

\index{Whyteleafe , Tandridge@Whyteleafe, \emph{Tandridge}}

Disqualification (non-attendance) of Tom Dempsey (C).

\noindent
\begin{tabular*}{\columnwidth}{@{\extracolsep{\fill}} p{0.545\columnwidth} >{\itshape}l r @{\extracolsep{\fill}}}
David Lee & LD & 393\\
Peter Sweeney & C & 274\\
Martin Ferguson & UKIP & 119\\
\end{tabular*}

\council{Waverley}

At the May 2015 ordinary election there was an unfilled vacancy in Cranleigh East ward due to the death of Janet Somerville (C).
\index{Cranleigh East , Waverley@Cranleigh E., \emph{Waverley}}

\council{Woking}

\subsubsection*{Goldsworth East \hspace*{\fill}\nolinebreak[1]%
\enspace\hspace*{\fill}
\finalhyphendemerits=0
[8th October]}

\index{Goldsworth East , Woking@Goldsworth E., \emph{Woking}}

Resignation of Amanda Coulson (LD).

\noindent
\begin{tabular*}{\columnwidth}{@{\extracolsep{\fill}} p{0.545\columnwidth} >{\itshape}l r @{\extracolsep{\fill}}}
James Sanderson & LD & 594\\
Sonia Elbaraka & C & 562\\
James Butcher & Lab & 262\\
Tim Read & UKIP & 154\\
\end{tabular*}

\subsubsection*{Goldsworth West \hspace*{\fill}\nolinebreak[1]%
\enspace\hspace*{\fill}
\finalhyphendemerits=0
[8th October; C gain from LD]}

\index{Goldsworth West , Woking@Goldsworth W., \emph{Woking}}

Resignation of Denzil Coulson (LD).

\noindent
\begin{tabular*}{\columnwidth}{@{\extracolsep{\fill}} p{0.545\columnwidth} >{\itshape}l r @{\extracolsep{\fill}}}
Chitra Rana & C & 367\\
Tina Liddington & LD & 349\\
Robina Shaheen & Lab & 105\\
Troy de Leon & UKIP & 97\\
\end{tabular*}

\section{Warwickshire}

\subsection*{County Council}\index{Warwickshire}

\subsubsection*{Nuneaton Whitestone \hspace*{\fill}\nolinebreak[1]%
\enspace\hspace*{\fill}
\finalhyphendemerits=0
[13th August]}

\index{Nuneaton Whitestone , Warwickshire@Nuneaton Whitestone, \emph{Warks.}}

Resignation of Martin Heatley (C).

\noindent
\begin{tabular*}{\columnwidth}{@{\extracolsep{\fill}} p{0.545\columnwidth} >{\itshape}l r @{\extracolsep{\fill}}}
Jeff Morgan & C & 1281\\
Andrew Crichton & Lab & 503\\
Alwyn Waine & UKIP & 292\\
Mick Ludford & Grn & 119\\
\end{tabular*}

\council{Stratford-on-Avon}

At the May 2015 ordinary election there was an unfilled vacancy in Aston Cantlow ward due to the death of Sir William Lawrence, Bt.\ (C).
\index{Aston Cantlow , Stratford-on-Avon@Aston Cantlow \emph{Stratford-on-Avon}}

\council{Warwick}

At the May 2015 ordinary election there was an unfilled vacancy in Stoneleigh ward due to the death of Bertie Mackay (Ind).
\index{Stoneleigh , Warwick@Stoneleigh, \emph{Warwick}}

\section{Wiltshire}

\council{Wiltshire}

\subsubsection*{Bromham, Rowde and Potterne \hspace*{\fill}\nolinebreak[1]%
\enspace\hspace*{\fill}
\finalhyphendemerits=0
[7th May]}

\index{Bromham, Rowde and Potterne , Wiltshire@Bromham, Rowde \& Potterne, \emph{Wilts.}}

Resignation of Alan Rushton (C).

\noindent
\begin{tabular*}{\columnwidth}{@{\extracolsep{\fill}} p{0.545\columnwidth} >{\itshape}l r @{\extracolsep{\fill}}}
Anna Cuthbert & C & 1641\\
Paul Carter & UKIP & 519\\
Alan Rankin & LD & 401\\
David Wearn & Lab & 334\\
\end{tabular*}

\subsubsection*{Chippenham Hardenhuish \hspace*{\fill}\nolinebreak[1]%
\enspace\hspace*{\fill}
\finalhyphendemerits=0
[7th May; C gain from LD]}

\index{Chippenham Hardenhuish , Wiltshire@Chippenham Hardenhuish, \emph{Wilts.}}

Resignation of Nick Watts (LD).

\noindent
\begin{tabular*}{\columnwidth}{@{\extracolsep{\fill}} p{0.545\columnwidth} >{\itshape}l r @{\extracolsep{\fill}}}
Melody Thompson & C & 1154\\
Kenneth Brough & LD & 864\\
Tina Johnston & Grn & 416\\
\end{tabular*}

\subsubsection*{Salisbury St Edmund and Milford \hspace*{\fill}\nolinebreak[1]%
\enspace\hspace*{\fill}
\finalhyphendemerits=0
[26th November; C gain from LD]}

\index{Salisbury Saint Edmund and Milford , Wiltshire@Salisbury St Edmund \& Milford, \emph{Wilts.}}

Resignation of Helena McKeown (LD).

\noindent
\begin{tabular*}{\columnwidth}{@{\extracolsep{\fill}} p{0.545\columnwidth} >{\itshape}l r @{\extracolsep{\fill}}}
Atiqul Hoque & C & 425\\
Greg Condliffe & LD & 242\\
Mark Timbrell & Lab & 232\\
Michael Pope & Grn & 215\\
Diana Dallimore & Ind & 45\\
\end{tabular*}

\section{Worcestershire}

\sloppyword{ICHC = Independent Community and Health Concern}

\subsection*{County Council}\index{Worcestershire}

\subsubsection*{Stourport-on-Severn \hspace*{\fill}\nolinebreak[1]%
\enspace\hspace*{\fill}
\finalhyphendemerits=0
[17th December; C gain from ICHC]}

\index{Stourport-on-Severn , Worcestershire@Stourport-on-Severn, \emph{Worcs.}}

Death of Jim Parish (ICHC).

\noindent
\begin{tabular*}{\columnwidth}{@{\extracolsep{\fill}} p{0.545\columnwidth} >{\itshape}l r @{\extracolsep{\fill}}}
Tony Muir & C & 763\\
Nicky Martin & ICHC & 725\\
Jill Hawes & Lab & 581\\
John Holden & UKIP & 547\\
Phil Oliver & Grn & 42\\
\end{tabular*}

\council{Bromsgrove}

At the May 2015 ordinary election there was an unfilled vacancy in Stoke Prior ward due to the death of John Tidmarsh (C).
\index{Stoke Prior , Bromsgrove@Stoke Prior, \emph{Bromsgrove}}

\council{Malvern Hills}

\subsubsection*{Teme Valley \hspace*{\fill}\nolinebreak[1]%
\enspace\hspace*{\fill}
\finalhyphendemerits=0
[3rd December]}

\index{Teme Valley , Malvern Hills@Teme Valley, \emph{Malvern Hills}}

Death of Gill Farmer (C).

\noindent
\begin{tabular*}{\columnwidth}{@{\extracolsep{\fill}} p{0.545\columnwidth} >{\itshape}l r @{\extracolsep{\fill}}}
Caroline Palethorpe & C & 268\\
Daniel Walton & Lab & 96\\
Andrew Dolan & UKIP & 56\\
\end{tabular*}

\council{Redditch}

\subsubsection*{West \hspace*{\fill}\nolinebreak[1]%
\enspace\hspace*{\fill}
\finalhyphendemerits=0
[7th May]}

\index{West , Redditch@West, \emph{Redditch}}

Resignation of Michael Braley (Ind elected as C).

Combined with the 2015 ordinary election.
%; see page \pageref{WestRedditch} for the result.

\council{Wychavon}

At the May 2015 ordinary election there were unfilled vacancies in Badsey, and Elmley Castle and Somerville wards due to the death of Reg Jakeman and the resignation of Roma Kirke (both C) respectively.
\index{Badsey , Wychavon@Badsey, \emph{Wychavon}}
\index{Elmley Castle and Somerville , Wychavon@Elmley Castle \& Somerville, \emph{Wychavon}}

\subsubsection*{Droitwich East \hspace*{\fill}\nolinebreak[1]%
\enspace\hspace*{\fill}
\finalhyphendemerits=0
[30th July]}

\index{Droitwich East , Wychavon@Droitwich E., \emph{Wychavon}}

Death of Glenise Noyes (C).

\noindent
\begin{tabular*}{\columnwidth}{@{\extracolsep{\fill}} p{0.545\columnwidth} >{\itshape}l r @{\extracolsep{\fill}}}
Karen Tomalin & C & 496\\
Jacqui O'Reilly & Lab & 175\\
Andy Morgan & UKIP & 171\\
Rory Roberson & LD & 108\\
\end{tabular*}

\council{Wyre Forest}

\subsubsection*{Areley Kings and Riverside (3) \hspace*{\fill}\nolinebreak[1]%
\enspace\hspace*{\fill}
\finalhyphendemerits=0
[Tuesday 9th June]}

\index{Areley Kings and Riverside , Wyre Forest@Areley Kings \& Riverside, \emph{Wyre Forest}}

Ordinary election postponed from 7th May: death of candidate Nigel Thomas (ICHC).

New ward: boundary changes.

\noindent
\begin{tabular*}{\columnwidth}{@{\extracolsep{\fill}} p{0.545\columnwidth} >{\itshape}l r @{\extracolsep{\fill}}}
Ken Henderson & C & 662\\
Jamie Shaw & Lab & 633\\
Lin Henderson & C & 564\\
Vi Higgs & Lab & 561\\
Rob Lloyd & Lab & 532\\
Malcolm Gough & C & 492\\
John Thomas & ICHC & 404\\
Cliff Brewer & ICHC & 378\\
Dixon Sheppard & ICHC & 326\\
Martin Clapton & UKIP & 213\\
Ian Jones & UKIP & 209\\
Trevor Newman & UKIP & 196\\
John Davis & Grn & 66\\
\end{tabular*}

\subsubsection*{Blakebrook and Habberley South \hspace*{\fill}\nolinebreak[1]%
\enspace\hspace*{\fill}
\finalhyphendemerits=0
[24th September]}

\index{Blakebrook and Habberley South , Wyre Forest@Blakebrook \& Habberley S., \emph{Wyre Forest}}

Resignation of Ruth Gregory (C).

\noindent
\begin{tabular*}{\columnwidth}{@{\extracolsep{\fill}} p{0.545\columnwidth} >{\itshape}l r @{\extracolsep{\fill}}}
Tracey Onslow & C & 595\\
Michael Wrench & UKIP & 252\\
Bernadette Connor & Lab & 247\\
Natalie McVey & Grn & 173\\
Jim Lawson & ICHC & 167\\
Adrian Beavis & LD & 54\\
\end{tabular*}

\section{Glamorgan}

\council{Bridgend}

\subsubsection*{Newcastle \hspace*{\fill}\nolinebreak[1]%
\enspace\hspace*{\fill}
\finalhyphendemerits=0
[7th May]}

\index{Newcastle , Bridgend@Newcastle, \emph{Bridgend}}

Resignation of Christina Rees (Lab).

\noindent
\begin{tabular*}{\columnwidth}{@{\extracolsep{\fill}} p{0.545\columnwidth} >{\itshape}l r @{\extracolsep{\fill}}}
Neelo Farr & Lab & 1164\\
Alex Williams & C & 938\\
Hazel Kendall & UKIP & 526\\
\end{tabular*}

\subsubsection*{Ogmore Vale \hspace*{\fill}\nolinebreak[1]%
\enspace\hspace*{\fill}
\finalhyphendemerits=0
[12th November; Lab gain from Ind]}

\index{Ogmore Vale , Bridgend@Ogmore Vale, \emph{Bridgend}}

Resignation of Della Hughes (Ind).

\noindent
\begin{tabular*}{\columnwidth}{@{\extracolsep{\fill}} p{0.545\columnwidth} >{\itshape}l r @{\extracolsep{\fill}}}
Dhanisha Patel & Lab & 365\\
Ralph Shepherd & Ind & 203\\
Tim Thomas & PC & 101\\
Jamie Wallis & C & 47\\
Sally Hyde & UKIP & 32\\
\end{tabular*}

\council{Cardiff}

\subsubsection*{Pentyrch \hspace*{\fill}\nolinebreak[1]%
\enspace\hspace*{\fill}
\finalhyphendemerits=0
[Tuesday 30th June]}

\index{Pentyrch , Cardiff@Pentyrch, \emph{Cadiff}}

Resignation of Craig Williams (C).

\noindent
\begin{tabular*}{\columnwidth}{@{\extracolsep{\fill}} p{0.545\columnwidth} >{\itshape}l r @{\extracolsep{\fill}}}
Gavin Hill John & C & 561\\
Hywel Wigley & PC & 543\\
Paul Fisher & Lab & 234\\
Munawar Mughal & Ind & 24\\
Ruth Osner & Grn & 22\\
Cadan ap Tomos & LD & 10\\
\end{tabular*}

\subsubsection*{Riverside \hspace*{\fill}\nolinebreak[1]%
\enspace\hspace*{\fill}
\finalhyphendemerits=0
[Wednesday 7th October]}

\index{Riverside , Cardiff@Riverside, \emph{Cardiff}}

Resignation of Cecilia Love (Lab).

\noindent
\begin{tabular*}{\columnwidth}{@{\extracolsep{\fill}} p{0.545\columnwidth} >{\itshape}l r @{\extracolsep{\fill}}}
Caro Wild & Lab & 1071\\
Ruksana Begum & PC & 780\\
Sean Driscoll & C & 155\\
Gareth Bennett & UKIP & 110\\
Hannah Pudner & Grn & 109\\
Gwilym Owen & LD & 85\\
Steffan Bateman & TUSC & 21\\
\end{tabular*}

\council{Merthyr Tydfil}

\subsubsection*{Gurnos \hspace*{\fill}\nolinebreak[1]%
\enspace\hspace*{\fill}
\finalhyphendemerits=0
[7th May]}

\index{Gurnos , Merthyr Tydfil@Gurnos, \emph{Merthyr Tydfil}}

Death of David Jarrett (Lab).

\noindent
\begin{tabular*}{\columnwidth}{@{\extracolsep{\fill}} p{0.545\columnwidth} >{\itshape}l r @{\extracolsep{\fill}}}
Mike O'Neill & Lab & 706\\
Gregory Canavan & Ind & 698\\
Lee Davies & Ind & 340\\
\end{tabular*}

\subsubsection*{Town \hspace*{\fill}\nolinebreak[1]%
\enspace\hspace*{\fill}
\finalhyphendemerits=0
[7th May]}

\index{Town , Merthyr Tydfil@Town, \emph{Merthyr Tydfil}}

Resignation of Graham Davies (Lab).

\noindent
\begin{tabular*}{\columnwidth}{@{\extracolsep{\fill}} p{0.545\columnwidth} >{\itshape}l r @{\extracolsep{\fill}}}
Sian Slater & Lab & 2182\\
Bob Griffin & LD & 1126\\
\end{tabular*}

\council{Neath Port Talbot}

\subsubsection*{Cwmllynfell \hspace*{\fill}\nolinebreak[1]%
\enspace\hspace*{\fill}
\finalhyphendemerits=0
[7th May]}

\index{Cwmllynfell , Neath Port Talbot@Cwmllynfell, \emph{Neath Port Talbot}}

Resignation of Cliff Richards (Lab).

\noindent
\begin{tabular*}{\columnwidth}{@{\extracolsep{\fill}} p{0.545\columnwidth} >{\itshape}l r @{\extracolsep{\fill}}}
Kris Lloyd & Lab & 365\\
Meirion Jordan & PC & 222\\
Jude Bailey-Murfin & Grn & 52\\
\end{tabular*}

\council{Swansea}

\subsubsection*{Cockett \hspace*{\fill}\nolinebreak[1]%
\enspace\hspace*{\fill}
\finalhyphendemerits=0
[7th May]}

\index{Cockett , Swansea@Cockett, \emph{Swansea}}

Resignation of Mitch Theaker (Lab).

\noindent
\begin{tabular*}{\columnwidth}{@{\extracolsep{\fill}} p{0.545\columnwidth} >{\itshape}l r @{\extracolsep{\fill}}}
Elliott King & Lab & 2741\\
Dai Lloyd & PC & 1345\\
Rebecca Singh & C & 899\\
Nicola Holley & LD & 776\\
Mike Whittall & Grn & 314\\
Dave Phillips & TUSC & 128\\
\end{tabular*}

\subsubsection*{Morriston \hspace*{\fill}\nolinebreak[1]%
\enspace\hspace*{\fill}
\finalhyphendemerits=0
[7th May]}

\index{Morriston , Swansea@Morriston, \emph{Swansea}}

Resignation of John Davies (Lab).

\noindent
\begin{tabular*}{\columnwidth}{@{\extracolsep{\fill}} p{0.545\columnwidth} >{\itshape}l r @{\extracolsep{\fill}}}
Ceri Evans & Lab & 3988\\
Thomas Morgan & C & 1469\\
Dic Jones & PC & 1344\\
John Jenkins & LD & 372\\
\end{tabular*}

\subsubsection*{Townhill \hspace*{\fill}\nolinebreak[1]%
\enspace\hspace*{\fill}
\finalhyphendemerits=0
[7th May]}

\index{Townhill , Swansea@Townhill, \emph{Swansea}}

Resignation of Nick Bradley (Lab).

\noindent
\begin{tabular*}{\columnwidth}{@{\extracolsep{\fill}} p{0.545\columnwidth} >{\itshape}l r @{\extracolsep{\fill}}}
Cyril Anderson & Lab & 1773\\
Tim Orr & UKIP & 523\\
Patrick Powell & PC & 159\\
Tom Giffard & C & 143\\
Kevin Searley & Grn & 130\\
Michael O'Carroll & LD & 82\\
Owen Herbert & TUSC & 56\\
\end{tabular*}

\council{Vale of Glamorgan}

Llantwit = Llantwit First Independent

\subsubsection*{Llantwit Major \hspace*{\fill}\nolinebreak[1]%
\enspace\hspace*{\fill}
\finalhyphendemerits=0
[26th March; C gain from Llantwit]}

\index{Llantwit Major , Vale of Glamorgan@Llantwit Major, \emph{Vale of Glamorgan}}

Death of Keith Geary (Llantwit).

\noindent
\begin{tabular*}{\columnwidth}{@{\extracolsep{\fill}} p{0.495\columnwidth} >{\itshape}l r @{\extracolsep{\fill}}}
Tony Bennett & C & 1016\\
Mick Mason & Llantwit & 1004\\
Jack Hawkins & Lab & 378\\
Wynford Bellin & PC & 95\\
\end{tabular*}

\section{Gwent}

\council{Caerphilly}

\subsubsection*{New Tredegar \hspace*{\fill}\nolinebreak[1]%
\enspace\hspace*{\fill}
\finalhyphendemerits=0
[23rd July]}

\index{New Tredegar , Caerphilly@New Tredegar, \emph{Caerphilly}}

Resignation of Gerald Jones (Lab).

\noindent
\begin{tabular*}{\columnwidth}{@{\extracolsep{\fill}} p{0.495\columnwidth} >{\itshape}l r @{\extracolsep{\fill}}}
Mark Evans & Lab & 648\\
Ian Gorman & UKIP & 90\\
Robert Lea & C & 47\\
\end{tabular*}

\subsubsection*{Crosskeys \hspace*{\fill}\nolinebreak[1]%
\enspace\hspace*{\fill}
\finalhyphendemerits=0
[13th August]}

\index{Crosskeys , Caerphilly@Crosskeys, \emph{Caerphilly}}

Resignation of Michael Gray (Lab).

\noindent
\begin{tabular*}{\columnwidth}{@{\extracolsep{\fill}} p{0.495\columnwidth} >{\itshape}l r @{\extracolsep{\fill}}}
Julian Simmonds & Lab & 354\\
Chris Cook & PC & 179\\
Joe Smyth & UKIP & 166\\
\end{tabular*}

\subsubsection*{Bedwas, Trethomas and Machen \hspace*{\fill}\nolinebreak[1]%
\enspace\hspace*{\fill}
\finalhyphendemerits=0
[3rd September]}

\index{Bedwas, Trethomas and Machen , Caerphilly@Bedwas, Trethomas \& Machen, \emph{Caerphilly}}

Death of Ray Davies (Lab).

\noindent
\begin{tabular*}{\columnwidth}{@{\extracolsep{\fill}} p{0.495\columnwidth} >{\itshape}l r @{\extracolsep{\fill}}}
Lisa Jones & Lab & 1002\\
Ron Davies & PC & 509\\
Bobby Douglas & UKIP & 223\\
John Dew & Ind & 184\\
Rita Lukins & C & 119\\
Ray Davies & Ind & 106\\
\end{tabular*}

\council{Newport}

\subsubsection*{Bettws \hspace*{\fill}\nolinebreak[1]%
\enspace\hspace*{\fill}
\finalhyphendemerits=0
[26th November]}

\index{Bettws , Newport@Bettws, \emph{Newport}}

Resignation of Noel Trigg (Ind).

\noindent
\begin{tabular*}{\columnwidth}{@{\extracolsep{\fill}} p{0.495\columnwidth} >{\itshape}l r @{\extracolsep{\fill}}}
Janet Cleverly & Ind & 336\\
Glyn Jarvis & Lab & 294\\
Jason Jordan & Ind & 275\\
Lewis Williams & C & 114\\
Peter Varley & Grn & 29\\
Paul l'Allier & LD & 7\\
\end{tabular*}

\section{Mid and West Wales}

\council{Carmarthenshire}

PF = People First

\subsubsection*{Hengoed \hspace*{\fill}\nolinebreak[1]%
\enspace\hspace*{\fill}
\finalhyphendemerits=0
[19th February]}

\index{Hengoed , Carmarthenshire@Hengoed, \emph{Carmarthenshire}}

Death of George Edwards (Lab).

\noindent
\begin{tabular*}{\columnwidth}{@{\extracolsep{\fill}} p{0.545\columnwidth} >{\itshape}l r @{\extracolsep{\fill}}}
Penny Edwards & Lab & 335\\
Martin Davies & PC & 313\\
Bramwell Richards & UKIP & 152\\
Wynford Samuel & PF & 80\\
Edward Skinner & Ind & 76\\
Stephen Davies & C & 54\\
\end{tabular*}

\subsubsection*{Kidwelly \hspace*{\fill}\nolinebreak[1]%
\enspace\hspace*{\fill}
\finalhyphendemerits=0
[19th November]}

\index{Kidwelly , Carmarthenshire@Kidwelly, \emph{Carmarthenshire}}

Death of Keith Davies (Lab).

\noindent
\begin{tabular*}{\columnwidth}{@{\extracolsep{\fill}} p{0.545\columnwidth} >{\itshape}l r @{\extracolsep{\fill}}}
Ryan Thomas & Lab & 288\\
Dilwyn Jones & PC & 248\\
Fran Burke-Lloyd & Ind & 177\\
Stephen Bowen & PF & 58\\
Stephen Davies & C & 53\\
Vivian Summers & Ind & 28\\
\end{tabular*}

\council{Powys}

\subsubsection*{Glasbury \hspace*{\fill}\nolinebreak[1]%
\enspace\hspace*{\fill}
\finalhyphendemerits=0
[13th August; LD gain from C]}

\index{Glasbury , Powys@Glasbury, \emph{Powys}}

Resignation of Chris Davies MP (C).

\noindent
\begin{tabular*}{\columnwidth}{@{\extracolsep{\fill}} p{0.545\columnwidth} >{\itshape}l r @{\extracolsep{\fill}}}
James Gibson-Watt & LD & 457\\
James Evans & C & 415\\
David Hood & Ind & 106\\
Louise Davies & Grn & 52\\
\end{tabular*}

\section{North Wales}

\council{Conwy}

\subsubsection*{Eglwysbach \hspace*{\fill}\nolinebreak[1]%
\enspace\hspace*{\fill}
\finalhyphendemerits=0
[12th November]}

\index{Eglwysbach , Conwy@Eglwysbach, \emph{Conwy}}

Resignation of Mike Rayner (PC).

\noindent
\begin{tabular*}{\columnwidth}{@{\extracolsep{\fill}} p{0.545\columnwidth} >{\itshape}l r @{\extracolsep{\fill}}}
Austin Roberts & PC & 369\\
Hazel Meredith & C & 56\\
\end{tabular*}

\subsubsection*{Gogarth \hspace*{\fill}\nolinebreak[1]%
\enspace\hspace*{\fill}
\finalhyphendemerits=0
[Wednesday 16th December]}

\index{Gogarth , Conwy@Gogarth, \emph{Conwy}}

Resignation of Janet Howarth (C).

\noindent
\begin{tabular*}{\columnwidth}{@{\extracolsep{\fill}} p{0.545\columnwidth} >{\itshape}l r @{\extracolsep{\fill}}}
Harry Saville & C & 318\\
Deborah Good & Lab & 187\\
Greg Robbins & PC & 167\\
Roger Boon & LD & 49\\
Mark Pavey & Ind & 28\\
\end{tabular*}

\council{Denbighshire}

\subsubsection*{Rhyl South West \hspace*{\fill}\nolinebreak[1]%
\enspace\hspace*{\fill}
\finalhyphendemerits=0
[19th March]}

\index{Rhyl South West , Denbighshire@Rhyl S.W., \emph{Denbighshire}}

Death of Margaret McCarroll (Lab).

\noindent
\begin{tabular*}{\columnwidth}{@{\extracolsep{\fill}} p{0.545\columnwidth} >{\itshape}l r @{\extracolsep{\fill}}}
Pete Prendergast & Lab & 474\\
Melanie Jones & C & 131\\
David Wilmot & PC & 62\\
Diana Hannam & Ind & 52\\
David Dear & LD & 26\\
\end{tabular*}

\subsubsection*{Prestatyn East \hspace*{\fill}\nolinebreak[1]%
\enspace\hspace*{\fill}
\finalhyphendemerits=0
[16th July]}

\index{Prestatyn East , Denbighshire@Prestatyn E., \emph{Denbighshire}}

Resignation of James Davies (C).

\noindent
\begin{tabular*}{\columnwidth}{@{\extracolsep{\fill}} p{0.545\columnwidth} >{\itshape}l r @{\extracolsep{\fill}}}
Anton Sampson & C & 528\\
Kenneth Wells & Lab & 373\\
Michael German & Ind & 76\\
Janice Williams & PC & 60\\
\end{tabular*}

\council{Flintshire}

\subsubsection*{Caergwrle \hspace*{\fill}\nolinebreak[1]%
\enspace\hspace*{\fill}
\finalhyphendemerits=0
[7th May]}

\index{Caergwrle , Flintshire@Caergwrle, \emph{Flintshire}}

Death of Stella Jones (Lab).

\noindent
\begin{tabular*}{\columnwidth}{@{\extracolsep{\fill}} p{0.6\columnwidth} >{\itshape}l r @{\extracolsep{\fill}}}
David Healey & Lab & 380\\
Christine Cunnah & Ind & 246\\
Jacqueline Hirst & PC & 84\\
Patrick Heesom & Ind & 72\\
David Chamberlain-Jones & C & 67\\
\end{tabular*}

\council{Gwynedd}

LlG = Llais Gwynedd

\subsubsection*{Cadnant \hspace*{\fill}\nolinebreak[1]%
\enspace\hspace*{\fill}
\finalhyphendemerits=0
[2nd April; Lab gain from PC]}

\index{Cadnant , Gwynedd@Cadnant, \emph{Gwynedd}}

Death of Huw Edwards (PC).

\noindent
\begin{tabular*}{\columnwidth}{@{\extracolsep{\fill}} p{0.545\columnwidth} >{\itshape}l r @{\extracolsep{\fill}}}
Glyn Thomas & Lab & 233\\
Glyn Tomos & PC & 185\\
Jason Parry & LlG & 148\\
Maria Sarnacki & Ind & 94\\
David Hibbert & C & 22\\
\end{tabular*}

\subsubsection*{Morfa Nefyn \hspace*{\fill}\nolinebreak[1]%
\enspace\hspace*{\fill}
\finalhyphendemerits=0
[9th July]}

\index{Morfa Nefyn , Gwynedd@Morfa Nefyn, \emph{Gwynedd}}

Resignation of Liz Saville Roberts (PC).

\noindent
\begin{tabular*}{\columnwidth}{@{\extracolsep{\fill}} p{0.545\columnwidth} >{\itshape}l r @{\extracolsep{\fill}}}
Sian Hughes & PC & 315\\
Wini Jones Lewis & LlG & 123\\
\end{tabular*}

\subsubsection*{Dewi \hspace*{\fill}\nolinebreak[1]%
\enspace\hspace*{\fill}
\finalhyphendemerits=0
[19th November]}

\index{Dewi , Gwynedd@Dewi, \emph{Gwynedd}}

Resignation of Eddie Dogan (PC).

\noindent
\begin{tabular*}{\columnwidth}{@{\extracolsep{\fill}} p{0.545\columnwidth} >{\itshape}l r @{\extracolsep{\fill}}}
Gareth Roberts & PC & 189\\
Eirian Roberts & Lab & 110\\
Andrew Joyce & LD & 19\\
\end{tabular*}

\subsubsection*{Llanaelhaearn \hspace*{\fill}\nolinebreak[1]%
\enspace\hspace*{\fill}
\finalhyphendemerits=0
[19th November; PC gain from LlG]}

\index{Llanaelhaearn , Gwynedd@Llanaelhaearn, \emph{Gwynedd}}

Resignation of Llywarch Jones (LlG).

\noindent
\begin{tabular*}{\columnwidth}{@{\extracolsep{\fill}} p{0.545\columnwidth} >{\itshape}l r @{\extracolsep{\fill}}}
Aled Jones & PC & 200\\
Wynne Isaac & LlG & 112\\
Eric Cullen & Ind & 99\\
\end{tabular*}

\subsubsection*{Pwllheli South \hspace*{\fill}\nolinebreak[1]%
\enspace\hspace*{\fill}
\finalhyphendemerits=0
[26th November; Ind gain from LlG]}

\index{Pwllheli South , Gwynedd@Pwllheli S., \emph{Gwynedd}}

Death of Bob Wright (LlG).

\noindent
\begin{tabular*}{\columnwidth}{@{\extracolsep{\fill}} p{0.545\columnwidth} >{\itshape}l r @{\extracolsep{\fill}}}
Hefin Underwood & Ind & 269\\
Alan Williams & PC & 168\\
Michael Parry & Ind & 106\\
Peta Pollitt & LlG & 49\\
\end{tabular*}

\council{Wrexham}

\subsubsection*{Llay \hspace*{\fill}\nolinebreak[1]%
\enspace\hspace*{\fill}
\finalhyphendemerits=0
[16th July; LD gain from Lab]}

\index{Llay , Wrexham@Llay, \emph{Wrexham}}

Death of Malcolm Taylor (Lab).

\noindent
\begin{tabular*}{\columnwidth}{@{\extracolsep{\fill}} p{0.545\columnwidth} >{\itshape}l r @{\extracolsep{\fill}}}
Rob Walsh & LD & 700\\
Pete Evans & Lab & 353\\
David Broderick & Ind & 124\\
Emlyn Phennah & C & 64\\
Jeanette Bassford-Barton & UKIP & 60\\
David Dodd & Ind & 41\\
\end{tabular*}

\section{Aberdeen City and Shire}

\council{Aberdeen}

\subsubsection*{Hilton\slash Woodside\slash Stockethill \hspace*{\fill}\nolinebreak[1]%
\enspace\hspace*{\fill}
\finalhyphendemerits=0
[30th July]}

\index{Hilton/Woodside/Stockethill , Aberdeen@\sloppyword{Hilton\slash Woodside\slash Stockethill, \emph{Aberdeen}}}

Resignation of Kirsty Blackman (SNP).

This ward was renamed in 2013: it was previously known as Hilton\slash Stockethill.

\noindent
\begin{tabular*}{\columnwidth}{@{\extracolsep{\fill}} p{0.545\columnwidth} >{\itshape}l r @{\extracolsep{\fill}}}
Neil Copland & SNP & 1690\\
Charlie Pirie & Lab & 771\\
Roy Begg & C & 350\\
Peter Kennedy & Grn & 130\\
Jonathan Waddell & LD & 125\\
\end{tabular*}

\subsubsection*{Kincorth\slash Nigg\slash Cove \hspace*{\fill}\nolinebreak[1]%
\enspace\hspace*{\fill}
\finalhyphendemerits=0
[30th July]}

\index{Kincorth/Nigg/Cove , Aberdeen@Kincorth\slash Nigg\slash Cove, \emph{Aberdeen}}

Resignation of Callum McCaig (SNP).

This ward was renamed in 2013: it was previously known as Kincorth\slash Loirston.

\noindent
\begin{tabular*}{\columnwidth}{@{\extracolsep{\fill}} p{0.545\columnwidth} >{\itshape}l r @{\extracolsep{\fill}}}
Stephen Flynn & SNP & 1939\\
Donna Clark & Lab & 606\\
Philip Sellar & C & 313\\
Ken McLeod & LD & 207\\
Dan Yeats & Grn & 114\\
\end{tabular*}

\subsubsection*{George Street\slash Harbour \hspace*{\fill}\nolinebreak[1]%
\enspace\hspace*{\fill}
\finalhyphendemerits=0
[1st October]}

\index{George Street/Harbour , Aberdeen@George St.\slash Harbour, \emph{Aberdeen}}

Resignation of Andy May (SNP).

\noindent
\begin{tabular*}{\columnwidth}{@{\extracolsep{\fill}} p{0.545\columnwidth} >{\itshape}l r @{\extracolsep{\fill}}}
Michael Hutchison & SNP & 961\\
Mike Scott & Lab & 490\\
Brian Davidson & C & 195\\
Alex Jarvis & Grn & 136\\
Euan Davidson & LD & 96\\
\end{tabular*}

\columnbreak

\subsubsection*{Midstocket\slash Rosemount \hspace*{\fill}\nolinebreak[1]%
\enspace\hspace*{\fill}
\finalhyphendemerits=0
[1st October; SNP gain from C]}

\index{Midstocket/Rosemount , Aberdeen@Midstocket\slash Rosemount, \emph{Aberdeen}}

Resignation of Fraser Forsyth (Ind elected as C).

\noindent
\begin{tabular*}{\columnwidth}{@{\extracolsep{\fill}} p{0.545\columnwidth} >{\itshape}l r @{\extracolsep{\fill}}}
\emph{First preferences}\\
Alex Nicoll & SNP & 1168\\
Tom Mason & C & 672\\
Howard Gemmell & Lab & 605\\
Ken McLeod & LD & 238\\
Jennifer Phillips & Grn & 170\\
\end{tabular*}

\emph{McLeod and Phillips eliminated}: Nicoll 1275 Mason 771 Gemmell 692

\noindent
\begin{tabular*}{\columnwidth}{@{\extracolsep{\fill}} p{0.545\columnwidth} >{\itshape}l r @{\extracolsep{\fill}}}
\emph{Gemmell eliminated}\\
Alex Nicoll & SNP & 1433\\
Tom Mason & C & 927\\
\end{tabular*}

\council{Aberdeenshire}

Libtn = Libertarian Party

\subsubsection*{Huntly, Strathbogie and Howe of Alford (2) \hspace*{\fill}\nolinebreak[1]%
\enspace\hspace*{\fill}
\finalhyphendemerits=0
[5th November; 1 C gain from LD]}

\index{Huntly, Strathbogie and Howe of Alford , Aberdeenshire@Huntly, Strathbogie \& Howe of Alford, \emph{Aberdeenshire}}

Death of Joanna Strathdee (SNP) and resignation of Alastair Ross (LD).

\noindent
\begin{tabular*}{\columnwidth}{@{\extracolsep{\fill}} p{0.545\columnwidth} >{\itshape}l r @{\extracolsep{\fill}}}
\emph{First preferences}\\
Margo Stewart & C & 1469\\
Gwyneth Petrie & SNP & 1433\\
Daniel Millican & LD & 928\\
Sarah Flavell & Lab & 196\\
Derek Scott & Libtn & 20\\
\end{tabular*}

\emph{Quota = 1349, Stewart and Petrie elected, count complete.}

\section{Ayrshire Councils}

\council{East Ayrshire}

\subsubsection*{Irvine Valley \hspace*{\fill}\nolinebreak[1]%
\enspace\hspace*{\fill}
\finalhyphendemerits=0
[1st October]}

\index{Irvine Valley , East Ayrshire@Irvine Valley, \emph{E. Ayrshire}}

Resignation of Alan Brown (SNP).

\noindent
\begin{tabular*}{\columnwidth}{@{\extracolsep{\fill}} p{0.545\columnwidth} >{\itshape}l r @{\extracolsep{\fill}}}
\emph{First preferences}\\
Elena Whitham & SNP & 1797\\
Susan McFadzean & C & 865\\
Alex Walsh & Lab & 860\\
Jen Broadhurst & Grn & 88\\
\end{tabular*}

\noindent
\begin{tabular*}{\columnwidth}{@{\extracolsep{\fill}} p{0.545\columnwidth} >{\itshape}l r @{\extracolsep{\fill}}}
\emph{Broadhurst eliminated}\\
Elena Whitham & SNP & 1832\\
Susan McFadzean & C & 884\\
Alex Walsh & Lab & 884\\
\end{tabular*}

\council{South Ayrshire}

\subsubsection*{Ayr East \hspace*{\fill}\nolinebreak[1]%
\enspace\hspace*{\fill}
\finalhyphendemerits=0
[17th September]}

\index{Ayr East , South Ayrshire@Ayr E., \emph{S. Ayrshire}}

Resignation of Corri Wilson (SNP).

\noindent
\begin{tabular*}{\columnwidth}{@{\extracolsep{\fill}} p{0.545\columnwidth} >{\itshape}l r @{\extracolsep{\fill}}}
\emph{First preferences}\\
Dan McCroskrie & C & 1527\\
John Wallace & SNP & 1507\\
Susan Wilson & Lab & 642\\
Andrew Bryden & Ind & 218\\
Boyd Murdoch & Grn & 76\\
\end{tabular*}

\noindent
\begin{tabular*}{\columnwidth}{@{\extracolsep{\fill}} p{0.545\columnwidth} >{\itshape}l r @{\extracolsep{\fill}}}
\multicolumn{3}{@{\extracolsep{\fill}}l}{\emph{Wilson, Bryden and Murdoch eliminated}}\\
%\emph{}\\
John Wallace & SNP & 1775\\
Dan McCroskrie & C & 1740\\
\end{tabular*}

\section{Clyde Councils}

\council{Glasgow}

\subsubsection*{Anderston\slash City \hspace*{\fill}\nolinebreak[1]%
\enspace\hspace*{\fill}
\finalhyphendemerits=0
[6th August]}

\index{Anderston/City , Glasgow@Anderston\slash City, \emph{Glasgow}}

Resignation of Martin Docherty (SNP).

\noindent
\begin{tabular*}{\columnwidth}{@{\extracolsep{\fill}} p{0.545\columnwidth} >{\itshape}l r @{\extracolsep{\fill}}}
\emph{First preferences}\\
Eva Bolander & SNP & 1441\\
Katie Ford & Lab & 857\\
Christy Mearns & Grn & 414\\
Ary Jaff & C & 164\\
Gary McLelland & LD & 66\\
Janice MacKay & UKIP & 43\\
Stevie Creighton & Libtn & 12\\
\end{tabular*}

\emph{McLelland, MacKay and Creighton eliminated}: Bolander 1457 Ford 881 Mearns 442 Jaff 184

\noindent
\begin{tabular*}{\columnwidth}{@{\extracolsep{\fill}} p{0.545\columnwidth} >{\itshape}l r @{\extracolsep{\fill}}}
\emph{Jaff eliminated}\\
Eva Bolander & SNP & 1473\\
Katie Ford & Lab & 943\\
Christy Mearns & Grn & 467\\
\end{tabular*}

\subsubsection*{Calton \hspace*{\fill}\nolinebreak[1]%
\enspace\hspace*{\fill}
\finalhyphendemerits=0
[6th August]}

\index{Calton , Glasgow@Calton, \emph{Glasgow}}

Resignation of Alison Thewliss (SNP).

\noindent
\begin{tabular*}{\columnwidth}{@{\extracolsep{\fill}} p{0.545\columnwidth} >{\itshape}l r @{\extracolsep{\fill}}}
Greg Hepburn & SNP & 1507\\
Thomas Rannachan & Lab & 814\\
Thomas Kerr & C & 129\\
Karen King & UKIP & 103\\
Malachy Clarke & Grn & 99\\
Tommy Ramsay & Ind & 47\\
Chris Young & LD & 18\\
\end{tabular*}

\subsubsection*{Craigton \hspace*{\fill}\nolinebreak[1]%
\enspace\hspace*{\fill}
\finalhyphendemerits=0
[6th August]}

\index{Craigton , Glasgow@Craigton, \emph{Glasgow}}

Resignation of Iris Gibson (SNP).

\noindent
\begin{tabular*}{\columnwidth}{@{\extracolsep{\fill}} p{0.545\columnwidth} >{\itshape}l r @{\extracolsep{\fill}}}
Alex Wilson & SNP & 2674\\
Kevin O'Donnell & Lab & 1643\\
Phillip Charles & C & 300\\
Katie Noble & Grn & 136\\
Arthur Thackeray & UKIP & 95\\
Isabel Nelson & LD & 87\\
\end{tabular*}

\subsubsection*{Langside \hspace*{\fill}\nolinebreak[1]%
\enspace\hspace*{\fill}
\finalhyphendemerits=0
[6th August; SNP gain from Grn]}

\index{Langside , Glasgow@Langside, \emph{Glasgow}}

Resignation of Liam Hainey (Grn).

\noindent
\begin{tabular*}{\columnwidth}{@{\extracolsep{\fill}} p{0.545\columnwidth} >{\itshape}l r @{\extracolsep{\fill}}}
\emph{First preferences}\\
Anna Richardson & SNP & 2134\\
Eileen Dinning & Lab & 932\\
Robert Pollock & Grn & 579\\
Kyle Thornton & C & 379\\
Will Millinship & LD & 125\\
Cailean Mongan & UKIP & 65\\
Ian Leech & TUSC & 62\\
\end{tabular*}

\noindent
\begin{tabular*}{\columnwidth}{@{\extracolsep{\fill}} p{0.545\columnwidth} >{\itshape}l r @{\extracolsep{\fill}}}
\emph{Leech eliminated}\\
Anna Richardson & SNP & 2143\\
Eileen Dinning & Lab & 945\\
Robert Pollock & Grn & 598\\
Kyle Thornton & C & 379\\
Will Millinship & LD & 125\\
Cailean Mongan & UKIP & 66\\
\end{tabular*}

\council{North Lanarkshire}

\subsubsection*{Thorniewood \hspace*{\fill}\nolinebreak[1]%
\enspace\hspace*{\fill}
\finalhyphendemerits=0
[9th July]}

\index{Thorniewood , North Lanarkshire@Thorniewood, \emph{N. Lanarks.}}

Resignation of Duncan McShannon (SNP).

\noindent
\begin{tabular*}{\columnwidth}{@{\extracolsep{\fill}} p{0.545\columnwidth} >{\itshape}l r @{\extracolsep{\fill}}}
\emph{First preferences}\\
Steven Bonnar & SNP & 1555\\
Hugh Gaffney & Lab & 1410\\
Meghan Gallacher & C & 149\\
Liam McCabe & SSP & 81\\
Patrick McAleer & Grn & 51\\
Craig Smith & Chr & 33\\
Matt Williams & UKIP & 29\\
\end{tabular*}

\emph{Williams eliminated}: Bonnar 1556 Gaffney 1417 Gallacher 158 McCabe 81 McAleer 55 Smith 37

\emph{Smith eliminated}: Bonnar 1565 Gaffney 1422 Gallacher 167 McCabe 82 McAleer 61

\emph{McCabe and McAleer eliminated}: Bonnar 1622 Gaffney 1456 Gallacher 175

\noindent
\begin{tabular*}{\columnwidth}{@{\extracolsep{\fill}} p{0.545\columnwidth} >{\itshape}l r @{\extracolsep{\fill}}}
\emph{Gallacher eliminated}\\
Steven Bonnar & SNP & 1647\\
Hugh Gaffney & Lab & 1517\\
\end{tabular*}

\subsubsection*{Wishaw \hspace*{\fill}\nolinebreak[1]%
\enspace\hspace*{\fill}
\finalhyphendemerits=0
[13th August]}

\index{Wishaw , North Lanarkshire@Wishaw, \emph{N. Lanarks.}}

Resignation of Marion Fellows (SNP).

\noindent
\begin{tabular*}{\columnwidth}{@{\extracolsep{\fill}} p{0.545\columnwidth} >{\itshape}l r @{\extracolsep{\fill}}}
Rosa Zambonini & SNP & 1915\\
Peter McDade & Lab & 1230\\
Marjory Borthwick & C & 385\\
Maria Feeney & SSP & 117\\
Neil Wilson & UKIP & 67\\
Gerard Neary & LD & 37\\
\end{tabular*}

\council{South Lanarkshire}

\subsubsection*{Hamilton South \hspace*{\fill}\nolinebreak[1]%
\enspace\hspace*{\fill}
\finalhyphendemerits=0
[6th August]}

\index{Hamilton South , South Lanarkshire@Hamilton S., \emph{S. Lanarks.}}

Resignation of Angela Crawley (SNP).

\noindent
\begin{tabular*}{\columnwidth}{@{\extracolsep{\fill}} p{0.545\columnwidth} >{\itshape}l r @{\extracolsep{\fill}}}
\emph{First preferences}\\
John Ross & SNP & 1881\\
Jim Lee & Lab & 1396\\
Lynne Nailon & C & 349\\
John Kane & Grn & 127\\
Craig Smith & Chr & 77\\
Donald MacKay & UKIP & 43\\
Matthew Cockburn & LD & 32\\
Andrew McCallum & Pirate & 13\\
\end{tabular*}

\emph{McCallum eliminated}: Ross 1883 Lee 1398 Nailon 349 Kane 128 Smith 80 MacKay 44 Cockburn 34

\emph{MacKay and Cockburn eliminated}: Ross 1895 Lee 1410 Nailon 365 Kane 135 Smith 84

\emph{Smith eliminated}: Ross 1909 Lee 1425 Nailon 389 Kane 146

\noindent
\begin{tabular*}{\columnwidth}{@{\extracolsep{\fill}} p{0.545\columnwidth} >{\itshape}l r @{\extracolsep{\fill}}}
\emph{Kane eliminated}\\
John Ross & SNP & 1978\\
Jim Lee & Lab & 1460\\
Lynne Nailon & C & 398\\
\end{tabular*}

\subsubsection*{Blantyre \hspace*{\fill}\nolinebreak[1]%
\enspace\hspace*{\fill}
\finalhyphendemerits=0
[10th December]}

\index{Blantyre , South Lanarkshire@Blantyre, \emph{S. Lanarks.}}

Death of Jim Handibode (Lab).

\noindent
\begin{tabular*}{\columnwidth}{@{\extracolsep{\fill}} p{0.545\columnwidth} >{\itshape}l r @{\extracolsep{\fill}}}
\emph{First preferences}\\
Mo Razzaq & Lab & 1476\\
Gerry Chambers & SNP & 1236\\
Taylor Muir & C & 140\\
Sean Baillie & SSP & 122\\
Stephen Reid & LD & 92\\
Emma Docherty & UKIP & 59\\
\end{tabular*}

\sloppyword{\emph{Docherty eliminated}: Razzaq 1483 Chambers 1246 Muir 156 Baillie 125 Reid 97}

\emph{Reid eliminated}: Razzaq 1504 Chambers 1259 Muir 172 Baillie 133

\noindent
\begin{tabular*}{\columnwidth}{@{\extracolsep{\fill}} p{0.545\columnwidth} >{\itshape}l r @{\extracolsep{\fill}}}
\emph{Baillie eliminated}\\
Mo Razzaq & Lab & 1536\\
Gerry Chambers & SNP & 1314\\
Taylor Muir & C & 173\\
\end{tabular*}

\section{Forth Councils}

\council{Edinburgh}

LftUnity = Left Unity

Libtn = Libertarian Party

\subsubsection*{Leith Walk (2) \hspace*{\fill}\nolinebreak[1]%
\enspace\hspace*{\fill}
\finalhyphendemerits=0
[10th September; 1 Lab gain from Grn]}

\index{Leith Walk , Edinburgh@Leith Walk, \emph{Edinburgh}}

Resignations of Deirdre Brock (SNP) and Maggie Chapman (Grn).

\noindent
\begin{tabular*}{\columnwidth}{@{\extracolsep{\fill}} p{0.49\columnwidth} >{\itshape}l r @{\extracolsep{\fill}}}
\emph{First preferences}\\
John Ritchie & SNP & 2290\\
Marion Donaldson & Lab & 1623\\
Susan Rae & Grn & 1381\\
Gordon Murdie & C & 501\\
Mo Hussain & LD & 255\\
Alan Melville & UKIP & 102\\
Natalie Reid & SSP & 97\\
Bruce Whitehead & LftUnity & 32\\
John Scott & Ind & 26\\
Tom Laird & Libtn & 17\\
\end{tabular*}

\emph{Quota = 2109, Ritchie elected.  Ritchie's surplus transferred:} Donaldson 1650 Rae 1464 Murdie 503 Hussain 261 Reid 107 Melville 105 Whitehead 33 Scott 29 Laird 17

\emph{Whitehead, Scott and Laird eliminated}: Donaldson 1661 Rae 1495 Murdie 507 Hussain 265 Reid 122 Melville 106

\emph{Hussain, Reid and Melville eliminated}: Donaldson 1798 Rae 1653 Murdie 574

\noindent
\begin{tabular*}{\columnwidth}{@{\extracolsep{\fill}} p{0.545\columnwidth} >{\itshape}l r @{\extracolsep{\fill}}}
\emph{Murdie eliminated}\\
John Ritchie & SNP & 2109\\
Marion Donaldson & Lab & 1990\\
Susan Rae & Grn & 1729\\
\end{tabular*}

\council{Falkirk}

\subsubsection*{Denny and Banknock \hspace*{\fill}\nolinebreak[1]%
\enspace\hspace*{\fill}
\finalhyphendemerits=0
[13th August]}

\index{Denny and Banknock , Falkirk@Denny \& Banknock, \emph{Falkirk}}

Resignation of John McNally (SNP).

\noindent
\begin{tabular*}{\columnwidth}{@{\extracolsep{\fill}} p{0.545\columnwidth} >{\itshape}l r @{\extracolsep{\fill}}}
Paul Garner & SNP & 2576\\
Andrew Bell & Lab & 549\\
David Grant & C & 431\\
Brian Capaloff & Grn & 170\\
\end{tabular*}

\columnbreak

\council{Fife}

\subsubsection*{Kirkcaldy East \hspace*{\fill}\nolinebreak[1]%
\enspace\hspace*{\fill}
\finalhyphendemerits=0
[22nd January]}

\index{Kirkcaldy East , Fife@Kirkcaldy E., \emph{Fife}}

Resignation of Arthur Morrison (SNP).

\noindent
\begin{tabular*}{\columnwidth}{@{\extracolsep{\fill}} p{0.545\columnwidth} >{\itshape}l r @{\extracolsep{\fill}}}
\emph{First preferences}\\
Marie Penman & SNP & 1460\\
Liz Easton & Lab & 1088\\
Edgar Cook & C & 223\\
Claire Reid & Grn & 126\\
Peter Adams & UKIP & 117\\
Callum Leslie & LD & 40\\
Ronald Hunter & Ind & 19\\
Alastair Macintyre & Ind & 12\\
\end{tabular*}

\emph{Leslie, Hunter and Macintyre eliminated}: Penman 1472 Easton 1097 Cook 231 Reid 138 Adams 123

\emph{Adams eliminated}: Penman 1484 Easton 1120 Cook 266 Reid 150

\noindent
\begin{tabular*}{\columnwidth}{@{\extracolsep{\fill}} p{0.545\columnwidth} >{\itshape}l r @{\extracolsep{\fill}}}
\emph{Reid eliminated}\\
Marie Penman & SNP & 1553\\
Liz Easton & Lab & 1148\\
Edgar Cook & C & 274\\
\end{tabular*}

\subsubsection*{Glenrothes West and Kinglassie \hspace*{\fill}\nolinebreak[1]%
\enspace\hspace*{\fill}
\finalhyphendemerits=0
[26th March; SNP gain from Lab]}

\index{Glenrothes West and Kinglassie , Fife@Glenrothes W. \& Kinglassie, \emph{Fife}}

Death of Betty Campbell (Lab).

\noindent
\begin{tabular*}{\columnwidth}{@{\extracolsep{\fill}} p{0.545\columnwidth} >{\itshape}l r @{\extracolsep{\fill}}}
Craig Walker & SNP & 2539\\
Alan Seath & Lab & 1643\\
John Wheatley & C & 202\\
Martin Green & UKIP & 146\\
Jane Liston & LD & 61\\
\end{tabular*}

\subsubsection*{Dunfermline South \hspace*{\fill}\nolinebreak[1]%
\enspace\hspace*{\fill}
\finalhyphendemerits=0
[7th May; SNP gain from Lab]}

\index{Dunfermline South , Fife@Dunfermline S., \emph{Fife}}

Resignation of Cara Hilton (Lab).

\noindent
\begin{tabular*}{\columnwidth}{@{\extracolsep{\fill}} p{0.545\columnwidth} >{\itshape}l r @{\extracolsep{\fill}}}
Fay Sinclair & SNP & 5899\\
Andrew Verrecchia & Lab & 3185\\
David Ross & C & 1324\\
James Calder & LD & 1041\\
\end{tabular*}

\subsubsection*{Glenrothes West and Kinglassie \hspace*{\fill}\nolinebreak[1]%
\enspace\hspace*{\fill}
\finalhyphendemerits=0
[1st October]}

\index{Glenrothes West and Kinglassie , Fife@Glenrothes W. \& Kinglassie, \emph{Fife}}

Resignation of Peter Grant (SNP).

\noindent
\begin{tabular*}{\columnwidth}{@{\extracolsep{\fill}} p{0.545\columnwidth} >{\itshape}l r @{\extracolsep{\fill}}}
Julie Ford & SNP & 2235\\
Alan Seath & Lab & 1207\\
Jonathan Gray & C & 234\\
Lorna Ross & Grn & 113\\
\end{tabular*}

\subsubsection*{Dunfermline North \hspace*{\fill}\nolinebreak[1]%
\enspace\hspace*{\fill}
\finalhyphendemerits=0
[26th November]}

\index{Dunfermline North , Fife@Dunfermline N., \emph{Fife}}

Resignation of David Mogg (SNP).

\noindent
\begin{tabular*}{\columnwidth}{@{\extracolsep{\fill}} p{0.545\columnwidth} >{\itshape}l r 
@{\extracolsep{\fill}}}
\emph{First preferences}\\
Ian Ferguson & SNP & 1056\\
Joe Long & Lab & 719\\
James Reekie & C & 304\\
James Calder & LD & 230\\
Lewis Campbell & Grn & 63\\
Chloanne Dodds & UKIP & 58\\
\end{tabular*}

\emph{Campbell and Dodds eliminated}: Ferguson 1083 Long 733 Reekie 321 Calder 253

\emph{Calder eliminated}: Ferguson 1122 Long 805 Reekie 389

\noindent
\begin{tabular*}{\columnwidth}{@{\extracolsep{\fill}} p{0.545\columnwidth} >{\itshape}l r 
@{\extracolsep{\fill}}}
\emph{Reekie eliminated}\\
Ian Ferguson & SNP & 1144\\
Joe Long & Lab & 912\\
\end{tabular*}

\subsubsection*{Rosyth \hspace*{\fill}\nolinebreak[1]%
\enspace\hspace*{\fill}
\finalhyphendemerits=0
[26th November]}

\index{Rosyth , Fife@Rosyth, \emph{Fife}}

Resignation of Douglas Chapman (SNP).

\noindent
\begin{tabular*}{\columnwidth}{@{\extracolsep{\fill}} p{0.545\columnwidth} >{\itshape}l r 
@{\extracolsep{\fill}}}
\emph{First preferences}\\
Sharon Wilson & SNP & 1214\\
Vikki Fairweather & Lab & 926\\
David Ross & C & 245\\
Matthew Hall & LD & 97\\
Colin Mitchelson & UKIP & 88\\
Alastair MacIntyre & Ind & 66\\
Cairinne Macdonald & Grn & 51\\
\end{tabular*}

\emph{Macdonald eliminated}: Wilson 1235 Fairweather 939 Ross 246 Hall 102 Mitchelson 90 MacIntyre 68

\emph{MacIntyre eliminated}: Wilson 1241 Fairweather 950 Ross 256 Hall 112 Mitchelson 97

\emph{Mitchelson eliminated}: Wilson 1249 Fairweather 966 Ross 281 Hall 122

\emph{Hall eliminated}: Wilson 1263 Fairweather 1012 Ross 309

\noindent
\begin{tabular*}{\columnwidth}{@{\extracolsep{\fill}} p{0.545\columnwidth} >{\itshape}l r 
@{\extracolsep{\fill}}}
\emph{Ross eliminated}\\
Sharon Wilson & SNP & 1286\\
Vikki Fairweather & Lab & 1117\\
\end{tabular*}

\council{Midlothian}

\subsubsection*{Midlothian West \hspace*{\fill}\nolinebreak[1]%
\enspace\hspace*{\fill}
\finalhyphendemerits=0
[10th September]}

\index{Midlothian West , Midlothian@Midlothian W., \emph{Midlothian}}

Resignation of Owen Thompson (SNP).

\noindent
\begin{tabular*}{\columnwidth}{@{\extracolsep{\fill}} p{0.545\columnwidth} >{\itshape}l r @{\extracolsep{\fill}}}
\emph{First preferences}\\
Kelly Perry & SNP & 1540\\
Ian Miller & Lab & 945\\
Pauline Winchester & C & 524\\
Daya Feldwick & Grn & 372\\
Jane Davidson & LD & 162\\
David Tedford & Ind & 25\\
\end{tabular*}

\emph{Davidson and Tedford eliminated}: Perry 1558 Miller 977 Winchester 570 Feldwick 420

\noindent
\begin{tabular*}{\columnwidth}{@{\extracolsep{\fill}} p{0.545\columnwidth} >{\itshape}l r @{\extracolsep{\fill}}}
\emph{Feldwick eliminated}\\
Kelly Perry & SNP & 1701\\
Ian Miller & Lab & 1082\\
Pauline Winchester & C & 618\\
\end{tabular*}

\council{Stirling}

\subsubsection*{Stirling East \hspace*{\fill}\nolinebreak[1]%
\enspace\hspace*{\fill}
\finalhyphendemerits=0
[1st October]}

\index{Stirling East , Stirling@Stirling E., \emph{Stirling}}

Resignation of Steven Paterson (SNP).

\noindent
\begin{tabular*}{\columnwidth}{@{\extracolsep{\fill}} p{0.545\columnwidth} >{\itshape}l r @{\extracolsep{\fill}}}
\emph{First preferences}\\
Gerry McLaughlan & SNP & 1311\\
Chris Kane & Lab & 1094\\
Luke Davison & C & 343\\
Alasdair Tollemache & Grn & 152\\
\end{tabular*}

\emph{Tollemache eliminated}: McLaughlan 1367 Kane 1134 Davison 352

\noindent
\begin{tabular*}{\columnwidth}{@{\extracolsep{\fill}} p{0.545\columnwidth} >{\itshape}l r @{\extracolsep{\fill}}}
\emph{Davison eliminated}\\
Gerry McLaughlan & SNP & 1388\\
Chris Kane & Lab & 1272\\
\end{tabular*}

\columnbreak

\council{West Lothian}

\subsubsection*{Armadale and Blackridge \hspace*{\fill}\nolinebreak[1]%
\enspace\hspace*{\fill}
\finalhyphendemerits=0
[26th March]}

\index{Armadale and Blackridge , West Lothian@Armadale \& Blackridge, \emph{W. Lothian}}

Disqualification (non-attendance) of Isabel Hutton (SNP).

\noindent
\begin{tabular*}{\columnwidth}{@{\extracolsep{\fill}} p{0.545\columnwidth} >{\itshape}l r @{\extracolsep{\fill}}}
\emph{First preferences}\\
Sarah King & SNP & 1620\\
Andrew McGuire & Lab & 1009\\
Scott Mackay & Ind & 756\\
Ian Burgess & C & 255\\
Jenny Johnson & Grn & 90\\
\end{tabular*}

\emph{Burgess and Johnson eliminated}: King 1676 McGuire 1074 Mackay 868

\noindent
\begin{tabular*}{\columnwidth}{@{\extracolsep{\fill}} p{0.545\columnwidth} >{\itshape}l r @{\extracolsep{\fill}}}
\emph{Mackay eliminated}\\
Sarah King & SNP & 1874\\
Andrew McGuire & Lab & 1382\\
\end{tabular*}

\subsubsection*{Linlithgow \hspace*{\fill}\nolinebreak[1]%
\enspace\hspace*{\fill}
\finalhyphendemerits=0
[1st October]}

\index{Linlithgow , West Lothian@Linlithgow, \emph{W. Lothian}}

Resignation of Martin Day (SNP).

\noindent
\begin{tabular*}{\columnwidth}{@{\extracolsep{\fill}} p{0.545\columnwidth} >{\itshape}l r @{\extracolsep{\fill}}}
\emph{First preferences}\\
David Tait & SNP & 2049\\
David Manion & Lab & 1088\\
Ian Burgess & C & 973\\
Maire McCormack & Grn & 282\\
Brenda Galloway & Ind & 230\\
Caron Lindsay & LD & 133\\
\end{tabular*}

\emph{Lindsey eliminated}: Tait 2061 Manion 1131 Burgess 992 McCormack 303 Galloway 246

\emph{McCormack and Galloway eliminated}: Tait 2234 Manion 1279 Burgess 1070

\noindent
\begin{tabular*}{\columnwidth}{@{\extracolsep{\fill}} p{0.545\columnwidth} >{\itshape}l r @{\extracolsep{\fill}}}
\emph{Burgess eliminated}\\
David Tait & SNP & 2325\\
David Manion & Lab & 1644\\
\end{tabular*}

\section{Highland Councils}

\council{Highland}

\subsubsection*{Nairn \hspace*{\fill}\nolinebreak[1]%
\enspace\hspace*{\fill}
\finalhyphendemerits=0
[7th May]}

\index{Nairn , Highland@Nairn, \emph{Highland}}

Resignation of Colin Macaulay (SNP).

\noindent
\begin{tabular*}{\columnwidth}{@{\extracolsep{\fill}} p{0.545\columnwidth} >{\itshape}l r @{\extracolsep{\fill}}}
\emph{First preferences}\\
Stephen Fuller & SNP & 2709\\
Ritchie Cunningham & LD & 1589\\
Paul McIvor & Ind & 1011\\
Mairi MacGregor & Ind & 837\\
Chris Johnson & Lab & 455\\
\end{tabular*}

\emph{Johnson eliminated}: Fuller 2782 Cunningham 1701 McIvor 1042 MacGregor 893

\emph{MacGregor eliminated}: Fuller 2854 Cunningham 1890 McIvor 1440

\noindent
\begin{tabular*}{\columnwidth}{@{\extracolsep{\fill}} p{0.545\columnwidth} >{\itshape}l r @{\extracolsep{\fill}}}
\emph{McIvor eliminated}\\
Stephen Fuller & SNP & 3135\\
Ritchie Cunningham & LD & 2406\\
\end{tabular*}

\subsubsection*{Aird and Loch Ness \hspace*{\fill}\nolinebreak[1]%
\enspace\hspace*{\fill}
\finalhyphendemerits=0
[8th October; LD gain from SNP]}

\index{Aird and Loch Ness , Highland@Aird \& Loch Ness, \emph{Highland}}

Resignation of Drew Hendry (SNP).

\noindent
\begin{tabular*}{\columnwidth}{@{\extracolsep{\fill}} p{0.545\columnwidth} >{\itshape}l r @{\extracolsep{\fill}}}
\emph{First preferences}\\
Jean Davis & LD & 1029\\
Emma Knox & SNP & 1000\\
George Cruickshank & C & 467\\
Zofia Fraser & Ind & 293\\
Vikki Trelfer & Grn & 287\\
\end{tabular*}

\emph{Trelfer eliminated}: Davis 1099 Knox 1097 Cruickshank 480 Fraser 330

\noindent
\begin{tabular*}{\columnwidth}{@{\extracolsep{\fill}} p{0.545\columnwidth} >{\itshape}l r @{\extracolsep{\fill}}}
\multicolumn{3}{@{\extracolsep{\fill}}l}{\emph{Cruickshank and Fraser eliminated}}\\
Jean Davis & LD & 1511\\
Emma Knox & SNP & 1167\\
\end{tabular*}

\council{Moray}

\subsubsection*{Buckie \hspace*{\fill}\nolinebreak[1]%
\enspace\hspace*{\fill}
\finalhyphendemerits=0
[26th March; SNP gain from Ind]}

\index{Buckie , Moray@Buckie, \emph{Moray}}

Death of Joe Mackay (Ind).

\noindent
\begin{tabular*}{\columnwidth}{@{\extracolsep{\fill}} p{0.545\columnwidth} >{\itshape}l r @{\extracolsep{\fill}}}
Sonya Warren & SNP & 1485\\
Norman Calder & Ind & 696\\
Tim Eagle & C & 315\\
\end{tabular*}

\subsubsection*{Heldon and Laich \hspace*{\fill}\nolinebreak[1]%
\enspace\hspace*{\fill}
\finalhyphendemerits=0
[1st October]}

\index{Heldon and Laich , Moray@Heldon \& Laich, \emph{Moray}}

Resignation of Eric McGillivray (Ind).

\noindent
\begin{tabular*}{\columnwidth}{@{\extracolsep{\fill}} p{0.6\columnwidth} >{\itshape}l r @{\extracolsep{\fill}}}
\emph{First preferences}\\
Dennis Slater & Ind & 1323\\
Joyce O'Hara & SNP & 1003\\
Pete Bloomfield & C & 703\\
James MacKessack-Leitch & Grn & 192\\
\end{tabular*}

\noindent
\begin{tabular*}{\columnwidth}{@{\extracolsep{\fill}} p{0.545\columnwidth} >{\itshape}l r @{\extracolsep{\fill}}}
\multicolumn{3}{@{\extracolsep{\fill}}l}{\emph{Two candidates eliminated}}\\
Dennis Slater & Ind & 1775\\
Joyce O'Hara & SNP & 1100\\
\end{tabular*}

\section{Island Councils}

\council{Eilean Siar}

\subsubsection*{An Taobh Siar agus Nis \hspace*{\fill}\nolinebreak[1]%
\enspace\hspace*{\fill}
\finalhyphendemerits=0
[12th March]}

\index{An Taobh Siar agus Nis , Eilean Siar@An Taobh Siar \& Nis, \emph{Eilean Siar}}

Death of Kenny Murray (Ind).

\noindent
\begin{tabular*}{\columnwidth}{@{\extracolsep{\fill}} p{0.545\columnwidth} >{\itshape}l r @{\extracolsep{\fill}}}
Alistair Maclennan & Ind & \emph{unop.}\\
\end{tabular*}

\subsubsection*{Beinn na Foghla agus Uibhist a Tuath \hspace*{\fill}\nolinebreak[1]%
\enspace\hspace*{\fill}
\finalhyphendemerits=0
[26th March; Ind gain from Lab]}

\index{Beinn na Foghla agus Uibhist a Tuath , Eilean Siar@Beinn na Foghla \& Uibhist a T., \emph{Eilean Siar}}

Resignation of Archie Campbell (Lab).

\noindent
\begin{tabular*}{\columnwidth}{@{\extracolsep{\fill}} p{0.545\columnwidth} >{\itshape}l r @{\extracolsep{\fill}}}
Andrew Walker & Ind & 437\\
Roslyn Macpherson & SNP & 302\\
\end{tabular*}

\subsubsection*{An Taobh Siar agus Nis \hspace*{\fill}\nolinebreak[1]%
\enspace\hspace*{\fill}
\finalhyphendemerits=0
[Wednesday 7th October]}

\index{An Taobh Siar agus Nis , Eilean Siar@An Taobh Siar \& Nis, \emph{Eilean Siar}}

Death of Iain Morrison (Ind).

\noindent
\begin{tabular*}{\columnwidth}{@{\extracolsep{\fill}} p{0.545\columnwidth} >{\itshape}l r @{\extracolsep{\fill}}}
John Macleod & Ind & 886\\
Richard Froggatt & Ind & 75\\
Gavin Humphreys & Grn & 59\\
\end{tabular*}

\clearpage

\council{Orkney}

OMG = Orkney Manifesto Group

\subsubsection*{West Mainland \hspace*{\fill}\nolinebreak[1]%
\enspace\hspace*{\fill}
\finalhyphendemerits=0
[Tuesday 18th August]}

\index{West Mainland , Orkney@West Mainland, \emph{Orkney}}

Death of Alistair Gordon (Ind).

\noindent
\begin{tabular*}{\columnwidth}{@{\extracolsep{\fill}} p{0.545\columnwidth} >{\itshape}l r @{\extracolsep{\fill}}}
Rachael King & OMG & 593\\
Barbara Foulkes & Ind & 446\\
Fiona Grahame & Grn & 115\\
\end{tabular*}

\section{Tay Councils}

\council{Perth and Kinross}

\subsubsection*{Perth City Centre \hspace*{\fill}\nolinebreak[1]%
\enspace\hspace*{\fill}
\finalhyphendemerits=0
[7th May]}

\index{Perth City Centre , Perth and Kinross@Perth City Centre, \emph{Perth \& Kinross}}

Resignation of Jack Coburn (SNP).

\noindent
\begin{tabular*}{\columnwidth}{@{\extracolsep{\fill}} p{0.545\columnwidth} >{\itshape}l r @{\extracolsep{\fill}}}
Andrew Parrott & SNP & 3589\\
Chris Ahern & C & 1679\\
Lorna Redford & Lab & 939\\
Philip Brown & LD & 701\\
Ian Thomson & Ind & 119\\
\end{tabular*}

\end{resultsiii}

\part{2016}
\renewcommand\resultsyear{2016}

%\part{Referendums}

\chapter{Referendums in 2016}

\section{Bath and North East Somerset mayoral referendum}
\index{Bath and North East Somerset!Mayoral Referendum}

A referendum was held in Bath and North East Somerset on 10th March on the question of whether the district should have a directly elected mayor.

\noindent
\begin{tabular*}{\columnwidth}{@{\extracolsep{\fill}} p{0.545\columnwidth} >{\itshape}l r @{\extracolsep{\fill}}}
& Yes & 8054\\
& No & 30557\\
\end{tabular*}

\section{North Tyneside mayoral abolition referendum}
\index{North Tyneside!Mayoral Retention Referendum}

A referendum was held in North Tyneside on 5th May on the question of whether the borough should continue to have a directly elected mayor.

\noindent
\begin{tabular*}{\columnwidth}{@{\extracolsep{\fill}} p{0.545\columnwidth} >{\itshape}l r @{\extracolsep{\fill}}}
& Yes & 32546\\
& No & 23703\\
\end{tabular*}

\section{Torbay mayoral abolition referendum}
\index{Torbay!Mayoral Retention Referendum}

A referendum was held in Torbay on 5th May on the question of whether the borough should continue to have a directly elected mayor.

\noindent
\begin{tabular*}{\columnwidth}{@{\extracolsep{\fill}} p{0.545\columnwidth} >{\itshape}l r @{\extracolsep{\fill}}}
& Yes & 9511\\
& No & 15846\\
\end{tabular*}

\section{West Dorset committee system referendum}
\index{West Dorset!Committee System Referendum}

A referendum was held in West Dorset on 5th May on the question of whether the district should move from the leader and cabinet executive to a committee system.

\noindent
\begin{tabular*}{\columnwidth}{@{\extracolsep{\fill}} p{0.545\columnwidth} >{\itshape}l r @{\extracolsep{\fill}}}
& Yes & 16534\\
& No & 8811\\
\end{tabular*}

\section{United Kingdom EU membership referendum}
\index{Referendums!EU Membership}

A nationwide referendum was held on 23rd June on the question of the UK's EU membership.

\emph{Should the United Kingdom remain a member of the European Union or leave the European Union?}

\noindent
\begin{tabular*}{\columnwidth}{@{\extracolsep{\fill}} p{0.545\columnwidth} >{\itshape}l r @{\extracolsep{\fill}}}
& Remain & 16\,141\,241\\
& Leave & 17\,410\,742\\
\end{tabular*}

\section{Guildford mayoral referendum}
\index{Guildford!Mayoral Referendum}

A referendum was held in Guildford on 13th October on the question of whether the district should have a directly elected mayor.

\noindent
\begin{tabular*}{\columnwidth}{@{\extracolsep{\fill}} p{0.545\columnwidth} >{\itshape}l r @{\extracolsep{\fill}}}
& Yes & 4948\\
& No & 20639\\
\end{tabular*}

%\part{By-elections}

\chapter{Parliamentary by-elections}

There were four parliamentary by-elections in 2016.

Ecc = Eccentric Party of Great Britain

Elmo = Give Me Back Elmo

Elvis = Bus-Pass Elvis Party

EngInd = English Independence Party

Immig = Immigrants Political Party

LibtyGB = Liberty GB

LincsInd = Lincolnshire Independents

NHAction = National Health Action Party

OneLove = One Love Party

Yorks1st = Yorkshire First

\vfill

\section*{Ogmore \hspace*{\fill}\nolinebreak[1]%
\enspace\hspace*{\fill}
\finalhyphendemerits=0
[5th May]}

\index{Ogmore , House of Commons@Ogmore, \emph{House of Commons}}

Resignation of Huw Irranca-Davies (Lab).

\noindent
\begin{tabular*}{\columnwidth}{@{\extracolsep{\fill}} p{0.53\columnwidth} >{\itshape}l r @{\extracolsep{\fill}}}
Christopher Elmore & Lab & 12383\\
Glenda Davies & UKIP & 3808\\
Abi Thomas & PC & 3683\\
Alex Williams & C & 2956\\
Janet Ellard & LD & 702\\
\end{tabular*}

\vfill

\section*{Sheffield Brightside and Hillsborough \hspace*{\fill}\nolinebreak[1]%
\enspace\hspace*{\fill}
\finalhyphendemerits=0
[5th May]}

\index{Sheffield Brightside and Hillsborough , House of Commons@\sloppyword{Shef{fi}eld Brightside \& Hillsborough, \emph{House of Commons}}}

Death of Harry Harpham (Lab).

\noindent
\begin{tabular*}{\columnwidth}{@{\extracolsep{\fill}} p{0.53\columnwidth} >{\itshape}l r @{\extracolsep{\fill}}}
Gill Furniss & Lab & 14087\\
Steven Winstone & UKIP & 4497\\
Shaffaq Mohammed & LD & 1385\\
Spencer Pitfeld & C & 1267\\
Christine Gilligan Kubo & Grn & 938\\
Stevie Manion & Yorks1st & 349\\
Bobby Smith & Elmo & 58\\
\end{tabular*}

\eject

\section*{Tooting \hspace*{\fill}\nolinebreak[1]%
\enspace\hspace*{\fill}
\finalhyphendemerits=0
[16th June]}

\index{Tooting , House of Commons@Tooting, \emph{House of Commons}}

Resignation of Sadiq Khan (Lab).

\noindent
\begin{tabular*}{\columnwidth}{@{\extracolsep{\fill}} p{0.53\columnwidth} >{\itshape}l r @{\extracolsep{\fill}}}
Rosena Allin-Khan & Lab & 17894\\
Dan Watkins & C & 11537\\
Esther Obiri-Darko & Grn & 830\\
Alexander Glassbrook & LD & 820\\
Elizabeth Jones & UKIP & 507\\
Des Coke & CPA & 164\\
Howling Laud Hope & Loony & 54\\
Graham Moore & EDP & 50\\
Akbar Malik & Immig & 44\\
Ankit Love & OneLove & 32\\
Zirwa Javaid & Ind & 30\\
Zia Samadani & Ind & 23\\
Bobby Smith & Elmo & 9\\
Smiley Smillie & Ind & 5\\
\end{tabular*}

\vfill

\section*{Batley and Spen \hspace*{\fill}\nolinebreak[1]%
\enspace\hspace*{\fill}
\finalhyphendemerits=0
[20th October]}

\index{Batley and Spen , House of Commons@Batley \& Spen, \emph{House of Commons}}

Death of Jo Cox (Lab).

\noindent
\begin{tabular*}{\columnwidth}{@{\extracolsep{\fill}} p{0.53\columnwidth} >{\itshape}l r @{\extracolsep{\fill}}}
Tracy Brabin & Lab & 17506\\
Therese Hirst & EDP & 969\\
David Furness & BNP & 548\\
Garry Kitchin & Ind & 517\\
Corbyn Anti & EngInd & 241\\
Jack Buckby & LibtyGB & 220\\
Henry Mayhew & Ind & 153\\
Waqas Khan & Ind & 118\\
Richard Edmonds & NF & 87\\
Ankit Love & OneLove & 34\\
\end{tabular*}

\vfill

\section*{Witney \hspace*{\fill}\nolinebreak[1]%
\enspace\hspace*{\fill}
\finalhyphendemerits=0
[20th October]}

\index{Witney , House of Commons@Witney, \emph{House of Commons}}

Resignation of David Cameron (C).

\noindent
\begin{tabular*}{\columnwidth}{@{\extracolsep{\fill}} p{0.53\columnwidth} >{\itshape}l r @{\extracolsep{\fill}}}
Robert Courts & C & 17313\\
Elizabeth Leffman & LD & 11611\\
Duncan Enright & Lab & 5765\\
Larry Sanders & Grn & 1363\\
Dickie Bird & UKIP & 1354\\
Helen Salisbury & NHAction & 433\\
Daniel Skidmore & Ind & 151\\
Mad Hatter & Loony & 129\\
Nicholas Ward & Ind & 93\\
David Bishop & Elvis & 61\\
Lord Toby Jug & Ecc & 59\\
Winston McKenzie & EDP & 52\\
Emilia Arno & OneLove & 44\\
Adam Knight & Ind & 27\\
\end{tabular*}

\eject

\section*{Richmond Park \hspace*{\fill}\nolinebreak[1]%
\enspace\hspace*{\fill}
\finalhyphendemerits=0
[1st December; LD gain from C]}

\index{Richmond Park , House of Commons@Richmond Park, \emph{House of Commons}}

Resignation of Zac Goldsmith (C).

\noindent
\begin{tabular*}{\columnwidth}{@{\extracolsep{\fill}} p{0.53\columnwidth} >{\itshape}l r @{\extracolsep{\fill}}}
Sarah Olney & LD & 20510\\
Zac Goldsmith & Ind & 18638\\
Christian Wolmar & Lab & 1515\\
Howling Laud Hope & Loony & 184\\
Fiona Syms & Ind & 173\\
Dominic Stockford & CPA & 164\\
Maharaja Jammu and Kashmir & OneLove & 67\\
David Powell & Ind & 32\\
\end{tabular*}

\section*{Sleaford and North Hykeham \hspace*{\fill}\nolinebreak[1]%
\enspace\hspace*{\fill}
\finalhyphendemerits=0
[8th December]}

\index{Sleaford and North Hykeham , House of Commons@Sleaford \& N. Hykeham, \emph{House of Commons}}

Resignation of Stephen Phillips (C).

\noindent
\begin{tabular*}{\columnwidth}{@{\extracolsep{\fill}} p{0.53\columnwidth} >{\itshape}l r @{\extracolsep{\fill}}}
Caroline Johnson & C & 17570\\
Victoria Ayling & UKIP & 4426\\
Jim Clarke & Lab & 3363\\
Ross Pepper & LD & 3006\\
Marianne Overton & LincsInd & 2892\\
Sarah Stock & Ind & 462\\
The Iconic Arty-Pole & Loony & 200\\
Paul Coyne & Ind & 186\\
Mark Suffield & Ind & 74\\
David Bishop & Elvis & 55\\
\end{tabular*}

\chapter{By-elections to devolved assemblies, the European Parliament, and police and crime commissionerships}

\section{Greater London Authority}

There were no by-elections in 2016 to the Greater London Authority.

\section{National Assembly for Wales}

There were no by-elections in 2016 to the National Assembly for Wales.

\section{Scottish Parliament}

There were no by-elections in 2016 to the Scottish Parliament.

Margo MacDonald (Ind, Lothian) died on 4th April 2014.  Her seat was vacant at the 2016 ordinary election.

Richard Baker (Lab, North East Scotland) resigned on 11th January and was replaced from the list by Lesley Brennan.

Alex Johnstone (C, North East Scotland) died on 7th December and was replaced from the list by Bill Bowman.

\section{Northern Ireland Assembly}

Vacancies in the Northern Ireland Assembly are filled by co-option.
%No co-options were made in 2016.
%
The following members were co-opted to the Assembly in 2016:
\begin{itemize}
\item Philip McGuigan (SF) replaced Daith\'i McKay following his resignation on 19th August (North Antrim).
\item \'Orlaith\'i Flynn (SF) replaced Jennifer McCann following her resignation on 30th November (Belfast West).
\end{itemize}

\section{European Parliament}

UK vacancies in the European Parliament are filled by the next available person from the party list at the most recent election (which was held in 2014).
No replacements were made in 2016.
%The following replacement was made in 2010:
%\begin{itemize}
%\item Keith Taylor (Grn) replaced Caroline Lucas following her resignation on 17th May (South East).
%\end{itemize}

\section{Police and crime commissioners}

There were no by-elections in 2016 for vacant police and crime commissioner posts.

\chapter{Local by-elections and unfilled vacancies}

\begin{resultsiii}

\section{North London}

\council{City of London}

\subsubsection*{Tower \hspace*{\fill}\nolinebreak[1]%
\enspace\hspace*{\fill}
\finalhyphendemerits=0
[Wednesday 16th March]}

\index{Tower , City of London@Tower, \emph{City of London}}

Resignation of Gerald Pulman (Ind).

\noindent
\begin{tabular*}{\columnwidth}{@{\extracolsep{\fill}} p{0.53\columnwidth} >{\itshape}l r @{\extracolsep{\fill}}}
Anne Fairweather & Ind & 172\\
Lady Xuelin Li Bates & Ind & 46\\
\end{tabular*}

\subsubsection*{Lime Street \hspace*{\fill}\nolinebreak[1]%
\enspace\hspace*{\fill}
\finalhyphendemerits=0
[17th March]}

\index{Lime Street , City of London@Lime St., \emph{City of London}}

Resignation of Robert Howard (Ind).

\noindent
\begin{tabular*}{\columnwidth}{@{\extracolsep{\fill}} p{0.53\columnwidth} >{\itshape}l r @{\extracolsep{\fill}}}
Dominic Christian & Ind & \emph{unop.}\\
\end{tabular*}

\subsubsection*{Cordwainer \hspace*{\fill}\nolinebreak[1]%
\enspace\hspace*{\fill}
\finalhyphendemerits=0
[3rd November]}

\index{Cordwainer , City of London@Cordwainer, \emph{City of London}}

Aldermanic election: Sir Roger Gifford (Ind) sought re-election.

\noindent
\begin{tabular*}{\columnwidth}{@{\extracolsep{\fill}} p{0.53\columnwidth} >{\itshape}l r @{\extracolsep{\fill}}}
Roger Gifford & Ind & \emph{unop.}\\
\end{tabular*}

\subsubsection*{Queenhithe \hspace*{\fill}\nolinebreak[1]%
\enspace\hspace*{\fill}
\finalhyphendemerits=0
[Monday 21st November]}

\index{Queenhithe , City of London@Queenhithe, \emph{City of London}}

Aldermanic election: resignation of Gordon Haines (Ind).

\noindent
\begin{tabular*}{\columnwidth}{@{\extracolsep{\fill}} p{0.53\columnwidth} >{\itshape}l r @{\extracolsep{\fill}}}
Alistair King & Ind & 122\\
Patrick Streeter & Ind & 3\\
\end{tabular*}

\subsubsection*{Walbrook \hspace*{\fill}\nolinebreak[1]%
\enspace\hspace*{\fill}
\finalhyphendemerits=0
[1st December]}

\index{Walbrook , City of London@Walbrook, \emph{City of London}}

Resignation of Lucy Frew (Ind).

\noindent
\begin{tabular*}{\columnwidth}{@{\extracolsep{\fill}} p{0.53\columnwidth} >{\itshape}l r @{\extracolsep{\fill}}}
Peter Bennett & Ind & 72\\
Sophia Morrell & Ind & 46\\
Xuelin Bates & Ind & 36\\
\end{tabular*}

\subsection*{Barnet}
\index{Barnet}

\subsubsection*{Underhill \hspace*{\fill}\nolinebreak[1]%
\enspace\hspace*{\fill}
\finalhyphendemerits=0
[5th May]}

\index{Underhill , Barnet@Underhill, \emph{Barnet}}

Resignation of Amy Trevethan (Lab).

\noindent
\begin{tabular*}{\columnwidth}{@{\extracolsep{\fill}} p{0.53\columnwidth} >{\itshape}l r @{\extracolsep{\fill}}}
Jess Brayne & Lab & 2314\\
Lesley Evans & C & 1979\\
Barry Ryan & UKIP & 459\\
Duncan Macdonald & LD & 452\\
Phil Fletcher & Grn & 387\\
\end{tabular*}

\subsection*{Brent}
\index{Brent}

\subsubsection*{Kilburn \hspace*{\fill}\nolinebreak[1]%
\enspace\hspace*{\fill}
\finalhyphendemerits=0
[5th May]}

\index{Kilburn , Brent@Kilburn, \emph{Brent}}

Death of Tayo Oladapo (Lab).

\noindent
\begin{tabular*}{\columnwidth}{@{\extracolsep{\fill}} p{0.53\columnwidth} >{\itshape}l r @{\extracolsep{\fill}}}
Barbara Pitruzzela & Lab & 2841\\
Calvin Robinson & C & 802\\
Tilly Boulter & LD & 459\\
Peter Murry & Grn & 452\\
Janice North & UKIP & 232\\
Elcena Jeffers & Ind & 35\\
\end{tabular*}

\subsection*{Hackney}
\index{Hackney}

OneLove = One Love Party

\subsubsection*{Hackney Downs \hspace*{\fill}\nolinebreak[1]%
\enspace\hspace*{\fill}
\finalhyphendemerits=0
[5th May]}

\index{Hackney Downs , Hackney@Hackney Downs, \emph{Hackney}}

Resignation of Rick Muir (Lab).

\noindent
\begin{tabular*}{\columnwidth}{@{\extracolsep{\fill}} p{0.6\columnwidth} >{\itshape}l r @{\extracolsep{\fill}}}
Sem Moema & Lab & 2614\\
Alastair Binnie-Lubbock & Grn & 1067\\
Nicola Benjamin & C & 350\\
Mohammed Sadiq & LD & 338\\
\end{tabular*}

\subsubsection*{Stoke Newington \hspace*{\fill}\nolinebreak[1]%
\enspace\hspace*{\fill}
\finalhyphendemerits=0
[5th May]}

\index{Stoke Newington , Hackney@Stoke Newington, \emph{Hackney}}

Resignation of Louisa Thomson (Lab).

\noindent
\begin{tabular*}{\columnwidth}{@{\extracolsep{\fill}} p{0.53\columnwidth} >{\itshape}l r @{\extracolsep{\fill}}}
Patrick Moule & Lab & 3241\\
Halita Obineche & Grn & 1132\\
Christopher Sills & C & 450\\
Victor de Almeida & LD & 303\\
Mick Cotter & TUSC & 136\\
\end{tabular*}

\subsubsection*{Hackney Central \hspace*{\fill}\nolinebreak[1]%
\enspace\hspace*{\fill}
\finalhyphendemerits=0
[21st July]}

\index{Hackney Central , Hackney@Hackney C., \emph{Hackney}}

Resignation of Sophie Linden (Lab).

\noindent
\begin{tabular*}{\columnwidth}{@{\extracolsep{\fill}} p{0.53\columnwidth} >{\itshape}l r @{\extracolsep{\fill}}}
Sophie Conway & Lab & 1354\\
Siobhan MacMahon & Grn & 178\\
Russell French & LD & 113\\
Christopher Sills & C & 101\\
Mustafa Korel & Ind & 55\\
\end{tabular*}

\subsubsection*{Mayor of Hackney \hspace*{\fill}\nolinebreak[1]%
\enspace\hspace*{\fill}
\finalhyphendemerits=0
[15th September]}

\index{Elected Mayors!Hackney}

Resignation of Jules Pipe (Lab).

\noindent
\begin{tabular*}{\columnwidth}{@{\extracolsep{\fill}} p{0.45\columnwidth} >{\itshape}l r @{\extracolsep{\fill}}}
Philip Glanville & Lab & 22595\\
Samir Jeraj & Grn & 4338\\
Amy Gray & C & 3533\\
Dave Raval & LD & 1818\\
Dawa Ma & OneLove & 494\\
\end{tabular*}

\subsubsection*{Hoxton West \hspace*{\fill}\nolinebreak[1]%
\enspace\hspace*{\fill}
\finalhyphendemerits=0
[3rd November]}

\index{Hoxton West , Hackney@Hoxton W., \emph{Hackney}}

Election of Philip Glanville (Lab) as Mayor of Hackney.

\noindent
\begin{tabular*}{\columnwidth}{@{\extracolsep{\fill}} p{0.53\columnwidth} >{\itshape}l r @{\extracolsep{\fill}}}
Yvonne Maxwell & Lab & 951\\
Christopher Sills & C & 185\\
Chantel Encavey & LD & 133\\
Morgan James & Grn & 123\\
\end{tabular*}

\subsection*{Haringey}
\index{Haringey}

\subsubsection*{Harringay \hspace*{\fill}\nolinebreak[1]%
\enspace\hspace*{\fill}
\finalhyphendemerits=0
[28th July]}

\index{Harringay , Haringey@Harringay, \emph{Haringey}}

Resignation of James Ryan (Lab).

\noindent
\begin{tabular*}{\columnwidth}{@{\extracolsep{\fill}} p{0.53\columnwidth} >{\itshape}l r @{\extracolsep{\fill}}}
Zena Brabazon & Lab & 1054\\
Karen Alexander & LD & 765\\
Jarelle Francis & Grn & 325\\
Cansoy Elmaz & C & 99\\
Neville Watson & UKIP & 36\\
\end{tabular*}

\subsubsection*{St Ann's \hspace*{\fill}\nolinebreak[1]%
\enspace\hspace*{\fill}
\finalhyphendemerits=0
[6th October]}

\index{Saint Ann's , Haringey@St Ann's, \emph{Haringey}}

Resignation of Peter Morton (Lab).

\noindent
\begin{tabular*}{\columnwidth}{@{\extracolsep{\fill}} p{0.53\columnwidth} >{\itshape}l r @{\extracolsep{\fill}}}
Noah Tucker & Lab & 1177\\
Ronald Stewart & Grn & 323\\
Josh Dixon & LD & 189\\
Ellis Turrell & C & 106\\
Janus Polenceusz & UKIP & 54\\
\end{tabular*}

\subsection*{Havering}
\index{Havering}

\subsubsection*{Heaton \hspace*{\fill}\nolinebreak[1]%
\enspace\hspace*{\fill}
\finalhyphendemerits=0
[5th May; Lab gain from UKIP]}

\index{Heaton , Havering@Heaton, \emph{Havering}}

Resignation of Philip Hyde (Ind elected as UKIP).

\noindent
\begin{tabular*}{\columnwidth}{@{\extracolsep{\fill}} p{0.53\columnwidth} >{\itshape}l r @{\extracolsep{\fill}}}
Denis O'Flynn & Lab & 1122\\
Keith Wells & C & 951\\
John Thurtle & UKIP & 864\\
Christopher Cooper & Ind & 515\\
Peter Caton & Grn & 107\\
Jonathan Coles & LD & 86\\
Denise Underwood & BNP & 73\\
Kevin Layzell & NF & 14\\
\end{tabular*}

\subsection*{Hounslow}
\index{Hounslow}

\subsubsection*{Cranford \hspace*{\fill}\nolinebreak[1]%
\enspace\hspace*{\fill}
\finalhyphendemerits=0
[11th February]}

\index{Cranford , Hounslow@Cranford, \emph{Hounslow}}

Death of Sohan Sangha (Lab).

\noindent
\begin{tabular*}{\columnwidth}{@{\extracolsep{\fill}} p{0.53\columnwidth} >{\itshape}l r @{\extracolsep{\fill}}}
Sukhbir Dhaliwal & Lab & 1264\\
Sukhdev Singh Maras & C & 638\\
Hina Malik & LD & 265\\
George Radulski & UKIP & 96\\
Nico Fekete & Grn & 48\\
\end{tabular*}

\subsection*{Islington}
\index{Islington}

\subsubsection*{Barnsbury \hspace*{\fill}\nolinebreak[1]%
\enspace\hspace*{\fill}
\finalhyphendemerits=0
[14th July]}

\index{Barnsbury , Islington@Barnsbury, \emph{Islington}}

Resignation of James Murray (Lab).

\noindent
\begin{tabular*}{\columnwidth}{@{\extracolsep{\fill}} p{0.53\columnwidth} >{\itshape}l r @{\extracolsep{\fill}}}
Rowena Champion & Lab & 1192\\
Bradley Hillier-Smith & LD & 409\\
Edward Waldegrave & C & 367\\
Ernestas Jegorovas & Grn & 302\\
Robert Capper & Ind & 40\\
\end{tabular*}

\council{Kensington and Chelsea}

\subsubsection*{Abingdon (2) \hspace*{\fill}\nolinebreak[1]%
\enspace\hspace*{\fill}
\finalhyphendemerits=0
[5th May]}

\index{Abingdon , Kensington and Chelsea@Abingdon, \emph{Kensington \& Chelsea}}

Resignations of Victoria Borwick MP and Joanna Gardner (both C).

\noindent
\begin{tabular*}{\columnwidth}{@{\extracolsep{\fill}} p{0.53\columnwidth} >{\itshape}l r @{\extracolsep{\fill}}}
Sarah Addenbrooke & C & 1716\\
Anne Cyron & C & 1470\\
Benjamin Fernando & Lab & 395\\
Nigel Wilkins & Lab & 298\\
Jeremy Good & LD & 220\\
Jonathan Owen & LD & 210\\
Richard Braine & UKIP & 85\\
Jack Bovill & UKIP & 76\\
\end{tabular*}

\subsection*{Newham}
\index{Newham}

\subsubsection*{Forest Gate North \hspace*{\fill}\nolinebreak[1]%
\enspace\hspace*{\fill}
\finalhyphendemerits=0
[14th July]}

\index{Forest Gate North , Newham@Forest Gate N., \emph{Newham}}

Resignation of Ellie Robinson (Lab).

\noindent
\begin{tabular*}{\columnwidth}{@{\extracolsep{\fill}} p{0.53\columnwidth} >{\itshape}l r @{\extracolsep{\fill}}}
Anamul Islam & Lab & 1150\\
Elisabeth Whitebread & Grn & 681\\
John Oxley & C & 301\\
James Rumsby & LD & 57\\
\end{tabular*}

\subsection*{Redbridge}
\index{Redbridge}

APP = All People's Party

\subsubsection*{Roding \hspace*{\fill}\nolinebreak[1]%
\enspace\hspace*{\fill}
\finalhyphendemerits=0
[5th May; Lab gain from C]}

\index{Roding , Redbridge@Roding, \emph{Redbridge}}

Resignation of Sarah Blaber (C).

\noindent
\begin{tabular*}{\columnwidth}{@{\extracolsep{\fill}} p{0.53\columnwidth} >{\itshape}l r @{\extracolsep{\fill}}}
Lloyd Duddridge & Lab & 1832\\
Ruth Clark & C & 1254\\
Richard Clare & LD & 983\\
Jonathan Seymour & UKIP & 216\\
Barry Cooper & Grn & 169\\
Marilyn Moore & APP & 22\\
\end{tabular*}

\subsection*{Tower Hamlets}
\index{Tower Hamlets}

\subsubsection*{Whitechapel \hspace*{\fill}\nolinebreak[1]%
\enspace\hspace*{\fill}
\finalhyphendemerits=0
[1st December; Ind gain from Tower Hamlets First]}

\index{Whitechapel , Tower Hamlets@Whitechapel, \emph{Tower Hamlets}}

Disqualification (sentenced to five months' imprisonment, housing fraud) of Shahed Ali (Ind elected as Tower Hamlets First).

\noindent
\begin{tabular*}{\columnwidth}{@{\extracolsep{\fill}} p{0.53\columnwidth} >{\itshape}l r @{\extracolsep{\fill}}}
Shafi Ahmed & Ind & 1147\\
Victoria Obaze & Lab & 823\\
Will Fletcher & C & 217\\
Emanuel Andjelic & LD & 173\\
James Wilson & Grn & 170\\
Martin Smith & UKIP & 34\\
\end{tabular*}

\columnbreak

\subsection*{Westminster}
\index{Westminster}

Pirate = Pirate Party UK

\subsubsection*{Church Street \hspace*{\fill}\nolinebreak[1]%
\enspace\hspace*{\fill}
\finalhyphendemerits=0
[5th May]}

\index{Church Street , Westminster@Church St., \emph{Westminster}}

Resignation of Vincenzo Rampulla (Lab).

\noindent
\begin{tabular*}{\columnwidth}{@{\extracolsep{\fill}} p{0.53\columnwidth} >{\itshape}l r @{\extracolsep{\fill}}}
Aicha Less & Lab & 2174\\
Rachid Boufas & C & 512\\
Alistair Barr & LD & 205\\
Jill de Quincey & UKIP & 175\\
Andreas Haberland & Pirate & 26\\
\end{tabular*}

\section{South London}

APP = All People's Party

\subsection*{Bexley}
\index{Bexley}

\subsubsection*{St Michael's \hspace*{\fill}\nolinebreak[1]%
\enspace\hspace*{\fill}
\finalhyphendemerits=0
[30th June]}

\index{Saint Michael's , Bexley@St Michael's, \emph{Bexley}}

Resignation of Joe Pollard (C).

\noindent
\begin{tabular*}{\columnwidth}{@{\extracolsep{\fill}} p{0.53\columnwidth} >{\itshape}l r @{\extracolsep{\fill}}}
Ray Sams & C & 939\\
Sam Marchant & Lab & 840\\
Keith Forster & UKIP & 456\\
Simone Reynolds & LD & 117\\
Michael Jones & BNP & 105\\
Derek Moran & Grn & 54\\
\end{tabular*}

\subsection*{Croydon}
\index{Croydon}

AIE = An Independence from Europe

\subsubsection*{West Thornton \hspace*{\fill}\nolinebreak[1]%
\enspace\hspace*{\fill}
\finalhyphendemerits=0
[5th May]}

\index{West Thornton , Croydon@West Thornton, \emph{Croydon}}

Resignation of Emily Benn (Lab).

\noindent
\begin{tabular*}{\columnwidth}{@{\extracolsep{\fill}} p{0.53\columnwidth} >{\itshape}l r @{\extracolsep{\fill}}}
Callton Young & Lab & 3136\\
Scott Roche & C & 989\\
David Beall & Grn & 289\\
Ace Nnorom & UKIP & 145\\
Geoff Morley & LD & 140\\
Peter Morgan & AIE & 77\\
Winston McKenzie & EDP & 70\\
\end{tabular*}

\subsection*{Greenwich}
\index{Greenwich}

\subsubsection*{Glyndon \hspace*{\fill}\nolinebreak[1]%
\enspace\hspace*{\fill}
\finalhyphendemerits=0
[5th May]}

\index{Glyndon , Greenwich@Glyndon, \emph{Greenwich}}

Resignation of Radha Rabadia (Lab).

\noindent
\begin{tabular*}{\columnwidth}{@{\extracolsep{\fill}} p{0.53\columnwidth} >{\itshape}l r @{\extracolsep{\fill}}}
Tonia Ashikodi & Lab & 2583\\
Matt Browne & C & 561\\
Daniel Garrun & Grn & 402\\
Rita Hamilton & UKIP & 380\\
Stewart Christie & LD & 376\\
Ebru Ogun & Ind & 157\\
Abiola Olaore & APP & 64\\
\end{tabular*}

\subsubsection*{Eltham North \hspace*{\fill}\nolinebreak[1]%
\enspace\hspace*{\fill}
\finalhyphendemerits=0
[10th November; C gain from Lab]}

\index{Eltham North , Greenwich@Eltham N., \emph{Greenwich}}

Resignation of Wynn Davies (Lab).

\noindent
\begin{tabular*}{\columnwidth}{@{\extracolsep{\fill}} p{0.53\columnwidth} >{\itshape}l r @{\extracolsep{\fill}}}
Charlie Davis & C & 1335\\
Simon Peirce & Lab & 1297\\
Sam Macaulay & LD & 279\\
Barbara Ray & UKIP & 160\\
Matt Browne & Grn & 110\\
\end{tabular*}

\subsection*{Lambeth}
\index{Lambeth}

\subsubsection*{Gipsy Hill \hspace*{\fill}\nolinebreak[1]%
\enspace\hspace*{\fill}
\finalhyphendemerits=0
[9th June]}

\index{Gipsy Hill , Lambeth@Gipsy Hill, \emph{Lambeth}}

Death of Niranjan Francis (Lab).

\noindent
\begin{tabular*}{\columnwidth}{@{\extracolsep{\fill}} p{0.53\columnwidth} >{\itshape}l r @{\extracolsep{\fill}}}
Luke Murphy & Lab & 1220\\
Pete Elliott & Grn & 1184\\
Leslie Maruziva & C & 210\\
Rosa Jesse & LD & 84\\
Elizabeth Jones & UKIP & 73\\
Robin Lambert & Ind & 24\\
Steven Nally & TUSC & 19\\
\end{tabular*}

\subsection*{Lewisham}
\index{Lewisham}

LPBP = Lewisham People Before Profit

WEq = Women's Equality Party

\subsubsection*{Bellingham \hspace*{\fill}\nolinebreak[1]%
\enspace\hspace*{\fill}
\finalhyphendemerits=0
[21st July]}

\index{Bellingham , Lewisham@Bellingham, \emph{Lewisham}}

Resignation of Ami Ibitson (Lab).

\noindent
\begin{tabular*}{\columnwidth}{@{\extracolsep{\fill}} p{0.53\columnwidth} >{\itshape}l r @{\extracolsep{\fill}}}
Sue Hordijenko & Lab & 940\\
Ross Archer & C & 302\\
Ed Veasey & LD & 180\\
David Hamilton & LPBP & 129\\
Edwin Smith & UKIP & 104\\
\end{tabular*}

\subsubsection*{Brockley \hspace*{\fill}\nolinebreak[1]%
\enspace\hspace*{\fill}
\finalhyphendemerits=0
[13th October]}

\index{Brockley , Lewisham@Brockley, \emph{Lewisham}}

Resignation of Baroness Kennedy of Cradley (Lab).

\noindent
\begin{tabular*}{\columnwidth}{@{\extracolsep{\fill}} p{0.53\columnwidth} >{\itshape}l r @{\extracolsep{\fill}}}
Sophie McGeevor & Lab & 1190\\
Clare Phipps & Grn & 631\\
Bobby Dean & LD & 259\\
Andrew Hughes & C & 195\\
Rebecca Manson Jones & WEq & 173\\
Hugh Waine & UKIP & 33\\
\end{tabular*}

\subsubsection*{Evelyn \hspace*{\fill}\nolinebreak[1]%
\enspace\hspace*{\fill}
\finalhyphendemerits=0
[13th October]}

\index{Evelyn , Lewisham@Evelyn, \emph{Lewisham}}

Death of Crada Onuegbu (Lab).

\noindent
\begin{tabular*}{\columnwidth}{@{\extracolsep{\fill}} p{0.53\columnwidth} >{\itshape}l r @{\extracolsep{\fill}}}
Joyce Jacca & Lab & 1028\\
Ray Barron-Woolford & LPBP & 314\\
James Clark & C & 183\\
Scott Barkwith & Ind & 173\\
Andrea Carey-Fuller & Grn & 119\\
Lucy Salek & LD & 107\\
\end{tabular*}

\subsection*{Merton}
\index{Merton}

\subsubsection*{Figge's Marsh \hspace*{\fill}\nolinebreak[1]%
\enspace\hspace*{\fill}
\finalhyphendemerits=0
[5th May]}

\index{Figge's Marsh , Merton@Figge's Marsh, \emph{Merton}}

Resignation of Peter Walker (Lab).

\noindent
\begin{tabular*}{\columnwidth}{@{\extracolsep{\fill}} p{0.53\columnwidth} >{\itshape}l r @{\extracolsep{\fill}}}
Mike Brunt & Lab & 2824\\
Hamna Qureshi & C & 443\\
Andrew Mills & UKIP & 254\\
Penelope Homer & Grn & 184\\
Rachel Waitt & LD & 136\\
\end{tabular*}

\subsubsection*{St Helier \hspace*{\fill}\nolinebreak[1]%
\enspace\hspace*{\fill}
\finalhyphendemerits=0
[19th May]}

\index{Saint Helier , Merton@St Helier, \emph{Merton}}

Death of Maxi Martin (Lab).

\noindent
\begin{tabular*}{\columnwidth}{@{\extracolsep{\fill}} p{0.53\columnwidth} >{\itshape}l r @{\extracolsep{\fill}}}
Jerome Neil & Lab & 1436\\
Susan Edwards & C & 282\\
Richard Hilton & UKIP & 191\\
Asif Ashraf & LD & 59\\
John Barraball & Grn & 55\\
\end{tabular*}

\subsection*{Southwark}
\index{Southwark}

\subsubsection*{Faraday \hspace*{\fill}\nolinebreak[1]%
\enspace\hspace*{\fill}
\finalhyphendemerits=0
[21st January]}

\index{Faraday , Southwark@Faraday, \emph{Southwark}}

Resignation of Dan Garfield (Lab).

\noindent
\begin{tabular*}{\columnwidth}{@{\extracolsep{\fill}} p{0.545\columnwidth} >{\itshape}l r @{\extracolsep{\fill}}}
Samantha Jury-Dada & Lab & 1072\\
\sloppyword{Lauren Pemberton-Nelson} & LD & 255\\
Nick Hooper & Grn & 138\\
David Furze & C & 117\\
Toby Prescott & UKIP & 93\\
Dean Porter & Ind & 47\\
Alhaji Kanumansa & APP & 38\\
\end{tabular*}

\subsubsection*{College \hspace*{\fill}\nolinebreak[1]%
\enspace\hspace*{\fill}
\finalhyphendemerits=0
[5th May]}

\index{College , Southwark@College, \emph{Southwark}}

Resignation of Helen Hayes (Lab).

\noindent
\begin{tabular*}{\columnwidth}{@{\extracolsep{\fill}} p{0.53\columnwidth} >{\itshape}l r @{\extracolsep{\fill}}}
Catherine Rose & Lab & 2258\\
Kate Bramson & C & 1024\\
Brigid Gardner & LD & 699\\
Dale Rapley & Grn & 371\\
Toby Prescott & UKIP & 118\\
Denis Pougin & APP & 25\\
\end{tabular*}

\subsubsection*{Newington \hspace*{\fill}\nolinebreak[1]%
\enspace\hspace*{\fill}
\finalhyphendemerits=0
[5th May]}

\index{Newington , Southwark@Newington, \emph{Southwark}}

Resignation of Neil Coyle MP (Lab).

\noindent
\begin{tabular*}{\columnwidth}{@{\extracolsep{\fill}} p{0.53\columnwidth} >{\itshape}l r @{\extracolsep{\fill}}}
James Coldwell & Lab & 2829\\
Martin Shapland & LD & 694\\
Nick Hooper & Grn & 464\\
Siobhan Aarons & C & 458\\
Gawain Towler & UKIP & 237\\
Terry Adewale & APP & 45\\
Michelle Baharier & Ind & 45\\
\end{tabular*}

\subsubsection*{Surrey Docks \hspace*{\fill}\nolinebreak[1]%
\enspace\hspace*{\fill}
\finalhyphendemerits=0
[9th June]}

\index{Surrey Docks , Southwark@Surrey Docks, \emph{Southwark}}

Resignation of Lisa Rajan (LD).

\noindent
\begin{tabular*}{\columnwidth}{@{\extracolsep{\fill}} p{0.53\columnwidth} >{\itshape}l r @{\extracolsep{\fill}}}
Dan Whitehead & LD & 1523\\
Will Holmes & Lab & 619\\
Craig Cox & C & 380\\
Colin Boyle & Grn & 218\\
Toby Prescott & UKIP & 187\\
John Hellings & Ind & 10\\
\end{tabular*}

\subsection*{Sutton}
\index{Sutton}

\subsubsection*{Carshalton Central \hspace*{\fill}\nolinebreak[1]%
\enspace\hspace*{\fill}
\finalhyphendemerits=0
[28th July]}

\index{Carshalton Central , Sutton@Carshalton C., \emph{Sutton}}

Resignation of Alan Salter (LD).

\noindent
\begin{tabular*}{\columnwidth}{@{\extracolsep{\fill}} p{0.53\columnwidth} >{\itshape}l r @{\extracolsep{\fill}}}
Chris Williams & LD & 1250\\
Melissa Pearce & C & 1061\\
Ross Hemingway & Grn & 211\\
Sarah Gwynn & Lab & 176\\
Bill Main-Ian & UKIP & 150\\
Ashley Dickenson & CPA & 29\\
\end{tabular*}

\subsection*{Wandsworth}
\index{Wandsworth}

\subsubsection*{Tooting \hspace*{\fill}\nolinebreak[1]%
\enspace\hspace*{\fill}
\finalhyphendemerits=0
[18th August]}

\index{Tooting , Wandsworth@Tooting, \emph{Wandsworth}}

Resignation of Ben Johnson (Lab).

\noindent
\begin{tabular*}{\columnwidth}{@{\extracolsep{\fill}} p{0.53\columnwidth} >{\itshape}l r @{\extracolsep{\fill}}}
Paul White & Lab & 1467\\
Thom Norman & C & 644\\
Eileen Arms & LD & 267\\
Albert Vickery & Grn & 116\\
Alexander Balkan & SDP & 15\\
\end{tabular*}

\subsubsection*{Queenstown \hspace*{\fill}\nolinebreak[1]%
\enspace\hspace*{\fill}
\finalhyphendemerits=0
[10th November]}

\index{Queenstown , Wandsworth@Queenstown, \emph{Wandsworth}}

Death of Sally-Ann Ephson (Lab).

\noindent
\begin{tabular*}{\columnwidth}{@{\extracolsep{\fill}} p{0.53\columnwidth} >{\itshape}l r @{\extracolsep{\fill}}}
Aydin Osborne Dikerdem & Lab & 1551\\
Rhodri Morgan & C & 987\\
Richard Davis & LD & 249\\
Stella Baker & Grn & 122\\
\end{tabular*}

\section{Greater Manchester}

\subsection*{Bolton}
\index{Bolton}

At the May 2016 ordinary election there was an unfilled vacancy in Hulton ward due to the death of Alan Walsh (C).
\index{Hulton , Bolton@Hulton, \emph{Bolton}}

\subsubsection*{Crompton \hspace*{\fill}\nolinebreak[1]%
\enspace\hspace*{\fill}
\finalhyphendemerits=0
[11th February]}

\index{Crompton , Bolton@Crompton, \emph{Bolton}}

Death of Sufrana Bashir Ismail (Lab).

\noindent
\begin{tabular*}{\columnwidth}{@{\extracolsep{\fill}} p{0.53\columnwidth} >{\itshape}l r @{\extracolsep{\fill}}}
Bilkis Ismail & Lab & 1961\\
Paul Eccles & UKIP & 320\\
Ryan Haslam & C & 302\\
Garry Veevers & LD & 117\\
Laura Diggle & Grn & 65\\
\end{tabular*}

\subsubsection*{Rumworth \hspace*{\fill}\nolinebreak[1]%
\enspace\hspace*{\fill}
\finalhyphendemerits=0
[6th October]}

\index{Rumworth , Bolton@Rumworth, \emph{Bolton}}

Death of Rosa Kay (Lab).

\noindent
\begin{tabular*}{\columnwidth}{@{\extracolsep{\fill}} p{0.53\columnwidth} >{\itshape}l r @{\extracolsep{\fill}}}
Shamim Abdullah & Lab & 2125\\
Joseph Baxendale & UKIP & 251\\
Mohammed Waqas & C & 167\\
Alan Johnson & Grn & 126\\
Becky Forrest & LD & 96\\
\end{tabular*}

\subsection*{Manchester}
\index{Manchester}

At the May 2016 ordinary election there was an unfilled vacancy in Charlestown ward due to the resignation of Mark Hackett (Lab).
\index{Charlestown , Manchester@Charlestown, \emph{Manchester}}

Pirate = Pirate Party UK

\subsubsection*{Higher Blackley \hspace*{\fill}\nolinebreak[1]%
\enspace\hspace*{\fill}
\finalhyphendemerits=0
[18th February]}

\index{Higher Blackley , Manchester@Higher Blackley, \emph{Manchester}}

Resignation of Anna Trotman (Lab).

\noindent
\begin{tabular*}{\columnwidth}{@{\extracolsep{\fill}} p{0.53\columnwidth} >{\itshape}l r @{\extracolsep{\fill}}}
Paula Sadler & Lab & 1062\\
Martin Power & UKIP & 308\\
David Semple & C & 130\\
Peter Matthews & LD & 78\\
Anne Power & Grn & 40\\
George Walkden & Pirate & 9\\
\end{tabular*}

\subsection*{Salford}
\index{Salford}

\subsubsection*{Little Hulton \hspace*{\fill}\nolinebreak[1]%
\enspace\hspace*{\fill}
\finalhyphendemerits=0
[5th May]}

\index{Little Hulton , Salford@Little Hulton, \emph{Salford}}

Resignation of Eric Burgoyne (Lab).

Combined with the 2016 ordinary election.
%; see page \pageref{LittleHultonSalford} for the result.

\subsection*{Stockport}
\index{Stockport}

At the May 2016 ordinary election there was an unfilled vacancy in Cheadle Hulme South ward due to the death of Lenny Grice (LD).
\index{Cheadle Hulme South , Stockport@Cheadle Hulme S., \emph{Stockport}}

\council{Tameside}

\subsubsection*{Mossley \hspace*{\fill}\nolinebreak[1]%
\enspace\hspace*{\fill}
\finalhyphendemerits=0
[5th May]}

\index{Mossley , Tameside@Mossley, \emph{Tameside}}

Resignation of Idu Miah (Lab).

Combined with the 2016 ordinary election.
%; see page \pageref{MossleyTameside} for the result.

\section{Merseyside}

\subsection*{Knowsley}
\index{Knowsley}

At the May 2016 ordinary election there were unfilled vacancies in St Bartholomews and Shevington wards due to the deaths of Tony Cunningham and Mal Sharp (both Lab) respectively.%
\index{Saint Bartholomews , Knowsley@St Bartholomews, \emph{Knowsley}}%
\index{Shevington , Knowsley@Shevington, \emph{Knowsley}}

\subsection*{Liverpool}
\index{Liverpool}

\subsubsection*{Belle Vale \hspace*{\fill}\nolinebreak[1]%
\enspace\hspace*{\fill}
\finalhyphendemerits=0
[5th May]}

\index{Belle Vale , Liverpool@Belle Vale, \emph{Liverpool}}

Resignation of Janet Kent (Lab).

Combined with the 2016 ordinary election.
%; see page \pageref{BelleValeLiverpool} for the result.

\subsubsection*{Norris Green \hspace*{\fill}\nolinebreak[1]%
\enspace\hspace*{\fill}
\finalhyphendemerits=0
[5th May]}

\index{Norris Green , Liverpool@Norris Green, \emph{Liverpool}}

Resignation of Debbie Caine (Lab).

Combined with the 2016 ordinary election.
%; see page \pageref{NorrisGreenLiverpool} for the result.

\subsection*{St Helens}
\index{Saint Helens@St Helens}

At the May 2016 ordinary election there was an unfilled vacancy in West Park ward due to the resignation of Robbie Ayres (Lab).
\index{West Park , Saint Helens@West Park, \emph{St Helens}}

\subsubsection*{Thatto Heath \hspace*{\fill}\nolinebreak[1]%
\enspace\hspace*{\fill}
\finalhyphendemerits=0
[21st January]}

\index{Thatto Heath , Saint Helens@Thatto Heath, \emph{St Helens}}

Death of Sheila Seddon (Lab).

\noindent
\begin{tabular*}{\columnwidth}{@{\extracolsep{\fill}} p{0.53\columnwidth} >{\itshape}l r @{\extracolsep{\fill}}}
Nova Charlton & Lab & 964\\
Alastair Sutcliffe & UKIP & 182\\
Lisa Mackarell & C & 147\\
Damien Clarke & Grn & 62\\
\end{tabular*}

\subsubsection*{West Park \hspace*{\fill}\nolinebreak[1]%
\enspace\hspace*{\fill}
\finalhyphendemerits=0
[5th May]}

\index{West Park , Saint Helens@West Park, \emph{St Helens}}

Resignation of Marie Rimmer (Lab).

Combined with the 2016 ordinary election.
%; see page \pageref{WestParkStHelens} for the result.

\subsection*{Sefton}
\index{Sefton}

\subsubsection*{St Oswald \hspace*{\fill}\nolinebreak[1]%
\enspace\hspace*{\fill}
\finalhyphendemerits=0
[5th May]}

\index{Saint Oswald , Sefton@St Oswald, \emph{Sefton}}

Resignation of Peter Dowd (Lab).

Combined with the 2016 ordinary election.
%; see page \pageref{StOswaldSefton} for the result.

\subsection*{Wirral}
\index{Wirral}

\subsubsection*{Liscard \hspace*{\fill}\nolinebreak[1]%
\enspace\hspace*{\fill}
\finalhyphendemerits=0
[5th May]}

\index{Liscard , Wirral@Liscard, \emph{Wirral}}

Resignation of Matthew Daniel (Lab).

Combined with the 2016 ordinary election.
%; see page \pageref{LiscardWirral} for the result.

\section{South Yorkshire}

\subsection*{Doncaster}
\index{Doncaster}

\subsubsection*{Edenthorpe and Kirk Sandall \hspace*{\fill}\nolinebreak[1]%
\enspace\hspace*{\fill}
\finalhyphendemerits=0
[5th May]}

\index{Edenthorpe and Kirk Sandall , Doncaster@Edenthorpe \& Kirk Sandall, \emph{Doncaster}}

Death of Tony Revill (Lab).

\noindent
\begin{tabular*}{\columnwidth}{@{\extracolsep{\fill}} p{0.53\columnwidth} >{\itshape}l r @{\extracolsep{\fill}}}
Andrea Robinson & Lab & 1148\\
Paul Bissett & UKIP & 960\\
Martin Greenhalgh & C & 268\\
Stephen Porter & LD & 202\\
\end{tabular*}

\subsection*{Sheffield}
\index{Sheffield}

\subsubsection*{Mosborough \hspace*{\fill}\nolinebreak[1]%
\enspace\hspace*{\fill}
\finalhyphendemerits=0
[8th September; LD gain from Lab]}

\index{Mosborough , Sheffield@Mosborough, \emph{Sheffield}}

Death of Isobel Bowler (Lab).

\noindent
\begin{tabular*}{\columnwidth}{@{\extracolsep{\fill}} p{0.53\columnwidth} >{\itshape}l r @{\extracolsep{\fill}}}
Gail Smith & LD & 1711\\
Julie Grocutt & Lab & 1279\\
Joanne Parkin & UKIP & 466\\
Andrew Taylor & C & 229\\
Julie White & Grn & 67\\
\end{tabular*}

\section{Tyne and Wear}

\subsection*{Gateshead}
\index{Gateshead}

\subsubsection*{Chopwell and Rowlands Gill \hspace*{\fill}\nolinebreak[1]%
\enspace\hspace*{\fill}
\finalhyphendemerits=0
[22nd September]}

\index{Chopwell and Rowlands Gill , Gateshead@Chopwell \& Rowlands Hill, \emph{Gateshead}}

Resignation of John Hamilton (Lab).

\noindent
\begin{tabular*}{\columnwidth}{@{\extracolsep{\fill}} p{0.53\columnwidth} >{\itshape}l r @{\extracolsep{\fill}}}
Dave Bradford & Lab & 1066\\
Ray Tolley & UKIP & 282\\
Amelia Ord & LD & 221\\
John Lathan & C & 156\\
Dave Castleton & Grn & 79\\
\end{tabular*}

\subsection*{Newcastle upon Tyne}
\index{Newcastle upon Tyne}

NuTCFP = Newcastle upon Tyne Community First Party

\subsubsection*{Blakelaw \hspace*{\fill}\nolinebreak[1]%
\enspace\hspace*{\fill}
\finalhyphendemerits=0
[15th September]}

\index{Blakelaw , Newcastle upon Tyne@Blakelaw, \emph{Newcastle upon Tyne}}

Resignation of David Stockdale (Lab).

\noindent
\begin{tabular*}{\columnwidth}{@{\extracolsep{\fill}} p{0.53\columnwidth} >{\itshape}l r @{\extracolsep{\fill}}}
Nora Casey & Lab & 1004\\
Ciaran Morrissey & LD & 654\\
Ritchie Lane & UKIP & 443\\
James Langley & C & 190\\
Brendan Derham & Grn & 105\\
\end{tabular*}

\subsubsection*{Blakelaw \hspace*{\fill}\nolinebreak[1]%
\enspace\hspace*{\fill}
\finalhyphendemerits=0
[24th November]}

\index{Blakelaw , Newcastle upon Tyne@Blakelaw, \emph{Newcastle upon Tyne}}

Resignation of Ben Riley (Lab).

\noindent
\begin{tabular*}{\columnwidth}{@{\extracolsep{\fill}} p{0.51\columnwidth} >{\itshape}l r @{\extracolsep{\fill}}}
Oskar Avery & Lab & 892\\
Ciaran Morrissey & LD & 784\\
John Gordon & NuTCFP & 164\\
Gerry Langley & C & 148\\
\end{tabular*}

\section{West Midlands}

\subsection*{Coventry}
\index{Coventry}

At the May 2016 ordinary election there was an unfilled vacancy in Sherbourne ward due to the death of Alison Gingell (Lab).
\index{Sherbourne , Coventry@Sherbourne, \emph{Coventry}}

\subsubsection*{Lower Stoke \hspace*{\fill}\nolinebreak[1]%
\enspace\hspace*{\fill}
\finalhyphendemerits=0
[11th February]}

\index{Lower Stoke , Coventry@Lower Stoke, \emph{Coventry}}

Death of Phil Townshend (Lab).

\noindent
\begin{tabular*}{\columnwidth}{@{\extracolsep{\fill}} p{0.53\columnwidth} >{\itshape}l r @{\extracolsep{\fill}}}
Rupinder Singh & Lab & 1235\\
Elaine Yebkal & C & 344\\
Harjinder Singh Sehmi & UKIP & 290\\
Aimee Challenor & Grn & 165\\
Christopher Glenn & LD & 124\\
\end{tabular*}

\subsection*{Dudley}
\index{Dudley}

\subsubsection*{\sloppyword{Kingswinford North and Wall Heath} \hspace*{\fill}\nolinebreak[1]%
\enspace\hspace*{\fill}
\finalhyphendemerits=0
[25th February]}

\index{Kingswinford North and Wall Heath , Dudley@Kingswinford N. \& Wall Heath, \emph{Dudley}}

Resignation of Natalie Neale (C).

\noindent
\begin{tabular*}{\columnwidth}{@{\extracolsep{\fill}} p{0.53\columnwidth} >{\itshape}l r @{\extracolsep{\fill}}}
Edward Lawrence & C & 1456\\
Lynn Boleyn & Lab & 934\\
Mick Forsyth & UKIP & 262\\
Andi Mohr & Grn & 52\\
\end{tabular*}

\subsubsection*{St James's \hspace*{\fill}\nolinebreak[1]%
\enspace\hspace*{\fill}
\finalhyphendemerits=0
[25th February]}

\index{Saint James's , Dudley@St James's, \emph{Dudley}}

Death of Mary Roberts (Lab).

\noindent
\begin{tabular*}{\columnwidth}{@{\extracolsep{\fill}} p{0.53\columnwidth} >{\itshape}l r @{\extracolsep{\fill}}}
Cathryn Bayton & Lab & 847\\
Graeme Lloyd & UKIP & 554\\
Wayne Sullivan & C & 427\\
\end{tabular*}

\subsubsection*{Pedmore and Stourbridge East \hspace*{\fill}\nolinebreak[1]%
\enspace\hspace*{\fill}
\finalhyphendemerits=0
[5th May]}

\index{Pedmore and Stourbridge East , Dudley@Pedmore \& Stourbridge E., \emph{Dudley}}

Resignation of Mike Wood (C).

Combined with the 2016 ordinary election.
%; see page \pageref{PedmoreStourbridgeEastDudley} for the result.

\subsection*{Sandwell}
\index{Sandwell}

At the May 2016 ordinary election there was an unfilled vacancy in Soho and Victoria ward due to the death of Darren Cooper (Lab).
\index{Soho and Victoria , Sandwell@Soho \& Victoria, \emph{Sandwell}}

\subsection*{Wolverhampton}
\index{Wolverhampton}

At the May 2016 ordinary election there was an unfilled vacancy in Bilston East ward due to the death of Bert Turner (Lab).
\index{Bilston East , Wolverhampton@Bilston E., \emph{Wolverhampton}}

\section{West Yorkshire}

\subsection*{Bradford}
\index{Bradford}

\subsubsection*{Wibsey \hspace*{\fill}\nolinebreak[1]%
\enspace\hspace*{\fill}
\finalhyphendemerits=0
[14th July]}

\index{Wibsey , Bradford@Wibsey, \emph{Bradford}}

Death of Lynne Smith (Lab).

\noindent
\begin{tabular*}{\columnwidth}{@{\extracolsep{\fill}} p{0.53\columnwidth} >{\itshape}l r @{\extracolsep{\fill}}}
Joanne Sharp & Lab & 1207\\
Jason Smith & UKIP & 655\\
Richard Sheard & C & 451\\
Angharad Griffiths & LD & 70\\
\end{tabular*}

\columnbreak

\section{Bedfordshire}

\subsection*{Luton}
\index{Luton}

\subsubsection*{High Town \hspace*{\fill}\nolinebreak[1]%
\enspace\hspace*{\fill}
\finalhyphendemerits=0
[30th June]}

\index{High Town , Luton@High Town, \emph{Luton}}

Resignation of Aysegul Gurbuz (Lab).

\noindent
\begin{tabular*}{\columnwidth}{@{\extracolsep{\fill}} p{0.53\columnwidth} >{\itshape}l r @{\extracolsep{\fill}}}
Maahwish Mirza & Lab & 505\\
Lyn Bliss & Grn & 273\\
Clive Mead & LD & 181\\
Sue Garrett & C & 141\\
John French & Ind & 102\\
Grace Froggatt & UKIP & 69\\
\end{tabular*}

\section{Berkshire}

\subsection*{Bracknell Forest}
\index{Bracknell Forest}

\subsubsection*{Central Sandhurst \hspace*{\fill}\nolinebreak[1]%
\enspace\hspace*{\fill}
\finalhyphendemerits=0
[20th October]}

\index{Central Sandhurst , Bracknell Forest@Central Sandhurst, \emph{Bracknell Forest}}

Resignation of Philip King (C).

\noindent
\begin{tabular*}{\columnwidth}{@{\extracolsep{\fill}} p{0.53\columnwidth} >{\itshape}l r @{\extracolsep{\fill}}}
Gaby Kennedy & C & 476\\
Anne Brunton & Lab & 211\\
\end{tabular*}

\subsection*{Reading}
\index{Reading}

\subsubsection*{Southcote \hspace*{\fill}\nolinebreak[1]%
\enspace\hspace*{\fill}
\finalhyphendemerits=0
[21st July]}

\index{Southcote , Reading@Southcote, \emph{Reading}}

Resignation of Matt Lawrence (Lab).

\noindent
\begin{tabular*}{\columnwidth}{@{\extracolsep{\fill}} p{0.53\columnwidth} >{\itshape}l r @{\extracolsep{\fill}}}
Jason Brook & Lab & 934\\
Russell Martin & C & 381\\
Mark Cole & LD & 77\\
Alan Lockey & Grn & 66\\
\end{tabular*}

\council{Windsor and Maidenhead}

\subsubsection*{Maidenhead Riverside \hspace*{\fill}\nolinebreak[1]%
\enspace\hspace*{\fill}
\finalhyphendemerits=0
[10th March]}

\index{Maidenhead Riverside , Windsor and Maidenhead@\sloppyword{Maidenhead Riverside, \emph{Windsor \& Maidenhead}}}

Resignation of Andrew Jenner (C).

\noindent
\begin{tabular*}{\columnwidth}{@{\extracolsep{\fill}} p{0.53\columnwidth} >{\itshape}l r @{\extracolsep{\fill}}}
Judith Diment & C & 916\\
Saghir Ahmed & LD & 397\\
Jeff Lloyd & Ind & 162\\
Nigel Smith & Lab & 144\\
George Chamberlaine & UKIP & 95\\
\end{tabular*}

\subsubsection*{Sunninghill and South Ascot \hspace*{\fill}\nolinebreak[1]%
\enspace\hspace*{\fill}
\finalhyphendemerits=0
[23rd June]}

\index{Sunninghill and South Ascot , Windsor and Maidenhead@Sunninghill \& South Ascot, \emph{Windsor \& Maidenhead}}

Resignation of George Bathurst (C).

\noindent
\begin{tabular*}{\columnwidth}{@{\extracolsep{\fill}} p{0.53\columnwidth} >{\itshape}l r @{\extracolsep{\fill}}}
Julian Sharpe & C & 1443\\
Spike Humphrey & Lab & 601\\
Tamasin Barnbrook & LD & 264\\
Nicole Fowler & UKIP & 214\\
\end{tabular*}

\subsection*{Wokingham}
\index{Wokingham}

\subsubsection*{Bulmershe and Whitegates \hspace*{\fill}\nolinebreak[1]%
\enspace\hspace*{\fill}
\finalhyphendemerits=0
[5th May]}

\index{Bulmershe and Whitegates , Wokingham@Bulmershe \& Whitegates, \emph{Wokingham}}

Resignation of Nicky Jerrome (Lab).

Combined with the 2016 ordinary election.
%; see page \pageref{BulmersheWhitegatesWokingham} for the result.

\section{Bristol}
\index{Bristol}

At the May 2016 ordinary election there was an unfilled vacancy in St George West ward due to the death of Ron Stone (Lab).
\index{Saint George West , Bristol@St George W., \emph{Bristol}}



\section{Buckinghamshire}

\subsection*{Chiltern}
\index{Chiltern}

\subsubsection*{Amersham Town \hspace*{\fill}\nolinebreak[1]%
\enspace\hspace*{\fill}
\finalhyphendemerits=0
[18th February; C gain from LD]}

\index{Amersham Town , Chiltern@Amersham Town, \emph{Chiltern}}

Death of Davida Allen (LD).

\noindent
\begin{tabular*}{\columnwidth}{@{\extracolsep{\fill}} p{0.53\columnwidth} >{\itshape}l r @{\extracolsep{\fill}}}
Jules Cook & C & 489\\
Richard Williams & LD & 354\\
Richard Phoenix & UKIP & 67\\
Robin Walters & Lab & 64\\
\end{tabular*}

\subsection*{Milton Keynes}
\index{Milton Keynes}

\subsubsection*{Bletchley East \hspace*{\fill}\nolinebreak[1]%
\enspace\hspace*{\fill}
\finalhyphendemerits=0
[5th May]}

\index{Bletchley East , Milton Keynes@Bletchley E., \emph{Milton Keynes}}

Resignation of Carole Baume (Lab).

Combined with the 2016 ordinary election.
%; see page \pageref{BletchleyEastMiltonKeynes} for the result.

\subsection*{South Bucks}
\index{South Bucks}

\subsubsection*{Farnham and Hedgerley \hspace*{\fill}\nolinebreak[1]%
\enspace\hspace*{\fill}
\finalhyphendemerits=0
[5th May]}

\index{Farnham and Hedgerley , South Bucks@Farnham \& Hedgerley, \emph{S. Bucks}}

Resignation of David Vincent (C).

\noindent
\begin{tabular*}{\columnwidth}{@{\extracolsep{\fill}} p{0.53\columnwidth} >{\itshape}l r @{\extracolsep{\fill}}}
Claire Lewis & C & 962\\
Delphine Gray-Fisk & UKIP & 339\\
Ryan Sains & Grn & 239\\
\end{tabular*}

\section{Cambridgeshire}

\subsection*{County Council}
\index{Cambridgeshire}

\subsubsection*{Sutton \hspace*{\fill}\nolinebreak[1]%
\enspace\hspace*{\fill}
\finalhyphendemerits=0
[18th February; LD gain from C]}

\index{Sutton , Cambridgeshire@Sutton, \emph{Cambs.}}

Death of Philip Read (C).

\noindent
\begin{tabular*}{\columnwidth}{@{\extracolsep{\fill}} p{0.53\columnwidth} >{\itshape}l r @{\extracolsep{\fill}}}
Lorna Dupre & LD & 1063\\
Mike Bradley & C & 651\\
Pete Bigsby & UKIP & 208\\
Owen Winters & Ind & 102\\
\end{tabular*}

\subsubsection*{St Neots Eaton Socon and Eynesbury \hspace*{\fill}\nolinebreak[1]%
\enspace\hspace*{\fill}
\finalhyphendemerits=0
[5th May]}

\index{Saint Neots Eaton Socon and Eynesbury , Cambridgeshire@St Neots Eaton Socon \& Eynesbury, \emph{Cambs.}}

Death of Steven van de Kerkhove (Ind).

\noindent
\begin{tabular*}{\columnwidth}{@{\extracolsep{\fill}} p{0.53\columnwidth} >{\itshape}l r @{\extracolsep{\fill}}}
Simone Taylor & Ind & 1114\\
Karl Wainwright & C & 1024\\
Nik Johnson & Lab & 625\\
James Corley & Ind & 479\\
\end{tabular*}

\council{East Cambridgeshire}

\subsubsection*{Bottisham \hspace*{\fill}\nolinebreak[1]%
\enspace\hspace*{\fill}
\finalhyphendemerits=0
[4th February]}

\index{Bottisham , East Cambridgeshire@Bottisham, \emph{E. Cambs.}}

Resignation of Vince Campbell (C).

\noindent
\begin{tabular*}{\columnwidth}{@{\extracolsep{\fill}} p{0.53\columnwidth} >{\itshape}l r @{\extracolsep{\fill}}}
Alan Sharp & C & 421\\
Steven Aronson & LD & 403\\
Steven O'Dell & Lab & 99\\
Daniel Divine & UKIP & 43\\
\end{tabular*}

\subsection*{Huntindgdonshire}
\index{Huntingdonshire}

\subsubsection*{St Neots Eynesbury \hspace*{\fill}\nolinebreak[1]%
\enspace\hspace*{\fill}
\finalhyphendemerits=0
[5th May]}

\index{Saint Neots Eynesbury , Cambridgeshire@St Neots Eynesbury, \emph{Hunts.}}

Death of Steven van de Kerkhove (Ind).

Combined with the 2016 ordinary election.
%; see page \pageref{StNeotsEynesburyHuntingdonshire} for the result.

\council{South Cambridgeshire}

At the May 2016 ordinary election there was an unfilled vacancy in Haslingfield and The Eversdens ward due to the resignation of Robin Page (Ind).
\index{Haslingfield and Eversdens , South Cambridgeshire@\sloppyword{Hasling{fi}eld \& The Eversdens, \emph{S. Cambs.}}}

\section{Cornwall}
\index{Cornwall}

%\subsection*{Cornwall}

MK = Mebyon Kernow

\subsubsection*{Launceston Central \hspace*{\fill}\nolinebreak[1]%
\enspace\hspace*{\fill}
\finalhyphendemerits=0
[14th January]}

\index{Launceston Central , Cornwall@Launceston C., \emph{Cornwall}}

Resignation of Alex Folkes (LD).

\noindent
\begin{tabular*}{\columnwidth}{@{\extracolsep{\fill}} p{0.53\columnwidth} >{\itshape}l r @{\extracolsep{\fill}}}
Gemma Massey & LD & 515\\
Val Bugden-Cawsey & C & 226\\
Roger Creagh-Osborne & Grn & 65\\
John Allman & CPA & 12\\
\end{tabular*}

\subsubsection*{Menheniot \hspace*{\fill}\nolinebreak[1]%
\enspace\hspace*{\fill}
\finalhyphendemerits=0
[14th April]}

\index{Menheniot , Cornwall@Menheniot, \emph{Cornwall}}

Death of Bernie Ellis (C).

\noindent
\begin{tabular*}{\columnwidth}{@{\extracolsep{\fill}} p{0.53\columnwidth} >{\itshape}l r @{\extracolsep{\fill}}}
Phil Seeva & C & 532\\
Charles Boney & LD & 472\\
Duncan Odgers & UKIP & 177\\
Martin Menear & Lab & 67\\
Richard Sedgley & Grn & 65\\
\end{tabular*}

\subsubsection*{Wadebridge West \hspace*{\fill}\nolinebreak[1]%
\enspace\hspace*{\fill}
\finalhyphendemerits=0
[14th April; LD gain from C]}

\index{Wadebridge West , Cornwall@Wadebridge W., \emph{Cornwall}}

Resignation of Scott Mann (C).

\noindent
\begin{tabular*}{\columnwidth}{@{\extracolsep{\fill}} p{0.53\columnwidth} >{\itshape}l r @{\extracolsep{\fill}}}
Karen McHugh & LD & 604\\
Sally Dunn & C & 356\\
Adrian Jones & Lab & 222\\
Helen Hyland & Ind & 111\\
Amanda Pennington & Grn & 95\\
\end{tabular*}

\subsubsection*{Newquay Treviglas \hspace*{\fill}\nolinebreak[1]%
\enspace\hspace*{\fill}
\finalhyphendemerits=0
[14th July; LD gain from UKIP]}

\index{Newquay Treviglas , Cornwall@Newquay Treviglas, \emph{Cornwall}}

Resignation of Mark Hicks (UKIP).

\noindent
\begin{tabular*}{\columnwidth}{@{\extracolsep{\fill}} p{0.53\columnwidth} >{\itshape}l r @{\extracolsep{\fill}}}
Paul Summers & LD & 486\\
Carl Leadbetter & C & 210\\
Julian Grover & Lab & 87\\
Roy Edwards & Ind & 58\\
\end{tabular*}

\subsubsection*{St Teath and St Breward \hspace*{\fill}\nolinebreak[1]%
\enspace\hspace*{\fill}
\finalhyphendemerits=0
[14th July; LD gain from Ind]}

\index{Saint Teath and Saint Breward , Cornwall@St Teath \& St Breward, \emph{Cornwall}}

Resignation of John Lugg (Ind).

\noindent
\begin{tabular*}{\columnwidth}{@{\extracolsep{\fill}} p{0.53\columnwidth} >{\itshape}l r @{\extracolsep{\fill}}}
Dominic Fairman & LD & 620\\
William Kitto & Ind & 242\\
Jeremy Stanford-Davis & C & 202\\
Susan Theobald & Ind & 181\\
Eddie Jones & Ind & 73\\
David Garrigan & Lab & 66\\
\end{tabular*}

\subsubsection*{Newlyn and Goonhavern \hspace*{\fill}\nolinebreak[1]%
\enspace\hspace*{\fill}
\finalhyphendemerits=0
[28th July; LD gain from C]}

\index{Newlyn and Goonhavern , Cornwall@Newlyn \& Goonhavern, \emph{Cornwall}}

Resignation of Lisa Gorman (elected as Lisa Shuttlewood) (C).

\noindent
\begin{tabular*}{\columnwidth}{@{\extracolsep{\fill}} p{0.53\columnwidth} >{\itshape}l r @{\extracolsep{\fill}}}
Maggie Vale & LD & 247\\
Paul Charlesworth & C & 234\\
Kenneth Yeo & Ind & 163\\
Rod Toms & MK & 161\\
Vicky Crowther & Lab & 77\\
James Tucker & Ind & 75\\
Rob Thomas & Ind & 54\\
\end{tabular*}

\subsubsection*{Four Lanes \hspace*{\fill}\nolinebreak[1]%
\enspace\hspace*{\fill}
\finalhyphendemerits=0
[1st September; LD gain from UKIP]}

\index{Four Lanes , Cornwall@Four Lanes, \emph{Cornwall}}

Resignation of Derek Elliott (UKIP).

\noindent
\begin{tabular*}{\columnwidth}{@{\extracolsep{\fill}} p{0.53\columnwidth} >{\itshape}l r @{\extracolsep{\fill}}}
Nathan Billings & LD & 300\\
Bernard Webb & Ind & 144\\
Peter Sheppard & C & 128\\
Peter Williams & Lab & 125\\
Christopher Lawrence & MK & 111\\
Dan Hall & UKIP & 57\\
\end{tabular*}

\section{Cumbria}

\subsection*{County Council}
\index{Cumbria}

\subsubsection*{Kendal Strickland and Fell \hspace*{\fill}\nolinebreak[1]%
\enspace\hspace*{\fill}
\finalhyphendemerits=0
[10th March]}

\index{Kendal Strickland and Fell , Cumbria@Kendal Strickland \& Fell, \emph{Cumbria}}

Death of John McCreesh (LD).

\noindent
\begin{tabular*}{\columnwidth}{@{\extracolsep{\fill}} p{0.53\columnwidth} >{\itshape}l r @{\extracolsep{\fill}}}
Peter Thornton & LD & 1067\\
Virginia Branney & Lab & 307\\
Harry Taylor & C & 172\\
Andy Mason & Grn & 128\\
David Walker & UKIP & 106\\
\end{tabular*}

\subsubsection*{Windermere \hspace*{\fill}\nolinebreak[1]%
\enspace\hspace*{\fill}
\finalhyphendemerits=0
[13th October]}

\index{Windermere , Cumbria@Windermere, \emph{Cumbria}}

Resignation of Colin Jones (LD).

\noindent
\begin{tabular*}{\columnwidth}{@{\extracolsep{\fill}} p{0.53\columnwidth} >{\itshape}l r @{\extracolsep{\fill}}}
Steve Rooke & LD & 1009\\
Ben Berry & C & 785\\
Penny Henderson & Lab & 88\\
Kate Threadgold & Grn & 46\\
\end{tabular*}

\subsection*{Allerdale}
\index{Allerdale}

\subsubsection*{Dalton \hspace*{\fill}\nolinebreak[1]%
\enspace\hspace*{\fill}
\finalhyphendemerits=0
[24th March; Ind gain from C]}

\index{Dalton , Allerdale@Dalton, \emph{Allerdale}}

Resignation of Colin Sharpe (C).

\noindent
\begin{tabular*}{\columnwidth}{@{\extracolsep{\fill}} p{0.53\columnwidth} >{\itshape}l r @{\extracolsep{\fill}}}
Marion Fitzgerald & Ind & 133\\
Ross Hayman & Lab & 118\\
Mike Johnson & C & 93\\
Eric Atkinson & UKIP & 53\\
Flic Crowley & Grn & 22\\
\end{tabular*}

\subsubsection*{Moss Bay \hspace*{\fill}\nolinebreak[1]%
\enspace\hspace*{\fill}
\finalhyphendemerits=0
[24th March]}

\index{Moss Bay , Allerdale@Moss Bay, \emph{Allerdale}}

Death of Bill Bacon (Lab).

\noindent
\begin{tabular*}{\columnwidth}{@{\extracolsep{\fill}} p{0.53\columnwidth} >{\itshape}l r @{\extracolsep{\fill}}}
Frank Johnston & Lab & 411\\
Bob Hardon & UKIP & 189\\
Louise Donnelly & C & 33\\
\end{tabular*}

\subsubsection*{Christchurch \hspace*{\fill}\nolinebreak[1]%
\enspace\hspace*{\fill}
\finalhyphendemerits=0
[22nd September; Lab gain from C]}

\index{Christchurch , Allerdale@Christchurch, \emph{Allerdale}}

Resignation of Margaret Jackson (C).

\noindent
\begin{tabular*}{\columnwidth}{@{\extracolsep{\fill}} p{0.53\columnwidth} >{\itshape}l r @{\extracolsep{\fill}}}
Joan Ellis & Lab & 324\\
Debbie Taylor & LD & 234\\
Simon Nicholson & C & 206\\
Eric Atkinson & UKIP & 32\\
\end{tabular*}

\subsection*{Barrow-in-Furness}
\index{Barrow-in-Furness}

\subsubsection*{Dalton South \hspace*{\fill}\nolinebreak[1]%
\enspace\hspace*{\fill}
\finalhyphendemerits=0
[5th May; Lab gain from C]}

\index{Dalton South , Barrow-in-Furness@Dalton S., \emph{Barrow-in-Furness}}

Death of Bill Bleasdale (C).

\noindent
\begin{tabular*}{\columnwidth}{@{\extracolsep{\fill}} p{0.53\columnwidth} >{\itshape}l r @{\extracolsep{\fill}}}
Shaun Blezard & Lab & 595\\
Des English & C & 415\\
Dick Young & UKIP & 237\\
\end{tabular*}

\subsubsection*{Parkside \hspace*{\fill}\nolinebreak[1]%
\enspace\hspace*{\fill}
\finalhyphendemerits=0
[8th September]}

\index{Parkside , Barrow-in-Furness@Parkside, \emph{Barrow-in-Furness}}

Death of Susan Opie (Lab).

\noindent
\begin{tabular*}{\columnwidth}{@{\extracolsep{\fill}} p{0.53\columnwidth} >{\itshape}l r @{\extracolsep{\fill}}}
Lee Roberts & Lab & 317\\
Roy Worthington & C & 257\\
Colin Rudd & UKIP & 34\\
\end{tabular*}

\subsection*{Carlisle}
\index{Carlisle}

\subsubsection*{Botcherby \hspace*{\fill}\nolinebreak[1]%
\enspace\hspace*{\fill}
\finalhyphendemerits=0
[7th January; Ind gain from Lab]}

\index{Botcherby , Carlisle@Botcherby, \emph{Carlisle}}

Death of Terry Scarborough (Lab).

\noindent
\begin{tabular*}{\columnwidth}{@{\extracolsep{\fill}} p{0.53\columnwidth} >{\itshape}l r @{\extracolsep{\fill}}}
Jack Paton & Ind & 381\\
Stephen Sidgwick & Lab & 250\\
Robert Currie & C & 115\\
\end{tabular*}

\subsubsection*{Castle \hspace*{\fill}\nolinebreak[1]%
\enspace\hspace*{\fill}
\finalhyphendemerits=0
[15th September]}

\index{Castle , Carlisle@Castle, \emph{Carlisle}}

Death of Gerald Caig (Lab).

\noindent
\begin{tabular*}{\columnwidth}{@{\extracolsep{\fill}} p{0.53\columnwidth} >{\itshape}l r @{\extracolsep{\fill}}}
Anne Glendinning & Lab & 398\\
Melissa Andrews & C & 228\\
Robbie Reid-Sinclair & UKIP & 107\\
Alison Hobson & LD & 88\\
Deborah Brown & Grn & 34\\
\end{tabular*}

\subsubsection*{Castle \hspace*{\fill}\nolinebreak[1]%
\enspace\hspace*{\fill}
\finalhyphendemerits=0
[24th November]}

\index{Castle , Carlisle@Castle, \emph{Carlisle}}

Resignation of Barrie Osgood (Lab).

\noindent
\begin{tabular*}{\columnwidth}{@{\extracolsep{\fill}} p{0.53\columnwidth} >{\itshape}l r @{\extracolsep{\fill}}}
Stephen Sidgwick & Lab & 350\\
John North & C & 194\\
Michael Story & UKIP & 79\\
David Wood & LD & 51\\
Neil Boothman & Grn & 36\\
\end{tabular*}

\subsection*{Eden}
\index{Eden}

\subsubsection*{Appleby (Appleby) \hspace*{\fill}\nolinebreak[1]%
\enspace\hspace*{\fill}
\finalhyphendemerits=0
[7th July]}

\index{Appleby Appleby , Eden@Appleby (Appleby), \emph{Eden}}

Death of Keith Morgan (Ind).

\noindent
\begin{tabular*}{\columnwidth}{@{\extracolsep{\fill}} p{0.53\columnwidth} >{\itshape}l r @{\extracolsep{\fill}}}
Karen Greenwood & Ind & 187\\
Philip Guest & C & 67\\
\end{tabular*}

\subsubsection*{Alston Moor \hspace*{\fill}\nolinebreak[1]%
\enspace\hspace*{\fill}
\finalhyphendemerits=0
[4th August; LD gain from C]}

\index{Alston Moor , Eden@Alston Moor, \emph{Eden}}

Resignation of David Hymers (C).

\noindent
\begin{tabular*}{\columnwidth}{@{\extracolsep{\fill}} p{0.53\columnwidth} >{\itshape}l r @{\extracolsep{\fill}}}
Thomas Sheriff & LD & 302\\
Stephen Harrison & C & 251\\
\end{tabular*}

\subsection*{South Lakeland}
\index{South Lakeland}

\subsubsection*{Windermere Bowness North \hspace*{\fill}\nolinebreak[1]%
\enspace\hspace*{\fill}
\finalhyphendemerits=0
[13th October]}

\index{Windermere Bowness North , South Lakeland@Windermere Bowness N., \emph{S. Lakeland}}

Resignation of Colin Jones (LD).

\noindent
\begin{tabular*}{\columnwidth}{@{\extracolsep{\fill}} p{0.53\columnwidth} >{\itshape}l r @{\extracolsep{\fill}}}
Andrew Jarvis & LD & 441\\
Martin Hall & C & 256\\
Kate Threadgold & Grn & 37\\
\end{tabular*}

\section{Derbyshire}

\subsection*{Derby}
\index{Derby}

\subsubsection*{Allestree \hspace*{\fill}\nolinebreak[1]%
\enspace\hspace*{\fill}
\finalhyphendemerits=0
[29th September]}

\index{Allestree , Derby@Allestree, \emph{Derby}}

Resignation of Richard Smalley (C).

\noindent
\begin{tabular*}{\columnwidth}{@{\extracolsep{\fill}} p{0.53\columnwidth} >{\itshape}l r @{\extracolsep{\fill}}}
Ged Potter & C & 2006\\
Deena Smith & LD & 1053\\
Oleg Sotnicenko & Lab & 409\\
Marten Kats & Grn & 115\\
Gaurav Pandey & UKIP & 91\\
\end{tabular*}

\council{North East Derbyshire}

BPP = British People's Party (2015)

\subsubsection*{Tupton \hspace*{\fill}\nolinebreak[1]%
\enspace\hspace*{\fill}
\finalhyphendemerits=0
[15th September; LD gain from Lab]}

\index{Tupton , North East Derbyshire@Tupton, \emph{N.E. Derbys.}}

Resignation of Wayne Lilleyman (Lab).

\noindent
\begin{tabular*}{\columnwidth}{@{\extracolsep{\fill}} p{0.53\columnwidth} >{\itshape}l r @{\extracolsep{\fill}}}
David Hancock & LD & 340\\
Cathy Goodyer & Lab & 308\\
Andrew Lovell & C & 155\\
Alan Garfitt & UKIP & 79\\
Ben Marshall & BPP & 6\\
\end{tabular*}

\section{Devon}

\subsection*{East Devon}
\index{East Devon}

EDevon = Independent East Devon Alliance

\subsubsection*{Exmouth Littleham \hspace*{\fill}\nolinebreak[1]%
\enspace\hspace*{\fill}
\finalhyphendemerits=0
[21st July]}

\index{Exmouth Littleham , East Devon@Exmouth Littleham, \emph{E. Devon}}

Death of Alison Greenhalgh (C).

\noindent
\begin{tabular*}{\columnwidth}{@{\extracolsep{\fill}} p{0.53\columnwidth} >{\itshape}l r @{\extracolsep{\fill}}}
Bruce de Saram & C & 547\\
Alex Sadiq & LD & 467\\
Keith Edwards & Lab & 193\\
\end{tabular*}

\subsubsection*{Honiton St Michael's \hspace*{\fill}\nolinebreak[1]%
\enspace\hspace*{\fill}
\finalhyphendemerits=0
[21st July]}

\index{Honiton Saint Michael's , East Devon@Honiton St Michael's, \emph{E. Devon}}

Resignation of David Foster (C).

\noindent
\begin{tabular*}{\columnwidth}{@{\extracolsep{\fill}} p{0.53\columnwidth} >{\itshape}l r @{\extracolsep{\fill}}}
Jenny Brown & C & 362\\
John Taylor & EDevon & 211\\
Henry Brown & Lab & 197\\
Ashley Alder & UKIP & 140\\
\end{tabular*}

\subsubsection*{Exmouth Brixington \hspace*{\fill}\nolinebreak[1]%
\enspace\hspace*{\fill}
\finalhyphendemerits=0
[6th October]}

\index{Exmouth Brixington , East Devon@Exmouth Brixington, \emph{E. Devon}}

Death of David Chapman (C).

\noindent
\begin{tabular*}{\columnwidth}{@{\extracolsep{\fill}} p{0.53\columnwidth} >{\itshape}l r @{\extracolsep{\fill}}}
Darryl Nicholas & C & 425\\
Robin Humphreys & EDevon & 324\\
Alex Sadiq & LD & 286\\
\end{tabular*}

\subsection*{Exeter}
\index{Exeter}

At the May 2016 ordinary election there was an unfilled vacancy in Pennsylvania ward due to the resignation of Jake Donovan (C).
\index{Pennsylvania , Exeter@Pennsylvania, \emph{Exeter}}

\subsection*{South Hams}
\index{South Hams}

\subsubsection*{Totnes \hspace*{\fill}\nolinebreak[1]%
\enspace\hspace*{\fill}
\finalhyphendemerits=0
[Wednesday 27th July; LD gain from Lab]}

\index{Totnes , South Hams@Totnes, \emph{S. Hams}}

Resignation of David Horsburgh (Lab).

\noindent
\begin{tabular*}{\columnwidth}{@{\extracolsep{\fill}} p{0.53\columnwidth} >{\itshape}l r @{\extracolsep{\fill}}}
John Birch & LD & 812\\
Alan White & Grn & 499\\
Alex Mockridge & Ind & 391\\
Andrew Barrand & C & 137\\
\end{tabular*}

\subsection*{Teignbridge}
\index{Teignbridge}

\subsubsection*{Teignmouth Central \hspace*{\fill}\nolinebreak[1]%
\enspace\hspace*{\fill}
\finalhyphendemerits=0
[22nd September; LD gain from C]}

\index{Teignmouth Central , Teignbridge@Teignmouth C., \emph{Teignbridge}}

Death of Geoff Bladon (C).

\noindent
\begin{tabular*}{\columnwidth}{@{\extracolsep{\fill}} p{0.53\columnwidth} >{\itshape}l r @{\extracolsep{\fill}}}
Alison Eden & LD & 491\\
Nick Maylam & C & 286\\
Steven Harvey & UKIP & 111\\
Malcolm Tipper & Lab & 72\\
\end{tabular*}

\subsubsection*{Bovey \hspace*{\fill}\nolinebreak[1]%
\enspace\hspace*{\fill}
\finalhyphendemerits=0
[15th December; LD gain from C]}

\index{Bovey , Teignbridge@Bovey, \emph{Teignbridge}}

Death of Anna Klinkenberg (C).

\noindent
\begin{tabular*}{\columnwidth}{@{\extracolsep{\fill}} p{0.53\columnwidth} >{\itshape}l r @{\extracolsep{\fill}}}
Sally Morgan & LD & 838\\
Taff Evans & C & 631\\
Eoghan Kelly & Ind & 169\\
Christopher Robillard & Lab & 103\\
Anne Bracher & UKIP & 98\\
Charlie West & Ind & 68\\
\end{tabular*}

\subsubsection*{Chudleigh \hspace*{\fill}\nolinebreak[1]%
\enspace\hspace*{\fill}
\finalhyphendemerits=0
[15th December; LD gain from C]}

\index{Chudleigh , Teignbridge@Chudleigh, \emph{Teignbridge}}

Death of Patricia Johnson-King (C).

\noindent
\begin{tabular*}{\columnwidth}{@{\extracolsep{\fill}} p{0.53\columnwidth} >{\itshape}l r @{\extracolsep{\fill}}}
Richard Keeling & LD & 680\\
Chris Webb & C & 470\\
Steven Harvey & UKIP & 89\\
Janette Parker & Lab & 81\\
\end{tabular*}

\subsection*{Torbay}
\index{Torbay}

\subsubsection*{Tormohun \hspace*{\fill}\nolinebreak[1]%
\enspace\hspace*{\fill}
\finalhyphendemerits=0
[5th May; LD gain from C]}

\index{Tormohun , Torbay@Tormohun, \emph{Torbay}}

Resignation of Andy Lang (C).

\noindent
\begin{tabular*}{\columnwidth}{@{\extracolsep{\fill}} p{0.53\columnwidth} >{\itshape}l r @{\extracolsep{\fill}}}
Nick Pentney & LD & 1126\\
Jackie Wakeham & C & 533\\
Darren Cowell & Lab & 344\\
Steve Walsh & UKIP & 315\\
Stephen Morley & Grn & 66\\
Michelle Goodman & TUSC & 27\\
\end{tabular*}

\section{Dorset}

\subsection*{County Council}
\index{Dorset}

\subsubsection*{Sherborne Rural \hspace*{\fill}\nolinebreak[1]%
\enspace\hspace*{\fill}
\finalhyphendemerits=0
[2nd June; LD gain from C]}

\index{Sherborne Rural , Dorset@Sherborne Rural, \emph{Dorset}}

Resignation of Michael Bevan (C).

\noindent
\begin{tabular*}{\columnwidth}{@{\extracolsep{\fill}} p{0.53\columnwidth} >{\itshape}l r @{\extracolsep{\fill}}}
Matthew Hall & LD & 1287\\
Mary Penfold & C & 1212\\
Geoff Freeman & Lab & 95\\
\end{tabular*}

\subsubsection*{Ferndown \hspace*{\fill}\nolinebreak[1]%
\enspace\hspace*{\fill}
\finalhyphendemerits=0
[1st September]}

\index{Ferndown , Dorset@Ferndown, \emph{Dorset}}

Death of John Wilson (C).

\noindent
\begin{tabular*}{\columnwidth}{@{\extracolsep{\fill}} p{0.53\columnwidth} >{\itshape}l r @{\extracolsep{\fill}}}
Steven Lugg & C & 2046\\
Peter Lucas & UKIP & 1092\\
Jason Jones & LD & 260\\
Peter Stokes & Lab & 190\\
\end{tabular*}

\subsubsection*{Ferndown \hspace*{\fill}\nolinebreak[1]%
\enspace\hspace*{\fill}
\finalhyphendemerits=0
[1st December; C gain from UKIP]}

\index{Ferndown , Dorset@Ferndown, \emph{Dorset}}

Resignation of Ian Smith (UKIP).

\noindent
\begin{tabular*}{\columnwidth}{@{\extracolsep{\fill}} p{0.53\columnwidth} >{\itshape}l r @{\extracolsep{\fill}}}
Andrew Parry & C & 1463\\
Lawrence Wilson & UKIP & 831\\
Jason Jones & LD & 301\\
Peter Stokes & Lab & 160\\
\end{tabular*}

\subsection*{Bournemouth}
\index{Bournemouth}

\subsubsection*{Kinson North \hspace*{\fill}\nolinebreak[1]%
\enspace\hspace*{\fill}
\finalhyphendemerits=0
[1st September]}

\index{Kinson North , Bournemouth@Kinson N., \emph{Bournemouth}}

Resignation of David Turtle (C).

\noindent
\begin{tabular*}{\columnwidth}{@{\extracolsep{\fill}} p{0.53\columnwidth} >{\itshape}l r @{\extracolsep{\fill}}}
John Perkins & C & 556\\
Dennis Gritt & Lab & 517\\
Duane Farr & UKIP & 313\\
Stephen Plant & LD & 116\\
Carla Gregory-May & Grn & 102\\
\end{tabular*}

\subsection*{East Dorset}
\index{East Dorset}

\subsubsection*{Alderholt \hspace*{\fill}\nolinebreak[1]%
\enspace\hspace*{\fill}
\finalhyphendemerits=0
[3rd March]}

\index{Alderholt , East Dorset@Alderholt, \emph{E. Dorset}}

Resignation of Ian Monks (C).

\noindent
\begin{tabular*}{\columnwidth}{@{\extracolsep{\fill}} p{0.53\columnwidth} >{\itshape}l r @{\extracolsep{\fill}}}
Gina Logan & C & 384\\
David Tooke & LD & 376\\
Chris Archibold & Lab & 49\\
\end{tabular*}

\subsubsection*{Parley \hspace*{\fill}\nolinebreak[1]%
\enspace\hspace*{\fill}
\finalhyphendemerits=0
[1st September]}

\index{Parley , East Dorset@Parley, \emph{E. Dorset}}

Death of John Wilson (C).

\noindent
\begin{tabular*}{\columnwidth}{@{\extracolsep{\fill}} p{0.53\columnwidth} >{\itshape}l r @{\extracolsep{\fill}}}
Andrew Parry & C & 631\\
Lawrence Wilson & UKIP & 369\\
Jason Jones & LD & 84\\
Brian Cropper & Lab & 52\\
\end{tabular*}

\subsection*{North Dorset}
\index{North Dorset}

\subsubsection*{Blandford Hilltop \hspace*{\fill}\nolinebreak[1]%
\enspace\hspace*{\fill}
\finalhyphendemerits=0
[5th May]}

\index{Blandford Hilltop , North Dorset@Blandford Hilltop, \emph{N. Dorset}}

Resignation of Mark Leonard (C).

\noindent
\begin{tabular*}{\columnwidth}{@{\extracolsep{\fill}} p{0.53\columnwidth} >{\itshape}l r @{\extracolsep{\fill}}}
Traci Handford & C & 176\\
Hugo Mieville & LD & 170\\
Haydn White & Lab & 100\\
William Woodhouse & UKIP & 96\\
\end{tabular*}

\subsubsection*{Hill Forts \hspace*{\fill}\nolinebreak[1]%
\enspace\hspace*{\fill}
\finalhyphendemerits=0
[5th May]}

\index{Hill Forts , North Dorset@Hill Forts, \emph{N. Dorset}}

Resignation of James Schwier (C).

\noindent
\begin{tabular*}{\columnwidth}{@{\extracolsep{\fill}} p{0.53\columnwidth} >{\itshape}l r @{\extracolsep{\fill}}}
Piers Brown & C & 946\\
John England & UKIP & 417\\
David Tibbles & LD & 318\\
Keith Yarwood & Lab & 273\\
\end{tabular*}

\subsection*{Poole}
\index{Poole}

\subsubsection*{Broadstone \hspace*{\fill}\nolinebreak[1]%
\enspace\hspace*{\fill}
\finalhyphendemerits=0
[13th October; LD gain from C]}

\index{Broadstone , Poole@Broadstone, \emph{Poole}}

Resignation of Joanne Tomlin (C).

\noindent
\begin{tabular*}{\columnwidth}{@{\extracolsep{\fill}} p{0.53\columnwidth} >{\itshape}l r @{\extracolsep{\fill}}}
Vikki Slade & LD & 2184\\
Mark Ujvari & C & 733\\
Alan Gerring & UKIP & 132\\
Mark Chivers & Grn & 57\\
Jason Sanderson & Lab & 45\\
\end{tabular*}

\council{Weymouth and Portland}

\subsubsection*{Wyke Regis \hspace*{\fill}\nolinebreak[1]%
\enspace\hspace*{\fill}
\finalhyphendemerits=0
[5th May]}

\index{Wyke Regis , Weymouth and Portland@Wyke Regis, \emph{Weymouth \& Portland}}

Resignation of Craig Martin (Lab).

Combined with the 2016 ordinary election.
%; see page \pageref{WykeRegisWeymouthPortland} for the result.

\subsubsection*{Wey Valley \hspace*{\fill}\nolinebreak[1]%
\enspace\hspace*{\fill}
\finalhyphendemerits=0
[20th October]}

\index{Wey Valley , Weymouth and Portland@Wey Valley, \emph{Weymouth \& Portland}}

Resignation of Cory Russell (C).

\noindent
\begin{tabular*}{\columnwidth}{@{\extracolsep{\fill}} p{0.53\columnwidth} >{\itshape}l r @{\extracolsep{\fill}}}
Tony Ferrari & C & 475\\
Robin Vaughan & LD & 118\\
Grafton Straker & Lab & 96\\
James Askew & Grn & 74\\
\end{tabular*}

\section{Durham}

\subsection*{Hartlepool}
\index{Hartlepool}

NHA = National Health Action

PHF = Putting Hartlepool First

\subsubsection*{Headland and Harbour \hspace*{\fill}\nolinebreak[1]%
\enspace\hspace*{\fill}
\finalhyphendemerits=0
[6th October; UKIP gain from Lab]}

\index{Headland and Harbour , Hartlepool@Headland \& Harbour, \emph{Hartlepool}}

Resignation of Peter Jackson (Lab).

\noindent
\begin{tabular*}{\columnwidth}{@{\extracolsep{\fill}} p{0.53\columnwidth} >{\itshape}l r @{\extracolsep{\fill}}}
Tim Fleming & UKIP & 496\\
Trevor Rogan & Lab & 255\\
Steve Latimer & PHF & 155\\
Benjamin Marshall & C & 41\\
John Price & NHA & 36\\
Chris Broadbent & Ind & 26\\
\end{tabular*}

\subsection*{Stockton-on-Tees}
\index{Stockton-on-Tees}

\subsubsection*{Parkfield and Oxbridge \hspace*{\fill}\nolinebreak[1]%
\enspace\hspace*{\fill}
\finalhyphendemerits=0
[28th January]}

\index{Parkfield and Oxbridge , Stockton-on-Tees@Parkfield \& Oxbridge, \emph{Stockton-on-Tees}}

Resignation of David Rose (Lab).

\noindent
\begin{tabular*}{\columnwidth}{@{\extracolsep{\fill}} p{0.53\columnwidth} >{\itshape}l r @{\extracolsep{\fill}}}
Allan Mitchell & Lab & 598\\
Stephen Richardson & C & 363\\
Peter Braney & UKIP & 113\\
Drew Durning & LD & 65\\
\end{tabular*}

\subsubsection*{Grangefield \hspace*{\fill}\nolinebreak[1]%
\enspace\hspace*{\fill}
\finalhyphendemerits=0
[1st September; C gain from Lab]}

\index{Grangefield , Stockton-on-Tees@Grangefield, \emph{Stockton-on-Tees}}

Death of Mike Clark (Lab).

\noindent
\begin{tabular*}{\columnwidth}{@{\extracolsep{\fill}} p{0.53\columnwidth} >{\itshape}l r @{\extracolsep{\fill}}}
Stephen Richardson & C & 807\\
Eleanor Clark & Lab & 689\\
Daniel Dalton & UKIP & 58\\
Nick Webb & LD & 44\\
\end{tabular*}

\section{East Sussex}

\subsection*{County Council}
\index{East Sussex}

\subsubsection*{St Helens and Silverhill \hspace*{\fill}\nolinebreak[1]%
\enspace\hspace*{\fill}
\finalhyphendemerits=0
[5th May]}

\index{Saint Helens and Silverhill , East Sussex@St Helens \& Silverhill, \emph{E. Sussex}}

Death of John Hodges (Lab).

\noindent
\begin{tabular*}{\columnwidth}{@{\extracolsep{\fill}} p{0.53\columnwidth} >{\itshape}l r @{\extracolsep{\fill}}}
Judy Rogers & Lab & 1441\\
Martin Clarke & C & 1253\\
Julie Hilton & Grn & 214\\
Stewart Rayment & LD & 212\\
\end{tabular*}

\council{Brighton and Hove}

\subsubsection*{East Brighton \hspace*{\fill}\nolinebreak[1]%
\enspace\hspace*{\fill}
\finalhyphendemerits=0
[4th August]}

\index{East Brighton , Brighton and Hove@East Brighton, \emph{Brighton \& Hove}}

Resignation of Maggie Barradell (Lab).

\noindent
\begin{tabular*}{\columnwidth}{@{\extracolsep{\fill}} p{0.53\columnwidth} >{\itshape}l r @{\extracolsep{\fill}}}
Lloyd Russell-Moyle & Lab & 1488\\
David Plant & C & 514\\
Mitch Alexander & Grn & 286\\
Leigh Farrow & UKIP & 152\\
Andrew England & LD & 116\\
Ramon Sammut & Ind & 31\\
\end{tabular*}

\subsection*{Eastbourne}
\index{Eastbourne}

\subsubsection*{Sovereign \hspace*{\fill}\nolinebreak[1]%
\enspace\hspace*{\fill}
\finalhyphendemerits=0
[24th November]}

\index{Sovereign , Eastbourne@Sovereign, \emph{Eastbourne}}

Resignation of Ray Blakebrough (C).

\noindent
\begin{tabular*}{\columnwidth}{@{\extracolsep{\fill}} p{0.53\columnwidth} >{\itshape}l r @{\extracolsep{\fill}}}
Paul Metcalfe & C & 1276\\
Roger Howarth & LD & 528\\
Louis Thorburn & Lab & 152\\
\end{tabular*}

\subsection*{Hastings}
\index{Hastings}

\subsubsection*{Old Hastings \hspace*{\fill}\nolinebreak[1]%
\enspace\hspace*{\fill}
\finalhyphendemerits=0
[5th May]}

\index{Old Hastings , Hastings@Old Hastings, \emph{Hastings}}

Death of John Hodges (Lab).

Combined with the 2016 ordinary election.
%; see page \pageref{OldHastingsHastings} for the result.

\subsection*{Lewes}
\index{Lewes}

\subsubsection*{Lewes Bridge \hspace*{\fill}\nolinebreak[1]%
\enspace\hspace*{\fill}
\finalhyphendemerits=0
[2nd June]}

\index{Lewes Bridge , Lewes@Lewes Bridge, \emph{Lewes}}

Resignation of Daisy Cooper (LD).

\noindent
\begin{tabular*}{\columnwidth}{@{\extracolsep{\fill}} p{0.53\columnwidth} >{\itshape}l r @{\extracolsep{\fill}}}
Will Elliott & LD & 543\\
Johnny Denis & Grn & 345\\
Richard Hurn & Lab & 212\\
Roy Burman & C & 117\\
\end{tabular*}

\subsection*{Rother}
\index{Rother}

\subsubsection*{Collington \hspace*{\fill}\nolinebreak[1]%
\enspace\hspace*{\fill}
\finalhyphendemerits=0
[27th October]}

\index{Collington , Rother@Collington, \emph{Rother}}

Resignation of Tony Mansi (Ind).

\noindent
\begin{tabular*}{\columnwidth}{@{\extracolsep{\fill}} p{0.53\columnwidth} >{\itshape}l r @{\extracolsep{\fill}}}
Deirdre Earl-Williams & Ind & 818\\
Andrew Burton & C & 393\\
Sara Watson & Lab & 87\\
Michael Phillips & UKIP & 66\\
\end{tabular*}

\subsubsection*{Darwell \hspace*{\fill}\nolinebreak[1]%
\enspace\hspace*{\fill}
\finalhyphendemerits=0
[27th October]}

\index{Darwell , Rother@Darwell, \emph{Rother}}

Resignation of Emily Rowlinson (C).

\noindent
\begin{tabular*}{\columnwidth}{@{\extracolsep{\fill}} p{0.53\columnwidth} >{\itshape}l r @{\extracolsep{\fill}}}
John Barnes & C & 359\\
Mary Varrall & LD & 259\\
Antonia Berelson & Lab & 79\\
Andrew Wedmore & Grn & 69\\
Edward Smith & UKIP & 60\\
\end{tabular*}

\subsection*{Wealden}
\index{Wealden}

\subsubsection*{Crowborough East \hspace*{\fill}\nolinebreak[1]%
\enspace\hspace*{\fill}
\finalhyphendemerits=0
[21st January]}

\index{Crowborough East , Wealden@Crowborough E., \emph{Wealden}}

Death of Peter Cowie (C).

\noindent
\begin{tabular*}{\columnwidth}{@{\extracolsep{\fill}} p{0.53\columnwidth} >{\itshape}l r @{\extracolsep{\fill}}}
Philip Lunn & C & 517\\
Jane Clark & LD & 198\\
Linda Scotson & Lab & 93\\
\end{tabular*}

\section{East Yorkshire}

\subsection*{East Riding}
\index{East Riding}

Beverley = Beverley Party

\subsubsection*{Pocklington Provincial \hspace*{\fill}\nolinebreak[1]%
\enspace\hspace*{\fill}
\finalhyphendemerits=0
[7th April; Ind gain from C]}

\index{Pocklington Provincial , East Riding@Pocklington Provincial, \emph{E. Riding}}

Death of Stephen Lane (C).

\noindent
\begin{tabular*}{\columnwidth}{@{\extracolsep{\fill}} p{0.53\columnwidth} >{\itshape}l r @{\extracolsep{\fill}}}
Andy Strangeway & Ind & 1032\\
Paul West & C & 980\\
Lucie Spadone & Lab & 490\\
Neil Tate & UKIP & 215\\
\end{tabular*}

\subsubsection*{East Wolds and Coastal \hspace*{\fill}\nolinebreak[1]%
\enspace\hspace*{\fill}
\finalhyphendemerits=0
[5th May]}

\index{East Wolds and Coastal , East Riding@East Wolds \& Coastal, \emph{E. Riding}}

Death of Margaret Chapman (C).

\noindent
\begin{tabular*}{\columnwidth}{@{\extracolsep{\fill}} p{0.53\columnwidth} >{\itshape}l r @{\extracolsep{\fill}}}
Paul Lisseter & C & 1885\\
Tom Lee & Lab & 860\\
Peter Watts & UKIP & 835\\
\end{tabular*}

\subsubsection*{South East Holderness \hspace*{\fill}\nolinebreak[1]%
\enspace\hspace*{\fill}
\finalhyphendemerits=0
[4th August]}

\index{South East Holderness , East Riding@South East Holderness, \emph{E. Riding}}

Resignation of Arthur Hodgson (C).

\noindent
\begin{tabular*}{\columnwidth}{@{\extracolsep{\fill}} p{0.53\columnwidth} >{\itshape}l r @{\extracolsep{\fill}}}
David Tucker & C & 917\\
Patrick Wilkinson & Lab & 806\\
Andrew Weaver & UKIP & 390\\
Dave Edwards & Ind & 173\\
Helen Wright & LD & 98\\
\end{tabular*}

\subsubsection*{St Mary's \hspace*{\fill}\nolinebreak[1]%
\enspace\hspace*{\fill}
\finalhyphendemerits=0
[20th October; LD gain from C]}

\index{Saint Mary's , East Riding@St Mary's, \emph{E. Riding}}

Death of Irene Charis (C).

\noindent
\begin{tabular*}{\columnwidth}{@{\extracolsep{\fill}} p{0.5\columnwidth} >{\itshape}l r @{\extracolsep{\fill}}}
Denis Healy & LD & 1497\\
Roy Begg & C & 947\\
Margaret Pinder & Lab & 689\\
Bea Willar & Beverley & 364\\
Chris Harrod & Ind & 141\\
John Kitchener & UKIP & 101\\
\end{tabular*}

\columnbreak

\section{Essex}

\subsection*{County Council}
\index{Essex}

HOSRA = Holland-on-Sea Residents Association

\subsubsection*{Clacton East \hspace*{\fill}\nolinebreak[1]%
\enspace\hspace*{\fill}
\finalhyphendemerits=0
[31st March; HOSRA gain from Tendring First]}

\index{Clacton East , Essex@Clacton E., \emph{Essex}}

Disqualification (sentenced to two years in prison, suspended, fraud) of Pierre Oxley (Tendring First).

\noindent
\begin{tabular*}{\columnwidth}{@{\extracolsep{\fill}} p{0.4975\columnwidth} >{\itshape}l r @{\extracolsep{\fill}}}
Colin Sargeant & HOSRA & 1781\\
Ben Smith & UKIP & 961\\
Richard Bleach & C & 628\\
Christopher Bird & Lab & 387\\
Rain Welham-Cobb & LD & 49\\
\end{tabular*}

\subsubsection*{Basildon Laindon Park and Fryerns \hspace*{\fill}\nolinebreak[1]%
\enspace\hspace*{\fill}
\finalhyphendemerits=0
[9th June; UKIP gain from Lab]}

\index{Basildon Laindon Park and Fryerns , Essex@Basildon Laindon Park \& Fryerns, \emph{Essex}}

Death of William Archibald (Lab).

\noindent
\begin{tabular*}{\columnwidth}{@{\extracolsep{\fill}} p{0.53\columnwidth} >{\itshape}l r @{\extracolsep{\fill}}}
Frank Ferguson & UKIP & 2034\\
Gavin Callaghan & Lab & 1600\\
Gary Maylin & C & 878\\
Philip Rackley & Grn & 264\\
\end{tabular*}

\subsection*{Basildon}
\index{Basildon}

At the May 2016 ordinary election there was an unfilled vacancy in Lee Chapel North ward due to the resignation of Trevor Malsbury (UKIP).
\index{Lee Chapel North , Basildon@Lee Chapel N., \emph{Basildon}}

\subsection*{Braintree}
\index{Braintree}

\subsubsection*{Witham South \hspace*{\fill}\nolinebreak[1]%
\enspace\hspace*{\fill}
\finalhyphendemerits=0
[5th May]}

\index{Witham South , Braintree@Witham S., \emph{Braintree}}

Resignation of Corinne Thompson (C).

\noindent
\begin{tabular*}{\columnwidth}{@{\extracolsep{\fill}} p{0.53\columnwidth} >{\itshape}l r @{\extracolsep{\fill}}}
Gavin Maclure & C & 446\\
Paul Heath & Lab & 376\\
Stephen Hicks & Grn & 184\\
\end{tabular*}

\subsubsection*{Bumpstead \hspace*{\fill}\nolinebreak[1]%
\enspace\hspace*{\fill}
\finalhyphendemerits=0
[20th October]}

\index{Bumpstead , Braintree@Bumpstead, \emph{Braintree}}

Resignation of Robert Bolton (C).

\noindent
\begin{tabular*}{\columnwidth}{@{\extracolsep{\fill}} p{0.53\columnwidth} >{\itshape}l r @{\extracolsep{\fill}}}
Diana Garrod & C & 350\\
Debbie Shaw & UKIP & 84\\
Bill Edwards & Lab & 45\\
Steve Bolter & LD & 40\\
Jenny Bishop & Grn & 23\\
\end{tabular*}

\subsubsection*{Witham North \hspace*{\fill}\nolinebreak[1]%
\enspace\hspace*{\fill}
\finalhyphendemerits=0
[20th October; Lab gain from C]}

\index{Witham North , Braintree@Witham N., \emph{Braintree}}

Resignation of Christopher Bailey (C).

\noindent
\begin{tabular*}{\columnwidth}{@{\extracolsep{\fill}} p{0.53\columnwidth} >{\itshape}l r @{\extracolsep{\fill}}}
Phil Barlow & Lab & 339\\
Lorne Campbell & C & 308\\
Michelle Weeks & Grn & 227\\
Mark Scott & LD & 31\\
\end{tabular*}

\subsection*{Brentwood}
\index{Brentwood}

\subsubsection*{Tipps Cross \hspace*{\fill}\nolinebreak[1]%
\enspace\hspace*{\fill}
\finalhyphendemerits=0
[5th May]}

\index{Tipps Cross , Brentwood@Tipps Cross, \emph{Brentwood}}

Resignation of Madeline Henwood (C).

Combined with the 2016 ordinary election.
%; see page \pageref{TippsCrossBrentwood} for the result.

\subsection*{Castle Point}
\index{Castle Point}

At the May 2016 ordinary election there was an unfilled vacancy in Canvey Island East ward due to the resignation of Colin Letchford (Ind).
\index{Canvey Island East , Castle Point@Canvey Island E., \emph{Castle Point}}

\subsubsection*{St George's \hspace*{\fill}\nolinebreak[1]%
\enspace\hspace*{\fill}
\finalhyphendemerits=0
[5th May]}

\index{Saint George's , Castle Point@St George's, \emph{Castle Point}}

Death of Jacqui Govier (C).

Combined with the 2016 ordinary election.
%; see page \pageref{StGeorgesCastlePoint} for the result.

\subsection*{Colchester}
\index{Colchester}

At the May 2016 ordinary election there was an unfilled vacancy in Stanway ward due to the resignation of Laura Sykes (Ind elected as LD).
\index{Stanway , Colchester@Stanway, \emph{Colchester}}

\subsection*{Epping Forest}
\index{Epping Forest}

Loughton = Loughton Residents Association

At the May 2016 ordinary election there was an unfilled vacancy in Loughton Roding ward due to the death of Ken Angold-Stephens (Loughton).
\index{Loughton Roding , Epping Forest@Loughton Roding, \emph{Epping Forest}}

\subsubsection*{Loughton Forest \hspace*{\fill}\nolinebreak[1]%
\enspace\hspace*{\fill}
\finalhyphendemerits=0
[5th May]}

\index{Loughton Forest , Epping Forest@Loughton Forest, \emph{Epping Forest}}

Resignation of Sharon Weston (Loughton).

Combined with the 2016 ordinary election.
%; see page \pageref{LoughtonForestEppingForest} for the result.

\subsection*{Maldon}
\index{Maldon}

\subsubsection*{Maldon West \hspace*{\fill}\nolinebreak[1]%
\enspace\hspace*{\fill}
\finalhyphendemerits=0
[8th December; Ind gain from C]}

\index{Maldon West , Maldon@Maldon W., \emph{Maldon}}

Death of Charles Mackenzie (C).

\noindent
\begin{tabular*}{\columnwidth}{@{\extracolsep{\fill}} p{0.53\columnwidth} >{\itshape}l r @{\extracolsep{\fill}}}
Flo Shaughnessy & Ind & 279\\
Martin Harvey & C & 172\\
Andrew Francis & UKIP & 114\\
Janet Carden & Grn & 69\\
Richard Perry & BNP & 51\\
John Sweeney & Lab & 47\\
\end{tabular*}

\subsection*{Tendring}
\index{Tendring}

\subsubsection*{St Pauls \hspace*{\fill}\nolinebreak[1]%
\enspace\hspace*{\fill}
\finalhyphendemerits=0
[5th May]}

\index{Saint Pauls , Tendring@St Pauls, \emph{Tendring}}

Resignation of Ashley Mooney (UKIP).

\noindent
\begin{tabular*}{\columnwidth}{@{\extracolsep{\fill}} p{0.53\columnwidth} >{\itshape}l r @{\extracolsep{\fill}}}
Jack Parsons & UKIP & 424\\
Danny Mayzes & C & 311\\
William Hones & Ind & 248\\
Chris Bird & Lab & 148\\
\end{tabular*}

\section{Gloucestershire}

\subsection*{County Council}
\index{Gloucestershire}

\subsubsection*{Churchdown \hspace*{\fill}\nolinebreak[1]%
\enspace\hspace*{\fill}
\finalhyphendemerits=0
[5th May]}

\index{Churchdown , Gloucestershire@Churchdown, \emph{Glos.}}

Death of Bill Whelan (LD).

\noindent
\begin{tabular*}{\columnwidth}{@{\extracolsep{\fill}} p{0.53\columnwidth} >{\itshape}l r @{\extracolsep{\fill}}}
Jack Williams & LD & 1700\\
Graham Bocking & C & 1062\\
Ed Buxton & Lab & 359\\
\end{tabular*}

\subsection*{Cotswold}
\index{Cotswold}

\subsubsection*{Stow \hspace*{\fill}\nolinebreak[1]%
\enspace\hspace*{\fill}
\finalhyphendemerits=0
[29th September; LD gain from C]}

\index{Stow , Cotswold@Stow, \emph{Cotswold}}

Death of Barry Dare (C).

\noindent
\begin{tabular*}{\columnwidth}{@{\extracolsep{\fill}} p{0.53\columnwidth} >{\itshape}l r @{\extracolsep{\fill}}}
Dilys Neill & LD & 555\\
David Penman & C & 300\\
\end{tabular*}

\subsection*{Gloucester}
\index{Gloucester}

\subsubsection*{Longlevens \hspace*{\fill}\nolinebreak[1]%
\enspace\hspace*{\fill}
\finalhyphendemerits=0
[3rd November]}

\index{Longlevens , Gloucester@Longlevens, \emph{Gloucester}}

Death of Jim Porter (C).

\noindent
\begin{tabular*}{\columnwidth}{@{\extracolsep{\fill}} p{0.53\columnwidth} >{\itshape}l r @{\extracolsep{\fill}}}
Clive Walford & C & 1066\\
Linda Castle & LD & 852\\
Terry Haines & Lab & 223\\
Daniel Woolf & UKIP & 167\\
\end{tabular*}

\section{Hampshire}

\subsection*{County Council}
\index{Hampshire}

\subsubsection*{Fareham Town \hspace*{\fill}\nolinebreak[1]%
\enspace\hspace*{\fill}
\finalhyphendemerits=0
[5th May]}

\index{Fareham Town , Hampshire@Fareham Town, \emph{Hants.}}

Resignation of George Ringrow (C).

\noindent
\begin{tabular*}{\columnwidth}{@{\extracolsep{\fill}} p{0.53\columnwidth} >{\itshape}l r @{\extracolsep{\fill}}}
Christopher Matthews & C & 4408\\
Paul Sturgess & UKIP & 2164\\
Peter Trott & LD & 1905\\
James Carr & Lab & 1360\\
David Harrison & Grn & 673\\
\end{tabular*}

\subsubsection*{Headley \hspace*{\fill}\nolinebreak[1]%
\enspace\hspace*{\fill}
\finalhyphendemerits=0
[5th May]}

\index{Headley , Hampshire@Headley, \emph{Hants.}}

Resignation of Ferris Cowper (C).

\noindent
\begin{tabular*}{\columnwidth}{@{\extracolsep{\fill}} p{0.53\columnwidth} >{\itshape}l r @{\extracolsep{\fill}}}
Floss Mitchell & C & 2201\\
Trevor Maroney & LD & 1321\\
Peter Baillie & UKIP & 791\\
\end{tabular*}

\council{Basingstoke and Deane}

\subsubsection*{Basing \hspace*{\fill}\nolinebreak[1]%
\enspace\hspace*{\fill}
\finalhyphendemerits=0
[6th October]}

\index{Basing , Basingstoke and Deane@Basing, \emph{Basingstoke \& Deane}}

Resignation of Clive Pinder (C).

\noindent
\begin{tabular*}{\columnwidth}{@{\extracolsep{\fill}} p{0.53\columnwidth} >{\itshape}l r @{\extracolsep{\fill}}}
Paul Gaskell & C & 1051\\
Richard Lilleker & LD & 323\\
Andrew Toal & Lab & 184\\
\end{tabular*}

\subsubsection*{Tadley South \hspace*{\fill}\nolinebreak[1]%
\enspace\hspace*{\fill}
\finalhyphendemerits=0
[24th November]}

\index{Tadley South , Basingstoke and Deane@Tadley S., \emph{Basingstoke \& Deane}}

Death of Rob Musson (C).

\noindent
\begin{tabular*}{\columnwidth}{@{\extracolsep{\fill}} p{0.53\columnwidth} >{\itshape}l r @{\extracolsep{\fill}}}
Kerri Carruthers & C & 456\\
Jo Slimin & LD & 342\\
Claire Ballard & Lab & 88\\
Phil Heath & UKIP & 41\\
\end{tabular*}

\council{East Hampshire}

JACP = Justice and Anti-Corruption Party

\subsubsection*{Clanfield and Finchdean \hspace*{\fill}\nolinebreak[1]%
\enspace\hspace*{\fill}
\finalhyphendemerits=0
[5th May]}

\index{Clanfield and Finchdean , East Hampshire@Clanfield \& Finchdean, \emph{E. Hants.}}

Resignation of Tony Denton (C).

\noindent
\begin{tabular*}{\columnwidth}{@{\extracolsep{\fill}} p{0.53\columnwidth} >{\itshape}l r @{\extracolsep{\fill}}}
Nigel Wren & C & 765\\
Elaine Woodard & LD & 410\\
David Alexander & UKIP & 303\\
\end{tabular*}

\subsubsection*{The Hangers and Forest \hspace*{\fill}\nolinebreak[1]%
\enspace\hspace*{\fill}
\finalhyphendemerits=0
[Tuesday 26th July]}

\index{Hangers and Forest , East Hampshire@The Hangers \& Forest, \emph{E. Hants.}}

Death of Judy Onslow (C).

\noindent
\begin{tabular*}{\columnwidth}{@{\extracolsep{\fill}} p{0.53\columnwidth} >{\itshape}l r @{\extracolsep{\fill}}}
Keith Budden & C & 236\\
Roger Mullenger & LD & 227\\
Don Jerrard & JACP & 41\\
Neil Owsnett & Lab & 17\\
\end{tabular*}

\subsection*{Eastleigh}
\index{Eastleigh}

\subsubsection*{West End North \hspace*{\fill}\nolinebreak[1]%
\enspace\hspace*{\fill}
\finalhyphendemerits=0
[11th February]}

\index{West End North , Eastleigh@West End N., \emph{Eastleigh}}

Death of Tony Noyce (LD).

\noindent
\begin{tabular*}{\columnwidth}{@{\extracolsep{\fill}} p{0.53\columnwidth} >{\itshape}l r @{\extracolsep{\fill}}}
Janice Asman & LD & 582\\
Steven Broomfield & C & 315\\
Hugh McGuinness & UKIP & 115\\
Andy Andrews & Lab & 58\\
Glynn Fleming & Grn & 28\\
\end{tabular*}

\subsubsection*{Fair Oak and Horton Heath \hspace*{\fill}\nolinebreak[1]%
\enspace\hspace*{\fill}
\finalhyphendemerits=0
[3rd November]}

\index{Fair Oak and Horton Heath , Eastleigh@Fair Oak \& Horton Heath, \emph{Eastleigh}}

Death of Roger Smith (LD).

\noindent
\begin{tabular*}{\columnwidth}{@{\extracolsep{\fill}} p{0.53\columnwidth} >{\itshape}l r @{\extracolsep{\fill}}}
Nicholas Couldrey & LD & 828\\
Steven Broomfield & C & 553\\
Hugh McGuinness & UKIP & 286\\
John Sorley & Lab & 132\\
\end{tabular*}

\subsubsection*{Hedge End Wildern \hspace*{\fill}\nolinebreak[1]%
\enspace\hspace*{\fill}
\finalhyphendemerits=0
[22nd December]}

\index{Hedge End Wildern , Eastleigh@Hedge End Wildern, \emph{Eastleigh}}

Resignation of Emma Norman (LD).

\noindent
\begin{tabular*}{\columnwidth}{@{\extracolsep{\fill}} p{0.53\columnwidth} >{\itshape}l r @{\extracolsep{\fill}}}
Ian Corben & LD & 672\\
Ben Burcombe-Filer & C & 263\\
Terry Crow & Lab & 107\\
\end{tabular*}

\subsection*{Fareham}
\index{Fareham}

\subsubsection*{Fareham East \hspace*{\fill}\nolinebreak[1]%
\enspace\hspace*{\fill}
\finalhyphendemerits=0
[5th May]}

\index{Fareham East , Fareham@Fareham E., \emph{Fareham}}

Resignation of Paul Whittle (LD).

Combined with the 2016 ordinary election.
%; see page \pageref{FarehamEastFareham} for the result.

\subsection*{Havant}
\index{Havant}

\subsubsection*{Bondfields \hspace*{\fill}\nolinebreak[1]%
\enspace\hspace*{\fill}
\finalhyphendemerits=0
[3rd March]}

\index{Bondfields , Havant@Bondfields, \emph{Havant}}

Death of Frida Edwards (C).

\noindent
\begin{tabular*}{\columnwidth}{@{\extracolsep{\fill}} p{0.53\columnwidth} >{\itshape}l r @{\extracolsep{\fill}}}
Lance Quantrill & C & 207\\
Catherine Billam & LD & 187\\
Tony Berry & Lab & 148\\
Geoff Whiffen & UKIP & 143\\
\end{tabular*}

\subsection*{Portsmouth}
\index{Portsmouth}

At the May 2016 ordinary election there was an unfilled vacancy in Nelson ward due to the resignation of Ken Ferrett (Ind elected as Lab).
\index{Nelson , Portsmouth@Nelson, \emph{Portsmouth}}

\subsection*{Rushmoor}
\index{Rushmoor}

At the May 2016 ordinary election there was an unfilled vacancy in Manor Park ward due to the death of Ron Hughes (C).
\index{Manor Park , Rushmoor@Manor Park, \emph{Rushmoor}}

\subsubsection*{Aldershot Park \hspace*{\fill}\nolinebreak[1]%
\enspace\hspace*{\fill}
\finalhyphendemerits=0
[2nd June]}

\index{Aldershot Park , Rushmoor@Aldershot Park, \emph{Rushmoor}}

Ordinary election postponed from 5th May; death of candidate Ron Hughes (C).

\noindent
\begin{tabular*}{\columnwidth}{@{\extracolsep{\fill}} p{0.53\columnwidth} >{\itshape}l r @{\extracolsep{\fill}}}
Mike Roberts & Lab & 525\\
Jeffery Boxall & UKIP & 314\\
Matthew Collins & C & 264\\
Lucy Perrin & Grn & 41\\
\end{tabular*}

\subsection*{Southampton}
\index{Southampton}

\subsubsection*{Harefield \hspace*{\fill}\nolinebreak[1]%
\enspace\hspace*{\fill}
\finalhyphendemerits=0
[5th May]}

\index{Harefield , Southampton@Harefield, \emph{Southampton}}

Resignation of Royston Smith (C).

Combined with the 2016 ordinary election.
%; see page \pageref{HarefieldSouthampton} for the result.

\subsubsection*{Woolston \hspace*{\fill}\nolinebreak[1]%
\enspace\hspace*{\fill}
\finalhyphendemerits=0
[5th May]}

\index{Woolston , Southampton@Woolston, \emph{Southampton}}

Resignation of Caran Chamberlain (Lab).

Combined with the 2016 ordinary election.
%; see page \pageref{WoolstonSouthampton} for the result.

\section{Hertfordshire}

\subsection*{County Council}
\index{Hertfordshire}

\subsubsection*{Bushey North \hspace*{\fill}\nolinebreak[1]%
\enspace\hspace*{\fill}
\finalhyphendemerits=0
[21st January]}

\index{Bushey North , Hertfordshire@Bushey N., \emph{Herts.}}

Death of Steve O'Brien (C).

\noindent
\begin{tabular*}{\columnwidth}{@{\extracolsep{\fill}} p{0.53\columnwidth} >{\itshape}l r @{\extracolsep{\fill}}}
Jane West & C & 881\\
Shailan Shah & LD & 333\\
Seamus Williams & Lab & 286\\
David Hoy & UKIP & 176\\
\end{tabular*}

\subsection*{Dacorum}
\index{Dacorum}

\subsubsection*{Adeyfield West \hspace*{\fill}\nolinebreak[1]%
\enspace\hspace*{\fill}
\finalhyphendemerits=0
[29th September; LD gain from C]}

\index{Adeyfield West , Dacorum@Adeyfield W., \emph{Dacorum}}

Death of Sharon Adshead (C).

\noindent
\begin{tabular*}{\columnwidth}{@{\extracolsep{\fill}} p{0.53\columnwidth} >{\itshape}l r @{\extracolsep{\fill}}}
Adrian England & LD & 520\\
Tony Gallagher & C & 233\\
Gary Cook & Lab & 166\\
Rachel Biggs & UKIP & 115\\
Angela Lynch & Grn & 17\\
\end{tabular*}

\council{East Hertfordshire}

\subsubsection*{Puckeridge \hspace*{\fill}\nolinebreak[1]%
\enspace\hspace*{\fill}
\finalhyphendemerits=0
[15th September]}

\index{Puckeridge , East Hertfordshire@Puckeridge, \emph{E. Herts.}}

Resignation of James Cartwright (C).

\noindent
\begin{tabular*}{\columnwidth}{@{\extracolsep{\fill}} p{0.53\columnwidth} >{\itshape}l r @{\extracolsep{\fill}}}
Peter Boylan & C & 179\\
Geoffrey Miles & UKIP & 79\\
Sara Mihajlovic & LD & 75\\
David Bell & Lab & 46\\
Tabitha Evans & Grn & 38\\
\end{tabular*}

\council{North Hertfordshire}

\subsubsection*{Hitchin Oughton \hspace*{\fill}\nolinebreak[1]%
\enspace\hspace*{\fill}
\finalhyphendemerits=0
[10th November]}

\index{Hitchin Oughton , North Hertfordshire@Hitchin Oughton, \emph{N. Herts.}}

Resignation of Simon Watson (Lab).

\noindent
\begin{tabular*}{\columnwidth}{@{\extracolsep{\fill}} p{0.6\columnwidth} >{\itshape}l r @{\extracolsep{\fill}}}
Martin Stears-Handscomb & Lab & 258\\
Jackie McDonald & Ind & 200\\
Serena Farrow & C & 158\\
Louise Peace & LD & 150\\
George Howe & Grn & 42\\
\end{tabular*}

\subsection*{St Albans}
\index{Saint Albans@St Albans}

\subsubsection*{Clarence \hspace*{\fill}\nolinebreak[1]%
\enspace\hspace*{\fill}
\finalhyphendemerits=0
[20th October]}

\index{Clarence , Saint Albans@Clarence, \emph{St Albans}}

Resignation of Sam Rowlands (LD).

\noindent
\begin{tabular*}{\columnwidth}{@{\extracolsep{\fill}} p{0.53\columnwidth} >{\itshape}l r @{\extracolsep{\fill}}}
Ellie Hudspith & LD & 916\\
Michael Roth & C & 388\\
Liz Mills & Lab & 193\\
Keith Cotton & Grn & 98\\
David Dickson & UKIP & 16\\
\end{tabular*}

\subsection*{Three Rivers}
\index{Three Rivers}

\subsubsection*{Carpenders Park \hspace*{\fill}\nolinebreak[1]%
\enspace\hspace*{\fill}
\finalhyphendemerits=0
[5th May]}

\index{Carpenders Park , Three Rivers@Carpenders Park, \emph{Three Rivers}}

Resignation of Angela Roberts (C).

Combined with the 2016 ordinary election.
%; see page \pageref{CarpendersParkThreeRivers} for the result.

\subsubsection*{South Oxhey \hspace*{\fill}\nolinebreak[1]%
\enspace\hspace*{\fill}
\finalhyphendemerits=0
[5th May]}

\index{South Oxhey , Three Rivers@South Oxhey, \emph{Three Rivers}}

Death of Len Tippen (Lab).

Combined with the 2016 ordinary election.
%; see page \pageref{SouthOxheyThreeRivers} for the result.

\subsection*{Welwyn Hatfield}
\index{Welwyn Hatfield}

At the May 2016 ordinary election there was an unfilled vacancy in Northaw and Cuffley ward due to the death of Adrian Prest (C).
\index{Northaw and Cuffley , Welwyn Hatfield@Northaw \& Cuffley, \emph{Welwyn Hatfield}}

\subsubsection*{Haldens \hspace*{\fill}\nolinebreak[1]%
\enspace\hspace*{\fill}
\finalhyphendemerits=0
[17th November]}

\index{Haldens , Welwyn Hatfield@Haldens, \emph{Welwyn Hatfield}}

Resignation of Malcolm Spinks (C).

\noindent
\begin{tabular*}{\columnwidth}{@{\extracolsep{\fill}} p{0.53\columnwidth} >{\itshape}l r @{\extracolsep{\fill}}}
Nathaniel Chapman & C & 502\\
Astrid Thorpe & Lab & 454\\
Anthony Dennis & LD & 437\\
Lynne Allison & Grn & 81\\
\end{tabular*}

\section{Kent}

\subsection*{County Council}
\index{Kent}

\subsubsection*{Gravesham East \hspace*{\fill}\nolinebreak[1]%
\enspace\hspace*{\fill}
\finalhyphendemerits=0
[18th August; C gain from Lab]}

\index{Gravesham East , Kent@Gravesham E., \emph{Kent}}

Death of Jane Cribbon (Lab).

\noindent
\begin{tabular*}{\columnwidth}{@{\extracolsep{\fill}} p{0.53\columnwidth} >{\itshape}l r @{\extracolsep{\fill}}}
Diane Marsh & C & 1758\\
Lyn Milner & Lab & 1538\\
Tina Brooker & UKIP & 1272\\
Martin Wilson & Grn & 209\\
Mark Marsh & LD & 110\\
\end{tabular*}

\subsubsection*{Swanley \hspace*{\fill}\nolinebreak[1]%
\enspace\hspace*{\fill}
\finalhyphendemerits=0
[13th October]}

\index{Swanley , Kent@Swanley, \emph{Kent}}

Death of Robert Brookbank (C).

\noindent
\begin{tabular*}{\columnwidth}{@{\extracolsep{\fill}} p{0.53\columnwidth} >{\itshape}l r @{\extracolsep{\fill}}}
Michael Horwood & C & 717\\
James Halford & UKIP & 615\\
Angela George & Lab & 518\\
Robert Woodbridge & LD & 362\\
\end{tabular*}

\subsection*{Ashford}
\index{Ashford}

\subsubsection*{Beaver \hspace*{\fill}\nolinebreak[1]%
\enspace\hspace*{\fill}
\finalhyphendemerits=0
[4th August; UKIP gain from Lab]}

\index{Beaver , Ashford@Beaver, \emph{Ashford}}

Resignation of Jill Britcher (Lab).

\noindent
\begin{tabular*}{\columnwidth}{@{\extracolsep{\fill}} p{0.53\columnwidth} >{\itshape}l r @{\extracolsep{\fill}}}
Ryan Macpherson & UKIP & 373\\
Caroline Harris & Lab & 243\\
Jo Gideon & C & 240\\
Elizabeth Wright & Grn & 31\\
\end{tabular*}

\subsection*{Canterbury}
\index{Canterbury}

\subsubsection*{Reculver \hspace*{\fill}\nolinebreak[1]%
\enspace\hspace*{\fill}
\finalhyphendemerits=0
[5th May]}

\index{Reculver , Canterbury@Reculver, \emph{Canterbury}}

Resignation of Guy Foster (C).

\noindent
\begin{tabular*}{\columnwidth}{@{\extracolsep{\fill}} p{0.53\columnwidth} >{\itshape}l r @{\extracolsep{\fill}}}
Ann Taylor & C & 421\\
Mick O'Brien & UKIP & 306\\
Simon Warley & Lab & 251\\
Ann Anderson & LD & 93\\
\end{tabular*}

\subsection*{Dover}
\index{Dover}

\subsubsection*{Aylesham \hspace*{\fill}\nolinebreak[1]%
\enspace\hspace*{\fill}
\finalhyphendemerits=0
[22nd December]}

\index{Aylesham , Dover@Aylesham, \emph{Dover}}

Resignation of Tommy Johnstone (Lab).

\noindent
\begin{tabular*}{\columnwidth}{@{\extracolsep{\fill}} p{0.53\columnwidth} >{\itshape}l r @{\extracolsep{\fill}}}
Gordon Cowan & Lab & 460\\
Pauline Catterall & C & 283\\
\end{tabular*}

\subsection*{Gravesham}
\index{Gravesham}

\subsubsection*{Pelham \hspace*{\fill}\nolinebreak[1]%
\enspace\hspace*{\fill}
\finalhyphendemerits=0
[18th August]}

\index{Pelham , Gravesham@Pelham, \emph{Gravesham}}

Death of Jane Cribbon (Lab).

\noindent
\begin{tabular*}{\columnwidth}{@{\extracolsep{\fill}} p{0.53\columnwidth} >{\itshape}l r @{\extracolsep{\fill}}}
Jenny Wallace & Lab & 494\\
Conrad Broadley & C & 325\\
Sharan Virk & LD & 101\\
Gary Harding & UKIP & 91\\
Marna Gilligan & Grn & 35\\
Emma Foreman & EDP & 24\\
\end{tabular*}

\subsection*{Maidstone}
\index{Maidstone}

\subsubsection*{Shepway South \hspace*{\fill}\nolinebreak[1]%
\enspace\hspace*{\fill}
\finalhyphendemerits=0
[8th September]}

\index{Shepway South , Maidstone@Shepway S., \emph{Maidstone}}

Death of Dave Sargeant (UKIP).

\noindent
\begin{tabular*}{\columnwidth}{@{\extracolsep{\fill}} p{0.53\columnwidth} >{\itshape}l r @{\extracolsep{\fill}}}
John Barned & UKIP & 432\\
Bob Hinder & C & 215\\
Dan Wilkinson & Lab & 183\\
Jon Hicks & Ind & 88\\
Milden Choongo & LD & 31\\
\end{tabular*}

\columnbreak

\subsection*{Medway}
\index{Medway}

\subsubsection*{Strood South \hspace*{\fill}\nolinebreak[1]%
\enspace\hspace*{\fill}
\finalhyphendemerits=0
[20th October; C gain from UKIP]}

\index{Strood South , Medway@Strood S., \emph{Medway}}

\sloppyword{Resignation of Catriona Brown-Reckless (UKIP).}

\noindent
\begin{tabular*}{\columnwidth}{@{\extracolsep{\fill}} p{0.53\columnwidth} >{\itshape}l r @{\extracolsep{\fill}}}
Josie Iles & C & 724\\
Isaac Igwe & Lab & 521\\
Karl Weller & UKIP & 480\\
Stephen Dyke & Grn & 74\\
Isabelle Cherry & LD & 62\\
Mike Russell & EDP & 23\\
\end{tabular*}

\subsubsection*{Rainham Central \hspace*{\fill}\nolinebreak[1]%
\enspace\hspace*{\fill}
\finalhyphendemerits=0
[3rd November]}

\index{Rainham Central , Medway@Rainham C., \emph{Medway}}

Death of Mike O'Brien (C).

\noindent
\begin{tabular*}{\columnwidth}{@{\extracolsep{\fill}} p{0.53\columnwidth} >{\itshape}l r @{\extracolsep{\fill}}}
Jan Aldous & C & 1448\\
Mark Mencattelli & UKIP & 389\\
Simon Allen & Lab & 320\\
Paul Chaplin & LD & 137\\
George Meegan & Grn & 61\\
Mike Russell & EDP & 14\\
\end{tabular*}

\subsection*{Sevenoaks}
\index{Sevenoaks}

\subsubsection*{Swanley Christchurch and Swanley Village \hspace*{\fill}\nolinebreak[1]%
\enspace\hspace*{\fill}
\finalhyphendemerits=0
[13th October]}

\index{Swanley Christchurch and Swanley Village , Sevenoaks@Swanley Christchurch \& Swanley Village, \emph{Sevenoaks}}

Death of Robert Brookbank (C).

\noindent
\begin{tabular*}{\columnwidth}{@{\extracolsep{\fill}} p{0.53\columnwidth} >{\itshape}l r @{\extracolsep{\fill}}}
Clare Barnes & C & 311\\
Jacqueline Griffiths & Lab & 274\\
Krish Shanmuganathan & LD & 183\\
Medina Hall & UKIP & 131\\
\end{tabular*}

\subsection*{Thanet}
\index{Thanet}

UThanet = Party for a United Thanet

\subsubsection*{Newington \hspace*{\fill}\nolinebreak[1]%
\enspace\hspace*{\fill}
\finalhyphendemerits=0
[21st January; Lab gain from UKIP]}

\index{Newington , Thanet@Newington, \emph{Thanet}}

Resignation of Vince Munday (UKIP).

\noindent
\begin{tabular*}{\columnwidth}{@{\extracolsep{\fill}} p{0.53\columnwidth} >{\itshape}l r @{\extracolsep{\fill}}}
Karen Constantine & Lab & 288\\
Duncan Smithson & UKIP & 229\\
Adam Dark & C & 156\\
Alan Hodder & Ind & 49\\
Ian Driver & Grn & 20\\
Jordan Williams & LD & 12\\
Grahame Birchall & Ind & 10\\
\end{tabular*}

\subsubsection*{Newington \hspace*{\fill}\nolinebreak[1]%
\enspace\hspace*{\fill}
\finalhyphendemerits=0
[Friday 1st July]}

\index{Newington , Thanet@Newington, \emph{Thanet}}

Resignation of Mo Leys (UKIP).

\noindent
\begin{tabular*}{\columnwidth}{@{\extracolsep{\fill}} p{0.53\columnwidth} >{\itshape}l r @{\extracolsep{\fill}}}
Roy Potts & UKIP & 295\\
David Green & Lab & 281\\
Adam Dark & C & 125\\
Matthew Brown & LD & 33\\
\end{tabular*}

\subsubsection*{Northwood (2) \hspace*{\fill}\nolinebreak[1]%
\enspace\hspace*{\fill}
\finalhyphendemerits=0
[18th August]}

\index{Northwood , Thanet@Northwood, \emph{Thanet}}

Resignations of Konnor Collins and Helen Smith (both Democratic Independent Group elected as UKIP).

\noindent
\begin{tabular*}{\columnwidth}{@{\extracolsep{\fill}} p{0.52\columnwidth} >{\itshape}l r @{\extracolsep{\fill}}}
Lynda Piper & UKIP & 488\\
George Rusiecki & UKIP & 394\\
Helen Crittenden & Lab & 356\\
Kaz Peet & Lab & 325\\
Charlie Leys & C & 282\\
Marc Rattigan & C & 246\\
Colin Grostate & Ind & 136\\
Jordan Williams & LD & 64\\
John Finnegan & LD & 48\\
Grahame Birchall & UThanet & 44\\
\end{tabular*}

\council{Tonbridge and Malling}

\subsubsection*{Trench \hspace*{\fill}\nolinebreak[1]%
\enspace\hspace*{\fill}
\finalhyphendemerits=0
[8th December]}

\index{Trench , Tonbridge and Malling@Trench, \emph{Tonbridge \& Malling}}

Death of Jean Atkinson (C).

\noindent
\begin{tabular*}{\columnwidth}{@{\extracolsep{\fill}} p{0.52\columnwidth} >{\itshape}l r @{\extracolsep{\fill}}}
Georgina Thomas & C & 603\\
Fred Long & Lab & 204\\
David Allen & UKIP & 178\\
\end{tabular*}

\section{Lancashire}

\subsection*{County Council}
\index{Lancashire}

\subsubsection*{Lancaster East \hspace*{\fill}\nolinebreak[1]%
\enspace\hspace*{\fill}
\finalhyphendemerits=0
[5th May]}

\index{Lancaster East , Lancashire@Lancaster E., \emph{Lancs.}}

\sloppyword{Death of Richard Newman-Thompson (Lab).}

\noindent
\begin{tabular*}{\columnwidth}{@{\extracolsep{\fill}} p{0.53\columnwidth} >{\itshape}l r @{\extracolsep{\fill}}}
Lizzi Collinge & Lab & 1758\\
Tim Hamilton-Cox & Grn & 1408\\
Robin Long & LD & 231\\
Steve Metcalfe & TUSC & 60\\
\end{tabular*}

\subsubsection*{Chorley Rural North \hspace*{\fill}\nolinebreak[1]%
\enspace\hspace*{\fill}
\finalhyphendemerits=0
[21st July]}

\index{Chorley Rural North , Lancashire@Chorley Rural N., \emph{Lancs.}}

Death of Mike Devaney (C).

\noindent
\begin{tabular*}{\columnwidth}{@{\extracolsep{\fill}} p{0.53\columnwidth} >{\itshape}l r @{\extracolsep{\fill}}}
Alan Cullens & C & 1144\\
Yvonne Hargreaves & Lab & 1042\\
Christopher Suart & UKIP & 303\\
Stephen Fenn & LD & 125\\
\end{tabular*}

\subsubsection*{Burnley Central East \hspace*{\fill}\nolinebreak[1]%
\enspace\hspace*{\fill}
\finalhyphendemerits=0
[3rd November]}

\index{Burnley Central East , Lancashire@Burnley C.E., \emph{Lancs.}}

Resignation of Misfar Hassan (Lab).

\noindent
\begin{tabular*}{\columnwidth}{@{\extracolsep{\fill}} p{0.53\columnwidth} >{\itshape}l r @{\extracolsep{\fill}}}
Sobia Malik & Lab & 1348\\
Emma Payne & LD & 276\\
Mark Girven & UKIP & 249\\
Laura Fisk & Grn & 84\\
\end{tabular*}

\subsection*{Blackburn with Darwen}
\index{Blackburn with Darwen}

\subsubsection*{Higher Croft \hspace*{\fill}\nolinebreak[1]%
\enspace\hspace*{\fill}
\finalhyphendemerits=0
[15th December]}

\index{Higher Croft , Blackburn with Darwen@Higher Croft, \emph{Blackburn with Darwen}}

Death of Mike Johnson (Lab).

\noindent
\begin{tabular*}{\columnwidth}{@{\extracolsep{\fill}} p{0.53\columnwidth} >{\itshape}l r @{\extracolsep{\fill}}}
Amy Johnson & Lab & 435\\
Ian Grimshaw & UKIP & 187\\
Maureen McGarvey & C & 125\\
\end{tabular*}

\subsection*{Blackpool}
\index{Blackpool}

\subsubsection*{Bloomfield \hspace*{\fill}\nolinebreak[1]%
\enspace\hspace*{\fill}
\finalhyphendemerits=0
[3rd March]}

\index{Bloomfield , Blackpool@Bloomfield, \emph{Blackpool}}

Resignation of John Jones (Lab).

\noindent
\begin{tabular*}{\columnwidth}{@{\extracolsep{\fill}} p{0.53\columnwidth} >{\itshape}l r @{\extracolsep{\fill}}}
Jim Hobson & Lab & 450\\
Tony Jones & C & 150\\
Spencer Shackleton & UKIP & 118\\
Phill Armstrong & Grn & 32\\
Neil Close & LD & 31\\
\end{tabular*}

\subsubsection*{Tyldesley \hspace*{\fill}\nolinebreak[1]%
\enspace\hspace*{\fill}
\finalhyphendemerits=0
[29th September]}

\index{Tyldesley , Blackpool@Tyldesley, \emph{Blackpool}}

Death of Eddie Collett (Lab).

\noindent
\begin{tabular*}{\columnwidth}{@{\extracolsep{\fill}} p{0.53\columnwidth} >{\itshape}l r @{\extracolsep{\fill}}}
David Collett & Lab & 535\\
Moira Graham & C & 297\\
Kim Knight & UKIP & 238\\
Paul Hindley & LD & 37\\
\end{tabular*}

\subsection*{Hyndburn}
\index{Hyndburn}

At the May 2016 ordinary election there was an unfilled vacancy in Church ward due to the resignation of Joan Smith (Lab).
\index{Church , Hyndburn@Church, \emph{Hyndburn}}



\subsection*{Lancaster}
\index{Lancaster}

MBI = Morecambe Bay Independent

\subsubsection*{Carnforth and Millhead \hspace*{\fill}\nolinebreak[1]%
\enspace\hspace*{\fill}
\finalhyphendemerits=0
[5th May; Lab gain from C]}

\index{Carnforth and Millhead , Lancaster@Carnforth \& Millhead, \emph{Lancaster}}

Death of George Askew (C).

\noindent
\begin{tabular*}{\columnwidth}{@{\extracolsep{\fill}} p{0.53\columnwidth} >{\itshape}l r @{\extracolsep{\fill}}}
John Reynolds & Lab & 702\\
John Bassinder & C & 671\\
Robert Gillespie & UKIP & 134\\
Phil Dunster & LD & 74\\
Cait Sinclair & Grn & 49\\
\end{tabular*}

\subsubsection*{John O'Gaunt \hspace*{\fill}\nolinebreak[1]%
\enspace\hspace*{\fill}
\finalhyphendemerits=0
[5th May]}

\index{John O'Gaunt , Lancaster@John O'Gaunt, \emph{Lancaster}}

\sloppyword{Death of Richard Newman-Thompson (Lab).}

\noindent
\begin{tabular*}{\columnwidth}{@{\extracolsep{\fill}} p{0.53\columnwidth} >{\itshape}l r @{\extracolsep{\fill}}}
Oscar Thynne & Lab & 857\\
Paul Stubbins & Grn & 650\\
Kieran Cooke & C & 221\\
Niall Semple & UKIP & 112\\
Bethany Frost & LD & 85\\
\end{tabular*}

\subsubsection*{Westgate \hspace*{\fill}\nolinebreak[1]%
\enspace\hspace*{\fill}
\finalhyphendemerits=0
[13th October]}

\index{Westgate , Lancaster@Westgate, \emph{Lancaster}}

Resignation of David Smith (Lab).

\noindent
\begin{tabular*}{\columnwidth}{@{\extracolsep{\fill}} p{0.53\columnwidth} >{\itshape}l r @{\extracolsep{\fill}}}
Ian Clift & Lab & 463\\
Roger Dennison & MBI & 193\\
Michelle Ogden & UKIP & 183\\
Daniel Gibbins & C & 178\\
Louise Stansfield & LD & 41\\
Richard Moriarty & Grn & 26\\
\end{tabular*}

\subsubsection*{University and Scotforth Rural \hspace*{\fill}\nolinebreak[1]%
\enspace\hspace*{\fill}
\finalhyphendemerits=0
[8th December]}

\index{University and Scotforth Rural , Lancaster@University \& Scotforth Rural, \emph{Lancaster}}

Resignation of Matt Mann (Lab).

\noindent
\begin{tabular*}{\columnwidth}{@{\extracolsep{\fill}} p{0.53\columnwidth} >{\itshape}l r @{\extracolsep{\fill}}}
Nathan Burns & Lab & 98\\
Xeina Aveyard & Grn & 79\\
Luke Brandon & C & 68\\
Pippa Hepworth & LD & 36\\
\end{tabular*}

\subsection*{Pendle}
\index{Pendle}

\subsubsection*{Barrowford \hspace*{\fill}\nolinebreak[1]%
\enspace\hspace*{\fill}
\finalhyphendemerits=0
[5th May]}

\index{Barrowford , Pendle@Barrowford, \emph{Pendle}}

Resignation of Christopher Jowett (C).

Combined with the 2016 ordinary election.
%; see page \pageref{BarrowfordPendle} for the result.

\subsubsection*{Craven \hspace*{\fill}\nolinebreak[1]%
\enspace\hspace*{\fill}
\finalhyphendemerits=0
[5th May]}

\index{Craven , Pendle@Craven, \emph{Pendle}}

Resignation of Richard Milner (LD).

Combined with the 2016 ordinary election.
%; see page \pageref{CravenPendle} for the result.

\subsubsection*{Reedley \hspace*{\fill}\nolinebreak[1]%
\enspace\hspace*{\fill}
\finalhyphendemerits=0
[24th November; C gain from Lab]}

\index{Reedley , Pendle@Reedley, \emph{Pendle}}

Resignation of Robert Allen (Lab).

\noindent
\begin{tabular*}{\columnwidth}{@{\extracolsep{\fill}} p{0.53\columnwidth} >{\itshape}l r @{\extracolsep{\fill}}}
Pauline McCormick & C & 1267\\
Mohammad Hanif & Lab & 1156\\
James Wood & LD & 57\\
\end{tabular*}

\subsection*{Rossendale}
\index{Rossendale}

At the May 2016 ordinary election there was an unfilled vacancy in Helmshore ward due to the resignation of Peter Evans (C).
\index{Helmshore , Rossendale@Helmshore, \emph{Rossendale}}

\subsection*{South Ribble}
\index{South Ribble}

\subsubsection*{Seven Stars \hspace*{\fill}\nolinebreak[1]%
\enspace\hspace*{\fill}
\finalhyphendemerits=0
[5th May]}

\index{Seven Stars , South Ribble@Seven Stars, \emph{S. Ribble}}

Death of Fred Heyworth (Lab).

\noindent
\begin{tabular*}{\columnwidth}{@{\extracolsep{\fill}} p{0.53\columnwidth} >{\itshape}l r @{\extracolsep{\fill}}}
Malcolm Donoghue & Lab & 625\\
Anthony Green & C & 405\\
James Pattison & LD & 102\\
\end{tabular*}

\section{Leicestershire}

\subsection*{Harborough}
\index{Harborough}

\subsubsection*{Misterton \hspace*{\fill}\nolinebreak[1]%
\enspace\hspace*{\fill}
\finalhyphendemerits=0
[17th November]}

\index{Misterton , Harborough@Misterton, \emph{Harborough}}

Resignation of John Everett (C).

\noindent
\begin{tabular*}{\columnwidth}{@{\extracolsep{\fill}} p{0.53\columnwidth} >{\itshape}l r @{\extracolsep{\fill}}}
Jonathan Bateman & C & 257\\
Liz Marsh & Lab & 119\\
Martin Sarfas & LD & 77\\
Bill Piper & UKIP & 57\\
\end{tabular*}

\subsection*{Melton}
\index{Melton}

\subsubsection*{Melton Egerton \hspace*{\fill}\nolinebreak[1]%
\enspace\hspace*{\fill}
\finalhyphendemerits=0
[5th May; Lab gain from C]}

\index{Melton Egerton , Melton@Melton Egerton, \emph{Melton}}

Resignation of Peter Faulkner (C).

\noindent
\begin{tabular*}{\columnwidth}{@{\extracolsep{\fill}} p{0.53\columnwidth} >{\itshape}l r @{\extracolsep{\fill}}}
Michael Blase & Lab & 255\\
Maddie Smith & C & 158\\
John Scutter & UKIP & 136\\
Marilyn Gordon & Ind & 70\\
Andrea Lovegrove & Ind & 40\\
\end{tabular*}

\council{North West Leicestershire}

\subsubsection*{Measham South \hspace*{\fill}\nolinebreak[1]%
\enspace\hspace*{\fill}
\finalhyphendemerits=0
[4th February]}

\index{Measham South , North West Leicestershire@Measham S., \emph{N.W. Leics.}}

Resignation of Tom Neilson (Lab).

\noindent
\begin{tabular*}{\columnwidth}{@{\extracolsep{\fill}} p{0.53\columnwidth} >{\itshape}l r @{\extracolsep{\fill}}}
Sean Sheahan & Lab & 257\\
Annette Bridge & C & 202\\
Martin Green & UKIP & 141\\
\end{tabular*}

\section{Lincolnshire}

\council{North East Lincolnshire}

At the May 2016 ordinary election there was an unfilled vacancy in East Marsh ward due to the resignation of Becci Bishell (UKIP).
\index{East Marsh , North East Lincolnshire@East Marsh, \emph{N.E. Lincs.}}

\subsubsection*{South \hspace*{\fill}\nolinebreak[1]%
\enspace\hspace*{\fill}
\finalhyphendemerits=0
[9th June]}

\index{South , North East Lincolnshire@South, \emph{N.E. Lincs.}}

Resignation of Chris Stanland (Lab).

\noindent
\begin{tabular*}{\columnwidth}{@{\extracolsep{\fill}} p{0.53\columnwidth} >{\itshape}l r @{\extracolsep{\fill}}}
Janet Goodwin & Lab & 758\\
Stephen Whittingham & UKIP & 462\\
Paul Batson & C & 312\\
Loyd Emmerson & Grn & 40\\
Val O'Flynn & TUSC & 26\\
\end{tabular*}



\subsection*{North Kesteven}
\index{North Kesteven}

LincsInd = Lincolnshire Independents

\subsubsection*{Ashby de la Launde and Cranwell \hspace*{\fill}\nolinebreak[1]%
\enspace\hspace*{\fill}
\finalhyphendemerits=0
[17th March; LincsInd gain from C]}

\index{Ashby de la Launde and Cranwell , North Kesteven@Ashby de la Launde \& Cranwell, \emph{N. Kesteven}}

Death of Geoffrey Whittle (C).

\noindent
\begin{tabular*}{\columnwidth}{@{\extracolsep{\fill}} p{0.52\columnwidth} >{\itshape}l r @{\extracolsep{\fill}}}
Steve Clegg & LincsInd & 457\\
Luke Mitchell & C & 296\\
Clare Newton & LD & 69\\
\end{tabular*}

\subsubsection*{Cliff Villages \hspace*{\fill}\nolinebreak[1]%
\enspace\hspace*{\fill}
\finalhyphendemerits=0
[13th October]}

\index{Cliff Villages , North Kesteven@Cliff Villages, \emph{N. Kesteven}}

Resignation of Laura Conway (LincsInd).

\noindent
\begin{tabular*}{\columnwidth}{@{\extracolsep{\fill}} p{0.52\columnwidth} >{\itshape}l r @{\extracolsep{\fill}}}
Cat Mills & LincsInd & 721\\
Daniel Gray & C & 372\\
Aarron Smith & LD & 49\\
\end{tabular*}

\subsection*{South Kesteven}
\index{South Kesteven}

\subsubsection*{Deeping St James \hspace*{\fill}\nolinebreak[1]%
\enspace\hspace*{\fill}
\finalhyphendemerits=0
[5th May]}

\index{Deeping Saint James , South Kesteven@Deeping St James, \emph{S. Kesteven}}

Resignation of Leigh Johnson (C).

\noindent
\begin{tabular*}{\columnwidth}{@{\extracolsep{\fill}} p{0.6\columnwidth} >{\itshape}l r @{\extracolsep{\fill}}}
Steve Benn & C & 784\\
Adam Brookes & LD & 436\\
Lisa Holmes & Lab & 286\\
Gerhard Lohmann-Bond & Grn & 70\\
\end{tabular*}

\subsection*{West Lindsey}
\index{West Lindsey}

\subsubsection*{Cherry Willingham \hspace*{\fill}\nolinebreak[1]%
\enspace\hspace*{\fill}
\finalhyphendemerits=0
[29th September]}

\index{Cherry Willingham , West Lindsey@Cherry Willingham, \emph{W. Lindsey}}

Resignation of Alexander Bridgwood (C).

\noindent
\begin{tabular*}{\columnwidth}{@{\extracolsep{\fill}} p{0.53\columnwidth} >{\itshape}l r @{\extracolsep{\fill}}}
Maureen Palmer & C & 555\\
Wendy Beckett & Lab & 288\\
Trevor Bridgwood & UKIP & 244\\
\end{tabular*}

\section{Norfolk}

\subsection*{Breckland}
\index{Breckland}

\subsubsection*{\sloppyword{Attleborough Queens and Besthorpe} \hspace*{\fill}\nolinebreak[1]%
\enspace\hspace*{\fill}
\finalhyphendemerits=0
[5th May]}

\index{Attleborough Queens and Besthorpe , Breckland@\sloppyword{Attleborough Queens \& Besthorpe, \emph{Breckland}}}

Resignation of Karen Pettitt (C).

\noindent
\begin{tabular*}{\columnwidth}{@{\extracolsep{\fill}} p{0.53\columnwidth} >{\itshape}l r @{\extracolsep{\fill}}}
Stephen Askew & C & 546\\
Phil Spiby & Lab & 393\\
Tony Crouch & Ind & 360\\
\end{tabular*}

\subsection*{Broadland}
\index{Broadland}

\subsubsection*{Aylsham \hspace*{\fill}\nolinebreak[1]%
\enspace\hspace*{\fill}
\finalhyphendemerits=0
[17th March; LD gain from C]}

\index{Aylsham , Broadland@Aylsham, \emph{Broadland}}

Resignation of Jo Cottingham (C).

\noindent
\begin{tabular*}{\columnwidth}{@{\extracolsep{\fill}} p{0.53\columnwidth} >{\itshape}l r @{\extracolsep{\fill}}}
Steve Riley & LD & 829\\
Hal Turkmen & C & 654\\
Christopher Jenner & Lab & 243\\
\end{tabular*}

\council{King's Lynn and West Norfolk}

\subsubsection*{Valley Hill \hspace*{\fill}\nolinebreak[1]%
\enspace\hspace*{\fill}
\finalhyphendemerits=0
[16th June; C gain from Ind]}

\index{Valley Hill , King's Lynn and West Norfolk@Valley Hill, \emph{King's Lynn \& W. Norfolk}}

Death of Mike Tilbury (Ind).

\noindent
\begin{tabular*}{\columnwidth}{@{\extracolsep{\fill}} p{0.53\columnwidth} >{\itshape}l r @{\extracolsep{\fill}}}
Tim Tilbrook & C & 266\\
Edward Robb & Lab & 157\\
Kate Sayer & LD & 102\\
Andrew Carr & UKIP & 96\\
Andrew de Whalley & Grn & 27\\
\end{tabular*}

\subsubsection*{Heacham \hspace*{\fill}\nolinebreak[1]%
\enspace\hspace*{\fill}
\finalhyphendemerits=0
[20th October; Ind gain from C]}

\index{Heacham , King's Lynn and West Norfolk@Heacham, \emph{King's Lynn \& W. Norfolk}}

Resignation of Peter Colvin (C).

\noindent
\begin{tabular*}{\columnwidth}{@{\extracolsep{\fill}} p{0.53\columnwidth} >{\itshape}l r @{\extracolsep{\fill}}}
Terry Parish & Ind & 400\\
Simon Eyre & C & 342\\
Rob Colwell & LD & 83\\
Debbie le May & UKIP & 83\\
Michael Press & Ind & 79\\
Edward Robb & Lab & 74\\
\end{tabular*}

\subsection*{North Norfolk}
\index{North Norfolk}

\subsubsection*{Astley \hspace*{\fill}\nolinebreak[1]%
\enspace\hspace*{\fill}
\finalhyphendemerits=0
[14th July; LD gain from C]}

\index{Astley , North Norfolk@Astley, \emph{N. Norfolk}}

Resignation of Steven Ward (C).

\noindent
\begin{tabular*}{\columnwidth}{@{\extracolsep{\fill}} p{0.53\columnwidth} >{\itshape}l r @{\extracolsep{\fill}}}
Pierre Butikofer & LD & 319\\
Jo Copplestone & C & 198\\
David Ramsbotham & UKIP & 133\\
Mandy Huntridge & Grn & 81\\
Callum Ringer & Lab & 51\\
\end{tabular*}

\subsubsection*{Glaven Valley \hspace*{\fill}\nolinebreak[1]%
\enspace\hspace*{\fill}
\finalhyphendemerits=0
[29th September]}

\index{Glaven Valley , North Norfolk@Glaven Valley, \emph{N. Norfolk}}

Resignation of Andrew Wells (LD).

\noindent
\begin{tabular*}{\columnwidth}{@{\extracolsep{\fill}} p{0.53\columnwidth} >{\itshape}l r @{\extracolsep{\fill}}}
Karen Ward & LD & 429\\
Andrew Livsey & C & 281\\
John Dymond & UKIP & 32\\
Stephen Burke & Lab & 23\\
Alicia Hull & Grn & 12\\
\end{tabular*}

\columnbreak

\section{North Yorkshire}

Yorks1st = Yorkshire First

\subsection*{County Council}
\index{North Yorkshire}

\subsubsection*{Northallerton \hspace*{\fill}\nolinebreak[1]%
\enspace\hspace*{\fill}
\finalhyphendemerits=0
[26th May]}

\index{Northallerton , North Yorkshire@Northallerton, \emph{N. Yorks.}}

Death of Tony Hall (C).

\noindent
\begin{tabular*}{\columnwidth}{@{\extracolsep{\fill}} p{0.53\columnwidth} >{\itshape}l r @{\extracolsep{\fill}}}
Caroline Dickinson & C & 654\\
Stephen Place & UKIP & 278\\
David Tickle & Lab & 233\\
Chris Pearson & Yorks1st & 131\\
Michael Chaloner & Grn & 58\\
\end{tabular*}

\subsection*{Craven}
\index{Craven}

\subsubsection*{Embsay-with-Eastby \hspace*{\fill}\nolinebreak[1]%
\enspace\hspace*{\fill}
\finalhyphendemerits=0
[31st March; Ind gain from C]}

\index{Embsay-with-Eastby , Craven@Embsay-with-Eastby, \emph{Craven}}

Death of Andy Quinn (C).

\noindent
\begin{tabular*}{\columnwidth}{@{\extracolsep{\fill}} p{0.53\columnwidth} >{\itshape}l r @{\extracolsep{\fill}}}
Brian Shuttleworth & Ind & 466\\
Trevor Kent & C & 117\\
\end{tabular*}

\subsection*{Hambleton}
\index{Hambleton}

\subsubsection*{Northallerton South \hspace*{\fill}\nolinebreak[1]%
\enspace\hspace*{\fill}
\finalhyphendemerits=0
[26th May]}

\index{Northallerton South , Hambleton@Northallerton S., \emph{Hambleton}}

Death of Tony Hall (C).

\noindent
\begin{tabular*}{\columnwidth}{@{\extracolsep{\fill}} p{0.53\columnwidth} >{\itshape}l r @{\extracolsep{\fill}}}
Caroline Dickinson & C & 541\\
David Tickle & Lab & 232\\
Dave Robertson & UKIP & 222\\
Chris Pearson & Yorks1st & 133\\
\end{tabular*}

\subsection*{Middlesbrough}
\index{Middlesbrough}

\subsubsection*{Coulby Newham \hspace*{\fill}\nolinebreak[1]%
\enspace\hspace*{\fill}
\finalhyphendemerits=0
[5th May]}

\index{Coulby Newham , Middlesbrough@Coulby Newham, \emph{Middlesbrough}}

Resignation of Joe Culley (Lab).

\noindent
\begin{tabular*}{\columnwidth}{@{\extracolsep{\fill}} p{0.53\columnwidth} >{\itshape}l r @{\extracolsep{\fill}}}
David Branson & Lab & 732\\
Alison Huggan & Ind & 475\\
Lewis Melvin & C & 352\\
Ian Jones & LD & 48\\
\end{tabular*}

\subsubsection*{Central \hspace*{\fill}\nolinebreak[1]%
\enspace\hspace*{\fill}
\finalhyphendemerits=0
[20th October]}

\index{Central , Middlesbrough@Central, \emph{Middlesbrough}}

Resignation of Ansab Shan (Lab).

\noindent
\begin{tabular*}{\columnwidth}{@{\extracolsep{\fill}} p{0.53\columnwidth} >{\itshape}l r @{\extracolsep{\fill}}}
Matthew Storey & Lab & 732\\
Dale Clark & Ind & 149\\
Ron Armstrong & C & 70\\
Elliott Sabin-Motson & LD & 53\\
\end{tabular*}

\council{Redcar and Cleveland}

NEast = North East Party

\subsubsection*{Hutton \hspace*{\fill}\nolinebreak[1]%
\enspace\hspace*{\fill}
\finalhyphendemerits=0
[17th March]}

\index{Hutton , Redcar and Cleveland@Hutton, \emph{Redcar \& Cleveland}}

Death of Peter Spencer (C).

\noindent
\begin{tabular*}{\columnwidth}{@{\extracolsep{\fill}} p{0.53\columnwidth} >{\itshape}l r @{\extracolsep{\fill}}}
Caroline Jackson & C & 879\\
Graeme Kidd & LD & 536\\
Ian Taylor & Lab & 368\\
Harry Lilleker & UKIP & 116\\
George Tinsley & Ind & 56\\
\end{tabular*}

\subsubsection*{Ormesby \hspace*{\fill}\nolinebreak[1]%
\enspace\hspace*{\fill}
\finalhyphendemerits=0
[18th August]}

\index{Ormesby , Redcar and Cleveland@Ormesby, \emph{Redcar \& Cleveland}}

Resignation of Ann Wilson (LD).

\noindent
\begin{tabular*}{\columnwidth}{@{\extracolsep{\fill}} p{0.51\columnwidth} >{\itshape}l r @{\extracolsep{\fill}}}
Carole Morgan & LD & 980\\
Ian Neil & UKIP & 138\\
Alison Suthers & Lab & 126\\
Cameron Brown & C & 41\\
Philip Lockey & NEast & 15\\
\end{tabular*}

\subsection*{Richmondshire}
\index{Richmondshire}

RIG = Richmondshire Independent Group

\subsubsection*{Catterick \hspace*{\fill}\nolinebreak[1]%
\enspace\hspace*{\fill}
\finalhyphendemerits=0
[25th February]}

\index{Catterick , Richmondshire@Catterick, \emph{Richmondshire}}

Death of Derek Sankey (C).

\noindent
\begin{tabular*}{\columnwidth}{@{\extracolsep{\fill}} p{0.53\columnwidth} >{\itshape}l r @{\extracolsep{\fill}}}
Simon Young & C & 308\\
Jill McMullon & Ind & 203\\
\end{tabular*}

\subsubsection*{Richmond Central \hspace*{\fill}\nolinebreak[1]%
\enspace\hspace*{\fill}
\finalhyphendemerits=0
[31st March; RIG gain from LD]}

\index{Richmond Central , Richmondshire@Richmond C., \emph{Richmondshire}}

Death of John Robinson (LD).

\noindent
\begin{tabular*}{\columnwidth}{@{\extracolsep{\fill}} p{0.53\columnwidth} >{\itshape}l r @{\extracolsep{\fill}}}
Lorraine Hodgson & RIG & 236\\
Philip Knowles & LD & 205\\
Nathalie Carter & C & 156\\
Anna Jackson & Grn & 77\\
\end{tabular*}

\subsubsection*{Catterick \hspace*{\fill}\nolinebreak[1]%
\enspace\hspace*{\fill}
\finalhyphendemerits=0
[18th August; C gain from Ind]}

\index{Catterick , Richmondshire@Catterick, \emph{Richmondshire}}

Death of Tony Pelton (Ind).

\noindent
\begin{tabular*}{\columnwidth}{@{\extracolsep{\fill}} p{0.53\columnwidth} >{\itshape}l r @{\extracolsep{\fill}}}
Stephen Wyrill & C & 228\\
David Coates & LD & 203\\
Jill McMullon & Ind & 112\\
Robbie Kelly & Grn & 3\\
\end{tabular*}

\subsection*{Selby}
\index{Selby}

\subsubsection*{Byram and Brotherton \hspace*{\fill}\nolinebreak[1]%
\enspace\hspace*{\fill}
\finalhyphendemerits=0
[14th July; C gain from Lab]}

\index{Byram and Brotherton , Selby@Byram \& Brotherton, \emph{Selby}}

Death of Jack Crawford (Lab).

\noindent
\begin{tabular*}{\columnwidth}{@{\extracolsep{\fill}} p{0.53\columnwidth} >{\itshape}l r @{\extracolsep{\fill}}}
Bryn Sage & C & 251\\
Steven Shaw-Wright & Lab & 224\\
Chris Whitwood & Yorks1st & 91\\
\end{tabular*}

\section{Northamptonshire}

\subsection*{Kettering}
\index{Kettering}

\subsubsection*{St Peter's \hspace*{\fill}\nolinebreak[1]%
\enspace\hspace*{\fill}
\finalhyphendemerits=0
[24th March]}

\index{Saint Peter's , Kettering@St Peter's, \emph{Kettering}}

Resignation of Terence Freer (C).

\noindent
\begin{tabular*}{\columnwidth}{@{\extracolsep{\fill}} p{0.53\columnwidth} >{\itshape}l r @{\extracolsep{\fill}}}
Ian Jelley & C & 468\\
Eugene Dalton-Ruark & Lab & 180\\
Kevin Sills & UKIP & 149\\
Kirsty Berry & Grn & 93\\
Mel Gosliga & LD & 28\\
\end{tabular*}

\subsubsection*{Rothwell \hspace*{\fill}\nolinebreak[1]%
\enspace\hspace*{\fill}
\finalhyphendemerits=0
[20th October; C gain from Lab]}

\index{Rothwell , Kettering@Rothwell, \emph{Kettering}}

Death of Alan Mills (Lab).

\noindent
\begin{tabular*}{\columnwidth}{@{\extracolsep{\fill}} p{0.53\columnwidth} >{\itshape}l r @{\extracolsep{\fill}}}
Cedwein Brown & C & 700\\
Kathleen Harris & Lab & 498\\
Sam Watts & UKIP & 108\\
Stevie Jones & Grn & 75\\
Malcolm Adcock & LD & 67\\
\end{tabular*}

\subsection*{Northampton}
\index{Northampton}

\subsubsection*{Westone \hspace*{\fill}\nolinebreak[1]%
\enspace\hspace*{\fill}
\finalhyphendemerits=0
[21st July; LD gain from C]}

\index{Westone , Northampton@Westone, \emph{Northampton}}

Resignation of Matthew Lynch (C).

\noindent
\begin{tabular*}{\columnwidth}{@{\extracolsep{\fill}} p{0.53\columnwidth} >{\itshape}l r @{\extracolsep{\fill}}}
Brian Markham & LD & 583\\
Greg Lunn & C & 319\\
Toby Birch & Lab & 270\\
\end{tabular*}

\council{South Northamptonshire}

\subsubsection*{Old Stratford \hspace*{\fill}\nolinebreak[1]%
\enspace\hspace*{\fill}
\finalhyphendemerits=0
[22nd September]}

\index{Old Stratford , South Northamptonshire@Old Stratford, \emph{S. Northants.}}

Resignation of Stephen Mold (C).

\noindent
\begin{tabular*}{\columnwidth}{@{\extracolsep{\fill}} p{0.53\columnwidth} >{\itshape}l r @{\extracolsep{\fill}}}
Ken Pritchard & C & 369\\
Rose Gibbins & UKIP & 109\\
\end{tabular*}

\subsubsection*{Grange Park \hspace*{\fill}\nolinebreak[1]%
\enspace\hspace*{\fill}
\finalhyphendemerits=0
[1st December]}

\index{Grange Park , South Northamptonshire@Grange Park, \emph{S. Northants.}}

Resignation of Simon Clifford (C).

\noindent
\begin{tabular*}{\columnwidth}{@{\extracolsep{\fill}} p{0.53\columnwidth} >{\itshape}l r @{\extracolsep{\fill}}}
Andrew Grant & C & 244\\
Ian Grant & Lab & 105\\
Rose Gibbins & UKIP & 49\\
Andy Clarke & Grn & 20\\
\end{tabular*}

\subsection*{Wellingborough}
\index{Wellingborough}

\subsubsection*{Finedon \hspace*{\fill}\nolinebreak[1]%
\enspace\hspace*{\fill}
\finalhyphendemerits=0
[29th September]}

\index{Finedon , Wellingborough@Finedon, \emph{Wellingborough}}

Death of John Bailey (C).

\noindent
\begin{tabular*}{\columnwidth}{@{\extracolsep{\fill}} p{0.53\columnwidth} >{\itshape}l r @{\extracolsep{\fill}}}
Barbara Bailey & C & 758\\
Steve Ayland & Lab & 235\\
Allan Shipham & UKIP & 137\\
John Wheaver & LD & 86\\
\end{tabular*}

\section{Northumberland}
\index{Northumberland}

\subsubsection*{Hexham West \hspace*{\fill}\nolinebreak[1]%
\enspace\hspace*{\fill}
\finalhyphendemerits=0
[4th February; Ind gain from C]}

\index{Hexham West , Northumberland@Hexham W., \emph{Northd}}

Resignation of Colin Cessford (C).

\noindent
\begin{tabular*}{\columnwidth}{@{\extracolsep{\fill}} p{0.53\columnwidth} >{\itshape}l r @{\extracolsep{\fill}}}
Derek Kennedy & Ind & 501\\
Tom Gillanders & C & 454\\
Nuala Rose & Lab & 200\\
Anne Pickering & Ind & 125\\
Lee Williscroft-Ferris & Grn & 89\\
\end{tabular*}

\section{Nottinghamshire}

\subsection*{County Council}
\index{Nottinghamshire}

\subsubsection*{Bingham \hspace*{\fill}\nolinebreak[1]%
\enspace\hspace*{\fill}
\finalhyphendemerits=0
[4th August]}

\index{Bingham , Nottinghamshire@Bingham, \emph{Notts.}}

Death of Martin Suthers (C).

\noindent
\begin{tabular*}{\columnwidth}{@{\extracolsep{\fill}} p{0.53\columnwidth} >{\itshape}l r @{\extracolsep{\fill}}}
Francis Purdue-Horan & C & 1270\\
Tracey Kerry & Ind & 1232\\
Alan Walker & Lab & 382\\
\end{tabular*}

\subsection*{Broxtowe}
\index{Broxtowe}

\subsubsection*{Greasley \hspace*{\fill}\nolinebreak[1]%
\enspace\hspace*{\fill}
\finalhyphendemerits=0
[18th February]}

\index{Greasley , Broxtowe@Greasley, \emph{Broxtowe}}

Death of Stuart Rowland (C).

\noindent
\begin{tabular*}{\columnwidth}{@{\extracolsep{\fill}} p{0.53\columnwidth} >{\itshape}l r @{\extracolsep{\fill}}}
Eddie Cubley & C & 656\\
Chris Chandler & Lab & 300\\
Tracey Cahill & UKIP & 230\\
Keith Longdon & LD & 158\\
\end{tabular*}

\subsubsection*{Toton and Chilwell Meadows \hspace*{\fill}\nolinebreak[1]%
\enspace\hspace*{\fill}
\finalhyphendemerits=0
[18th February]}

\index{Toton and Chilwell Meadows , Broxtowe@Toton \& Chilwell Meadows, \emph{Broxtowe}}

Resignation of Natalie Harvey (C).

\noindent
\begin{tabular*}{\columnwidth}{@{\extracolsep{\fill}} p{0.53\columnwidth} >{\itshape}l r @{\extracolsep{\fill}}}
Stephanie Kerry & C & 910\\
Lisa Clarke & Lab & 368\\
Graham Heal & LD & 363\\
Gordon Stoner & Grn & 111\\
\end{tabular*}

\subsection*{Mansfield}
\index{Mansfield}

MIF = Mansfield Independent Forum

\subsubsection*{Yeoman Hill \hspace*{\fill}\nolinebreak[1]%
\enspace\hspace*{\fill}
\finalhyphendemerits=0
[8th September]}

\index{Yeoman Hill , Mansfield@Yeoman Hill, \emph{Mansfield}}

Death of Lee Probert (Lab).

\noindent
\begin{tabular*}{\columnwidth}{@{\extracolsep{\fill}} p{0.53\columnwidth} >{\itshape}l r @{\extracolsep{\fill}}}
John Coxhead & Lab & 278\\
Neil Williams & MIF & 148\\
David Hamilton & UKIP & 105\\
Daniel Redfern & C & 41\\
Philip Shields & Ind & 36\\
\end{tabular*}

\subsubsection*{Warsop Carrs \hspace*{\fill}\nolinebreak[1]%
\enspace\hspace*{\fill}
\finalhyphendemerits=0
[24th November]}

\index{Warsop Carrs , Mansfield@Warsop Carrs, \emph{Mansfield}}

Death of Peter Crawford (Lab).

\noindent
\begin{tabular*}{\columnwidth}{@{\extracolsep{\fill}} p{0.53\columnwidth} >{\itshape}l r @{\extracolsep{\fill}}}
Andew Burgin & Lab & 285\\
Debra Barlow & Ind & 211\\
Raymond Forster & UKIP & 74\\
Daniel Redfern & C & 25\\
\end{tabular*}

\council{Newark and Sherwood}

\subsubsection*{Balderton South \hspace*{\fill}\nolinebreak[1]%
\enspace\hspace*{\fill}
\finalhyphendemerits=0
[21st July]}

\index{Balderton South , Newark and Sherwood@Balderton S., \emph{Newark \& Sherwood}}

Death of Gordon Brooks (C).

\noindent
\begin{tabular*}{\columnwidth}{@{\extracolsep{\fill}} p{0.53\columnwidth} >{\itshape}l r @{\extracolsep{\fill}}}
Lydia Hurst & C & 483\\
Marylyn Rayner & LD & 103\\
\end{tabular*}

\subsection*{Rushcliffe}
\index{Rushcliffe}

\subsubsection*{Cranmer \hspace*{\fill}\nolinebreak[1]%
\enspace\hspace*{\fill}
\finalhyphendemerits=0
[4th August]}

\index{Cranmer , Rushcliffe@Cranmer, \emph{Rushcliffe}}

Death of Martin Suthers (C).

\noindent
\begin{tabular*}{\columnwidth}{@{\extracolsep{\fill}} p{0.53\columnwidth} >{\itshape}l r @{\extracolsep{\fill}}}
Maureen Stockwood & C & 318\\
Tracey Kerry & Ind & 138\\
Chris Grocock & Lab & 130\\
\end{tabular*}

\section{Oxfordshire}

\subsection*{Cherwell}
\index{Cherwell}

\subsubsection*{Adderbury, Bloxham and Bodicote \hspace*{\fill}\nolinebreak[1]%
\enspace\hspace*{\fill}
\finalhyphendemerits=0
[22nd September]}

\index{Adderbury, Bloxham and Bodicote , Cherwell@Adderbury, Bloxham \& Bodicote, \emph{Cherwell}}

Resignation of Nigel Randall (C).

\noindent
\begin{tabular*}{\columnwidth}{@{\extracolsep{\fill}} p{0.53\columnwidth} >{\itshape}l r @{\extracolsep{\fill}}}
Andrew McHugh & C & 1015\\
Sue Christie & Lab & 286\\
Naomi Kanetsuka & Grn & 278\\
Ian Thomas & LD & 189\\
\end{tabular*}

\section{Rutland}
\index{Rutland}

\subsubsection*{Whissendine \hspace*{\fill}\nolinebreak[1]%
\enspace\hspace*{\fill}
\finalhyphendemerits=0
[3rd March]}

\index{Whissendine , Rutland@Whissendine, \emph{Rutland}}

Resignation of Sam Asplin (LD).

\noindent
\begin{tabular*}{\columnwidth}{@{\extracolsep{\fill}} p{0.53\columnwidth} >{\itshape}l r @{\extracolsep{\fill}}}
Kevin Thomas & LD & 265\\
Christopher Clark & C & 109\\
Marietta King & UKIP & 33\\
\end{tabular*}

\subsubsection*{Greetham \hspace*{\fill}\nolinebreak[1]%
\enspace\hspace*{\fill}
\finalhyphendemerits=0
[5th May]}

\index{Greetham , Rutland@Greetham, \emph{Rutland}}

Death of Roger Begy (C).

\noindent
\begin{tabular*}{\columnwidth}{@{\extracolsep{\fill}} p{0.53\columnwidth} >{\itshape}l r @{\extracolsep{\fill}}}
Nick Begy & C & \emph{unop.}\\
\end{tabular*}

\section{Shropshire}

\subsection*{Shropshire}
\index{Shropshire}

\subsubsection*{Oswestry South \hspace*{\fill}\nolinebreak[1]%
\enspace\hspace*{\fill}
\finalhyphendemerits=0
[4th February; Grn gain from C]}

\index{Oswestry South , Shropshire@Oswestry S., \emph{Shrops.}}

Resignation of Keith Barrow (C).

\noindent
\begin{tabular*}{\columnwidth}{@{\extracolsep{\fill}} p{0.53\columnwidth} >{\itshape}l r @{\extracolsep{\fill}}}
Duncan Kerr & Grn & 518\\
Christopher Schofield & C & 367\\
Carl Hopley & Lab & 95\\
Amanda Woof & LD & 81\\
\end{tabular*}

\subsubsection*{Bishop's Castle \hspace*{\fill}\nolinebreak[1]%
\enspace\hspace*{\fill}
\finalhyphendemerits=0
[15th September]}

\index{Bishop's Castle , Shropshire@Bishop's Castle, \emph{Shrops.}}

Resignation of Charlotte Barnes (LD).

\noindent
\begin{tabular*}{\columnwidth}{@{\extracolsep{\fill}} p{0.53\columnwidth} >{\itshape}l r @{\extracolsep{\fill}}}
Jonny Keeley & LD & 862\\
Georgie Ellis & C & 430\\
Judith Payne & Lab & 95\\
Steve Hale & Grn & 37\\
\end{tabular*}

\council{Telford and Wrekin}

\subsubsection*{Horsehay and Lightmoor \hspace*{\fill}\nolinebreak[1]%
\enspace\hspace*{\fill}
\finalhyphendemerits=0
[8th December; Lab gain from C]}

\index{Horsehay and Lightmoor , Telford and Wrekin@Horsehay \& Lightmoor, \emph{Telford \& Wrekin}}

Death of Clive Mollett (C).

\noindent
\begin{tabular*}{\columnwidth}{@{\extracolsep{\fill}} p{0.53\columnwidth} >{\itshape}l r @{\extracolsep{\fill}}}
Rajash Mehta & Lab & 358\\
Robert Cadman & C & 292\\
Denis Allen & UKIP & 124\\
\end{tabular*}

\section{Somerset}

\council{Bath and North East Somerset}

\subsubsection*{Abbey \hspace*{\fill}\nolinebreak[1]%
\enspace\hspace*{\fill}
\finalhyphendemerits=0
[17th November; C gain from Grn]}

\index{Abbey , Bath and North East Somerset@Abbey, \emph{Bath \& N.E. Somerset}}

Resignation of Jonathan Carr (Grn).

\noindent
\begin{tabular*}{\columnwidth}{@{\extracolsep{\fill}} p{0.53\columnwidth} >{\itshape}l r @{\extracolsep{\fill}}}
Lizzie Gladwyn & C & 350\\
Gerry Curran & LD & 273\\
Vipul Patel & Grn & 252\\
Vicky Drew & Lab & 126\\
Jenny Knight & Ind & 43\\
Marc Hooper & UKIP & 23\\
\end{tabular*}

\subsection*{South Somerset}
\index{South Somerset}

\subsubsection*{Turn Hill \hspace*{\fill}\nolinebreak[1]%
\enspace\hspace*{\fill}
\finalhyphendemerits=0
[24th November]}

\index{Turn Hill , South Somerset@Turn Hill, \emph{S. Somerset}}

Resignation of Shane Pledger (C).

\noindent
\begin{tabular*}{\columnwidth}{@{\extracolsep{\fill}} p{0.53\columnwidth} >{\itshape}l r @{\extracolsep{\fill}}}
Gerard Tucker & C & 452\\
Julia Gadd & LD & 354\\
Sean Dromgoole & Lab & 74\\
\end{tabular*}

\subsection*{Taunton Deane}
\index{Taunton Deane}

\subsubsection*{Taunton Halcon \hspace*{\fill}\nolinebreak[1]%
\enspace\hspace*{\fill}
\finalhyphendemerits=0
[14th April]}

\index{Taunton Halcon , Taunton Deane@Taunton Halcon, \emph{Taunton Deane}}

Resignation of Christopher Appleby (LD).

\noindent
\begin{tabular*}{\columnwidth}{@{\extracolsep{\fill}} p{0.53\columnwidth} >{\itshape}l r @{\extracolsep{\fill}}}
Chris Booth & LD & 389\\
Livvi Grant & C & 222\\
Kieran Canham & Lab & 133\\
Robert Bainbridge & UKIP & 118\\
Craig Rossiter & Grn & 42\\
\end{tabular*}

\subsubsection*{Blackdown \hspace*{\fill}\nolinebreak[1]%
\enspace\hspace*{\fill}
\finalhyphendemerits=0
[15th December; LD gain from C]}

\index{Blackdown , Taunton Deane@Blackdown, \emph{Taunton Deane}}

Resignation of Charlotte Edwards (C).

\noindent
\begin{tabular*}{\columnwidth}{@{\extracolsep{\fill}} p{0.53\columnwidth} >{\itshape}l r @{\extracolsep{\fill}}}
Ross Henley & LD & 440\\
Giuseppe Fraschini & C & 139\\
Carl Benneyworth & Ind & 39\\
\end{tabular*}



\section{Staffordshire}

\subsection*{County Council}
\index{Staffordshire}

\subsubsection*{Uttoxeter Town \hspace*{\fill}\nolinebreak[1]%
\enspace\hspace*{\fill}
\finalhyphendemerits=0
[5th May]}

\index{Uttoxeter Town , Staffordshire@Uttoxeter Town, \emph{Staffs.}}

Death of Geoff Morrison (C).

\noindent
\begin{tabular*}{\columnwidth}{@{\extracolsep{\fill}} p{0.53\columnwidth} >{\itshape}l r @{\extracolsep{\fill}}}
David Brookes & C & 1247\\
John McKiernan & Lab & 967\\
Julian Lee & UKIP & 713\\
\end{tabular*}

\subsection*{Cannock Chase}
\index{Cannock Chase}

At the May 2016 ordinary election there was an unfilled vacancy in Cannock West ward due to the resignation of Chris Anslow (C).
\index{Cannock West , Cannock Chase@Cannock W., \emph{Cannock Chase}}

\council{East Staffordshire}

\subsubsection*{Stapenhill \hspace*{\fill}\nolinebreak[1]%
\enspace\hspace*{\fill}
\finalhyphendemerits=0
[26th May; Lab gain from UKIP]}

\index{Stapenhill , East Staffordshire@Stapenhill, \emph{E. Staffs.}}

Resignation of Steven Dyche (UKIP).

\noindent
\begin{tabular*}{\columnwidth}{@{\extracolsep{\fill}} p{0.53\columnwidth} >{\itshape}l r @{\extracolsep{\fill}}}
Craig Jones & Lab & 536\\
Sally Green & UKIP & 348\\
Michael Teasel & C & 208\\
Susan Paxton & Ind & 75\\
Thomas Hadley & Grn & 24\\
Hugh Warner & LD & 18\\
\end{tabular*}

\subsection*{Lichfield}
\index{Lichfield}

\subsubsection*{Chadsmead \hspace*{\fill}\nolinebreak[1]%
\enspace\hspace*{\fill}
\finalhyphendemerits=0
[18th February]}

\index{Chadsmead , Lichfield@Chadsmead, \emph{Lichfield}}

Resignation of Marion Bland (LD).

\noindent
\begin{tabular*}{\columnwidth}{@{\extracolsep{\fill}} p{0.53\columnwidth} >{\itshape}l r @{\extracolsep{\fill}}}
Paul Ray & LD & 300\\
Colin Ball & Lab & 195\\
Brian McMullan & C & 159\\
Jan Higgins & UKIP & 73\\
Adam Elsdon & Grn & 23\\
\end{tabular*}

\subsection*{Newcastle-under-Lyme}
\index{Newcastle-under-Lyme}

At the May 2016 ordinary election there was an unfilled vacancy in Town ward due to the resignation of Robert Wallace (Lab).
\index{Town , Newcastle-under-Lyme@Town, \emph{Newcastle-under-Lyme}}

\subsubsection*{Silverdale and Parksite \hspace*{\fill}\nolinebreak[1]%
\enspace\hspace*{\fill}
\finalhyphendemerits=0
[4th August; Lab gain from UKIP]}

\index{Silverdale and Parksite , Newcastle-under-Lyme@Silverdale \& Parksite, \emph{Newcastle-under-Lyme}}

Death of Eileen Braithwaite (Ind elected as UKIP).

\noindent
\begin{tabular*}{\columnwidth}{@{\extracolsep{\fill}} p{0.53\columnwidth} >{\itshape}l r @{\extracolsep{\fill}}}
Gareth Snell & Lab & 399\\
Lynn Dean & UKIP & 174\\
James Vernon & C & 80\\
Gary White & Ind & 54\\
\end{tabular*}

\subsubsection*{Madeley \hspace*{\fill}\nolinebreak[1]%
\enspace\hspace*{\fill}
\finalhyphendemerits=0
[8th December; Ind gain from Lab]}

\index{Madeley , Newcastle-under-Lyme@\sloppyword{Madeley, \emph{Newcastle-under-Lyme}}}

Death of Billy Welsh (Ind elected as Lab).

\noindent
\begin{tabular*}{\columnwidth}{@{\extracolsep{\fill}} p{0.53\columnwidth} >{\itshape}l r @{\extracolsep{\fill}}}
Gary White & Ind & 458\\
David Whitmore & C & 112\\
Peter Andras & LD & 75\\
Stephen French & Lab & 62\\
\end{tabular*}

\council{South Staffordshire}

\subsubsection*{Great Wyrley Town \hspace*{\fill}\nolinebreak[1]%
\enspace\hspace*{\fill}
\finalhyphendemerits=0
[21st July]}

\index{Great Wyrley Town , South Staffordshire@Great Wyrley Town, \emph{S. Staffs.}}

Resignation of Brian Bates (C).

\noindent
\begin{tabular*}{\columnwidth}{@{\extracolsep{\fill}} p{0.53\columnwidth} >{\itshape}l r @{\extracolsep{\fill}}}
Mike Lawrence & C & 357\\
George Bullock & Lab & 230\\
Malcolm McKenzie & UKIP & 114\\
\end{tabular*}

\section{Suffolk}

WSuffolk = West Suffolk Independent

\subsection*{County Council}
\index{Suffolk}

\subsubsection*{Newmarket and Red Lodge \hspace*{\fill}\nolinebreak[1]%
\enspace\hspace*{\fill}
\finalhyphendemerits=0
[18th February]}

\index{Newmarket and Red Lodge , Suffolk@Newmarket \& Red Lodge, \emph{Suffolk}}

Resignation of Lisa Chambers (C).

\noindent
\begin{tabular*}{\columnwidth}{@{\extracolsep{\fill}} p{0.51\columnwidth} >{\itshape}l r @{\extracolsep{\fill}}}
Robin Millar & C & 644\\
Roger Dicker & UKIP & 494\\
Michael Jefferys & Lab & 284\\
Andrew Appleby & WSuffolk & 123\\
Tim Huggan & LD & 76\\
\end{tabular*}

\subsubsection*{Bixley \hspace*{\fill}\nolinebreak[1]%
\enspace\hspace*{\fill}
\finalhyphendemerits=0
[5th May]}

\index{Bixley , Suffolk@Bixley, \emph{Suffolk}}

Resignation of Alan Murray (C).

\noindent
\begin{tabular*}{\columnwidth}{@{\extracolsep{\fill}} p{0.53\columnwidth} >{\itshape}l r @{\extracolsep{\fill}}}
Paul West & C & 1117\\
Rob Bridgeman & Lab & 634\\
Tony Gould & UKIP & 344\\
Colin Boyd & LD & 154\\
\end{tabular*}

\subsubsection*{Haverhill Cangle \hspace*{\fill}\nolinebreak[1]%
\enspace\hspace*{\fill}
\finalhyphendemerits=0
[5th May; UKIP gain from C]}

\index{Haverhill Cangle , Suffolk@Haverhill Cangle, \emph{Suffolk}}

Death of Tim Marks (C).

\noindent
\begin{tabular*}{\columnwidth}{@{\extracolsep{\fill}} p{0.53\columnwidth} >{\itshape}l r @{\extracolsep{\fill}}}
John Burns & UKIP & 1273\\
Margaret Marks & C & 1168\\
David Smith & Lab & 838\\
Kenneth Rolph & LD & 178\\
\end{tabular*}

\subsubsection*{Carlford \hspace*{\fill}\nolinebreak[1]%
\enspace\hspace*{\fill}
\finalhyphendemerits=0
[7th July]}

\index{Carlford , Suffolk@Carlford, \emph{Suffolk}}

Death of Peter Bellfield (C).

\noindent
\begin{tabular*}{\columnwidth}{@{\extracolsep{\fill}} p{0.53\columnwidth} >{\itshape}l r @{\extracolsep{\fill}}}
Robin Vickery & C & 1142\\
Graham Hedger & Lab & 344\\
Jon Neal & LD & 228\\
Jacqueline Barrow & Grn & 176\\
\end{tabular*}

\subsubsection*{Hadleigh \hspace*{\fill}\nolinebreak[1]%
\enspace\hspace*{\fill}
\finalhyphendemerits=0
[22nd September; LD gain from C]}

\index{Hadleigh , Suffolk@Hadleigh, \emph{Suffolk}}

Resignation of Brian Riley (C).

\noindent
\begin{tabular*}{\columnwidth}{@{\extracolsep{\fill}} p{0.53\columnwidth} >{\itshape}l r @{\extracolsep{\fill}}}
Trevor Sheldrick & LD & 642\\
Kathryn Grandon & C & 460\\
Sue Monks & Lab & 397\\
Stephen Laing & UKIP & 204\\
Lisa Gordon & Grn & 70\\
\end{tabular*}

\subsection*{Forest Heath}
\index{Forest Heath}

\subsubsection*{Brandon West \hspace*{\fill}\nolinebreak[1]%
\enspace\hspace*{\fill}
\finalhyphendemerits=0
[5th May; WSuffolk gain from C]}

\index{Brandon West , Forest Heath@Brandon W., \emph{Forest Heath}}

Resignation of David Bimson (C).

\noindent
\begin{tabular*}{\columnwidth}{@{\extracolsep{\fill}} p{0.51\columnwidth} >{\itshape}l r @{\extracolsep{\fill}}}
Victor Lukaniuk & WSuffolk & 220\\
Ian Smith & UKIP & 197\\
Edward Stewart & Ind & 180\\
Anthony Simmons & C & 161\\
\end{tabular*}

\subsubsection*{South \hspace*{\fill}\nolinebreak[1]%
\enspace\hspace*{\fill}
\finalhyphendemerits=0
[5th May; UKIP gain from C]}

\index{South , Forest Heath@South, \emph{Forest Heath}}

Resignation of James Lay (C).

\noindent
\begin{tabular*}{\columnwidth}{@{\extracolsep{\fill}} p{0.53\columnwidth} >{\itshape}l r @{\extracolsep{\fill}}}
Roger Dicker & UKIP & 334\\
Karen Soons & C & 309\\
\end{tabular*}

\subsection*{Ipswich}
\index{Ipswich}

\subsubsection*{Castle Hill \hspace*{\fill}\nolinebreak[1]%
\enspace\hspace*{\fill}
\finalhyphendemerits=0
[5th May]}

\index{Castle Hill , Ipswich@Castle Hill, \emph{Ipswich}}

Resignation of Chris Stewart (C).

Combined with the 2016 ordinary election.
%; see page \pageref{CastleHillIpswich} for the result.

\subsection*{Mid Suffolk}
\index{Mid Suffolk}

\subsubsection*{Barking and Somersham \hspace*{\fill}\nolinebreak[1]%
\enspace\hspace*{\fill}
\finalhyphendemerits=0
[2nd June; Grn gain from C]}

\index{Barking and Somersham , Mid Suffolk@Barking \& Somersham, \emph{Mid Suffolk}}

Resignation of David Card (C).

\noindent
\begin{tabular*}{\columnwidth}{@{\extracolsep{\fill}} p{0.53\columnwidth} >{\itshape}l r @{\extracolsep{\fill}}}
Anne Killett & Grn & 212\\
Jemma Lynch & C & 210\\
William Marsburg & Lab & 154\\
Mark Valladares & LD & 38\\
\end{tabular*}

\subsection*{St Edmundsbury}
\index{Saint Edmundsbury@St Edmundsbury}

\subsubsection*{Haverhill North \hspace*{\fill}\nolinebreak[1]%
\enspace\hspace*{\fill}
\finalhyphendemerits=0
[5th May; UKIP gain from C]}

\index{Haverhill North , Saint Edmundsbury@\sloppyword{Haverhill N., \emph{St Edmundsbury}}}

Death of Tim Marks (C).

\noindent
\begin{tabular*}{\columnwidth}{@{\extracolsep{\fill}} p{0.53\columnwidth} >{\itshape}l r @{\extracolsep{\fill}}}
Anthony Williams & UKIP & 563\\
Maureen Byrne & Lab & 460\\
Quillon Fox & C & 409\\
Kenneth Rolph & LD & 78\\
\end{tabular*}

\subsubsection*{Bardwell \hspace*{\fill}\nolinebreak[1]%
\enspace\hspace*{\fill}
\finalhyphendemerits=0
[24th November]}

\index{Bardwell , Saint Edmundsbury@Bardwell, \emph{St Edmundsbury}}

Resignation of Paula Wade (C).

\noindent
\begin{tabular*}{\columnwidth}{@{\extracolsep{\fill}} p{0.53\columnwidth} >{\itshape}l r @{\extracolsep{\fill}}}
Andrew Smith & C & \emph{unop.}\\
\end{tabular*}

\subsubsection*{Moreton Hall \hspace*{\fill}\nolinebreak[1]%
\enspace\hspace*{\fill}
\finalhyphendemerits=0
[15th December; Ind gain from C]}

\index{Moreton Hall , Saint Edmundsbury@Moreton Hall, \emph{St Edmundsbury}}

Resignation of Terry Buckle (C).

\noindent
\begin{tabular*}{\columnwidth}{@{\extracolsep{\fill}} p{0.53\columnwidth} >{\itshape}l r @{\extracolsep{\fill}}}
Trevor Beckwith & Ind & 550\\
Sue Bill & C & 213\\
Chris Lale & LD & 102\\
Alex Griffin & Lab & 71\\
Julian Flood & UKIP & 47\\
\end{tabular*}

\subsection*{Waveney}
\index{Waveney}

\subsubsection*{Wrentham \hspace*{\fill}\nolinebreak[1]%
\enspace\hspace*{\fill}
\finalhyphendemerits=0
[5th May]}

\index{Wrentham , Waveney@Wrentham, \emph{Waveney}}

Resignation of Martin Parsons (C).

\noindent
\begin{tabular*}{\columnwidth}{@{\extracolsep{\fill}} p{0.545\columnwidth} >{\itshape}l r @{\extracolsep{\fill}}}
Craig Rivett & C & 335\\
Paul Tyack & Lab & 252\\
Andrew Bols & UKIP & 156\\
Chris Thomas & LD & 46\\
\sloppyword{David Brambley-Crawshaw} & Grn & 40\\
\end{tabular*}



\section{Surrey}

Farnham = Farnham Residents

\subsection*{County Council}
\index{Surrey}

\subsubsection*{Staines South and Ashford West \hspace*{\fill}\nolinebreak[1]%
\enspace\hspace*{\fill}
\finalhyphendemerits=0
[5th May; C gain from UKIP]}

\index{Staines South and Ashford West , Surrey@Staines S. \& Ashford W., \emph{Surrey}}

Resignation of Daniel Jenkins (UKIP).

\noindent
\begin{tabular*}{\columnwidth}{@{\extracolsep{\fill}} p{0.53\columnwidth} >{\itshape}l r @{\extracolsep{\fill}}}
Denise Turner-Stewart & C & 1585\\
Peter Appleford & UKIP & 695\\
Iain Raymond & Lab & 543\\
Christopher Bateson & LD & 382\\
Andrew McLuskey & Grn & 145\\
Matthew Clarke & TUSC & 33\\
\end{tabular*}

\subsubsection*{Farnham South \hspace*{\fill}\nolinebreak[1]%
\enspace\hspace*{\fill}
\finalhyphendemerits=0
[18th August]}

\index{Farnham South , Surrey@Farnham S., \emph{Surrey}}

Resignation of David Munro (C).

\noindent
\begin{tabular*}{\columnwidth}{@{\extracolsep{\fill}} p{0.51\columnwidth} >{\itshape}l r @{\extracolsep{\fill}}}
Robert Ramsdale & C & 932\\
Jerry Hyman & Farnham & 754\\
Joanne Aylwin & LD & 269\\
Mark Westcott & Ind & 139\\
Paul Chapman & UKIP & 89\\
Fabian Wood & Lab & 77\\
\end{tabular*}

\subsection*{Elmbridge}
\index{Elmbridge}

At the May 2016 ordinary election there was an unfilled vacancy in Walton South ward due to the death of Stuart Hawkins (C).
\index{Walton South , Elmbridge@Walton S., \emph{Elmbridge}}

\subsection*{Guildford}
\index{Guildford}

\subsubsection*{Stoke \hspace*{\fill}\nolinebreak[1]%
\enspace\hspace*{\fill}
\finalhyphendemerits=0
[5th May; Lab gain from C]}

\index{Stoke , Guildford@Stoke, \emph{Guildford}}

Resignation of Will Chesterfield (C).

\noindent
\begin{tabular*}{\columnwidth}{@{\extracolsep{\fill}} p{0.53\columnwidth} >{\itshape}l r @{\extracolsep{\fill}}}
James Walsh & Lab & 528\\
Barry Keane & C & 497\\
Hannah Thompson & LD & 492\\
\end{tabular*}

\columnbreak

\subsection*{Mole Valley}
\index{Mole Valley}

Ashtead = Ashtead Independent

\subsubsection*{Ashtead Common \hspace*{\fill}\nolinebreak[1]%
\enspace\hspace*{\fill}
\finalhyphendemerits=0
[5th May]}

\index{Ashtead Common , Mole Valley@Ashtead Common, \emph{Mole Valley}}

Resignation of John Northcott (Ashtead).

Combined with the 2016 ordinary election.
%; see page \pageref{AshteadCommonMoleValley} for the result.

\subsubsection*{Leatherhead North \hspace*{\fill}\nolinebreak[1]%
\enspace\hspace*{\fill}
\finalhyphendemerits=0
[30th June; LD gain from C]}

\index{Leatherhead North , Mole Valley@Leatherhead N., \emph{Mole Valley}}

Resignation of Santiago Mondejar Flores (C).

\noindent
\begin{tabular*}{\columnwidth}{@{\extracolsep{\fill}} p{0.53\columnwidth} >{\itshape}l r @{\extracolsep{\fill}}}
Joe Crome & LD & 862\\
Tracy Keeley & C & 340\\
Simon Chambers & UKIP & 157\\
Marc Green & Lab & 135\\
Vicki Elcoate & Grn & 28\\
\end{tabular*}

\council{Reigate and Banstead}

\subsubsection*{Kingswood with Burgh Heath \hspace*{\fill}\nolinebreak[1]%
\enspace\hspace*{\fill}
\finalhyphendemerits=0
[3rd November]}

\index{Kingswood with Burgh Heath , Reigate and Banstead@Kingswood with Burgh Heath, \emph{Reigate \& Banstead}}

Death of Joan Spiers (C).

\noindent
\begin{tabular*}{\columnwidth}{@{\extracolsep{\fill}} p{0.53\columnwidth} >{\itshape}l r @{\extracolsep{\fill}}}
Rod Ashford & C & 839\\
Gerard Hever & UKIP & 155\\
Tony Robinson & Lab & 96\\
Shasha Khan & Grn & 55\\
\end{tabular*}

\subsection*{Runnymede}
\index{Runnymede}

\subsubsection*{Woodham \hspace*{\fill}\nolinebreak[1]%
\enspace\hspace*{\fill}
\finalhyphendemerits=0
[5th May]}

\index{Woodham , Runnymede@Woodham, \emph{Runnymede}}

Resignation of Robert Mackin (C).

Combined with the 2016 ordinary election.
%; see page \pageref{WoodhamRunnymede} for the result.

\subsection*{Spelthorne}
\index{Spelthorne}

\subsubsection*{\sloppyword{Ashford North and Stanwell South} \hspace*{\fill}\nolinebreak[1]%
\enspace\hspace*{\fill}
\finalhyphendemerits=0
[5th May]}

\index{Ashford North and Stanwell South , Spelthorne@Ashford N. \& Stanwell S., \emph{Spelthorne}}

Resignation of Anne-Marie Neale (C).

\noindent
\begin{tabular*}{\columnwidth}{@{\extracolsep{\fill}} p{0.53\columnwidth} >{\itshape}l r @{\extracolsep{\fill}}}
John Boughtflower & C & 658\\
Iain Raymond & Lab & 487\\
Gerald Gravett & UKIP & 393\\
Gordon Douglas & Grn & 133\\
Alan Mockford & LD & 59\\
Paul Couchman & TUSC & 30\\
\end{tabular*}

\subsection*{Tandridge}
\index{Tandridge}

OLRG = Oxted and Limpsfield Residents Group

\subsubsection*{Warlingham West \hspace*{\fill}\nolinebreak[1]%
\enspace\hspace*{\fill}
\finalhyphendemerits=0
[21st July]}

\index{Warlingham West , Tandridge@Warlingham W., \emph{Tandridge}}

Death of Glynis Whittle (C).

\noindent
\begin{tabular*}{\columnwidth}{@{\extracolsep{\fill}} p{0.53\columnwidth} >{\itshape}l r @{\extracolsep{\fill}}}
Keith Prew & C & 367\\
Celia Caulcott & LD & 218\\
Martin Haley & UKIP & 64\\
\end{tabular*}

\subsubsection*{Limpsfield \hspace*{\fill}\nolinebreak[1]%
\enspace\hspace*{\fill}
\finalhyphendemerits=0
[13th October; OLRG gain from C]}

\index{Limpsfield , Tandridge@Limpsfield, \emph{Tandridge}}

Resignation of John Pannett (C).

\noindent
\begin{tabular*}{\columnwidth}{@{\extracolsep{\fill}} p{0.53\columnwidth} >{\itshape}l r @{\extracolsep{\fill}}}
Phil Davies & OLRG & 713\\
Neil O'Brien & C & 472\\
Sheelagh Crampton & LD & 33\\
Simon Charles & Lab & 25\\
\end{tabular*}

\subsubsection*{Valley \hspace*{\fill}\nolinebreak[1]%
\enspace\hspace*{\fill}
\finalhyphendemerits=0
[24th November]}

\index{Valley , Tandridge@Valley, \emph{Tandridge}}

Resignation of Jill Caudle (LD).

\noindent
\begin{tabular*}{\columnwidth}{@{\extracolsep{\fill}} p{0.53\columnwidth} >{\itshape}l r @{\extracolsep{\fill}}}
Dorinda Cooper & LD & 444\\
Paul Shipway & C & 215\\
Jeffrey Bolter & UKIP & 145\\
Mark Wood & Lab & 57\\
\end{tabular*}

\subsection*{Waverley}
\index{Waverley}

\subsubsection*{Farnham Castle \hspace*{\fill}\nolinebreak[1]%
\enspace\hspace*{\fill}
\finalhyphendemerits=0
[18th August; Farnham gain from C]}

\index{Farnham Castle , Waverley@Farnham Castle, \emph{Waverley}}

Resignation of Patrick Blagden (C).

\noindent
\begin{tabular*}{\columnwidth}{@{\extracolsep{\fill}} p{0.51\columnwidth} >{\itshape}l r @{\extracolsep{\fill}}}
Jerry Hyman & Farnham & 386\\
Stewart Edge & LD & 292\\
Nicholas le Gal & C & 229\\
George Hesse & UKIP & 43\\
\end{tabular*}

\subsubsection*{Farnham Shortheath and Boundstone \hspace*{\fill}\nolinebreak[1]%
\enspace\hspace*{\fill}
\finalhyphendemerits=0
[18th August; Farnham gain from C]}

\index{Farnham Shortheath and Boundstone , Waverley@\sloppyword{Farnham Shortheath \& Boundstone, \emph{Waverley}}}

Resignation of David Munro (C).

\noindent
\begin{tabular*}{\columnwidth}{@{\extracolsep{\fill}} p{0.51\columnwidth} >{\itshape}l r @{\extracolsep{\fill}}}
John Ward & Farnham & 356\\
Donal O'Neill & C & 233\\
Sylvia Jacobs & LD & 90\\
Andrew Jones & Ind & 43\\
Jim Burroughs & UKIP & 26\\
\end{tabular*}

\subsubsection*{Cranleigh West \hspace*{\fill}\nolinebreak[1]%
\enspace\hspace*{\fill}
\finalhyphendemerits=0
[Wednesday 21st December]}

\index{Cranleigh West , Waverley@Cranleigh W., \emph{Waverley}}

Death of Brian Ellis (C).

\noindent
\begin{tabular*}{\columnwidth}{@{\extracolsep{\fill}} p{0.51\columnwidth} >{\itshape}l r @{\extracolsep{\fill}}}
Liz Townsend & C & 377\\
Richard Cole & LD & 187\\
Rosaleen Egan & UKIP & 78\\
\end{tabular*}

\section{Warwickshire}

\council{North Warwickshire}

\subsubsection*{Arley and Whitacre \hspace*{\fill}\nolinebreak[1]%
\enspace\hspace*{\fill}
\finalhyphendemerits=0
[22nd September; Lab gain from C]}

\index{Arley and Whitacre , North Warwickshire@Arley \& Whitacre, \emph{N. Warks.}}

Resignation of Andrew Watkins (C).

\noindent
\begin{tabular*}{\columnwidth}{@{\extracolsep{\fill}} p{0.53\columnwidth} >{\itshape}l r @{\extracolsep{\fill}}}
Jodie Gosling & Lab & 577\\
Karen Barber & C & 390\\
\end{tabular*}

\council{Nuneaton and Bedworth}

\subsubsection*{Exhall \hspace*{\fill}\nolinebreak[1]%
\enspace\hspace*{\fill}
\finalhyphendemerits=0
[5th May]}

\index{Exhall , Nuneaton and Bedworth@Exhall, \emph{Nuneaton \& Bedworth}}

Resignation of Roma Taylor (Lab).

Combined with the 2016 ordinary election.
%; see page \pageref{ExhallNuneatonBedworth} for the result.

\subsection*{Stratford-on-Avon}
\index{Stratford-on-Avon}

\subsubsection*{Studley with Sambourne \hspace*{\fill}\nolinebreak[1]%
\enspace\hspace*{\fill}
\finalhyphendemerits=0
[25th February; LD gain from C]}

\index{Studley with Sambourne , Stratford-on-Avon@Studley with Sambourne, \emph{Stratford-on-Avon}}

Resignation of Nick Moon (C).

\noindent
\begin{tabular*}{\columnwidth}{@{\extracolsep{\fill}} p{0.53\columnwidth} >{\itshape}l r @{\extracolsep{\fill}}}
Hazel Wright & LD & 632\\
Paul Beaman & C & 233\\
Karen Somner-Brown & Lab & 156\\
Nick Moon & Ind & 66\\
Nigel Rogers & UKIP & 55\\
\end{tabular*}

\subsection*{Warwick}
\index{Warwick}

\subsubsection*{Myton and Heathcote \hspace*{\fill}\nolinebreak[1]%
\enspace\hspace*{\fill}
\finalhyphendemerits=0
[1st December]}

\index{Myton and Heathcote , Warwick@Myton \& Heathcote, \emph{Warwick}}

Death of Raj Mann (C).

\noindent
\begin{tabular*}{\columnwidth}{@{\extracolsep{\fill}} p{0.53\columnwidth} >{\itshape}l r @{\extracolsep{\fill}}}
Mary Noone & C & 488\\
Nick Solman & LD & 228\\
Ben Wesson & Lab & 194\\
\end{tabular*}



\section{West Sussex}

\subsection*{Chichester}
\index{Chichester}

\subsubsection*{Southbourne \hspace*{\fill}\nolinebreak[1]%
\enspace\hspace*{\fill}
\finalhyphendemerits=0
[1st December; LD gain from C]}

\index{Southbourne , Chichester@Southbourne, \emph{Chichester}}

Resignation of Bruce Finch (C).

\noindent
\begin{tabular*}{\columnwidth}{@{\extracolsep{\fill}} p{0.53\columnwidth} >{\itshape}l r @{\extracolsep{\fill}}}
Jonathan Brown & LD & 646\\
David Harwood & C & 289\\
Patricia Hunt & UKIP & 132\\
Rebecca Hamlet & Lab & 53\\
\end{tabular*}

\subsection*{Horsham}
\index{Horsham}

\subsubsection*{Southwater \hspace*{\fill}\nolinebreak[1]%
\enspace\hspace*{\fill}
\finalhyphendemerits=0
[10th November]}

\index{Southwater , Horsham@Southwater, \emph{Horsham}}

Death of Ian Howard (C).

\noindent
\begin{tabular*}{\columnwidth}{@{\extracolsep{\fill}} p{0.53\columnwidth} >{\itshape}l r @{\extracolsep{\fill}}}
Billy Greening & C & 1046\\
Richard Greenwood & LD & 308\\
Kevin O'Sullivan & Lab & 118\\
Uri Baran & UKIP & 109\\
\end{tabular*}

\section{Wiltshire}

\subsection*{Wiltshire}
\index{Wiltshire}

\subsubsection*{Amesbury East \hspace*{\fill}\nolinebreak[1]%
\enspace\hspace*{\fill}
\finalhyphendemerits=0
[5th May; LD gain from C]}

\index{Amesbury East , Wiltshire@Amesbury E., \emph{Wilts.}}

Resignation of John Noeken (C).

\noindent
\begin{tabular*}{\columnwidth}{@{\extracolsep{\fill}} p{0.53\columnwidth} >{\itshape}l r @{\extracolsep{\fill}}}
Jamie Capp & LD & 361\\
Robert Yuill & C & 356\\
Andy Derry & Ind & 292\\
Les Webster & UKIP & 217\\
Steve McAuliffe & Lab & 133\\
Joshua Baker & Grn & 60\\
\end{tabular*}

\columnbreak

\subsubsection*{Trowbridge Grove \hspace*{\fill}\nolinebreak[1]%
\enspace\hspace*{\fill}
\finalhyphendemerits=0
[14th July; LD gain from Ind]}

\index{Trowbridge Grove , Wiltshire@Trowbridge Grove, \emph{Wilts.}}

Death of Jeff Osborn (Ind).

\noindent
\begin{tabular*}{\columnwidth}{@{\extracolsep{\fill}} p{0.53\columnwidth} >{\itshape}l r @{\extracolsep{\fill}}}
Chris Auckland & LD & 421\\
David Halik & C & 196\\
Simon Selby & UKIP & 123\\
Shaun Henley & Lab & 77\\
Robert Wall & Ind & 74\\
Philip Randle & Grn & 27\\
\end{tabular*}

\section{Worcestershire}

\subsection*{County Council}
\index{Worcestershire}

\subsubsection*{Ombersley \hspace*{\fill}\nolinebreak[1]%
\enspace\hspace*{\fill}
\finalhyphendemerits=0
[11th August]}

\index{Ombersley , Worcestershire@Ombersley, \emph{Worcs.}}

Death of Maurice Broomfield (C).

\noindent
\begin{tabular*}{\columnwidth}{@{\extracolsep{\fill}} p{0.53\columnwidth} >{\itshape}l r @{\extracolsep{\fill}}}
Peter Tomlinson & C & 956\\
Peter Evans & LD & 224\\
Richard Keel & UKIP & 212\\
Doug Ingram & Ind & 120\\
\end{tabular*}

\subsection*{Wychavon}
\index{Wychavon}

\subsubsection*{Droitwich West \hspace*{\fill}\nolinebreak[1]%
\enspace\hspace*{\fill}
\finalhyphendemerits=0
[28th July]}

\index{Droitwich West , Wychavon@Droitwich W., \emph{Wychavon}}

Resignation of Catherine Powell (C).

\noindent
\begin{tabular*}{\columnwidth}{@{\extracolsep{\fill}} p{0.53\columnwidth} >{\itshape}l r @{\extracolsep{\fill}}}
George Duffy & C & 281\\
Alan Humphries & Lab & 161\\
Andy Morgan & UKIP & 132\\
Adrian Key & LD & 97\\
\end{tabular*}

\section{Glamorgan}

\subsection*{Cardiff}
\index{Cardiff}

\subsubsection*{Plasnewydd \hspace*{\fill}\nolinebreak[1]%
\enspace\hspace*{\fill}
\finalhyphendemerits=0
[Tuesday 20th September; LD gain from Lab]}

\index{Plasnewydd , Cardiff@Plasnewydd, \emph{Cardiff}}

Death of Mohammad Javed (Lab).

\noindent
\begin{tabular*}{\columnwidth}{@{\extracolsep{\fill}} p{0.53\columnwidth} >{\itshape}l r @{\extracolsep{\fill}}}
Robin Rea & LD & 1258\\
Peter Wong & Lab & 910\\
Glenn Page & PC & 177\\
Munawar Mughal & C & 115\\
Michael Cope & Grn & 93\\
Lawrence Gwynn & UKIP & 62\\
\end{tabular*}

\subsubsection*{Grangetown \hspace*{\fill}\nolinebreak[1]%
\enspace\hspace*{\fill}
\finalhyphendemerits=0
[3rd November; PC gain from Lab]}

\index{Grangetown , Cardiff@Grangetown, \emph{Cardiff}}

Death of Chris Lomax (Lab).

\noindent
\begin{tabular*}{\columnwidth}{@{\extracolsep{\fill}} p{0.53\columnwidth} >{\itshape}l r @{\extracolsep{\fill}}}
Tariq Awan & PC & 1163\\
Maliika Kaaba & Lab & 1049\\
Michael Bryan & C & 287\\
Asghar Ali & LD & 187\\
Richard Lewis & UKIP & 141\\
\end{tabular*}

\subsection*{Neath Port Talbot}
\index{Neath Port Talbot}

\subsubsection*{Blaengwrach \hspace*{\fill}\nolinebreak[1]%
\enspace\hspace*{\fill}
\finalhyphendemerits=0
[20th October; PC gain from Lab]}

\index{Blaengwrach , Neath Port Talbot@Blaengwrach, \emph{Neath Port Talbot}}

Death of Alf Siddley (Lab).

\noindent
\begin{tabular*}{\columnwidth}{@{\extracolsep{\fill}} p{0.53\columnwidth} >{\itshape}l r @{\extracolsep{\fill}}}
Carolyn Edwards & PC & 225\\
Sarah Price & Lab & 143\\
Thomas Evans & Ind & 58\\
Richard Pritchard & UKIP & 39\\
Peter Crocker-Jacques & C & 4\\
\end{tabular*}

\subsection*{Swansea}
\index{Swansea}

I@S = Independents @ Swansea

\subsubsection*{Mynyddbach \hspace*{\fill}\nolinebreak[1]%
\enspace\hspace*{\fill}
\finalhyphendemerits=0
[5th May]}

\index{Mynyddbach , Swansea@Mynyddbach, \emph{Swansea}}

Resignation of Byron Owen (Lab).

\noindent
\begin{tabular*}{\columnwidth}{@{\extracolsep{\fill}} p{0.53\columnwidth} >{\itshape}l r @{\extracolsep{\fill}}}
Mike Lewis & Lab & 1525\\
Shan Couch & PC & 340\\
Patrick Morgan & C & 321\\
Noel West & I@S & 263\\
Charlene Webster & LD & 120\\
Ashley Wakeling & Ind & 82\\
\end{tabular*}

\subsection*{Vale of Glamorgan}
\index{Vale of Glamorgan}

Pirate = Pirate Party UK

\subsubsection*{Rhoose \hspace*{\fill}\nolinebreak[1]%
\enspace\hspace*{\fill}
\finalhyphendemerits=0
[30th June]}

\index{Rhoose , Vale of Glamorgan@Rhoose, \emph{Vale of Glamorgan}}

Death of Philip Clarke (Ind).

\noindent
\begin{tabular*}{\columnwidth}{@{\extracolsep{\fill}} p{0.545\columnwidth} >{\itshape}l r @{\extracolsep{\fill}}}
Adam Riley & Ind & 598\\
Gordon Kemp & C & 520\\
\sloppyword{Graham Loveluck-Edwards} & Lab & 401\\
Rachel Banner & Ind & 399\\
Ian Perry & PC & 104\\
Robin Lynn & LD & 24\\
James Fyfe & Pirate & 4\\
\end{tabular*}

\subsubsection*{Gibbonsdown \hspace*{\fill}\nolinebreak[1]%
\enspace\hspace*{\fill}
\finalhyphendemerits=0
[3rd November]}

\index{Gibbonsdown , Vale of Glamorgan@Gibbonsdown, \emph{Vale of Glamorgan}}

Resignation of Rob Curtis (Lab).

\noindent
\begin{tabular*}{\columnwidth}{@{\extracolsep{\fill}} p{0.53\columnwidth} >{\itshape}l r @{\extracolsep{\fill}}}
Julie Aviet & Lab & 404\\
Shirley Hodges & PC & 161\\
Dennis Harkus & Ind & 113\\
Leighton Rowlands & C & 104\\
Robin Hunter-Clarke & UKIP & 54\\
Jennifer Geroni & LD & 7\\
\end{tabular*}

\section{Gwent}

\subsection*{Blaenau Gwent}
\index{Blaenau Gwent}

\subsubsection*{Brynmawr \hspace*{\fill}\nolinebreak[1]%
\enspace\hspace*{\fill}
\finalhyphendemerits=0
[29th September]}

\index{Brynmawr , Blaenau Gwent@Brynmawr, \emph{Blaenau Gwent}}

Death of John Hopkins (Ind).

\noindent
\begin{tabular*}{\columnwidth}{@{\extracolsep{\fill}} p{0.53\columnwidth} >{\itshape}l r @{\extracolsep{\fill}}}
Wayne Hogdins & Ind & 1085\\
Julian Gardner & Lab & 270\\
\end{tabular*}

\subsection*{Caerphilly}
\index{Caerphilly}

\subsubsection*{Moriah \hspace*{\fill}\nolinebreak[1]%
\enspace\hspace*{\fill}
\finalhyphendemerits=0
[31st March]}

\index{Moriah , Caerphilly@Moriah, \emph{Caerphilly}}

Death of Gina Bevan (Lab).

\noindent
\begin{tabular*}{\columnwidth}{@{\extracolsep{\fill}} p{0.53\columnwidth} >{\itshape}l r @{\extracolsep{\fill}}}
David Harse & Lab & 464\\
Peter Bailie & Ind & 196\\
Mervyn Diggle & Ind & 89\\
Ian Gorman & UKIP & 77\\
Nigel Godfrey & C & 7\\
\end{tabular*}

\subsubsection*{Ynysddu \hspace*{\fill}\nolinebreak[1]%
\enspace\hspace*{\fill}
\finalhyphendemerits=0
[7th April]}

\index{Ynysddu , Caerphilly@Ynysddu, \emph{Caerphilly}}

Resignation of Colin Durham (Lab).

\noindent
\begin{tabular*}{\columnwidth}{@{\extracolsep{\fill}} p{0.53\columnwidth} >{\itshape}l r @{\extracolsep{\fill}}}
Philippa Marsden & Lab & 502\\
Joe Smyth & UKIP & 180\\
Marina Pritchard & PC & 134\\
Matthew Kidner & LD & 36\\
\end{tabular*}

\subsubsection*{Gilfach \hspace*{\fill}\nolinebreak[1]%
\enspace\hspace*{\fill}
\finalhyphendemerits=0
[6th October]}

\index{Gilfach , Caerphilly@Gilfach, \emph{Caerphilly}}

Resignation of Harry Andrews (Lab).

\noindent
\begin{tabular*}{\columnwidth}{@{\extracolsep{\fill}} p{0.53\columnwidth} >{\itshape}l r @{\extracolsep{\fill}}}
Lindsey Harding & Lab & 254\\
Ken Houston & PC & 150\\
Glenys Griffiths & UKIP & 28\\
Andrew Creak & Grn & 7\\
\end{tabular*}

\subsubsection*{Risca East \hspace*{\fill}\nolinebreak[1]%
\enspace\hspace*{\fill}
\finalhyphendemerits=0
[6th October]}

\index{Risca East , Caerphilly@Risca E., \emph{Caerphilly}}

Resignation of Rhianon Passmore (Lab).

\noindent
\begin{tabular*}{\columnwidth}{@{\extracolsep{\fill}} p{0.53\columnwidth} >{\itshape}l r @{\extracolsep{\fill}}}
Arianna Passmore & Lab & 400\\
Matthew Farrell & PC & 120\\
Joe Smyth & UKIP & 117\\
Matthew Kidner & LD & 32\\
\end{tabular*}

\subsection*{Newport}
\index{Newport}

\subsubsection*{St Julians \hspace*{\fill}\nolinebreak[1]%
\enspace\hspace*{\fill}
\finalhyphendemerits=0
[28th July]}

\index{Saint Julians , Newport@St Julians, \emph{Newport}}

Death of Ed Townsend (LD).

\noindent
\begin{tabular*}{\columnwidth}{@{\extracolsep{\fill}} p{0.53\columnwidth} >{\itshape}l r @{\extracolsep{\fill}}}
Carmel Townsend & LD & 948\\
Phil Hourahine & Lab & 432\\
Andrew Byers & UKIP & 156\\
Carol Bader & C & 135\\
Chris Priest & PC & 71\\
Mirka Virtanen & Grn & 25\\
\end{tabular*}

\section{Mid and West Wales}

\subsection*{Carmarthenshire}
\index{Carmarthenshire}

PF = People First

\subsubsection*{Cilycwm \hspace*{\fill}\nolinebreak[1]%
\enspace\hspace*{\fill}
\finalhyphendemerits=0
[22nd September; PC gain from Ind]}

\index{Cilycwm , Carmarthenshire@Cilycwm, \emph{Carmarthenshire}}

Death of Tom Theophilus (Ind).

\noindent
\begin{tabular*}{\columnwidth}{@{\extracolsep{\fill}} p{0.53\columnwidth} >{\itshape}l r @{\extracolsep{\fill}}}
Dafydd Tomos & PC & 201\\
Thomas Davies & Ind & 151\\
Maria Carroll & Lab & 123\\
Matthew Paul & Ind & 106\\
Jacqui Thompson & PF & 64\\
Catherine Nakielny & LD & 62\\
Stephen Holmes & C & 15\\
\end{tabular*}

\subsection*{Powys}
\index{Powys}

\subsubsection*{Welshpool Llanerchyddol \hspace*{\fill}\nolinebreak[1]%
\enspace\hspace*{\fill}
\finalhyphendemerits=0
[15th December]}

\index{Welshpool Llanerchyddol , Powys@Welshpool Llanerchyddol, \emph{Powys}}

Death of Ann Holloway (Ind).

\noindent
\begin{tabular*}{\columnwidth}{@{\extracolsep{\fill}} p{0.53\columnwidth} >{\itshape}l r @{\extracolsep{\fill}}}
Graham Breeze & Ind & 323\\
Richard Church & LD & 212\\
Ruth Canning & C & 126\\
\end{tabular*}

\section{North Wales}

\subsection*{Conwy}
\index{Conwy}

\subsubsection*{Mostyn \hspace*{\fill}\nolinebreak[1]%
\enspace\hspace*{\fill}
\finalhyphendemerits=0
[7th July]}

\index{Mostyn , Conwy@Mostyn, \emph{Conwy}}

Resignation of Jobi Hold (Lab).

\noindent
\begin{tabular*}{\columnwidth}{@{\extracolsep{\fill}} p{0.53\columnwidth} >{\itshape}l r @{\extracolsep{\fill}}}
Emily Owen & Lab & 248\\
Greg Robbins & C & 200\\
Penelope Appleton & LD & 126\\
John Thomas & Ind & 88\\
John Humberstone & UKIP & 75\\
Richard Enston & Ind & 27\\
\end{tabular*}

\subsubsection*{Abergele Pensarn \hspace*{\fill}\nolinebreak[1]%
\enspace\hspace*{\fill}
\finalhyphendemerits=0
[20th October; Ind gain from Lab]}

\index{Abergele Pensarn , Conwy@Abergele Pensarn, \emph{Conwy}}

Resignation of Rick Stubbs (Lab).

\noindent
\begin{tabular*}{\columnwidth}{@{\extracolsep{\fill}} p{0.53\columnwidth} >{\itshape}l r @{\extracolsep{\fill}}}
Alan Hunter & Ind & 170\\
Michael Smith & Ind & 146\\
David Hancock & Lab & 136\\
Bernice McLoughlin & C & 87\\
\end{tabular*}

\subsection*{Denbighshire}
\index{Denbighshire}

\subsubsection*{Denbigh Lower \hspace*{\fill}\nolinebreak[1]%
\enspace\hspace*{\fill}
\finalhyphendemerits=0
[2nd June]}

\index{Denbigh Lower , Denbighshire@Denbigh Lower, \emph{Denbighshire}}

Death of Richard Davies (Ind).

\noindent
\begin{tabular*}{\columnwidth}{@{\extracolsep{\fill}} p{0.53\columnwidth} >{\itshape}l r @{\extracolsep{\fill}}}
Mark Young & Ind & 389\\
Rhys Thomas & PC & 315\\
Lara Pritchard & C & 159\\
John McGuire & Lab & 108\\
Gwyn Williams & LD & 101\\
\end{tabular*}

\subsubsection*{Dyserth \hspace*{\fill}\nolinebreak[1]%
\enspace\hspace*{\fill}
\finalhyphendemerits=0
[2nd June; Ind gain from C]}

\index{Dyserth , Denbighshire@Dyserth, \emph{Denbighshire}}

Death of Peter Owen (C).

\noindent
\begin{tabular*}{\columnwidth}{@{\extracolsep{\fill}} p{0.53\columnwidth} >{\itshape}l r @{\extracolsep{\fill}}}
David Williams & Ind & 177\\
Dave Parry & Ind & 159\\
Andy Hughes & C & 140\\
Heather Prydderch & LD & 107\\
Richard Jones-Abbas & Lab & 87\\
Janice Williams & PC & 21\\
\end{tabular*}

\subsubsection*{Rhyl West \hspace*{\fill}\nolinebreak[1]%
\enspace\hspace*{\fill}
\finalhyphendemerits=0
[27th October]}

\index{Rhyl West , Denbighshire@Rhyl W., \emph{Denbighshire}}

Resignation of Ian Armstrong (Lab).

\noindent
\begin{tabular*}{\columnwidth}{@{\extracolsep{\fill}} p{0.53\columnwidth} >{\itshape}l r @{\extracolsep{\fill}}}
Alan James & Lab & 199\\
Les Harker & C & 93\\
Norman Shone & Ind & 55\\
Mark Webster & Ind & 42\\
Keith Kirwan & LD & 26\\
\end{tabular*}

\subsection*{Flintshire}

\subsubsection*{New Brighton \hspace*{\fill}\nolinebreak[1]%
\enspace\hspace*{\fill}
\finalhyphendemerits=0
[18th February]}

\index{New Brighton , Flintshire@New Brighton, \emph{Flintshire}}

Resignation of Amanda Brigg (LD).

\noindent
\begin{tabular*}{\columnwidth}{@{\extracolsep{\fill}} p{0.53\columnwidth} >{\itshape}l r @{\extracolsep{\fill}}}
Sara Parker & LD & 365\\
Linda Pierce & Lab & 241\\
Zulya Taylor & C & 141\\
John Yorke & Ind & 60\\
\end{tabular*}

\subsection*{Gwynedd}
\index{Gwynedd}

\subsubsection*{Marchog \hspace*{\fill}\nolinebreak[1]%
\enspace\hspace*{\fill}
\finalhyphendemerits=0
[14th July]}

\index{Marchog , Gwynedd@Marchog, \emph{Gwynedd}}

Resignation of Chris O'Neal (Ind).

\noindent
\begin{tabular*}{\columnwidth}{@{\extracolsep{\fill}} p{0.53\columnwidth} >{\itshape}l r @{\extracolsep{\fill}}}
Dylan Fernley & Ind & 211\\
Luke Tugwell & Lab & 112\\
\end{tabular*}

\subsubsection*{Y Felinheli \hspace*{\fill}\nolinebreak[1]%
\enspace\hspace*{\fill}
\finalhyphendemerits=0
[14th July]}

\index{Y Felinheli , Gwynedd@Y Felinheli, \emph{Gwynedd}}

Resignation of Siân Gwenllian AM (PC).

\noindent
\begin{tabular*}{\columnwidth}{@{\extracolsep{\fill}} p{0.53\columnwidth} >{\itshape}l r @{\extracolsep{\fill}}}
Gareth Griffith & PC & 614\\
Andrew Kinsman & C & 49\\
\end{tabular*}

\subsubsection*{Waunfawr \hspace*{\fill}\nolinebreak[1]%
\enspace\hspace*{\fill}
\finalhyphendemerits=0
[21st July]}

\index{Waunfawr , Gwynedd@Waunfawr, \emph{Gwynedd}}

Resignation of Eurig Wyn (PC).

\noindent
\begin{tabular*}{\columnwidth}{@{\extracolsep{\fill}} p{0.53\columnwidth} >{\itshape}l r @{\extracolsep{\fill}}}
Edgar Owen & PC & 358\\
Paul Scott & Lab & 114\\
\end{tabular*}

\section[Aberdeenshire Councils]{\sloppyword{Aberdeenshire Councils}}

\subsection*{Aberdeenshire}
\index{Aberdeenshire}

\subsubsection*{Banff and District \hspace*{\fill}\nolinebreak[1]%
\enspace\hspace*{\fill}
\finalhyphendemerits=0
[3rd November; C gain from SNP]}

\index{Banff and District , Aberdeenshire@Banff \& District, \emph{Aberdeenshire}}

Death of Ian Gray (SNP).

\noindent
\begin{tabular*}{\columnwidth}{@{\extracolsep{\fill}} p{0.53\columnwidth} >{\itshape}l r @{\extracolsep{\fill}}}
\emph{First preferences}\\
Iain Taylor & C & 1170\\
Glen Reynolds & SNP & 962\\
Alistair Mason & LD & 526\\
\end{tabular*}

\noindent
\begin{tabular*}{\columnwidth}{@{\extracolsep{\fill}} p{0.53\columnwidth} >{\itshape}l r @{\extracolsep{\fill}}}
\emph{Mason eliminated}\\
Iain Taylor & C & 1378\\
Glen Reynolds & SNP & 1097\\
\end{tabular*}

\subsubsection*{Inverurie and District \hspace*{\fill}\nolinebreak[1]%
\enspace\hspace*{\fill}
\finalhyphendemerits=0
[3rd November; C gain from LD]}

\index{Inverurie and District , Aberdeenshire@Inverurie \& District, \emph{Aberdeenshire}}

Resignation of Martin Kitts-Hayes (LD).

\noindent
\begin{tabular*}{\columnwidth}{@{\extracolsep{\fill}} p{0.53\columnwidth} >{\itshape}l r @{\extracolsep{\fill}}}
\emph{First preferences}\\
Colin Clark & C & 1302\\
Neil Baillie & SNP & 1164\\
Alison Auld & LD & 755\\
Sarah Flavell & Lab & 139\\
\end{tabular*}

\noindent
\begin{tabular*}{\columnwidth}{@{\extracolsep{\fill}} p{0.53\columnwidth} >{\itshape}l r @{\extracolsep{\fill}}}
\multicolumn{3}{@{\extracolsep{\fill}}l}{\emph{Auld and Flavell eliminated}}\\
Colin Clark & C & 1701\\
Neil Baillie & SNP & 1341\\
\end{tabular*}

\section{Ayrshire Councils}

\subsection*{North Ayrshire}
\index{North Ayrshire}

\subsubsection*{Irvine West \hspace*{\fill}\nolinebreak[1]%
\enspace\hspace*{\fill}
\finalhyphendemerits=0
[11th August; Lab gain from SNP]}

\index{Irvine West , North Ayrshire@Irvine W., \emph{N. Ayrshire}}

Resignation of Ruth Maguire (SNP).

\noindent
\begin{tabular*}{\columnwidth}{@{\extracolsep{\fill}} p{0.53\columnwidth} >{\itshape}l r @{\extracolsep{\fill}}}
\emph{First preferences}\\
Robin Sturgeon & SNP & 1164\\
Louise McPhater & Lab & 1029\\
Angela Stephen & C & 639\\
Bobby Cochrane & SocLab & 131\\
Joan McCormick & Grn & 94\\
Nick Smith & LD & 48\\
\end{tabular*}

\noindent
\begin{tabular*}{\columnwidth}{@{\extracolsep{\fill}} p{0.53\columnwidth} >{\itshape}l r @{\extracolsep{\fill}}}
\multicolumn{3}{@{\extracolsep{\fill}}l}{\emph{Four candidates eliminated}}\\
Louise McPhater & Lab & 1301\\
Robin Sturgeon & SNP & 1277\\
\end{tabular*}

\section{Border Councils}

\council{Dumfries and Galloway}

\subsubsection*{Annandale North \hspace*{\fill}\nolinebreak[1]%
\enspace\hspace*{\fill}
\finalhyphendemerits=0
[17th November]}

\index{Annandale North , Dumfries and Galloway@Annandale N., \emph{Dumfries \& Galloway}}

Resignation of Graeme Tait (Lab elected as C).

\noindent
\begin{tabular*}{\columnwidth}{@{\extracolsep{\fill}} p{0.53\columnwidth} >{\itshape}l r @{\extracolsep{\fill}}}
Douglas Fairbairn & C & 2041\\
Sylvia Moffat & SNP & 749\\
Adam Wilson & Lab & 611\\
Chris Ballance & Grn & 152\\
\end{tabular*}

\section{Clyde Councils}

\subsection*{Glasgow}
\index{Glasgow}

\subsubsection*{Anderston/City \hspace*{\fill}\nolinebreak[1]%
\enspace\hspace*{\fill}
\finalhyphendemerits=0
[5th May; SNP gain from Lab]}

\index{Anderston/City , Glasgow@Anderston\slash City, \emph{Glasgow}}

Resignation of Gordon Matheson (Lab).

\noindent
\begin{tabular*}{\columnwidth}{@{\extracolsep{\fill}} p{0.53\columnwidth} >{\itshape}l r @{\extracolsep{\fill}}}
\emph{First preferences}\\
Angus Millar & SNP & 3467\\
Steven Livingston & Lab & 1698\\
Christy Mearns & Grn & 1621\\
Philip Charles & C & 869\\
Ryan Ross & LD & 248\\
Karen King & UKIP & 154\\
\end{tabular*}

\emph{Charles, Ross and King eliminated}: Millar 3586, Livingston 2112, Mearns 1869.

\noindent
\begin{tabular*}{\columnwidth}{@{\extracolsep{\fill}} p{0.53\columnwidth} >{\itshape}l r @{\extracolsep{\fill}}}
\emph{Mearns eliminated}\\
Angus Millar & SNP & 4436\\
Steven Livingston & Lab & 2603\\
\end{tabular*}

\subsubsection*{Garscadden/Scotstounhill \hspace*{\fill}\nolinebreak[1]%
\enspace\hspace*{\fill}
\finalhyphendemerits=0
[6th October; SNP gain from Lab]}

\index{Garscadden/Scotstounhill , Glasgow@\sloppyword{Garscadden\slash Scotstounhill, \emph{Glasgow}}}

Death of John Kelly (Lab).

\noindent
\begin{tabular*}{\columnwidth}{@{\extracolsep{\fill}} p{0.53\columnwidth} >{\itshape}l r @{\extracolsep{\fill}}}
\emph{First preferences}\\
Chris Cunningham & SNP & 2135\\
Ian Cruikshank & Lab & 1944\\
Ary Jaff & C & 510\\
Gillian Macdonald & Grn & 242\\
James Speirs & LD & 97\\
Donald Mackay & UKIP & 83\\
\end{tabular*}

\emph{Macdonald, Speirs and Mackay eliminated}: Cunningham 2275, Cruikshank 2037, Jaff 564.

\noindent
\begin{tabular*}{\columnwidth}{@{\extracolsep{\fill}} p{0.53\columnwidth} >{\itshape}l r @{\extracolsep{\fill}}}
\emph{Jaff eliminated}\\
Chris Cunningham & SNP & 2321\\
Ian Cruikshank & Lab & 2204\\
\end{tabular*}

\subsection*{North Lanarkshire}
\index{North Lanarkshire}

\subsubsection*{Coatbridge North and Glenboig \hspace*{\fill}\nolinebreak[1]%
\enspace\hspace*{\fill}
\finalhyphendemerits=0
[22nd September; Lab gain from SNP]}

\index{Coatbridge North and Glenboig , North Lanarkshire@Coatbridge N. \& Glenboig, \emph{N. Lanarks.}}

Resignation of Fulton MacGregor (SNP).

\noindent
\begin{tabular*}{\columnwidth}{@{\extracolsep{\fill}} p{0.53\columnwidth} >{\itshape}l r @{\extracolsep{\fill}}}
\emph{First preferences}\\
Alex McVey & Lab & 1350\\
Stephen Kirley & SNP & 1261\\
Ben Callaghan & C & 366\\
John Wilson & Grn & 195\\
Neil Wilson & UKIP & 63\\
\end{tabular*}

\noindent
\begin{tabular*}{\columnwidth}{@{\extracolsep{\fill}} p{0.53\columnwidth} >{\itshape}l r @{\extracolsep{\fill}}}
\multicolumn{3}{@{\extracolsep{\fill}}l}{\emph{Three candidates eliminated}}\\
Alex McVey & Lab & 1572\\
Stephen Kirley & SNP & 1378\\
\end{tabular*}

\subsection*{Renfrewshire}
\index{Renfrewshire}

\subsubsection*{Renfrew South and Gallowhill \hspace*{\fill}\nolinebreak[1]%
\enspace\hspace*{\fill}
\finalhyphendemerits=0
[11th August; SNP gain from Lab]}

\index{Renfrew South and Gallowhill , Renfrewshire@Renfrew S. \& Gallowhill, \emph{Renfs.}}

Death of Eddie Grady (Lab).

\noindent
\begin{tabular*}{\columnwidth}{@{\extracolsep{\fill}} p{0.53\columnwidth} >{\itshape}l r @{\extracolsep{\fill}}}
\emph{First preferences}\\
Jim Paterson & SNP & 1309\\
Edward Grady & Lab & 1012\\
Mark Dougan & C & 366\\
Ross Stalker & LD & 53\\
\end{tabular*}

\noindent
\begin{tabular*}{\columnwidth}{@{\extracolsep{\fill}} p{0.53\columnwidth} >{\itshape}l r @{\extracolsep{\fill}}}
\multicolumn{3}{@{\extracolsep{\fill}}l}{\emph{Dougan and Stalker eliminated}}\\
Jim Paterson & SNP & 1356\\
Edward Grady & Lab & 1180\\
\end{tabular*}

\subsection*{South Lanarkshire}
\index{South Lanarkshire}

\subsubsection*{Hamilton North and East \hspace*{\fill}\nolinebreak[1]%
\enspace\hspace*{\fill}
\finalhyphendemerits=0
[21st January]}

\index{Hamilton North and East , South Lanarkshire@Hamilton N. \& E., \emph{S. Lanarks.}}

Death of Lynn Adams (SNP).

\noindent
\begin{tabular*}{\columnwidth}{@{\extracolsep{\fill}} p{0.53\columnwidth} >{\itshape}l r @{\extracolsep{\fill}}}
\emph{First preferences}\\
Stephanie Callaghan & SNP & 1089\\
Lyndsay Clelland & Lab & 855\\
James Mackay & C & 469\\
Steven Hannigan & Grn & 83\\
Norman Rae & LD & 45\\
\end{tabular*}

\noindent
\begin{tabular*}{\columnwidth}{@{\extracolsep{\fill}} p{0.53\columnwidth} >{\itshape}l r @{\extracolsep{\fill}}}
\multicolumn{3}{@{\extracolsep{\fill}}l}{\emph{Mackay, Hannigan and Rae eliminated}}\\
%\emph{Mackay, Hannigan \& Rae eliminated}\\
Stephanie Callaghan & SNP & 1206\\
Lyndsay Clelland & Lab & 1052\\
\end{tabular*}

\section{Forth Councils}

\subsection*{Fife}
\index{Fife}

\subsubsection*{The Lochs \hspace*{\fill}\nolinebreak[1]%
\enspace\hspace*{\fill}
\finalhyphendemerits=0
[25th August; Lab gain from Ind]}

\index{Lochs , Fife@The Lochs, \emph{Fife}}

Resignation of Willie Clarke (Ind).

\noindent
\begin{tabular*}{\columnwidth}{@{\extracolsep{\fill}} p{0.53\columnwidth} >{\itshape}l r @{\extracolsep{\fill}}}
\emph{First preferences}\\
Mary Lockhart & Lab & 1318\\
Lea McLelland & SNP & 1079\\
Malcolm McDonald & C & 270\\
Thomas Kirby & Comm & 86\\
Bradford Oliver & LD & 45\\
\end{tabular*}

\emph{Kirby and Oliver eliminated}: Lockhart 1365, McLelland 1106, McDonald 280.

\noindent
\begin{tabular*}{\columnwidth}{@{\extracolsep{\fill}} p{0.53\columnwidth} >{\itshape}l r @{\extracolsep{\fill}}}
\emph{McDonald eliminated}\\
Mary Lockhart & Lab & 1459\\
Lea McLelland & SNP & 1124\\
\end{tabular*}

\subsubsection*{Leven, Kennoway and Largo \hspace*{\fill}\nolinebreak[1]%
\enspace\hspace*{\fill}
\finalhyphendemerits=0
[15th December]}

\index{Leven, Kennoway and Largo , Fife@Leven, Kennoway \& Largo, \emph{Fife}}

Resignation of Alistair Hunter (SNP).

\noindent
\begin{tabular*}{\columnwidth}{@{\extracolsep{\fill}} p{0.53\columnwidth} >{\itshape}l r @{\extracolsep{\fill}}}
\emph{First preferences}\\
Alistair Suttie & SNP & 1501\\
Colin Davidson & Lab & 1155\\
Graham Ritchie & C & 752\\
Steve Wood & LD & 580\\
Iain Morrice & Grn & 74\\
\end{tabular*}

\emph{Wood and Morrice eliminated}: Suttie 1615, Davidson 1302, Ritchie 954.

\noindent
\begin{tabular*}{\columnwidth}{@{\extracolsep{\fill}} p{0.53\columnwidth} >{\itshape}l r @{\extracolsep{\fill}}}
\emph{Ritchie eliminated}\\
Alistair Suttie & SNP & 1668\\
Colin Davidson & Lab & 1620\\
\end{tabular*}

\section{Highland Councils}

\council{Argyll and Bute}

\subsubsection*{Oban North and Lorn \hspace*{\fill}\nolinebreak[1]%
\enspace\hspace*{\fill}
\finalhyphendemerits=0
[18th February; SNP gain from Ind]}

\index{Oban North and Lorn , Argyll and Bute@Oban N. \& Lorn, \emph{Argyll \& Bute}}

Resignation of Duncan MacIntyre (Ind).

\noindent
\begin{tabular*}{\columnwidth}{@{\extracolsep{\fill}} p{0.53\columnwidth} >{\itshape}l r @{\extracolsep{\fill}}}
\emph{First preferences}\\
Julie McKenzie & SNP & 1113\\
Andrew Vennard & C & 609\\
Kieron Green & Ind & 608\\
Pat Tyrrell & Grn & 300\\
\end{tabular*}

\sloppyword{\emph{Tyrrell eliminated}: McKenzie 1186 Green 721 Vennard 640}

\noindent
\begin{tabular*}{\columnwidth}{@{\extracolsep{\fill}} p{0.53\columnwidth} >{\itshape}l r @{\extracolsep{\fill}}}
\emph{Vennard eliminated}\\
Julie McKenzie & SNP & 1241\\
Kieron Green & Ind & 1048\\
\end{tabular*}

\subsubsection*{Oban North and Lorn \hspace*{\fill}\nolinebreak[1]%
\enspace\hspace*{\fill}
\finalhyphendemerits=0
[2nd June]}

\index{Oban North and Lorn , Argyll and Bute@Oban N. \& Lorn, \emph{Argyll \& Bute}}

Resignation of Iain MacDonald (SNP elected as Ind).

\noindent
\begin{tabular*}{\columnwidth}{@{\extracolsep{\fill}} p{0.53\columnwidth} >{\itshape}l r @{\extracolsep{\fill}}}
\emph{First preferences}\\
Breege Smyth & SNP & 1055\\
Kieron Green & Ind & 711\\
Andrew Vennard & C & 591\\
David Pollard & LD & 294\\
\end{tabular*}

\emph{Pollard eliminated}: Smyth 1097, Green 821, Vennard 668.

\noindent
\begin{tabular*}{\columnwidth}{@{\extracolsep{\fill}} p{0.53\columnwidth} >{\itshape}l r @{\extracolsep{\fill}}}
\emph{Vennard eliminated}\\
Kieron Green & Ind & 1160\\
Breege Smyth & SNP & 1138\\
\end{tabular*}

\subsection*{Highland}
\index{Highland}

\subsubsection*{Culloden and Ardersier \hspace*{\fill}\nolinebreak[1]%
\enspace\hspace*{\fill}
\finalhyphendemerits=0
[6th October; LD gain from Lab]}

\index{Culloden and Ardersier , Highland@Culloden \& Ardersier, \emph{Highland}}

Death of John Ford (Lab).

\noindent
\begin{tabular*}{\columnwidth}{@{\extracolsep{\fill}} p{0.53\columnwidth} >{\itshape}l r @{\extracolsep{\fill}}}
\emph{First preferences}\\
Pauline Munro & SNP & 753\\
Trish Robertson & LD & 463\\
Andrew Jarvie & C & 439\\
John Ross & Ind & 315\\
Duncan Macpherson & Ind & 274\\
Isla Macleod-O'Reilly & Grn & 180\\
Andrew Mackintosh & Lab & 163\\
David McGrath & Ind & 158\\
Thomas Lamont & Ind & 23\\
\end{tabular*}

\emph{Lamont eliminated}: Munro 755, Robertson 464, Jarvie 442, Ross 317, Macpherson 285, Macleod-O'Reilly 182, Mackintosh 163, McGrath 158.

\sloppyword{\emph{McGrath eliminated}: Munro 775, Robertson 479, Jarvie 452, Ross 330, Macpherson 324, Macleod-O'Reilly 188, Mackintosh 180.}

\emph{Mackintosh eliminated}: Munro 796, Robertson 515, Jarvie 468, Macpherson 346, Ross 339, Macleod-O'Reilly 209.

\emph{Macleod-O'Reilly eliminated}: Munro 862, Robertson 564, Jarvie 478, Macpherson 369, Ross 362.

\emph{Ross eliminated}: Munro 908, Robertson 703, Jarvie 515, Macpherson 414.

\emph{Macpherson eliminated}: Munro 970, Robertson 793, Jarvie 589.

\noindent
\begin{tabular*}{\columnwidth}{@{\extracolsep{\fill}} p{0.53\columnwidth} >{\itshape}l r @{\extracolsep{\fill}}}
\emph{Jarvie eliminated}\\
Trish Robertson & LD & 1026\\
Pauline Munro & SNP & 1001\\
\end{tabular*}

\section{Tay Councils}

\subsection*{Angus}
\index{Angus}

\subsubsection*{Arbroath East and Lunan \hspace*{\fill}\nolinebreak[1]%
\enspace\hspace*{\fill}
\finalhyphendemerits=0
[Monday 28th November; SNP gain from Ind]}

\index{Arbroath East and Lunan , Angus@Arbroath E. \& Lunan, \emph{Angus}}

Resignation of Bob Spink (Ind).

\noindent
\begin{tabular*}{\columnwidth}{@{\extracolsep{\fill}} p{0.53\columnwidth} >{\itshape}l r @{\extracolsep{\fill}}}
\emph{First preferences}\\
Brenda Durno & SNP & 919\\
Derek Wann & C & 709\\
Lois Speed & Ind & 452\\
Kevin Smith & Ind & 309\\
John Ruddy & Lab & 177\\
Richard Moore & LD & 60\\
\end{tabular*}

\emph{Ruddy and Moore eliminated}: Durno 956, Wann 741, Speed 482, Smith 352.

\emph{Smith eliminated}: Durno 1010, Wann 799, Speed 619.

\noindent
\begin{tabular*}{\columnwidth}{@{\extracolsep{\fill}} p{0.53\columnwidth} >{\itshape}l r @{\extracolsep{\fill}}}
\emph{Speed eliminated}\\
Brenda Durno & SNP & 1172\\
Derek Wann & C & 928\\
\end{tabular*}

\subsubsection*{Carnoustie and District \hspace*{\fill}\nolinebreak[1]%
\enspace\hspace*{\fill}
\finalhyphendemerits=0
[Monday 5th December; Ind gain from SNP]}

\index{Carnoustie and District , Angus@Carnoustie \& District, \emph{Angus}}

Death of Helen Oswald (SNP).

\noindent
\begin{tabular*}{\columnwidth}{@{\extracolsep{\fill}} p{0.53\columnwidth} >{\itshape}l r @{\extracolsep{\fill}}}
\emph{First preferences}\\
David Cheape & Ind & 1401\\
Mark McDonald & SNP & 1033\\
Derek Shaw & C & 568\\
Ray Strachan & Lab & 141\\
Beth Morrison & LD & 75\\
\end{tabular*}

\sloppyword{\emph{Strachan and Morrison eliminated}: Cheape 1477, McDonald 1078, Shaw 606.}

\noindent
\begin{tabular*}{\columnwidth}{@{\extracolsep{\fill}} p{0.53\columnwidth} >{\itshape}l r @{\extracolsep{\fill}}}
\emph{Shaw eliminated}\\
David Cheape & Ind & 1637\\
Mark McDonald & SNP & 1117\\
\end{tabular*}

\subsection*{Dundee}
\index{Dundee}

\subsubsection*{Maryfield \hspace*{\fill}\nolinebreak[1]%
\enspace\hspace*{\fill}
\finalhyphendemerits=0
[31st March]}

\index{Maryfield , Dundee@Maryfield, \emph{Dundee}}

Resignation of Craig Melville (SNP).

\noindent
\begin{tabular*}{\columnwidth}{@{\extracolsep{\fill}} p{0.53\columnwidth} >{\itshape}l r @{\extracolsep{\fill}}}
\emph{First preferences}\\
Lynne Short & SNP & 1383\\
Alan Cowan & Lab & 634\\
James Clancy & C & 294\\
Stuart Fairweather & TUSC & 142\\
Jacob Ellis & Grn & 116\\
Christopher McIntyre & LD & 85\\
Brian McLeod & Ind & 73\\
Calum Walker & UKIP & 69\\
\end{tabular*}

\noindent
\begin{tabular*}{\columnwidth}{@{\extracolsep{\fill}} p{0.53\columnwidth} >{\itshape}l r @{\extracolsep{\fill}}}
\emph{Walker eliminated}\\
Lynne Short & SNP & 1389\\
Alan Cowan & Lab & 637\\
James Clancy & C & 304\\
Stuart Fairweather & TUSC & 147\\
Jacob Ellis & Grn & 118\\
Christopher McIntyre & LD & 87\\
Brian McLeod & Ind & 81\\
\end{tabular*}

\council{Perth and Kinross}

\subsubsection*{Almond and Earn \hspace*{\fill}\nolinebreak[1]%
\enspace\hspace*{\fill}
\finalhyphendemerits=0
[7th April; C gain from Ind]}

\index{Almond and Earn , Perth and Kinross@Almond \& Earn, \emph{Perth \& Kinross}}

Death of Alan Jack (Ind).

\noindent
\begin{tabular*}{\columnwidth}{@{\extracolsep{\fill}} p{0.53\columnwidth} >{\itshape}l r @{\extracolsep{\fill}}}
\emph{First preferences}\\
Kathleen Baird & C & 1651\\
Wilma Lumsden & SNP & 1327\\
Dave Mackenzie & Lab & 219\\
George Hayton & LD & 157\\
Denise Baykal & UKIP & 77\\
\end{tabular*}

\emph{Baykal eliminated}: Baird 1681, Lumsden 1334, Mackenzie 228, Hayton 163.

\noindent
\begin{tabular*}{\columnwidth}{@{\extracolsep{\fill}} p{0.53\columnwidth} >{\itshape}l r @{\extracolsep{\fill}}}
\emph{Hayton eliminated}\\
Kathleen Baird & C & 1720\\
Wilma Lumsden & SNP & 1370\\
Dave Mackenzie & Lab & 280\\
\end{tabular*}

\end{resultsiii}

\documentclass[a4paper,openany]{book}
\usepackage{graphicx}
\usepackage{url}
\usepackage{microtype}

\usepackage[plainpages=false,pdfpagelabels,pdfauthor={Andrew Teale},pdftitle={Local Election Results 2017},hidelinks]{hyperref}

\usepackage[cp1252]{inputenc}

% put all the other packages here:

\usepackage{election06-test}
	
\renewcommand\resultsyear{2017}

\pdfminorversion=5 
\pdfobjcompresslevel=2 

\setboolean{maps}{false}

\begin{document}

% Title page

\begin{titlepage}

\begin{center}

\Huge Local Election Results

2017

\bigskip

\Large Andrew Teale

\vfill

\newcommand\versionno{0.10.1}

%Version \versionno

\today

\end{center}

\end{titlepage}

% Copyright notice

\begin{center}

\bigskip

Typeset by \LaTeX{} 

\bigskip

Compilation and design \copyright\ Andrew Teale, 2017.

 Permission is granted to copy, distribute and/or modify this document
 under the terms of the GNU Free Documentation License, Version 1.3
 or any later version published by the Free Software Foundation;
 with no Invariant Sections, no Front-Cover Texts, and no Back-Cover Texts.
 A copy of the license is included in the section entitled ``GNU
 Free Documentation License''.

\bigskip

This file is available for download from
\url{http://www.andrewteale.me.uk/}

\bigskip

Please advise the author of any corrections which need to be made by
email: \url{andrewteale@yahoo.co.uk}

\vfill
\end{center}

\section*{Change Log}

%8 June 2014: Added result for Clydesdale South by-election.

%24 November 2013: Corrected results for Vassall ward, Lambeth (typing error) and Reddish North ward, Stockport (LD candidate incorrectly shown as Labour).
%
%21 November 2013: First version.

\tableofcontents

% Introduction

% Introduction

% \chapter*{Introduction and Abbreviations}
% \addcontentsline{toc}{chapter}{Introduction and Abbreviations}
% %\markright{INTRODUCTION AND ABBREVIATIONS}
% 
% Elections were held on 6th May 2010 to all London boroughs and metropolitan boroughs, and some unitary authorities and shire districts in England. These elections were combined with a general election which was held on the same day. 
%
% The voting system used for all elections covered here was
% first-past-the-post, with multi-member FPTP being used where more than
% one seat was up for election. 
% 
%The results of the general election are shown in Part I. The information in Part I is taken from the Electoral Commission.
%
% All of the seats on the 32 London borough councils were up for election. The vast majority of London boroughs use multi-member wards electing three councillors each; there are also a handful of single-member and two-member wards. Elections to the London boroughs are covered in Part~II, which has been split into two chapters (North and South London).
% 
% The 36 metropolitan boroughs are all elected by thirds. Each ward has
% three councillors, with the winning councillor from the 2006 election being up for
% election in each ward. In some cases two seats were up for election,
% due to the death or resignation of another councillor for the ward
% within six months of the election. Results of these elections are
% contained in Part~III, which each of the former metropolitan counties
% constituting a separate chapter.
% 
% The English unitary authorities and shire districts may have up to
% three councillors in each ward, and may hold elections either all at
% once or by thirds. 
% Only those councils which elect by thirds held
% elections this year; those councils which elect all at once were
% last elected in 2007 and will next be elected in 2011. A few
% districts elect by halves every two years; all of these districts held
% an election this year. Where districts elect by thirds generally not
% all of the wards in the district hold an election every year. A full explanation of the electoral arrangements is
% given at the head of each council's entry.
%
% Due to a botched attempt at local government reorganisation, the 2010 elections to Exeter and Norwich city councils were held on 9th September. It had originally been intended to change these councils to unitary status, which resulted in the scheduled 2010 elections to these councils being cancelled with the intention that the first elections to the new unitary councils would take place in 2011, the councillors elected in 2006 to have their terms extended until 2011. However, when the unitary plans were abandoned by the coalition government following the general election, the High Court ruled that the councillors elected in 2006 had come to the end of their four-year term and could no longer continue in office.
% 
% Unitary election results are shown in Part~IV 
% with shire district results in Part~V. Part~IV is
% divided into eight chapters based on region, while Part~V has one
% chapter for each county.
% 
% For the first time in this series, referendums (Part VI) and by-elections (Part VII) held in 2010 are also included. Scottish local by-elections are held using the Alternative Vote; while details of transfers are shown, for reasons of space some elimination stages have been omitted.
% 
% Finally, at the back you will find an Index of Wards.
% 
% Where a candidate in an election dies, the election in that ward or division is
% cancelled and rearranged for a later date. This happened in the
% following wards or divisions at this election:
% 
% \begin{results}
% \begin{itemize}
% \item Haverstock, Camden\index{Haverstock , Camden@Haverstock, Camden}
% \item Ore, Hastings\index{Ore , Hastings@Ore, Hastings}
% \end{itemize}
% \end{results}
% 
% Here is a list of abbreviations used in this book for major parties
% and selected other parties which fought several councils. This list
% is not exhaustive; parties which put up only a few candidates will
% generally have their abbreviation listed at the head of the entry for
% the relevant council. Please note that the ``Lab'' label includes
% candidates who were jointly sponsored by the Labour and Co-operative
% Parties.
% 
% \begin{results}
% BNP - British National Party
% 
% C - Conservative Party
% 
% Grn - Green Party
% 
% Ind - Independent
% 
% Lab - Labour Party
% 
% LD - Liberal Democrat
% 
% Lib - Liberal Party
% 
% Loony - Monster Raving Loony Party
% 
% Respect - Respect, the Unity Coalition
% 
% SocLab - Socialist Labour Party
% 
% UKIP - UK Independence Party
% 
% \end{results}
%
%Errors in a work of this size are inevitable. I take full responsibility for any errors which may have crept in, undertake to correct any errors which I am made aware of, and hope that any errors which you may spot do not substantially affect any use you may make of this book.
%
% I would like to close this section by thanking all those who have
% supplied me with results and sources of information, most notably David Boothroyd, John Cartwright, James Doyle, Keith Edkins, the Electoral Commission, Tom Harris, Paul Harwood, ``hullenedge'', ``Listener'', ``MaxQue'', Philip Mutton, John Swarbrick, Andrew Stidwell, Pete Whitehead and all the members of the Vote UK Forum, and particularly those scores of council
% webpages without which this work would not have been possible. 

 
 
% Here beginneth the content

% 2017 results to go here


\part{Referendums}

\chapter{Referendums in 2017}

%There were no referendums in 2017.

\section{Burnley mayoral referendum}

A referendum was held in Burnley on 4th May on the question of whether the district should have a directly elected mayor.

\noindent
\begin{tabular*}{\columnwidth}{@{\extracolsep{\fill}} p{0.545\columnwidth} >{\itshape}l r @{\extracolsep{\fill}}}
& Yes & -\\
& No & -\\
\end{tabular*}

\part{By-elections}

\chapter{Parliamentary by-elections}

There were three parliamentary by-elections in 2017.

\section*{Copeland \hspace*{\fill}\nolinebreak[1]%
\enspace\hspace*{\fill}
\finalhyphendemerits=0
[23rd February; C gain from Lab]}

\index{Copeland , House of Commons@Copeland, \emph{House of Commons}}

Resignation of Jamie Reed (Lab).

\noindent
\begin{tabular*}{\columnwidth}{@{\extracolsep{\fill}} p{0.53\columnwidth} >{\itshape}l r @{\extracolsep{\fill}}}
Trudy Harrison & C & 13748\\
Gillian Troughton & Lab & 11601\\
Rebecca Hanson & LD & 2252\\
Fiona Mills & UKIP & 2025\\
Michael Guest & Ind & 811\\
Jack Lenox & Grn & 515\\
Roy Ivinson & Ind & 116\\
\end{tabular*}

\section*{Stoke-on-Trent Central \hspace*{\fill}\nolinebreak[1]%
\enspace\hspace*{\fill}
\finalhyphendemerits=0
[23rd February]}

\index{Stoke-on-Trent Central , House of Commons@Stoke-on-Trent C., \emph{House of Commons}}

Resignation of Tristram Hunt (Lab).

\noindent
\begin{tabular*}{\columnwidth}{@{\extracolsep{\fill}} p{0.53\columnwidth} >{\itshape}l r @{\extracolsep{\fill}}}
Gareth Snell & Lab & 7853\\
Paul Nuttall & UKIP & 5233\\
Jack Brereton & C & 5154\\
Zulfiqar Ali & LD & 2083\\
Adam Colclough & Grn & 294\\
Barbara Fielding-Morriss & Ind & 137\\
The Incredible Flying Brick & Loony & 127\\
David Furness & BNP & 124\\
Godfrey Davies & CPA & 109\\
Mohammed Akram & Ind & 56\\
\end{tabular*}

\section*{Manchester Gorton \hspace*{\fill}\nolinebreak[1]%
\enspace\hspace*{\fill}
\finalhyphendemerits=0
[pending]}

\index{Manchester Gorton , House of Commons@Manchester Gorton, \emph{House of Commons}}

Death of Sir Gerald Kaufman (Lab).

\chapter{By-elections to devolved assemblies, the European Parliament, and police and crime commissionerships}

\section{Greater London Authority}

There were no by-elections in 2017 to the Greater London Authority.

\section{National Assembly for Wales}

There were no by-elections in 2017 to the National Assembly for Wales.

\section{Scottish Parliament}

There were no by-elections in 2017 to the Scottish Parliament.

\section{Northern Ireland Assembly}

Vacancies in the Northern Ireland Assembly are filled by co-option.
No co-options were made in 2017.
%
%The following members were co-opted to the Assembly in 2017:
%\begin{itemize}
%\item \emph{to be confirmed} (SF) replaced Jennifer McCann following her resignation on 30th November (Belfast West).
%\end{itemize}

\section{European Parliament}

UK vacancies in the European Parliament are filled by the next available person from the party list at the most recent election (which was held in 2014). 
No replacements were made in 2017.
%The following replacement was made in 2010:
%\begin{itemize}
%\item Keith Taylor (Grn) replaced Caroline Lucas following her resignation on 17th May (South East).
%\end{itemize}

\section{Police and crime commissioners}

There were no by-elections in 2017 for vacant police and crime commissioner posts.

\chapter{Local by-elections and unfilled vacancies}

\begin{resultsiii}

\section[North London]{\sloppyword{North London}}

\subsection*{City of London Corporation}

At the March 2017 ordinary election there were unfilled vacancies in Bishopsgate Without and Queenhithe wards due to the resignation of Billy Dove and the election of Alastair King as alderman (both Ind) respectively.

\subsection*{Camden}

\subsubsection*{Gospel Oak \hspace*{\fill}\nolinebreak[1]%
\enspace\hspace*{\fill}
\finalhyphendemerits=0
[pending]}

\index{Gospel Oak , Camden@Gospel Oak, \emph{Camden}}

Resignation of Maeve McCormack (Lab).

\subsection*{Harrow}

\subsubsection*{Roxbourne \hspace*{\fill}\nolinebreak[1]%
\enspace\hspace*{\fill}
\finalhyphendemerits=0
[9th March]}

\index{Roxbourne , Harrow@Roxbourne, \emph{Harrow}}

Death of Bob Currie (Lab).

\noindent
\begin{tabular*}{\columnwidth}{@{\extracolsep{\fill}} p{0.53\columnwidth} >{\itshape}l r @{\extracolsep{\fill}}}
Maxine Henson & Lab & 1554\\
Annabel Singh & C & 533\\
Marshel Amutharasan & LD & 240\\
Herbert Crossman & UKIP & 148\\
\end{tabular*}

\subsubsection*{Kenton East \hspace*{\fill}\nolinebreak[1]%
\enspace\hspace*{\fill}
\finalhyphendemerits=0
[20th April]}

\index{Kenton East , Harrow@Kenton E., \emph{Harrow}}

Death of Mitzi Green (Lab).

\section[Greater Manchester]{\sloppyword{Greater Manchester}}

\subsection*{Manchester}

\subsubsection*{Rusholme \hspace*{\fill}\nolinebreak[1]%
\enspace\hspace*{\fill}
\finalhyphendemerits=0
[pending]}

\index{Rusholme , Manchester@Rusholme, \emph{Manchester}}

Resignation (to be confirmed) of Kate Chappell (Lab).

\subsection*{Oldham}

\subsubsection*{Failsworth East \hspace*{\fill}\nolinebreak[1]%
\enspace\hspace*{\fill}
\finalhyphendemerits=0
[16th February]}

\index{Failsworth East , Oldham@Failsworth E., \emph{Oldham}}

Resignation of Jim McMahon (Lab).

\noindent
\begin{tabular*}{\columnwidth}{@{\extracolsep{\fill}} p{0.53\columnwidth} >{\itshape}l r @{\extracolsep{\fill}}}
Paul Jacques & Lab & 829\\
Antony Cahill & C & 360\\
Nicholas Godleman & UKIP & 166\\
Andy Hunter-Rossall & Grn & 49\\
Shaun Duffy & LD & 16\\
\end{tabular*}

\subsection*{Salford}

\subsubsection*{Kersal \hspace*{\fill}\nolinebreak[1]%
\enspace\hspace*{\fill}
\finalhyphendemerits=0
[2nd March; C gain from Lab]}

\index{Kersal , Salford@Kersal, \emph{Salford}}

Death of Harry Davies (Lab).

\noindent
\begin{tabular*}{\columnwidth}{@{\extracolsep{\fill}} p{0.53\columnwidth} >{\itshape}l r @{\extracolsep{\fill}}}
Arnie Saunders & C & 850\\
Mike Pevitt & Lab & 553\\
Jonny Wineberg & Ind & 354\\
Christopher Barnes & UKIP & 182\\
Jason Reading & Grn & 48\\
Adam Slack & LD & 39\\
\end{tabular*}

\subsection*{Trafford}

\subsubsection*{Broadheath \hspace*{\fill}\nolinebreak[1]%
\enspace\hspace*{\fill}
\finalhyphendemerits=0
[pending]}

\index{Broadheath , Trafford@Broadheath, \emph{Trafford}}

Resignation of Louise Dagnall (Lab).

\section[Merseyside]{\sloppyword{Merseyside}}

\subsection*{Wirral}

\subsubsection*{Claughton \hspace*{\fill}\nolinebreak[1]%
\enspace\hspace*{\fill}
\finalhyphendemerits=0
[4th May]}

\index{Claughton , Wirral@Claughton, \emph{Wirral}}

Death of Denise Roberts (Lab).

\section[South Yorkshire]{\sloppyword{South Yorkshire}}

\subsection*{Doncaster}

At the May 2017 ordinary election there was an unfilled vacancy in Norton and Askern ward due to the death of Alan Jones (Lab).
\index{Norton and Askern , Doncaster@Norton \& Askern, \emph{Doncaster}}

\subsection*{Rotherham}

\subsubsection*{Brinsworth and Catcliffe \hspace*{\fill}\nolinebreak[1]%
\enspace\hspace*{\fill}
\finalhyphendemerits=0
[2nd February; LD gain from Lab]}

\index{Brinsworth and Catcliffe , Rotherham@Brinsworth \& Catcliffe, \emph{Rotherham}}

Resignation of Andrew Roddison (Lab).

\noindent
\begin{tabular*}{\columnwidth}{@{\extracolsep{\fill}} p{0.53\columnwidth} >{\itshape}l r @{\extracolsep{\fill}}}
Adam Carter & LD & 2000\\
Shabana Ahmed & Lab & 519\\
Steve Webster & UKIP & 389\\
John Oliver & C & 91\\
Bex Whyman & Grn & 30\\
\end{tabular*}

\subsubsection*{Dinnington \hspace*{\fill}\nolinebreak[1]%
\enspace\hspace*{\fill}
\finalhyphendemerits=0
[2nd February; Lab gain from UKIP]}

\index{Dinnington , Rotherham@Dinnington, \emph{Rotherham}}

Resignation of Ian Finnie (UKIP).

\noindent
\begin{tabular*}{\columnwidth}{@{\extracolsep{\fill}} p{0.53\columnwidth} >{\itshape}l r @{\extracolsep{\fill}}}
John Vjestica & Lab & 670\\
Lee Hunter & UKIP & 303\\
Christopher Middleton & C & 238\\
Dave Smith & Ind & 232\\
Jean Hart & Ind & 180\\
Steven Scott & Ind & 81\\
David Foulstone & Grn & 78\\
Stephen Thornley & LD & 75\\
\end{tabular*}

\section{Tyne and Wear}

\subsection*{Newcastle upon Tyne}

\subsubsection*{South Heaton \hspace*{\fill}\nolinebreak[1]%
\enspace\hspace*{\fill}
\finalhyphendemerits=0
[16th March]}

\index{South Heaton , Newcastle upon Tyne@South Heaton, \emph{Newcastle upon Tyne}}

Resignation of Sophie White (Lab).

\subsection*{Sunderland}

\subsubsection*{Sandhill \hspace*{\fill}\nolinebreak[1]%
\enspace\hspace*{\fill}
\finalhyphendemerits=0
[12th January; LD gain from Lab]}

\index{Sandhill , Sunderland@Sandhill, \emph{Sunderland}}

Disqualification of Jacqui Gallagher (Lab): non-attendance.

\noindent
\begin{tabular*}{\columnwidth}{@{\extracolsep{\fill}} p{0.53\columnwidth} >{\itshape}l r @{\extracolsep{\fill}}}
Stephen O'Brien & LD & 824\\
Gary Waller & Lab & 458\\
Bryan Foster & UKIP & 343\\
Gavin Wilson & C & 184\\
Helmut Izaks & Grn & 23\\
\end{tabular*}

\section{West Midlands}

\subsection*{Birmingham}

\subsubsection*{Hall Green \hspace*{\fill}\nolinebreak[1]%
\enspace\hspace*{\fill}
\finalhyphendemerits=0
[pending]}

\index{Hall Green , Birmingham@Hall Green, \emph{Birmingham}}

Resignation of Sam Burden (Lab).

\subsubsection*{Perry Barr \hspace*{\fill}\nolinebreak[1]%
\enspace\hspace*{\fill}
\finalhyphendemerits=0
[pending]}

\index{Perry Barr , Birmingham@Perry Barr, \emph{Birmingham}}

Death of Ray Hassall (LD).

\subsection*{Dudley}

\subsubsection*{St Thomas's \hspace*{\fill}\nolinebreak[1]%
\enspace\hspace*{\fill}
\finalhyphendemerits=0
[16th February]}

\index{Saint Thomas's , Dudley@St Thomas's, \emph{Dudley}}

Resignation of Glenis Simms (Lab).

\noindent
\begin{tabular*}{\columnwidth}{@{\extracolsep{\fill}} p{0.53\columnwidth} >{\itshape}l r @{\extracolsep{\fill}}}
Shaneila Mughal & Lab & 1466\\
Phil Wimlett & UKIP & 653\\
Jonathan Elliott & C & 249\\
Francis Sheppard & Grn & 52\\
\end{tabular*}

\section{West Yorkshire}

\subsection*{Calderdale}

\subsubsection*{Hipperholme and Lightcliffe \hspace*{\fill}\nolinebreak[1]%
\enspace\hspace*{\fill}
\finalhyphendemerits=0
[6th April]}

\index{Hipperholme and Lightcliffe , Calderdale@Hipperholme \& Lightcliffe, \emph{Calderdale}}

Death of Graham Hall (C).

\section{Bedfordshire}

\subsection*{Central Bedfordshire}

\subsubsection*{Shefford \hspace*{\fill}\nolinebreak[1]%
\enspace\hspace*{\fill}
\finalhyphendemerits=0
[pending]}

\index{Shefford , Central Bedfordshire@Shefford, \emph{C. Beds.}}

Death of Lewis Birt (C).

\section{Berkshire}

\subsection*{Slough}

\subsubsection*{Haymill and Lynch Hill \hspace*{\fill}\nolinebreak[1]%
\enspace\hspace*{\fill}
\finalhyphendemerits=0
[pending]}

\index{Haymill and Lynch Hill , Slough@Haymill \& Lynch Hill, \emph{Slough}}

Death of Darren Morris (C).

\subsection*{Wokingham}

\subsubsection*{Emmbrook \hspace*{\fill}\nolinebreak[1]%
\enspace\hspace*{\fill}
\finalhyphendemerits=0
[Friday 17th February; LD gain from C]}

\index{Emmbrook , Wokingham@Emmbrook, \emph{Wokingham}}

Resignation of Christopher Singleton (C).

\noindent
\begin{tabular*}{\columnwidth}{@{\extracolsep{\fill}} p{0.53\columnwidth} >{\itshape}l r @{\extracolsep{\fill}}}
Imogen Shepherd-Dubey & LD & 1575\\
Kevin Morgan & C & 879\\
Phil Ray & UKIP & 104\\
Christopher Everett & Lab & 79\\
\end{tabular*}

\section[Buckinghamshire]{\sloppyword{Buckinghamshire}}

\subsection*{County Council}

At the May 2017 ordinary election there was an unfilled vacancy in Amersham and Chesham Bois division due to the death of Martin Phillips (C).
\index{Amersham and Chesham Bois , Buckinghamshire@Amersham \& Chesham Bois, \emph{Bucks.}}

\subsection*{Aylesbury Vale}

\subsubsection*{Elmhurst \hspace*{\fill}\nolinebreak[1]%
\enspace\hspace*{\fill}
\finalhyphendemerits=0
[6th April]}

\index{Elmhurst , Aylesbury Vale@Elmhurst, \emph{Aylesbury Vale}}

Resignation of Andy Hetherington (UKIP).

\subsubsection*{Wendover and Halton \hspace*{\fill}\nolinebreak[1]%
\enspace\hspace*{\fill}
\finalhyphendemerits=0
[4th May]}

\index{Wendover and Halton , Aylesbury Vale@Wendover \& Halton, \emph{Aylesbury Vale}}

Resignation of Andrew Southam (C).

\subsection*{Chiltern}

\subsubsection*{Great Missenden \hspace*{\fill}\nolinebreak[1]%
\enspace\hspace*{\fill}
\finalhyphendemerits=0
[pending]}

\index{Great Missenden , Chiltern@Great Missenden, \emph{Chiltern}}

Resignation of Seb Berry (Ind).

\section[Cambridgeshire]{\sloppyword{Cambridgeshire}}

\subsection*{Huntingdonshire}

\subsubsection*{St Neots Eaton Ford \hspace*{\fill}\nolinebreak[1]%
\enspace\hspace*{\fill}
\finalhyphendemerits=0
[pending]}

\index{Saint Neots Eaton Ford , Huntingdonshire@St Neots Eaton Ford, \emph{Hunts.}}

Resignation of David Harty (C).

\subsection*{South Cambridgeshire}

\subsubsection*{Bourn \hspace*{\fill}\nolinebreak[1]%
\enspace\hspace*{\fill}
\finalhyphendemerits=0
[pending]}

\index{Bourn , South Cambridgeshire@Bourn, \emph{S. Cambs.}}

Resignation of Mervyn Loyes (C).

\section{Cheshire}

\subsection*{Cheshire East}

Boll1st = Bollington First

\subsubsection*{Bollington \hspace*{\fill}\nolinebreak[1]%
\enspace\hspace*{\fill}
\finalhyphendemerits=0
[16th February; Boll1st gain from C]}

\index{Bollington , Cheshire East@Bollington, \emph{Ches. E.}}

Resignation of Jon Weston (C).

\noindent
\begin{tabular*}{\columnwidth}{@{\extracolsep{\fill}} p{0.53\columnwidth} >{\itshape}l r @{\extracolsep{\fill}}}
James Nicholas & Boll1st & 939\\
Philip Bolton & C & 319\\
Rob Vernon & Lab & 239\\
Sam al-Hamdani & LD & 198\\
Richard Purslow & Grn & 162\\
\end{tabular*}

\subsection*{Cheshire West and Chester}

\subsubsection*{Blacon \hspace*{\fill}\nolinebreak[1]%
\enspace\hspace*{\fill}
\finalhyphendemerits=0
[20th April]}

\index{Blacon , Cheshire West and Chester@Blacon, \emph{Ches. W. \& Chester}}

Resignation of Reg Jones (Lab).

\section{Cumbria}

\subsection*{Carlisle}

\subsubsection*{Belle Vue \hspace*{\fill}\nolinebreak[1]%
\enspace\hspace*{\fill}
\finalhyphendemerits=0
[pending]}

\index{Belle Vue , Carlisle@Belle Vue, \emph{Carlisle}}

Resignation of Jacqui Franklin (Lab).

\subsubsection*{Yewdale \hspace*{\fill}\nolinebreak[1]%
\enspace\hspace*{\fill}
\finalhyphendemerits=0
[pending]}

\index{Yewdale , Carlisle@Yewdale, \emph{Carlisle}}

Resignation of Tom Dodd (Lab).

\section{Derbyshire}

\subsection*{Derby}

\subsubsection*{Derwent \hspace*{\fill}\nolinebreak[1]%
\enspace\hspace*{\fill}
\finalhyphendemerits=0
[9th March; C gain from UKIP]}

\index{Derwent , Derby@Derwent, \emph{Derby}}

Death of Bill Wright (UKIP).

\noindent
\begin{tabular*}{\columnwidth}{@{\extracolsep{\fill}} p{0.53\columnwidth} >{\itshape}l r @{\extracolsep{\fill}}}
Steve Willoughby & C & 789\\
Nadine Peatfield & Lab & 611\\
Tony Crawley & UKIP & 537\\
Simon Ferrigno & LD & 192\\
\end{tabular*}

\subsection*{South Derbyshire}

\subsubsection*{Woodville \hspace*{\fill}\nolinebreak[1]%
\enspace\hspace*{\fill}
\finalhyphendemerits=0
[pending]}

\index{Woodville , South Derbyshire@Woodville, \emph{S. Derbys.}}

Death of Gill Farrington (C).

\section{Devon}

\subsection*{Exeter}

\subsubsection*{St Thomas \hspace*{\fill}\nolinebreak[1]%
\enspace\hspace*{\fill}
\finalhyphendemerits=0
[pending]}

\index{Saint Thomas , Exeter@St Thomas, \emph{Exeter}}

Death of Paul Bull (Lab).

\subsection*{South Hams}

\subsubsection*{Charterlands \hspace*{\fill}\nolinebreak[1]%
\enspace\hspace*{\fill}
\finalhyphendemerits=0
[23rd February; LD gain from C]}

\index{Charterlands , South Hams@Charterlands, \emph{S. Hams}}

Resignation of Lindsay Ward (C).

\noindent
\begin{tabular*}{\columnwidth}{@{\extracolsep{\fill}} p{0.53\columnwidth} >{\itshape}l r @{\extracolsep{\fill}}}
Elizabeth Huntley & LD & 473\\
Jonathan Bell & C & 404\\
David Trigger & Lab & 110\\
Janet Chapman & Grn & 40\\
\end{tabular*}

\subsection*{Teignbridge}

\subsubsection*{Bushell \hspace*{\fill}\nolinebreak[1]%
\enspace\hspace*{\fill}
\finalhyphendemerits=0
[pending]}

\index{Bushell , Teignbridge@Bushell, \emph{Teignbridge}}

Reason for vacancy not known: missing councillor is Judy Grainger (C).

\subsection*{West Devon}

\subsubsection*{Bere Ferrers \hspace*{\fill}\nolinebreak[1]%
\enspace\hspace*{\fill}
\finalhyphendemerits=0
[pending]}

\index{Bere Ferrers , West Devon@Bere Ferrers, \emph{W. Devon}}

Death of Mike Benson (C).

\section{Dorset}

\subsection*{Christchurch}

\subsubsection*{Mudeford and Friars Cliff \hspace*{\fill}\nolinebreak[1]%
\enspace\hspace*{\fill}
\finalhyphendemerits=0
[2nd March]}

\index{Mudeford and Friars Cliff , Christchurch@Mudeford \& Friars Cliff, \emph{Christchurch}}

Resignation of Andy Barfield (C).

\noindent
\begin{tabular*}{\columnwidth}{@{\extracolsep{\fill}} p{0.53\columnwidth} >{\itshape}l r @{\extracolsep{\fill}}}
Paul Hilliard & C & 629\\
Sheila Gray & Ind & 466\\
Julian Spurr & Lab & 91\\
Lawrence Wilson & UKIP & 85\\
Fiona Cownie & Grn & 72\\
\end{tabular*}

\subsection*{West Dorset}

\subsubsection*{Piddle Valley \hspace*{\fill}\nolinebreak[1]%
\enspace\hspace*{\fill}
\finalhyphendemerits=0
[13th April]}

\index{Piddle Valley , West Dorset@Piddle Valley, \emph{W. Dorset}}

Resignation of Peter Hiscock (C).

\section{Durham}

\subsection*{Durham}

At the May 2017 ordinary election there were unfilled vacancies in Chester-le-Street East and Coxhoe divisions due to the resignations of Lawson Armstrong and Mac Williams (both Lab) respectively.
\index{Chester-le-Street East , Durham@Chester-le-Street E., \emph{Durham}}
\index{Coxhoe , Durham@Coxhoe, \emph{Durham}}

\section{Essex}

\subsection*{County Council}

At the May 2017 ordinary election there was an unfilled vacancy in Halstead division due to the death of Joe Pike (C).
\index{Halstead , Essex@Halstead, \emph{Essex}}

\subsection*{Epping Forest}

\subsubsection*{Chigwell Village \hspace*{\fill}\nolinebreak[1]%
\enspace\hspace*{\fill}
\finalhyphendemerits=0
[23rd February]}

\index{Chigwell Village , Epping Forest@Chigwell Village, \emph{Epping Forest}}

Resignation of Lesley Wagland (C).

\noindent
\begin{tabular*}{\columnwidth}{@{\extracolsep{\fill}} p{0.53\columnwidth} >{\itshape}l r @{\extracolsep{\fill}}}
Darshan Singh Sunger & C & 453\\
Joanne Alexander-Sefre & LD & 143\\
\end{tabular*}

\subsection*{Tendring}

\subsubsection*{Great and Little Oakley \hspace*{\fill}\nolinebreak[1]%
\enspace\hspace*{\fill}
\finalhyphendemerits=0
[9th February; UKIP gain from Ind]}

\index{Great and Little Oakley , Tendring@Great \& Little Oakley, \emph{Tendring}}

Resignation of Tom Howard (Ind).

\noindent
\begin{tabular*}{\columnwidth}{@{\extracolsep{\fill}} p{0.53\columnwidth} >{\itshape}l r @{\extracolsep{\fill}}}
Mike Bush & UKIP & 216\\
Andy Erskine & C & 171\\
Robert Shephard & Lab & 117\\
Matthew Bensilum & LD & 83\\
\end{tabular*}

\subsubsection*{St James \hspace*{\fill}\nolinebreak[1]%
\enspace\hspace*{\fill}
\finalhyphendemerits=0
[6th April]}

\index{Saint James , Tendring@St James, \emph{Tendring}}

Death of John Hughes (Ind elected as UKIP).

\subsection*{Uttlesford}

R4U = Residents for Uttlesford

\subsubsection*{Elsenham and Henham (2) \hspace*{\fill}\nolinebreak[1]%
\enspace\hspace*{\fill}
\finalhyphendemerits=0
[16th February; 2 R4U gains from LD]}

\index{Elsenham and Henham , Uttlesford@Elsenham \& Henham, \emph{Uttlesford}}

Resignations of Rory Gleeson and Lizzie Parr (both LD).

\noindent
\begin{tabular*}{\columnwidth}{@{\extracolsep{\fill}} p{0.53\columnwidth} >{\itshape}l r @{\extracolsep{\fill}}}
Petrina Lees & R4U & 834\\
Garry LeCount & R4U & 716\\
Sinead Holland & LD & 316\\
Lorraine Flawn & LD & 259\\
Joe Rich & C & 141\\
Alexis Beeching & C & 120\\
Sharron Coker & UKIP & 68\\
David Allum & UKIP & 64\\
Hilary Todd & Lab & 39\\
Carl Steward & Lab & 28\\
Paul Allington & Grn & 8\\
Karmel Stannard & Grn & 6\\
\end{tabular*}

\section[Gloucestershire]{\sloppyword{Gloucestershire}}

\subsection*{County Council}

At the May 2017 ordinary election there was an unfilled vacancy in Drybrook and Lydbrook division due to the resignation of Colin Guyton (Ind elected as UKIP).
\index{Drybrook and Lydbrook , Gloucestershire@Drybrook \& Lydbrook, \emph{Glos.}}

\subsection*{Cotswold}

\subsubsection*{Fairford North \hspace*{\fill}\nolinebreak[1]%
\enspace\hspace*{\fill}
\finalhyphendemerits=0
[9th February; LD gain from C]}

\index{Fairford North , Cotswold@Fairford N., \emph{Cotswold}}

Resignation of Abagail Beccle (C).

\noindent
\begin{tabular*}{\columnwidth}{@{\extracolsep{\fill}} p{0.53\columnwidth} >{\itshape}l r @{\extracolsep{\fill}}}
Andrew Doherty & LD & 610\\
Dom Morris & C & 270\\
Xanthe Messenger & Grn & 16\\
\end{tabular*}

\subsection*{Forest of Dean}

\subsubsection*{Lydbrook and Ruardean \hspace*{\fill}\nolinebreak[1]%
\enspace\hspace*{\fill}
\finalhyphendemerits=0
[16th February; Grn gain from UKIP]}

\index{Lydbrook and Ruardean , Forest of Dean@Lydbrook \& Ruardean, \emph{Forest of Dean}}

Resignation of Colin Guyton (Ind elected as UKIP).

\noindent
\begin{tabular*}{\columnwidth}{@{\extracolsep{\fill}} p{0.53\columnwidth} >{\itshape}l r @{\extracolsep{\fill}}}
Sid Phelps & Grn & 360\\
Kevin White & C & 248\\
Karen Brown & Lab & 231\\
Roy Bardo & UKIP & 113\\
Heather Lusty & LD & 67\\
\end{tabular*}

\section{Hampshire}

\subsection*{County Council}

At the May 2017 ordinary election there was an unfilled vacancy in Bedhampton and Leigh Park division due to the resignation of Ray Finch (UKIP).
\index{Bedhampton and Leigh Park , Hampshire@Bedhampton \& Leigh Park, \emph{Hants.}}

\subsection*{Basingstoke and Deane}

\subsubsection*{Winklebury \hspace*{\fill}\nolinebreak[1]%
\enspace\hspace*{\fill}
\finalhyphendemerits=0
[Tuesday 21st February; Lab gain from C]}

\index{Winklebury , Basingstoke and Deane@Winklebury, \emph{Basingstoke \& Deane}}

Resignation of Joseph Smith (C).

\noindent
\begin{tabular*}{\columnwidth}{@{\extracolsep{\fill}} p{0.53\columnwidth} >{\itshape}l r @{\extracolsep{\fill}}}
Angie Freeman & Lab & 824\\
Chris Hendon & C & 472\\
Zoe-Marie Rogers & LD & 42\\
\end{tabular*}

\subsection*{Eastleigh}

\subsubsection*{Eastleigh Central \hspace*{\fill}\nolinebreak[1]%
\enspace\hspace*{\fill}
\finalhyphendemerits=0
[pending]}

\index{Eastleigh Central , Eastleigh@Eastleigh C., \emph{Eastleigh}}

Resignation of Keith Trenchard (LD).

\subsection*{Havant}

\subsubsection*{Emsworth \hspace*{\fill}\nolinebreak[1]%
\enspace\hspace*{\fill}
\finalhyphendemerits=0
[pending]}

\index{Emsworth , Havant@Emsworth, \emph{Havant}}

Death of Colin Mackey (C).

\section{Herefordshire}

\subsubsection*{Leominster South \hspace*{\fill}\nolinebreak[1]%
\enspace\hspace*{\fill}
\finalhyphendemerits=0
[23rd March]}

\index{Leominster South , Herefordshire@Leominster S., \emph{Herefs.}}

Death of Peter McCaull (Ind).

\section{Hertfordshire}

\subsection*{County Council}

At the May 2017 ordinary election there was an unfilled vacancy in Broadwater division due to the death of Sherma Batson (Lab).
\index{Broadwater , Hertfordshire@Broadwater, \emph{Herts.}}

\subsection*{Broxbourne}

\subsubsection*{Waltham Cross \hspace*{\fill}\nolinebreak[1]%
\enspace\hspace*{\fill}
\finalhyphendemerits=0
[9th March; C gain from Lab]}

\index{Waltham Cross , Broxbourne@Waltham Cross, \emph{Broxbourne}}

Death of Malcolm Aitken (Lab).

\noindent
\begin{tabular*}{\columnwidth}{@{\extracolsep{\fill}} p{0.53\columnwidth} >{\itshape}l r @{\extracolsep{\fill}}}
Patsy Spears & C & 650\\
Christian Durugo & Lab & 646\\
Steve Coster & UKIP & 200\\
Brendan Wyer & LD & 89\\
\end{tabular*}

\subsection*{Dacorum}

\subsubsection*{Berkhamsted West \hspace*{\fill}\nolinebreak[1]%
\enspace\hspace*{\fill}
\finalhyphendemerits=0
[pending]}

\index{Berkhamsted West , Dacorum@Berkhamsted W., \emph{Dacorum}}

Resignation of Julian Ashbourn (C).

\subsection*{East Hertfordshire}

\subsubsection*{Hertford Castle \hspace*{\fill}\nolinebreak[1]%
\enspace\hspace*{\fill}
\finalhyphendemerits=0
[9th March]}

\index{Hertford Castle , East Hertfordshire@Hertford Castle, \emph{E. Herts.}}

Resignation of Kevin Brush (C).

\noindent
\begin{tabular*}{\columnwidth}{@{\extracolsep{\fill}} p{0.53\columnwidth} >{\itshape}l r @{\extracolsep{\fill}}}
Linda Radford & C & 593\\
Veronica Fraser & Lab & 207\\
Freya Waterhouse & LD & 188\\
Tony Tarrega & Grn & 157\\
Mike Shaw & UKIP & 65\\
\end{tabular*}

\subsection*{North Hertfordshire}

\subsubsection*{Hitchin Priory \hspace*{\fill}\nolinebreak[1]%
\enspace\hspace*{\fill}
\finalhyphendemerits=0
[pending]}

\index{Hitchin Priory , North Hertfordshire@Hitchin Priory, \emph{N. Herts.}}

Resignation of Allison Ashley (C).

\subsubsection*{Royston Heath \hspace*{\fill}\nolinebreak[1]%
\enspace\hspace*{\fill}
\finalhyphendemerits=0
[pending]}

\index{Royston Heath , North Hertfordshire@Royston Heath, \emph{N. Herts.}}

Resignation of Peter Burt (C).

\subsection*{Stevenage}

\subsubsection*{Roebuck \hspace*{\fill}\nolinebreak[1]%
\enspace\hspace*{\fill}
\finalhyphendemerits=0
[pending]}

\index{Roebuck , Stevenage@Roebuck, \emph{Stevenage}}

Death of Sherma Batson (Lab).

\subsection*{Three Rivers}

\subsubsection*{Gade Valley \hspace*{\fill}\nolinebreak[1]%
\enspace\hspace*{\fill}
\finalhyphendemerits=0
[12th January; LD gain from C]}

\index{Gade Valley , Three Rivers@Gade Valley, \emph{Three Rivers}}

Death of Leslie Proctor (C).

\noindent
\begin{tabular*}{\columnwidth}{@{\extracolsep{\fill}} p{0.53\columnwidth} >{\itshape}l r @{\extracolsep{\fill}}}
Alex Michaels & LD & 626\\
Dee Ward & C & 196\\
Bruce Prochnik & Lab & 119\\
David Bennett & UKIP & 69\\
Roberta Curran & Grn & 18\\
\end{tabular*}

\section{Isle of Wight}

At the May 2017 ordinary election there was an unfilled vacancy in Arreton and Newchurch division due to the death of Colin Richards (Ind).
\index{Arreton and Newchurch , Isle of Wight@Arreton \& Newchurch, \emph{I.O.W.}}

\section{Kent}

\subsection*{Dover}

\subsubsection*{Maxton, Elms Vale and Priory \hspace*{\fill}\nolinebreak[1]%
\enspace\hspace*{\fill}
\finalhyphendemerits=0
[pending]}

\index{Maxton, Elms Vale and Priory , Dover@Maxton, Elms Vale \& Priory, \emph{Dover}}

Resignation of Andrew Richardson (UKIP).

\subsection*{Maidstone}

\subsubsection*{Bearsted \hspace*{\fill}\nolinebreak[1]%
\enspace\hspace*{\fill}
\finalhyphendemerits=0
[pending]}

\index{Bearsted , Maidstone@Bearsted, \emph{Maidstone}}

Resignation of Mike Revell (C).

\section[Lancashire]{\sloppyword{Lancashire}}

\subsection*{County Council}

At the May 2017 ordinary election there were unfilled vacancies in Great Harwood and Lancaster Rural North divisions due to the disqualification (non-attendance) of Gareth Molineux (Lab) and the resignation of Alycia James (C) respectively.
\index{Great Harwood , Lancashire@Great Harwood, \emph{Lancs.}}
\index{Lancaster Rural North , Lancashire@Lancaster Rural N., \emph{Lancs.}}

\subsection*{Blackburn with Darwen}

\subsubsection*{Higher Croft \hspace*{\fill}\nolinebreak[1]%
\enspace\hspace*{\fill}
\finalhyphendemerits=0
[23rd March]}

\index{Higher Croft , Blackburn with Darwen@Higher Croft, \emph{Blackburn with Darwen}}

Death of Don McKinlay (Lab).

\subsubsection*{Marsh House \hspace*{\fill}\nolinebreak[1]%
\enspace\hspace*{\fill}
\finalhyphendemerits=0
[pending]}

\index{Marsh House , Blackburn with Darwen@Marsh House, \emph{Blackburn with Darwen}}

Death of John Roberts (Lab).

\subsection*{Blackpool}

\subsubsection*{Warbreck \hspace*{\fill}\nolinebreak[1]%
\enspace\hspace*{\fill}
\finalhyphendemerits=0
[16th March]}

\index{Warbreck , Blackpool@Warbreck, \emph{Blackpool}}

Death of Tony Brown (C).

\subsection*{Fylde}

Fylde = Fylde Ratepayers

\subsubsection*{St Johns \hspace*{\fill}\nolinebreak[1]%
\enspace\hspace*{\fill}
\finalhyphendemerits=0
[9th February]}

\index{Saint Johns , Fylde@St Johns, \emph{Fylde}}

Disqualification of Mark Bamforth (Fylde): non-attendance.

\noindent
\begin{tabular*}{\columnwidth}{@{\extracolsep{\fill}} p{0.53\columnwidth} >{\itshape}l r @{\extracolsep{\fill}}}
Mark Bamforth & Fylde & 564\\
Paul Lomax & C & 278\\
Jayne Boardman & Lab & 45\\
Paul Hill & Grn & 40\\
\end{tabular*}

\subsection*{Preston}

\subsubsection*{Ashton \hspace*{\fill}\nolinebreak[1]%
\enspace\hspace*{\fill}
\finalhyphendemerits=0
[4th May]}

\index{Ashton , Preston@Ashton, \emph{Preston}}

Resignation of Angela Vodden (Lab).

\subsubsection*{Preston Rural East \hspace*{\fill}\nolinebreak[1]%
\enspace\hspace*{\fill}
\finalhyphendemerits=0
[4th May]}

\index{Preston Rural East , Preston@Preston Rural E., \emph{Preston}}

Death of Tom Davies (C).

\subsection*{South Ribble}

\subsubsection*{Walton-le-Dale East \hspace*{\fill}\nolinebreak[1]%
\enspace\hspace*{\fill}
\finalhyphendemerits=0
[16th March]}

\index{Walton-le-Dale East , South Ribble@Walton-le-Dale E., \emph{S. Ribble}}

Resignation of Andrea Ball (C).

\section{Leicestershire}

\subsection*{Melton}

\subsubsection*{Melton Sysonby \hspace*{\fill}\nolinebreak[1]%
\enspace\hspace*{\fill}
\finalhyphendemerits=0
[pending]}

\index{Melton Sysonby , Melton@Melton Sysonby, \emph{Melton}}

Death of Valerie Manderson (C).

\section{Lincolnshire}

\subsection*{County Council}

At the May 2017 ordinary election there was an unfilled vacancy in Mablethorpe division due to the resignation of Anne Reynolds (UKIP).
\index{Mablethorpe , Lincolnshire@Mablethorpe, \emph{Lincs.}}

\section{Norfolk}

\subsection*{County Council}

At the May 2017 ordinary election there was an unfilled vacancy in Lothingland division due to the death of Colin Aldred (UKIP).
\index{Lothingland , Norfolk@Lothingland, \emph{Norfolk}}

\subsection*{Breckland}

\subsubsection*{Saham Toney \hspace*{\fill}\nolinebreak[1]%
\enspace\hspace*{\fill}
\finalhyphendemerits=0
[16th March]}

\index{Saham Toney , Breckland@Saham Toney, \emph{Breckland}}

Resignation of Charles Carter (C).

\subsection*{King's Lynn and West Norfolk}

\subsubsection*{Fairstead \hspace*{\fill}\nolinebreak[1]%
\enspace\hspace*{\fill}
\finalhyphendemerits=0
[pending]}

\index{Fairstead , King's Lynn and West Norfolk@Fairstead, \emph{King's Lynn \& W. Norfolk}}

Death of Ian Gourlay (Lab).

\subsection*{North Norfolk}

\subsubsection*{Waterside \hspace*{\fill}\nolinebreak[1]%
\enspace\hspace*{\fill}
\finalhyphendemerits=0
[9th February; LD gain from C]}

\index{Waterside , North Norfolk@Waterside, \emph{N. Norfolk}}

Disqualification of Ben Jarvis (C): non-attendance.

\noindent
\begin{tabular*}{\columnwidth}{@{\extracolsep{\fill}} p{0.53\columnwidth} >{\itshape}l r @{\extracolsep{\fill}}}
Marion Millership & LD & 649\\
Tony Lumbard & C & 410\\
Barry Whitehouse & UKIP & 77\\
David Russell & Lab & 41\\
\end{tabular*}

\section[North Yorkshire]{\sloppyword{North Yorkshire}}

\subsection*{County Council}

At the May 2017 ordinary election there was an unfilled vacancy in Richmondshire North division due to the death of Michael Heseltine (C).
\index{Richmondshire North , North Yorkshire@Richmondshire N., \emph{N. Yorks.}}

\subsection*{Middlesbrough}

\subsubsection*{Coulby Newham \hspace*{\fill}\nolinebreak[1]%
\enspace\hspace*{\fill}
\finalhyphendemerits=0
[13th April]}

\index{Coulby Newham , Middlesbrough@Coulby Newham, \emph{Middlesbrough}}

Resignation of Geoff Cole (Lab).

\subsection*{Redcar and Cleveland}

\subsubsection*{Hutton \hspace*{\fill}\nolinebreak[1]%
\enspace\hspace*{\fill}
\finalhyphendemerits=0
[2nd March]}

\index{Hutton , Redcar and Cleveland@Hutton, \emph{Redcar \& Cleveland}}

Resignation of Valerie Halton (C).

\noindent
\begin{tabular*}{\columnwidth}{@{\extracolsep{\fill}} p{0.53\columnwidth} >{\itshape}l r @{\extracolsep{\fill}}}
Alma Thrower & C & 860\\
Graeme Kidd & LD & 326\\
Ian Urwin & Lab & 183\\
Barry Hudson & UKIP & 129\\
\end{tabular*}

\subsubsection*{Newcomen \hspace*{\fill}\nolinebreak[1]%
\enspace\hspace*{\fill}
\finalhyphendemerits=0
[2nd March]}

\index{Newcomen , Redcar and Cleveland@Newcomen, \emph{Redcar \& Cleveland}}

Death of Chris Abbott (LD).

\noindent
\begin{tabular*}{\columnwidth}{@{\extracolsep{\fill}} p{0.53\columnwidth} >{\itshape}l r @{\extracolsep{\fill}}}
Laura Benson & LD & 426\\
Charlie Brady & Lab & 259\\
Andrea Turner & UKIP & 153\\
Mark Hannon & Ind & 52\\
Dave Stones & Ind & 36\\
Maret Ward & C & 29\\
\end{tabular*}

\subsection*{Richmondshire}

\subsubsection*{Reeth and Arkengarthdale \hspace*{\fill}\nolinebreak[1]%
\enspace\hspace*{\fill}
\finalhyphendemerits=0
[pending]}

\index{Reeth and Arkengarthdale , Richmondshire@Reeth \& Arkengarthdale, \emph{Richmondshire}}

Death of Richard Beal (Ind).

\subsection*{York}

\subsubsection*{Micklegate \hspace*{\fill}\nolinebreak[1]%
\enspace\hspace*{\fill}
\finalhyphendemerits=0
[pending]}

\index{Micklegate , York@Micklegate, \emph{York}}

Resignation (effective by 31st March 2017) of Julie Gunnell (Lab).

\section[Northamptonshire]{\sloppyword{Northamptonshire}}

\subsection*{Corby}

\subsubsection*{Kingswood and Hazel Leys \hspace*{\fill}\nolinebreak[1]%
\enspace\hspace*{\fill}
\finalhyphendemerits=0
[9th February]}

\index{Kingswood and Hazel Leys , Corby@Kingswood \& Hazel Leys, \emph{Corby}}

Disqualification of Kenneth Carratt (Lab): non-attendance.

\noindent
\begin{tabular*}{\columnwidth}{@{\extracolsep{\fill}} p{0.53\columnwidth} >{\itshape}l r @{\extracolsep{\fill}}}
Isabel McNab & Lab & 610\\
Stan Heggs & C & 252\\
Michael Mahon & Grn & 82\\
\end{tabular*}

\subsection*{Daventry}

\subsubsection*{Long Buckby \hspace*{\fill}\nolinebreak[1]%
\enspace\hspace*{\fill}
\finalhyphendemerits=0
[pending]}

\index{Long Buckby , Daventry@Long Buckby, \emph{Daventry}}

Resignation of Diana Osborne (C).

\subsection*{East Northamptonshire}

\subsubsection*{Prebendal \hspace*{\fill}\nolinebreak[1]%
\enspace\hspace*{\fill}
\finalhyphendemerits=0
[pending]}

\index{Prebendal , East Northamptonshire@Prebendal, \emph{E. Northants.}}

Reason for vacancy not known: missing councillor is Valerie Raven-Hill (C).

\subsection*{Kettering}

\subsubsection*{Barton \hspace*{\fill}\nolinebreak[1]%
\enspace\hspace*{\fill}
\finalhyphendemerits=0
[23rd February; LD gain from C]}

\index{Barton , Kettering@Barton, \emph{Kettering}}

Resignation of Stephen Bellamy (C).

\noindent
\begin{tabular*}{\columnwidth}{@{\extracolsep{\fill}} p{0.53\columnwidth} >{\itshape}l r @{\extracolsep{\fill}}}
Andrew Dutton & LD & 644\\
Dianne Miles-Zanger & C & 337\\
Robert Clements & UKIP & 106\\
Rob Reeves & Grn & 42\\
\end{tabular*}

\subsubsection*{Burton Latimer \hspace*{\fill}\nolinebreak[1]%
\enspace\hspace*{\fill}
\finalhyphendemerits=0
[pending]}

\index{Burton Latimer , Kettering@Burton Latimer, \emph{Kettering}}

Resignation of Derek Zanger (C).

\subsection*{\sloppyword{South Northamptonshire}}

\subsubsection*{King's Sutton \hspace*{\fill}\nolinebreak[1]%
\enspace\hspace*{\fill}
\finalhyphendemerits=0
[pending]}

\index{King's Sutton , South Northamptonshire@King's Sutton, \emph{S. Northants.}}

Resignation of Ian Morris (C).

\section[Northumberland]{\sloppyword{Northumberland}}

At the May 2017 ordinary election there was an unfilled vacancy in Lynemouth division due to the death of Milburn Douglas (Lab).
\index{Lynemouth , Northumberland@Lynemouth, \emph{Northd.}}

\section{Nottinghamshire}

\subsection*{Rushcliffe}

\subsubsection*{Thoroton \hspace*{\fill}\nolinebreak[1]%
\enspace\hspace*{\fill}
\finalhyphendemerits=0
[pending]}

\index{Thoroton , Rushcliffe@Thoroton, \emph{Rushcliffe}}

Resignation of Adeline Pell (C).

\section{Oxfordshire}

\subsection*{Oxford}

\subsubsection*{Barton and Sandhills \hspace*{\fill}\nolinebreak[1]%
\enspace\hspace*{\fill}
\finalhyphendemerits=0
[pending]}

\index{Barton and Sandhills , Oxford@Barton \& Sandhills, \emph{Oxford}}

Death of Van Coulter (Lab).

\subsection*{West Oxfordshire}

\subsubsection*{Hailey, Minster Lovell and Leafield \hspace*{\fill}\nolinebreak[1]%
\enspace\hspace*{\fill}
\finalhyphendemerits=0
[9th March; LD gain from C]}

\index{Hailey, Minster Lovell and Leafield , West Oxfordshire@Hailey, Minster Lovell \& Leafield, \emph{W. Oxon.}}

Death of Warwick Robinson (C).

\noindent
\begin{tabular*}{\columnwidth}{@{\extracolsep{\fill}} p{0.53\columnwidth} >{\itshape}l r @{\extracolsep{\fill}}}
Kieran Mullins & LD & 567\\
Brendan Kay & C & 504\\
Calvert McGibbon & Lab & 71\\
Andrew Wright & Grn & 38\\
Jim Stanley & UKIP & 35\\
\end{tabular*}

\section{Rutland}

\subsubsection*{Exton \hspace*{\fill}\nolinebreak[1]%
\enspace\hspace*{\fill}
\finalhyphendemerits=0
[9th March]}

\index{Exton , Rutland@Exton, \emph{Rutland}}

Resignation of Terry King (C).

\noindent
\begin{tabular*}{\columnwidth}{@{\extracolsep{\fill}} p{0.53\columnwidth} >{\itshape}l r @{\extracolsep{\fill}}}
June Fox & C & 238\\
Joanna Burrows & LD & 123\\
Claire Barks & UKIP & 39\\
\end{tabular*}

\section{Somerset}

\subsection*{Bath and North East Somerset}

\subsubsection*{Walcot \hspace*{\fill}\nolinebreak[1]%
\enspace\hspace*{\fill}
\finalhyphendemerits=0
[6th April]}

\index{Walcot , Bath and North East Somerset@Walcot, \emph{Bath \& N.E. Somerset}}

Resignation of Lisa Brett (LD).

\subsection*{Sedgemoor}

\subsubsection*{Bridgwater Eastover \hspace*{\fill}\nolinebreak[1]%
\enspace\hspace*{\fill}
\finalhyphendemerits=0
[pending]}

\index{Bridgwater Eastover , Sedgemoor@Bridgwater Eastover, \emph{Sedgemoor}}

Resignation of Moira Brown (Lab).

\subsection*{South Somerset}

\subsubsection*{Blackmoor Vale \hspace*{\fill}\nolinebreak[1]%
\enspace\hspace*{\fill}
\finalhyphendemerits=0
[pending]}

\index{Blackmoor Vale , South Somerset@Blackmoor Vale, \emph{S. Somerset}}

Death of Tim Inglefield (C).

\subsection*{West Somerset}

\subsubsection*{Dunster and Timberscombe \hspace*{\fill}\nolinebreak[1]%
\enspace\hspace*{\fill}
\finalhyphendemerits=0
[23rd March]}

\index{Dunster and Timberscombe , West Somerset@Dunster \& Timberscombe, \emph{W. Somerset}}

Resignation of Bryan Leaker (C).

\section{Staffordshire}

\subsection*{East Staffordshire}

\subsubsection*{Town \hspace*{\fill}\nolinebreak[1]%
\enspace\hspace*{\fill}
\finalhyphendemerits=0
[2nd February]}

\index{Town , East Staffordshire@Town, \emph{E. Staffs.}}

Resignation of Karen Haberfield (C).

\noindent
\begin{tabular*}{\columnwidth}{@{\extracolsep{\fill}} p{0.53\columnwidth} >{\itshape}l r @{\extracolsep{\fill}}}
Philip Hudson & C & 627\\
Zdzislaw Krupski & Lab & 359\\
Norman Moir & UKIP & 213\\
\end{tabular*}

\subsubsection*{Burton \hspace*{\fill}\nolinebreak[1]%
\enspace\hspace*{\fill}
\finalhyphendemerits=0
[16th February]}

\index{Burton , East Staffordshire@Burton, \emph{E. Staffs.}}

Resignation of Michael Rodgers (LD).

\noindent
\begin{tabular*}{\columnwidth}{@{\extracolsep{\fill}} p{0.53\columnwidth} >{\itshape}l r @{\extracolsep{\fill}}}
Helen Hall & LD & 271\\
Phil Hutchinson & Lab & 127\\
Peter Levis & UKIP & 60\\
Hamid Asghar & C & 56\\
\end{tabular*}

\subsection*{Lichfield}

\subsubsection*{Fazeley \hspace*{\fill}\nolinebreak[1]%
\enspace\hspace*{\fill}
\finalhyphendemerits=0
[pending]}

\index{Fazeley , Lichfield@Fazeley, \emph{Lichfield}}

Resignation of John Mills (UKIP).

\subsection*{South Staffordshire}

\subsubsection*{Penkridge West \hspace*{\fill}\nolinebreak[1]%
\enspace\hspace*{\fill}
\finalhyphendemerits=0
[pending]}

\index{Penkridge West , South Staffordshire@Penkridge W., \emph{S. Staffs.}}

Resignation of Don Cartwright (C).

\subsection*{Stafford}

\subsubsection*{Highfields and Western Downs (2) \hspace*{\fill}\nolinebreak[1]%
\enspace\hspace*{\fill}
\finalhyphendemerits=0
[pending]}

\index{Highfields and Western Downs , Stafford@Highfields \& Western Downs, \emph{Stafford}}

Resignations of Maureen Bowen and Stephen O'Connor (both Lab).

\section{Suffolk}

\subsection*{County Council}

At the May 2017 ordinary election there was an unfilled vacancy in Lowestoft South division due to the resignation of Derek Hackett (UKIP).
\index{Lowestoft South , Suffolk@Lowestoft S., \emph{Suffolk}}

\subsection*{Ipswich}

\subsubsection*{Stoke Park \hspace*{\fill}\nolinebreak[1]%
\enspace\hspace*{\fill}
\finalhyphendemerits=0
[pending]}

\index{Stoke Park , Ipswich@Stoke Park, \emph{Ipswich}}

Resignation of Robin Hyde-Chambers (C).

\subsection*{Mid Suffolk}

\subsubsection*{Eye \hspace*{\fill}\nolinebreak[1]%
\enspace\hspace*{\fill}
\finalhyphendemerits=0
[pending]}

\index{Eye , Mid Suffolk@Eye, \emph{Mid Suffolk}}

Resignation of Charles Flatman (Ind).

\section{Surrey}

\subsection*{Spelthorne}

\subsubsection*{Ashford East \hspace*{\fill}\nolinebreak[1]%
\enspace\hspace*{\fill}
\finalhyphendemerits=0
[4th May]}

\index{Ashford East , Spelthorne@Ashford E., \emph{Spelthorne}}

Resignation of Chris Frazer (C).

\section[Warwickshire]{\sloppyword{Warwickshire}}

\subsection*{Rugby}

\subsubsection*{Coton and Boughton \hspace*{\fill}\nolinebreak[1]%
\enspace\hspace*{\fill}
\finalhyphendemerits=0
[pending]}

\index{Coton and Boughton , Rugby@Coton \& Boughton, \emph{Rugby}}

Resignation of Helen Taylor (C).

\subsubsection*{New Bilton \hspace*{\fill}\nolinebreak[1]%
\enspace\hspace*{\fill}
\finalhyphendemerits=0
[pending]}

\index{New Bilton , Rugby@New Bilton, \emph{Rugby}}

Resignation of Steve Birkett (Lab).

\subsection*{Stratford-on-Avon}

\subsubsection*{Red Horse \hspace*{\fill}\nolinebreak[1]%
\enspace\hspace*{\fill}
\finalhyphendemerits=0
[9th March]}

\index{Red Horse , Stratford-on-Avon@Red Horse, \emph{Stratford-on-Avon}}

Resignation of Bart Mura (C).

\noindent
\begin{tabular*}{\columnwidth}{@{\extracolsep{\fill}} p{0.53\columnwidth} >{\itshape}l r @{\extracolsep{\fill}}}
John Feilding & C & 476\\
Philip Vial & LD & 266\\
Edward Fila & UKIP & 92\\
Pat Hotson & Grn & 58\\
\end{tabular*}

\subsubsection*{Ettington \hspace*{\fill}\nolinebreak[1]%
\enspace\hspace*{\fill}
\finalhyphendemerits=0
[pending]}

\index{Ettington , Stratford-on-Avon@Ettington, \emph{Stratford-on-Avon}}

Resignation of Philip Seccombe (C).

\section[West Sussex]{\sloppyword{West Sussex}}

\subsection*{County Council}

At the May 2017 ordinary election there were unfilled vacancies in East Grinstead South and Ashurst Wood, and Roffey divisions due to the resignations of John O'Brien and Jim Rae (both C) respectively.
\index{East Grinstead South and Ashurst Wood , West Sussex@East Grinstead S. \& Ashurst Wood, \emph{W. Sussex}}
\index{Roffey , West Sussex@Roffey, \emph{W. Sussex}}

\subsection*{Mid Sussex}

\subsubsection*{Bolney \hspace*{\fill}\nolinebreak[1]%
\enspace\hspace*{\fill}
\finalhyphendemerits=0
[pending]}

\index{Bolney , Mid Sussex@Bolney, \emph{Mid Sussex}}

Resignation of John Allen (C).

\subsubsection*{Burgess Hill Franklands \hspace*{\fill}\nolinebreak[1]%
\enspace\hspace*{\fill}
\finalhyphendemerits=0
[pending]}

\index{Burgess Hill Franklands , Mid Sussex@Burgess Hill Franklands, \emph{Mid Sussex}}

Resignation of Chris King (C).

\section{Wiltshire}

\subsection*{Wiltshire}

At the May 2017 ordinary election there was an unfilled vacancy in Trowbridge Lambrok division due to the resignation of Helen Osborn (Ind).
\index{Trowbridge Lambrok , Wiltshire@Trowbridge Lambrok, \emph{Wilts.}}

\section[Worcestershire]{\sloppyword{Worcestershire}}

\subsection*{County Council}

At the May 2017 ordinary election there was an unfilled vacancy in Redditch South division due to the resignation of Philip Gretton (C).
\index{Redditch South , Worcestershire@Redditch S., \emph{Worcs.}}

\subsection*{Bromsgrove}

\subsubsection*{Norton \hspace*{\fill}\nolinebreak[1]%
\enspace\hspace*{\fill}
\finalhyphendemerits=0
[19th January]}

\index{Norton , Bromsgrove@Norton, \emph{Bromsgrove}}

Death of Pete Lammas (C).

\noindent
\begin{tabular*}{\columnwidth}{@{\extracolsep{\fill}} p{0.53\columnwidth} >{\itshape}l r @{\extracolsep{\fill}}}
Michael Webb & C & 219\\
Rory Shannon & Lab & 186\\
Adrian Smart & UKIP & 82\\
Michelle Baker & Grn & 20\\
\end{tabular*}

\subsection*{Malvern Hills}

\subsubsection*{West \hspace*{\fill}\nolinebreak[1]%
\enspace\hspace*{\fill}
\finalhyphendemerits=0
[pending]}

\index{West , Malvern Hills@West, \emph{Malvern Hills}}

Resignation of Julian Roskams (Grn).

\subsection*{Worcester}

\subsubsection*{Bedwardine \hspace*{\fill}\nolinebreak[1]%
\enspace\hspace*{\fill}
\finalhyphendemerits=0
[pending]}

\index{Bedwardine , Worcester@Bedwardine, \emph{Worcester}}

Resignation of David Wilkinson (C).

\subsection*{Wychavon}

\subsubsection*{Evesham South \hspace*{\fill}\nolinebreak[1]%
\enspace\hspace*{\fill}
\finalhyphendemerits=0
[pending]}

\index{Evesham South , Wychavon@Evesham S., \emph{Wychavon}}

Reason for vacancy not known: missing councillor is Kenneth Barclay-Timmis (C).

\subsection*{Wyre Forest}

\subsubsection*{Mitton \hspace*{\fill}\nolinebreak[1]%
\enspace\hspace*{\fill}
\finalhyphendemerits=0
[pending]}

\index{Mitton , Wyre Forest@Mitton, \emph{Wyre Forest}}

Resignation of Sara Fearn (C).

\section[Glamorgan]{\sloppyword{Glamorgan}}

\subsection*{Bridgend}

At the May 2017 ordinary election there was an unfilled vacancy in Cornelly division due to the resignation of Megan Butcher (Ind).
\index{Cornelly , Bridgend@Cornelly, \emph{Bridgend}}

\subsection*{Cardiff}

At the May 2017 ordinary election there were unfilled vacancies in Cyncoed, Llandaff North and Llanrumney divisions due to the resignations of Margaret Jones (LD) and Sue White (Lab) and the death of Derrick Morgan (Lab) respectively.
\index{Cyncoed , Cardiff@Cyncoed, \emph{Cardiff}}
\index{Llandaff North , Cardiff@Llandaff N., \emph{Cardiff}}
\index{Llanrumney , Cardiff@Llanrumney, \emph{Cardiff}}

\section[Mid and West Wales]{\sloppyword{Mid and West Wales}}

\subsection*{Powys}

At the May 2017 ordinary election there was an unfilled vacancy in Builth division due to the resignation of Avril York (Ind).
\index{Builth , Powys@Builth, \emph{Powys}}

\section[North Wales]{\sloppyword{North Wales}}

\subsection*{Denbighshire}

At the May 2017 ordinary election there was an unfilled vacancy in Denbigh Lower division due to the death of John Bartley (Ind).
\index{Denbigh Lower , Denbighshire@Denbigh Lower, \emph{Denbighshire}}

\subsection*{Flintshire}

At the May 2017 ordinary election there was an unfilled vacancy in Caerwys division due to the resignation of Jim Falshaw (C).
\index{Caerwys , Flintshire@Caerwys, \emph{Flintshire}}

\subsection*{Isle of Anglesey}

At the May 2017 ordinary election there was an unfilled vacancy in Lligwy division due to the resignation of Derlwyn Hughes (Ind).
\index{Lligwy , Isle of Anglesey@Lligwy, \emph{Isle of Anglesey}}

\section[Aberdeen City and Shire]{\sloppyword{Aberdeen City and Shire}}

\subsection*{Aberdeenshire}

At the May 2017 ordinary election there were unfilled vacancies in Aboyne, Upper Deeside and Donside; and Peterhead South and Cruden wards due to the resignations of Katrina Farquhar (C) and Tom Malone (Ind) respectively.
\index{Aboyne, Upper Deeside and Donside , Aberdeenshire@Aboyne, Upper Deeside \& Donside, \emph{Aberdeenshire}}
\index{Peterhead South and Cruden , Aberdeenshire@Peterhead S. \& Cruden, \emph{Aberdeenshire}}

\section[Ayrshire Councils]{\sloppyword{Ayrshire Councils}}

\subsection*{East Ayrshire}

Libtn = Libertarian Party

\subsubsection*{Kilmarnock East and Hurlford \hspace*{\fill}\nolinebreak[1]%
\enspace\hspace*{\fill}
\finalhyphendemerits=0
[26th January]}

\index{Kilmarnock East and Hurlford , East Ayrshire@Kilmarnock E. \& Hurlford, \emph{E. Ayrshire}}

Death of James Buchanan (SNP).

\noindent
\begin{tabular*}{\columnwidth}{@{\extracolsep{\fill}} p{0.53\columnwidth} >{\itshape}l r @{\extracolsep{\fill}}}
\emph{First preferences}\\
Fiona Campbell & SNP & 1461\\
Dave Meechan & Lab & 881\\
Jon Herd & C & 602\\
Stephen McNamara & Libtn & 53\\
\end{tabular*}

\noindent
\begin{tabular*}{\columnwidth}{@{\extracolsep{\fill}} p{0.53\columnwidth} >{\itshape}l r @{\extracolsep{\fill}}}
\emph{Herd and McNamara eliminated}\\
Fiona Campbell & SNP & 1531\\
Dave Meechan & Lab & 1122\\
\end{tabular*}

\section[Clyde Councils]{\sloppyword{Clyde Councils}}

\subsection*{Glasgow}

At the May 2017 ordinary election there was an unfilled vacancy in Canal ward due to the resignation of Chris Kelly (Lab).
\index{Canal , Glasgow@Canal, \emph{Glasgow}}

\subsection*{Renfrewshire}

At the May 2017 ordinary election there was an unfilled vacancy in Houston, Crosslee and Linwood ward due to the resignation of Stuart Clark (Lab).
\index{Houston, Crosslee and Linwood , Renfrewshire@Houston, Crosslee \& Linwood, \emph{Renfs.}}

\section[Forth Councils]{\sloppyword{Forth Councils}}

\subsection*{Fife}

At the May 2017 ordinary election there was an unfilled vacancy in Kirkcaldy East ward due to the death of Kay Carrington (Lab).
\index{Kirkcaldy East , Fife@Kirkcaldy E., \emph{Fife}}

\section[Highland Councils]{\sloppyword{Highland Councils}}

\subsection*{Highland}

At the May 2017 ordinary election there was an unfilled vacancy in Wick ward due to the resignation of Gail Ross (SNP).
\index{Wick , Highland@Wick, \emph{Highland}}

\end{resultsiii}

% Index:
\clearpage
\phantomsection
%\addcontentsline{toc}{chapter}{Index of Wards}
{\scriptsize%\raggedright
\frenchspacing\printindex}
\thispagestyle{plain}

%% GNU Free Documentation License
\chapter*{{GNU Free Documentation License}}
\phantomsection % so hyperref creates bookmarks
\addcontentsline{toc}{chapter}{GNU Free Documentation License}
%\label{label_fdl}
\pagestyle{plain}

 Version 1.3, 3 November 2008


 Copyright \copyright{} 2000, 2001, 2002, 2007, 2008 Free Software Foundation, Inc.
 
 \bigskip
 
 <\url{http://fsf.org/}>
 
 \bigskip
 
 Everyone is permitted to copy and distribute verbatim copies
 of this license document, but changing it is not allowed.

\begin{results}
\tiny

\subsubsection*{Preamble}

The purpose of this License is to make a manual, textbook, or other
functional and useful document ``free'' in the sense of freedom: to
assure everyone the effective freedom to copy and redistribute it,
with or without modifying it, either commercially or noncommercially.
Secondarily, this License preserves for the author and publisher a way
to get credit for their work, while not being considered responsible
for modifications made by others.

This License is a kind of ``copyleft'', which means that derivative
works of the document must themselves be free in the same sense. It
complements the GNU General Public License, which is a copyleft
license designed for free software.

We have designed this License in order to use it for manuals for free
software, because free software needs free documentation: a free
program should come with manuals providing the same freedoms that the
software does. But this License is not limited to software manuals;
it can be used for any textual work, regardless of subject matter or
whether it is published as a printed book. We recommend this License
principally for works whose purpose is instruction or reference.

\subsubsection*{1. Applicability and definitions}

This License applies to any manual or other work, in any medium, that
contains a notice placed by the copyright holder saying it can be
distributed under the terms of this License. Such a notice grants a
world-wide, royalty-free license, unlimited in duration, to use that
work under the conditions stated herein. The ``\emph{Document}'', below,
refers to any such manual or work. Any member of the public is a
licensee, and is addressed as ``\emph{you}''. You accept the license if you
copy, modify or distribute the work in a way requiring permission
under copyright law.

A ``\emph{Modified Version}'' of the Document means any work containing the
Document or a portion of it, either copied verbatim, or with
modifications and/or translated into another language.

A ``\emph{Secondary Section}'' is a named appendix or a front-matter section of
the Document that deals exclusively with the relationship of the
publishers or authors of the Document to the Document's overall subject
(or to related matters) and contains nothing that could fall directly
within that overall subject. (Thus, if the Document is in part a
textbook of mathematics, a Secondary Section may not explain any
mathematics.) The relationship could be a matter of historical
connection with the subject or with related matters, or of legal,
commercial, philosophical, ethical or political position regarding
them.

The ``\emph{Invariant Sections}'' are certain Secondary Sections whose titles
are designated, as being those of Invariant Sections, in the notice
that says that the Document is released under this License. If a
section does not fit the above definition of Secondary then it is not
allowed to be designated as Invariant. The Document may contain zero
Invariant Sections. If the Document does not identify any Invariant
Sections then there are none.

The ``\emph{Cover Texts}'' are certain short passages of text that are listed,
as Front-Cover Texts or Back-Cover Texts, in the notice that says that
the Document is released under this License. A Front-Cover Text may
be at most 5 words, and a Back-Cover Text may be at most 25 words.

A ``\emph{Transparent}'' copy of the Document means a machine-readable copy,
represented in a format whose specification is available to the
general public, that is suitable for revising the document
straightforwardly with generic text editors or (for images composed of
pixels) generic paint programs or (for drawings) some widely available
drawing editor, and that is suitable for input to text formatters or
for automatic translation to a variety of formats suitable for input
to text formatters. A copy made in an otherwise Transparent file
format whose markup, or absence of markup, has been arranged to thwart
or discourage subsequent modification by readers is not Transparent.
An image format is not Transparent if used for any substantial amount
of text. A copy that is not ``Transparent'' is called ``\emph{Opaque}''.

Examples of suitable formats for Transparent copies include plain
ASCII without markup, Texinfo input format, \LaTeX{} input format, SGML
or XML using a publicly available DTD, and standard-conforming simple
HTML, PostScript or PDF designed for human modification. Examples of
transparent image formats include PNG, XCF and JPG. Opaque formats
include proprietary formats that can be read and edited only by
proprietary word processors, SGML or XML for which the DTD and/or
processing tools are not generally available, and the
machine-generated HTML, PostScript or PDF produced by some word
processors for output purposes only.

The ``\emph{Title Page}'' means, for a printed book, the title page itself,
plus such following pages as are needed to hold, legibly, the material
this License requires to appear in the title page. For works in
formats which do not have any title page as such, ``Title Page'' means
the text near the most prominent appearance of the work's title,
preceding the beginning of the body of the text.

The ``\emph{publisher}'' means any person or entity that distributes
copies of the Document to the public.

A section ``\emph{Entitled XYZ}'' means a named subunit of the Document whose
title either is precisely XYZ or contains XYZ in parentheses following
text that translates XYZ in another language. (Here XYZ stands for a
specific section name mentioned below, such as ``\emph{Acknowledgements}'',
``\emph{Dedications}'', ``\emph{Endorsements}'', or ``\emph{History}''.) 
To ``\emph{Preserve the Title}''
of such a section when you modify the Document means that it remains a
section ``Entitled XYZ'' according to this definition.

The Document may include Warranty Disclaimers next to the notice which
states that this License applies to the Document. These Warranty
Disclaimers are considered to be included by reference in this
License, but only as regards disclaiming warranties: any other
implication that these Warranty Disclaimers may have is void and has
no effect on the meaning of this License.

\subsubsection*{2. Verbatim copying}

You may copy and distribute the Document in any medium, either
commercially or noncommercially, provided that this License, the
copyright notices, and the license notice saying this License applies
to the Document are reproduced in all copies, and that you add no other
conditions whatsoever to those of this License. You may not use
technical measures to obstruct or control the reading or further
copying of the copies you make or distribute. However, you may accept
compensation in exchange for copies. If you distribute a large enough
number of copies you must also follow the conditions in section~3.

You may also lend copies, under the same conditions stated above, and
you may publicly display copies.

\subsubsection*{3. Copying in quantity}

If you publish printed copies (or copies in media that commonly have
printed covers) of the Document, numbering more than 100, and the
Document's license notice requires Cover Texts, you must enclose the
copies in covers that carry, clearly and legibly, all these Cover
Texts: Front-Cover Texts on the front cover, and Back-Cover Texts on
the back cover. Both covers must also clearly and legibly identify
you as the publisher of these copies. The front cover must present
the full title with all words of the title equally prominent and
visible. You may add other material on the covers in addition.
Copying with changes limited to the covers, as long as they preserve
the title of the Document and satisfy these conditions, can be treated
as verbatim copying in other respects.

If the required texts for either cover are too voluminous to fit
legibly, you should put the first ones listed (as many as fit
reasonably) on the actual cover, and continue the rest onto adjacent
pages.

If you publish or distribute Opaque copies of the Document numbering
more than 100, you must either include a machine-readable Transparent
copy along with each Opaque copy, or state in or with each Opaque copy
a computer-network location from which the general network-using
public has access to download using public-standard network protocols
a complete Transparent copy of the Document, free of added material.
If you use the latter option, you must take reasonably prudent steps,
when you begin distribution of Opaque copies in quantity, to ensure
that this Transparent copy will remain thus accessible at the stated
location until at least one year after the last time you distribute an
Opaque copy (directly or through your agents or retailers) of that
edition to the public.

It is requested, but not required, that you contact the authors of the
Document well before redistributing any large number of copies, to give
them a chance to provide you with an updated version of the Document.

\subsubsection*{4. Modifications}

You may copy and distribute a Modified Version of the Document under
the conditions of sections 2 and 3 above, provided that you release
the Modified Version under precisely this License, with the Modified
Version filling the role of the Document, thus licensing distribution
and modification of the Modified Version to whoever possesses a copy
of it. In addition, you must do these things in the Modified Version:

\begin{itemize}
\item[A.] 
 Use in the Title Page (and on the covers, if any) a title distinct
 from that of the Document, and from those of previous versions
 (which should, if there were any, be listed in the History section
 of the Document). You may use the same title as a previous version
 if the original publisher of that version gives permission.
 
\item[B.]
 List on the Title Page, as authors, one or more persons or entities
 responsible for authorship of the modifications in the Modified
 Version, together with at least five of the principal authors of the
 Document (all of its principal authors, if it has fewer than five),
 unless they release you from this requirement.
 
\item[C.]
 State on the Title page the name of the publisher of the
 Modified Version, as the publisher.
 
\item[D.]
 Preserve all the copyright notices of the Document.
 
\item[E.]
 Add an appropriate copyright notice for your modifications
 adjacent to the other copyright notices.
 
\item[F.]
 Include, immediately after the copyright notices, a license notice
 giving the public permission to use the Modified Version under the
 terms of this License, in the form shown in the Addendum below.
 
\item[G.]
 Preserve in that license notice the full lists of Invariant Sections
 and required Cover Texts given in the Document's license notice.
 
\item[H.]
 Include an unaltered copy of this License.
 
\item[I.]
 Preserve the section Entitled ``History'', Preserve its Title, and add
 to it an item stating at least the title, year, new authors, and
 publisher of the Modified Version as given on the Title Page. If
 there is no section Entitled ``History'' in the Document, create one
 stating the title, year, authors, and publisher of the Document as
 given on its Title Page, then add an item describing the Modified
 Version as stated in the previous sentence.
 
\item[J.]
 Preserve the network location, if any, given in the Document for
 public access to a Transparent copy of the Document, and likewise
 the network locations given in the Document for previous versions
 it was based on. These may be placed in the ``History'' section.
 You may omit a network location for a work that was published at
 least four years before the Document itself, or if the original
 publisher of the version it refers to gives permission.
 
\item[K.]
 For any section Entitled ``Acknowledgements'' or ``Dedications'',
 Preserve the Title of the section, and preserve in the section all
 the substance and tone of each of the contributor acknowledgements
 and/or dedications given therein.
 
\item[L.]
 Preserve all the Invariant Sections of the Document,
 unaltered in their text and in their titles. Section numbers
 or the equivalent are not considered part of the section titles.
 
\item[M.]
 Delete any section Entitled ``Endorsements''. Such a section
 may not be included in the Modified Version.
 
\item[N.]
 Do not retitle any existing section to be Entitled ``Endorsements''
 or to conflict in title with any Invariant Section.
 
\item[O.]
 Preserve any Warranty Disclaimers.
\end{itemize}

If the Modified Version includes new front-matter sections or
appendices that qualify as Secondary Sections and contain no material
copied from the Document, you may at your option designate some or all
of these sections as invariant. To do this, add their titles to the
list of Invariant Sections in the Modified Version's license notice.
These titles must be distinct from any other section titles.

You may add a section Entitled ``Endorsements'', provided it contains
nothing but endorsements of your Modified Version by various
parties---for example, statements of peer review or that the text has
been approved by an organization as the authoritative definition of a
standard.

You may add a passage of up to five words as a Front-Cover Text, and a
passage of up to 25 words as a Back-Cover Text, to the end of the list
of Cover Texts in the Modified Version. Only one passage of
Front-Cover Text and one of Back-Cover Text may be added by (or
through arrangements made by) any one entity. If the Document already
includes a cover text for the same cover, previously added by you or
by arrangement made by the same entity you are acting on behalf of,
you may not add another; but you may replace the old one, on explicit
permission from the previous publisher that added the old one.

The author(s) and publisher(s) of the Document do not by this License
give permission to use their names for publicity for or to assert or
imply endorsement of any Modified Version.

\subsubsection*{5. Combining documents}

You may combine the Document with other documents released under this
License, under the terms defined in section~4 above for modified
versions, provided that you include in the combination all of the
Invariant Sections of all of the original documents, unmodified, and
list them all as Invariant Sections of your combined work in its
license notice, and that you preserve all their Warranty Disclaimers.

The combined work need only contain one copy of this License, and
multiple identical Invariant Sections may be replaced with a single
copy. If there are multiple Invariant Sections with the same name but
different contents, make the title of each such section unique by
adding at the end of it, in parentheses, the name of the original
author or publisher of that section if known, or else a unique number.
Make the same adjustment to the section titles in the list of
Invariant Sections in the license notice of the combined work.

In the combination, you must combine any sections Entitled ``History''
in the various original documents, forming one section Entitled
``History''; likewise combine any sections Entitled ``Acknowledgements'',
and any sections Entitled ``Dedications''. You must delete all sections
Entitled ``Endorsements''.

\subsubsection*{6. Collections of documents}

You may make a collection consisting of the Document and other documents
released under this License, and replace the individual copies of this
License in the various documents with a single copy that is included in
the collection, provided that you follow the rules of this License for
verbatim copying of each of the documents in all other respects.

You may extract a single document from such a collection, and distribute
it individually under this License, provided you insert a copy of this
License into the extracted document, and follow this License in all
other respects regarding verbatim copying of that document.

\subsubsection*{7. Aggregation with independent works}

A compilation of the Document or its derivatives with other separate
and independent documents or works, in or on a volume of a storage or
distribution medium, is called an ``aggregate'' if the copyright
resulting from the compilation is not used to limit the legal rights
of the compilation's users beyond what the individual works permit.
When the Document is included in an aggregate, this License does not
apply to the other works in the aggregate which are not themselves
derivative works of the Document.

If the Cover Text requirement of section~3 is applicable to these
copies of the Document, then if the Document is less than one half of
the entire aggregate, the Document's Cover Texts may be placed on
covers that bracket the Document within the aggregate, or the
electronic equivalent of covers if the Document is in electronic form.
Otherwise they must appear on printed covers that bracket the whole
aggregate.

\subsubsection*{8. Translation}

Translation is considered a kind of modification, so you may
distribute translations of the Document under the terms of section~4.
Replacing Invariant Sections with translations requires special
permission from their copyright holders, but you may include
translations of some or all Invariant Sections in addition to the
original versions of these Invariant Sections. You may include a
translation of this License, and all the license notices in the
Document, and any Warranty Disclaimers, provided that you also include
the original English version of this License and the original versions
of those notices and disclaimers. In case of a disagreement between
the translation and the original version of this License or a notice
or disclaimer, the original version will prevail.

If a section in the Document is Entitled ``Acknowledgements'',
``Dedications'', or ``History'', the requirement (section~4) to Preserve
its Title (section~1) will typically require changing the actual
title.

\subsubsection*{9. Termination}

You may not copy, modify, sublicense, or distribute the Document
except as expressly provided under this License. Any attempt
otherwise to copy, modify, sublicense, or distribute it is void, and
will automatically terminate your rights under this License.

However, if you cease all violation of this License, then your license
from a particular copyright holder is reinstated (a) provisionally,
unless and until the copyright holder explicitly and finally
terminates your license, and (b) permanently, if the copyright holder
fails to notify you of the violation by some reasonable means prior to
60 days after the cessation.

Moreover, your license from a particular copyright holder is
reinstated permanently if the copyright holder notifies you of the
violation by some reasonable means, this is the first time you have
received notice of violation of this License (for any work) from that
copyright holder, and you cure the violation prior to 30 days after
your receipt of the notice.

Termination of your rights under this section does not terminate the
licenses of parties who have received copies or rights from you under
this License. If your rights have been terminated and not permanently
reinstated, receipt of a copy of some or all of the same material does
not give you any rights to use it.

\subsubsection*{10. Future revisions of this License}

The Free Software Foundation may publish new, revised versions
of the GNU Free Documentation License from time to time. Such new
versions will be similar in spirit to the present version, but may
differ in detail to address new problems or concerns. See
\url{http://www.gnu.org/copyleft/}.

Each version of the License is given a distinguishing version number.
If the Document specifies that a particular numbered version of this
License ``or any later version'' applies to it, you have the option of
following the terms and conditions either of that specified version or
of any later version that has been published (not as a draft) by the
Free Software Foundation. If the Document does not specify a version
number of this License, you may choose any version ever published (not
as a draft) by the Free Software Foundation. If the Document
specifies that a proxy can decide which future versions of this
License can be used, that proxy's public statement of acceptance of a
version permanently authorizes you to choose that version for the
Document.

\subsubsection*{11. Relicensing}

``Massive Multiauthor Collaboration Site'' (or ``MMC Site'') means any
World Wide Web server that publishes copyrightable works and also
provides prominent facilities for anybody to edit those works. A
public wiki that anybody can edit is an example of such a server. A
``Massive Multiauthor Collaboration'' (or ``MMC'') contained in the
site means any set of copyrightable works thus published on the MMC
site.

``CC-BY-SA'' means the Creative Commons Attribution-Share Alike 3.0
license published by Creative Commons Corporation, a not-for-profit
corporation with a principal place of business in San Francisco,
California, as well as future copyleft versions of that license
published by that same organization.

``Incorporate'' means to publish or republish a Document, in whole or
in part, as part of another Document.

An MMC is ``eligible for relicensing'' if it is licensed under this
License, and if all works that were first published under this License
somewhere other than this MMC, and subsequently incorporated in whole
or in part into the MMC, (1) had no cover texts or invariant sections,
and (2) were thus incorporated prior to November 1, 2008.

The operator of an MMC Site may republish an MMC contained in the site
under CC-BY-SA on the same site at any time before August 1, 2009,
provided the MMC is eligible for relicensing.

\end{results}

\end{document}


\renewcommand\resultsyear{2018}

\part{2018}

%\chapter{Referendums in 2018}
%
%There were no referendums in 2018.

%\section{Burnley mayoral referendum}
%
%A referendum was held in Burnley on 4th May on the question of whether the district should have a directly elected mayor.
%
%\noindent
%\begin{tabular*}{\columnwidth}{@{\extracolsep{\fill}} p{0.545\columnwidth} >{\itshape}l r @{\extracolsep{\fill}}}
%& Yes & 8694\\
%& No & 10986\\
%\end{tabular*}

%\part{By-elections in 2018}

\chapter{Parliamentary by-elections}

There were two parliamentary by-elections in 2018.

DVP = Democrats and Veterans Party

ForBritn = For Britain Movement

Libtn = Libertarian Party

Radical = Radical Party

WomensEq = Women's Equality Party

YPP = Young People's Party
%
%CommLg = Communist League
%
%CPA = Christian Peoples Alliance
%
\section*{West Tyrone \hspace*{\fill}\nolinebreak[1]%
\enspace\hspace*{\fill}
\finalhyphendemerits=0
[3rd May]}

\index{West Tyrone , House of Commons@West Tyrone, \emph{House of Commons}}

Resignation of Barry McElduff (SF).

\noindent
\begin{tabular*}{\columnwidth}{@{\extracolsep{\fill}} p{0.53\columnwidth} >{\itshape}l r @{\extracolsep{\fill}}}
Órfhlaith Begley & SF & 16346\\
Thomas Buchanan & DUP & 8390\\
Daniel McCrossan & SDLP & 6254\\
Chris Smyth & UUP & 2909\\
Stephen Donnelly & All & 1130\\
\end{tabular*}

\section*{Lewisham East \hspace*{\fill}\nolinebreak[1]%
\enspace\hspace*{\fill}
\finalhyphendemerits=0
[14th June]}

\index{Lewisham East , House of Commons@Lewisham E., \emph{House of Commons}}

Resignation of Heidi Alexander (Lab).

\noindent
\begin{tabular*}{\columnwidth}{@{\extracolsep{\fill}} p{0.53\columnwidth} >{\itshape}l r @{\extracolsep{\fill}}}
Janet Daby & Lab & 11033\\
Lucy Salek & LD & 5404\\
Ross Archer & C & 3161\\
Rosamund Adoo-Kissi-Debrah & Grn & 788\\
Mandu Reid & WomenEq & 506\\
David Kurten & UKIP & 380\\
Anne Marie Waters & ForBritn & 266\\
Maureen Martin & CPA & 168\\
Howling Laud Hope & Loony & 93\\
Massimo Dimambro & DVP & 67\\
Sean Finch & Libtn & 38\\
Charles Carey & Ind & 37\\
Patrick Gray & Radical & 20\\
Thomas Hall & YPP & 18\\
\end{tabular*}

%\noindent
%\begin{tabular*}{\columnwidth}{@{\extracolsep{\fill}} p{0.53\columnwidth} >{\itshape}l r @{\extracolsep{\fill}}}
%Trudy Harrison & C & 13748\\
%Gillian Troughton & Lab & 11601\\
%Rebecca Hanson & LD & 2252\\
%Fiona Mills & UKIP & 2025\\
%Michael Guest & Ind & 811\\
%Jack Lenox & Grn & 515\\
%Roy Ivinson & Ind & 116\\
%\end{tabular*}
%
%\section*{Stoke-on-Trent Central \hspace*{\fill}\nolinebreak[1]%
%\enspace\hspace*{\fill}
%\finalhyphendemerits=0
%[23rd February]}
%
%\index{Stoke-on-Trent Central , House of Commons@Stoke-on-Trent C., \emph{House of Commons}}
%
%Resignation of Tristram Hunt (Lab).
%
%\noindent
%\begin{tabular*}{\columnwidth}{@{\extracolsep{\fill}} p{0.53\columnwidth} >{\itshape}l r @{\extracolsep{\fill}}}
%Gareth Snell & Lab & 7853\\
%Paul Nuttall & UKIP & 5233\\
%Jack Brereton & C & 5154\\
%Zulfiqar Ali & LD & 2083\\
%Adam Colclough & Grn & 294\\
%Barbara Fielding-Morriss & Ind & 137\\
%The Incredible Flying Brick & Loony & 127\\
%David Furness & BNP & 124\\
%Godfrey Davies & CPA & 109\\
%Mohammed Akram & Ind & 56\\
%\end{tabular*}
%
%At the dissolution of Parliament on 3rd May 2017 there was an unfilled vacancy in Manchester Gorton due to the death of Sir Gerald Kaufman (Lab).  A by-election to fill this seat had been scheduled for 4th May but was superseded by the dissolution.
%\index{Manchester Gorton , House of Commons@Manchester Gorton, \emph{House of Commons}}

\chapter{By-elections to devolved assemblies, the European Parliament, and police and crime commissionerships}

\section{Greater London Authority}

There were no by-elections in 2018 to the Greater London Authority.

%Kemi Badenoch (C, London-wide list) resigned in June 2017.  She was replaced from the list by Susan Hall.

\section{National Assembly for Wales}

There was one by-election in 2018 to the National Assembly for Wales.

\subsection*{Alyn and Deeside \hspace*{\fill}\nolinebreak[1]%
\enspace\hspace*{\fill}
\finalhyphendemerits=0
[Tuesday 6th February]}

\index{Alyn and Deeside , Welsh Assembly@Alyn \& Deeside, \emph{Welsh Assembly}}

Death of Carl Sargeant (Lab).

\noindent
\begin{tabular*}{\columnwidth}{@{\extracolsep{\fill}} p{0.53\columnwidth} >{\itshape}l r @{\extracolsep{\fill}}}
Jack Sargeant & Lab & 11267\\
Sarah Atherton & C & 4722\\
Donna Lalek & LD & 1176\\
Carrie Harper & PC & 1059\\
Duncan Rees & Grn & 353\\
\end{tabular*}

Simon Thomas (PC, Mid and West Wales) resigned on 25 July 2018.  He was replaced from the list by Helen Mary Jones.

\section{Scottish Parliament}

There were no by-elections in 2018 to the Scottish Parliament.
%There was one by-election in 2017 to the Scottish Parliament.
%
%\subsubsection*{Ettrick, Roxburgh and Berwickshire \hspace*{\fill}\nolinebreak[1]%
%\enspace\hspace*{\fill}
%\finalhyphendemerits=0
%[8th June]}
%
%\index{Ettrick, Roxburgh and Berwickshire , Scottish Parliament@Ettrick, Roxburgh \& Berwickshire, \emph{Scottish Parliament}}
%
%Resignation of John Lamont (C).
%
%\noindent
%\begin{tabular*}{\columnwidth}{@{\extracolsep{\fill}} p{0.53\columnwidth} >{\itshape}l r @{\extracolsep{\fill}}}
%Rachael Hamilton & C & 20658\\
%Gail Hendry & SNP & 11320\\
%Sally Prentice & Lab & 3406\\
%Catriona Bhatia & LD & 3196\\
%\end{tabular*}
%
%Rachael Hamilton (C, South of Scotland) resigned on 3 May 2017 in order to contest the above by-election.  She was replaced from the list by Michelle Ballantyne.
%
%Douglas Ross MP (C, Highlands and Islands) resigned on 9 June 2017.  He was replaced from the list by Jamie Halcro Johnston.
%
%Ross Thomson MP (C, North East) resigned on 9 June 2017.  He was replaced from the list by Tom Mason.

\section{Northern Ireland Assembly}

Vacancies in the Northern Ireland Assembly are filled by co-option.
No co-options were made in 2018.
%
%The following members were co-opted to the Assembly in 2017:
%\begin{itemize}
%\item Colm Gildernew (SF) replaced Michelle Gildernew MP following her resignation on 8th June (Fermanagh and South Tyrone).
%\item Trevor Clarke (DUP) replaced Paul Girvan MP following his resignation on 8th June (South Antrim).
%\item Emma Rogan (SF) replaced Chris Hazzard MP following his resignation on 8th June (South Down).
%\item Karen Mullan (SF) replaced Elisha McCallion MP following her resignation on 8th June (Foyle).
%\item Catherine Kelly (SF) replaced Barry McElduff MP following his resignation on 8th June (West Tyrone).
%\end{itemize}

\section{European Parliament}

UK vacancies in the European Parliament are filled by the next available person from the party list at the most recent election (which was held in 2014).
No replacements were made in 2018.
%The following replacements were made in 2017:
%\begin{itemize}
%\item John Howarth (Lab) replaced Anneliese Dodds following her resignation on 8th June (South East).
%\item John Flack (C) replaced Vicky Ford following her resignation on 8th June (Eastern).
%\item Wajid Khan (Lab) replaced Afzal Khan following his resignation on 8th June (North West).
%\item Rupert Matthews (C) replaced Andrew Lewer following his resignation on 8th June (East Midlands).
%\item Baroness Mobarik (C) replaced Ian Duncan following his resignation on 22nd June (Scotland).
%\item Jonathan Bullock (UKIP) replaced Roger Helmer following his resignation on 28th July (East Midlands).
%\item Rory Palmer (Lab) replaced Dame Glenis Willmott following her resignation on 2nd October (East Midlands).
%\end{itemize}

\section{Police and crime commissioners}

There were no by-elections in 2018 for vacant police and crime commissioner posts.

\chapter{Local by-elections and unfilled vacancies}

\begin{resultsiii}

%\section[North London]{\sloppyword{North London}}
%
%\subsection*{City of London Corporation}
%
%At the March 2017 ordinary election there were unfilled vacancies in Bishopsgate Without and Queenhithe wards due to the resignation of Billy Dove and the election of Alastair King as alderman (both Ind) respectively.
%\index{Bishopsgate Without , City of London@Bishopsgate Wt., \emph{City of London}}
%\index{Queenhithe , City of London@Queenhithe, \emph{City of London}}
%
%\subsubsection*{Bishopsgate \hspace*{\fill}\nolinebreak[1]%
%\enspace\hspace*{\fill}
%\finalhyphendemerits=0
%[Tuesday 14th November]}
%
%\index{Bishopsgate , City of London@Bishopsgate, \emph{City of London}}
%
%Resignation of Pooja Suri Tank (Ind).
%
%\noindent
%\begin{tabular*}{\columnwidth}{@{\extracolsep{\fill}} p{0.53\columnwidth} >{\itshape}l r @{\extracolsep{\fill}}}
%Timothy Becker & Ind & -\\
%Benjamin Murphy & Ind & -\\
%Patrick Streeter & Ind & -\\
%\end{tabular*}
%
%\subsubsection*{Portsoken \hspace*{\fill}\nolinebreak[1]%
%\enspace\hspace*{\fill}
%\finalhyphendemerits=0
%[14th December]}
%
%\index{Portsoken , City of London@Portsoken, \emph{City of London}}
%
%Aldermanic election: resignation of Alderman Michael Bear (Ind).

\section{North London}

\council{City of London}

\subsubsection*{Bishopsgate \hspace*{\fill}\nolinebreak[1]%
\enspace\hspace*{\fill}
\finalhyphendemerits=0
[Tuesday 20th March]}

\index{Bishopsgate , City of London@Bishopsgate, \emph{City of London}}

Election of Prem Goyal (Ind) as Alderman for Portsoken ward.

\noindent
\begin{tabular*}{\columnwidth}{@{\extracolsep{\fill}} p{0.53\columnwidth} >{\itshape}l r @{\extracolsep{\fill}}}
Shravan Joshi & Ind & 72\\
Patrick Streeter & Ind & 57\\
Joanne Abeyie & Ind & 24\\
Adedamola Aminu & Lab & 20\\
\end{tabular*}

\subsubsection*{Billingsgate \hspace*{\fill}\nolinebreak[1]%
\enspace\hspace*{\fill}
\finalhyphendemerits=0
[22nd March]}

\index{Billingsgate , City of London@Billingsgate, \emph{City of London}}

Resignation of Michael Welbank (Ind).

\noindent
\begin{tabular*}{\columnwidth}{@{\extracolsep{\fill}} p{0.53\columnwidth} >{\itshape}l r @{\extracolsep{\fill}}}
John Allen-Petrie & Ind & 40\\
Dawn Wright & Ind & 30\\
Alpa Raja & Ind & 14\\
Timothy Becker & Ind & 6\\
\end{tabular*}

\subsubsection*{Candlewick \hspace*{\fill}\nolinebreak[1]%
\enspace\hspace*{\fill}
\finalhyphendemerits=0
[5th July]}

\index{Candlewick , City of London@Candlewick, \emph{City of London}}

Aldermanic election: resignation of Dame Fiona Woolf (Ind).

\noindent
\begin{tabular*}{\columnwidth}{@{\extracolsep{\fill}} p{0.53\columnwidth} >{\itshape}l r @{\extracolsep{\fill}}}
Emma Edhem & Ind & 50\\
Havilland de Sausmarez & Ind & 43\\
William Charnley & Ind & 30\\
Jonathan Bewes & Ind & 27\\
\end{tabular*}

\subsubsection*{Cheap \hspace*{\fill}\nolinebreak[1]%
\enspace\hspace*{\fill}
\finalhyphendemerits=0
[5th July]}

\index{Cheap , City of London@Cheap, \emph{City of London}}

Aldermanic election: resignation of Lord Mountevans (Ind).

\noindent
\begin{tabular*}{\columnwidth}{@{\extracolsep{\fill}} p{0.65\columnwidth} >{\itshape}l r @{\extracolsep{\fill}}}
Robert Hughes-Penney & Ind & 65\\
Andrew Heath-Richardson & Ind & 51\\
Andrew Marsden & Ind & 42\\
Richard Hills & Ind & 39\\
Timothy Haywood & Ind & 33\\
Anthony Samuels & Ind & 15\\
Timothy Becker & Ind & 2\\
\end{tabular*}

\subsubsection*{Aldgate \hspace*{\fill}\nolinebreak[1]%
\enspace\hspace*{\fill}
\finalhyphendemerits=0
[12th July]}

\index{Aldgate , City of London@Aldgate, \emph{City of London}}

Aldermanic election: Peter Hewitt (Ind) sought re-election.

\noindent
\begin{tabular*}{\columnwidth}{@{\extracolsep{\fill}} p{0.53\columnwidth} >{\itshape}l r @{\extracolsep{\fill}}}
Susan Langley & Ind & 174\\
Peter Hewitt & Ind & 54\\
\end{tabular*}

\subsubsection*{Bridge and Bridge Without \hspace*{\fill}\nolinebreak[1]%
\enspace\hspace*{\fill}
\finalhyphendemerits=0
[12th July]}

\index{Bridge and Bridge Without , City of London@Bridge \& Bridge Wt., \emph{City of London}}

Aldermanic election: Sir Alan Yarrow (Ind) sought re-election.

\noindent
\begin{tabular*}{\columnwidth}{@{\extracolsep{\fill}} p{0.53\columnwidth} >{\itshape}l r @{\extracolsep{\fill}}}
Sir Alan Yarrow & Ind & \emph{unop.}\\
\end{tabular*}

\subsubsection*{Bread Street \hspace*{\fill}\nolinebreak[1]%
\enspace\hspace*{\fill}
\finalhyphendemerits=0
[30th August]}

\index{Bread Street , City of London@Bread St., \emph{City of London}}

Aldermanic election: William Russell (Ind) sought re-election.

\noindent
\begin{tabular*}{\columnwidth}{@{\extracolsep{\fill}} p{0.53\columnwidth} >{\itshape}l r @{\extracolsep{\fill}}}
William Russell & Ind & \emph{unop.}\\
\end{tabular*}

\subsubsection*{Castle Baynard \hspace*{\fill}\nolinebreak[1]%
\enspace\hspace*{\fill}
\finalhyphendemerits=0
[Tuesday 9th October; Lab gain from Ind]}

\index{Castle Baynard , City of London@Castle Baynard, \emph{City of London}}

Election of Emma Edhem (Ind) as Alderman for Candlewick ward.

\noindent
\begin{tabular*}{\columnwidth}{@{\extracolsep{\fill}} p{0.53\columnwidth} >{\itshape}l r @{\extracolsep{\fill}}}
Natasha Lloyd-Owen & Lab & 77\\
Julian Malins & Ind & 59\\
Alpa Raja & Ind & 37\\
Merlene Emerson & Ind & 36\\
Virginia Rounding & Ind & 28\\
Richard Humphreys & Ind & 23\\
Deborah Oliver & Ind & 12\\
Timothy Becker & Ind & 5\\
\end{tabular*}

\subsubsection*{Broad Street \hspace*{\fill}\nolinebreak[1]%
	\enspace\hspace*{\fill}
	\finalhyphendemerits=0
	[12th December]}

\index{Broad Street , City of London@Broad St., \emph{City of London}}

Aldermanic election: Michael Mainelli (Ind) sought re-election.

\noindent
\begin{tabular*}{\columnwidth}{@{\extracolsep{\fill}} p{0.53\columnwidth} >{\itshape}l r @{\extracolsep{\fill}}}
Michael Mainelli & Ind & \emph{unop.}\\
\end{tabular*}

\council{Barking and Dagenham}

At the May 2018 ordinary election there was an unfilled vacancy in Heath ward due to the resignation of Dan Young (Ind elected as Lab).\index{Heath , Barking and Dagenham@Heath, \emph{Barking \& Dagenham}}

\council{Brent}

\subsubsection*{Willesden Green (3)
\hspace*{\fill}\nolinebreak[1]%
\enspace\hspace*{\fill}
\finalhyphendemerits=0
[21st June]}

\index{Willesden Green , Brent@Willesden Green, \emph{Brent}}

Ordinary election postponed from 3rd May: death of outgoing councillor Lesley Jones (Lab) who was seeking re-election.

\noindent
\begin{tabular*}{\columnwidth}{@{\extracolsep{\fill}} p{0.53\columnwidth} >{\itshape}l r @{\extracolsep{\fill}}}
Fleur Donnelly-Jackson & Lab & 1683\\
Elliot Chappell & Lab & 1679\\
Tom Miller & Lab & 1618\\
Shaka Lish & Grn & 289\\
Harry Goodwill & C & 280\\
Peter Murry & Grn & 256\\
Felicity Dunn & LD & 254\\
William Relton & Grn & 250\\
Shahin Chowdhury & C & 237\\
Ali al-Jawad & C & 218\\
Ulla Thiessen & LD & 189\\
Christopher Wheatley & LD & 184\\
\end{tabular*}

\council{Ealing}

\subsubsection*{Dormers Wells
	\hspace*{\fill}\nolinebreak[1]%
	\enspace\hspace*{\fill}
	\finalhyphendemerits=0
	[8th November]}

\index{Dormers Wells , Ealing@Dormers Wells, \emph{Ealing}}

Death of Tej Ram Bagha (Lab).

\noindent
\begin{tabular*}{\columnwidth}{@{\extracolsep{\fill}} p{0.53\columnwidth} >{\itshape}l r @{\extracolsep{\fill}}}
Mohinda Kaur Midha & Lab & 1868\\
Amandeep Singh Gill & C & 429\\
Nigel Bakhai & LD & 188\\
Meena Hans & Grn & 106\\
\end{tabular*}

\council{Enfield}

At the May 2018 ordinary election there was an unfilled vacancy in Ponders End ward due to the resignation of Donald McGowan (Lab).\index{Ponders End , Enfield@Ponders End, \emph{Enfield}}

WEP = Women's Equality Party

\subsubsection*{Bush Hill Park
	\hspace*{\fill}\nolinebreak[1]%
	\enspace\hspace*{\fill}
	\finalhyphendemerits=0
	[22nd November]}

\index{Bush Hill Park , Enfield@Bush Hill Park, \emph{Enfield}}

Resignation of Jon Daniels (C).

\noindent
\begin{tabular*}{\columnwidth}{@{\extracolsep{\fill}} p{0.53\columnwidth} >{\itshape}l r @{\extracolsep{\fill}}}
	James Hockney & C & 1540\\
	Bevin Betton & Lab & 828\\
	Robert Wilson & LD & 313\\
	Benjamin Maydon & Grn & 127\\
	Tulip Hambleton & WEP & 79\\
	Erol Ovayolu & Ind & 50\\
\end{tabular*}

\council{Hackney}

WEP = Women's Equality Party

\subsubsection*{Victoria
\hspace*{\fill}\nolinebreak[1]%
\enspace\hspace*{\fill}
\finalhyphendemerits=0
[18th October]}

\index{Victoria , Hackney@Victoria, \emph{Hackney}}

Resignation of Alex Kuye (Lab).

\noindent
\begin{tabular*}{\columnwidth}{@{\extracolsep{\fill}} p{0.53\columnwidth} >{\itshape}l r @{\extracolsep{\fill}}}
Penny Wrout & Lab & 1311\\
Pippa Morgan & LD & 436\\
Wendy Robinson & Grn & 296\\
Christopher Sills & C & 148\\
Harini Iyengar & WEP & 84\\
\end{tabular*}

\council{Haringey}

\subsubsection*{West Green
	\hspace*{\fill}\nolinebreak[1]%
	\enspace\hspace*{\fill}
	\finalhyphendemerits=0
	[13th December]}

\index{West Green , Haringey@West Green, \emph{Haringey}}

Resignation of Ishamel Osamor (Lab).

\noindent
\begin{tabular*}{\columnwidth}{@{\extracolsep{\fill}} p{0.53\columnwidth} >{\itshape}l r @{\extracolsep{\fill}}}
Semma Chandwani & Lab & 1273\\
Elizabeth Payne & LD & 621\\
Cecily Spelling & Grn & 243\\
Hammad Baig & C & 114\\
\end{tabular*}

\council{Hillingdon}

At the May 2018 ordinary election there was an unfilled vacancy in Botwell ward due to the death of Mo Khursheed (Lab).\index{Botwell , Hillingdon@Botwell, \emph{Hillingdon}}

\council{Hounslow}

At the May 2018 ordinary election there were unfilled vacancies in Bedfont, Chiswick Riverside and Hanworth Park wards due to the resignations of Sachin Gupta (Lab), Felicity Barwood (C) and Tina Howe (Lab) respectively.\index{Bedfont , Hounslow@Bedfont, \emph{Hounslow}}\index{Chsiwick Riverside , Hounslow@Chiswick Riverside, \emph{Hounslow}}\index{Hanworth Park , Hounslow@Hanworth Park, \emph{Hounslow}}

\council{Newham}

\subsubsection*{Boleyn
	\hspace*{\fill}\nolinebreak[1]%
	\enspace\hspace*{\fill}
	\finalhyphendemerits=0
	[1st November]}

\index{Boleyn , Newham@Boleyn, \emph{Newham}}

Resignation of Veronica Oakeshott (Lab).

\noindent
\begin{tabular*}{\columnwidth}{@{\extracolsep{\fill}} p{0.53\columnwidth} >{\itshape}l r @{\extracolsep{\fill}}}
Moniba Khan & Lab & 1725\\
Md Fazlul Karim & C & 327\\
Frankie-Rose Taylor & Grn & 172\\
Arunsalam Pirapaharan & LD & 83\\
\end{tabular*}

\council{Westminster}

\subsubsection*{Lancaster Gate
	\hspace*{\fill}\nolinebreak[1]%
	\enspace\hspace*{\fill}
	\finalhyphendemerits=0
	[22nd November]}

\index{Lancaster Gate , Westminster@Lancaster Gate, \emph{Westminster}}

Resignation of Robert Davis (C).

\noindent
\begin{tabular*}{\columnwidth}{@{\extracolsep{\fill}} p{0.53\columnwidth} >{\itshape}l r @{\extracolsep{\fill}}}
	Margot Bright & C & 913\\
	Angela Piddick & Lab & 684\\
	Sally Gray & LD & 275\\
	Zack Polanski & Grn & 62\\
\end{tabular*}

\section{South London}

\council{Bexley}

At the May 2018 ordinary election there was an unfilled vacancy in Lesnes Abbey ward due to the resignation of Esther Amaning (Lab).\index{Lesnes Abbey , Bexley@Lesnes Abbey, \emph{Bexley}}

\council{Bromley}

\subsubsection*{Kelsey and Eden Park
	\hspace*{\fill}\nolinebreak[1]%
	\enspace\hspace*{\fill}
	\finalhyphendemerits=0
	[29th November]}

\index{Kelsey and Eden Park , Bromley@Kelsey \& Eden Park, \emph{Bromley}}

Resignation of Dave Wibberley (C).

\noindent
\begin{tabular*}{\columnwidth}{@{\extracolsep{\fill}} p{0.53\columnwidth} >{\itshape}l r @{\extracolsep{\fill}}}
Christine Harris & C & 1626\\
Marie Bardsley & Lab & 1046\\
Julie Ireland & LD & 633\\
Graham Reakes & UKIP & 219\\
Paul Brock & Grn & 73\\
\end{tabular*}

\council{Croydon}

At the May 2018 ordinary election there were unfilled vacancies in Selsdon and Ballards, and Thornton Heath wards due to the resignation of Sara Bashford (C) and the disqualification (non-attendance) of Matthew Kyeremeh (Ind elected as Lab).\index{Selsdon and Ballards , Croydon@Selsdon \& Ballards, \emph{Croydon}}\index{Thornton Heath , Croydon@Thornton Heath, \emph{Croydon}}

\council{Lambeth}

WEP = Women's Equality Party

\subsubsection*{Coldharbour
\hspace*{\fill}\nolinebreak[1]%
\enspace\hspace*{\fill}
\finalhyphendemerits=0
[13th September]}

\index{Coldharbour , Lambeth@Coldharbour, \emph{Lambeth}}

Death of Mathew Parr (Lab).

\noindent
\begin{tabular*}{\columnwidth}{@{\extracolsep{\fill}} p{0.57\columnwidth} >{\itshape}l r @{\extracolsep{\fill}}}
Scarlett O'Hara & Lab & 1739\\
Michael Groce & Grn & 912\\
Doug Buist & LD & 148\\
Yvonne Stewart-Williams & C & 119\\
Sian Fogden & WEP & 47\\
Robert Stephenson & UKIP & 21\\
\end{tabular*}

\council{Richmond upon Thames}

At the May 2018 ordinary election there was an unfilled vacancy in Heathfield ward due to the resignation of Annie Hambidge (C).\index{Heathfield , Richmond upon Thames@Heathfield, \emph{Richmond upon Thames}}

\council{Southwark}

\subsubsection*{\sloppyword{London Bridge and West Bermondsey (3)}
\hspace*{\fill}\nolinebreak[1]%
\enspace\hspace*{\fill}
\finalhyphendemerits=0
[14th June]}

\index{London Bridge and West Bermondsey , Southwark@London Bridge \& West Bermondsey, \emph{Southwark}}

Ordinary election postponed from 3rd May: death of candidate Toby Eckersley (C).

\noindent
\begin{tabular*}{\columnwidth}{@{\extracolsep{\fill}} p{0.53\columnwidth} >{\itshape}l r @{\extracolsep{\fill}}}
Humaira Ali & LD & 1340\\
Damian O'Brien & LD & 1281\\
William Houngbo & LD & 1270\\
Julie Eyles & Lab & 1239\\
John Batteson & Lab & 1215\\
Edward McDonagh & Lab & 1171\\
Hannah Ginnett & C & 221\\
Richard Packer & C & 219\\
Claude Werner & Grn & 215\\
Nathan Newport Gay & C & 205\\
Bernard Creely & Grn & 191\\
\end{tabular*}

\council{Sutton}

At the May 2018 ordinary election there was an unfilled vacancy in Stonecot ward due to the death of Adrian Davey (LD).\index{Stonecot , Sutton@Stonecot, \emph{Sutton}}

\subsubsection*{Belmont
\hspace*{\fill}\nolinebreak[1]%
\enspace\hspace*{\fill}
\finalhyphendemerits=0
[25th October]}

\index{Belmont , Sutton@Belmont, \emph{Sutton}}

Resignation of Patrick McManus (C).

\noindent
\begin{tabular*}{\columnwidth}{@{\extracolsep{\fill}} p{0.53\columnwidth} >{\itshape}l r @{\extracolsep{\fill}}}
Neil Garratt & C & 1328\\
Dean Juster & LD & 1069\\
Marian Wingrove & Lab & 303\\
Claire Jackson-Prior & Grn & 63\\
John Bannon & UKIP & 50\\
Ashley Dickenson & CPA & 30\\
\end{tabular*}

\section{Greater Manchester}

\council{Bolton}

At the May 2018 ordinary election there was an unfilled vacancy in Smithills ward due to the resignation of Andrew Martin (LD).\index{Smithills , Bolton@Smithills, \emph{Bolton}}

FKF = Farnworth and Kearsley First

\subsubsection*{Hulton \hspace*{\fill}\nolinebreak[1]%
\enspace\hspace*{\fill}
\finalhyphendemerits=0
[18th January; C gain from Lab]}

\index{Hulton , Bolton@Hulton, \emph{Bolton}}

Death of Darren Whitehead (Lab).

\noindent
\begin{tabular*}{\columnwidth}{@{\extracolsep{\fill}} p{0.53\columnwidth} >{\itshape}l r @{\extracolsep{\fill}}}
Toby Hewitt & C & 1455\\
Rabiya Jiva & Lab & 1179\\
Bev Fletcher & UKIP & 190\\
Derek Gradwell & LD & 67\\
James Tomkinson & Grn & 52\\
\end{tabular*}

\subsubsection*{Farnworth \hspace*{\fill}\nolinebreak[1]%
\enspace\hspace*{\fill}
\finalhyphendemerits=0
[8th March; FKF gain from Lab]}

\index{Farnworth , Bolton@Farnworth, \emph{Bolton}}

Resignation of Asif Ibrahim (Lab).

\noindent
\begin{tabular*}{\columnwidth}{@{\extracolsep{\fill}} p{0.53\columnwidth} >{\itshape}l r @{\extracolsep{\fill}}}
Paul Sanders & FKF & 1204\\
Rebecca Minors & Lab & 969\\
Dave Harvey & UKIP & 169\\
Matthew Littler & C & 153\\
David Walsh & LD & 23\\
David Figgins & Grn & 18\\
\end{tabular*}

\subsubsection*{Bromley Cross \hspace*{\fill}\nolinebreak[1]%
\enspace\hspace*{\fill}
\finalhyphendemerits=0
[3rd May]}

\index{Bromley Cross , Bolton@Bromley Cross, \emph{Bolton}}

Resignation of Alan Wilkinson (C).

Combined with the 2018 ordinary election.
%; see page \pageref{BromleyCrossBolton} for the result.

\council{Bury}

\subsubsection*{Besses \hspace*{\fill}\nolinebreak[1]%
\enspace\hspace*{\fill}
\finalhyphendemerits=0
[19th July]}

\index{Besses , Bury@Besses, \emph{Bury}}

Resignation of Elizabeth Fitzgerald (Lab).

\noindent
\begin{tabular*}{\columnwidth}{@{\extracolsep{\fill}} p{0.53\columnwidth} >{\itshape}l r @{\extracolsep{\fill}}}
Lucy Smith & Lab & 999\\
Jordan Lewis & C & 708\\
Stephen Morris & EDP & 72\\
Gareth Lloyd-Johnson & LD & 71\\
Glyn Heath & Grn & 55\\
Michael Zwierzanski & UKIP & 49\\
\end{tabular*}

\subsubsection*{East \hspace*{\fill}\nolinebreak[1]%
\enspace\hspace*{\fill}
\finalhyphendemerits=0
[16th August]}

\index{East , Bury@East, \emph{Bury}}

Resignation of Mike Connolly (Lab).

\noindent
\begin{tabular*}{\columnwidth}{@{\extracolsep{\fill}} p{0.53\columnwidth} >{\itshape}l r @{\extracolsep{\fill}}}
Gavin McGill & Lab & 1419\\
Sohail Raja & C & 557\\
Angela Zwierzanski & UKIP & 107\\
Nicole Haydock & Grn & 77\\
Andy Minty & LD & 49\\
\end{tabular*}

\council{Manchester}

At the May 2018 ordinary election there were unfilled vacancies in Chorlton and City Centre wards due to the death of Sheila Newman and the resignation of Beth Knowles (both Lab) respectively.\index{Chorlton , Manchester@Chorlton, \emph{Manchester}}\index{City Centre , Manchester@City Centre, \emph{Manchester}}

\council{Oldham}

At the May 2018 ordinary election there was an unfilled vacancy in Shaw ward due to the resignation of Rod Blyth (LD).\index{Shaw , Oldham@Shaw, \emph{Oldham}}

\subsubsection*{Chadderton Central \hspace*{\fill}\nolinebreak[1]%
\enspace\hspace*{\fill}
\finalhyphendemerits=0
[3rd May]}

\index{Chadderton Central , Oldham@Chadderton C., \emph{Oldham}}

Death of Susan Dearden (Lab).

Combined with the 2018 ordinary election.
%; see page \pageref{ChaddertonCentralOldham} for the result.

\subsubsection*{Hollinwood \hspace*{\fill}\nolinebreak[1]%
\enspace\hspace*{\fill}
\finalhyphendemerits=0
[3rd May]}

\index{Hollinwood , Oldham@Hollinwood, \emph{Oldham}}

Death of Brian Ames (Lab).

Combined with the 2018 ordinary election.
%; see page \pageref{HollinwoodOldham} for the result.

\subsubsection*{Failsworth East \hspace*{\fill}\nolinebreak[1]%
	\enspace\hspace*{\fill}
	\finalhyphendemerits=0
	[29th November]}

\index{Failsworth East , Oldham@Failsworth E., \emph{Oldham}}

Resignation of Cheryl Brock (Lab).

\noindent
\begin{tabular*}{\columnwidth}{@{\extracolsep{\fill}} p{0.53\columnwidth} >{\itshape}l r @{\extracolsep{\fill}}}
Elizabeth Jacques & Lab & 677\\
Antony Cahill & C & 336\\
Warren Bates & Ind & 94\\
Paul Goldring & UKIP & 32\\
Stephen Barrow & LD & 18\\
\end{tabular*}

\columnbreak

\council{Salford}

WEP = Women's Equality Party

\subsubsection*{Eccles \hspace*{\fill}\nolinebreak[1]%
\enspace\hspace*{\fill}
\finalhyphendemerits=0
[27th September]}

\index{Eccles , Salford@Eccles, \emph{Salford}}

Resignation of Peter Wheeler (Lab).

\noindent
\begin{tabular*}{\columnwidth}{@{\extracolsep{\fill}} p{0.53\columnwidth} >{\itshape}l r @{\extracolsep{\fill}}}
Mike McCusker & Lab & 1071\\
Andrew Darlington & C & 474\\
Jake Overend & LD & 156\\
Helen Alker & Grn & 123\\
Keith Hallam & UKIP & 100\\
Caroline Dean & WEP & 39\\
\end{tabular*}

\council{Stockport}

\subsubsection*{Edgeley and Cheadle Heath \hspace*{\fill}\nolinebreak[1]%
\enspace\hspace*{\fill}
\finalhyphendemerits=0
[24th May]}

\index{Edgeley and Cheadle Heath , Stockport@Edgeley \& Cheadle Heath, \emph{Stockport}}

Ordinary election postponed from 3rd May: death of candidate Maureen Baldwin-Moore (C).

\noindent
\begin{tabular*}{\columnwidth}{@{\extracolsep{\fill}} p{0.53\columnwidth} >{\itshape}l r @{\extracolsep{\fill}}}
Philip Harding & Lab & 1709\\
Oliver Harrison & LD & 203\\
Pat Leck & C & 187\\
Camilla Luff & Grn & 144\\
Peter Behan & UKIP & 71\\
\end{tabular*}

\council{Tameside}

\subsubsection*{Droylsden East \hspace*{\fill}\nolinebreak[1]%
\enspace\hspace*{\fill}
\finalhyphendemerits=0
[8th March]}

\index{Droylsden East , Tameside@Droylsden E., \emph{Tameside}}

Death of Kieran Quinn (Lab).

\noindent
\begin{tabular*}{\columnwidth}{@{\extracolsep{\fill}} p{0.53\columnwidth} >{\itshape}l r @{\extracolsep{\fill}}}
Laura Boyle & Lab & 986\\
Matt Stevenson & C & 489\\
Annie Train & Grn & 98\\
Shaun Offerman & LD & 30\\
\end{tabular*}

\subsubsection*{Ashton Waterloo \hspace*{\fill}\nolinebreak[1]%
\enspace\hspace*{\fill}
\finalhyphendemerits=0
[6th September]}

\index{Ashton Waterloo , Tameside@Ashton Waterloo, \emph{Tameside}}

Death of Catherine Piddington (Lab).

\noindent
\begin{tabular*}{\columnwidth}{@{\extracolsep{\fill}} p{0.53\columnwidth} >{\itshape}l r @{\extracolsep{\fill}}}
Pauline Hollinshead & Lab & 889\\
Lee Huntbach & Grn & 448\\
Therese Costello & C & 357\\
\end{tabular*}

\council{Trafford}

At the May 2018 ordinary election there was an unfilled vacancy in Altrincham ward due to the disqualification (sentenced to two years and nine months in prison, child pornography offences) of Matthew Sephton (C).\index{Altrincham , Trafford@Altrincham, \emph{Trafford}}

\subsubsection*{Altrincham \hspace*{\fill}\nolinebreak[1]%
\enspace\hspace*{\fill}
\finalhyphendemerits=0
[3rd May]}

\index{Altrincham , Trafford@Altrincham, \emph{Trafford}}

Resignation of Alex Williams (C).

Combined with the 2018 ordinary election.
%; see page \pageref{AltrinchamTrafford} for the result.

\council{Wigan}

\subsubsection*{Bryn \hspace*{\fill}\nolinebreak[1]%
\enspace\hspace*{\fill}
\finalhyphendemerits=0
[22nd February]}

\index{Bryn , Wigan@Bryn, \emph{Wigan}}

Resignation of Steve Jones (Ind).

\emph{This by-election was cancelled.}  Jones had submitted a postdated resignation letter to the council and then withdrew his resignation before it was due to take effect.  The Returning Officer had decided that Jones' resignation took effect immediately upon the letter being received, and accordingly organised a poll.  On 21st February the High Court ruled that Jones had not resigned his office and ordered that the by-election be cancelled.

\section{Merseyside}

\council{Knowsley}

\subsubsection*{Page Moss \hspace*{\fill}\nolinebreak[1]%
\enspace\hspace*{\fill}
\finalhyphendemerits=0
[29th March]}

\index{Page Moss , Knowsley@Page Moss, \emph{Knowsley}}

Death of Veronica McNeill (Lab).

\noindent
\begin{tabular*}{\columnwidth}{@{\extracolsep{\fill}} p{0.53\columnwidth} >{\itshape}l r @{\extracolsep{\fill}}}
Del Arnall & Lab & 657\\
Kirk Sandringham & Grn & 74\\
Fred Fricker & UKIP & 68\\
Aaron Waters & C & 41\\
\end{tabular*}

\subsubsection*{Halewood South \hspace*{\fill}\nolinebreak[1]%
\enspace\hspace*{\fill}
\finalhyphendemerits=0
[23rd August]}

\index{Halewood South , Knowsley@Halewood S., \emph{Knowsley}}

Death of Tina Harris (Lab).

\noindent
\begin{tabular*}{\columnwidth}{@{\extracolsep{\fill}} p{0.53\columnwidth} >{\itshape}l r @{\extracolsep{\fill}}}
Gary See & Lab & 1012\\
Bob Swann & Ind & 778\\
Jenny McNeilis & LD & 118\\
Victoria Smart & C & 54\\
\end{tabular*}

\council{Liverpool}

At the May 2018 ordinary election there was an unfilled vacancy in Everton ward due to the death of John McIntosh (Lab).\index{Everton , Liverpool@Everton, \emph{Liverpool}}

\subsubsection*{Knotty Ash \hspace*{\fill}\nolinebreak[1]%
\enspace\hspace*{\fill}
\finalhyphendemerits=0
[3rd May]}

\index{Knotty Ash , Liverpool@Knotty Ash, \emph{Liverpool}}

Resignation of Jacqui Taylor (Lab).

Combined with the 2018 ordinary election.
%; see page \pageref{KnottyAshLiverpool} for the result.

\council{Sefton}

At the May 2018 ordinary election there was an unfilled vacancy in Blundellsands ward due to the resignation of Andy Dams (Lab).\index{Blundellsands , Sefton@Blundellsands, \emph{Sefton}}

\subsubsection*{Ford \hspace*{\fill}\nolinebreak[1]%
\enspace\hspace*{\fill}
\finalhyphendemerits=0
[3rd May]}

\index{Ford , Sefton@Ford, \emph{Sefton}}

Resignation of Kevin Cluskey (Lab).

Combined with the 2018 ordinary election.
%; see page \pageref{FordSefton} for the result.

\council{Wirral}

At the May 2018 ordinary election there was an unfilled vacancy in Bebington ward due to the death of Walter Smith (Lab).\index{Bebington , Wirral@Bebington, \emph{Wirral}}

\subsubsection*{Hoylake and Meols \hspace*{\fill}\nolinebreak[1]%
\enspace\hspace*{\fill}
\finalhyphendemerits=0
[3rd May]}

\index{Hoylake and Meols , Wirral@Hoylake \& Meols, \emph{Wirral}}

Resignation of John Hale (C).

Combined with the 2018 ordinary election.
%; see page \pageref{HoylakeMeolsWirral} for the result.

\subsubsection*{Bromborough \hspace*{\fill}\nolinebreak[1]%
\enspace\hspace*{\fill}
\finalhyphendemerits=0
[23rd August]}

\index{Bromborough , Wirral@Bromborough, \emph{Wirral}}

Resignation of Warren Ward (Lab).

\noindent
\begin{tabular*}{\columnwidth}{@{\extracolsep{\fill}} p{0.53\columnwidth} >{\itshape}l r @{\extracolsep{\fill}}}
Jo Bird & Lab & 1253\\
Des Drury & C & 749\\
Vicky Downie & LD & 454\\
Steve Niblock & Ind & 147\\
Susan Braddock & Grn & 59\\
\end{tabular*}

\subsubsection*{Upton \hspace*{\fill}\nolinebreak[1]%
	\enspace\hspace*{\fill}
	\finalhyphendemerits=0
	[22nd November]}

\index{Upton , Wirral@Upton, \emph{Wirral}}

Resignation of Matthew Patrick (Lab).

\noindent
\begin{tabular*}{\columnwidth}{@{\extracolsep{\fill}} p{0.53\columnwidth} >{\itshape}l r @{\extracolsep{\fill}}}
Jean Robinson & Lab & 1490\\
Emma Sellman & C & 705\\
Lily Clough & Grn & 151\\
Alan Davies & LD & 83\\
\end{tabular*}

\section{South Yorkshire}

Yorks = Yorkshire Party

\council{Barnsley}

DVP = Democrats and Veterans Party

\subsubsection*{Old Town \hspace*{\fill}\nolinebreak[1]%
\enspace\hspace*{\fill}
\finalhyphendemerits=0
[12th July]}

\index{Old Town , Barnsley@Old Town, \emph{Barnsley}}

Resignation of Anita Cherryholme (Lab).

\noindent
\begin{tabular*}{\columnwidth}{@{\extracolsep{\fill}} p{0.53\columnwidth} >{\itshape}l r @{\extracolsep{\fill}}}
Jo Newing & Lab & 548\\
Gavin Felton & DVP & 338\\
Clive Watkinson & C & 157\\
Kevin Bennett & LD & 124\\
Tony Devoy & Yorks & 47\\
Christopher Houston & BNP & 25\\
\end{tabular*}

\council{Doncaster}

\subsubsection*{Armthorpe \hspace*{\fill}\nolinebreak[1]%
\enspace\hspace*{\fill}
\finalhyphendemerits=0
[15th February]}

\index{Armthorpe , Doncaster@Armthorpe, \emph{Doncaster}}

Death of Tony Corden (Lab).

\noindent
\begin{tabular*}{\columnwidth}{@{\extracolsep{\fill}} p{0.53\columnwidth} >{\itshape}l r @{\extracolsep{\fill}}}
Frank Tyas & Lab & 1431\\
Martin Williams & Ind & 466\\
\end{tabular*}

\subsubsection*{Town \hspace*{\fill}\nolinebreak[1]%
\enspace\hspace*{\fill}
\finalhyphendemerits=0
[14th June]}

\index{Town , Doncaster@Town, \emph{Doncaster}}

Resignation of John McHale (Lab).

\noindent
\begin{tabular*}{\columnwidth}{@{\extracolsep{\fill}} p{0.53\columnwidth} >{\itshape}l r @{\extracolsep{\fill}}}
Tosh McDonald & Lab & 1084\\
Chris Whitwood & Yorks & 570\\
Julie Buckley & Grn & 294\\
Carol Greenhalgh & C & 260\\
Ian Smith & LD & 66\\
Gareth Pendry & Ind & 43\\
\end{tabular*}

\section{Tyne and Wear}

\council{Gateshead}

\subsubsection*{Ryton, Crookhill and Stella \hspace*{\fill}\nolinebreak[1]%
\enspace\hspace*{\fill}
\finalhyphendemerits=0
[3rd May]}

\index{Ryton, Crookhill and Stella , Gateshead@Ryton, Crookhill \& Stella, \emph{Gateshead}}

Resignation of Liz Twist MP (Lab).

Combined with the 2018 ordinary election.
%; see page \pageref{RytonCrookhillStellaGateshead} for the result.

\council{Newcastle upon Tyne}

At the May 2018 ordinary election there was an unfilled vacancy in South Jesmond ward due to the resignation of Kerry Alibhai (Lab).\index{South Jesmond , Newcastle upon Tyne@South Jesmond, \emph{Newcastle upon Tyne}}

\council{Sunderland}

\subsubsection*{Pallion \hspace*{\fill}\nolinebreak[1]%
\enspace\hspace*{\fill}
\finalhyphendemerits=0
[1st February; LD gain from Lab]}

\index{Pallion , Sunderland@Pallion, \emph{Sunderland}}

Death of Paul Watson (Lab).

\noindent
\begin{tabular*}{\columnwidth}{@{\extracolsep{\fill}} p{0.53\columnwidth} >{\itshape}l r @{\extracolsep{\fill}}}
Martin Haswell & LD & 1251\\
Gordon Chalk & Lab & 807\\
Grant Shearer & C & 126\\
Steven Bewick & UKIP & 97\\
Craig Hardy & Grn & 93\\
\end{tabular*}

\section{West Midlands}

\council{Birmingham}

At the May 2018 ordinary election there was an unfilled vacancy in Kings Norton ward due to the resignation of Valerie Seabright (Lab).\index{Kings Norton , Birmingham@Kings Norton, \emph{Birmingham}}

\council{Dudley}

At the May 2018 ordinary election there were unfilled vacancies in Coseley East and Halesowen South wards due to the resignation of Star Anderton (elected as Star Etheridge, UKIP) and the disqualification (non-attendance) of Nick Gregory (C).\index{Coseley East , Dudley@Coseley E., \emph{Dudley}}\index{Halesowen South , Dudley@Halesowen S., \emph{Dudley}}

\council{Sandwell}

\subsubsection*{Greets Green and Lyng \hspace*{\fill}\nolinebreak[1]%
\enspace\hspace*{\fill}
\finalhyphendemerits=0
[3rd May]}

\index{Greets Green and Lyng , Sandwell@Greets Green \& Lyng, \emph{Sandwell}}

Resignation of Gurcharan Singh Sidhu (Lab).

Combined with the 2018 ordinary election.
%; see page \pageref{GreetsGreenLyngSandwell} for the result.

\subsubsection*{St Pauls \hspace*{\fill}\nolinebreak[1]%
\enspace\hspace*{\fill}
\finalhyphendemerits=0
[3rd May]}

\index{Saint Pauls , Sandwell@St Pauls, \emph{Sandwell}}

Resignation of Preet Gill MP (Lab).

Combined with the 2018 ordinary election.
%; see page \pageref{StPaulsSandwell} for the result.

\council{Solihull}

\subsubsection*{Blythe \hspace*{\fill}\nolinebreak[1]%
\enspace\hspace*{\fill}
\finalhyphendemerits=0
[1st March]}

\index{Blythe , Solihull@Blythe, \emph{Solihull}}

Resignation of Alex Insley (C).

\noindent
\begin{tabular*}{\columnwidth}{@{\extracolsep{\fill}} p{0.53\columnwidth} >{\itshape}l r @{\extracolsep{\fill}}}
James Butler & C & 1252\\
Sardul Singh Marwa & Lab & 224\\
Charles Robinson & LD & 174\\
\end{tabular*}

\council{Walsall}

\subsubsection*{Paddock \hspace*{\fill}\nolinebreak[1]%
\enspace\hspace*{\fill}
\finalhyphendemerits=0
[3rd May]}

\index{Paddock , Walsall@Paddock, \emph{Walsall}}

Resignation of Peter Washbrook (C).

Combined with the 2018 ordinary election.
%; see page \pageref{PaddockWalsall} for the result.

\council{Wolverhampton}

\subsubsection*{Graiseley \hspace*{\fill}\nolinebreak[1]%
\enspace\hspace*{\fill}
\finalhyphendemerits=0
[3rd May]}

\index{Graiseley , Wolverhampton@Graiseley, \emph{Wolverhampton}}

Death of Elias Mattu (Lab).

Combined with the 2018 ordinary election.
%; see page \pageref{GraiseleyWolverhampton} for the result.

\subsubsection*{Oxley \hspace*{\fill}\nolinebreak[1]%
\enspace\hspace*{\fill}
\finalhyphendemerits=0
[3rd May]}

\index{Oxley , Wolverhampton@Oxley, \emph{Wolverhampton}}

Resignation of Ian Claymore (Lab).

Combined with the 2018 ordinary election.
%; see page \pageref{OxleyWolverhampton} for the result.

\section{West Yorkshire}

\council{Bradford}

At the May 2018 ordinary election there was an unfilled vacancy in Little Horton ward due to the resignation of Naveeda Ikram (Lab).\index{Little Horton , Bradford@Little Horton, \emph{Bradford}}

\council{Kirklees}

\subsubsection*{Birstall and Birkenshaw \hspace*{\fill}\nolinebreak[1]%
\enspace\hspace*{\fill}
\finalhyphendemerits=0
[3rd May]}

\index{Birstall and Birkenshaw , Kirklees@Birstall \& Birkenshaw, \emph{Kirklees}}

Resignation of Andrew Palfreeman (C).

Combined with the 2018 ordinary election.
%; see page \pageref{BirstallBirkenshawKirklees} for the result.

\subsubsection*{Denby Dale \hspace*{\fill}\nolinebreak[1]%
	\enspace\hspace*{\fill}
	\finalhyphendemerits=0
	[1st November; Lab gain from C]}

\index{Denby Dale , Kirklees@Denby Dale, \emph{Kirklees}}

Resignation of Billy Jewitt (C).

\noindent
\begin{tabular*}{\columnwidth}{@{\extracolsep{\fill}} p{0.53\columnwidth} >{\itshape}l r @{\extracolsep{\fill}}}
Will Simpson & Lab & 1834\\
Paula Kemp & C & 1689\\
Alison Baskeyfield & LD & 289\\
Isabel Walters & Grn & 116\\
\end{tabular*}

\section{Bedfordshire}

\council{Luton}

\subsubsection*{Limbury \hspace*{\fill}\nolinebreak[1]%
\enspace\hspace*{\fill}
\finalhyphendemerits=0
[20th September]}

\index{Limbury , Luton@Limbury, \emph{Luton}}

Resignation of Jennifer Rowlands (Lab).

\noindent
\begin{tabular*}{\columnwidth}{@{\extracolsep{\fill}} p{0.53\columnwidth} >{\itshape}l r @{\extracolsep{\fill}}}
Amy Nicholls & Lab & 692\\
Heather Baker & C & 396\\
Steve Moore & LD & 344\\
\end{tabular*}

\section{Berkshire}

\council{Reading}

\subsubsection*{Church \hspace*{\fill}\nolinebreak[1]%
\enspace\hspace*{\fill}
\finalhyphendemerits=0
[3rd May]}

\index{Church , Reading@Church, \emph{Reading}}

Resignation of Paul Woodward (Lab).

Combined with the 2018 ordinary election.
%; see page \pageref{ChurchReading} for the result.

\subsubsection*{Katesgrove \hspace*{\fill}\nolinebreak[1]%
\enspace\hspace*{\fill}
\finalhyphendemerits=0
[3rd May]}

\index{Katesgrove , Reading@Katesgrove, \emph{Reading}}

Resignation of Matt Rodda MP (Lab).

Combined with the 2018 ordinary election.
%; see page \pageref{KatesgroveReading} for the result.

\subsubsection*{Kentwood \hspace*{\fill}\nolinebreak[1]%
\enspace\hspace*{\fill}
\finalhyphendemerits=0
[3rd May]}

\index{Kentwood , Reading@Kentwood, \emph{Reading}}

Resignation of Tom Steele (C).

Combined with the 2018 ordinary election.
%; see page \pageref{KentwoodReading} for the result.

\council{West Berkshire}

\subsubsection*{Thatcham West \hspace*{\fill}\nolinebreak[1]%
\enspace\hspace*{\fill}
\finalhyphendemerits=0
[19th April; LD gain from C]}

\index{Thatcham West , West Berkshire@Thatcham W., \emph{W. Berks.}}

Disqualification (non-attendance) of Nick Goodes (C).

\noindent
\begin{tabular*}{\columnwidth}{@{\extracolsep{\fill}} p{0.53\columnwidth} >{\itshape}l r @{\extracolsep{\fill}}}
Jeff Brooks & LD & 820\\
Ellen Crumly & C & 523\\
Louise Coulson & Lab & 130\\
Jane Livermore & Grn & 130\\
Gary Johnson & UKIP & 91\\
\end{tabular*}

\council{Windsor and Maidenhead}

NFPP = National Flood Prevention Party

\subsubsection*{Datchet \hspace*{\fill}\nolinebreak[1]%
	\enspace\hspace*{\fill}
	\finalhyphendemerits=0
	[22nd November]}

\index{Datchet , Windsor and Maidenhead@Datchet, \emph{Windsor \& Maidenhead}}

Death of Jesse Gray (C).

\noindent
\begin{tabular*}{\columnwidth}{@{\extracolsep{\fill}} p{0.53\columnwidth} >{\itshape}l r @{\extracolsep{\fill}}}
	David Cannon & C & 525\\
	Ewan Larcombe & NFPP & 223\\
	Deborah Foster & Lab & 121\\
	Tim O'Flynn & LD & 48\\
	Christopher Moss & Grn & 21\\
\end{tabular*}

\section{Bristol}\index{Bristol}

\subsubsection*{Westbury-on-Trym and Henleaze \hspace*{\fill}\nolinebreak[1]%
\enspace\hspace*{\fill}
\finalhyphendemerits=0
[24th May; C gain from LD]}

\index{Westbury-on-Trym and Henleaze , Bristol@Westbury-on-Trym \& Henleaze, \emph{Bristol}}

Resignation of Clare Campion-Smith (LD).

\noindent
\begin{tabular*}{\columnwidth}{@{\extracolsep{\fill}} p{0.53\columnwidth} >{\itshape}l r @{\extracolsep{\fill}}}
Steve Smith & C & 2900\\
Graham Donald & LD & 2704\\
Teresa Stratford & Lab & 891\\
Ian Moss & Grn & 355\\
\end{tabular*}

\section{Buckinghamshire}

\subsection*{County Council}\index{Buckinghamshire}

\subsubsection*{Aylesbury North-West \hspace*{\fill}\nolinebreak[1]%
	\enspace\hspace*{\fill}
	\finalhyphendemerits=0
	[29th November]}

\index{Aylesbury North-West , Buckinghamshire@Aylesbury N.W., \emph{Bucks.}}

Resignation of Martin Farrow (LD).

\noindent
\begin{tabular*}{\columnwidth}{@{\extracolsep{\fill}} p{0.53\columnwidth} >{\itshape}l r @{\extracolsep{\fill}}}
Anders Christensen & LD & 654\\
Ashley Waite & C & 507\\
Liz Hind & Lab & 426\\
Mark Wheeler & Grn & 77\\
\end{tabular*}

\council{Aylesbury Vale}

\subsubsection*{Central and Walton \hspace*{\fill}\nolinebreak[1]%
\enspace\hspace*{\fill}
\finalhyphendemerits=0
[22nd March; LD gain from C]}

\index{Central and Walton , Aylesbury Vale@Central \& Walton, \emph{Aylesbury Vale}}

Resignation of Edward Sims (C).

\noindent
\begin{tabular*}{\columnwidth}{@{\extracolsep{\fill}} p{0.53\columnwidth} >{\itshape}l r @{\extracolsep{\fill}}}
Waheed Raja & LD & 551\\
Lou Redding & C & 425\\
Philip Jacques & Lab & 267\\
Matt Williams & Grn & 61\\
Kyle Michael & Ind & 44\\
\end{tabular*}

\subsubsection*{Quainton \hspace*{\fill}\nolinebreak[1]%
\enspace\hspace*{\fill}
\finalhyphendemerits=0
[3rd May; LD gain from C]}

\index{Quainton , Aylesbury Vale@Quainton, \emph{Aylesbury Vale}}

Death of Kevin Hewson (C).

\noindent
\begin{tabular*}{\columnwidth}{@{\extracolsep{\fill}} p{0.53\columnwidth} >{\itshape}l r @{\extracolsep{\fill}}}
Scott Raven & LD & 564\\
Steven Walker & C & 492\\
Maxine Myatt & Lab & 113\\
Deborah Lovatt & Grn & 47\\
\end{tabular*}

\council{Chiltern}

\subsubsection*{Ridgeway \hspace*{\fill}\nolinebreak[1]%
\enspace\hspace*{\fill}
\finalhyphendemerits=0
[22nd March; C gain from Ind]}

\index{Ridgeway , Chiltern@Ridgeway, \emph{Chiltern}}

Death of Derek Lacey (Ind).

\noindent
\begin{tabular*}{\columnwidth}{@{\extracolsep{\fill}} p{0.6\columnwidth} >{\itshape}l r @{\extracolsep{\fill}}}
Nick Southworth & C & 268\\
Mohammad Zafir Bhatti & Lab & 230\\
Frances Kneller & LD & 203\\
\end{tabular*}

\council{Milton Keynes}

\subsubsection*{Newport Pagnell North and Hanslope \hspace*{\fill}\nolinebreak[1]%
\enspace\hspace*{\fill}
\finalhyphendemerits=0
[18th January]}

\index{Newport Pagnell North and Hanslope , Milton Keynes@Newport Pagnell N. \& Hanslope, \emph{Milton Keynes}}

Death of Jeannette Green (C).

\noindent
\begin{tabular*}{\columnwidth}{@{\extracolsep{\fill}} p{0.53\columnwidth} >{\itshape}l r @{\extracolsep{\fill}}}
Bill Green & C & 1604\\
Nick Phillips & Lab & 749\\
Jane Carr & LD & 672\\
\end{tabular*}

\subsubsection*{Newport Pagnell South \hspace*{\fill}\nolinebreak[1]%
\enspace\hspace*{\fill}
\finalhyphendemerits=0
[3rd May]}

\index{Newport Pagnell South , Milton Keynes@Newport Pagnell S., \emph{Milton Keynes}}

Resignation of Derek Eastman (LD).

Combined with the 2018 ordinary election.
%; see page \pageref{NewportPagnellSouthMiltonKeynes} for the result.

\subsubsection*{Bletchley East \hspace*{\fill}\nolinebreak[1]%
\enspace\hspace*{\fill}
\finalhyphendemerits=0
[19th July]}

\index{Bletchley East , Milton Keynes@Bletchley E., \emph{Milton Keynes}}

Resignation of Alan Webb (Lab).

\noindent
\begin{tabular*}{\columnwidth}{@{\extracolsep{\fill}} p{0.53\columnwidth} >{\itshape}l r @{\extracolsep{\fill}}}
Emily Darlington & Lab & 1355\\
Angela Kennedy & C & 1026\\
Jo Breen & Grn & 131\\
Vince Peddle & UKIP & 101\\
Richard Greenwood & LD & 50\\
\end{tabular*}

\section{Cambridgeshire}

\subsection*{County Council}\index{Cambridgeshire}

\subsubsection*{Soham North and Isleham \hspace*{\fill}\nolinebreak[1]%
	\enspace\hspace*{\fill}
	\finalhyphendemerits=0
	[4th October]}

\index{Soham North and Isleham , Cambridgeshire@Soham N. \& Isleham, \emph{Cambs.}}

Resignation of Paul Raynes (C).

\noindent
\begin{tabular*}{\columnwidth}{@{\extracolsep{\fill}} p{0.53\columnwidth} >{\itshape}l r @{\extracolsep{\fill}}}
Mark Goldsack & C & 858\\
Victoria Charlesworth & LD & 527\\
Lee Jinks & Lab & 191\\
Geoffrey Woollard & Ind & 182\\
\end{tabular*}

\council{Cambridge}

\subsubsection*{East Chesterton \hspace*{\fill}\nolinebreak[1]%
\enspace\hspace*{\fill}
\finalhyphendemerits=0
[3rd May]}

\index{East Chesterton , Cambridge@East Chesterton, \emph{Cambridge}}

Resignation of Margery Abbott (Lab).

Combined with the 2018 ordinary election.
%; see page \pageref{EastChestertonCambridge} for the result.

\subsubsection*{Petersfield \hspace*{\fill}\nolinebreak[1]%
\enspace\hspace*{\fill}
\finalhyphendemerits=0
[13th September]}

\index{Petersfield , Cambridge@Petersfield, \emph{Cambridge}}

Resignation of Ann Sinnott (Lab).

\noindent
\begin{tabular*}{\columnwidth}{@{\extracolsep{\fill}} p{0.53\columnwidth} >{\itshape}l r @{\extracolsep{\fill}}}
Kelley Green & Lab & 873\\
Sarah Brown & LD & 663\\
Virgin Ierubino & Grn & 171\\
Othman Cole & C & 115\\
\end{tabular*}

\council{Fenland}

\subsubsection*{Birch \hspace*{\fill}\nolinebreak[1]%
\enspace\hspace*{\fill}
\finalhyphendemerits=0
[21st June]}

\index{Birch, Fenland@Birch, \emph{Fenland}}

Resignation of Dave Green (C).

\noindent
\begin{tabular*}{\columnwidth}{@{\extracolsep{\fill}} p{0.53\columnwidth} >{\itshape}l r @{\extracolsep{\fill}}}
Ian Benney & C & 326\\
Helena Minton & LD & 113\\
Steve Nicholson & Ind & 86\\
\end{tabular*}

\council{Peterborough}

\subsubsection*{Orton Longueville \hspace*{\fill}\nolinebreak[1]%
\enspace\hspace*{\fill}
\finalhyphendemerits=0
[2nd August]}

\index{Orton Longueville, Peterborough@Orton Longueville, \emph{Peterborough}}

Resignation of June Bull (C).

\noindent
\begin{tabular*}{\columnwidth}{@{\extracolsep{\fill}} p{0.53\columnwidth} >{\itshape}l r @{\extracolsep{\fill}}}
Gavin Elsey & C & 713\\
Heather Skibsted & Lab & 657\\
Daniel Gibbs & LD & 237\\
Alex Airey & Grn & 201\\
Graham Whitehead & UKIP & 143\\
\end{tabular*}

\council{South Cambridgeshire}

At the May 2018 ordinary election there was an unfilled vacancy in Bourn ward due to the resignation of Simon Crocker (C).\index{Bourn , South Cambridgeshire@Bourn, \emph{S. Cambs.}}

\section{Cheshire}

\council{Cheshire East}

\subsubsection*{Bunbury \hspace*{\fill}\nolinebreak[1]%
\enspace\hspace*{\fill}
\finalhyphendemerits=0
[22nd March]}

\index{Bunbury , Cheshire East@Bunbury, \emph{Ches. E.}}

Disqualification (non-attendance) of Michael Jones (Ind elected as C).

\noindent
\begin{tabular*}{\columnwidth}{@{\extracolsep{\fill}} p{0.53\columnwidth} >{\itshape}l r @{\extracolsep{\fill}}}
Chris Green & C & 663\\
Mark Jones & LD & 342\\
Jake Lomax & Lab & 178\\
Mark Sharkey & Grn & 60\\
\end{tabular*}

\council{Cheshire West and Chester}

\subsubsection*{Ellesmere Port Town \hspace*{\fill}\nolinebreak[1]%
\enspace\hspace*{\fill}
\finalhyphendemerits=0
[3rd May]}

\index{Ellesmere Port Town , Cheshire West and Chester@Ellesmere Port Town, \emph{Ches. W. \& Chester}}

Death of Lynn Clare (Lab).

\noindent
\begin{tabular*}{\columnwidth}{@{\extracolsep{\fill}} p{0.53\columnwidth} >{\itshape}l r @{\extracolsep{\fill}}}
Mike Edwardson & Lab & 1447\\
Robert Griffiths & C & 239\\
Mathew Roberts & Grn & 60\\
\end{tabular*}

\council{Halton}

\subsubsection*{Halton Castle \hspace*{\fill}\nolinebreak[1]%
\enspace\hspace*{\fill}
\finalhyphendemerits=0
[15th February]}

\index{Halton Castle , Halton@Halton Castle, \emph{Halton}}

Death of Arthur Cole (Lab).

\noindent
\begin{tabular*}{\columnwidth}{@{\extracolsep{\fill}} p{0.53\columnwidth} >{\itshape}l r @{\extracolsep{\fill}}}
Christopher Carlin & Lab & 522\\
Darrin Whyte & Ind & 133\\
Ian Adams & C & 88\\
\end{tabular*}

\subsubsection*{Halton View \hspace*{\fill}\nolinebreak[1]%
\enspace\hspace*{\fill}
\finalhyphendemerits=0
[3rd May]}

\index{Halton View , Halton@Halton View, \emph{Halton}}

Resignation of Stan Parker (Lab).

Combined with the 2018 ordinary election.
%; see page \pageref{HaltonViewHalton} for the result.

\subsubsection*{Ditton \hspace*{\fill}\nolinebreak[1]%
\enspace\hspace*{\fill}
\finalhyphendemerits=0
[11th October]}

\index{Ditton , Halton@Ditton, \emph{Halton}}

Disqualification (non-attendance) of Shaun Osborne (Lab).

\noindent
\begin{tabular*}{\columnwidth}{@{\extracolsep{\fill}} p{0.53\columnwidth} >{\itshape}l r @{\extracolsep{\fill}}}
Edward Dourley & Lab & 644\\
Daniel Clark & C & 135\\
David Coveney & LD & 97\\
\end{tabular*}

\council{Warrington}

\subsubsection*{Lymm South \hspace*{\fill}\nolinebreak[1]%
\enspace\hspace*{\fill}
\finalhyphendemerits=0
[19th April; LD gain from C]}

\index{Lymm South , Warrington@Lymm S., \emph{Warrington}}

Death of Sheila Woodyatt (C).

\noindent
\begin{tabular*}{\columnwidth}{@{\extracolsep{\fill}} p{0.53\columnwidth} >{\itshape}l r @{\extracolsep{\fill}}}
Anna Fradgley & LD & 769\\
Stephen Taylor & C & 649\\
Trish Cockayne & Lab & 328\\
Derek Ashington & UKIP & 25\\
Michael Wass & Grn & 24\\
\end{tabular*}

\subsubsection*{Penketh and Cuerdley \hspace*{\fill}\nolinebreak[1]%
\enspace\hspace*{\fill}
\finalhyphendemerits=0
[11th October; Ind gain from Lab]}

\index{Penketh and Cuerdley , Warrington@Penketh \& Cuerdley, \emph{Warrington}}

Death of Allin Dirir (Lab).

\noindent
\begin{tabular*}{\columnwidth}{@{\extracolsep{\fill}} p{0.53\columnwidth} >{\itshape}l r @{\extracolsep{\fill}}}
Geoff Fellows & Ind & 784\\
Kenny Watson & Lab & 691\\
Philip Hayward & C & 479\\
David Crowther & LD & 100\\
Ian Wilson & UKIP & 69\\
Stephanie Davies & Grn & 47\\
\end{tabular*}

\section{Cornwall}

\council{Cornwall}

\subsubsection*{Falmouth Smithick \hspace*{\fill}\nolinebreak[1]%
\enspace\hspace*{\fill}
\finalhyphendemerits=0
[1st February]}

\index{Falmouth Smithick , Cornwall@Falmouth Smithick, \emph{Cornwall}}

Death of Candy Atherton (Lab).

\noindent
\begin{tabular*}{\columnwidth}{@{\extracolsep{\fill}} p{0.53\columnwidth} >{\itshape}l r @{\extracolsep{\fill}}}
Jayne Kirkham & Lab & 643\\
Richard Cunningham & C & 184\\
John Spargo & LD & 184\\
Tom Scott & Grn & 57\\
\end{tabular*}

\subsubsection*{Newquay Treviglas \hspace*{\fill}\nolinebreak[1]%
\enspace\hspace*{\fill}
\finalhyphendemerits=0
[9th August; C gain from LD]}

\index{Newquay Treviglas , Cornwall@Newquay Treviglas, \emph{Cornwall}}

Death of Paul Summers (LD).

\noindent
\begin{tabular*}{\columnwidth}{@{\extracolsep{\fill}} p{0.53\columnwidth} >{\itshape}l r @{\extracolsep{\fill}}}
Mark Formosa & C & 363\\
Steven Daniell & LD & 306\\
Brod Ross & Lab & 131\\
\end{tabular*}

\subsubsection*{Bude \hspace*{\fill}\nolinebreak[1]%
\enspace\hspace*{\fill}
\finalhyphendemerits=0
[23rd August]}

\index{Bude , Cornwall@Bude, \emph{Cornwall}}

Resignation of Nigel Pearce (LD).

\noindent
\begin{tabular*}{\columnwidth}{@{\extracolsep{\fill}} p{0.53\columnwidth} >{\itshape}l r @{\extracolsep{\fill}}}
David Parsons & LD & 1010\\
Bob Willingham & Ind & 475\\
Alex Dart & C & 264\\
Ray Shemilt & Lab & 148\\
\end{tabular*}

\section{Cumbria}

\subsection*{County Council}\index{Cumbria}

\subsubsection*{Denton Holme \hspace*{\fill}\nolinebreak[1]%
\enspace\hspace*{\fill}
\finalhyphendemerits=0
[6th September]}

\index{Denton Holme , Cumbria@Denton Holme, \emph{Cumbria}}

Death of Hugh McDevitt (Lab).

\noindent
\begin{tabular*}{\columnwidth}{@{\extracolsep{\fill}} p{0.53\columnwidth} >{\itshape}l r @{\extracolsep{\fill}}}
Karen Lockney & Lab & 625\\
Geoffrey Osborne & C & 292\\
Helen Davison & Grn & 94\\
Phil Douglass & UKIP & 46\\
\end{tabular*}

\subsubsection*{Kent Estuary \hspace*{\fill}\nolinebreak[1]%
	\enspace\hspace*{\fill}
	\finalhyphendemerits=0
	[20th December]}

\index{Kent Estuary , Cumbria@Kent Estuary, \emph{Cumbria}}

Death of Ian Stewart (LD).

\noindent
\begin{tabular*}{\columnwidth}{@{\extracolsep{\fill}} p{0.53\columnwidth} >{\itshape}l r @{\extracolsep{\fill}}}
Pete McSweeney & LD & 1381\\
Tom Harvey & C & 666\\
Jill Abel & Grn & 109\\
Kate Love & Lab & 70\\
\end{tabular*}

\council{Carlisle}

\subsubsection*{Denton Holme \hspace*{\fill}\nolinebreak[1]%
\enspace\hspace*{\fill}
\finalhyphendemerits=0
[6th September]}

\index{Denton Holme , Carlisle@Denton Holme, \emph{Carlisle}}

Death of Hugh McDevitt (Lab).

\noindent
\begin{tabular*}{\columnwidth}{@{\extracolsep{\fill}} p{0.53\columnwidth} >{\itshape}l r @{\extracolsep{\fill}}}
Lisa Brown & Lab & 647\\
Syed Ali & C & 254\\
Rob Morrison & Grn & 78\\
Phil Douglass & UKIP & 57\\
\end{tabular*}

\council{Eden}

\subsubsection*{Hartside \hspace*{\fill}\nolinebreak[1]%
\enspace\hspace*{\fill}
\finalhyphendemerits=0
[8th February]}

\index{Hartside , Eden@Hartside, \emph{Eden}}

Death of Sheila Orchard (C).

\noindent
\begin{tabular*}{\columnwidth}{@{\extracolsep{\fill}} p{0.53\columnwidth} >{\itshape}l r @{\extracolsep{\fill}}}
Robin Orchard & C & 175\\
Susan Castle-Clarke & Ind & 98\\
Richard Henry & Grn & 58\\
\end{tabular*}

\council{South Lakeland}

At the May 2018 ordinary election there was an unfilled vacancy in Burneside ward due to the resignation of Keith Hurst-Jones (LD).\index{Burneside , South Lakeland@Burneside, \emph{S. Lakeland}}

\subsubsection*{Arnside and Milnthorpe \hspace*{\fill}\nolinebreak[1]%
	\enspace\hspace*{\fill}
	\finalhyphendemerits=0
	[20th December]}

\index{Arnside and Milnthorpe , South Lakeland@Arnside \& Milnthorpe, \emph{S. Lakeland}}

Death of Ian Stewart (LD).

\noindent
\begin{tabular*}{\columnwidth}{@{\extracolsep{\fill}} p{0.53\columnwidth} >{\itshape}l r @{\extracolsep{\fill}}}
Helen Chaffey & LD & 1319\\
Rachel Ashburner & C & 709\\
Jill Abel & Grn & 125\\
Kate Love & Lab & 68\\
\end{tabular*}

\section{Derbyshire}

\council{Chesterfield}

\subsubsection*{Moor \hspace*{\fill}\nolinebreak[1]%
\enspace\hspace*{\fill}
\finalhyphendemerits=0
[4th October; LD gain from Lab]}

\index{Moor , Chesterfield@Moor, \emph{Chesterfield}}

Death of Keith Brown (Lab).

\noindent
\begin{tabular*}{\columnwidth}{@{\extracolsep{\fill}} p{0.53\columnwidth} >{\itshape}l r @{\extracolsep{\fill}}}
Tony Rogers & LD & 532\\
Ron Mihaly & Lab & 445\\
Gordon Partington & C & 84\\
Barry Thompson & UKIP & 69\\
\end{tabular*}

\council{North East Derbyshire}

\subsubsection*{Grassmoor \hspace*{\fill}\nolinebreak[1]%
\enspace\hspace*{\fill}
\finalhyphendemerits=0
[15th February]}

\index{Grassmoor , North East Derbyshire@Grassmoor, \emph{N.E. Derbys.}}

Resignation of Julie Hill (Lab).

\noindent
\begin{tabular*}{\columnwidth}{@{\extracolsep{\fill}} p{0.53\columnwidth} >{\itshape}l r @{\extracolsep{\fill}}}
Dick Marriott & Lab & 459\\
Josh Broadhurst & C & 368\\
Ben Marshall & LD & 111\\
\end{tabular*}

\council{South Derbyshire}

\subsubsection*{Linton \hspace*{\fill}\nolinebreak[1]%
	\enspace\hspace*{\fill}
	\finalhyphendemerits=0
	[25th October]}

\index{Linton , South Derbyshire@Linton, \emph{S. Derbys.}}

Death of Bob Wheeler (C).

\noindent
\begin{tabular*}{\columnwidth}{@{\extracolsep{\fill}} p{0.53\columnwidth} >{\itshape}l r @{\extracolsep{\fill}}}
Dan Pegg & C & 623\\
Ben Stuart & Lab & 316\\
Lorraine Johnson & LD & 48\\
\end{tabular*}

\section{Devon}

\council{East Devon}

\subsubsection*{Exmouth Town \hspace*{\fill}\nolinebreak[1]%
\enspace\hspace*{\fill}
\finalhyphendemerits=0
[1st March]}

\index{Exmouth Town , East Devon@Exmouth Town, \emph{E. Devon}}

Resignation of Pat Graham (LD).

\noindent
\begin{tabular*}{\columnwidth}{@{\extracolsep{\fill}} p{0.53\columnwidth} >{\itshape}l r @{\extracolsep{\fill}}}
Tim Dumper & LD & 187\\
Daphne Currier & Ind & 176\\
Tony Hill & C & 142\\
Dilys Hadley & Lab & 86\\
Robert Masding & Grn & 71\\
\end{tabular*}

\subsubsection*{Ottery St Mary Rural \hspace*{\fill}\nolinebreak[1]%
\enspace\hspace*{\fill}
\finalhyphendemerits=0
[20th September]}

\index{Ottery Saint Mary Rural , East Devon@Ottery St Mary Rural, \emph{E. Devon}}

Disqualification (non-attendance) of Matt Coppell (Ind East Devon Alliance).

\noindent
\begin{tabular*}{\columnwidth}{@{\extracolsep{\fill}} p{0.53\columnwidth} >{\itshape}l r @{\extracolsep{\fill}}}
Geoff Pratt & Ind & 755\\
John Sheaves & C & 421\\
Nick Benson & LD & 51\\
Margaret Bargmann & Grn & 24\\
Richard May & Lab & 20\\
\end{tabular*}

\council{Mid Devon}

\subsubsection*{Cranmore \hspace*{\fill}\nolinebreak[1]%
\enspace\hspace*{\fill}
\finalhyphendemerits=0
[7th June]}

\index{Cranmore , Mid Devon@Cranmore, \emph{Mid Devon}}

Death of Clarissa Slade (C).

\noindent
\begin{tabular*}{\columnwidth}{@{\extracolsep{\fill}} p{0.53\columnwidth} >{\itshape}l r @{\extracolsep{\fill}}}
Lance Kennedy & C & 479\\
Les Cruwys & LD & 346\\
Steve Bush & Lab & 238\\
\end{tabular*}

\council{North Devon}

\subsubsection*{Fremington \hspace*{\fill}\nolinebreak[1]%
\enspace\hspace*{\fill}
\finalhyphendemerits=0
[28th June]}

\index{Fremington , North Devon@Fremington, \emph{N. Devon}}

Death of Tony Wood (Ind).

\noindent
\begin{tabular*}{\columnwidth}{@{\extracolsep{\fill}} p{0.53\columnwidth} >{\itshape}l r @{\extracolsep{\fill}}}
Jayne Mackie & Ind & 577\\
Jim Pilkington & C & 356\\
Graham Lofthouse & LD & 119\\
Blake Ladley & Lab & 65\\
Lou Goodger & Grn & 19\\
\end{tabular*}

\columnbreak

\council{Plymouth}

AFPl = Active for Plymouth

\subsubsection*{Stoke \hspace*{\fill}\nolinebreak[1]%
\enspace\hspace*{\fill}
\finalhyphendemerits=0
[26th July]}

\index{Stoke , Plymouth@Stoke, \emph{Plymouth}}

Resignation of Philippa Davey (Lab).

\noindent
\begin{tabular*}{\columnwidth}{@{\extracolsep{\fill}} p{0.53\columnwidth} >{\itshape}l r @{\extracolsep{\fill}}}
Jemima Laing & Lab & 1427\\
Kathy Watkin & C & 981\\
Connor Clarke & LD & 174\\
Iuliu Popescu & AFPl & 123\\
\end{tabular*}

\council{Teignbridge}

\subsubsection*{Chudleigh \hspace*{\fill}\nolinebreak[1]%
\enspace\hspace*{\fill}
\finalhyphendemerits=0
[15th February; LD gain from C]}

\index{Chudleigh , Teignbridge@Chudleigh, \emph{Teignbridge}}

Disqualification (sentenced to three years' imprisonment, wounding with intent to cause grievous bodily harm) of Doug Laing (C).

\noindent
\begin{tabular*}{\columnwidth}{@{\extracolsep{\fill}} p{0.53\columnwidth} >{\itshape}l r @{\extracolsep{\fill}}}
Lorraine Evans & LD & 575\\
Pam Elliott & C & 564\\
Janette Parker & Lab & 262\\
\end{tabular*}

\subsubsection*{Dawlish Central and North East \hspace*{\fill}\nolinebreak[1]%
\enspace\hspace*{\fill}
\finalhyphendemerits=0
[15th February; LD gain from C]}

\index{Dawlish Central and North East , Teignbridge@Dawlish C. \& N.E., \emph{Teignbridge}}

Resignation of Graham Price (C).

\noindent
\begin{tabular*}{\columnwidth}{@{\extracolsep{\fill}} p{0.53\columnwidth} >{\itshape}l r @{\extracolsep{\fill}}}
Martin Wrigley & LD & 1287\\
Angela Fenne & C & 535\\
\end{tabular*}

\council{Torridge}

\subsubsection*{Bideford East \hspace*{\fill}\nolinebreak[1]%
\enspace\hspace*{\fill}
\finalhyphendemerits=0
[3rd May; C gain from UKIP]}

\index{Bideford East , Torridge@Bideford E., \emph{Torridge}}

Death of Sam Robinson (UKIP).

\noindent
\begin{tabular*}{\columnwidth}{@{\extracolsep{\fill}} p{0.6\columnwidth} >{\itshape}l r @{\extracolsep{\fill}}}
James Hellyer & C & 383\\
Anne Brenton & Lab & 273\\
Jamie McKenzie & LD & 236\\
Jude Gubb & Ind & 231\\
Gregory de Freyne-Martin & Grn & 62\\
Pauline Davies & Ind & 56\\
\end{tabular*}

\subsubsection*{Hartland and Bradworthy \hspace*{\fill}\nolinebreak[1]%
\enspace\hspace*{\fill}
\finalhyphendemerits=0
[26th July; C gain from LD]}

\index{Hartland and Bradworthy , Torridge@Hartland \& Bradworthy, \emph{Torridge}}

Resignation of Jane Leaper (LD).

\noindent
\begin{tabular*}{\columnwidth}{@{\extracolsep{\fill}} p{0.53\columnwidth} >{\itshape}l r @{\extracolsep{\fill}}}
Richard Boughton & C & 408\\
Martin Hill & LD & 204\\
John Sanders & Grn & 85\\
\end{tabular*}

\subsubsection*{Holsworthy \hspace*{\fill}\nolinebreak[1]%
	\enspace\hspace*{\fill}
	\finalhyphendemerits=0
	[8th November]}

\index{Holsworthy , Torridge@Holsworthy, \emph{Torridge}}

Death of Ken Carroll (C).

\noindent
\begin{tabular*}{\columnwidth}{@{\extracolsep{\fill}} p{0.6\columnwidth} >{\itshape}l r @{\extracolsep{\fill}}}
Jon Hutchings & C & 698\\
John Allen & Ind & 314\\
Christopher Styles-Power & LD & 151\\
Tom Hammett & Lab & 75\\
\end{tabular*}

\section{Dorset}

\subsection*{County Council}\index{Dorset}

\subsubsection*{Bridport \hspace*{\fill}\nolinebreak[1]%
\enspace\hspace*{\fill}
\finalhyphendemerits=0
[22nd February; C gain from LD]}

\index{Bridport , Dorset@Bridport, \emph{Dorset}}

Resignation of Ros Kayes (LD).

\noindent
\begin{tabular*}{\columnwidth}{@{\extracolsep{\fill}} p{0.53\columnwidth} >{\itshape}l r @{\extracolsep{\fill}}}
Mark Roberts & C & 1660\\
Dave Rickard & LD & 1451\\
Rose Allwork & Lab & 691\\
Kelvin Clayton & Grn & 388\\
\end{tabular*}

\subsubsection*{Ferndown \hspace*{\fill}\nolinebreak[1]%
\enspace\hspace*{\fill}
\finalhyphendemerits=0
[25th October]}

\index{Ferndown , Dorset@Ferndown, \emph{Dorset}}

Death of Steve Lugg (C).

\noindent
\begin{tabular*}{\columnwidth}{@{\extracolsep{\fill}} p{0.53\columnwidth} >{\itshape}l r @{\extracolsep{\fill}}}
Mike Parkes & C & 1878\\
Matthew Coussell & LD & 647\\
Lawrence Wilson & UKIP & 540\\
\end{tabular*}

\council{Bournemouth}

\subsubsection*{Throop and Muscliff \hspace*{\fill}\nolinebreak[1]%
\enspace\hspace*{\fill}
\finalhyphendemerits=0
[18th January]}

\index{Throop and Muscliff , Bournemouth@Throop \& Muscliff, \emph{Bournemouth}}

Resignation of Anne Rey (Ind).

\noindent
\begin{tabular*}{\columnwidth}{@{\extracolsep{\fill}} p{0.53\columnwidth} >{\itshape}l r @{\extracolsep{\fill}}}
Kieron Wilson & Ind & 533\\
Hazel Allen & C & 511\\
Rob Bassinder & Lab & 402\\
Peter Lucas & Ind & 117\\
Muriel Turner & LD & 107\\
Jane Bull & Grn & 33\\
\end{tabular*}

\council{East Dorset}

\subsubsection*{Verwood East \hspace*{\fill}\nolinebreak[1]%
\enspace\hspace*{\fill}
\finalhyphendemerits=0
[12th July]}

\index{Verwood East , East Dorset@Verwood E., \emph{E. Dorset}}

Death of Boyd Mortimer (C).

\noindent
\begin{tabular*}{\columnwidth}{@{\extracolsep{\fill}} p{0.53\columnwidth} >{\itshape}l r @{\extracolsep{\fill}}}
Colin Beck & C & 706\\
Sandra Turner & Lab & 234\\
\end{tabular*}

\subsubsection*{Ferndown Central \hspace*{\fill}\nolinebreak[1]%
\enspace\hspace*{\fill}
\finalhyphendemerits=0
[25th October]}

\index{Ferndown Central , East Dorset@Ferndown C., \emph{E. Dorset}}

Death of Steve Lugg (C).

\noindent
\begin{tabular*}{\columnwidth}{@{\extracolsep{\fill}} p{0.53\columnwidth} >{\itshape}l r @{\extracolsep{\fill}}}
Mike Parkes & C & 899\\
Matthew Coussell & LD & 355\\
Lawrence Wilson & UKIP & 246\\
\end{tabular*}

\council{West Dorset}

\subsubsection*{Bridport North \hspace*{\fill}\nolinebreak[1]%
\enspace\hspace*{\fill}
\finalhyphendemerits=0
[22nd February; C gain from LD]}

\index{Bridport North , West Dorset@Bridport N., \emph{W. Dorset}}

Resignation of Ros Kayes (LD).

\noindent
\begin{tabular*}{\columnwidth}{@{\extracolsep{\fill}} p{0.53\columnwidth} >{\itshape}l r @{\extracolsep{\fill}}}
Derek Bussell & C & 600\\
Sarah Williams & LD & 500\\
Phyllida Culpin & Lab & 383\\
Julian Jones & Grn & 184\\
\end{tabular*}

\council{Weymouth and Portland}

\subsubsection*{Tophill East \hspace*{\fill}\nolinebreak[1]%
\enspace\hspace*{\fill}
\finalhyphendemerits=0
[8th February; C gain from Ind]}

\index{Tophill East , Weymouth and Portland@Tophill E., \emph{Weymouth \& Portland}}

Resignation of David Hawkins (Ind).

\noindent
\begin{tabular*}{\columnwidth}{@{\extracolsep{\fill}} p{0.53\columnwidth} >{\itshape}l r @{\extracolsep{\fill}}}
Katharine Garcia & C & 362\\
Becky Blake & Lab & 354\\
Sara Harpley & Grn & 56\\
\end{tabular*}

\subsubsection*{Tophill West \hspace*{\fill}\nolinebreak[1]%
\enspace\hspace*{\fill}
\finalhyphendemerits=0
[8th February]}

\index{Tophill West , Weymouth and Portland@Tophill W., \emph{Weymouth \& Portland}}

Resignation of Jason Webb (C).

\noindent
\begin{tabular*}{\columnwidth}{@{\extracolsep{\fill}} p{0.53\columnwidth} >{\itshape}l r @{\extracolsep{\fill}}}
Kerry Baker & C & 511\\
Giovanna Lewis & Lab & 356\\
Carole Timmons & Grn & 82\\
\end{tabular*}

\subsubsection*{Weymouth West \hspace*{\fill}\nolinebreak[1]%
\enspace\hspace*{\fill}
\finalhyphendemerits=0
[3rd May]}

\index{Weymouth West , Weymouth and Portland@Weymouth W., \emph{Weymouth \& Portland}}

Resignation of Claudia Moore (Grn, elected as Claudia Webb, C).

\noindent
\begin{tabular*}{\columnwidth}{@{\extracolsep{\fill}} p{0.53\columnwidth} >{\itshape}l r @{\extracolsep{\fill}}}
Richard Nickinson & C & 558\\
Val Graves & Grn & 508\\
David Greenhalf & Lab & 354\\
\end{tabular*}

\section{Durham}

\council{Darlington}

ForBritn = For Britain Movement

\subsubsection*{Cockerton \hspace*{\fill}\nolinebreak[1]%
\enspace\hspace*{\fill}
\finalhyphendemerits=0
[12th July]}

\index{Cockerton , Darlington@Cockerton, \emph{Darlington}}

Death of David Regan (Lab).

\noindent
\begin{tabular*}{\columnwidth}{@{\extracolsep{\fill}} p{0.53\columnwidth} >{\itshape}l r @{\extracolsep{\fill}}}
Eddie Heslop & Lab & 555\\
Scott Durham & C & 239\\
Charlie Curry & LD & 104\\
Joel Alexander & Ind & 93\\
Kevin Brack & ForBritn & 63\\
Terri Hankinson & Grn & 34\\
\end{tabular*}

\council{Hartlepool}

\subsubsection*{Rural West \hspace*{\fill}\nolinebreak[1]%
\enspace\hspace*{\fill}
\finalhyphendemerits=0
[12th July]}

\index{Rural West , Hartlepool@Rural W., \emph{Hartlepool}}

Resignation of Ray Martin-Wells (C).

\noindent
\begin{tabular*}{\columnwidth}{@{\extracolsep{\fill}} p{0.53\columnwidth} >{\itshape}l r @{\extracolsep{\fill}}}
Mike Young & C & 678\\
James Brewer & Ind & 546\\
Yousuf Khan & Lab & 184\\
Michael Holt & Grn & 87\\
\end{tabular*}

\subsubsection*{Hart \hspace*{\fill}\nolinebreak[1]%
	\enspace\hspace*{\fill}
	\finalhyphendemerits=0
	[11th October; Ind gain from Lab]}

\index{Hart , Hartlepool@Hart, \emph{Hartlepool}}

Resignation of Paul Beck (Lab).

\noindent
\begin{tabular*}{\columnwidth}{@{\extracolsep{\fill}} p{0.53\columnwidth} >{\itshape}l r @{\extracolsep{\fill}}}
James Brewer & Ind & 637\\
Aileen Kendon & Lab & 582\\
Cameron Stokell & C & 200\\
Michael Holt & Grn & 27\\
\end{tabular*}

\section{East Sussex}

\council{Brighton and Hove}

\subsubsection*{East Brighton \hspace*{\fill}\nolinebreak[1]%
\enspace\hspace*{\fill}
\finalhyphendemerits=0
[8th February]}

\index{East Brighton , Brighton and Hove@East Brighton, \emph{Brighton \& Hove}}

Resignation of Lloyd Russell-Moyle MP (Lab).

\noindent
\begin{tabular*}{\columnwidth}{@{\extracolsep{\fill}} p{0.53\columnwidth} >{\itshape}l r @{\extracolsep{\fill}}}
Nancy Platts & Lab & 1889\\
Edward Wilson & C & 481\\
Ed Baker & Grn & 316\\
George Taylor & LD & 114\\
\end{tabular*}

\council{Lewes}

\subsubsection*{Chailey and Wivelsfield \hspace*{\fill}\nolinebreak[1]%
\enspace\hspace*{\fill}
\finalhyphendemerits=0
[12th July]}

\index{Chailey and Wivelsfield , Lewes@Chailey \& Wivelsfield, \emph{Lewes}}

Resignation of Cyril Sugarman (C).

\noindent
\begin{tabular*}{\columnwidth}{@{\extracolsep{\fill}} p{0.53\columnwidth} >{\itshape}l r @{\extracolsep{\fill}}}
Nancy Bikson & C & 563\\
Marion Hughes & LD & 324\\
Nicholas Belcher & Lab & 104\\
Brenda Barnes & Grn & 60\\
\end{tabular*}

\section{Essex}

\council{Basildon}

\subsubsection*{Pitsea South East \hspace*{\fill}\nolinebreak[1]%
\enspace\hspace*{\fill}
\finalhyphendemerits=0
[3rd May]}

\index{Pitsea South East , Basildon@Pitsea S.E., \emph{Basildon}}

Resignation of Amanda Arnold (C).

Combined with the 2018 ordinary election.
%; see page \pageref{PitseaSoutEastBasildon} for the result.

\subsubsection*{Lee Chapel North \hspace*{\fill}\nolinebreak[1]%
\enspace\hspace*{\fill}
\finalhyphendemerits=0
[21st June]}

\index{Lee Chapel North , Basildon@Lee Chapel N., \emph{Basildon}}

Resignation of Alan Bennett (Lab).

\noindent
\begin{tabular*}{\columnwidth}{@{\extracolsep{\fill}} p{0.53\columnwidth} >{\itshape}l r @{\extracolsep{\fill}}}
Kayode Adeniran & Lab & 612\\
Spencer Warner & C & 267\\
Frank Ferguson & UKIP & 145\\
Christine Winter & BNP & 42\\
\end{tabular*}

\subsubsection*{Pitsea South East \hspace*{\fill}\nolinebreak[1]%
\enspace\hspace*{\fill}
\finalhyphendemerits=0
[21st June; Lab gain from UKIP]}

\index{Pitsea South East , Basildon@Pitsea S.E., \emph{Basildon}}

Resignation of Jose Carrion (UKIP).

\noindent
\begin{tabular*}{\columnwidth}{@{\extracolsep{\fill}} p{0.53\columnwidth} >{\itshape}l r @{\extracolsep{\fill}}}
Andrew Ansell & Lab & 718\\
Yetunde Adeshile & C & 710\\
Richard Morris & UKIP & 130\\
\end{tabular*}

\council{Braintree}

\subsubsection*{Bocking North \hspace*{\fill}\nolinebreak[1]%
\enspace\hspace*{\fill}
\finalhyphendemerits=0
[3rd May; Lab gain from C]}

\index{Bocking North , Braintree@Bocking N., \emph{Braintree}}

Resignation of Stephanie Paul (C).

\noindent
\begin{tabular*}{\columnwidth}{@{\extracolsep{\fill}} p{0.53\columnwidth} >{\itshape}l r @{\extracolsep{\fill}}}
Tony Everard & Lab & 619\\
Dean Wallace & C & 540\\
Sam Cowie & UKIP & 78\\
Dawn Holmes & Grn & 44\\
\end{tabular*}

\subsubsection*{Hatfield Peverel and Terling \hspace*{\fill}\nolinebreak[1]%
\enspace\hspace*{\fill}
\finalhyphendemerits=0
[3rd May]}

\index{Hatfield Peverel and Terling , Braintree@Hatfield Peverel \& Terling, \emph{Braintree}}

Resignation of Daryn Hufton-Rees (C).

\noindent
\begin{tabular*}{\columnwidth}{@{\extracolsep{\fill}} p{0.53\columnwidth} >{\itshape}l r @{\extracolsep{\fill}}}
James Coleridge & C & 965\\
Jonathan Barker & Grn & 293\\
Lucy Barlow & Lab & 234\\
\end{tabular*}

\council{Brentwood}

At the May 2018 ordinary election there was an unfilled vacancy in Herongate, Ingrave and West Horndon ward due to the resignation of Jo Squirrell (LD elected as Brentwood First).\index{Herongate, Ingrave and West Horndon , Brentwood@Herongate, Ingrave \& West Horndon, \emph{Brentwood}}

\council{Epping Forest}

At the May 2018 ordinary election there were unfilled vacancies in Buckhurst Hill West ward and in Chipping Ongar, Greensted and Marden Ash ward due to the resignation of Sylvia Watson (Ind elected as C) and the death of Brian Surtees (LD) respectively.\index{Buckhurst Hill West , Epping Forest@Buckhurst Hill W., \emph{Epping Forest}}\index{Chipping Ongar, Greensted and Marden Ash , Epping Forest@Chipping Ongar, Greensted \& Marden Ash, \emph{Epping Forest}}

Loughton = Loughton Residents Association

\subsubsection*{Loughton Broadway \hspace*{\fill}\nolinebreak[1]%
\enspace\hspace*{\fill}
\finalhyphendemerits=0
[3rd May]}

\index{Loughton Broadway , Epping Forest@Loughton Broadway, \emph{Epping Forest}}

Resignation of Leon Girling (Loughton).

Combined with the 2018 ordinary election.
%; see page \pageref{LoughtonBroadwayEppingForest} for the result.

\subsubsection*{Moreton and Fyfield \hspace*{\fill}\nolinebreak[1]%
\enspace\hspace*{\fill}
\finalhyphendemerits=0
[3rd May]}

\index{Moreton and Fyfield , Epping Forest@Moreton \& Fyfield, \emph{Epping Forest}}

Resignation of Tony Boyce (C).

\noindent
\begin{tabular*}{\columnwidth}{@{\extracolsep{\fill}} p{0.53\columnwidth} >{\itshape}l r @{\extracolsep{\fill}}}
Ian Hadley & C & \emph{unop.}\\
\end{tabular*}

\subsubsection*{Waltham Abbey Honey Lane \hspace*{\fill}\nolinebreak[1]%
\enspace\hspace*{\fill}
\finalhyphendemerits=0
[3rd May]}

\index{Waltham Abbey Honey Lane , Epping Forest@Waltham Abbey Honey Lane, \emph{Epping Forest}}

Resignation of Glynis Shiell (C).

Combined with the 2018 ordinary election.
%; see page \pageref{WalthamAbbeyHoneyLaneEppingForest} for the result.

\council{Harlow}

HlowAll = Harlow Alliance Party

\subsubsection*{Little Parndon and Hare Street \hspace*{\fill}\nolinebreak[1]%
\enspace\hspace*{\fill}
\finalhyphendemerits=0
[8th March]}

\index{Little Parndon and Hare Street , Harlow@Little Parndon \& Hare St., \emph{Harlow}}

Resignation of John Clempner (Lab).

\noindent
\begin{tabular*}{\columnwidth}{@{\extracolsep{\fill}} p{0.53\columnwidth} >{\itshape}l r @{\extracolsep{\fill}}}
Chris Vince & Lab & 781\\
John Steer & C & 394\\
Patsy Long & UKIP & 80\\
\end{tabular*}

\subsubsection*{Bush Fair \hspace*{\fill}\nolinebreak[1]%
	\enspace\hspace*{\fill}
	\finalhyphendemerits=0
	[8th November]}

\index{Bush Fair , Harlow@Bush Fair, \emph{Harlow}}

Resignation of Ian Beckett (Lab).

\noindent
\begin{tabular*}{\columnwidth}{@{\extracolsep{\fill}} p{0.53\columnwidth} >{\itshape}l r @{\extracolsep{\fill}}}
Jodi Dunne & Lab & 543\\
Andreea Hardware & C & 460\\
Anita Long & UKIP & 103\\
Nicholas Taylor & HlowAll & 63\\
Lesley Rideout & LD & 39\\
\end{tabular*}

\subsubsection*{Netteswell \hspace*{\fill}\nolinebreak[1]%
	\enspace\hspace*{\fill}
	\finalhyphendemerits=0
	[8th November]}

\index{Netteswell , Harlow@Netteswell, \emph{Harlow}}

Resignation of Waida Forman (Lab).

\noindent
\begin{tabular*}{\columnwidth}{@{\extracolsep{\fill}} p{0.53\columnwidth} >{\itshape}l r @{\extracolsep{\fill}}}
Shannon Jezzard & Lab & 497\\
Jake Brackstone & C & 254\\
Alan Leverett & HlowAll & 99\\
Mark Gough & UKIP & 98\\
Robert Thurston & LD & 43\\
\end{tabular*}

\subsubsection*{Toddbrook \hspace*{\fill}\nolinebreak[1]%
	\enspace\hspace*{\fill}
	\finalhyphendemerits=0
	[13th December]}

\index{Toddbrook , Harlow@Toddbrook, \emph{Harlow}}

Resignation of Karen Clempner (Lab).

\noindent
\begin{tabular*}{\columnwidth}{@{\extracolsep{\fill}} p{0.53\columnwidth} >{\itshape}l r @{\extracolsep{\fill}}}
Frances Mason & Lab & 464\\
Tom Reynolds & C & 311\\
Dan Long & UKIP & 89\\
Christopher Millington & LD & 44\\
\end{tabular*}

\columnbreak

\council{Rochford}

Rochford = Rochford District Residents

\subsubsection*{Downhall and Rawreth \hspace*{\fill}\nolinebreak[1]%
\enspace\hspace*{\fill}
\finalhyphendemerits=0
[18th January]}

\index{Downhall and Rawreth , Rochford@Downhall \& Rawreth, \emph{Rochford}}

Death of Chris Black (LD).

\noindent
\begin{tabular*}{\columnwidth}{@{\extracolsep{\fill}} p{0.53\columnwidth} >{\itshape}l r @{\extracolsep{\fill}}}
Craig Cannell & LD & 794\\
Tony Hollis & C & 237\\
\end{tabular*}

\subsubsection*{Hockley \hspace*{\fill}\nolinebreak[1]%
\enspace\hspace*{\fill}
\finalhyphendemerits=0
[3rd May]}

\index{Hockley , Rochford@Hockley, \emph{Rochford}}

Resignation of Irena Cassar (Rochford).

Combined with the 2018 ordinary election.
%; see page \pageref{HockleyRochford} for the result.

\council{Tendring}

\subsubsection*{St Pauls \hspace*{\fill}\nolinebreak[1]%
\enspace\hspace*{\fill}
\finalhyphendemerits=0
[15th February; C gain from UKIP]}

\index{Saint Pauls , Tendring@St Pauls, \emph{Tendring}}

Resignation of Jack Parsons (Ind elected as UKIP).

\noindent
\begin{tabular*}{\columnwidth}{@{\extracolsep{\fill}} p{0.6\columnwidth} >{\itshape}l r @{\extracolsep{\fill}}}
Sue Honeywood & C & 378\\
Stephen Andrews & Ind & 160\\
William Hones & Ind & 134\\
Rosie-Roella Kevlin & Lab & 114\\
Keith Pitkin & LD & 79\\
Mike Vaughan-Chatfield & UKIP & 71\\
Robert Cockroft & Grn & 20\\
\end{tabular*}

\council{Thurrock}

Thurrock = Thurrock Independent

\subsubsection*{Ockendon \hspace*{\fill}\nolinebreak[1]%
\enspace\hspace*{\fill}
\finalhyphendemerits=0
[22nd March; C gain from UKIP]}

\index{Ockendon , Thurrock@Ockendon, \emph{Thurrock}}

Resignation of Kevin Wheeler (Thurrock elected as UKIP).  Result decided by drawing of lots.

\noindent
\begin{tabular*}{\columnwidth}{@{\extracolsep{\fill}} p{0.53\columnwidth} >{\itshape}l r @{\extracolsep{\fill}}}
Andrew Jefferies (\emph{el\-ect\-ed}) & C & 696\\
Les Strange (\emph{not elected}) & Lab & 696\\
Allen Mayes & Thurrock & 531\\
\end{tabular*}

\section{Gloucestershire}

\council{Cheltenham}

\subsubsection*{Leckhampton \hspace*{\fill}\nolinebreak[1]%
\enspace\hspace*{\fill}
\finalhyphendemerits=0
[3rd May]}

\index{Leckhampton , Cheltenham@Leckhampton, \emph{Cheltenham}}

Resignation of Ian Bickerton (Ind).

Combined with the 2018 ordinary election.
%; see page \pageref{LeckhamptonCheltenham} for the result.

\council{South Gloucestershire}

\subsubsection*{Dodington \hspace*{\fill}\nolinebreak[1]%
	\enspace\hspace*{\fill}
	\finalhyphendemerits=0
	[1st November]}

\index{Dodington , South Gloucestershire@Dodington, \emph{S. Glos.}}

Death of Gloria Stephen (LD).

\noindent
\begin{tabular*}{\columnwidth}{@{\extracolsep{\fill}} p{0.53\columnwidth} >{\itshape}l r @{\extracolsep{\fill}}}
Louise Harris & LD & 693\\
Ian Livermore & C & 564\\
John Malone & Lab & 158\\
\end{tabular*}

\council{Stroud}

\subsubsection*{Dursley \hspace*{\fill}\nolinebreak[1]%
	\enspace\hspace*{\fill}
	\finalhyphendemerits=0
	[15th November]}

\index{Dursley , Stroud@Dursley, \emph{Stroud}}

Resignation of Alison Hayward (Lab).

\noindent
\begin{tabular*}{\columnwidth}{@{\extracolsep{\fill}} p{0.53\columnwidth} >{\itshape}l r @{\extracolsep{\fill}}}
Trevor Hall & Lab & 889\\
Loraine Patrick & C & 704\\
Yvon Dignon & Grn & 90\\
\sloppyword{Richard Blackwell-Whitehead} & LD & 79\\
\end{tabular*}

\section{Hampshire}

\council{Basingstoke and Deane}

\subsubsection*{Kempshott \hspace*{\fill}\nolinebreak[1]%
\enspace\hspace*{\fill}
\finalhyphendemerits=0
[1st March]}

\index{Kempshott , Basingstoke and Deane@Kempshott, \emph{Basingstoke \& Deane}}

Death of Rita Burgess (C).

\noindent
\begin{tabular*}{\columnwidth}{@{\extracolsep{\fill}} p{0.53\columnwidth} >{\itshape}l r @{\extracolsep{\fill}}}
Tony Capon & C & 686\\
Grant Donohoe & Lab & 366\\
Stavroulla O'Doherty & LD & 113\\
\end{tabular*}

\subsubsection*{Kempshott \hspace*{\fill}\nolinebreak[1]%
\enspace\hspace*{\fill}
\finalhyphendemerits=0
[21st June]}

\index{Kempshott , Basingstoke and Deane@Kempshott, \emph{Basingstoke \& Deane}}

Disqualification (non-attendance) of Anne Court (C).

\noindent
\begin{tabular*}{\columnwidth}{@{\extracolsep{\fill}} p{0.53\columnwidth} >{\itshape}l r @{\extracolsep{\fill}}}
Anne Court & C & 884\\
Alex Lee & Lab & 475\\
Stavroulla O'Doherty & LD & 105\\
\end{tabular*}

\subsubsection*{Norden \hspace*{\fill}\nolinebreak[1]%
	\enspace\hspace*{\fill}
	\finalhyphendemerits=0
	[25th October]}

\index{Norden , Basingstoke and Deane@Norden, \emph{Basingstoke \& Deane}}

Resignation of George Hood (Lab).

\noindent
\begin{tabular*}{\columnwidth}{@{\extracolsep{\fill}} p{0.53\columnwidth} >{\itshape}l r @{\extracolsep{\fill}}}
Carolyn Wooldridge & Lab & 925\\
Michael Archer & C & 288\\
Phil Heath & Ind & 80\\
Zoe Rogers & LD & 64\\
\end{tabular*}

\council{East Hampshire}

\subsubsection*{Petersfield Bell Hill \hspace*{\fill}\nolinebreak[1]%
\enspace\hspace*{\fill}
\finalhyphendemerits=0
[8th March; Ind gain from C]}

\index{Petersfield Bell Hill , East Hampshire@Petersfield Bell Hill, \emph{E. Hants.}}

Resignation of Thomas Spencer (C).

\noindent
\begin{tabular*}{\columnwidth}{@{\extracolsep{\fill}} p{0.53\columnwidth} >{\itshape}l r @{\extracolsep{\fill}}}
Jamie Matthews & Ind & 178\\
David Podger & LD & 156\\
Clive Shore & C & 145\\
Steve Elder & Lab & 56\\
Jim Makin & UKIP & 11\\
\end{tabular*}

\council{Eastleigh}

At the May 2018 ordinary election there was an unfilled vacancy in Netley Abbey ward due to the resignation of Lizette van Niekerk (LD).\index{Netley Abbey , Eastleigh@Netley Abbey, \emph{Eastleigh}}

\council{Gosport}

\subsubsection*{Leesland \hspace*{\fill}\nolinebreak[1]%
\enspace\hspace*{\fill}
\finalhyphendemerits=0
[3rd May]}

\index{Leesland , Gosport@Leesland, \emph{Gosport}}

Resignation of Maria Diffey (LD).

Combined with the 2018 ordinary election.
%; see page \pageref{LeeslandGosport} for the result.

\council{Havant}

\subsubsection*{Hayling West \hspace*{\fill}\nolinebreak[1]%
\enspace\hspace*{\fill}
\finalhyphendemerits=0
[3rd May]}

\index{Hayling West , Havant@Hayling W., \emph{Havant}}

Resignation of Andy Lenaghan (C).

Combined with the 2018 ordinary election.
%; see page \pageref{HaylingWestHavant} for the result.

\council{New Forest}

\subsubsection*{Milford \hspace*{\fill}\nolinebreak[1]%
\enspace\hspace*{\fill}
\finalhyphendemerits=0
[5th April]}

\index{Milford , New Forest@Milford, \emph{New Forest}}

Death of Sophie Beeton (C).

\noindent
\begin{tabular*}{\columnwidth}{@{\extracolsep{\fill}} p{0.53\columnwidth} >{\itshape}l r @{\extracolsep{\fill}}}
Christine Hopkins & C & 1057\\
Wynford Davies & LD & 200\\
Sally Spicer & Lab & 126\\
\end{tabular*}

\subsubsection*{Fawley, Blackfield and Langley \hspace*{\fill}\nolinebreak[1]%
\enspace\hspace*{\fill}
\finalhyphendemerits=0
[26th July]}

\index{Fawley, Blackfield and Langley , New Forest@Fawley, Blackfield \& Langley, \emph{New Forest}}

Death of Bob Wappet (C).

\noindent
\begin{tabular*}{\columnwidth}{@{\extracolsep{\fill}} p{0.53\columnwidth} >{\itshape}l r @{\extracolsep{\fill}}}
Merv Langdale & C & 736\\
Craig Fletcher & LD & 525\\
\end{tabular*}

\subsubsection*{Pennington \hspace*{\fill}\nolinebreak[1]%
\enspace\hspace*{\fill}
\finalhyphendemerits=0
[13th September]}

\index{Pennington , New Forest@Pennington, \emph{New Forest}}

Resignation of Penny Jackman (C).

\noindent
\begin{tabular*}{\columnwidth}{@{\extracolsep{\fill}} p{0.53\columnwidth} >{\itshape}l r @{\extracolsep{\fill}}}
Andrew Gossage & C & 497\\
Jack Davies & LD & 445\\
Ted Jearrad & Ind & 144\\
Trina Hart & Lab & 97\\
\end{tabular*}

\council{Portsmouth}

At the May 2018 ordinary election there was an unfilled vacancy in Baffins ward due to the resignation of Steve Hastings (C elected as UKIP).\index{Baffins , Portsmouth@Baffins, \emph{Portsmouth}}

\council{Rushmoor}

\subsubsection*{West Heath \hspace*{\fill}\nolinebreak[1]%
\enspace\hspace*{\fill}
\finalhyphendemerits=0
[3rd May]}

\index{West Heath , Rushmoor@West Heath, \emph{Rushmoor}}

Resignation of Mark Staplehurst (UKIP).

Combined with the 2018 ordinary election.
%; see page \pageref{WestHeathRushmoor} for the result.

\council{Winchester}

At the May 2018 ordinary election there was an unfilled vacancy in Wonston and Micheldever ward due to the resignation of James Byrnes (C).\index{Wonston and Micheldever , Winchester@Wonston \& Micheldever, \emph{Winchester}}

\subsubsection*{Upper Meon Valley \hspace*{\fill}\nolinebreak[1]%
	\enspace\hspace*{\fill}
	\finalhyphendemerits=0
	[20th September]}

\index{Upper Meon Valley , Winchester@Upper Meon Valley, \emph{Winchester}}

Resignation of Amber Tresahar (C, elected as Amber Thacker).

\noindent
\begin{tabular*}{\columnwidth}{@{\extracolsep{\fill}} p{0.53\columnwidth} >{\itshape}l r @{\extracolsep{\fill}}}
Hugh Lumby & C & 1039\\
Lewis North & LD & 905\\
June Kershaw & Lab & 39\\
Andrew Wainwright & Grn & 31\\
\end{tabular*}

\section{Hertfordshire}

\subsection*{County Council}\index{Hertfordshire}

\subsubsection*{Goffs Oak and Bury Green \hspace*{\fill}\nolinebreak[1]%
\enspace\hspace*{\fill}
\finalhyphendemerits=0
[22nd February]}

\index{Goffs Oak and Bury Green , Hertfordshire@Goffs Oak \& Bury Green, \emph{Herts.}}

Death of Robert Gordon (C).

\noindent
\begin{tabular*}{\columnwidth}{@{\extracolsep{\fill}} p{0.53\columnwidth} >{\itshape}l r @{\extracolsep{\fill}}}
Lesley Greensmyth & C & 1390\\
David Payne & LD & 482\\
Selina Norgrove & Lab & 393\\
Sally Kemp & Grn & 69\\
\end{tabular*}

\subsubsection*{St Albans North \hspace*{\fill}\nolinebreak[1]%
\enspace\hspace*{\fill}
\finalhyphendemerits=0
[3rd May; Lab gain from LD]}

\index{Saint Albans North , Hertfordshire@St Albans N., \emph{Herts.}}

Resignation of Charlotte Hogg (LD).

\noindent
\begin{tabular*}{\columnwidth}{@{\extracolsep{\fill}} p{0.53\columnwidth} >{\itshape}l r @{\extracolsep{\fill}}}
Roma Mills & Lab & 1779\\
Karen Young & LD & 1460\\
Salih Gaygusuz & C & 1361\\
Simon Grover & Grn & 258\\
\end{tabular*}

\subsubsection*{Three Rivers Rural \hspace*{\fill}\nolinebreak[1]%
	\enspace\hspace*{\fill}
	\finalhyphendemerits=0
	[25th October; LD gain from C]}

\index{Three Rivers Rural , Hertfordshire@Three Rivers Rural, \emph{Herts.}}

Resignation of Chris Hayward (C).

\noindent
\begin{tabular*}{\columnwidth}{@{\extracolsep{\fill}} p{0.53\columnwidth} >{\itshape}l r @{\extracolsep{\fill}}}
Phil Williams & LD & 1846\\
Angela Killick & C & 1315\\
Jeni Swift Gillett & Lab & 144\\
David Bennett & UKIP & 86\\
Roan Alder & Grn & 68\\
\end{tabular*}

\council{Dacorum}

\subsubsection*{Northchurch \hspace*{\fill}\nolinebreak[1]%
\enspace\hspace*{\fill}
\finalhyphendemerits=0
[8th March; LD gain from C]}

\index{Northchurch , Dacorum@Northchurch, \emph{Dacorum}}

Death of Alan Fantham (C).

\noindent
\begin{tabular*}{\columnwidth}{@{\extracolsep{\fill}} p{0.53\columnwidth} >{\itshape}l r @{\extracolsep{\fill}}}
Lara Pringle & LD & 545\\
Rob McCarthy & C & 260\\
Gareth Hawden & Lab & 97\\
Joe Pitts & Grn & 19\\
\end{tabular*}

\council{East Hertfordshire}

\subsubsection*{Watton-at-Stone \hspace*{\fill}\nolinebreak[1]%
\enspace\hspace*{\fill}
\finalhyphendemerits=0
[23rd August; LD gain from C]}

\index{Watton-at-Stone , East Hertfordshire@Watton-at-Stone, \emph{E. Herts.}}

Resignation of Michael Freeman (C).

\noindent
\begin{tabular*}{\columnwidth}{@{\extracolsep{\fill}} p{0.53\columnwidth} >{\itshape}l r @{\extracolsep{\fill}}}
Sophie Bell & LD & 531\\
Andrew Huggins & C & 238\\
Veronica Fraser & Lab & 23\\
\end{tabular*}

\council{Hertsmere}

\subsubsection*{Borehamwood Cowley Hill \hspace*{\fill}\nolinebreak[1]%
\enspace\hspace*{\fill}
\finalhyphendemerits=0
[4th January; Lab gain from C]}

\index{Borehamwood Cowley Hill , Hertsmere@Borehamwood Cowley Hill, \emph{Hertsmere}}

Disqualification (sentenced to three months' imprisonment, suspended, sexual assault) of David Burcombe (C).

\noindent
\begin{tabular*}{\columnwidth}{@{\extracolsep{\fill}} p{0.53\columnwidth} >{\itshape}l r @{\extracolsep{\fill}}}
Rebecca Butler & Lab & 709\\
Sean Moore & C & 381\\
David Hoy & UKIP & 57\\
Paul Robinson & LD & 20\\
Nicholas Winston & Grn & 18\\
\end{tabular*}

\council{North Hertfordshire}

\subsubsection*{Letchworth Grange \hspace*{\fill}\nolinebreak[1]%
\enspace\hspace*{\fill}
\finalhyphendemerits=0
[3rd May]}

\index{Letchworth Grange , North Hertfordshire@Letchworth Grange, \emph{N. Herts.}}

Resignation of Clare Billing (Lab).

Combined with the 2018 ordinary election.
%; see page \pageref{LetchworthGrangeNorthHertfordshire} for the result.

\council{Watford}

\subsubsection*{Oxhey \hspace*{\fill}\nolinebreak[1]%
\enspace\hspace*{\fill}
\finalhyphendemerits=0
[21st June]}

\index{Oxhey , Watford@Oxhey, \emph{Watford}}

Election of Peter Taylor (LD) as Mayor of Watford.

\noindent
\begin{tabular*}{\columnwidth}{@{\extracolsep{\fill}} p{0.53\columnwidth} >{\itshape}l r @{\extracolsep{\fill}}}
Imran Hamid & LD & 828\\
Joseph Gornicki & C & 420\\
Sue Sleeman & Lab & 248\\
\end{tabular*}

\council{Welwyn Hatfield}

\subsubsection*{Hatfield Villages \hspace*{\fill}\nolinebreak[1]%
\enspace\hspace*{\fill}
\finalhyphendemerits=0
[3rd May]}

\index{Hatfield Villages , Welwyn Hatfield@Hatfield Villages, \emph{Welwyn Hatfield}}

Death of Lynne Sparks (C).

Combined with the 2018 ordinary election.
%; see page \pageref{HatfieldVillagesWelwynHatfield} for the result.

\subsubsection*{Northaw and Cuffley \hspace*{\fill}\nolinebreak[1]%
\enspace\hspace*{\fill}
\finalhyphendemerits=0
[3rd May]}

\index{Northaw and Cuffley , Welwyn Hatfield@Northaw \& Cuffley, \emph{Welwyn Hatfield}}

Resignation of Irene Dean (C).

Combined with the 2018 ordinary election.
%; see page \pageref{NorthawCuffleyWelwynHatfield} for the result.

\subsubsection*{Welham Green and Hatfield South \hspace*{\fill}\nolinebreak[1]%
\enspace\hspace*{\fill}
\finalhyphendemerits=0
[3rd May]}

\index{Welham Green and Hatfield South , Welwyn Hatfield@Welham Green \& Hatfield S., \emph{Welwyn Hatfield}}

Resignation of Tom Bailey (LD).

Combined with the 2018 ordinary election.
%; see page \pageref{WelhamGreenHatfieldSouthWelwynHatfield} for the result.

\subsubsection*{Welwyn West \hspace*{\fill}\nolinebreak[1]%
	\enspace\hspace*{\fill}
	\finalhyphendemerits=0
	[29th November]}

\index{Welwyn West , Welwyn Hatfield@Welwyn W., \emph{Welwyn Hatfield}}

Death of Mandy Perkins (C).

\noindent
\begin{tabular*}{\columnwidth}{@{\extracolsep{\fill}} p{0.53\columnwidth} >{\itshape}l r @{\extracolsep{\fill}}}
Paul Smith & C & 960\\
Christina Raven & LD & 604\\
Josh Chigwangwa & Lab & 72\\
\end{tabular*}

\section{Isle of Wight}
\index{Isle of Wight}

\subsubsection*{Central Wight \hspace*{\fill}\nolinebreak[1]%
\enspace\hspace*{\fill}
\finalhyphendemerits=0
[25th January]}

\index{Central Wight , Isle of Wight@Central Wight, \emph{Isle of Wight}}

Resignation of Bob Seely MP (C).

\noindent
\begin{tabular*}{\columnwidth}{@{\extracolsep{\fill}} p{0.53\columnwidth} >{\itshape}l r @{\extracolsep{\fill}}}
Steve Hastings & C & 547\\
Nick Stuart & LD & 286\\
Daniel James & Grn & 143\\
Simon Haytack & Lab & 101\\
Terry Brennan & UKIP & 24\\
\end{tabular*}

\section{Kent}

\subsection*{County Council}\index{Kent}

\subsubsection*{Birchington and Rural \hspace*{\fill}\nolinebreak[1]%
\enspace\hspace*{\fill}
\finalhyphendemerits=0
[11th January]}

\index{Birchington and Rural , Kent@Birchington \& Rural, \emph{Kent}}

Death of Ken Gregory (C).

\noindent
\begin{tabular*}{\columnwidth}{@{\extracolsep{\fill}} p{0.53\columnwidth} >{\itshape}l r @{\extracolsep{\fill}}}
Liz Hurst & C & 2534\\
Pauline Farrance & Lab & 856\\
Angie Curwen & LD & 561\\
Zita Wiltshire & UKIP & 357\\
Natasha Ransom & Grn & 169\\
\end{tabular*}

\subsubsection*{Canterbury North \hspace*{\fill}\nolinebreak[1]%
	\enspace\hspace*{\fill}
	\finalhyphendemerits=0
	[15th November]}

\index{Canterbury North , Kent@Canterbury N., \emph{Kent}}

Death of John Simmonds (C).

\noindent
\begin{tabular*}{\columnwidth}{@{\extracolsep{\fill}} p{0.53\columnwidth} >{\itshape}l r @{\extracolsep{\fill}}}
Robert Thomas & C & 1355\\
Alex Lister & LD & 756\\
Ben Hickman & Lab & 660\\
Henry Stanton & Grn & 157\\
Joe Egerton & Ind & 155\\
Joe Simons & UKIP & 120\\
\end{tabular*}

\columnbreak

\council{Ashford}

Ashford = Ashford Independent

\subsubsection*{Kennington \hspace*{\fill}\nolinebreak[1]%
	\enspace\hspace*{\fill}
	\finalhyphendemerits=0
	[25th October]}

\index{Kennington , Ashford@Kennington, \emph{Ashford}}

Resignation of Philip Sims (C).

\noindent
\begin{tabular*}{\columnwidth}{@{\extracolsep{\fill}} p{0.53\columnwidth} >{\itshape}l r @{\extracolsep{\fill}}}
Nathan Iliffe & C & 247\\
Ian Anderson & Ashford & 227\\
Dylan Jones & Lab & 85\\
Peter Morgan & Grn & 36\\
\end{tabular*}

\council{Maidstone}

\subsubsection*{Headcorn \hspace*{\fill}\nolinebreak[1]%
\enspace\hspace*{\fill}
\finalhyphendemerits=0
[13th September]}

\index{Headcorn , Maidstone@Headcorn, \emph{Maidstone}}

Resignation of Shelina Prendergast (C).

\noindent
\begin{tabular*}{\columnwidth}{@{\extracolsep{\fill}} p{0.53\columnwidth} >{\itshape}l r @{\extracolsep{\fill}}}
Karen Chappell-Tay & C & 686\\
Merilyn Fraser & LD & 409\\
Jim Grogan & Lab & 63\\
Derek Eagle & Grn & 40\\
\end{tabular*}

\council{Medway}

\subsubsection*{Rochester West \hspace*{\fill}\nolinebreak[1]%
\enspace\hspace*{\fill}
\finalhyphendemerits=0
[8th March; Lab gain from C]}

\index{Rochester West , Medway@Rochester W., \emph{Medway}}

Resignation of Kelly Tolhurst MP (C).

\noindent
\begin{tabular*}{\columnwidth}{@{\extracolsep{\fill}} p{0.53\columnwidth} >{\itshape}l r @{\extracolsep{\fill}}}
Alex Paterson & Lab & 1212\\
Alan Kew & C & 1007\\
Martin Rose & LD & 119\\
Sonia Hymer & Grn & 107\\
Rob McCulloch Martin & UKIP & 104\\
\end{tabular*}

\council{Sevenoaks}

\subsubsection*{\sloppyword{Farningham, Horton Kirby and South Darenth} \hspace*{\fill}\nolinebreak[1]%
\enspace\hspace*{\fill}
\finalhyphendemerits=0
[30th August]}

\index{Farningham, Horton Kirby and South Darenth , Sevenoaks@Farningham, Horton Kirby \& South Darenth, \emph{Sevenoaks}}

Disqualification (non-attendance) of Ingrid Tennessee (elected as Ingrid Chetram, C).

\noindent
\begin{tabular*}{\columnwidth}{@{\extracolsep{\fill}} p{0.53\columnwidth} >{\itshape}l r @{\extracolsep{\fill}}}
Brian Carroll & C & 542\\
Krish Shanmuganathan & LD & 260\\
Emily Asher & Lab & 171\\
\end{tabular*}

\council{Swale}

\subsubsection*{Sheppey East \hspace*{\fill}\nolinebreak[1]%
\enspace\hspace*{\fill}
\finalhyphendemerits=0
[3rd May]}

\index{Sheppey East , Swale@Sheppey E., \emph{Swale}}

Resignation of Lesley Ingham (C).

\noindent
\begin{tabular*}{\columnwidth}{@{\extracolsep{\fill}} p{0.53\columnwidth} >{\itshape}l r @{\extracolsep{\fill}}}
Lynd Taylor & C & 522\\
Gill Smith & Lab & 338\\
Sunny Nissanga & UKIP & 235\\
Marc Wilson & LD & 23\\
\end{tabular*}

\council{Thanet}

\subsubsection*{Thanet Villages \hspace*{\fill}\nolinebreak[1]%
\enspace\hspace*{\fill}
\finalhyphendemerits=0
[11th January]}

\index{Thanet Villages , Thanet@Thanet Villages, \emph{Thanet}}

Death of Ken Gregory (C).

\noindent
\begin{tabular*}{\columnwidth}{@{\extracolsep{\fill}} p{0.53\columnwidth} >{\itshape}l r @{\extracolsep{\fill}}}
Reece Pugh & C & 620\\
Angie Curwen & LD & 313\\
Pauline Farrance & Lab & 206\\
Natasha Ransom & Grn & 66\\
Sonia Smyth & Ind & 52\\
\end{tabular*}

\subsubsection*{Birchington South \hspace*{\fill}\nolinebreak[1]%
\enspace\hspace*{\fill}
\finalhyphendemerits=0
[26th July; C gain from UKIP]}

\index{Birchington South , Thanet@Birchington S., \emph{Thanet}}

Death of Alan Howes (Ind elected as UKIP).

\noindent
\begin{tabular*}{\columnwidth}{@{\extracolsep{\fill}} p{0.53\columnwidth} >{\itshape}l r @{\extracolsep{\fill}}}
Linda Wright & C & 651\\
Helen Whitehead & Lab & 265\\
Hannah Lloyd-Bowyer & LD & 117\\
\end{tabular*}

\section{Lancashire}

\subsection*{County Council}\index{Lancashire}

\subsubsection*{Wyre Rural Central \hspace*{\fill}\nolinebreak[1]%
\enspace\hspace*{\fill}
\finalhyphendemerits=0
[11th January]}

\index{Wyre Rural C. , Lancashire@Wyre Rural C., \emph{Lancs.}}

Death of Vivien Taylor (C).

\noindent
\begin{tabular*}{\columnwidth}{@{\extracolsep{\fill}} p{0.53\columnwidth} >{\itshape}l r @{\extracolsep{\fill}}}
Matthew Salter & C & 1745\\
Nic Fogg & Lab & 925\\
Susan Whyte & Grn & 237\\
\end{tabular*}

\subsubsection*{Morecambe North \hspace*{\fill}\nolinebreak[1]%
\enspace\hspace*{\fill}
\finalhyphendemerits=0
[15th February]}

\index{Morecambe North , Lancashire@Morecambe N., \emph{Lancs.}}

Resignation of Tony Jones (Ind elected as C).

\noindent
\begin{tabular*}{\columnwidth}{@{\extracolsep{\fill}} p{0.53\columnwidth} >{\itshape}l r @{\extracolsep{\fill}}}
Stuart Morris & C & 1332\\
Andrew Severn & LD & 809\\
Darren Clifford & Lab & 580\\
\end{tabular*}

\council{Fylde}

\subsubsection*{Heyhouses \hspace*{\fill}\nolinebreak[1]%
\enspace\hspace*{\fill}
\finalhyphendemerits=0
[5th April]}

\index{Heyhouses , Fylde@Heyhouses, \emph{Fylde}}

Death of Barbara Nash (C).

\noindent
\begin{tabular*}{\columnwidth}{@{\extracolsep{\fill}} p{0.53\columnwidth} >{\itshape}l r @{\extracolsep{\fill}}}
Sally Nash & C & 655\\
Lynn Goodwin & Lab & 202\\
Andrew Holland & LD & 138\\
Ian Roberts & Grn & 133\\
\end{tabular*}

\subsubsection*{Ansdell \hspace*{\fill}\nolinebreak[1]%
\enspace\hspace*{\fill}
\finalhyphendemerits=0
[2nd August]}

\index{Ansdell , Fylde@Ansdell, \emph{Fylde}}

Resignation of David Eaves (C).

\noindent
\begin{tabular*}{\columnwidth}{@{\extracolsep{\fill}} p{0.53\columnwidth} >{\itshape}l r @{\extracolsep{\fill}}}
Chris Dixon & C & 715\\
Gareth Nash & Lab & 272\\
\end{tabular*}

\council{Hyndburn}

At the May 2018 ordinary election there was an unfilled vacancy in St Oswald's ward due to the death of Paul Thompson (Ind elected as UKIP).\index{Saint Oswald's , Hyndburn@St Oswald's, \emph{Hyndburn}}

\council{Lancaster}

\subsubsection*{Skerton West \hspace*{\fill}\nolinebreak[1]%
\enspace\hspace*{\fill}
\finalhyphendemerits=0
[17th May]}

\index{Skerton West , Lancaster@Skerton W., \emph{Lancaster}}

Death of Roger Sherlock (Lab).

\noindent
\begin{tabular*}{\columnwidth}{@{\extracolsep{\fill}} p{0.53\columnwidth} >{\itshape}l r @{\extracolsep{\fill}}}
Peter Rivet & Lab & 587\\
Andy Kay & C & 279\\
Derek Kaye & LD & 95\\
Cait Sinclair & Grn & 59\\
\end{tabular*}

\subsubsection*{University and Scotforth Rural (2) \hspace*{\fill}\nolinebreak[1]%
\enspace\hspace*{\fill}
\finalhyphendemerits=0
[17th May; 1 Lab gain from Grn]}

\index{University and Scotforth Rural , Lancaster@University \& Scotforth Rural, \emph{Lancaster}}

Resignations of Sam Armstrong (Lab elected as Grn) and Lucy Atkinson (Lab).

\noindent
\begin{tabular*}{\columnwidth}{@{\extracolsep{\fill}} p{0.53\columnwidth} >{\itshape}l r @{\extracolsep{\fill}}}
Amara Betts-Patel & Lab & 518\\
Oliver Robinson & Lab & 423\\
Martin Paley & Grn & 264\\
Janet Maskell & Grn & 235\\
Callum Furner & C & 184\\
Guy Watts & C & 184\\
Jade Sullivan & LD & 120\\
Iain Embrey & LD & 114\\
\end{tabular*}

\council{Preston}

\subsubsection*{Greyfriars \hspace*{\fill}\nolinebreak[1]%
\enspace\hspace*{\fill}
\finalhyphendemerits=0
[3rd May]}

\index{Greyfriars , Preston@Greyfriars, \emph{Preston}}

Resignation of Damien Moore MP (C).

Combined with the 2018 ordinary election.
%; see page \pageref{GreyfriarsPreston} for the result.

\council{West Lancashire}

OWL = Our West Lancashire

\subsubsection*{Hesketh-with-Becconsall \hspace*{\fill}\nolinebreak[1]%
\enspace\hspace*{\fill}
\finalhyphendemerits=0
[19th July]}

\index{Hesketh-with-Becconsall , West Lancashire@Hesketh-with-Becconsall, \emph{W. Lancs.}}

Resignation of Paul Moon (C).

\noindent
\begin{tabular*}{\columnwidth}{@{\extracolsep{\fill}} p{0.53\columnwidth} >{\itshape}l r @{\extracolsep{\fill}}}
Joan Witter & C & 510\\
Nick Kemp & Lab & 432\\
Steve Kirby & Ind & 295\\
\end{tabular*}

\subsubsection*{Knowsley \hspace*{\fill}\nolinebreak[1]%
\enspace\hspace*{\fill}
\finalhyphendemerits=0
[26th July]}

\index{Knowsley , West Lancashire@Knowsley, \emph{W. Lancs.}}

Resignation of Michelle Aldridge (Lab).

\noindent
\begin{tabular*}{\columnwidth}{@{\extracolsep{\fill}} p{0.53\columnwidth} >{\itshape}l r @{\extracolsep{\fill}}}
Gareth Dowling & Lab & 641\\
Kate Mitchell & OWL & 567\\
Jeffrey Vernon & C & 364\\
\end{tabular*}

\subsubsection*{Tanhouse \hspace*{\fill}\nolinebreak[1]%
\enspace\hspace*{\fill}
\finalhyphendemerits=0
[11th October]}

\index{Tanhouse , West Lancashire@Tamhouse, \emph{W. Lancs.}}

Resignation of Robert Pendleton (Lab).

\noindent
\begin{tabular*}{\columnwidth}{@{\extracolsep{\fill}} p{0.53\columnwidth} >{\itshape}l r @{\extracolsep{\fill}}}
Ron Cooper & Lab & 464\\
Aaron Body & Ind & 129\\
Alexander Blundell & C & 49\\
\end{tabular*}

\council{Wyre}

\subsubsection*{Preesall \hspace*{\fill}\nolinebreak[1]%
\enspace\hspace*{\fill}
\finalhyphendemerits=0
[11th January]}

\index{Preesall , Wyre@Preesall, \emph{Wyre}}

Death of Vivien Taylor (C).

\noindent
\begin{tabular*}{\columnwidth}{@{\extracolsep{\fill}} p{0.53\columnwidth} >{\itshape}l r @{\extracolsep{\fill}}}
Peter Cartridge & C & 930\\
Nic Fogg & Lab & 753\\
\end{tabular*}

\section{Leicestershire}

\subsection*{County Council}\index{Leicestershire}

\subsubsection*{Stoney Stanton and Croft \hspace*{\fill}\nolinebreak[1]%
\enspace\hspace*{\fill}
\finalhyphendemerits=0
[3rd May]}

\index{Stoney Stanton and Croft , Leicestershire@Stoney Stanton \& Croft, \emph{Leics.}}

Death of Ernie White (C).

\noindent
\begin{tabular*}{\columnwidth}{@{\extracolsep{\fill}} p{0.53\columnwidth} >{\itshape}l r @{\extracolsep{\fill}}}
Maggie Wright & C & 1566\\
Danny Findlay & LD & 1032\\
Christina Emmett & Lab & 444\\
Nick Cox & Grn & 153\\
\end{tabular*}

\subsubsection*{Syston Ridgeway \hspace*{\fill}\nolinebreak[1]%
\enspace\hspace*{\fill}
\finalhyphendemerits=0
[28th June]}

\index{Syston Ridgeway , Leicestershire@Syston Ridgeway, \emph{Leics.}}

Death of David Slater (C).

\noindent
\begin{tabular*}{\columnwidth}{@{\extracolsep{\fill}} p{0.53\columnwidth} >{\itshape}l r @{\extracolsep{\fill}}}
Tom Barkley & C & 810\\
Claire Poole & Lab & 251\\
Nitesh Dave & LD & 149\\
Matthew Wise & Grn & 97\\
Andy McWilliam & UKIP & 49\\
\end{tabular*}

\council{Charnwood}

\subsubsection*{Quorn and Mountsorrel Castle \hspace*{\fill}\nolinebreak[1]%
\enspace\hspace*{\fill}
\finalhyphendemerits=0
[21st June]}

\index{Quorn and Mountsorrel Castle , Charnwood@Quorn \& Mountsorrel Castle, \emph{Charnwood}}

Death of David Slater (C).

\noindent
\begin{tabular*}{\columnwidth}{@{\extracolsep{\fill}} p{0.53\columnwidth} >{\itshape}l r @{\extracolsep{\fill}}}
Jane Hunt & C & 719\\
Chris Hughes & Lab & 305\\
Marianne Gilbert & LD & 232\\
Andy McWilliam & UKIP & 139\\
\end{tabular*}

\subsubsection*{Birstall Wanlip \hspace*{\fill}\nolinebreak[1]%
\enspace\hspace*{\fill}
\finalhyphendemerits=0
[13th September]}

\index{Birstall Wanlip , Charnwood@Birstall Wanlip, \emph{Charnwood}}

Resignation of Renata Jones (C).

\noindent
\begin{tabular*}{\columnwidth}{@{\extracolsep{\fill}} p{0.53\columnwidth} >{\itshape}l r @{\extracolsep{\fill}}}
Shona Rattray & C & 492\\
Abe Khayer & Lab & 340\\
Carolyn Thornborrow & LD & 128\\
Norman Cutting & UKIP & 50\\
Charlotte Clancy & Grn & 34\\
\end{tabular*}

\subsubsection*{Anstey \hspace*{\fill}\nolinebreak[1]%
	\enspace\hspace*{\fill}
	\finalhyphendemerits=0
	[20th December; C gain from LD]}

\index{Anstey , Charnwood@Anstey, \emph{Charnwood}}

Death of John Sutherington (LD).

\noindent
\begin{tabular*}{\columnwidth}{@{\extracolsep{\fill}} p{0.53\columnwidth} >{\itshape}l r @{\extracolsep{\fill}}}
Paul Baines & C & 523\\
Glyn McAllister & Lab & 507\\
\end{tabular*}

\council{Leicester}

\subsubsection*{Belgrave \hspace*{\fill}\nolinebreak[1]%
	\enspace\hspace*{\fill}
	\finalhyphendemerits=0
	[6th December]}

\index{Belgrave , Leicester@Belgrave, \emph{Leicester}}

Death of Mansukhlal Chohan (Lab).

\noindent
\begin{tabular*}{\columnwidth}{@{\extracolsep{\fill}} p{0.53\columnwidth} >{\itshape}l r @{\extracolsep{\fill}}}
Padmini Chamund & Lab & 5477\\
Khandubhai Patel & C & 412\\
Hash Chandarana & LD & 238\\
Ursula Bilson & Grn & 199\\
\end{tabular*}

\section{Lincolnshire}

\council{Boston}

BlueRevn = Blue Revolution

\subsubsection*{Old Leake and Wrangle \hspace*{\fill}\nolinebreak[1]%
\enspace\hspace*{\fill}
\finalhyphendemerits=0
[22nd February]}

\index{Old Leake and Wrangle , Boston@Old Leake \& Wrangle, \emph{Boston}}

Resignation of Maureen Dennis (C).

\noindent
\begin{tabular*}{\columnwidth}{@{\extracolsep{\fill}} p{0.52\columnwidth} >{\itshape}l r @{\extracolsep{\fill}}}
Tom Ashton & C & 536\\
Joseph Pearson & Lab & 123\\
Don Ransome & UKIP & 50\\
Richard Thornalley & BlueRevn & 13\\
\end{tabular*}

\council{North East Lincolnshire}

\subsubsection*{Immingham \hspace*{\fill}\nolinebreak[1]%
\enspace\hspace*{\fill}
\finalhyphendemerits=0
[3rd May]}

\index{Immingham , North East Lincolnshire@Immingham, \emph{N.E. Lincs.}}

Death of Mike Burton (Lab).

Combined with the 2018 ordinary election.
%; see page \pageref{ImminghamNorthEastLincolnshire} for the result.

\subsubsection*{Freshney \hspace*{\fill}\nolinebreak[1]%
\enspace\hspace*{\fill}
\finalhyphendemerits=0
[26th July]}

\index{Freshney , North East Lincolnshire@Freshney, \emph{N.E. Lincs.}}

Resignation of Ray Sutton (Lab).

\noindent
\begin{tabular*}{\columnwidth}{@{\extracolsep{\fill}} p{0.53\columnwidth} >{\itshape}l r @{\extracolsep{\fill}}}
Sheldon Mill & Lab & 692\\
Steve Holland & C & 650\\
Mick Kiff & Ind & 231\\
Barry Fisher & UKIP & 78\\
Loyd Emmerson & Grn & 24\\
\end{tabular*}

\council{North Kesteven}

LincsInd = Lincolnshire Independents

\subsubsection*{Eagle, Swinderby and Witham St Hughs \hspace*{\fill}\nolinebreak[1]%
\enspace\hspace*{\fill}
\finalhyphendemerits=0
[22nd February; C gain from Ind]}

\index{Eagle, Swinderby and Witham Saint Hughs , North Kesteven@Eagle, Swinderby \& Witham St Hughs, \emph{N. Kesteven}}

Death of Barbara Wells (Ind).

\noindent
\begin{tabular*}{\columnwidth}{@{\extracolsep{\fill}} p{0.53\columnwidth} >{\itshape}l r @{\extracolsep{\fill}}}
Peter Rothwell & C & 563\\
Nikki Dillon & LincsInd & 347\\
Corinne Byron & LD & 64\\
\end{tabular*}

\subsubsection*{Kirkby la Thorpe and South Kyme \hspace*{\fill}\nolinebreak[1]%
\enspace\hspace*{\fill}
\finalhyphendemerits=0
[24th May; LincsInd gain from C]}

\index{Kirkby la Thorpe and South Kyme , North Kesteven@Kirkby la Thorpe \& South Kyme, \emph{N. Kesteven}}

Resignation of Julia Harrison (C).

\noindent
\begin{tabular*}{\columnwidth}{@{\extracolsep{\fill}} p{0.53\columnwidth} >{\itshape}l r @{\extracolsep{\fill}}}
Mervyn Head & LincsInd & 278\\
Dean Harlow & C & 271\\
James Thomas & Lab & 30\\
Sue Hislop & LD & 27\\
\end{tabular*}

\subsubsection*{North Hykeham Mill \hspace*{\fill}\nolinebreak[1]%
\enspace\hspace*{\fill}
\finalhyphendemerits=0
[28th June]}

\index{North Hykeham Mill , North Kesteven@North Hykeham Mill, \emph{N. Kesteven}}

Resignation of Andrea Clarke (C).

\noindent
\begin{tabular*}{\columnwidth}{@{\extracolsep{\fill}} p{0.53\columnwidth} >{\itshape}l r @{\extracolsep{\fill}}}
Stephen Roe & C & 376\\
Nikki Dillon & LincsInd & 171\\
Mark Reynolds & Lab & 167\\
Corinne Byron & LD & 43\\
\end{tabular*}

\subsubsection*{Skellingthorpe \hspace*{\fill}\nolinebreak[1]%
\enspace\hspace*{\fill}
\finalhyphendemerits=0
[28th June; LincsInd gain from Ind]}

\index{Skellingthorpe , North Kesteven@Skellingthorpe, \emph{N. Kesteven}}

Resignation of Shirley Pannell (elected as Shirley Flint, Ind).

\noindent
\begin{tabular*}{\columnwidth}{@{\extracolsep{\fill}} p{0.53\columnwidth} >{\itshape}l r @{\extracolsep{\fill}}}
Richard Johnston & LincsInd & 348\\
Nicola Clarke & C & 201\\
Matthew Newman & Lab & 129\\
Tony Richardson & LD & 83\\
\end{tabular*}

\council{South Holland}

\subsubsection*{Donington, Quadring and Gosberton \hspace*{\fill}\nolinebreak[1]%
\enspace\hspace*{\fill}
\finalhyphendemerits=0
[3rd May]}

\index{Donington, Quadring and Gosberton , South Holland@Donington, Quadring \& Gosberton, \emph{S. Holland}}

Death of Robert Clark (C).

\noindent
\begin{tabular*}{\columnwidth}{@{\extracolsep{\fill}} p{0.53\columnwidth} >{\itshape}l r @{\extracolsep{\fill}}}
Sue Wray & C & 1021\\
Terri Cornwell & Ind & 185\\
Neil Oakman & LD & 169\\
Jennie Thomas & Lab & 138\\
\end{tabular*}

\council{South Kesteven}

\subsubsection*{Stamford St George's \hspace*{\fill}\nolinebreak[1]%
\enspace\hspace*{\fill}
\finalhyphendemerits=0
[15th March]}

\index{Stamford Saint George's , South Kesteven@Stamford St George's, \emph{S. Kesteven}}

Resignation of Katherine Brown (C).

\noindent
\begin{tabular*}{\columnwidth}{@{\extracolsep{\fill}} p{0.6\columnwidth} >{\itshape}l r @{\extracolsep{\fill}}}
Rachael Cooke & C & 309\\
Gloria Johnson & Ind & 174\\
Christopher Dennett & Lab & 114\\
Jack Stow & LD & 68\\
Gerhard Lohmann-Bond & Grn & 13\\
\end{tabular*}

\subsubsection*{Stamford St John's \hspace*{\fill}\nolinebreak[1]%
\enspace\hspace*{\fill}
\finalhyphendemerits=0
[15th March]}

\index{Stamford Saint John's , South Kesteven@Stamford St John's, \emph{S. Kesteven}}

Death of Terl Bryant (C).

\noindent
\begin{tabular*}{\columnwidth}{@{\extracolsep{\fill}} p{0.53\columnwidth} >{\itshape}l r @{\extracolsep{\fill}}}
David Taylor & C & 327\\
Steve Carroll & Ind & 267\\
Harrish Bisnauthsing & LD & 156\\
Cameron Clack & Lab & 66\\
Simon Whitmore & Grn & 15\\
\end{tabular*}

\section{Norfolk}

\subsection*{County Council}\index{Norfolk}

\subsubsection*{Yare and All Saints \hspace*{\fill}\nolinebreak[1]%
\enspace\hspace*{\fill}
\finalhyphendemerits=0
[12th July]}

\index{Yare and All Saints , Norfolk@Yare \& All SS., \emph{Norfolk}}

Resignation of Cliff Jordan (C).

\noindent
\begin{tabular*}{\columnwidth}{@{\extracolsep{\fill}} p{0.53\columnwidth} >{\itshape}l r @{\extracolsep{\fill}}}
Edward Connolly & C & 955\\
Harry Clarke & Lab & 337\\
Andrew Thorpe & LD & 182\\
\end{tabular*}

\council{Broadland}

\subsubsection*{Aylsham \hspace*{\fill}\nolinebreak[1]%
\enspace\hspace*{\fill}
\finalhyphendemerits=0
[24th May; LD gain from C]}

\index{Aylsham , Broadland@Aylsham, \emph{Broadland}}

Resignation of Ian Graham (C).

\noindent
\begin{tabular*}{\columnwidth}{@{\extracolsep{\fill}} p{0.53\columnwidth} >{\itshape}l r @{\extracolsep{\fill}}}
Sue Catchpole & LD & 1018\\
Hal Turkmen & C & 865\\
Peter Harwood & Lab & 328\\
\end{tabular*}

\council{Great Yarmouth}

\subsubsection*{Central and Northgate \hspace*{\fill}\nolinebreak[1]%
\enspace\hspace*{\fill}
\finalhyphendemerits=0
[3rd May]}

\index{Central and Northgate , Great Yarmouth@Central \& Northgate, \emph{Great Yarmouth}}

Resignation of Leanne Davis (Lab).

Combined with the 2018 ordinary election.
%; see page \pageref{CentralNorthgateGreatYarmouth} for the result.

\council{King's Lynn and West Norfolk}

\subsubsection*{Snettisham \hspace*{\fill}\nolinebreak[1]%
\enspace\hspace*{\fill}
\finalhyphendemerits=0
[2nd August]}

\index{Snettisham , King's Lynn and West Norfolk@Snettisham, \emph{King's Lynn \& W. Norfolk}}

Resignation of Avril Wright (C).

\noindent
\begin{tabular*}{\columnwidth}{@{\extracolsep{\fill}} p{0.53\columnwidth} >{\itshape}l r @{\extracolsep{\fill}}}
Stuart Dark & C & 613\\
Erika Coward & LD & 66\\
Nigel Walker & Grn & 65\\
Matthew Hannay & UKIP & 48\\
\end{tabular*}

\council{North Norfolk}

\subsubsection*{Worstead \hspace*{\fill}\nolinebreak[1]%
\enspace\hspace*{\fill}
\finalhyphendemerits=0
[15th February; LD gain from C]}

\index{Worstead , North Norfolk@Worstead, \emph{N. Norfolk}}

Resignation of Glyn Williams (C).

\noindent
\begin{tabular*}{\columnwidth}{@{\extracolsep{\fill}} p{0.53\columnwidth} >{\itshape}l r @{\extracolsep{\fill}}}
Saul Penfold & LD & 509\\
Robin Russell-Pavier & C & 118\\
David Spencer & Lab & 73\\
\end{tabular*}

\section{North Yorkshire}

\subsection*{County Council}\index{North Yorkshire}

\subsubsection*{Knaresborough \hspace*{\fill}\nolinebreak[1]%
\enspace\hspace*{\fill}
\finalhyphendemerits=0
[16th August; LD gain from C]}

\index{Knaresborough , North Yorkshire@Knaresborough, \emph{N. Yorks.}}

Resignation of Nicola Wilson (C).

\noindent
\begin{tabular*}{\columnwidth}{@{\extracolsep{\fill}} p{0.53\columnwidth} >{\itshape}l r @{\extracolsep{\fill}}}
David Goode & LD & 2051\\
Phil Ireland & C & 1313\\
Sharon-Theresa Calvert & Lab & 369\\
\end{tabular*}

\council{Hambleton}

Yorks = Yorkshire Party

\subsubsection*{Thirsk \hspace*{\fill}\nolinebreak[1]%
	\enspace\hspace*{\fill}
	\finalhyphendemerits=0
	[4th October]}

\index{Thirsk , Hambleton@Thirsk, \emph{Hambleton}}

Resignation of Janet Watson (C).

\noindent
\begin{tabular*}{\columnwidth}{@{\extracolsep{\fill}} p{0.53\columnwidth} >{\itshape}l r @{\extracolsep{\fill}}}
Dave Elders & C & 679\\
Trish Beadle & Lab & 251\\
Stewart Barber & Yorks & 108\\
\end{tabular*}

\council{Middlesbrough}

\subsubsection*{Brambles and Thorntree \hspace*{\fill}\nolinebreak[1]%
	\enspace\hspace*{\fill}
	\finalhyphendemerits=0
	[13th December]}

\index{Brambles and Thorntree , Middlesbrough@Brambles \& Thorntree, \emph{Middlesbrough}}

Death of Peter Purvis (Lab).

\noindent
\begin{tabular*}{\columnwidth}{@{\extracolsep{\fill}} p{0.53\columnwidth} >{\itshape}l r @{\extracolsep{\fill}}}
Janet Thompson & Lab & 321\\
Graham Wilson & Ind & 158\\
David Smith & C & 44\\
Paul Hamilton & LD & 27\\
\end{tabular*}

\council{Redcar and Cleveland}

\subsubsection*{Longbeck \hspace*{\fill}\nolinebreak[1]%
\enspace\hspace*{\fill}
\finalhyphendemerits=0
[15th March; C gain from Ind]}

\index{Longbeck , Redcar and Cleveland@Longbeck, \emph{Redcar \& Cleveland}}

Resignation of Mike Findley (Ind).

\noindent
\begin{tabular*}{\columnwidth}{@{\extracolsep{\fill}} p{0.53\columnwidth} >{\itshape}l r @{\extracolsep{\fill}}}
Vera Rider & C & 494\\
Marilyn Marshall & LD & 397\\
Darcie Shepherd & Lab & 337\\
Vic Jeffries & Ind & 282\\
\end{tabular*}

\council{York}

\subsubsection*{Holgate \hspace*{\fill}\nolinebreak[1]%
\enspace\hspace*{\fill}
\finalhyphendemerits=0
[15th February]}

\index{Holgate , York@Holgate, \emph{York}}

Resignation of Sonja Crisp (Lab).

\noindent
\begin{tabular*}{\columnwidth}{@{\extracolsep{\fill}} p{0.53\columnwidth} >{\itshape}l r @{\extracolsep{\fill}}}
Kallum Taylor & Lab & 1521\\
Emma Keef & LD & 982\\
Joe Pattinson & C & 334\\
Andreas Heinemeyer & Grn & 203\\
\end{tabular*}

\section{Northamptonshire}

\subsection*{County Council}\index{Northamptonshire}

\subsubsection*{Higham Ferrers \hspace*{\fill}\nolinebreak[1]%
\enspace\hspace*{\fill}
\finalhyphendemerits=0
[15th February]}

\index{Higham Ferrers , Northamptonshire@Higham Ferrers, \emph{Northants.}}

Death of Glenn Harwood (C).

\noindent
\begin{tabular*}{\columnwidth}{@{\extracolsep{\fill}} p{0.53\columnwidth} >{\itshape}l r @{\extracolsep{\fill}}}
Jason Smithers & C & 1414\\
Gary Day & Lab & 557\\
Suzanna Austin & LD & 336\\
Bill Cross & UKIP & 109\\
Simon Turner & Grn & 81\\
\end{tabular*}

\subsubsection*{St George \hspace*{\fill}\nolinebreak[1]%
\enspace\hspace*{\fill}
\finalhyphendemerits=0
[19th July]}

\index{Saint George , Northamptonshire@St George, \emph{Northants.}}

Resignation of Rachel Cooley (Lab).

\noindent
\begin{tabular*}{\columnwidth}{@{\extracolsep{\fill}} p{0.53\columnwidth} >{\itshape}l r @{\extracolsep{\fill}}}
Anjona Roy & Lab & 839\\
Martin Sawyer & LD & 564\\
Ausra Uzukauskaite & C & 285\\
Andy Smiles & UKIP & 111\\
Scott Mabbutt & Grn & 83\\
\end{tabular*}

\council{Daventry}

\subsubsection*{Walgrave \hspace*{\fill}\nolinebreak[1]%
\enspace\hspace*{\fill}
\finalhyphendemerits=0
[3rd May]}

\index{Walgrave , Daventry@Walgrave, \emph{Daventry}}

Resignation of Ann Carter (C).

\noindent
\begin{tabular*}{\columnwidth}{@{\extracolsep{\fill}} p{0.53\columnwidth} >{\itshape}l r @{\extracolsep{\fill}}}
Lesley Woolnough & C & 431\\
Grant Bowles & LD & 195\\
\end{tabular*}

\council{East Northamptonshire}

\subsubsection*{Higham Ferrers Lancaster \hspace*{\fill}\nolinebreak[1]%
\enspace\hspace*{\fill}
\finalhyphendemerits=0
[15th February]}

\index{Higham Ferrers Lancaster , East Northamptonshire@Higham Ferrers Lancaster, \emph{E. Northants.}}

Death of Glenn Harwood (C).

\noindent
\begin{tabular*}{\columnwidth}{@{\extracolsep{\fill}} p{0.53\columnwidth} >{\itshape}l r @{\extracolsep{\fill}}}
Harriet Pentland & C & 611\\
Suzanna Austin & LD & 244\\
Mark Smith & Lab & 189\\
Simon Turner & Grn & 33\\
Bill Cross & UKIP & 22\\
\end{tabular*}

\council{Northampton}

\subsubsection*{Delapre and Briar Hill \hspace*{\fill}\nolinebreak[1]%
	\enspace\hspace*{\fill}
	\finalhyphendemerits=0
	[29th November]}

\index{Delapre and Briar Hill , Northampton@Delapre \& Briar Hill, \emph{Northampton}}

Resignation of Victoria Culbard (Lab).

\noindent
\begin{tabular*}{\columnwidth}{@{\extracolsep{\fill}} p{0.53\columnwidth} >{\itshape}l r @{\extracolsep{\fill}}}
Emma Roberts & Lab & 914\\
Daniel Soan & C & 549\\
Nicola McKenna & Ind & 417\\
Michael Maher & LD & 133\\
Denise Donaldson & Grn & 95\\
\end{tabular*}

\council{South Northamptonshire}

\subsubsection*{Middleton Cheney \hspace*{\fill}\nolinebreak[1]%
\enspace\hspace*{\fill}
\finalhyphendemerits=0
[12th April]}

\index{Middleton Cheney , South Northamptonshire@Middleton Cheney, \emph{S. Northants.}}

Resignation of Judith Baxter (C).

\noindent
\begin{tabular*}{\columnwidth}{@{\extracolsep{\fill}} p{0.6\columnwidth} >{\itshape}l r @{\extracolsep{\fill}}}
Jonathan Riley & C & 391\\
Mark Allen & LD & 316\\
Richard Solesbury-Timms & Lab & 183\\
Adam Sear & Grn & 38\\
\end{tabular*}

\subsubsection*{Astwell \hspace*{\fill}\nolinebreak[1]%
\enspace\hspace*{\fill}
\finalhyphendemerits=0
[21st June]}

\index{Astwell , South Northamptonshire@Astwell, \emph{S. Northants.}}

Resignation of Simon Marinker (C).

\noindent
\begin{tabular*}{\columnwidth}{@{\extracolsep{\fill}} p{0.6\columnwidth} >{\itshape}l r @{\extracolsep{\fill}}}
Paul Wiltshire & C & 319\\
Richard Solesbury-Timms & Lab & 96\\
\end{tabular*}

\subsubsection*{Whittlewood \hspace*{\fill}\nolinebreak[1]%
\enspace\hspace*{\fill}
\finalhyphendemerits=0
[21st June; LD gain from C]}

\index{Whittlewood, South Northamptonshire@Whittlewood, \emph{S. Northants.}}

Disqualification (non-attendance) of Lizzy Bowen (C).

\noindent
\begin{tabular*}{\columnwidth}{@{\extracolsep{\fill}} p{0.53\columnwidth} >{\itshape}l r @{\extracolsep{\fill}}}
Abigail Medina & LD & 366\\
William Barter & C & 236\\
Adrian Scandrett & Lab & 44\\
\end{tabular*}

\section{Nottinghamshire}

\council{Ashfield}

Ashfield = Ashfield Independent

DVP = Democrats and Veterans Party

\subsubsection*{Sutton Junction and Harlow Wood \hspace*{\fill}\nolinebreak[1]%
	\enspace\hspace*{\fill}
	\finalhyphendemerits=0
	[Wednesday 12th December; Ashfield gain from Lab]}

\index{Sutton Junction and Harlow Wood , Ashfield@Sutton Junction \& Harlow Wood, \emph{Ashfield}}

\sloppyword{Resignation of Steven Carroll (Ash{fi}eld elected as Lab).}

\noindent
\begin{tabular*}{\columnwidth}{@{\extracolsep{\fill}} p{0.53\columnwidth} >{\itshape}l r @{\extracolsep{\fill}}}
Matthew Relf & Ashfield & 856\\
Kevin Ball & Lab & 97\\
Christine Self & C & 48\\
Stephen Crosby & DVP & 26\\
Moira Sansom & UKIP & 13\\
Martin Howes & LD & 5\\
\end{tabular*}

\council{Bassetlaw}

\subsubsection*{Worksop South-East \hspace*{\fill}\nolinebreak[1]%
\enspace\hspace*{\fill}
\finalhyphendemerits=0
[22nd March]}

\index{Worksop South-East , Bassetlaw@Worksop S.E., \emph{Bassetlaw}}

Resignation of Deirdre Foley (Lab).

\noindent
\begin{tabular*}{\columnwidth}{@{\extracolsep{\fill}} p{0.53\columnwidth} >{\itshape}l r @{\extracolsep{\fill}}}
Clayton Tindle & Lab & 1004\\
Lewis Stanniland & C & 197\\
Leon Duveen & LD & 98\\
\end{tabular*}

\subsubsection*{East Retford West \hspace*{\fill}\nolinebreak[1]%
	\enspace\hspace*{\fill}
	\finalhyphendemerits=0
	[15th November]}

\index{East Retford West , Bassetlaw@East Retford W., \emph{Bassetlaw}}

Disqualification (non-attendance) of Alan Chambers (Ind elected as Lab).

\noindent
\begin{tabular*}{\columnwidth}{@{\extracolsep{\fill}} p{0.53\columnwidth} >{\itshape}l r @{\extracolsep{\fill}}}
Matthew Callingham & Lab & 441\\
Emma Auckland & C & 296\\
Helen Tamblyn-Saville & LD & 146\\
\end{tabular*}

\council{Nottingham}

Elvis = Bus-Pass Elvis Party

NottmInd = Nottingham Independents Putting Clifton First

\subsubsection*{Wollaton West \hspace*{\fill}\nolinebreak[1]%
\enspace\hspace*{\fill}
\finalhyphendemerits=0
[8th March; Lab gain from C]}

\index{Wollaton West , Nottingham@Wollaton W., \emph{Nottingham}}

Death of Georgina Culley (C).

\noindent
\begin{tabular*}{\columnwidth}{@{\extracolsep{\fill}} p{0.53\columnwidth} >{\itshape}l r @{\extracolsep{\fill}}}
Cate Woodward & Lab & 2193\\
Paul Brittain & C & 1950\\
Tony Sutton & LD & 237\\
Adam McGregor & Grn & 72\\
David Bishop & Elvis & 41\\
\end{tabular*}

\subsubsection*{Clifton North \hspace*{\fill}\nolinebreak[1]%
\enspace\hspace*{\fill}
\finalhyphendemerits=0
[27th September; C gain from Lab]}

\index{Clifton North , Nottingham@Clifton N., \emph{Nottingham}}

Resignation of Pat Ferguson (Ind elected as Lab).

\noindent
\begin{tabular*}{\columnwidth}{@{\extracolsep{\fill}} p{0.475\columnwidth} >{\itshape}l r @{\extracolsep{\fill}}}
Roger Steel & C & 1311\\
Shuguftah Quddoos & Lab & 928\\
Kevin Clarke & NottmInd & 307\\
Rebecca Procter & LD & 92\\
Kirsty Jones & Grn & 64\\
David Bishop & Elvis & 46\\
\end{tabular*}

\council{Rushcliffe}

\subsubsection*{Gotham \hspace*{\fill}\nolinebreak[1]%
\enspace\hspace*{\fill}
\finalhyphendemerits=0
[23rd August]}

\index{Gotham , Rushcliffe@Gotham, \emph{Rushcliffe}}

Resignation of Stuart Matthews (C).

\noindent
\begin{tabular*}{\columnwidth}{@{\extracolsep{\fill}} p{0.53\columnwidth} >{\itshape}l r @{\extracolsep{\fill}}}
Rex Walker & C & 355\\
Lewis McAulay & Lab & 275\\
Stuart Matthews & Ind & 160\\
Jason Billin & LD & 63\\
Neil Pinder & Grn & 25\\
\end{tabular*}

\section{Oxfordshire}

\subsection*{County Council}\index{Oxfordshire}

\subsubsection*{Iffley Fields and St Mary's \hspace*{\fill}\nolinebreak[1]%
\enspace\hspace*{\fill}
\finalhyphendemerits=0
[18th October]}

\index{Iffley Fields and Saint Mary's , Oxfordshire@Iffley Fields \& St Mary's, \emph{Oxon.}}

Resignation of Helen Evans (Lab).

\noindent
\begin{tabular*}{\columnwidth}{@{\extracolsep{\fill}} p{0.53\columnwidth} >{\itshape}l r @{\extracolsep{\fill}}}
Damian Haywood & Lab & 1162\\
Arthur Williams & Grn & 1087\\
Paul Sims & C & 100\\
Josie Procter & LD & 43\\
\end{tabular*}

\subsubsection*{Grove and Wantage \hspace*{\fill}\nolinebreak[1]%
	\enspace\hspace*{\fill}
	\finalhyphendemerits=0
	[15th November]}

\index{Grove and Wantage , Oxfordshire@Grove \& Wantage, \emph{Oxon.}}

Resignation of Zoé Patrick (LD).

\noindent
\begin{tabular*}{\columnwidth}{@{\extracolsep{\fill}} p{0.53\columnwidth} >{\itshape}l r @{\extracolsep{\fill}}}
Jane Hanna & LD & 1925\\
Ben Mabbett & C & 1447\\
Dave Gernon & Lab & 459\\
Kevin Harris & Grn & 185\\
\end{tabular*}

\subsubsection*{Wheatley \hspace*{\fill}\nolinebreak[1]%
	\enspace\hspace*{\fill}
	\finalhyphendemerits=0
	[29th November]}

\index{Wheatley , Oxfordshire@Wheatley, \emph{Oxon.}}

Resignation of Kirsten Johnson (LD).

\noindent
\begin{tabular*}{\columnwidth}{@{\extracolsep{\fill}} p{0.53\columnwidth} >{\itshape}l r @{\extracolsep{\fill}}}
Tim Bearder & LD & 1380\\
John Walsh & C & 705\\
Michael Nixon & Lab & 178\\
\end{tabular*}

\council{Cherwell}

\subsubsection*{Cropredy, Sibfords and Wroxton \hspace*{\fill}\nolinebreak[1]%
\enspace\hspace*{\fill}
\finalhyphendemerits=0
[3rd May]}

\index{Cropredy, Sibfords and Wroxton , Cherwell@Cropredy, Sibfords \& Wroxton, \emph{Cherwell}}

Resignation of Ken Atack (C).

Combined with the 2018 ordinary election.
%; see page \pageref{CropredySibfordsWroxtonCherwell} for the result.

\subsubsection*{Bicester West \hspace*{\fill}\nolinebreak[1]%
\enspace\hspace*{\fill}
\finalhyphendemerits=0
[21st June; Ind gain from C]}

\index{Bicester West , Cherwell@Bicester W., \emph{Cherwell}}

Ordinary election postponed from 3rd May: death of outgoing councillor Jolanta Lis (C) who was seeking re-election.

\noindent
\begin{tabular*}{\columnwidth}{@{\extracolsep{\fill}} p{0.53\columnwidth} >{\itshape}l r @{\extracolsep{\fill}}}
John Broad & Ind & 877\\
David Lydiat & C & 716\\
Stuart Moss & Lab & 439\\
Robert Nixon & Grn & 72\\
Mark Chivers & LD & 64\\
\end{tabular*}

\council{Oxford}

At the May 2018 ordinary election there was an unfilled vacancy in Northfield Brook ward due to the death of Jennifer Pegg (Lab).\index{Northfield Brook , Oxford@Northfield Brook, \emph{Oxford}}

\subsubsection*{Headington \hspace*{\fill}\nolinebreak[1]%
\enspace\hspace*{\fill}
\finalhyphendemerits=0
[19th July]}

\index{Headington , Oxford@Headington, \emph{Oxford}}

Resignation of Ruth Wilkinson (LD).

\noindent
\begin{tabular*}{\columnwidth}{@{\extracolsep{\fill}} p{0.53\columnwidth} >{\itshape}l r @{\extracolsep{\fill}}}
Stef Garden & LD & 949\\
Simon Ottino & Lab & 419\\
Georgina Gibbs & C & 124\\
Ray Hitchins & Grn & 67\\
\end{tabular*}

\subsubsection*{Wolvercote \hspace*{\fill}\nolinebreak[1]%
	\enspace\hspace*{\fill}
	\finalhyphendemerits=0
	[6th December]}

\index{Wolvercote , Oxford@Wolvercote, \emph{Oxford}}

Death of Angie Goff (LD).

\noindent
\begin{tabular*}{\columnwidth}{@{\extracolsep{\fill}} p{0.53\columnwidth} >{\itshape}l r @{\extracolsep{\fill}}}
Liz Wade & LD & 998\\
Jenny Jackson & C & 404\\
Ibrahim el-Hendi & Lab & 162\\
Sarah Edwards & Grn & 86\\
\end{tabular*}

\council{South Oxfordshire}

\subsubsection*{Benson and Crowmarsh \hspace*{\fill}\nolinebreak[1]%
\enspace\hspace*{\fill}
\finalhyphendemerits=0
[7th June; LD gain from C]}

\index{Benson and Crowmarsh , South Oxfordshire@Benson \& Crowmarsh, \emph{S. Oxon.}}

Resignation of Richard Pullen (C).

\noindent
\begin{tabular*}{\columnwidth}{@{\extracolsep{\fill}} p{0.53\columnwidth} >{\itshape}l r @{\extracolsep{\fill}}}
Sue Cooper & LD & 1048\\
Domenic Papa & C & 658\\
William Sorenson & Lab & 121\\
\end{tabular*}

\council{West Oxfordshire}

\subsubsection*{Carterton South \hspace*{\fill}\nolinebreak[1]%
\enspace\hspace*{\fill}
\finalhyphendemerits=0
[15th February]}

\index{Carterton South , West Oxfordshire@Carterton S., \emph{W. Oxon.}}

Resignation of Mick Brennan (C).

\noindent
\begin{tabular*}{\columnwidth}{@{\extracolsep{\fill}} p{0.53\columnwidth} >{\itshape}l r @{\extracolsep{\fill}}}
Michele Mead & C & 388\\
Ben Lines & LD & 146\\
Simon Adderley & Lab & 83\\
\end{tabular*}

\subsubsection*{Freeland and Hanborough \hspace*{\fill}\nolinebreak[1]%
\enspace\hspace*{\fill}
\finalhyphendemerits=0
[3rd May]}

\index{Freeland and Hanborough , West Oxfordshire@Freeland \& Hanborough, \emph{W. Oxon.}}

Resignation of Carol Reynolds (C).

Combined with the 2018 ordinary election.
%; see page \pageref{FreelandHanboroughWestOxfordshire} for the result.

\section{Rutland}\index{Rutland}

\subsubsection*{Oakham South East \hspace*{\fill}\nolinebreak[1]%
\enspace\hspace*{\fill}
\finalhyphendemerits=0
[8th March; Ind gain from C]}

\index{Oakham South East , Rutland@Oakham S.E., \emph{Rutland}}

Resignation of Tony Mathias (C).

\noindent
\begin{tabular*}{\columnwidth}{@{\extracolsep{\fill}} p{0.53\columnwidth} >{\itshape}l r @{\extracolsep{\fill}}}
Adam Lowe & Ind & 300\\
Christopher Clark & C & 204\\
\end{tabular*}

\subsubsection*{Oakham South West \hspace*{\fill}\nolinebreak[1]%
\enspace\hspace*{\fill}
\finalhyphendemerits=0
[12th July; Ind gain from C, result decided on drawing of lots]}

\index{Oakham South West , Rutland@Oakham S.W., \emph{Rutland}}

Resignation of Richard Clifton (C).

\noindent
\begin{tabular*}{\columnwidth}{@{\extracolsep{\fill}} p{0.53\columnwidth} >{\itshape}l r @{\extracolsep{\fill}}}
Richard Alderman & Ind & 178\\
Joanna Burrows & LD & 177\\
Patsy Clifton & C & 163\\
Chris Brookes & Lab & 80\\
\end{tabular*}

\section{Shropshire}

\council{Shropshire}

\subsubsection*{Shifnal South and Cosford \hspace*{\fill}\nolinebreak[1]%
\enspace\hspace*{\fill}
\finalhyphendemerits=0
[5th July]}

\index{Shifnal South and Cosford , Shropshire@Shifnal S. \& Cosford, \emph{Shrops.}}

Resignation of Stuart West (C).

\noindent
\begin{tabular*}{\columnwidth}{@{\extracolsep{\fill}} p{0.53\columnwidth} >{\itshape}l r @{\extracolsep{\fill}}}
Edward Bird & C & 362\\
Andy Mitchell & Ind & 210\\
David Carey & Ind & 207\\
Jolyon Hartin & LD & 167\\
\end{tabular*}

\section{Somerset}

\council{Bath and North East Somerset}

\subsubsection*{Kingsmead \hspace*{\fill}\nolinebreak[1]%
\enspace\hspace*{\fill}
\finalhyphendemerits=0
[5th July; LD gain from C]}

\index{Kingsmead , Bath and North East Somerset@Kingsmead, \emph{Bath \& N.E. Somerset}}

Death of Chris Pearce (C).

\noindent
\begin{tabular*}{\columnwidth}{@{\extracolsep{\fill}} p{0.53\columnwidth} >{\itshape}l r @{\extracolsep{\fill}}}
Sue Craig & LD & 545\\
Sharon Gillings & Lab & 326\\
Tom Hobson & C & 282\\
Eric Lucas & Grn & 172\\
\end{tabular*}

\council{Mendip}

\subsubsection*{Wells St Thomas' \hspace*{\fill}\nolinebreak[1]%
\enspace\hspace*{\fill}
\finalhyphendemerits=0
[25th October]}

\index{Wells Saint Thomas' , Mendip@Wells St Thomas', \emph{Mendip}}

Death of Danny Unwin (C elected as LD).

\noindent
\begin{tabular*}{\columnwidth}{@{\extracolsep{\fill}} p{0.53\columnwidth} >{\itshape}l r @{\extracolsep{\fill}}}
Thomas Ronan & LD & 594\\
Richard Greenwell & C & 493\\
Den Carter & Lab & 131\\
\end{tabular*}

\council{Taunton Deane}

\subsubsection*{Wiveliscombe and West Deane \hspace*{\fill}\nolinebreak[1]%
\enspace\hspace*{\fill}
\finalhyphendemerits=0
[5th April; Grn gain from Ind]}

\index{Wiveliscombe and West Deane , Taunton Deane@Wiveliscombe \& West Deane, \emph{Taunton Deane}}

Resignation of Steve Ross (Ind).

\noindent
\begin{tabular*}{\columnwidth}{@{\extracolsep{\fill}} p{0.53\columnwidth} >{\itshape}l r @{\extracolsep{\fill}}}
Dave Mansell & Grn & 600\\
Susan Levinge & LD & 389\\
Phillip Thorne & C & 352\\
\end{tabular*}

\council{West Somerset}

\subsubsection*{Minehead South \hspace*{\fill}\nolinebreak[1]%
\enspace\hspace*{\fill}
\finalhyphendemerits=0
[22nd February; LD gain from Ind]}

\index{Minehead South , West Somerset@Minehead S., \emph{W. Somerset}}

Resignation of Tom Hall (UKIP elected as Ind).

\noindent
\begin{tabular*}{\columnwidth}{@{\extracolsep{\fill}} p{0.53\columnwidth} >{\itshape}l r @{\extracolsep{\fill}}}
Benet Allen & LD & 318\\
Gary Miele & C & 291\\
Maureen Smith & Lab & 125\\
\end{tabular*}

\subsubsection*{Alcombe \hspace*{\fill}\nolinebreak[1]%
\enspace\hspace*{\fill}
\finalhyphendemerits=0
[21st June; LD gain from UKIP]}

\index{Alcombe , West Somerset@Alcombe, \emph{W. Somerset}}

Resignation of Adrian Behan (UKIP).

\noindent
\begin{tabular*}{\columnwidth}{@{\extracolsep{\fill}} p{0.53\columnwidth} >{\itshape}l r @{\extracolsep{\fill}}}
Nicole Hawkins & LD & 256\\
Andy Parbrook & C & 177\\
Stephanie Stephens & Ind & 130\\
Andrew Mountford & Lab & 90\\
\end{tabular*}

\section{Staffordshire}

\subsection*{County Council}\index{Staffordshire}

\subsubsection*{Codsall \hspace*{\fill}\nolinebreak[1]%
\enspace\hspace*{\fill}
\finalhyphendemerits=0
[8th February]}

\index{Codsall , Staffordshire@Codsall, \emph{Staffs.}}

Death of Robert Marshall (C).

\noindent
\begin{tabular*}{\columnwidth}{@{\extracolsep{\fill}} p{0.53\columnwidth} >{\itshape}l r @{\extracolsep{\fill}}}
Bob Spencer & C & 1274\\
Gary Burnett & Grn & 329\\
Kevin McElduff & Lab & 283\\
\end{tabular*}

\council{East Staffordshire}

Stretton = Independent --- Save Our Stretton

\subsubsection*{Stretton \hspace*{\fill}\nolinebreak[1]%
\enspace\hspace*{\fill}
\finalhyphendemerits=0
[8th February]}

\index{Stretton , East Staffordshire@Stretton, \emph{E. Staffs.}}

Resignation of Dale Spedding (C).

\noindent
\begin{tabular*}{\columnwidth}{@{\extracolsep{\fill}} p{0.53\columnwidth} >{\itshape}l r @{\extracolsep{\fill}}}
Vicki Gould & C & 764\\
Graham Lamb & Stretton & 625\\
Elaine Pritchard & Lab & 347\\
Peter Levis & UKIP & 47\\
Rhys Buchan & LD & 14\\
\end{tabular*}

\subsubsection*{Crown \hspace*{\fill}\nolinebreak[1]%
\enspace\hspace*{\fill}
\finalhyphendemerits=0
[7th June]}

\index{Crown , East Staffordshire@Crown, \emph{E. Staffs.}}

Resignation of Stephen Smith (C).

\noindent
\begin{tabular*}{\columnwidth}{@{\extracolsep{\fill}} p{0.53\columnwidth} >{\itshape}l r @{\extracolsep{\fill}}}
Gordon Marjoram & C & 459\\
William Walker & Lab & 73\\
Michael Pettingale & LD & 71\\
\end{tabular*}

\council{Lichfield}

SthgNew = Something New

\subsubsection*{Stowe \hspace*{\fill}\nolinebreak[1]%
\enspace\hspace*{\fill}
\finalhyphendemerits=0
[22nd February]}

\index{Stowe , Lichfield@Stowe, \emph{Lichfield}}

Resignation of David Smedley (C).

\noindent
\begin{tabular*}{\columnwidth}{@{\extracolsep{\fill}} p{0.53\columnwidth} >{\itshape}l r @{\extracolsep{\fill}}}
Joanne Grange & C & 513\\
Don Palmer & Lab & 299\\
Jeyan Anketell & LD & 217\\
Philip Peter & SthgNew & 59\\
Mat Hayward & Grn & 56\\
\end{tabular*}

\subsubsection*{Curborough \hspace*{\fill}\nolinebreak[1]%
\enspace\hspace*{\fill}
\finalhyphendemerits=0
[5th July; Lab gain from C]}

\index{Curborough , Lichfield@Curborough, \emph{Lichfield}}

Death of Jeanette Allsopp (C).

\noindent
\begin{tabular*}{\columnwidth}{@{\extracolsep{\fill}} p{0.53\columnwidth} >{\itshape}l r @{\extracolsep{\fill}}}
Colin Ball & Lab & 309\\
Jayne Marks & C & 169\\
Lee Cadwallader-Allen & LD & 34\\
\end{tabular*}

\subsubsection*{Stowe \hspace*{\fill}\nolinebreak[1]%
\enspace\hspace*{\fill}
\finalhyphendemerits=0
[27th September]}

\index{Stowe , Lichfield@Stowe, \emph{Lichfield}}

Resignation of Joanne Grange (C).

\noindent
\begin{tabular*}{\columnwidth}{@{\extracolsep{\fill}} p{0.53\columnwidth} >{\itshape}l r @{\extracolsep{\fill}}}
Angela Lax & C & 499\\
Donald Palmer & Lab & 440\\
Richard Rathbone & LD & 193\\
\end{tabular*}

\council{South Staffordshire}

\subsubsection*{Codsall South \hspace*{\fill}\nolinebreak[1]%
\enspace\hspace*{\fill}
\finalhyphendemerits=0
[8th February]}

\index{Codsall South , South Staffordshire@Codsall S., \emph{S. Staffs.}}

Death of Robert Marshall (C).

\noindent
\begin{tabular*}{\columnwidth}{@{\extracolsep{\fill}} p{0.53\columnwidth} >{\itshape}l r @{\extracolsep{\fill}}}
Bob Spencer & C & 490\\
Kevin McElduff & Lab & 82\\
Ian Sadler & Grn & 50\\
\end{tabular*}

\council{Staffordshire Moorlands}

\subsubsection*{Leek West \hspace*{\fill}\nolinebreak[1]%
\enspace\hspace*{\fill}
\finalhyphendemerits=0
[22nd March; Lab gain from C]}

\index{Leek West , Staffordshire Moorlands@Leek W., \emph{Staffs. Moorlands}}

Death of Robert Plant (C).

\noindent
\begin{tabular*}{\columnwidth}{@{\extracolsep{\fill}} p{0.53\columnwidth} >{\itshape}l r @{\extracolsep{\fill}}}
Bill Cawley & Lab & 487\\
James Abberley & C & 370\\
George Herbert & LD & 218\\
Stephen Wales & Ind & 61\\
\end{tabular*}

\council{Tamworth}

\subsubsection*{Glascote \hspace*{\fill}\nolinebreak[1]%
\enspace\hspace*{\fill}
\finalhyphendemerits=0
[Friday 25th May]}

\index{Glascote , Tamworth@Glascote, \emph{Tamworth}}

Ordinary election postponed from 3rd May: death of candidate Sarah Walters (Grn).

\noindent
\begin{tabular*}{\columnwidth}{@{\extracolsep{\fill}} p{0.53\columnwidth} >{\itshape}l r @{\extracolsep{\fill}}}
Simon Peaple & Lab & 490\\
Allan Lunn & C & 478\\
Dennis Box & UKIP & 124\\
Kevin Jones & Grn & 55\\
\end{tabular*}

\section{Suffolk}

\subsection*{County Council}\index{Suffolk}

\subsubsection*{Bosmere \hspace*{\fill}\nolinebreak[1]%
	\enspace\hspace*{\fill}
	\finalhyphendemerits=0
	[25th October]}

\index{Bosmere , Suffolk@Bosmere, \emph{Suffolk}}

Death of Anne Whybrow (C).

\noindent
\begin{tabular*}{\columnwidth}{@{\extracolsep{\fill}} p{0.53\columnwidth} >{\itshape}l r @{\extracolsep{\fill}}}
Kay Oakes & C & 747\\
Steve Phillips & LD & 726\\
Emma Bonner-Morgan & Lab & 168\\
\end{tabular*}

\subsection*{St Edmundsbury}\index{Saint Edmundsbury@St Edmundsbury}

\subsubsection*{St Olaves \hspace*{\fill}\nolinebreak[1]%
\enspace\hspace*{\fill}
\finalhyphendemerits=0
[12th April]}

\index{Saint Olaves , Saint Edmundsbury@St Olaves, \emph{St Edmundsbury}}

Resignation of Bob Cockle (Lab).

\noindent
\begin{tabular*}{\columnwidth}{@{\extracolsep{\fill}} p{0.53\columnwidth} >{\itshape}l r @{\extracolsep{\fill}}}
Max Clarke & Lab & 365\\
Tom Murray & C & 150\\
Liam Byrne & Ind & 77\\
Helen Korfanty & LD & 31\\
\end{tabular*}

\subsubsection*{Haverhill East \hspace*{\fill}\nolinebreak[1]%
\enspace\hspace*{\fill}
\finalhyphendemerits=0
[3rd May]}

\index{Haverhill East , Saint Edmundsbury@Haverhill E., \emph{St Edmundsbury}}

Resignation of Ivor McLatchy (C).

\noindent
\begin{tabular*}{\columnwidth}{@{\extracolsep{\fill}} p{0.53\columnwidth} >{\itshape}l r @{\extracolsep{\fill}}}
Robin Pilley & C & 577\\
David Smith & Lab & 490\\
Saoirse O'Suilleabhán & LD & 151\\
\end{tabular*}

\subsubsection*{Haverhill North \hspace*{\fill}\nolinebreak[1]%
\enspace\hspace*{\fill}
\finalhyphendemerits=0
[3rd May]}

\index{Haverhill North , Saint Edmundsbury@Haverhill N., \emph{St Edmundsbury}}

Resignation of Betty McLatchy (C).

\noindent
\begin{tabular*}{\columnwidth}{@{\extracolsep{\fill}} p{0.53\columnwidth} >{\itshape}l r @{\extracolsep{\fill}}}
Elaine McManus & C & 752\\
Martin Jerram & Lab & 444\\
Peter Lord & LD & 148\\
\end{tabular*}

\council{Suffolk Coastal}

\subsubsection*{Leiston \hspace*{\fill}\nolinebreak[1]%
\enspace\hspace*{\fill}
\finalhyphendemerits=0
[17th May]}

\index{Leiston , Suffolk Coastal@Leiston, \emph{Suffolk Coastal}}

Resignation of Ian Pratt (C).

\noindent
\begin{tabular*}{\columnwidth}{@{\extracolsep{\fill}} p{0.53\columnwidth} >{\itshape}l r @{\extracolsep{\fill}}}
Susan Geater & C & 612\\
Freda Casagrande & Lab & 336\\
Sammy Betson & Ind & 293\\
Jules Ewart & LD & 213\\
\end{tabular*}

\subsubsection*{Wenhaston and Westleton \hspace*{\fill}\nolinebreak[1]%
\enspace\hspace*{\fill}
\finalhyphendemerits=0
[20th September]}

\index{Wenhaston and Westleton , Suffolk Coastal@Wenhaston \& Westleton, \emph{Suffolk Coastal}}

Death of Ray Catchpole (C).

\noindent
\begin{tabular*}{\columnwidth}{@{\extracolsep{\fill}} p{0.53\columnwidth} >{\itshape}l r @{\extracolsep{\fill}}}
Michael Gower & C & 431\\
Andrew Turner & LD & 340\\
Carl Bennett & Grn & 80\\
\end{tabular*}

\council{Waveney}

\subsubsection*{Pakefield \hspace*{\fill}\nolinebreak[1]%
\enspace\hspace*{\fill}
\finalhyphendemerits=0
[12th July; C gain from Lab]}

\index{Pakefield , Waveney@Pakefield, \emph{Waveney}}

Resignation of Sonia Barker (Lab).

\noindent
\begin{tabular*}{\columnwidth}{@{\extracolsep{\fill}} p{0.53\columnwidth} >{\itshape}l r @{\extracolsep{\fill}}}
Melanie Vigo di Gallidoro & C & 643\\
Paul Tyack & Lab & 600\\
Phillip Trindall & UKIP & 116\\
Peter Lang & Grn & 64\\
Adam Robertson & LD & 44\\
\end{tabular*}

\subsubsection*{Southwold and Reydon \hspace*{\fill}\nolinebreak[1]%
\enspace\hspace*{\fill}
\finalhyphendemerits=0
[12th July; LD gain from C]}

\index{Southwold and Reydon , Waveney@Southwold \& Reydon, \emph{Waveney}}

Death of Sue Allen (C).

\noindent
\begin{tabular*}{\columnwidth}{@{\extracolsep{\fill}} p{0.53\columnwidth} >{\itshape}l r @{\extracolsep{\fill}}}
David Beavan & LD & 1005\\
David Burrows & C & 307\\
John Cracknell & Lab & 78\\
Mike Shaw & UKIP & 18\\
\end{tabular*}

\section{Surrey}

\subsection*{County Council}\index{Surrey}

\subsubsection*{The Byfleets \hspace*{\fill}\nolinebreak[1]%
	\enspace\hspace*{\fill}
	\finalhyphendemerits=0
	[6th December; Ind gain from C]}

\index{Byfleets , Surrey@The Byfleets, \emph{Surrey}}

Death of Richard Wilson (C).

\noindent
\begin{tabular*}{\columnwidth}{@{\extracolsep{\fill}} p{0.53\columnwidth} >{\itshape}l r @{\extracolsep{\fill}}}
Amanda Boote & Ind & 1128\\
Gary Elson & C & 782\\
Ellen Nicholson & LD & 309\\
Lyn Sage & UKIP & 101\\
\end{tabular*}

\council{Elmbridge}

\subsubsection*{Oxshott and Stoke d'Abernon \hspace*{\fill}\nolinebreak[1]%
\enspace\hspace*{\fill}
\finalhyphendemerits=0
[12th July]}

\index{Oxshott and Stoke d'Abernon , Elmbridge@Oxshott \& Stoke d'Abernon, \emph{Elmbridge}}

Resignation of James Vickers (C).

\noindent
\begin{tabular*}{\columnwidth}{@{\extracolsep{\fill}} p{0.53\columnwidth} >{\itshape}l r @{\extracolsep{\fill}}}
David Lewis & C & 1297\\
Dorothy Ford & LD & 463\\
Nicholas Wood & UKIP & 42\\
\end{tabular*}

\council{Epsom and Ewell}

Nonsuch = Nonsuch Residents Association

RAEE = Residents Associations of Epsom and Ewell

WERRA = West Ewell and Ruxley Residents Association

\subsubsection*{Ruxley \hspace*{\fill}\nolinebreak[1]%
\enspace\hspace*{\fill}
\finalhyphendemerits=0
[15th February]}

\index{Ruxley , Epsom and Ewell@Ruxley, \emph{Epsom \& Ewell}}

Resignation of Keith Partridge (WERRA).

\noindent
\begin{tabular*}{\columnwidth}{@{\extracolsep{\fill}} p{0.53\columnwidth} >{\itshape}l r @{\extracolsep{\fill}}}
Alex Coley & RAEE & 398\\
Stephen Pontin & C & 340\\
Themba Msika & Lab & 264\\
Julia Kirkland & LD & 67\\
\end{tabular*}

\subsubsection*{Nonsuch \hspace*{\fill}\nolinebreak[1]%
\enspace\hspace*{\fill}
\finalhyphendemerits=0
[20th September]}

\index{Nonsuch , Epsom and Ewell@Nonsuch, \emph{Epsom \& Ewell}}

Death of David Wood (Nonsuch).

\noindent
\begin{tabular*}{\columnwidth}{@{\extracolsep{\fill}} p{0.53\columnwidth} >{\itshape}l r @{\extracolsep{\fill}}}
Colin Keane & RAEE & 766\\
Alastair Whitby & C & 227\\
Julian Freeman & LD & 92\\
Rosalind Godson & Lab & 41\\
\end{tabular*}

\council{Reigate and Banstead}

At the May 2018 ordinary election there was an unfilled vacancy in South Park and Woodhatch ward due to the resignation of Simon Rickman (C).\index{South Park and Woodhatch , Reigate and Banstead@South Park \& Woodhatch, \emph{Reigate \& Banstead}}

\subsubsection*{Horley West \hspace*{\fill}\nolinebreak[1]%
\enspace\hspace*{\fill}
\finalhyphendemerits=0
[3rd May]}

\index{Horley West , Reigate and Banstead@Horley W., \emph{Reigate \& Banstead}}

Resignation of David Jackson (C).

Combined with the 2018 ordinary election.
%; see page \pageref{HorleyWestReigateBanstead} for the result.

\council{Runnymede}

At the May 2018 ordinary election there was an unfilled vacancy in Chertsey South and Row Town ward due to the resignation of Barry Pitt (C).\index{Chertsey South and Row Town , Runnymede@Chertsey S. \& Row Town, \emph{Runnymede}}

\subsubsection*{Chertsey South and Row Town \hspace*{\fill}\nolinebreak[1]%
\enspace\hspace*{\fill}
\finalhyphendemerits=0
[3rd May]}

\index{Chertsey South and Row Town , Runnymede@Chertsey S. \& Row Town, \emph{Runnymede}}

Resignation of Terry Dicks (C).

Combined with the 2018 ordinary election.
%; see page \pageref{ChertseySouthRowTownRunnymede} for the result.

\subsubsection*{Foxhills \hspace*{\fill}\nolinebreak[1]%
\enspace\hspace*{\fill}
\finalhyphendemerits=0
[3rd May]}

\index{Foxhills , Runnymede@Foxhills, \emph{Runnymede}}

Resignation of Dannielle Khalique (C).

Combined with the 2018 ordinary election.
%; see page \pageref{FoxhillsRunnymede} for the result.

\council{Waverley}

Farnham = Farnham Residents

\subsubsection*{Farnham Castle \hspace*{\fill}\nolinebreak[1]%
\enspace\hspace*{\fill}
\finalhyphendemerits=0
[24th May]}

\index{Farnham Castle , Waverley@Farnham Castle, \emph{Waverley}}

Resignation of John Williamson (Farnham).

\noindent
\begin{tabular*}{\columnwidth}{@{\extracolsep{\fill}} p{0.525\columnwidth} >{\itshape}l r @{\extracolsep{\fill}}}
David Beaman & Farnham & 354\\
Jo Aylwin & LD & 338\\
Rashida Nasir & C & 175\\
Rebecca Kaye & Lab & 42\\
Mark Westcott & Ind & 26\\
\end{tabular*}

\section{Warwickshire}

\subsection*{County Council}\index{Warwickshire}

Strat1st = Stratford First Independents

\subsubsection*{Leamington Willes \hspace*{\fill}\nolinebreak[1]%
\enspace\hspace*{\fill}
\finalhyphendemerits=0
[3rd May]}

\index{Leamington Willes , Warwickshire@Leamington Willes, \emph{Warks.}}

Resignation of Matt Western MP (Lab).

\noindent
\begin{tabular*}{\columnwidth}{@{\extracolsep{\fill}} p{0.53\columnwidth} >{\itshape}l r @{\extracolsep{\fill}}}
Helen Adkins & Lab & 1164\\
Martin Luckhurst & Grn & 1139\\
Stacey Calder & C & 266\\
George Begg & LD & 36\\
\end{tabular*}

\subsubsection*{Stratford North \hspace*{\fill}\nolinebreak[1]%
	\enspace\hspace*{\fill}
	\finalhyphendemerits=0
	[29th November; LD gain from Strat1st]}

\index{Stratford North , Warwickshire@Stratford N., \emph{Warks.}}

Death of Keith Lloyd (Strat1st).

\noindent
\begin{tabular*}{\columnwidth}{@{\extracolsep{\fill}} p{0.53\columnwidth} >{\itshape}l r @{\extracolsep{\fill}}}
Dominic Skinner & LD & 877\\
Lynda Organ & C & 610\\
Juliet Short & Strat1st & 345\\
Joshua Payne & Lab & 180\\
John Riley & Grn & 144\\
\end{tabular*}

\council{North Warwickshire}

\subsubsection*{Newton Regis and Warton \hspace*{\fill}\nolinebreak[1]%
\enspace\hspace*{\fill}
\finalhyphendemerits=0
[23rd August]}

\index{Newton Regis and Warton , North Warwickshire@Newton Regis \& Warton, \emph{N. Warks.}}

Resignation of Patrick Davey (C).

\noindent
\begin{tabular*}{\columnwidth}{@{\extracolsep{\fill}} p{0.53\columnwidth} >{\itshape}l r @{\extracolsep{\fill}}}
Marian Humphreys & C & 451\\
Andrew Downes & Lab & 413\\
\end{tabular*}

\section{West Sussex}

\council{Adur}

\subsubsection*{Southlands \hspace*{\fill}\nolinebreak[1]%
\enspace\hspace*{\fill}
\finalhyphendemerits=0
[11th October; Lab gain from UKIP]}

\index{Southlands , Adur@Southlands, \emph{Adur}}

Resignation of Paul Graysmark (UKIP).

\noindent
\begin{tabular*}{\columnwidth}{@{\extracolsep{\fill}} p{0.53\columnwidth} >{\itshape}l r @{\extracolsep{\fill}}}
Debs Stainforth & Lab & 448\\
Andrew Bradbury & Grn & 395\\
Tony Nicklen & C & 132\\
\end{tabular*}

\council{Arun}

\subsubsection*{Marine \hspace*{\fill}\nolinebreak[1]%
\enspace\hspace*{\fill}
\finalhyphendemerits=0
[22nd February; LD gain from C]}

\index{Marine , Arun@Marine, \emph{Arun}}

Death of Dougal Maconachie (C).

\noindent
\begin{tabular*}{\columnwidth}{@{\extracolsep{\fill}} p{0.53\columnwidth} >{\itshape}l r @{\extracolsep{\fill}}}
Matt Stanley & LD & 309\\
Alison Sharples & Lab & 252\\
Kate Eccles & C & 242\\
Steve Goodheart & Ind & 141\\
\end{tabular*}

\council{Chichester}

\subsubsection*{Fishbourne \hspace*{\fill}\nolinebreak[1]%
\enspace\hspace*{\fill}
\finalhyphendemerits=0
[22nd February]}

\index{Fishbourne , Chichester@Fishbourne, \emph{Chichester}}

Resignation of Sandra Westacott (LD).

\noindent
\begin{tabular*}{\columnwidth}{@{\extracolsep{\fill}} p{0.53\columnwidth} >{\itshape}l r @{\extracolsep{\fill}}}
Adrian Moss & LD & 459\\
Libby Alexander & C & 294\\
Kevin Hughes & Lab & 88\\
\end{tabular*}

\subsubsection*{Rogate \hspace*{\fill}\nolinebreak[1]%
\enspace\hspace*{\fill}
\finalhyphendemerits=0
[12th April; LD gain from C]}

\index{Rogate , Chichester@Rogate, \emph{Chichester}}

Resignation of Gillian Keegan MP (C).

\noindent
\begin{tabular*}{\columnwidth}{@{\extracolsep{\fill}} p{0.53\columnwidth} >{\itshape}l r @{\extracolsep{\fill}}}
Kate O'Kelly & LD & 444\\
Robert Pettigrew & C & 319\\
Ray Davey & Lab & 21\\
Philip Naber & Grn & 12\\
\end{tabular*}

\council{Crawley}

At the May 2018 ordinary election there was an unfilled vacancy in Ifield ward due to the death of John Stanley (Lab).\index{Ifield , Crawley@Ifield, \emph{Crawley}}

\council{Horsham}

\subsubsection*{Cowfold, Shermanbury and West Grinstead \hspace*{\fill}\nolinebreak[1]%
\enspace\hspace*{\fill}
\finalhyphendemerits=0
[24th May]}

\index{Cowfold, Shermanbury and West Grinstead , Horsham@Cowfold, Shermanbury \& West Grinstead, \emph{Horsham}}

Death of Roger Clarke (C).

\noindent
\begin{tabular*}{\columnwidth}{@{\extracolsep{\fill}} p{0.53\columnwidth} >{\itshape}l r @{\extracolsep{\fill}}}
Lynn Lambert & C & 661\\
Kenneth Tyzack & Lab & 158\\
David Perry & LD & 148\\
\end{tabular*}

\section{Worcestershire}

\council{Worcester}

\subsubsection*{Warndon \hspace*{\fill}\nolinebreak[1]%
\enspace\hspace*{\fill}
\finalhyphendemerits=0
[3rd May]}

\index{Warndon , Worcester@Warndon, \emph{Worcester}}

Resignation of Elaine Williams (Lab).

Combined with the 2018 ordinary election.
%; see page \pageref{WarndonWorcester} for the result.

\council{Wyre Forest}

At the May 2018 ordinary election there was an unfilled vacancy in Wribbenhall and Arley ward due to the disqualification (non-attendance) of Gordon Yarranton (C).\index{Wribbenhall and Arley , Wyre Forest@Wribbenhall \& Arley, \emph{Wyre Forest}}

\subsubsection*{Bewdley and Rock \hspace*{\fill}\nolinebreak[1]%
\enspace\hspace*{\fill}
\finalhyphendemerits=0
[20th September]}

\index{Bewdley and Rock , Wyre Forest@Bewdley \& Rock, \emph{Wyre Forest}}

Death of Rod Wilson (C).

\noindent
\begin{tabular*}{\columnwidth}{@{\extracolsep{\fill}} p{0.53\columnwidth} >{\itshape}l r @{\extracolsep{\fill}}}
Anna Coleman & C & 734\\
Rod Stanczyszyn & Lab & 489\\
Clare Cassidy & LD & 109\\
John Davis & Grn & 85\\
\end{tabular*}

\section{Glamorgan}

\council{Merthyr Tydfil}

\subsubsection*{Gurnos \hspace*{\fill}\nolinebreak[1]%
\enspace\hspace*{\fill}
\finalhyphendemerits=0
[26th July; Ind gain from Lab]}

\index{Gurnos , Merthyr Tydfil@Gurnos, \emph{Merthyr Tydfil}}

Resignation of Rhonda Braithwaite (Lab).

\noindent
\begin{tabular*}{\columnwidth}{@{\extracolsep{\fill}} p{0.53\columnwidth} >{\itshape}l r @{\extracolsep{\fill}}}
Jeremy Davies & Ind & 375\\
Allyn Hooper & Lab & 368\\
Dillan Singh & Ind & 144\\
Laurel Ellis & C & 32\\
\end{tabular*}

\council{Neath Port Talbot}

\subsubsection*{Gwynfi \hspace*{\fill}\nolinebreak[1]%
\enspace\hspace*{\fill}
\finalhyphendemerits=0
[16th August; Ind gain from Lab]}

\index{Gwynfi , Neath Port Talbot@Gwynfi, \emph{Neath Port Talbot}}

Resignation of Ralph Thomas (Lab).

\noindent
\begin{tabular*}{\columnwidth}{@{\extracolsep{\fill}} p{0.53\columnwidth} >{\itshape}l r @{\extracolsep{\fill}}}
Jane Jones & Ind & 268\\
Katie Jones & PC & 73\\
Nicola Irwin & Lab & 60\\
David Joshua & Ind & 45\\
Jac Paul & Ind & 14\\
Orla Lowe & C & 4\\
\end{tabular*}

\section{Gwent}

\council{Torfaen}

\subsubsection*{Trevethin \hspace*{\fill}\nolinebreak[1]%
\enspace\hspace*{\fill}
\finalhyphendemerits=0
[22nd February]}

\index{Trevethin , Torfaen@Trevethin, \emph{Torfaen}}

Resignation of Matt Ford (Lab).

\noindent
\begin{tabular*}{\columnwidth}{@{\extracolsep{\fill}} p{0.6\columnwidth} >{\itshape}l r @{\extracolsep{\fill}}}
Emma Rapier & Lab & 233\\
Brynley Parker & Ind & 141\\
Frederick Wildgust & Ind & 69\\
Andrew Heygate-Browne & Grn & 15\\
\end{tabular*}

\section{Mid and West Wales}

\council{Carmarthenshire}

\subsubsection*{Saron \hspace*{\fill}\nolinebreak[1]%
\enspace\hspace*{\fill}
\finalhyphendemerits=0
[19th July]}

\index{Saron , Carmarthenshire@Saron, \emph{Carmarthenshire}}

Death of Alun Davies (PC).

\noindent
\begin{tabular*}{\columnwidth}{@{\extracolsep{\fill}} p{0.53\columnwidth} >{\itshape}l r @{\extracolsep{\fill}}}
Karen Davies & PC & 747\\
Tom Fallows & Lab & 239\\
Aled Crow & C & 146\\
Caryl Tandy & LD & 14\\
\end{tabular*}

\council{Pembrokeshire}

\subsubsection*{Pembroke: St Mary North \hspace*{\fill}\nolinebreak[1]%
\enspace\hspace*{\fill}
\finalhyphendemerits=0
[13th September; Ind gain from C]}

\index{Pembroke: Saint Mary North , Pembrokeshire@Pembroke: St Mary N., \emph{Pembrokeshire}}

Resignation of David Boswell (C).

\noindent
\begin{tabular*}{\columnwidth}{@{\extracolsep{\fill}} p{0.53\columnwidth} >{\itshape}l r @{\extracolsep{\fill}}}
Jon Harvey & Ind & 187\\
Daphne Bush & Ind & 79\\
Jonathan Nutting & Ind & 77\\
Maureen Bowen & Lab & 61\\
Bob Boucher & Ind & 59\\
Natalie Carey & C & 45\\
Lyn Edwards & Ind & 42\\
Al Williams & Ind & 26\\
\end{tabular*}

\section{Border Councils}

\council{Dumfries and Galloway}

\subsubsection*{Dee and Glenkens \hspace*{\fill}\nolinebreak[1]%
	\enspace\hspace*{\fill}
	\finalhyphendemerits=0
	[13th December]}

\index{Dee and Glenkens , Dumfries and Galloway@Dee \& Glenkens, \emph{Dumfries \& Galloway}}

Resignation of Patsy Gilroy (C).

\noindent
\begin{tabular*}{\columnwidth}{@{\extracolsep{\fill}} p{0.53\columnwidth} >{\itshape}l r @{\extracolsep{\fill}}}
\emph{First preferences}\\
Pauline Drysdale & C & 1682\\
Glen Murray & SNP & 1024\\
Colin Wyper & Ind & 569\\
Laura Moodie & Grn & 342\\
Jennifer Blue & UKIP & 46\\
\end{tabular*}

\emph{Moddie and Blue eliminated:} Drysdale 1740 Murray 1183 Wyper 651

\noindent
\begin{tabular*}{\columnwidth}{@{\extracolsep{\fill}} p{0.53\columnwidth} >{\itshape}l r @{\extracolsep{\fill}}}
	\emph{Wyper eliminated}\\
	Pauline Drysdale & C & 1956\\
	Glen Murray & SNP & 1357\\
\end{tabular*}

\columnbreak

\council{Scottish Borders}

\subsubsection*{Selkirkshire \hspace*{\fill}\nolinebreak[1]%
\enspace\hspace*{\fill}
\finalhyphendemerits=0
[22nd February; Ind gain from C]}

\index{Selkirkshre , Scottish Borders@Selkirkshire, \emph{Scottish Borders}}

Resignation of Michelle Ballantyne MSP (C).

\noindent
\begin{tabular*}{\columnwidth}{@{\extracolsep{\fill}} p{0.53\columnwidth} >{\itshape}l r @{\extracolsep{\fill}}}
\emph{First preferences}\\
Trevor Adams & C & 1247\\
Caroline Penman & Ind & 1040\\
John Mitchell & SNP & 691\\
Kenneth Gunn & Ind & 219\\
Scott Redpath & Lab & 134\\
Jack Clark & LD & 95\\
Barbra Harvie & Grn & 70\\
\end{tabular*}

\emph{Gunn, Redpath, Clark and Harvie eliminated:} Adams 1307, Penman 1231, Mitchell 797

\noindent
\begin{tabular*}{\columnwidth}{@{\extracolsep{\fill}} p{0.53\columnwidth} >{\itshape}l r @{\extracolsep{\fill}}}
\emph{Mitchell eliminated}\\
Caroline Penman & Ind & 1522\\
Trevor Adams & C & 1342\\
\end{tabular*}

\section{Clyde Councils}

\council{North Lanarkshire}

\subsubsection*{Coatbridge South \hspace*{\fill}\nolinebreak[1]%
	\enspace\hspace*{\fill}
	\finalhyphendemerits=0
	[25th October]}

\index{Coatbridge South , North Lanarkshire@Coatbridge S., \emph{N. Lanarks.}}

Death of Gordon Encinias (Lab).

\noindent
\begin{tabular*}{\columnwidth}{@{\extracolsep{\fill}} p{0.53\columnwidth} >{\itshape}l r @{\extracolsep{\fill}}}
\emph{First preferences}\\
Geraldine Woods & Lab & 1355\\
Lesley Mitchell & SNP & 1343\\
Ben Callaghan & C & 492\\
Rosemary McGowan & Grn & 47\\
Neil Wilson & UKIP & 14\\
Christopher Wilson & LD & 13\\
\end{tabular*}

\noindent
\begin{tabular*}{\columnwidth}{@{\extracolsep{\fill}} p{0.53\columnwidth} >{\itshape}l r @{\extracolsep{\fill}}}
\multicolumn{3}{@{\extracolsep{\fill}}l}{\emph{Four candidates eliminated}}\\
%	\emph{Callaghan, McGowan, Neil Wilson and Christopher Wilson eliminated}\\
	Geraldine Woods & Lab & 1549\\
	Lesley Mitchell & SNP & 1405\\
\end{tabular*}

\section{Forth Councils}

\council{Clackmannanshire}

\subsubsection*{Clackmannanshire North \hspace*{\fill}\nolinebreak[1]%
\enspace\hspace*{\fill}
\finalhyphendemerits=0
[1st March]}

\index{Clackmannanshire North , Clackmannanshire@Clackmannanshire N., \emph{Clacks.}}

Resignation of Archie Drummond (SNP).

\noindent
\begin{tabular*}{\columnwidth}{@{\extracolsep{\fill}} p{0.63\columnwidth} >{\itshape}l r @{\extracolsep{\fill}}}
\emph{First preferences}\\
Helen Lewis & SNP & 769\\
Alex Stewart & C & 659\\
Afifa Khanam & Lab & 493\\
Damian Sherwood-Johnson & LD & 84\\
Marion Robertson & Grn & 74\\
\end{tabular*}

\noindent
\begin{tabular*}{\columnwidth}{@{\extracolsep{\fill}} p{0.53\columnwidth} >{\itshape}l r @{\extracolsep{\fill}}}
\multicolumn{3}{@{\extracolsep{\fill}}l}{\emph{Three candidates eliminated}}\\
%\emph{Khanam, Sherwood-Johnson and Robertson eliminated}\\
Helen Lewis & SNP & 980\\
Alex Stewart & C & 784\\
\end{tabular*}

\council{Falkirk}

\subsubsection*{Bonnybridge and Larbert \hspace*{\fill}\nolinebreak[1]%
\enspace\hspace*{\fill}
\finalhyphendemerits=0
[15th February]}

\index{Bonnybridge and Larbert , Falkirk@Bonnybridge \& Larbert, \emph{Falkirk}}

Death of Tom Coleman (SNP).

\noindent
\begin{tabular*}{\columnwidth}{@{\extracolsep{\fill}} p{0.53\columnwidth} >{\itshape}l r @{\extracolsep{\fill}}}
\emph{First preferences}\\
Niall Coleman & SNP & 1295\\
George Stevenson & C & 1088\\
Linda Gow & Lab & 813\\
David Robertson & Grn & 124\\
Stuart Martin & UKIP & 35\\
\end{tabular*}

\noindent
\begin{tabular*}{\columnwidth}{@{\extracolsep{\fill}} p{0.53\columnwidth} >{\itshape}l r @{\extracolsep{\fill}}}
\multicolumn{3}{@{\extracolsep{\fill}}l}{\emph{Gow, Robertson and Martin eliminated}}\\
%\emph{Gow, Robertson and Martin eliminated}\\
Niall Coleman & SNP & 1619\\
George Stevenson & C & 1280\\
\end{tabular*}

\council{Fife}

Libtn = Libertarian Party

\subsubsection*{Inverkeithing and Dalgety Bay \hspace*{\fill}\nolinebreak[1]%
\enspace\hspace*{\fill}
\finalhyphendemerits=0
[6th September; C gain from Lab]}

\index{Inverkeithing and Dalgety Bay , Fife@Inverkeithing \& Dalgety Bay, \emph{Fife}}

Resignation of Lesley Laird MP (Lab).

\noindent
\begin{tabular*}{\columnwidth}{@{\extracolsep{\fill}} p{0.53\columnwidth} >{\itshape}l r @{\extracolsep{\fill}}}
\emph{First preferences}\\
Dave Coleman & C & 2309\\
Neale Hanvey & SNP & 1741\\
Billy Pollock & Lab & 744\\
Callum Hawthorne & LD & 566\\
Peter Collins & Ind & 521\\
Mags Hall & Grn & 257\\
Alastair Macintyre & Ind & 40\\
Calum Paul & Libtn & 13\\
\end{tabular*}

\emph{Hall, Macintyre and Paul eliminated}: Coleman 2338 Hanvey 1840 Pollock 794 Hawthorne 631 Collins 565

\emph{Collins eliminated}: Coleman 2455 Hanvey 1950 Pollock 867 Hawthorne 738

\emph{Hawthorne eliminated}: Coleman 2615 Hanvey 2076 Pollock 1058

\noindent
\begin{tabular*}{\columnwidth}{@{\extracolsep{\fill}} p{0.53\columnwidth} >{\itshape}l r @{\extracolsep{\fill}}}
\emph{Pollock eliminated}\\
Dave Coleman & C & 2839\\
Neale Hanvey & SNP & 2327\\
\end{tabular*}

\council{Midlothian}

\subsubsection*{Penicuik \hspace*{\fill}\nolinebreak[1]%
\enspace\hspace*{\fill}
\finalhyphendemerits=0
[22nd March; SNP gain from Lab]}

\index{Penicuik , Midlothian@Penicuik, \emph{Midlothian}}

Death of Adam Montgomery (Lab).

\noindent
\begin{tabular*}{\columnwidth}{@{\extracolsep{\fill}} p{0.53\columnwidth} >{\itshape}l r @{\extracolsep{\fill}}}
\emph{First preferences}\\
Joe Wallace & SNP & 1663\\
Murdo Macdonald & C & 1433\\
Vivienne Wallace & Lab & 1310\\
Helen Armstrong & Grn & 344\\
\end{tabular*}

\emph{Armstrong eliminated:} Joe Wallace 1803 Macdonald 1469 Vivienne Wallace 1414

\noindent
\begin{tabular*}{\columnwidth}{@{\extracolsep{\fill}} p{0.53\columnwidth} >{\itshape}l r @{\extracolsep{\fill}}}
\multicolumn{3}{@{\extracolsep{\fill}}l}{\emph{Vivienne Wallace eliminated}}\\
%\emph{Vivienne Wallace eliminated:}\\
Joe Wallace & SNP & 2237\\
Murdo Macdonald & C & 1788\\
\end{tabular*}

\section{Highland Councils}

\council{Highland}

\subsubsection*{Caol and Mallaig \hspace*{\fill}\nolinebreak[1]%
\enspace\hspace*{\fill}
\finalhyphendemerits=0
[5th April; LD gain from SNP]}

\index{Caol and Mallaig , Highland@Caol \& Mallaig, \emph{Highland}}

Death of Billy MacLachlan (SNP).

\noindent
\begin{tabular*}{\columnwidth}{@{\extracolsep{\fill}} p{0.53\columnwidth} >{\itshape}l r @{\extracolsep{\fill}}}
\emph{First preferences}\\
Denis Rixson & LD & 658\\
Alex MacInnes & SNP & 574\\
Colin ``Woody'' Wood & Ind & 454\\
Ian Smith & C & 183\\
Catherine MacKinnon & Ind & 146\\
Ronald Campbell & Ind & 98\\
\end{tabular*}

\emph{Smith, MacKinnon and Campbell eliminated:} Rixson 791 MacInnes 617 Wood 580

\noindent
\begin{tabular*}{\columnwidth}{@{\extracolsep{\fill}} p{0.53\columnwidth} >{\itshape}l r @{\extracolsep{\fill}}}
\emph{Wood eliminated}\\
Denis Rixson & LD & 968\\
Alex MacInnes & SNP & 737\\
\end{tabular*}

\subsubsection*{\sloppyword{Wester Ross, Strathpeffer and Lochalsh}
	\hspace*{\fill}\nolinebreak[1]%
	\enspace\hspace*{\fill}
	\finalhyphendemerits=0
	[6th December; SNP gain from LD]}

\index{Wester Ross, Strathpeffer and Lochalsh , Highland@Wester Ross, Strathpeffer \& Lochalsh, \emph{Highland}}

Resignation of Kate Stephen (LD).

\noindent
\begin{tabular*}{\columnwidth}{@{\extracolsep{\fill}} p{0.53\columnwidth} >{\itshape}l r @{\extracolsep{\fill}}}
\emph{First preferences}\\
Alexander MacInnes & SNP & 1318\\
Gavin Berkenheger & C & 1037\\
Richard Greene & Ind & 622\\
Irene Brandt & Grn & 359\\
George Scott & LD & 320\\
Christopher Birt & Lab & 174\\
Jean Davis & Ind & 131\\
Les Durance & UKIP & 16\\
Harry Christian & Libtn & 8\\
\end{tabular*}

\emph{Davis, Durance and Christian eliminated:} MacInnes 1327 Berkenheger 1061 Greene 677 Brandt 364 Scott 346 Birt 179

\emph{Birt eliminated:} MacInnes 1354 Berkenheger 1071 Greene 699 Brandt 411 Scott 379

\emph{Scott eliminated:} MacInnes 1397 Berkenheger 1147 Greene 785 Brandt 483

\emph{Brandt eliminated:} MacInnes 1575 Berkenheger 1186 Greene 905

\noindent
\begin{tabular*}{\columnwidth}{@{\extracolsep{\fill}} p{0.53\columnwidth} >{\itshape}l r @{\extracolsep{\fill}}}
\emph{Greene eliminated}\\
Alexander MacInnes & SNP & 1798\\
Gavin Berkenheger & C & 1374\\
\end{tabular*}

\section{Tay Councils}

\council{Perth and Kinross}

\subsubsection*{Highland \hspace*{\fill}\nolinebreak[1]%
\enspace\hspace*{\fill}
\finalhyphendemerits=0
[19th April]}

\index{Highland , Perth and Kinross@Highland, \emph{Perth \& Kinross}}

Death of Ian Campbell (C).

\noindent
\begin{tabular*}{\columnwidth}{@{\extracolsep{\fill}} p{0.53\columnwidth} >{\itshape}l r @{\extracolsep{\fill}}}
\emph{First preferences}\\
John Duff & C & 1907\\
John Kellas & SNP & 1466\\
Avril Taylor & Ind & 280\\
Jayne Ramage & Lab & 239\\
Mary McDougall & Grn & 104\\
Chris Rennie & LD & 78\\
Denise Baykal & Ind & 12\\
\end{tabular*}

\emph{McDougall, Rennie and Baykal eliminated:} Duff 1930 Kellas 1509 Taylor 325 Ramage 280

\emph{Ramage eliminated:} Duff 1977 Kellas 1594 Taylor 391

\noindent
\begin{tabular*}{\columnwidth}{@{\extracolsep{\fill}} p{0.53\columnwidth} >{\itshape}l r @{\extracolsep{\fill}}}
\emph{Taylor eliminated}\\
John Duff & C & 2084\\
John Kellas & SNP & 1712\\
\end{tabular*}

\section{Northern Ireland}

\council{Mid and East Antrim}

DVP = Democrats and Veterans Party

\subsubsection*{Carrick Castle \hspace*{\fill}\nolinebreak[1]%
\enspace\hspace*{\fill}
\finalhyphendemerits=0
[18th October; DUP gain from Ind]}

\index{Carrick Castle , Mid and East Antrim@Carrick Castle, \emph{Mid \& E. Antrim}}

Death of Jim Brown (Ind).

\noindent
\begin{tabular*}{\columnwidth}{@{\extracolsep{\fill}} p{0.53\columnwidth} >{\itshape}l r @{\extracolsep{\fill}}}
\emph{First preferences}\\
Peter Johnston & DUP & 1106\\
John McDermott & UUP & 668\\
Lauren Gray & All & 556\\
Si Harvey & DVP & 448\\
Will Sibley & Ind & 71\\
\end{tabular*}

\emph{Harvey and Sibley eliminated:} Johnston 1298 McDermott 823 Gray 645

\noindent
\begin{tabular*}{\columnwidth}{@{\extracolsep{\fill}} p{0.53\columnwidth} >{\itshape}l r @{\extracolsep{\fill}}}
	\emph{Gray eliminated}\\
	Peter Johnston & DUP & 1362\\
	John McDermott & UUP & 1147\\
\end{tabular*}

\end{resultsiii}

\documentclass[a4paper,openany]{book}
\usepackage[utf8]{inputenc}

% put all the other packages here:

\usepackage{election06-test}
\usepackage[t]{ebgaramond}
\usepackage{ebgaramond-maths}
	
\usepackage{graphicx}
\usepackage{url}
\usepackage{microtype}

\usepackage[plainpages=false,pdfpagelabels,pdfauthor={Andrew Teale},pdftitle={Local Election Results 2019},hidelinks]{hyperref}

\renewcommand\resultsyear{2019}

\setboolean{maps}{false}

\begin{document}

% Title page

\begin{titlepage}

\begin{center}

\Huge Local Election Results

2019

\bigskip

\Large Andrew Teale

\vfill

\newcommand\versionno{0.10.1}

%Version \versionno

\today

\end{center}

\end{titlepage}

% Copyright notice

\begin{center}

\bigskip

Typeset by \LaTeX{} 

\bigskip

Compilation and design \textcopyright\ Andrew Teale, 2019.

 Permission is granted to copy, distribute and/or modify this document
 under the terms of the GNU Free Documentation License, Version 1.3
 or any later version published by the Free Software Foundation;
 with no Invariant Sections, no Front-Cover Texts, and no Back-Cover Texts.
 A copy of the license is included in the section entitled ``GNU
 Free Documentation License''.

\bigskip

This file is available for download from
\url{http://www.andrewteale.me.uk/}

\bigskip

Please advise the author of any corrections which need to be made by
email: \url{andrewteale@yahoo.co.uk}

\vfill
\end{center}

\section*{Change Log}

%8 June 2014: Added result for Clydesdale South by-election.

%24 November 2013: Corrected results for Vassall ward, Lambeth (typing error) and Reddish North ward, Stockport (LD candidate incorrectly shown as Labour).
%
%21 November 2013: First version.

\tableofcontents

% Introduction

% Introduction

% \chapter*{Introduction and Abbreviations}
% \addcontentsline{toc}{chapter}{Introduction and Abbreviations}
% %\markright{INTRODUCTION AND ABBREVIATIONS}
% 
% Elections were held on 6th May 2010 to all London boroughs and metropolitan boroughs, and some unitary authorities and shire districts in England. These elections were combined with a general election which was held on the same day. 
%
% The voting system used for all elections covered here was
% first-past-the-post, with multi-member FPTP being used where more than
% one seat was up for election. 
% 
%The results of the general election are shown in Part I. The information in Part I is taken from the Electoral Commission.
%
% All of the seats on the 32 London borough councils were up for election. The vast majority of London boroughs use multi-member wards electing three councillors each; there are also a handful of single-member and two-member wards. Elections to the London boroughs are covered in Part~II, which has been split into two chapters (North and South London).
% 
% The 36 metropolitan boroughs are all elected by thirds. Each ward has
% three councillors, with the winning councillor from the 2006 election being up for
% election in each ward. In some cases two seats were up for election,
% due to the death or resignation of another councillor for the ward
% within six months of the election. Results of these elections are
% contained in Part~III, which each of the former metropolitan counties
% constituting a separate chapter.
% 
% The English unitary authorities and shire districts may have up to
% three councillors in each ward, and may hold elections either all at
% once or by thirds. 
% Only those councils which elect by thirds held
% elections this year; those councils which elect all at once were
% last elected in 2007 and will next be elected in 2011. A few
% districts elect by halves every two years; all of these districts held
% an election this year. Where districts elect by thirds generally not
% all of the wards in the district hold an election every year. A full explanation of the electoral arrangements is
% given at the head of each council's entry.
%
% Due to a botched attempt at local government reorganisation, the 2010 elections to Exeter and Norwich city councils were held on 9th September. It had originally been intended to change these councils to unitary status, which resulted in the scheduled 2010 elections to these councils being cancelled with the intention that the first elections to the new unitary councils would take place in 2011, the councillors elected in 2006 to have their terms extended until 2011. However, when the unitary plans were abandoned by the coalition government following the general election, the High Court ruled that the councillors elected in 2006 had come to the end of their four-year term and could no longer continue in office.
% 
% Unitary election results are shown in Part~IV 
% with shire district results in Part~V. Part~IV is
% divided into eight chapters based on region, while Part~V has one
% chapter for each county.
% 
% For the first time in this series, referendums (Part VI) and by-elections (Part VII) held in 2010 are also included. Scottish local by-elections are held using the Alternative Vote; while details of transfers are shown, for reasons of space some elimination stages have been omitted.
% 
% Finally, at the back you will find an Index of Wards.
% 
% Where a candidate in an election dies, the election in that ward or division is
% cancelled and rearranged for a later date. This happened in the
% following wards or divisions at this election:
% 
% \begin{results}
% \begin{itemize}
% \item Haverstock, Camden\index{Haverstock , Camden@Haverstock, Camden}
% \item Ore, Hastings\index{Ore , Hastings@Ore, Hastings}
% \end{itemize}
% \end{results}
% 
% Here is a list of abbreviations used in this book for major parties
% and selected other parties which fought several councils. This list
% is not exhaustive; parties which put up only a few candidates will
% generally have their abbreviation listed at the head of the entry for
% the retant council. Please note that the ``Lab'' label includes
% candidates who were jointly sponsored by the Labour and Co-operative
% Parties.
% 
% \begin{results}
% BNP - British National Party
% 
% C - Conservative Party
% 
% Grn - Green Party
% 
% Ind - Independent
% 
% Lab - Labour Party
% 
% LD - Liberal Democrat
% 
% Lib - Liberal Party
% 
% Loony - Monster Raving Loony Party
% 
% Respect - Respect, the Unity Coalition
% 
% SocLab - Socialist Labour Party
% 
% UKIP - UK Independence Party
% 
% \end{results}
%
%Errors in a work of this size are inevitable. I take full responsibility for any errors which may have crept in, undertake to correct any errors which I am made aware of, and hope that any errors which you may spot do not substantially affect any use you may make of this book.
%
% I would like to close this section by thanking all those who have
% supplied me with results and sources of information, most notably David Boothroyd, John Cartwright, James Doyle, Keith Edkins, the Electoral Commission, Tom Harris, Paul Harwood, ``hullenedge'', ``Listener'', ``MaxQue'', Philip Mutton, John Swarbrick, Andrew Stidwell, Pete Whitehead and all the members of the Vote UK Forum, and particularly those scores of council
% webpages without which this work would not have been possible. 

 
 
% Here beginneth the content

% 2019 results to go here


\part{Referendums}

\chapter{Referendums in 2019}

There were no referendums in 2019.

%\section{Burnley mayoral referendum}
%
%A referendum was held in Burnley on 4th May on the question of whether the district should have a directly elected mayor.
%
%\noindent
%\begin{tabular*}{\columnwidth}{@{\extracolsep{\fill}} p{0.545\columnwidth} >{\itshape}l r @{\extracolsep{\fill}}}
%& Yes & 8694\\
%& No & 10986\\
%\end{tabular*}

\part{By-elections}

\chapter{Parliamentary by-elections}

There were three parliamentary by-elections in 2019.

AWA = Abolish the Welsh Assembly Party

Brexit = Brexit Party

CGood = Common Good

DVP = Democrats and Veterans Party

ForBritn = The For Britain Movement

UKEU = UK European Union Party

\section*{Newport West \hspace*{\fill}\nolinebreak[1]%
\enspace\hspace*{\fill}
\finalhyphendemerits=0
[4th April]}

\index{Newport West , House of Commons@Newport W., \emph{House of Commons}}

Death of Paul Flynn (Lab).

\noindent
\begin{tabular*}{\columnwidth}{@{\extracolsep{\fill}} p{0.53\columnwidth} >{\itshape}l r @{\extracolsep{\fill}}}
Ruth Jones & Lab & 9308\\
Matthew Evans & C & 7357\\
Neil Hamilton & UKIP & 2023\\
Jonathan Clark & PC & 1185\\
Ryan Jones & LD & 1088\\
Amelia Womack & Grn & 924\\
June Davies & Renew & 879\\
Richard Suchorzewski & AWA & 205\\
Ian McLean & SDP & 202\\
Phillip Taylor & DVP & 185\\
Hugh Nicklin & ForBritn & 159\\
\end{tabular*}

\section*{Peterborough \hspace*{\fill}\nolinebreak[1]%
	\enspace\hspace*{\fill}
	\finalhyphendemerits=0
	[6th June]}

\index{Peterborough , House of Commons@Peterborough, \emph{House of Commons}}

Unseating of Fiona Onasanya (Lab) by recall petition.

\noindent
\begin{tabular*}{\columnwidth}{@{\extracolsep{\fill}} p{0.53\columnwidth} >{\itshape}l r @{\extracolsep{\fill}}}
	Lisa Forbes & Lab & 10484\\
	Mike Greene & Brexit & 9801\\
	Paul Bristow & C & 7243\\
	Andrew Moore & Ind & 101\\
	Beki Sellick & LD & 4159\\
	Joseph Wells & Grn & 1035\\
	John Whitby & UKIP & 400\\
	Tom Rogers & CPA & 162\\
	Stephen Goldspink & EDP & 153\\
	Patrick O'Flynn & SDP & 135\\
	Howling Laud Hope & Loony & 112\\
	Dick Rodgers & CGood & 60\\
	Peter Ward & Renew & 45\\
	Pierre Kirk & UKEU & 25\\
	Bobby Smith & Ind & 5\\
\end{tabular*}

\section*{Brecon and Radnorshire \hspace*{\fill}\nolinebreak[1]%
	\enspace\hspace*{\fill}
	\finalhyphendemerits=0
	[1st August; LD gain from C]}

\index{Brecon and Radnorshire , House of Commons@Brecon \& Radnorshire, \emph{House of Commons}}

Unseating of Christopher Davies (C) by recall petition.

\noindent
\begin{tabular*}{\columnwidth}{@{\extracolsep{\fill}} p{0.53\columnwidth} >{\itshape}l r @{\extracolsep{\fill}}}
Jane Dodds & LD & 13826\\
Chris Davies & C & 12401\\
Des Parkinson & Brexit & 3331\\
Tomos Davies & Lab & 1680\\
Lady Lily the Pink & Loony & 334\\
Liz Phillips & UKIP & 242\\
\end{tabular*}

At the December 2019 general election there were unfilled vacancies in Aylesbury and Bassetlaw following the resignation of John Bercow (Speaker) and the elevation of John Mann (Lab) to the peerage respectively.

\chapter{By-elections to devolved assemblies, the European Parliament, and police and crime commissionerships}

\section{Greater London Authority}

There were no by-elections in 2019 to the Greater London Authority.

%Kemi Badenoch (C, London-wide list) resigned in June 2017.  She was replaced from the list by Susan Hall.

\section{National Assembly for Wales}

There were no by-elections in 2019 to the National Assembly for Wales.

Steffan Lewis (PC, South Wales East) died on 11 January 2019.  He was replaced from the list by Delyth Jewell.

\section{Scottish Parliament}

There was one by-election in 2019 to the Scottish Parliament.

\subsubsection*{Shetland \hspace*{\fill}\nolinebreak[1]%
\enspace\hspace*{\fill}
\finalhyphendemerits=0
[29th August]}

\index{Shetland , Scottish Parliament@Shetland, \emph{Scottish Parliament}}

Resignation of Tavish Scott (LD).

\noindent
\begin{tabular*}{\columnwidth}{@{\extracolsep{\fill}} p{0.53\columnwidth} >{\itshape}l r @{\extracolsep{\fill}}}
Beatrice Wishart & LD & 5659\\
Tom Wills & SNP & 3822\\
Ryan Thomson & Ind & 1286\\
Brydon Goodlad & C & 425\\
Debra Nicolson & Grn & 189\\
Johan Adamson & Lab & 152\\
Michael Stout & Ind & 134\\
Ian Scott & Ind & 66\\
Stuart Martin & UKIP & 60\\
Peter Tait & Ind & 31\\
\end{tabular*}

%Rachael Hamilton (C, South of Scotland) resigned on 3 May 2017 in order to contest the above by-election.  She was replaced from the list by Michelle Ballantyne.
%
%Douglas Ross MP (C, Highlands and Islands) resigned on 9 June 2017.  He was replaced from the list by Jamie Halcro Johnston.
%
%Ross Thomson MP (C, North East) resigned on 9 June 2017.  He was replaced from the list by Tom Mason.

\section{Northern Ireland Assembly}

Vacancies in the Northern Ireland Assembly are filled by co-option.
No co-options were made in 2019.
%
%The following members were co-opted to the Assembly in 2017:
%\begin{itemize}
%\item Colm Gildernew (SF) replaced Michelle Gildernew MP following her resignation on 8th June (Fermanagh and South Tyrone).
%\item Trevor Clarke (DUP) replaced Paul Girvan MP following his resignation on 8th June (South Antrim).
%\item Emma Rogan (SF) replaced Chris Hazzard MP following his resignation on 8th June (South Down).
%\item Karen Mullan (SF) replaced Elisha McCallion MP following her resignation on 8th June (Foyle).
%\item Catherine Kelly (SF) replaced Barry McElduff MP following his resignation on 8th June (West Tyrone).
%\end{itemize}

\section{European Parliament}

UK vacancies in the European Parliament are filled by the next available person from the party list at the most recent election. 
No replacements were made in 2019.
%The following replacements were made in 2017:
%\begin{itemize}
%\item John Howarth (Lab) replaced Anneliese Dodds following her resignation on 8th June (South East).
%\item John Flack (C) replaced Vicky Ford following her resignation on 8th June (Eastern).
%\item Wajid Khan (Lab) replaced Afzal Khan following his resignation on 8th June (North West).
%\item Rupert Matthews (C) replaced Andrew Lewer following his resignation on 8th June (East Midlands).
%\item Baroness Mobarik (C) replaced Ian Duncan following his resignation on 22nd June (Scotland).
%\item Jonathan Bullock (UKIP) replaced Roger Helmer following his resignation on 28th July (East Midlands).
%\item Rory Palmer (Lab) replaced Dame Glenis Willmott following her resignation on 2nd October (East Midlands).
%\end{itemize}

\section{Police and crime commissioners}

There was one by-election in 2019 for a vacant police and crime commissioner post.

\subsection*{Northumbria \hspace*{\fill}\nolinebreak[1]%
	\enspace\hspace*{\fill}
	\finalhyphendemerits=0
	[18th July]}

\index{Northumbria Police and Crime Commissioner}

Resignation of Vera Baird (Lab).

\noindent
\begin{tabular*}{\columnwidth}{@{\extracolsep{\fill}} p{0.53\columnwidth} >{\itshape}l r @{\extracolsep{\fill}}}
	\emph{First preferences}\\
Cara McGuinness & Lab & 58355\\
Georgina Hill & Ind & 33704\\
Robbie Moore & C & 33267\\
Jonathan Wallace & LD & 28623\\
\end{tabular*}

\noindent
\begin{tabular*}{\columnwidth}{@{\extracolsep{\fill}} p{0.53\columnwidth} >{\itshape}l r @{\extracolsep{\fill}}}
	\emph{Runoff}\\
	Cara McGuinness & Lab & 67332\\
	Georgina Hill & Ind & 61633\\
\end{tabular*}

\chapter{Local by-elections and unfilled vacancies}

\begin{resultsiii}

\section{North London}

\subsection*{City of London}

\subsubsection*{Billingsgate
	\hspace*{\fill}\nolinebreak[1]%
	\enspace\hspace*{\fill}
	\finalhyphendemerits=0
	[31st January]}

\index{Billingsgate , City of London@Billingsgate, \emph{City of London}}

Aldermanic election: resignation of Matthew Richardson (Ind).

\noindent
\begin{tabular*}{\columnwidth}{@{\extracolsep{\fill}} p{0.53\columnwidth} >{\itshape}l r @{\extracolsep{\fill}}}
Bronek Masojada & Ind & 52\\
Andrew Heath-Richardson & Ind & 25\\
Rachel Kent & Ind & 15\\
Alpa Raja & Ind & 9\\
Havilland de Sausmarez & Ind & 8\\
Jonathan Bergdahl & SDP & 0\\
\end{tabular*}

\subsubsection*{Bassishaw
	\hspace*{\fill}\nolinebreak[1]%
	\enspace\hspace*{\fill}
	\finalhyphendemerits=0
	[Tuesday 30th April]}

\index{Bassishaw , City of London@Bassishaw, \emph{City of London}}

Aldermanic election: resignation of Timothy Hailes (Ind).

\noindent
\begin{tabular*}{\columnwidth}{@{\extracolsep{\fill}} p{0.53\columnwidth} >{\itshape}l r @{\extracolsep{\fill}}}
Timothy Hailes & Ind & 118\\
Ian Bishop-Laggett & Ind & 55\\
\end{tabular*}

\subsubsection*{Cordwainer
	\hspace*{\fill}\nolinebreak[1]%
	\enspace\hspace*{\fill}
	\finalhyphendemerits=0
	[Tuesday 30th April]}

\index{Cordwainer , City of London@Cordwainer, \emph{City of London}}

Resignation of Sir Mark Boleat (Ind).

\noindent
\begin{tabular*}{\columnwidth}{@{\extracolsep{\fill}} p{0.53\columnwidth} >{\itshape}l r @{\extracolsep{\fill}}}
Tracey Graham & Ind & 79\\
Timothy Becker & Ind & 5\\
\end{tabular*}

\subsubsection*{Farringdon Within
	\hspace*{\fill}\nolinebreak[1]%
	\enspace\hspace*{\fill}
	\finalhyphendemerits=0
	[Wednesday 24th July; result decided by drawing of lots]}

\index{Farringdon Within , City of London@Farringdon Wn., \emph{City of London}}

Resignation of Stuart Fraser (Ind).

\noindent
\begin{tabular*}{\columnwidth}{@{\extracolsep{\fill}} p{0.53\columnwidth} >{\itshape}l r @{\extracolsep{\fill}}}
John Edwards & Ind & 85\\
Virginia Rounding & Ind & 84\\
David Barker & Ind & 50\\
Paul O'Brien & Lab & 42\\
Ciara Murphy & Ind & 22\\
Emma Palmer & Ind & 3\\
\end{tabular*}

\subsubsection*{Coleman Street
	\hspace*{\fill}\nolinebreak[1]%
	\enspace\hspace*{\fill}
	\finalhyphendemerits=0
	[Tuesday 30th July]}

\index{Coleman Street , City of London@Coleman St., \emph{City of London}}

Resignation of Stuart Fraser (Ind).

\noindent
\begin{tabular*}{\columnwidth}{@{\extracolsep{\fill}} p{0.53\columnwidth} >{\itshape}l r @{\extracolsep{\fill}}}
Dawn Wright & Ind & 72\\
Alpa Raja & Ind & 39\\
Bren Albiston & Lab & 21\\
Timothy Becker & Ind & 15\\
\end{tabular*}

\subsubsection*{Lime Street
	\hspace*{\fill}\nolinebreak[1]%
	\enspace\hspace*{\fill}
	\finalhyphendemerits=0
	[Wednesday 7th August]}

\index{Lime Street , City of London@Lime St., \emph{City of London}}

Aldermanic election: resignation of Sir Charles Bowman (Ind).

\noindent
\begin{tabular*}{\columnwidth}{@{\extracolsep{\fill}} p{0.53\columnwidth} >{\itshape}l r @{\extracolsep{\fill}}}
Sir Charles Bowman & Ind & \emph{unop.}\\
\end{tabular*}

\subsubsection*{Vintry
	\hspace*{\fill}\nolinebreak[1]%
	\enspace\hspace*{\fill}
	\finalhyphendemerits=0
	[Wednesday 7th August]}

\index{Vintry , City of London@Vintry, \emph{City of London}}

Aldermanic election: resignation of Sir Andrew Parmley (Ind).

\noindent
\begin{tabular*}{\columnwidth}{@{\extracolsep{\fill}} p{0.53\columnwidth} >{\itshape}l r @{\extracolsep{\fill}}}
Sir Andrew Parmley & Ind & \emph{unop.}\\
\end{tabular*}

\subsubsection*{Aldersgate
	\hspace*{\fill}\nolinebreak[1]%
	\enspace\hspace*{\fill}
	\finalhyphendemerits=0
	[Wednesday 13th November]}

\index{Aldersgate , City of London@Aldersgate, \emph{City of London}}

Resignation of Richard Crossen (Lab).

\noindent
\begin{tabular*}{\columnwidth}{@{\extracolsep{\fill}} p{0.53\columnwidth} >{\itshape}l r @{\extracolsep{\fill}}}
Helen Fentimen & Lab & 260\\
Ian Burleigh & Ind & 137\\
Heather Thomas & Ind & 93\\
Paul Clifford & Ind & 38\\
Shahnan Bakth & Ind & 24\\
\end{tabular*}

\subsubsection*{Coleman Street
	\hspace*{\fill}\nolinebreak[1]%
	\enspace\hspace*{\fill}
	\finalhyphendemerits=0
	[19th December]}

\index{Coleman Street , City of London@Coleman St., \emph{City of London}}

Aldermanic election: resignation of Peter Estlin (Ind).

\noindent
\begin{tabular*}{\columnwidth}{@{\extracolsep{\fill}} p{0.53\columnwidth} >{\itshape}l r @{\extracolsep{\fill}}}
Peter Estlin & Ind & \emph{unop.}\\
\end{tabular*}

\subsubsection*{Dowgate
	\hspace*{\fill}\nolinebreak[1]%
	\enspace\hspace*{\fill}
	\finalhyphendemerits=0
	[19th December]}

\index{Dowgate , City of London@Dowgate, \emph{City of London}}

Aldermanic election: resignation of Alison Gowman (Ind).

\noindent
\begin{tabular*}{\columnwidth}{@{\extracolsep{\fill}} p{0.53\columnwidth} >{\itshape}l r @{\extracolsep{\fill}}}
Alison Gowman & Ind & \emph{unop.}\\
\end{tabular*}

\subsubsection*{Farrington Within
	\hspace*{\fill}\nolinebreak[1]%
	\enspace\hspace*{\fill}
	\finalhyphendemerits=0
	[19th December]}

\index{Farrington Within , City of London@Farrington Wn., \emph{City of London}}

Aldermanic election: resignation of Vincent Keaveny (Ind).

\noindent
\begin{tabular*}{\columnwidth}{@{\extracolsep{\fill}} p{0.53\columnwidth} >{\itshape}l r @{\extracolsep{\fill}}}
Vincent Keaveny & Ind & \emph{unop.}\\
\end{tabular*}

\subsection*{Camden}

\subsubsection*{Haverstock
	\hspace*{\fill}\nolinebreak[1]%
	\enspace\hspace*{\fill}
	\finalhyphendemerits=0
	[12th December]}

\index{Haverstock , Camden@Haverstock, \emph{Camden}}

Resignation of Abi Wood (Lab).

\noindent
\begin{tabular*}{\columnwidth}{@{\extracolsep{\fill}} p{0.53\columnwidth} >{\itshape}l r @{\extracolsep{\fill}}}
Gail McAnena Wood & Lab & 3121\\
Hunter Watts & Grn & 787\\
Catherine McQueen & C & 781\\
Jack Fleming & LD & 776\\
\end{tabular*}

\subsection*{Hackney}

\subsubsection*{Clissold
	\hspace*{\fill}\nolinebreak[1]%
	\enspace\hspace*{\fill}
	\finalhyphendemerits=0
	[12th December]}

\index{Clissold , Hackney@Clissold, \emph{Hackney}}

Resignation of Ned Hercock (Lab).

\noindent
\begin{tabular*}{\columnwidth}{@{\extracolsep{\fill}} p{0.53\columnwidth} >{\itshape}l r @{\extracolsep{\fill}}}
Kofo David & Lab & 3784\\
Marie Remy & Grn & 1597\\
Teresa Clark & LD & 612\\
Carmen Williams & C & 440\\
Tabitha Morton & WomensEq & 287\\
\end{tabular*}

\subsection*{Hmmersmith and Fulham}

\subsubsection*{Fulham Broadway
	\hspace*{\fill}\nolinebreak[1]%
	\enspace\hspace*{\fill}
	\finalhyphendemerits=0
	[19th September]}

\index{Fulham Broadway , Hammersmith and Fulham@Fulham Broadway, \emph{Hammersmith \& Fulham}}

Resignation of Alan de'Ath (Lab).

\noindent
\begin{tabular*}{\columnwidth}{@{\extracolsep{\fill}} p{0.53\columnwidth} >{\itshape}l r @{\extracolsep{\fill}}}
Helen Rowbottom & Lab & 1097\\
Jessie Venegas & LD & 755\\
Aliya Afzal Khan & C & 628\\
\end{tabular*}

\subsection*{Havering}

UCRA = Upminster and Cranham Residents Association

\subsubsection*{Cranham
	\hspace*{\fill}\nolinebreak[1]%
	\enspace\hspace*{\fill}
	\finalhyphendemerits=0
	[9th May]}

\index{Cranham , Havering@Cranham, \emph{Havering}}

Death of Clarence Barrett (UCRA).

\noindent
\begin{tabular*}{\columnwidth}{@{\extracolsep{\fill}} p{0.53\columnwidth} >{\itshape}l r @{\extracolsep{\fill}}}
Linda van den Hende & UCRA & 2421\\
Peter Caton & Grn & 312\\
Ben Sewell & C & 257\\
Adam Curtis & Lab & 219\\
Ben Buckland & UKIP & 208\\
Thomas Clarke & LD & 120\\
\end{tabular*}

\subsection*{Hounslow}

\subsubsection*{Feltham North
	\hspace*{\fill}\nolinebreak[1]%
	\enspace\hspace*{\fill}
	\finalhyphendemerits=0
	[12th December; C gain from Lab]}

\index{Feltham North , Hounslow@Feltham N., \emph{Hounslow}}

Death of John Chatt (Lab).

\noindent
\begin{tabular*}{\columnwidth}{@{\extracolsep{\fill}} p{0.53\columnwidth} >{\itshape}l r @{\extracolsep{\fill}}}
Kuldeep Tak & C & 2025\\
Adesh Farmahan & Lab & 1975\\
Joseph Bourke & LD & 351\\
Ioana-Simona Voicila & Grn & 209\\
Nooralhaq Nasimi & Ind & 91\\
\end{tabular*}

\subsubsection*{Heston West
	\hspace*{\fill}\nolinebreak[1]%
	\enspace\hspace*{\fill}
	\finalhyphendemerits=0
	[12th December]}

\index{Heston West , Hounslow@Heston W., \emph{Hounslow}}

Death of Rajinder Bath (Lab).

\noindent
\begin{tabular*}{\columnwidth}{@{\extracolsep{\fill}} p{0.53\columnwidth} >{\itshape}l r @{\extracolsep{\fill}}}
Balraj Sarai & Lab & 3440\\
Haresh Bhalsod & C & 1312\\
Hina Malik & LD & 377\\
Jon Elkon & Grn & 212\\
\end{tabular*}

\subsection*{Islington}

\subsubsection*{St George's
	\hspace*{\fill}\nolinebreak[1]%
	\enspace\hspace*{\fill}
	\finalhyphendemerits=0
	[12th December]}

\index{Saint George's , Islington@St George's, \emph{Islington}}

Resignation of Kat Fletcher (Lab).

\noindent
\begin{tabular*}{\columnwidth}{@{\extracolsep{\fill}} p{0.53\columnwidth} >{\itshape}l r @{\extracolsep{\fill}}}
Gulcin Ozdemir & Lab & 2918\\
Natasha Cox & Grn & 2501\\
Helen Redesdale & LD & 738\\
Guilene Marco & WomensEq & 268\\
\end{tabular*}

\subsection*{Kensington and Chelsea}

\subsubsection*{Dalgarno
\hspace*{\fill}\nolinebreak[1]%
\enspace\hspace*{\fill}
\finalhyphendemerits=0
[21st March]}

\index{Dalgarno , Kensington and Chelsea@Dalgarno, \emph{Kensington \& Chelsea}}

Resignation of Robert Thompson (Lab).

\noindent
\begin{tabular*}{\columnwidth}{@{\extracolsep{\fill}} p{0.53\columnwidth} >{\itshape}l r @{\extracolsep{\fill}}}
Kasim Ali & Lab & 719\\
Samia Bentayeb & C & 306\\
Alexandra Tatton-Brown & LD & 145\\
Callum Dorrington Hutton & UKIP & 68\\
Angela Georgievski & Grn & 61\\
\end{tabular*}

\subsection*{Tower Hamlets}

House = House Party

\subsubsection*{Lansbury
	\hspace*{\fill}\nolinebreak[1]%
	\enspace\hspace*{\fill}
	\finalhyphendemerits=0
	[7th February]}

\index{Lansbury , Tower Hamlets@Lansbury, \emph{Tower Hamlets}}

Resignation of Mohammad Harun (Lab).

\noindent
\begin{tabular*}{\columnwidth}{@{\extracolsep{\fill}} p{0.53\columnwidth} >{\itshape}l r @{\extracolsep{\fill}}}
Rajib Ahmed & Lab & 1308\\
Ohid Ahmed & Aspire & 1002\\
Muhammad Abul Asad & LD & 290\\
Paul Shea & UKIP & 176\\
Mumshad Afruz & C & 175\\
John Urpeth & Grn & 166\\
Terence McGrenera & House & 89\\
\end{tabular*}

\subsubsection*{Shadwell
	\hspace*{\fill}\nolinebreak[1]%
	\enspace\hspace*{\fill}
	\finalhyphendemerits=0
	[7th February; Aspire gain from Lab]}

\index{Shadwell , Tower Hamlets@Shadwell, \emph{Tower Hamlets}}

Resignation of Ruhul Amin (Lab).

\noindent
\begin{tabular*}{\columnwidth}{@{\extracolsep{\fill}} p{0.53\columnwidth} >{\itshape}l r @{\extracolsep{\fill}}}
Mohammad Harun Miah & Aspire & 1012\\
Asik Rahman & Lab & 914\\
Abjol Miah & LD & 484\\
Daryl Stafford & C & 185\\
Tim Kiely & Grn & 125\\
Kazi Gous-Miah & Ind & 119\\
Elena Scherbatykh & WEP & 65\\
\end{tabular*}

\section{South London}

\subsection*{Croydon}

\subsubsection*{Norbury and Pollards Hill
	\hspace*{\fill}\nolinebreak[1]%
	\enspace\hspace*{\fill}
	\finalhyphendemerits=0
	[14th March]}

\index{Norbury and Pollards Hill , Croydon@Norbury \& Pollards Hill, \emph{Croydon}}

Death of Maggie Mansell (Lab).

\noindent
\begin{tabular*}{\columnwidth}{@{\extracolsep{\fill}} p{0.53\columnwidth} >{\itshape}l r @{\extracolsep{\fill}}}
Leila Ben-Hassell & Lab & 1379\\
Tirena Gunter & C & 324\\
Mark O'Grady & Ind & 162\\
Rachel Chance & Grn & 91\\
Margaret Roznerska & Ind & 72\\
Guy Burchett & LD & 70\\
Kathleen Garnier & UKIP & 40\\
\end{tabular*}

\subsubsection*{Fairfield
	\hspace*{\fill}\nolinebreak[1]%
	\enspace\hspace*{\fill}
	\finalhyphendemerits=0
	[7th November]}

\index{Fairfield , Croydon@Fairfield, \emph{Croydon}}

Resignation of Niroshan Sirisena (Lab).

\noindent
\begin{tabular*}{\columnwidth}{@{\extracolsep{\fill}} p{0.53\columnwidth} >{\itshape}l r @{\extracolsep{\fill}}}
Caragh Skipper & Lab & 849\\
Jayde Edwards & C & 536\\
Andrew Rendle & LD & 397\\
Esther Sutton & Grn & 237\\
Heather Twidle & WEP & 40\\
Mark Samuel & Ind & 23\\
\end{tabular*}

\subsection*{Lambeth}

\subsubsection*{Thornton
	\hspace*{\fill}\nolinebreak[1]%
	\enspace\hspace*{\fill}
	\finalhyphendemerits=0
	[7th February]}

\index{Thornton , Lambeth@Thornton, \emph{Lambeth}}

Resignation of Jane Edbrooke (Lab).

\noindent
\begin{tabular*}{\columnwidth}{@{\extracolsep{\fill}} p{0.53\columnwidth} >{\itshape}l r @{\extracolsep{\fill}}}
Stephen Donnelly & Lab & 1154\\
Rebecca Macnair & LD & 845\\
Adrian Audsley & Grn & 251\\
Martin Read & C & 247\\
Leila Fazal & WEP & 46\\
John Plume & UKIP & 36\\
\end{tabular*}

\subsubsection*{Thornton
	\hspace*{\fill}\nolinebreak[1]%
	\enspace\hspace*{\fill}
	\finalhyphendemerits=0
	[11th April]}

\index{Thornton , Lambeth@Thornton, \emph{Lambeth}}

Resignation of Lib Peck (Lab).

\noindent
\begin{tabular*}{\columnwidth}{@{\extracolsep{\fill}} p{0.53\columnwidth} >{\itshape}l r @{\extracolsep{\fill}}}
Nanda Manley-Browne & Lab & 998\\
Matthew Bryant & LD & 979\\
Adrian Audsley & Grn & 171\\
Martin Read & C & 166\\
Leila Fazal & WEP & 53\\
John Plume & UKIP & 39\\
\end{tabular*}

\subsection*{Lewisham}

DVP = Democrats and Veterans Party

LPFP = Lewisham People Before Profit

\subsubsection*{Evelyn \hspace*{\fill}\nolinebreak[1]%
	\enspace\hspace*{\fill}
	\finalhyphendemerits=0
	[2nd May]}

\index{Evelyn , Lewisham@Evelyn, \emph{Lewisham}}

Resignation of Alex Feis-Bryce (Lab).

\noindent
\begin{tabular*}{\columnwidth}{@{\extracolsep{\fill}} p{0.53\columnwidth} >{\itshape}l r @{\extracolsep{\fill}}}
Lionel Openshaw & Lab & 1681\\
James Braun & Grn & 702\\
Eleanor Reader-Moore & C & 231\\
Bunmi Wajero & LD & 200\\
Joyce Jacca & LPFP & 151\\
Richard Day & UKIP & 140\\
Nicke Adebowale & WEP & 71\\
Matt Jenkins & DVP & 13\\
\end{tabular*}

\subsubsection*{Whitefoot \hspace*{\fill}\nolinebreak[1]%
	\enspace\hspace*{\fill}
	\finalhyphendemerits=0
	[2nd May]}

\index{Whitefoot , Lewisham@Whitefoot, \emph{Lewisham}}

Resignation of Janet Daby MP (Lab).

\noindent
\begin{tabular*}{\columnwidth}{@{\extracolsep{\fill}} p{0.53\columnwidth} >{\itshape}l r @{\extracolsep{\fill}}}
Kim Powell & Lab & 1314\\
Max Brockbank & LD & 514\\
Ben Blackmore & C & 313\\
Gwenton Sloley & LPBP & 218\\
Katherine Hortense & CPA & 52\\
Cairis Grant-Hickey & WEP & 41\\
Massimo Dimambro & DVP & 28\\
\end{tabular*}

\subsection*{Merton}

\subsubsection*{Cannon Hill \hspace*{\fill}\nolinebreak[1]%
	\enspace\hspace*{\fill}
	\finalhyphendemerits=0
	[20th June; LD gain from Lab]}

\index{Cannon Hill , Merton@Cannon Hill, \emph{Merton}}

Resignation of Mark Kenny (Lab).

\noindent
\begin{tabular*}{\columnwidth}{@{\extracolsep{\fill}} p{0.53\columnwidth} >{\itshape}l r @{\extracolsep{\fill}}}
Jenifer Gould & LD & 1060\\
Ryan Barnett & Lab & 875\\
Michael Paterson & C & 867\\
Susie O'Connor & Grn & 158\\
Andrew Mills & UKIP & 68\\
\end{tabular*}

\subsection*{Richmond upon Thames}

WomensEq = Women's Equality Party

\subsubsection*{East Sheen \hspace*{\fill}\nolinebreak[1]%
	\enspace\hspace*{\fill}
	\finalhyphendemerits=0
	[18th July]}

\index{East Sheen , Richmond upon Thames@East Sheen, \emph{Richmond upon Thames}}

Death of Mona Adams (LD).

\noindent
\begin{tabular*}{\columnwidth}{@{\extracolsep{\fill}} p{0.53\columnwidth} >{\itshape}l r @{\extracolsep{\fill}}}
Julia Cambridge & LD & 1809\\
Helen Edward & C & 1090\\
Trixie Rawlinson & WomensEq & 90\\
Giles Oakley & Lab & 82\\
\end{tabular*}

\subsection*{Sutton}

\subsubsection*{Wallington North \hspace*{\fill}\nolinebreak[1]%
	\enspace\hspace*{\fill}
	\finalhyphendemerits=0
	[28th March]}

\index{Wallington North , Sutton@Wallington N., \emph{Sutton}}

Resignation of Joyce Melican (LD).

\noindent
\begin{tabular*}{\columnwidth}{@{\extracolsep{\fill}} p{0.53\columnwidth} >{\itshape}l r @{\extracolsep{\fill}}}
Barry Lewis & LD & 1039\\
Charlotte Leonard & C & 709\\
Gervais Sawyer & Ind & 381\\
Sheila Berry & Lab & 301\\
Verity Thomson & Grn & 166\\
John Bannon & UKIP & 104\\
Ashley Dickenson & CPA & 17\\
\end{tabular*}

\subsection*{Wandsworth}

\subsubsection*{Furzedown \hspace*{\fill}\nolinebreak[1]%
	\enspace\hspace*{\fill}
	\finalhyphendemerits=0
	[20th June]}

\index{Furzedown , Wandsworth@Furzedown, \emph{Wandsworth}}

Resignation of Candida Jones (Lab).

\noindent
\begin{tabular*}{\columnwidth}{@{\extracolsep{\fill}} p{0.53\columnwidth} >{\itshape}l r @{\extracolsep{\fill}}}
Graham Loveland & Lab & 1811\\
Jon Irwin & LD & 887\\
Nabi Toktas & C & 681\\
Gerard Harrison & Grn & 318\\
\end{tabular*}

\section{Greater Manchester}

\subsection*{Bury}

Radcl1st = Radcliffe First

\subsubsection*{Radcliffe West \hspace*{\fill}\nolinebreak[1]%
	\enspace\hspace*{\fill}
	\finalhyphendemerits=0
	[29th August; Radcl1st gain from Lab]}

\index{Radcliffe West , Bury@Radcliffe W., \emph{Bury}}

Resignation of Rishi Shori (Lab).

\noindent
\begin{tabular*}{\columnwidth}{@{\extracolsep{\fill}} p{0.53\columnwidth} >{\itshape}l r @{\extracolsep{\fill}}}
Mike Smith & Radcl1st & 824\\
Jamie Walker & Lab & 708\\
Jordan Lewis & C & 283\\
Kingsley Jones & LD & 113\\
Anthony Clough & UKIP & 50\\
\end{tabular*}

\subsubsection*{Church \hspace*{\fill}\nolinebreak[1]%
	\enspace\hspace*{\fill}
	\finalhyphendemerits=0
	[12th December]}

\index{Church , Bury@Church, \emph{Bury}}

Death of Susan Nuttall (C).

\noindent
\begin{tabular*}{\columnwidth}{@{\extracolsep{\fill}} p{0.53\columnwidth} >{\itshape}l r @{\extracolsep{\fill}}}
Dene Vernon & C & 3317\\
Sam Turner & Lab & 2172\\
Charlie Allen & Grn & 235\\
Stephen Lewis & LD & 213\\
\end{tabular*}

\subsection*{Manchester}

\subsubsection*{Fallowfield \hspace*{\fill}\nolinebreak[1]%
	\enspace\hspace*{\fill}
	\finalhyphendemerits=0
	[2nd May]}

\index{Fallowfield , Manchester@Fallowfield, \emph{Manchester}}

Resignation of Grace Fletcher-Hackwood (Lab).

Combined with the 2019 ordinary election; see page \pageref{ManchesterFallowfield} for the result.

\subsection*{Salford}

\subsubsection*{Walkden South \hspace*{\fill}\nolinebreak[1]%
	\enspace\hspace*{\fill}
	\finalhyphendemerits=0
	[20th June; Lab gain from C]}

\index{Walkden South , Salford@Walkden S., \emph{Salford}}

Ordinary election postponed from 2nd May: death of candidate Andrew Darlington (C).

\noindent
\begin{tabular*}{\columnwidth}{@{\extracolsep{\fill}} p{0.53\columnwidth} >{\itshape}l r @{\extracolsep{\fill}}}
Joshua Brooks & Lab & 802\\
David Cawdrey & C & 654\\
Thomas Dylan & Grn & 254\\
John-Paul Atley & LD & 173\\
Tony Green & UKIP & 148\\
\end{tabular*}

\subsubsection*{Pendlebury \hspace*{\fill}\nolinebreak[1]%
	\enspace\hspace*{\fill}
	\finalhyphendemerits=0
	[12th December]}

\index{Pendlebury , Salford@Pendlebury, \emph{Salford}}

Death of John Ferguson (Lab).

\noindent
\begin{tabular*}{\columnwidth}{@{\extracolsep{\fill}} p{0.53\columnwidth} >{\itshape}l r @{\extracolsep{\fill}}}
Damian Bailey & Lab & -\\
Catherine Bisbey & C & -\\
Guy Otten & Grn & -\\
\end{tabular*}

\subsection*{Stockport}

\subsubsection*{Hazel Grove \hspace*{\fill}\nolinebreak[1]%
	\enspace\hspace*{\fill}
	\finalhyphendemerits=0
	[1st August]}

\index{Hazel Grove , Stockport@Hazel Grove, \emph{Stockport}}

Resignation of Jon Twigge (LD).

\noindent
\begin{tabular*}{\columnwidth}{@{\extracolsep{\fill}} p{0.53\columnwidth} >{\itshape}l r @{\extracolsep{\fill}}}
Charles Gibson & LD & 1401\\
Oliver Johnstone & C & 1194\\
Julie Wharton & Lab & 329\\
Michael Padfield & Grn & 142\\
\end{tabular*}

\subsection*{Tameside}

\subsubsection*{Denton West \hspace*{\fill}\nolinebreak[1]%
	\enspace\hspace*{\fill}
	\finalhyphendemerits=0
	[12th December]}

\index{Denton West , Tameside@Denton W., \emph{Tameside}}

Resignation of Dawson Lane (Lab).

\noindent
\begin{tabular*}{\columnwidth}{@{\extracolsep{\fill}} p{0.53\columnwidth} >{\itshape}l r @{\extracolsep{\fill}}}
Thomas Dunne & C & -\\
George Jones & Lab & -\\
Alice Mason-Power & LD & -\\
Jean Smee & Grn & -\\
\end{tabular*}

\section{Merseyside}

\subsection*{Liverpool}

OSAC = Old Swan Against the Cuts

\subsubsection*{Old Swan \hspace*{\fill}\nolinebreak[1]%
	\enspace\hspace*{\fill}
	\finalhyphendemerits=0
	[19th September]}

\index{Old Swan , Liverpool@Old Swan, \emph{Liverpool}}

Resignation of Peter Brennan (Lab).

\noindent
\begin{tabular*}{\columnwidth}{@{\extracolsep{\fill}} p{0.53\columnwidth} >{\itshape}l r @{\extracolsep{\fill}}}
William Shortall & Lab & 1153\\
Mick Coyne & Lib & 293\\
Chris Lea & LD & 272\\
Martin Ralph & OSAC & 138\\
George Maxwell & Grn & 130\\
Peter Andrew & C & 96\\
\end{tabular*}

\subsubsection*{Princes Park \hspace*{\fill}\nolinebreak[1]%
	\enspace\hspace*{\fill}
	\finalhyphendemerits=0
	[17th October]}

\index{Princes Park , Liverpool@Princes Park, \emph{Liverpool}}

Resignation of Timothy Moore (Lab).

\noindent
\begin{tabular*}{\columnwidth}{@{\extracolsep{\fill}} p{0.53\columnwidth} >{\itshape}l r @{\extracolsep{\fill}}}
Joanne Anderson & Lab & 1259\\
Stephanie Pitchers & Grn & 237\\
Lee Rowlands & LD & 148\\
Alma McGing & C & 79\\
\end{tabular*}

\subsubsection*{Clubmoor \hspace*{\fill}\nolinebreak[1]%
	\enspace\hspace*{\fill}
	\finalhyphendemerits=0
	[12th December]}

\index{Clubmoor , Liverpool@Clubmoor, \emph{Liverpool}}

Resignation of James Noakes (Lab).

\noindent
\begin{tabular*}{\columnwidth}{@{\extracolsep{\fill}} p{0.53\columnwidth} >{\itshape}l r @{\extracolsep{\fill}}}
Stephen Fitzsimmons & LD & -\\
Tim Jeeves & Lab & -\\
Paul Jones & Lib & -\\
Michael Stretton & Grn & -\\
\end{tabular*}

\subsubsection*{Picton \hspace*{\fill}\nolinebreak[1]%
	\enspace\hspace*{\fill}
	\finalhyphendemerits=0
	[12th December]}

\index{Picton , Liverpool@Picton, \emph{Liverpool}}

Resignation of Paul Kenyon (Lab).

\noindent
\begin{tabular*}{\columnwidth}{@{\extracolsep{\fill}} p{0.53\columnwidth} >{\itshape}l r @{\extracolsep{\fill}}}
Sophie Brown & Grn & -\\
Chris Hall & C & -\\
Adam Heatherington & Ind & -\\
Alan Oscroft & Lib & -\\
Calvin Smeda & Lab & -\\
Jenny Turner & LD & -\\
\end{tabular*}

\subsection*{Sefton}

\subsubsection*{Norwood \hspace*{\fill}\nolinebreak[1]%
	\enspace\hspace*{\fill}
	\finalhyphendemerits=0
	[2nd May]}

\index{Norwood , Sefton@Norwood, \emph{Sefton}}

Resignation of Bill Welsh (Lab elected as LD).

Combined with the 2019 ordinary election; see page \pageref{SeftonNorwood} for the result.

\section{Tyne and Wear}

\subsection*{Newcastle upon Tyne}

\subsubsection*{Monument \hspace*{\fill}\nolinebreak[1]%
	\enspace\hspace*{\fill}
	\finalhyphendemerits=0
	[2nd May]}

\index{Monument , Newcastle upon Tyne@Monument, \emph{Newcastle upon Tyne}}

Resignation of Rosie Hogg (Lab).

Combined with the 2019 ordinary election; see page \pageref{NewcastleTyneMonument} for the result.

\subsection*{South Tyneside}

\subsubsection*{Cleadon and East Boldon \hspace*{\fill}\nolinebreak[1]%
	\enspace\hspace*{\fill}
	\finalhyphendemerits=0
	[2nd May]}

\index{Cleadon and East Boldon , South Tyneside@Cleadon \& E. Boldon, \emph{S. Tyneside}}

Death of David Townsley (Lab).

Combined with the 2019 ordinary election; see page \pageref{SouthTynesideCleadonEastBoldon} for the result.

\subsection*{Sunderland}

At the May 2019 ordinary election there was an unfilled vacancy in Washington South ward due to the disqualification (non-attendance) of Paul Middleton (Lab).
\index{Washington South , Sunderland@Washington S., \emph{Sunderland}}

\subsubsection*{Sandhill \hspace*{\fill}\nolinebreak[1]%
	\enspace\hspace*{\fill}
	\finalhyphendemerits=0
	[2nd May]}

\index{Sandhill , Sunderland@Sandhill, \emph{Sunderland}}

Resignation of Lynn Appleby (Ind elected as LD).

Combined with the 2019 ordinary election; see page \pageref{SandhillSunderland} for the result.

\section{West Midlands}

\subsection*{Coventry}

\subsubsection*{Wainbody \hspace*{\fill}\nolinebreak[1]%
	\enspace\hspace*{\fill}
	\finalhyphendemerits=0
	[5th September]}

\index{Wainbody , Coventry@Wainbody, \emph{Coventry}}

Death of Gary Crookes (C).

\noindent
\begin{tabular*}{\columnwidth}{@{\extracolsep{\fill}} p{0.53\columnwidth} >{\itshape}l r @{\extracolsep{\fill}}}
Mattie Heaven & C & 1560\\
James Morshead & LD & 634\\
Abdul Jobbar & Lab & 544\\
George Beamish & Brexit & 193\\
\end{tabular*}

\subsection*{Wolverhampton}

\subsubsection*{Tettenhall Wightwick \hspace*{\fill}\nolinebreak[1]%
	\enspace\hspace*{\fill}
	\finalhyphendemerits=0
	[2nd May]}

\index{Tettenhall Wightwick , Wolverhampton@Tettenhall Wightwick, \emph{Wolverhampton}}

Resignation of Arun Photay (C).

Combined with the 2019 ordinary election; see page \pageref{WolverhamptonTettenhallWightwick} for the result.

\subsubsection*{Wednesfield South \hspace*{\fill}\nolinebreak[1]%
	\enspace\hspace*{\fill}
	\finalhyphendemerits=0
	[2nd May]}

\index{Wednesfield South , Wolverhampton@Wednesfield S., \emph{Wolverhampton}}

Resignation of Paula Brookfield (Lab).

Combined with the 2019 ordinary election; see page \pageref{WolverhamptonWednesfieldSouth} for the result.

\section{West Yorkshire}

\subsection*{Bradford}

\subsubsection*{Bolton and Undercliffe \hspace*{\fill}\nolinebreak[1]%
	\enspace\hspace*{\fill}
	\finalhyphendemerits=0
	[7th February; LD gain from Lab]}

\index{Bolton and Undercliffe , Bradford@Bolton \& Undercliffe, \emph{Bradford}}

Death of Ian Greenwood (Lab).

\noindent
\begin{tabular*}{\columnwidth}{@{\extracolsep{\fill}} p{0.53\columnwidth} >{\itshape}l r @{\extracolsep{\fill}}}
Rachel Sunderland & LD & 1733\\
Amriz Hussain & Lab & 1153\\
Ranbir Singh & C & 418\\
Phil Worsnop & Grn & 73\\
\end{tabular*}

\subsection*{Kirklees}

At the May 2019 ordinary election there was an unfilled vacancy in Bistall and Birkenshaw ward due to the resignation of Robert Light (C).
\index{Birstall and Birkenshaw , Kirklees@Birstall \& Birkenshaw, \emph{Kirklees}}

\subsubsection*{Colne Valley \hspace*{\fill}\nolinebreak[1]%
	\enspace\hspace*{\fill}
	\finalhyphendemerits=0
	[12th December; C gain from Lab]}

\index{Colne Valley , Kirklees@Colne Valley, \emph{Kirklees}}

Resignation of Neil Griffiths (Lab).

\noindent
\begin{tabular*}{\columnwidth}{@{\extracolsep{\fill}} p{0.53\columnwidth} >{\itshape}l r @{\extracolsep{\fill}}}
Donna Bellamy & C & 4504\\
Duggs Carre & Lab & 3308\\
Robert Iredale & LD & 1386\\
Ian Vincent & Grn & 646\\
\end{tabular*}

\subsubsection*{Dewsbury East \hspace*{\fill}\nolinebreak[1]%
	\enspace\hspace*{\fill}
	\finalhyphendemerits=0
	[12th December]}

\index{Dewsbury East , Kirklees@Dewsbury E., \emph{Kirklees}}

Resignation of Paul Kane (Lab).

\noindent
\begin{tabular*}{\columnwidth}{@{\extracolsep{\fill}} p{0.53\columnwidth} >{\itshape}l r @{\extracolsep{\fill}}}
Eric Firth & Lab & 3299\\
Keith Mallinson & C & 2669\\
Chris Stoner & HWDI & 1515\\
Dennis Hullock & LD & 380\\
\end{tabular*}

\subsection*{Leeds}

\subsubsection*{Wetherby \hspace*{\fill}\nolinebreak[1]%
	\enspace\hspace*{\fill}
	\finalhyphendemerits=0
	[12th December]}

\index{Wetherby , Leeds@Wetherby, \emph{Leeds}}

Death of Gerald Wilkinson (C).

\noindent
\begin{tabular*}{\columnwidth}{@{\extracolsep{\fill}} p{0.53\columnwidth} >{\itshape}l r @{\extracolsep{\fill}}}
Michael Bailey & Lab & -\\
Judith Dahlgreen & Grn & -\\
David Hopps & LD & -\\
Linda Richards & C & -\\
\end{tabular*}

\section{Bedfordshire}

\subsection*{Luton}

At the May 2019 ordinary election there was an unfilled vacancy in Stopsley ward due to the resignation of Meryl Dolling (LD).
\index{Stopsley , Luton@Stopsley, \emph{Luton}}

\subsubsection*{Icknield \hspace*{\fill}\nolinebreak[1]%
	\enspace\hspace*{\fill}
	\finalhyphendemerits=0
	[26th September; Lab gain from C]}

\index{Icknield , Luton@Icknield, \emph{Luton}}

Death of Mike Garrett (C).

\noindent
\begin{tabular*}{\columnwidth}{@{\extracolsep{\fill}} p{0.53\columnwidth} >{\itshape}l r @{\extracolsep{\fill}}}
Asif Masood & Lab & 585\\
John Baker & C & 563\\
Steve Moore & LD & 407\\
Marc Scheimann & Grn & 37\\
\end{tabular*}

\section{Berkshire}

\subsection*{Reading}

\subsubsection*{Thames \hspace*{\fill}\nolinebreak[1]%
	\enspace\hspace*{\fill}
	\finalhyphendemerits=0
	[2nd May]}

\index{Thames , Reading@Thames, \emph{Reading}}

Resignation of Ed Hopper (C).

Combined with the 2019 ordinary election; see page \pageref{ThamesReading} for the result.

\subsubsection*{Kentwood \hspace*{\fill}\nolinebreak[1]%
	\enspace\hspace*{\fill}
	\finalhyphendemerits=0
	[12th December]}

\index{Kentwood , Reading@Kentwood, \emph{Reading}}

Resignation of Emma Warman (C).

\noindent
\begin{tabular*}{\columnwidth}{@{\extracolsep{\fill}} p{0.53\columnwidth} >{\itshape}l r @{\extracolsep{\fill}}}
Gary Coster & LD & -\\
Glenn Dennis & Lab & -\\
Jenny Rynn & C & -\\
Richard Walker & Grn & -\\
\end{tabular*}

\subsection*{West Berkshire}

At the May 2019 ordinary election there was an unfilled vacancy in Greenham ward due to the resignation of Jeremy Bartlett (C).
\index{Greenham , West Berkshire@Greenham, \emph{W. Berks.}}

\subsection*{Windsor and Maidenhead}

tBf = the Borough first

WEP = Women's Equality Party

\subsubsection*{Riverside \hspace*{\fill}\nolinebreak[1]%
	\enspace\hspace*{\fill}
	\finalhyphendemerits=0
	[Wednesday 30th October]}

\index{Riverside , Windsor and Maidenhead@Riverside, \emph{Windsor \& Maidenhead}}

Resignation of Simon Dudley (C).

\noindent
\begin{tabular*}{\columnwidth}{@{\extracolsep{\fill}} p{0.53\columnwidth} >{\itshape}l r @{\extracolsep{\fill}}}
Gregory Jones & C & 794\\
Kashmir Singh & LD & 566\\
Claire Stretton & tBf & 428\\
Sharon Bunce & Lab & 70\\
Craig McDermott & Grn & 60\\
Deborah Mason & WEP & 16\\
\end{tabular*}

\subsection*{Wokingham}

\subsubsection*{Evendons \hspace*{\fill}\nolinebreak[1]%
	\enspace\hspace*{\fill}
	\finalhyphendemerits=0
	[7th February]}

\index{Evendons , Wokingham@Evendons, \emph{Wokingham}}

Resignation of Helen Power (LD).

\noindent
\begin{tabular*}{\columnwidth}{@{\extracolsep{\fill}} p{0.53\columnwidth} >{\itshape}l r @{\extracolsep{\fill}}}
Sarah Kerr & LD & 1441\\
Daniel Clawson & C & 729\\
Tim Lloyd & Lab & 115\\
\end{tabular*}

\section{Buckinghamshire}

EWycombe = East Wycombe Independents

\subsection*{County Council}

\subsubsection*{Totteridge and Bowerdean \hspace*{\fill}\nolinebreak[1]%
	\enspace\hspace*{\fill}
	\finalhyphendemerits=0
	[7th February; Lab gain from EWycombe]}

\index{Totteridge and Bowerdean , Buckinghamshire@Totteridge \& Bowerdean, \emph{Bucks.}}

Death of Chaudhary Ditta (EWycombe).

\noindent
\begin{tabular*}{\columnwidth}{@{\extracolsep{\fill}} p{0.53\columnwidth} >{\itshape}l r @{\extracolsep{\fill}}}
Israr Rashid & Lab & 978\\
Matt Knight & EWycombe & 668\\
Ben Holkham & LD & 508\\
Richard Peters & C & 245\\
\end{tabular*}

\subsection*{Aylesbury Vale}

\subsubsection*{Haddenham and Stone \hspace*{\fill}\nolinebreak[1]%
	\enspace\hspace*{\fill}
	\finalhyphendemerits=0
	[7th March; Grn gain from C]}

\index{Haddenham and Stone , Aylesbury Vale@Haddenham \& Stone, \emph{Aylesbury Vale}}

Death of Michael Edmonds (C).

\noindent
\begin{tabular*}{\columnwidth}{@{\extracolsep{\fill}} p{0.53\columnwidth} >{\itshape}l r @{\extracolsep{\fill}}}
David Lyons & Grn & 1210\\
Mark Bale & C & 781\\
Jim Brown & LD & 333\\
Jennifer Tuffley & Lab & 59\\
\end{tabular*}

\section{Cambridgeshire}

\subsection*{County Council}

\subsubsection*{Trumpington \hspace*{\fill}\nolinebreak[1]%
	\enspace\hspace*{\fill}
	\finalhyphendemerits=0
	[2nd May]}

\index{Trumpington , Cambridge@Trumpington, \emph{Cambridge}}

Resignation of Donald Adey (Ind elected as LD).

\noindent
\begin{tabular*}{\columnwidth}{@{\extracolsep{\fill}} p{0.53\columnwidth} >{\itshape}l r @{\extracolsep{\fill}}}
Barbara Ashwood & LD & 1328\\
Rob Grayston & Lab & 741\\
Shapour Meftah & C & 452\\
Beverley Carpenter & Grn & 325\\
\end{tabular*}

\subsection*{Cambridge}

\subsubsection*{King's Hedges \hspace*{\fill}\nolinebreak[1]%
	\enspace\hspace*{\fill}
	\finalhyphendemerits=0
	[2nd May]}

\index{King's Hedges , Cambridge@King's Hedges, \emph{Cambridge}}

Death of Nigel Gawthrope (Lab).

Combined with the 2019 ordinary election; see page \pageref{CambridgeKingsHedges} for the result.

\subsubsection*{Trumpington \hspace*{\fill}\nolinebreak[1]%
	\enspace\hspace*{\fill}
	\finalhyphendemerits=0
	[2nd May]}

\index{Trumpington , Cambridge@Trumpington, \emph{Cambridge}}

Resignation of Donald Adey (Ind elected as LD).

Combined with the 2019 ordinary election; see page \pageref{CambridgeTrumpington} for the result.

\subsubsection*{Newnham \hspace*{\fill}\nolinebreak[1]%
	\enspace\hspace*{\fill}
	\finalhyphendemerits=0
	[8th August]}

\index{Newnham , Cambridge@Newnham, \emph{Cambridge}}

Resignation of Lucy Nethsingha MEP (LD).

\noindent
\begin{tabular*}{\columnwidth}{@{\extracolsep{\fill}} p{0.53\columnwidth} >{\itshape}l r @{\extracolsep{\fill}}}
Josh Matthews & LD & 774\\
Niamh Sweeney & Lab & 235\\
Mark Slade & Grn & 149\\
Michael Spencer & C & 143\\
\end{tabular*}

\subsection*{Huntingdonshire}

\subsubsection*{Godmanchester and Hemingford Abbots \hspace*{\fill}\nolinebreak[1]%
	\enspace\hspace*{\fill}
	\finalhyphendemerits=0
	[1st August]}

\index{Godmanchester and Hemingford Abbots , Huntingdonshire@Godmanchester \& Hemingford Abbots, \emph{Hunts.}}

Resignation of David Underwood (LD).

\noindent
\begin{tabular*}{\columnwidth}{@{\extracolsep{\fill}} p{0.53\columnwidth} >{\itshape}l r @{\extracolsep{\fill}}}
Sarah Wilson & LD & 929\\
Paula Sparling & C & 666\\
Nigel Pauley & Ind & 333\\
\end{tabular*}

\subsubsection*{Alconbury \hspace*{\fill}\nolinebreak[1]%
	\enspace\hspace*{\fill}
	\finalhyphendemerits=0
	[12th December]}

\index{Alconbury , Huntingdonshire@Alconbury, \emph{Hunts.}}

Resignation of Jim White (C).

\noindent
\begin{tabular*}{\columnwidth}{@{\extracolsep{\fill}} p{0.53\columnwidth} >{\itshape}l r @{\extracolsep{\fill}}}
Ian Gardener & C & 1255\\
Alastair Henderson-Begg & LD & 365\\
Nick Sherratt & Lab & 269\\
Tom Maclennan & Ind & 235\\
Paul Buller & Ind & 89\\
\end{tabular*}

\subsection*{Peterborough}

At the May 2019 ordinary election there were unfilled vacancies in East and Park wards due to the resignations of Matthew Mahabadi and Richard Ferris (both Lab) respectively.
\index{East , Peterborough@East, \emph{Peterborough}}
\index{Park , Peterborough@Orton Longueville, \emph{Peterborough}}

\section{Cheshire}

\subsection*{Cheshire East}

At the May 2019 ordinary election there was an unfilled vacancy in Haslington ward due to the death of John Hammond (C).
\index{Haslington , Cheshire East@Haslington, \emph{Ches. E.}}

\section{Cornwall}

\subsection*{Cornwall}

\subsubsection*{Wadebridge West \hspace*{\fill}\nolinebreak[1]%
	\enspace\hspace*{\fill}
	\finalhyphendemerits=0
	[7th November; Ind gain from LD]}

\index{Wadebridge West , Cornwall@Wadebridge W., \emph{Cornwall}}

Resignation of Karen McHugh (LD).

\noindent
\begin{tabular*}{\columnwidth}{@{\extracolsep{\fill}} p{0.53\columnwidth} >{\itshape}l r @{\extracolsep{\fill}}}
Robin Moorcroft & Ind & 552\\
Philip Mitchell & C & 494\\
Julia Fletcher & LD & 250\\
Amanda Pennington & Grn & 123\\
Robyn Harris & Ind & 13\\
\end{tabular*}

\subsection*{Isles of Scilly}

\subsubsection*{St Mary's \hspace*{\fill}\nolinebreak[1]%
	\enspace\hspace*{\fill}
	\finalhyphendemerits=0
	[23rd May]}

\index{Saint Mary's , Isles of Scilly@St Mary's, \emph{Isles of Scilly}}

Resignation of Ted Moulson (Ind).

\noindent
\begin{tabular*}{\columnwidth}{@{\extracolsep{\fill}} p{0.53\columnwidth} >{\itshape}l r @{\extracolsep{\fill}}}
Tim Dean & Ind & 281\\
Jeanette Ware & Ind & 206\\
Steve Whomersley & Ind & 91\\
Andrew Coombes & Ind & 86\\
\end{tabular*}

\section{Cumbria}

\subsection*{County Council}

\subsubsection*{Thursby \hspace*{\fill}\nolinebreak[1]%
	\enspace\hspace*{\fill}
	\finalhyphendemerits=0
	[2nd May]}

\index{Thursby , Cumbria@Thursby, \emph{Cumbria}}

Death of Duncan Fairbairn (C).

\noindent
\begin{tabular*}{\columnwidth}{@{\extracolsep{\fill}} p{0.53\columnwidth} >{\itshape}l r @{\extracolsep{\fill}}}
Mike Johnson & C & \emph{unop.}\\
\end{tabular*}

\subsection*{Allerdale}

At the May 2019 ordinary election there were unfilled vacancies in Ewanrigg, Keswick, Moorclose and Warnell wards due to the resignation of Lee Williamson (Lab) and the deaths of Martin Pugmore and Denis Robertson (Ind) and Duncan Fairbairn (C) respectively.
\index{Ewanrigg , Allerdale@Ewanrigg, \emph{Allerdale}}
\index{Keswick , Allerdale@Keswick, \emph{Allerdale}}
\index{Moorclose , Allerdale@Moorclose, \emph{Allerdale}}
\index{Warnell , Allerdale@Warnell, \emph{Allerdale}}

\subsection*{Copeland}

At the May 2019 ordinary election there were unfilled vacancies in Distington and Kells wards due to the death of John Bowman (Lab) and the disqualification (non-attendance) of Michael Guest (Ind) respectively.
\index{Distington , Copeland@Distington, \emph{Copeland}}
\index{Kells , Copeland@Kells, \emph{Copeland}}

\subsection*{Eden}

Cumbria = Putting Cumbria First

\subsubsection*{Penrith South \hspace*{\fill}\nolinebreak[1]%
	\enspace\hspace*{\fill}
	\finalhyphendemerits=0
	[5th September; C gain from Ind]}

\index{Penrith South , Eden@Penrith S., \emph{Eden}}

Death of Paul Connor (Ind).

\noindent
\begin{tabular*}{\columnwidth}{@{\extracolsep{\fill}} p{0.53\columnwidth} >{\itshape}l r @{\extracolsep{\fill}}}
Helen Fearon & C & 222\\
Lee Quinn & Ind & 189\\
Dave Knaggs & Lab & 46\\
Kerryanne Wilde & Cumbria & 23\\
\end{tabular*}

\subsubsection*{Shap \hspace*{\fill}\nolinebreak[1]%
	\enspace\hspace*{\fill}
	\finalhyphendemerits=0
	[14th November; LD gain from C]}

\index{Shap , Eden@Shap, \emph{Eden}}

Resignation of John Owen (C).

\noindent
\begin{tabular*}{\columnwidth}{@{\extracolsep{\fill}} p{0.53\columnwidth} >{\itshape}l r @{\extracolsep{\fill}}}
Neil McCall & LD & 184\\
Sean Quinn & C & 128\\
Kerryanne Wilde & Cumbria & 67\\
\end{tabular*}

\section{Derbyshire}

\subsection*{Derbyshire Dales}

At the May 2019 ordinary election there was an unfilled vacancy in Ashbourne North ward due to the death of Tony Millward (C).
\index{Ashbourne North , Derbyshire Dales@Ashbourne N., \emph{Derbys. Dales}}

\subsection*{South Derbyshire}

At the May 2019 ordinary election there was an unfilled vacancy in Hilton ward due to the resignation of Amy Plenderleith (C).
\index{Hilton , South Derbyshire@Hilton, \emph{S. Derbys.}}

\section{Devon}

\subsection*{County Council}

ForBritn = The For Britain Movement

\subsubsection*{Heavitree and Whipton Barton \hspace*{\fill}\nolinebreak[1]%
	\enspace\hspace*{\fill}
	\finalhyphendemerits=0
	[24th October]}

\index{Heavitree and Whipton Barton , Devon@Heavitree \& Whipton Barton, \emph{Devon}}

Resignation of Emma Brennan (Lab).

\noindent
\begin{tabular*}{\columnwidth}{@{\extracolsep{\fill}} p{0.53\columnwidth} >{\itshape}l r @{\extracolsep{\fill}}}
Greg Sheldon & Lab & 1032\\
John Harvey & C & 992\\
Rowena Squires & LD & 576\\
Lizzie Woodman & Grn & 563\\
Frankie Rufolo & ForBritn & 70\\
\end{tabular*}

\subsection*{East Devon}

At the May 2019 ordinary election there was an unfilled vacancy in Exmouth Town ward due to the death of Bill Nash (C).
\index{Exmouth Town , East Devon@Exmouth Town, \emph{E. Devon}}

\subsection*{Exeter}

\subsubsection*{Priory \hspace*{\fill}\nolinebreak[1]%
	\enspace\hspace*{\fill}
	\finalhyphendemerits=0
	[2nd May]}

\index{Priory , Exeter@Priory, \emph{Exeter}}

Resignation of Kate Hannan (Lab).

Combined with the 2019 ordinary election; see page \pageref{ExeterPriory} for the result.

\subsubsection*{Topsham \hspace*{\fill}\nolinebreak[1]%
	\enspace\hspace*{\fill}
	\finalhyphendemerits=0
	[12th December]}

\index{Topsham , Exeter@Topsham, \emph{Exeter}}

Resignation of Caroline Pierce (C).

\noindent
\begin{tabular*}{\columnwidth}{@{\extracolsep{\fill}} p{0.53\columnwidth} >{\itshape}l r @{\extracolsep{\fill}}}
Keith Sparkes & C & 2315\\
Ivan Jordan & Lab & 1713\\
Christine Campion & LD & 1106\\
Frankie Rufolo & ForBritn & 245\\
\end{tabular*}

\subsection*{Mid Devon}

At the May 2019 ordinary election there was an unfilled vacancy in Silverton ward due to the death of Jenny Roach (Lib).
\index{Silverton , Mid Devon@Silverton, \emph{Mid Devon}}

\subsection*{North Devon}

At the May 2019 ordinary election there was an unfilled vacancy in Pilton ward due to the resignation of Mair Manuel (LD).
\index{Pilton , North Devon@Pilton, \emph{N. Devon}}

\subsubsection*{Chittlehampton \hspace*{\fill}\nolinebreak[1]%
	\enspace\hspace*{\fill}
	\finalhyphendemerits=0
	[13th June]}

\index{Chittlehampton , North Devon@Chittlehampton, \emph{N. Devon}}

Ordinary election postponed from 2nd May: death of outgoing councillor Walter White (Ind) who was seeking re-election.

\noindent
\begin{tabular*}{\columnwidth}{@{\extracolsep{\fill}} p{0.53\columnwidth} >{\itshape}l r @{\extracolsep{\fill}}}
Ray Jenkins & C & 323\\
Neil Basil & Grn & 248\\
Victoria Nel & LD & 221\\
Cecily Blyther & Lab & 16\\
\end{tabular*}

\subsection*{South Hams}

At the May 2019 ordinary election there was an unfilled vacancy in Charterlands ward due to the resignation of Elizabeth Huntley (LD).
\index{Charterlands , South Hams@Charterlands, \emph{S. Hams}}

\subsection*{Teignbridge}

At the May 2019 ordinary election there was an unfilled vacancy in Teign Valley ward due to the resignation of Amanda Ford (C).
\index{Teign Valley , Teignbridge@Teign Valley, \emph{Teignbridge}}

\subsection*{Torbay}

\subsubsection*{Goodrington with Roselands \hspace*{\fill}\nolinebreak[1]%
	\enspace\hspace*{\fill}
	\finalhyphendemerits=0
	[14th November; C gain from LD]}

\index{Goodrington with Roselands , Torbay@Goodrington with Roselands, \emph{Torbay}}

Resignation of Rick Heyse (LD).

\noindent
\begin{tabular*}{\columnwidth}{@{\extracolsep{\fill}} p{0.53\columnwidth} >{\itshape}l r @{\extracolsep{\fill}}}
Jane Barnby & C & 892\\
Dennis Shearman & LD & 641\\
Eddie Davis & Brexit & 168\\
Catherine Fritz & Lab & 72\\
Jane Hughes & Grn & 35\\
\end{tabular*}

\subsection*{Torridge}

At the May 2019 ordinary election there were unfilled vacancies in Bideford East and Winkleigh wards due to the death of Mervyn Langmead and the resignation of Betty Boundy (both C) respectively.
\index{Bideford East , Torridge@Bideford E., \emph{Torridge}}
\index{Winkleigh , Torridge@Winkleigh, \emph{Torridge}}

\section{Dorset}

\subsection*{Weymouth and Portland}

At the abolition of Weymouth and Portland council in April 2019 there were unfilled vacancies in Radipole and Tophill West wards due to the resignations of Ian Roebuck (LD) and Kerry Baker (C) respectively.
\index{Radipole , Weymouth and Portland@Radipole, \emph{Weymouth \& Portland}}
\index{Tophill West , Weymouth and Portland@Tophill W., \emph{Weymouth \& Portland}}

\section{Durham}

\subsection*{Durham}

ForBritn = For Britain Movement

NEP = North East Party

Spennymr = Spennymoor Independent

\subsubsection*{Wingate \hspace*{\fill}\nolinebreak[1]%
	\enspace\hspace*{\fill}
	\finalhyphendemerits=0
	[14th March]}

\index{Wingate , Durham@Wingate, \emph{Durham}}

Death of Leo Taylor (Lab).

\noindent
\begin{tabular*}{\columnwidth}{@{\extracolsep{\fill}} p{0.53\columnwidth} >{\itshape}l r @{\extracolsep{\fill}}}
John Higgins & Lab & 458\\
Edwin Simpson & LD & 163\\
Stephen Miles & NEP & 74\\
Gareth Fry & ForBritn & 20\\
\end{tabular*}

\subsubsection*{Esh and Witton Gilbert \hspace*{\fill}\nolinebreak[1]%
	\enspace\hspace*{\fill}
	\finalhyphendemerits=0
	[21st March]}

\index{Esh and Witton Gilbert , Durham@Esh \& Witton Gilbert, \emph{Durham}}

Resignation of Michael McGaun (LD).

\noindent
\begin{tabular*}{\columnwidth}{@{\extracolsep{\fill}} p{0.53\columnwidth} >{\itshape}l r @{\extracolsep{\fill}}}
Beverley Coult & LD & 1115\\
Anne Bonner & Lab & 366\\
Ryan Drion & Ind & 155\\
Richard Lawrie & C & 128\\
\end{tabular*}

\subsubsection*{Shildon and Dene Valley \hspace*{\fill}\nolinebreak[1]%
	\enspace\hspace*{\fill}
	\finalhyphendemerits=0
	[2nd May; LD gain from Lab]}

\index{Shildon and Dene Valley , Durham@Shildon \& Dene Valley, \emph{Durham}}

Death of Henry Nicholson (Lab).

\noindent
\begin{tabular*}{\columnwidth}{@{\extracolsep{\fill}} p{0.53\columnwidth} >{\itshape}l r @{\extracolsep{\fill}}}
James Huntington & LD & 1257\\
Samantha Townsend & Lab & 682\\
Alan Breeze & UKIP & 456\\
Robert Ingledew & Ind & 415\\
Marie Carter-Robb & C & 131\\
\end{tabular*}

\subsubsection*{Spennymoor \hspace*{\fill}\nolinebreak[1]%
	\enspace\hspace*{\fill}
	\finalhyphendemerits=0
	[2nd May; Ind gain from Spennymr]}

\index{Spennymoor , Durham@Spenymoor, \emph{Durham}}

Resignation of Geoffrey Darkes (Spennymr).

\noindent
\begin{tabular*}{\columnwidth}{@{\extracolsep{\fill}} p{0.53\columnwidth} >{\itshape}l r @{\extracolsep{\fill}}}
Ian Geldard & Ind & 489\\
Colin Nelson & Lab & 420\\
Martin Jones & LD & 373\\
Billy McAloon & Ind & 358\\
Pete Molloy & Ind & 332\\
Bob Purvis & UKIP & 281\\
Ronald Highley & Spennymr & 221\\
James Cosslett & C & 137\\
\end{tabular*}

\subsection*{Hartlepool}

ForBritn = For Britain Movement

IndU = Independent Union

\subsubsection*{Hart \hspace*{\fill}\nolinebreak[1]%
	\enspace\hspace*{\fill}
	\finalhyphendemerits=0
	[25th July]}

\index{Hart , Hartlepool@Hart, \emph{Hartlepool}}

Resignation of Jean Robinson (Ind elected as Lab).

\noindent
\begin{tabular*}{\columnwidth}{@{\extracolsep{\fill}} p{0.53\columnwidth} >{\itshape}l r @{\extracolsep{\fill}}}
Ann Johnson & Lab & 366\\
Ian Griffths & IndU & 358\\
Michael Ritchie & Grn & 196\\
Graham Craddy & ForBritn & 166\\
Graham Harrison & UKIP & 114\\
\end{tabular*}

\section{East Sussex}

\subsection*{County Council}

\subsubsection*{Bexhill West \hspace*{\fill}\nolinebreak[1]%
	\enspace\hspace*{\fill}
	\finalhyphendemerits=0
	[10th January]}

\index{Bexhill West , East Sussex@Bexhill W., \emph{E. Sussex}}

Death of Stuart Earl (Ind).

\noindent
\begin{tabular*}{\columnwidth}{@{\extracolsep{\fill}} p{0.53\columnwidth} >{\itshape}l r @{\extracolsep{\fill}}}
Deirdre Earl-Williams & Ind & 1761\\
Martin Kenward & C & 1071\\
Richard Thomas & LD & 261\\
Jacque Walker & Lab & 111\\
Polly Gray & Grn & 107\\
Geoffrey Bastin & UKIP & 81\\
\end{tabular*}

\subsection*{Brighton and Hove}

At the May 2019 ordinary election there was an unfilled vacancy in Hollingdean and Stanmer ward due to the resignation of Caroline Penn (Lab).
\index{Hollingdean and Stanmer , Brighton and Hove@Hollingdean \& Stanmer, \emph{Brighton \& Hove}}

\subsection*{Eastbourne}

At the May 2019 ordinary election there was an unfilled vacancy in Sovereign ward due to the resignation of Gordon Jenkins (C).
\index{Sovereign , Eastbourne@Sovereign, \emph{Eastbourne}}

\subsection*{Rother}

At the May 2019 ordinary election there was an unfilled vacancy in St Marks ward due to the resignation of Tom Graham (C).
\index{Saint Marks , Rother@St Marks, \emph{Rother}}

\subsubsection*{St Marks \hspace*{\fill}\nolinebreak[1]%
	\enspace\hspace*{\fill}
	\finalhyphendemerits=0
	[10th January]}

\index{Saint Marks , Rother@St Marks, \emph{Rother}}

Death of Stuart Earl (Ind).

\noindent
\begin{tabular*}{\columnwidth}{@{\extracolsep{\fill}} p{0.53\columnwidth} >{\itshape}l r @{\extracolsep{\fill}}}
Kathy Harmer & Ind & 1000\\
Gino Forte & C & 521\\
John Walker & Lab & 79\\
John Zipser & UKIP & 48\\
\end{tabular*}

\section{East Yorkshire}

\subsection*{East Riding}

Yorks = Yorkshire Party

\subsubsection*{Bridlington North \hspace*{\fill}\nolinebreak[1]%
	\enspace\hspace*{\fill}
	\finalhyphendemerits=0
	[11th July; LD gain from C]}

\index{Bridlington North , East Yorkshire@Bridlington N., \emph{E. Yorks.}}

Death of Richard Harrap (C).

\noindent
\begin{tabular*}{\columnwidth}{@{\extracolsep{\fill}} p{0.53\columnwidth} >{\itshape}l r @{\extracolsep{\fill}}}
Mike Heslop-Mullens & LD & 1308\\
Martin Burnhill & C & 815\\
Paul Walker & Yorks & 349\\
Gary Shores & UKIP & 196\\
Mike Dixon & Lab & 135\\
Terry Dixon & Ind & 125\\
David Robson & Ind & 76\\
Thelma Milns & Ind & 58\\
\end{tabular*}

\subsection*{Hull}

\subsubsection*{St Andrews and Docklands \hspace*{\fill}\nolinebreak[1]%
	\enspace\hspace*{\fill}
	\finalhyphendemerits=0
	[5th September]}

\index{Saint Andrews and Docklands , Hull@St Andrews \& Docklands, \emph{Hull}}

Death of Nadine Fudge (Lab).

\noindent
\begin{tabular*}{\columnwidth}{@{\extracolsep{\fill}} p{0.53\columnwidth} >{\itshape}l r @{\extracolsep{\fill}}}
Leanne Fudge & Lab & 837\\
Tracey Henry & LD & 805\\
Dan Bond & C & 193\\
\end{tabular*}

\section{Essex}

\subsection*{County Council}

HOSEM = Holland on Sea and Eastcliff Matters

\subsubsection*{Clacton East \hspace*{\fill}\nolinebreak[1]%
	\enspace\hspace*{\fill}
	\finalhyphendemerits=0
	[3rd October]}

\index{Clacton East , Essex@Clacton E., \emph{Essex}}

Resignation of David Sargeant (Ind).

\noindent
\begin{tabular*}{\columnwidth}{@{\extracolsep{\fill}} p{0.53\columnwidth} >{\itshape}l r @{\extracolsep{\fill}}}
Mark Stephenson & Ind & 1231\\
Chris Amos & C & 1223\\
K T King & HOSEM & 537\\
Callum Robertson & LD & 140\\
Geoff Ely & Lab & 111\\
Chris Southall & Grn & 47\\
\end{tabular*}

\subsection*{Basildon}

\subsubsection*{Vange \hspace*{\fill}\nolinebreak[1]%
	\enspace\hspace*{\fill}
	\finalhyphendemerits=0
	[21st March]}

\index{Vange , Basildon@Vange, \emph{Basildon}}

Resignation of Kayte Block (Lab).

\noindent
\begin{tabular*}{\columnwidth}{@{\extracolsep{\fill}} p{0.53\columnwidth} >{\itshape}l r @{\extracolsep{\fill}}}
Aidan McGurran & Lab & 504\\
Yetunde Adeshile & C & 478\\
\end{tabular*}

\subsection*{Braintree}

At the May 2019 ordinary election there was an unfilled vacancy in Hedingham ward due to the resignation of Joanne Beavis (C).
\index{Hedingham , Braintree@Hedingham, \emph{Braintree}}

\subsection*{Chelmsford}

\subsubsection*{Marconi \hspace*{\fill}\nolinebreak[1]%
	\enspace\hspace*{\fill}
	\finalhyphendemerits=0
	[7th November]}

\index{Marconi , Chelmsford@Marconi, \emph{Chelmsford}}

Resignation of Catherine Finnecy (LD).

\noindent
\begin{tabular*}{\columnwidth}{@{\extracolsep{\fill}} p{0.53\columnwidth} >{\itshape}l r @{\extracolsep{\fill}}}
Smita Rajesh & LD & 563\\
Ben McNally & C & 311\\
Paul Bishop & Lab & 156\\
Steven Chambers & Ind & 72\\
Ben Harvey & Grn & 69\\
\end{tabular*}

\subsection*{Rochford}

Rochford = Rochford District Residents

\subsubsection*{Sweyne Park and Grange \hspace*{\fill}\nolinebreak[1]%
	\enspace\hspace*{\fill}
	\finalhyphendemerits=0
	[26th September; C gain from Rochford]}

\index{Sweyne Park and Grange , Rochford@Sweyne Park \& Grange, \emph{Rochford}}

Resignation of Toby Mountain (Rochford).

\noindent
\begin{tabular*}{\columnwidth}{@{\extracolsep{\fill}} p{0.53\columnwidth} >{\itshape}l r @{\extracolsep{\fill}}}
Danielle Belton & C & 541\\
Lisa Newport & LD & 407\\
Jill Waight & Grn & 140\\
\end{tabular*}

\subsection*{Southend-on-Sea}

ForBritn = The For Britain Movement

\subsubsection*{Milton \hspace*{\fill}\nolinebreak[1]%
	\enspace\hspace*{\fill}
	\finalhyphendemerits=0
	[21st March]}

\index{Milton , Southend-on-Se@Milton, \emph{Southend-on-Sea}}

Death of Julian Ware-Lane (Lab).

\noindent
\begin{tabular*}{\columnwidth}{@{\extracolsep{\fill}} p{0.53\columnwidth} >{\itshape}l r @{\extracolsep{\fill}}}
Stephen George & Lab & 833\\
Garry Lowen & C & 528\\
Carol White & LD & 219\\
James Quail & ForBritn & 89\\
\end{tabular*}

\subsection*{Tendring}

\subsubsection*{St Osyth \hspace*{\fill}\nolinebreak[1]%
	\enspace\hspace*{\fill}
	\finalhyphendemerits=0
	[23rd May]}

\index{Saint Osyth , Tendring@St Osyth, \emph{Tendring}}

Ordinary election postponed from 2nd May; death of candidate Anita Bailey (C).

\noindent
\begin{tabular*}{\columnwidth}{@{\extracolsep{\fill}} p{0.53\columnwidth} >{\itshape}l r @{\extracolsep{\fill}}}
Michael Talbot & Ind & 856\\
John White & Ind & 850\\
Dawn Skeels & C & 437\\
Mark Skeels & C & 430\\
Tracey Osben & Lab & 177\\
\end{tabular*}

\subsection*{Thurrock}

Thurrock = Thurrock Independent

\subsubsection*{Aveley and Uplands \hspace*{\fill}\nolinebreak[1]%
	\enspace\hspace*{\fill}
	\finalhyphendemerits=0
	[21st March; C gain from UKIP]}

\index{Aveley and Uplands , Thurrock@Aveley \& Uplands, \emph{Thurrock}}

Resignation of Tim Aker MEP (Thurrock elected as UKIP).

\noindent
\begin{tabular*}{\columnwidth}{@{\extracolsep{\fill}} p{0.53\columnwidth} >{\itshape}l r @{\extracolsep{\fill}}}
David Van Day & C & 773\\
Alan Field & Thurrock & 551\\
Charles Curtis & Lab & 394\\
Tomas Pilvelis & LD & 55\\
\end{tabular*}

\subsubsection*{Chadwell St Mary \hspace*{\fill}\nolinebreak[1]%
	\enspace\hspace*{\fill}
	\finalhyphendemerits=0
	[2nd May]}

\index{Chadwell Saint Mary , Thurrock@Chadwell St Mary, \emph{Thurrock}}

Resignation of Barbara Rice (Lab).

Combined with the 2019 ordinary election; see page \pageref{ThurrockChadwellStMary} for the result.

\subsection*{Uttlesford}

R4U = Residents for Uttlesford

At the May 2019 ordinary election there was an unfilled vacancy in Saffron Walden Shire ward due to the resignation of Aisha Anjum (R4U).
\index{Saffron Walden Shire , Uttlesford@Saffron Walden Shire, \emph{Uttlesford}}

\section{Gloucestershire}

\subsection*{County Council}

\subsubsection*{Churchdown \hspace*{\fill}\nolinebreak[1]%
	\enspace\hspace*{\fill}
	\finalhyphendemerits=0
	[2nd May]}

\index{Churchdown , Gloucetsershire@Churchdown, \emph{Glos.}}

Death of Jack Williams (LD).

\noindent
\begin{tabular*}{\columnwidth}{@{\extracolsep{\fill}} p{0.53\columnwidth} >{\itshape}l r @{\extracolsep{\fill}}}
Benjamin Evans & LD & 1405\\
Graham Bocking & C & 811\\
Dick Bishop & Ind & 263\\
Cate Cody & Grn & 249\\
Robert McCormick & UKIP & 213\\
\end{tabular*}

\subsection*{Forest of Dean}

At the May 2019 ordinary election there was an unfilled vacancy in Coleford East ward due to the resignation of Martin Hill (UKIP).
\index{Coleford East , Forest of Dean@Coleford E., \emph{Forest of Dean}}

\subsubsection*{Newent and Taynton \hspace*{\fill}\nolinebreak[1]%
	\enspace\hspace*{\fill}
	\finalhyphendemerits=0
	[20th June]}

\index{Newent and Taynton , Forest of Dean@Newent \& Taynton, \emph{Forest of Dean}}

Ordinary election postponed from 2nd May; death of candidate David Humphreys (Grn).

\noindent
\begin{tabular*}{\columnwidth}{@{\extracolsep{\fill}} p{0.53\columnwidth} >{\itshape}l r @{\extracolsep{\fill}}}
Julia Gooch & Ind & 551\\
Gill Moseley & LD & 462\\
Vilnis Vesma & LD & 423\\
Len Lawton & C & 404\\
Eli Heathfield & C & 392\\
Bob Rhodes & Grn & 306\\
David Price & Grn & 282\\
Steve Martin & LD & 266\\
Johnny Back & Grn & 251\\
Nick Winter & C & 217\\
Edward Wood & Ind & 175\\
Simon Holmes & Ind & 170\\
Jean Sampson & Lab & 110\\
\end{tabular*}

\subsection*{Gloucester}

\subsubsection*{Barnwood \hspace*{\fill}\nolinebreak[1]%
	\enspace\hspace*{\fill}
	\finalhyphendemerits=0
	[25th July; LD gain from C]}

\index{Barnwood , Gloucester@Barnwood, \emph{Gloucester}}

Death of Lise Noakes (C).

\noindent
\begin{tabular*}{\columnwidth}{@{\extracolsep{\fill}} p{0.53\columnwidth} >{\itshape}l r @{\extracolsep{\fill}}}
Ashley Bowkett & LD & 676\\
Fred Ramsey & C & 496\\
Peter Sheehy & Brexit & 152\\
Chris Clee & Lab & 64\\
Jonathan Ingleby & Grn & 59\\
Matthew Young & UKIP & 6\\
\end{tabular*}

\subsubsection*{Podsmead \hspace*{\fill}\nolinebreak[1]%
	\enspace\hspace*{\fill}
	\finalhyphendemerits=0
	[25th July; LD gain from Lab]}

\index{Podsmead , Gloucester@Podsmead, \emph{Gloucester}}

Resignation of Deborah Smith (Lab).

\noindent
\begin{tabular*}{\columnwidth}{@{\extracolsep{\fill}} p{0.53\columnwidth} >{\itshape}l r @{\extracolsep{\fill}}}
Sebastian Field & LD & 203\\
Byron Davis & C & 200\\
Lisa Jevins & Lab & 122\\
Rob McCormick & Brexit & 111\\
Michael Byfield & Grn & 29\\
Simon Collins & UKIP & 11\\
\end{tabular*}

\subsection*{South Gloucestershire}

At the May 2019 ordinary election there was an unfilled vacancy in Patchway ward due to the resignation of Eve Orpen (Lab).
\index{Patchway , South Gloucestershire@Patchway, \emph{S. Glos.}}

\subsection*{Stroud}

\subsubsection*{Berkeley Vale \hspace*{\fill}\nolinebreak[1]%
	\enspace\hspace*{\fill}
	\finalhyphendemerits=0
	[28th February]}

\index{Berkeley Vale , Stroud@Berkeley Vale, \emph{Stroud}}

Death of Penny Wride (C).

\noindent
\begin{tabular*}{\columnwidth}{@{\extracolsep{\fill}} p{0.53\columnwidth} >{\itshape}l r @{\extracolsep{\fill}}}
Lindsay Green & C & 993\\
Liz Ashton & Lab & 686\\
Mike Stayte & LD & 231\\
Thomas Willetts & Grn & 82\\
\end{tabular*}

\section{Hampshire}

\subsection*{Basingstoke and Deane}

\subsubsection*{Bramley and Sherfield \hspace*{\fill}\nolinebreak[1]%
	\enspace\hspace*{\fill}
	\finalhyphendemerits=0
	[10th October]}

\index{Bramley and Sherfield , Basingstoke and Deane@Bramley \& Sherfield, \emph{Basingstoke \& Deane}}

Resignation of Venetia Rowland (C).

\noindent
\begin{tabular*}{\columnwidth}{@{\extracolsep{\fill}} p{0.53\columnwidth} >{\itshape}l r @{\extracolsep{\fill}}}
Chris Tomblin & Ind & 800\\
Angus Groom & C & 449\\
Joyce Bowyer & Ind & 150\\
\end{tabular*}

\subsection*{East Hampshire}

At the May 2019 ordinary election there was an unfilled vacancy in Petersfield St Mary's ward due to the resignation of Nicky Noble (C).
\index{Petersfield Saint Mary's , East Hampshire@Petersfield St Mary's, \emph{E. Hants.}}

\subsection*{Gosport}

BUSP = British Union and Sovereignty Party

\subsubsection*{Brockhurst \hspace*{\fill}\nolinebreak[1]%
	\enspace\hspace*{\fill}
	\finalhyphendemerits=0
	[30th May]}

\index{Brockhurst , Gosport@Brockhurst, \emph{Gosport}}

Death of Austin Hicks (LD).

\noindent
\begin{tabular*}{\columnwidth}{@{\extracolsep{\fill}} p{0.53\columnwidth} >{\itshape}l r @{\extracolsep{\fill}}}
Siobhan Mitchell & LD & 488\\
Pecs Uluviti & C & 214\\
Simon Bellord & BUSP & 165\\
Kirsty Smillie & Lab & 80\\
\end{tabular*}

\subsection*{Hart}

\subsubsection*{Hook \hspace*{\fill}\nolinebreak[1]%
	\enspace\hspace*{\fill}
	\finalhyphendemerits=0
	[2nd May]}

\index{Hook , Hart@Hook, \emph{Hart}}

Resignation of Mike Morris (C).

Combined with the 2019 ordinary election; see page \pageref{HartHook} for the result.

\subsection*{Havant}

At the May 2019 ordinary election there was an unfilled vacancy in Hart Plain ward due to the death of Gerald Shimbart (C).
\index{Hart Plain , Havant@Hart Plain, \emph{Havant}}

\subsubsection*{Purbrook \hspace*{\fill}\nolinebreak[1]%
	\enspace\hspace*{\fill}
	\finalhyphendemerits=0
	[2nd May]}

\index{Purbrook , Havant@Purbrook, \emph{Havant}}

Resignation of Adam Christie (C).

Combined with the 2019 ordinary election; see page \pageref{HavantPurbrook} for the result.

\subsection*{New Forest}

At the May 2019 ordinary election there was an unfilled vacancy in Totton Central ward due to the resignation of Brian Lucas (C).
\index{Totton Central , New Forest@Totton C., \emph{New Forest}}

\subsection*{Portsmouth}

\subsubsection*{Cosham \hspace*{\fill}\nolinebreak[1]%
	\enspace\hspace*{\fill}
	\finalhyphendemerits=0
	[2nd May]}

\index{Cosham , Portsmouth@Cosham, \emph{Portsmouth}}

Disqualification (non-attendance) of Jim Fleming (Ind elected as C).

Combined with the 2019 ordinary election; see page \pageref{PortsmouthCosham} for the result.

\subsection*{Rushmoor}

\subsubsection*{St Mark's \hspace*{\fill}\nolinebreak[1]%
	\enspace\hspace*{\fill}
	\finalhyphendemerits=0
	[12th September]}

\index{Saint Mark's , Rushmoor@St Mark's, \emph{Rushmoor}}

Resignation of Alain Dekker (LD).

\noindent
\begin{tabular*}{\columnwidth}{@{\extracolsep{\fill}} p{0.53\columnwidth} >{\itshape}l r @{\extracolsep{\fill}}}
Thomas Mitchell & LD & 687\\
Leon Hargreaves & C & 450\\
Carl Hewitt & Lab & 118\\
\end{tabular*}

\subsection*{Southampton}

IntegSO = Integrity Southampton

\subsubsection*{Coxford \hspace*{\fill}\nolinebreak[1]%
	\enspace\hspace*{\fill}
	\finalhyphendemerits=0
	[14th March; Lab gain from Ind]}

\index{Coxford , Southampton@Coxford, \emph{Southampton}}

Resignation of Keith Morrell (Ind).

\noindent
\begin{tabular*}{\columnwidth}{@{\extracolsep{\fill}} p{0.53\columnwidth} >{\itshape}l r @{\extracolsep{\fill}}}
Matthew Renyard & Lab & 668\\
Diana Galton & C & 529\\
Sam Chapman & LD & 450\\
Sue Atkins & SocAlt & 368\\
David Fletcher & IntegSO & 178\\
Richard McQuillan & Ind & 174\\
Derek Humber & UKIP & 123\\
Cara Sandys & Grn & 53\\
\end{tabular*}

\section{Herefordshire}

At the May 2019 ordinary election there was an unfilled vacancy in Ross North ward due to the death of Jenny Hyde (C).
\index{Ross North , Herefordshire@Ross N., \emph{Herefs.}}

IOCH = It's Our County (Herefordshire)

\subsubsection*{Ross North \hspace*{\fill}\nolinebreak[1]%
	\enspace\hspace*{\fill}
	\finalhyphendemerits=0
	[6th June]}

\index{Ross North , Herefordshire@Ross N., \emph{Herefs.}}

Ordinary election postponed from 2nd May; death of candidate Gareth Williams (UKIP).

\noindent
\begin{tabular*}{\columnwidth}{@{\extracolsep{\fill}} p{0.53\columnwidth} >{\itshape}l r @{\extracolsep{\fill}}}
Chris Bartrum & LD & 547\\
Nigel Gibbs & C & 136\\
Melvin Hodges & Lab & 45\\
\end{tabular*}

\subsubsection*{Whitecross \hspace*{\fill}\nolinebreak[1]%
	\enspace\hspace*{\fill}
	\finalhyphendemerits=0
	[11th July]}

\index{Whitecross , Herefordshire@Whitecross, \emph{Herefs.}}

Disqualification of Sue Boulter (IOCH): employed by the council.

\noindent
\begin{tabular*}{\columnwidth}{@{\extracolsep{\fill}} p{0.53\columnwidth} >{\itshape}l r @{\extracolsep{\fill}}}
Dave Boulter & IOCH & 304\\
Tricia Hales & LD & 141\\
Rob Williams & C & 56\\
\end{tabular*}

\section{Hertfordshire}

\subsection*{Broxbourne}

\subsubsection*{Broxbourne and Hoddesdon South \hspace*{\fill}\nolinebreak[1]%
	\enspace\hspace*{\fill}
	\finalhyphendemerits=0
	[2nd May]}

\index{Broxbourne and Hoddesdon South , Broxbourne@Broxbourne \& Hoddesdon S., \emph{Broxbourne}}

Resignation of Jim Metcalf (C).

Combined with the 2019 ordinary election; see page \pageref{BroxbourneBroxbourneHoddesdonSouth} for the result.

\subsection*{East Hertfordshire}

At the May 2019 ordinary election there were unfilled vacancies in Bishop's Stortford All Saints, and Mundens and Cottered wards due to the disqualification (non-attendance) of Stan Stainsby and the resignation of Paul Kenealy (both C) respectively.
\index{Bishop's Stortford All Saints , East Hertfordshire@Bishop's Stortford All SS., \emph{E. Herts.}}
\index{Mundens and Cottered , East Hertfordshire@Mundens \& Cottered, \emph{E. Herts.}}

\subsection*{St Albans}

\subsubsection*{Sopwell \hspace*{\fill}\nolinebreak[1]%
	\enspace\hspace*{\fill}
	\finalhyphendemerits=0
	[2nd May]}

\index{Sopwell , Saint Albans@Sopwell, \emph{St Albans}}

Resignation of Shakir Rahman (LD).

Combined with the 2019 ordinary election; see page \pageref{SopwellSaintAlbans} for the result.

\subsubsection*{Clarence \hspace*{\fill}\nolinebreak[1]%
	\enspace\hspace*{\fill}
	\finalhyphendemerits=0
	[3rd October]}

\index{Clarence , Saint Albans@Clarence, \emph{St Albans}}

Resignation of Caroline Brooke (LD).

\noindent
\begin{tabular*}{\columnwidth}{@{\extracolsep{\fill}} p{0.53\columnwidth} >{\itshape}l r @{\extracolsep{\fill}}}
Josie Madoc & LD & 1177\\
Don Deepthi & C & 314\\
Gary Chambers & Lab & 112\\
Rebecca Michel & Grn & 107\\
\end{tabular*}

\subsection*{Three Rivers}

\subsubsection*{Carpenders Park \hspace*{\fill}\nolinebreak[1]%
	\enspace\hspace*{\fill}
	\finalhyphendemerits=0
	[2nd May]}

\index{Carpenders Park , Three Rivers@Carpenders Park, \emph{Three Rivers}}

Resignation of David Coltman (C).

Combined with the 2019 ordinary election; see page \pageref{CarpendersParkThreeRivers} for the result.

\subsection*{Watford}

\subsubsection*{Meriden \hspace*{\fill}\nolinebreak[1]%
	\enspace\hspace*{\fill}
	\finalhyphendemerits=0
	[2nd May]}

\index{Meriden , Watford@Meriden, \emph{Watford}}

Resignation of Anthony Barton (LD).

Combined with the 2019 ordinary election; see page \pageref{WatfordMeriden} for the result.

\subsubsection*{Tudor \hspace*{\fill}\nolinebreak[1]%
	\enspace\hspace*{\fill}
	\finalhyphendemerits=0
	[10th October]}

\index{Tudor , Watford@Tudor, \emph{Watford}}

Resignation of Joe Fahmy (LD).

\noindent
\begin{tabular*}{\columnwidth}{@{\extracolsep{\fill}} p{0.53\columnwidth} >{\itshape}l r @{\extracolsep{\fill}}}
Bill Stanton & LD & 871\\
Carly Bishop & C & 490\\
Seamus Williams & Lab & 210\\
\end{tabular*}

\subsection*{Welwyn Hatfield}

\subsubsection*{Hatfield East \hspace*{\fill}\nolinebreak[1]%
	\enspace\hspace*{\fill}
	\finalhyphendemerits=0
	[2nd May]}

\index{Hatfield East , Welwyn Hatfield@Hatfield E., \emph{Welwyn Hatfield}}

Resignation of Kerstin Holman-Schmidt (C).

Combined with the 2019 ordinary election; see page \pageref{HatfieldEastWelwynHatfield} for the result.

\subsubsection*{Hatfield South West \hspace*{\fill}\nolinebreak[1]%
	\enspace\hspace*{\fill}
	\finalhyphendemerits=0
	[2nd May]}

\index{Hatfield South West , Welwyn Hatfield@Hatfield S.W., \emph{Welwyn Hatfield}}

Resignation of John Fitzpatrick (Lab).

Combined with the 2019 ordinary election; see page \pageref{HatfieldSouthWestWelwynHatfield} for the result.

\section{Isle of Wight}

IIN = Island Independent Network

\subsubsection*{Whippingham and Osborne \hspace*{\fill}\nolinebreak[1]%
	\enspace\hspace*{\fill}
	\finalhyphendemerits=0
	[20th June]}

\index{Whippingham and Osborne , Isle of Wight@Whippingham \& Osborne, \emph{Isle of Wight}}

Resignation of Julia Baker-Smith (Lab elected as Ind).

\noindent
\begin{tabular*}{\columnwidth}{@{\extracolsep{\fill}} p{0.53\columnwidth} >{\itshape}l r @{\extracolsep{\fill}}}
Stephen Hendry & C & 318\\
Julie Burridge & LD & 179\\
Michael Paler & Ind & 167\\
Luisa Hillard & Lab & 141\\
Karen Lucioni & IIN & 60\\
Rose Lynden-Bell & UKIP & 41\\
\end{tabular*}

\subsubsection*{Newport West \hspace*{\fill}\nolinebreak[1]%
	\enspace\hspace*{\fill}
	\finalhyphendemerits=0
	[12th December]}

\index{Newport West , Isle of Wight@Newport W., \emph{Isle of Wight}}

Resignation of Chris Whitehouse (C).

\noindent
\begin{tabular*}{\columnwidth}{@{\extracolsep{\fill}} p{0.53\columnwidth} >{\itshape}l r @{\extracolsep{\fill}}}
Richard Hollis & C & 605\\
Maria Villa Vine & Lab & 408\\
Joe Lever & Grn & 361\\
Nick Stewart & LD & 238\\
Stephen Reynolds & Ind & 47\\
Julian Harris & Ind & 36\\
W\end{tabular*}

\section{Kent}

\subsection*{County Council}

Swale = Swale Independents

\subsubsection*{Northfleet and Gravesend West \hspace*{\fill}\nolinebreak[1]%
	\enspace\hspace*{\fill}
	\finalhyphendemerits=0
	[2nd May]}

\index{Northfleet and Gravesend West , Kent@Northfleet \& Gravesend W., \emph{Kent}}

Resignation of Tan Dhesi MP (Lab).

\noindent
\begin{tabular*}{\columnwidth}{@{\extracolsep{\fill}} p{0.53\columnwidth} >{\itshape}l r @{\extracolsep{\fill}}}
John Burden & Lab & 3713\\
Jordan Meade & C & 2404\\
Emmanuel Feyisetan & UKIP & 1469\\
Marna Gilligan & Grn & 624\\
Ukonu Obasi & LD & 333\\
\end{tabular*}

\subsubsection*{Sittingbourne North \hspace*{\fill}\nolinebreak[1]%
	\enspace\hspace*{\fill}
	\finalhyphendemerits=0
	[2nd May; Swale gain from C]}

\index{Sittingbourne North , Kent@Sittingbourne N., \emph{Kent}}

Resignation of Sue Gent (C).

\noindent
\begin{tabular*}{\columnwidth}{@{\extracolsep{\fill}} p{0.53\columnwidth} >{\itshape}l r @{\extracolsep{\fill}}}
Jason Clinch & Swale & 1496\\
Tony Winckless & Lab & 1341\\
Sarah Aldridge & C & 1064\\
Alexander Stennings & LD & 284\\
\end{tabular*}

\subsection*{Ashford}

Ashford = Ashford Independent

\subsubsection*{Downs North \hspace*{\fill}\nolinebreak[1]%
	\enspace\hspace*{\fill}
	\finalhyphendemerits=0
	[18th July]}

\index{Downs North , Ashford@Downs N., \emph{Ashford}}

Death of Stephen Dehnel (C).

\noindent
\begin{tabular*}{\columnwidth}{@{\extracolsep{\fill}} p{0.53\columnwidth} >{\itshape}l r @{\extracolsep{\fill}}}
Charles Dehnel & C & 229\\
Geoff Meaden & Grn & 190\\
Adrian Gee-Turner & LD & 70\\
Rachael Carley & Ashford & 67\\
Philip Meads & UKIP & 22\\
Carly Ruppert Lingham & Lab & 17\\
Sarah Williams & Ind & 17\\
\end{tabular*}

\subsection*{Canterbury}

\subsubsection*{Chestfield \hspace*{\fill}\nolinebreak[1]%
	\enspace\hspace*{\fill}
	\finalhyphendemerits=0
	[19th September]}

\index{Chestfield , Canterbury@Chestfield, \emph{Canterbury}}

Death of Jenny Samper (C).

\noindent
\begin{tabular*}{\columnwidth}{@{\extracolsep{\fill}} p{0.53\columnwidth} >{\itshape}l r @{\extracolsep{\fill}}}
Ben Fitter-Harding & C & 728\\
Peter Old & LD & 562\\
Morag Warren & Lab & 140\\
Joe Egerton & Ind & 84\\
Nicole David & Grn & 68\\
\end{tabular*}

\subsection*{Dover}

At the May 2019 ordinary election there was an unfilled vacancy in Buckland ward due to the death of Simon Bannister (Lab).
\index{Buckland , Dover@Buckland, \emph{Dover}}

\subsubsection*{Guston, Kingsdown and St Margaret's-at-Cliffe \hspace*{\fill}\nolinebreak[1]%
	\enspace\hspace*{\fill}
	\finalhyphendemerits=0
	[12th December]}

\index{Guston, Kingsdown and Saint Margaret's-at-Cliffe , Dover@Guston, Kingsdown and St Margaret's-at-Cliffe, \emph{Dover}}

Resignation of Keith Morris (C).

\noindent
\begin{tabular*}{\columnwidth}{@{\extracolsep{\fill}} p{0.53\columnwidth} >{\itshape}l r @{\extracolsep{\fill}}}
Martin Bates & C & 2523\\
Eileen Rowbotham & Lab & 902\\
Sarah White-Gleave & Grn & 461\\
Robert Franklin & LD & 382\\
\end{tabular*}

\subsection*{Gravesham}

\subsubsection*{Westcourt \hspace*{\fill}\nolinebreak[1]%
	\enspace\hspace*{\fill}
	\finalhyphendemerits=0
	[17th October; C gain from Lab]}

\index{Westcourt , Gravesham@Westcourt, \emph{Gravesham}}

Death of Ruth Martin (Lab).

\noindent
\begin{tabular*}{\columnwidth}{@{\extracolsep{\fill}} p{0.53\columnwidth} >{\itshape}l r @{\extracolsep{\fill}}}
Helen Ashenden & C & 492\\
Lindsay Gordon & Lab & 314\\
Linda Talbot & UKIP & 116\\
Marna Gilligan & Grn & 60\\
\end{tabular*}

\subsection*{Medway}

At the May 2019 ordinary election there was an unfilled vacancy in Rainham North ward due to the death of David Carr (C).
\index{Rainham North , Medway@Rainham N., \emph{Medway}}

\subsection*{Thanet}

At the May 2019 ordinary election there was an unfilled vacancy in Margate Central ward due to the disqualification (indirectly employed by the council) of Ian Venables (Lab).
\index{Margate Central , Thanet@Margate C., \emph{Thanet}}

\subsection*{Tunbridge Wells}

\subsubsection*{Paddock Wood East \hspace*{\fill}\nolinebreak[1]%
	\enspace\hspace*{\fill}
	\finalhyphendemerits=0
	[2nd May]}

\index{Paddock Wood East , Tunbridge Wells@Paddock Wood E., \emph{Tunbridge Wells}}

Resignation of Allan Gooda (C).

Combined with the 2019 ordinary election; see page \pageref{TunbridgeWellsPaddockWoodEast} for the result.

\subsubsection*{Park \hspace*{\fill}\nolinebreak[1]%
	\enspace\hspace*{\fill}
	\finalhyphendemerits=0
	[2nd May]}

\index{Park , Tunbridge Wells@Park, \emph{Tunbridge Wells}}

Resignation of Peter Bulman (C).

Combined with the 2019 ordinary election; see page \pageref{TunbridgeWellsPark} for the result.

\subsubsection*{Culverden \hspace*{\fill}\nolinebreak[1]%
	\enspace\hspace*{\fill}
	\finalhyphendemerits=0
	[14th November; LD gain from C]}

\index{Culverden , Tunbridge Wells@Culverden, \emph{Tunbridge Wells}}

Death of Ronen Basu (C).

\noindent
\begin{tabular*}{\columnwidth}{@{\extracolsep{\fill}} p{0.53\columnwidth} >{\itshape}l r @{\extracolsep{\fill}}}
Justine Rutland & LD & 888\\
David Elliott & C & 474\\
Liz Orr & WEP & 193\\
Rachel Daly & TWells & 180\\
David Adams & Lab & 99\\
Aimee Taylor & Grn & 67\\
\end{tabular*}

\section{Lancashire}

\subsection*{Burnley}

BPIP = Burnley and Padiham Independent Party

\subsubsection*{Rosehill with Burnley Wood \hspace*{\fill}\nolinebreak[1]%
	\enspace\hspace*{\fill}
	\finalhyphendemerits=0
	[11th April]}

\index{Rosehill with Burnley Wood , Burnley@Rosehill with Burnley Wood, \emph{Burnley}}

Resignation of Christine White (BPIP elected as LD).

\noindent
\begin{tabular*}{\columnwidth}{@{\extracolsep{\fill}} p{0.53\columnwidth} >{\itshape}l r @{\extracolsep{\fill}}}
Peter McCann & LD & 341\\
Andy Devanney & Lab & 249\\
Paula Riley & BPIP & 154\\
Phil Chamberlain & C & 115\\
Victoria Alker & Grn & 51\\
\end{tabular*}

\subsection*{Chorley}

\subsubsection*{Eccleston and Mawdesley \hspace*{\fill}\nolinebreak[1]%
	\enspace\hspace*{\fill}
	\finalhyphendemerits=0
	[4th July]}

\index{Eccleston and Mawdesley , Chorley@Eccleston \& Mawdesley, \emph{Chorley}}

Death of Henry Caunce (C).

\noindent
\begin{tabular*}{\columnwidth}{@{\extracolsep{\fill}} p{0.53\columnwidth} >{\itshape}l r @{\extracolsep{\fill}}}
Val Caunce & C & 1050\\
Martin Fisher & Lab & 611\\
\end{tabular*}

\subsection*{Lancaster}

\subsubsection*{Overton \hspace*{\fill}\nolinebreak[1]%
	\enspace\hspace*{\fill}
	\finalhyphendemerits=0
	[12th December]}

\index{Overton , Lancaster@Overton, \emph{Lancaster}}

Resignation of Michael Smith (C).

\noindent
\begin{tabular*}{\columnwidth}{@{\extracolsep{\fill}} p{0.53\columnwidth} >{\itshape}l r @{\extracolsep{\fill}}}
Andrew Gardiner & C & -\\
Tom Porter & Lab & -\\
Amy Stanning & LD & -\\
\end{tabular*}

\subsection*{Ribble Valley}

At the May 2019 ordinary election there was an unfilled vacancy in Waddington and West Bradford ward due to the resignation of Paul Elms (Democratic Conservative elected as C).
\index{Waddington and West Bradford , Ribble Valley@Waddington \& West Bradford, \emph{Ribble Valley}}

\subsection*{Rossendale}

At the May 2019 ordinary election there was an unfilled vacancy in Cribden ward due to the resignation of Janet Graham (C).
\index{Cribden , Rossendale@Cribden, \emph{Rossendale}}

\subsection*{South Ribble}

At the May 2019 ordinary election there were unfilled vacancies in Bamber Bridge East and Middleforth wards due to the deaths of Dave Watts and David Wooldridge (both Lab) respectively.
\index{Bamber Bridge East , South Ribble@Bamber Bridge E., \emph{S. Ribble}}
\index{Middleforth , South Ribble@Middleforth, \emph{S. Ribble}}

\subsubsection*{Farington West \hspace*{\fill}\nolinebreak[1]%
	\enspace\hspace*{\fill}
	\finalhyphendemerits=0
	[20th June]}

\index{Farington West , South Ribble@Farington W., \emph{S. Ribble}}

Ordinary election postponed from 2nd May: death of outgoing councillor Graham Walton (C) who was seeking re-election.

\noindent
\begin{tabular*}{\columnwidth}{@{\extracolsep{\fill}} p{0.53\columnwidth} >{\itshape}l r @{\extracolsep{\fill}}}
Karen Walton & C & 536\\
Stephen Thurlbourn & C & 497\\
Emma Buchanan & Lab & 246\\
Ryan Hamilton & Lab & 171\\
Judith Davidson & LD & 114\\
Alison Hesketh-Holt & LD & 91\\
\end{tabular*}

\subsubsection*{Coupe Green and Gregson Lane \hspace*{\fill}\nolinebreak[1]%
	\enspace\hspace*{\fill}
	\finalhyphendemerits=0
	[24th October]}

\index{Coupe Green and Gregson Lane , South Ribble@Coupe Green \& Gregson Lane, \emph{S. Ribble}}

Resignation of Sarah Whittaker (C).

\noindent
\begin{tabular*}{\columnwidth}{@{\extracolsep{\fill}} p{0.53\columnwidth} >{\itshape}l r @{\extracolsep{\fill}}}
Gareth Watson & C & 437\\
Graham Dixon & Ind & 343\\
Stephanie Portersmith & LD & 110\\
\end{tabular*}

\subsection*{West Lancashire}

\subsubsection*{Birch Green \hspace*{\fill}\nolinebreak[1]%
	\enspace\hspace*{\fill}
	\finalhyphendemerits=0
	[21st November]}

\index{Birch Green , West Lancashire@Birch Green, \emph{W. Lancs.}}

Resignation of Claire Cooper (Lab).

\noindent
\begin{tabular*}{\columnwidth}{@{\extracolsep{\fill}} p{0.53\columnwidth} >{\itshape}l r @{\extracolsep{\fill}}}
Sue Gregson & Lab & 390\\
Andrew Taylor & SkemInd & 191\\
George Rear & C & 60\\
\end{tabular*}

\section{Leicestershire}

\subsection*{County Council}

\subsubsection*{Cosby and Countesthorpe \hspace*{\fill}\nolinebreak[1]%
	\enspace\hspace*{\fill}
	\finalhyphendemerits=0
	[12th December]}

\index{Cosby and Countesthorpe , Leicestershire@Cosby \& Countesthorpe, \emph{Leics.}}

Death of David Jennings (C).

\noindent
\begin{tabular*}{\columnwidth}{@{\extracolsep{\fill}} p{0.53\columnwidth} >{\itshape}l r @{\extracolsep{\fill}}}
Royston Bayliss & LD & -\\
Sandra Parkinson & Lab & -\\
Les Phillimore & C & -\\
Christiane Startin-Lorent & Grn & -\\
\end{tabular*}

\subsection*{Blaby}

At the May 2019 ordinary election there was an unfilled vacancy in Ellis ward due to the resignation of Scarlet Breckon (C).
\index{Ellis , Blaby@Ellis, \emph{Blaby}}

\subsection*{Charnwood}

\subsubsection*{Syston West \hspace*{\fill}\nolinebreak[1]%
	\enspace\hspace*{\fill}
	\finalhyphendemerits=0
	[3rd October]}

\index{Syston West , Charnwood@Syston W., \emph{Charnwood}}

Resignation of Eric Vardy (C).

\noindent
\begin{tabular*}{\columnwidth}{@{\extracolsep{\fill}} p{0.53\columnwidth} >{\itshape}l r @{\extracolsep{\fill}}}
Sue Gerrard & C & 406\\
Matthew Wise & Grn & 389\\
Sharon Brown & Lab & 114\\
\end{tabular*}

\subsection*{Melton}

At the May 2019 ordinary election there was an unfilled vacancy in Melton Egerton ward due to the death of Michael Blase (Lab).
\index{Melton Egerton , Melton@Melton Egerton, \emph{Melton}}

\subsection*{North West Leicestershire}

At the May 2019 ordinary election there was an unfilled vacancy in Ibstock East ward due to the resignation of Felix Fenning (Lab).
\index{Ibstock East , North West Leicestershire@Ibstock E., \emph{N.W. Leics.}}

\subsection*{Oadby and Wigston}

At the May 2019 ordinary election there was an unfilled vacancy in Oadby Grange ward due to the death of Bob Fahey (C).
\index{Oadby Grange , Oadby and Wigston@Oadby Grange, \emph{Oadby \& Wigston}}

\section{Lincolnshire}

\subsection*{Boston}

\subsubsection*{Kirton and Frampton \hspace*{\fill}\nolinebreak[1]%
	\enspace\hspace*{\fill}
	\finalhyphendemerits=0
	[12th December]}

\index{Kirton and Frampton , Boston@Kirton \& Frampton, \emph{Boston}}

Resignation of Shaun Blackman (C).

\noindent
\begin{tabular*}{\columnwidth}{@{\extracolsep{\fill}} p{0.53\columnwidth} >{\itshape}l r @{\extracolsep{\fill}}}
David Brown & C & 2018\\
Lorraine O'Connor & Ind & 453\\
Alan Taylor & LD & 415\\
\end{tabular*}

\subsubsection*{Skirbeck \hspace*{\fill}\nolinebreak[1]%
	\enspace\hspace*{\fill}
	\finalhyphendemerits=0
	[12th December]}

\index{Skirbeck , Boston@Skirbeck, \emph{Boston}}

Resignation of Alistair Arundell (C).

\noindent
\begin{tabular*}{\columnwidth}{@{\extracolsep{\fill}} p{0.53\columnwidth} >{\itshape}l r @{\extracolsep{\fill}}}
Martin Howard & C & 955\\
Jackie Barton & Lab & 437\\
Don Jenkins & Ind & 195\\
Sue Ransome & Ind & 112\\
Jason Stevenson & LD & 100\\
Christopher Moore & BlueRevn & 62\\
\end{tabular*}

\subsection*{Lincoln}

\subsubsection*{Witham \hspace*{\fill}\nolinebreak[1]%
	\enspace\hspace*{\fill}
	\finalhyphendemerits=0
	[12th December]}

\index{Witham , Lincoln@Witham, \emph{Lincoln}}

Resignation of Keith Weaver (C).

\noindent
\begin{tabular*}{\columnwidth}{@{\extracolsep{\fill}} p{0.53\columnwidth} >{\itshape}l r @{\extracolsep{\fill}}}
Bill Mara & C & -\\
Charles Parker & LD & -\\
Michele Servaud & Grn & -\\
Calum Watt & Lab & -\\
\end{tabular*}

\subsection*{North East Lincolnshire}

\subsubsection*{South \hspace*{\fill}\nolinebreak[1]%
	\enspace\hspace*{\fill}
	\finalhyphendemerits=0
	[2nd May]}

\index{South , North East Lincolnshire@South, \emph{N.E. Lincs.}}

Resignation of Ray Oxby (Lab).

Combined with the 2019 ordinary election; see page \pageref{NorthEastLincolnshireSouth} for the result.

\subsection*{North Kesteven}

LincsInd = Lincolnshire Independents

\subsubsection*{Billinghay, Martin and North Kyme \hspace*{\fill}\nolinebreak[1]%
	\enspace\hspace*{\fill}
	\finalhyphendemerits=0
	[13th June]}

\index{Billinghay, Martin and North Kyme , North Kesteven@Billinghay, Martin \& North Kyme, \emph{N. Kesteven}}

Insufficient nominations in the 2019 ordinary election.

\noindent
\begin{tabular*}{\columnwidth}{@{\extracolsep{\fill}} p{0.53\columnwidth} >{\itshape}l r @{\extracolsep{\fill}}}
Amanda Sanderson & C & 320\\
Tracy Giannisi & LincsInd & 160\\
Robert Greetham & Ind & 72\\
Garry Winterton & LD & 57\\
Matt Newman & Lab & 43\\
Steven Shanahan-Kluth & Ind & 9\\
\end{tabular*}

\subsection*{West Lindsey}

\subsubsection*{Torksey \hspace*{\fill}\nolinebreak[1]%
	\enspace\hspace*{\fill}
	\finalhyphendemerits=0
	[24th October]}

\index{Torksey , West Lindsey@Torksey, \emph{W. Lindsey}}

Resignation of Stuart Kinch (C).

\noindent
\begin{tabular*}{\columnwidth}{@{\extracolsep{\fill}} p{0.53\columnwidth} >{\itshape}l r @{\extracolsep{\fill}}}
Jayne Ellis & C & 378\\
Noel Mullally & LD & 346\\
Nick Pearson & Brexit & 299\\
Perry Smith & Lab & 37\\
\end{tabular*}

\section{Norfolk}

\subsection*{County Council}

\subsubsection*{Wroxham \hspace*{\fill}\nolinebreak[1]%
	\enspace\hspace*{\fill}
	\finalhyphendemerits=0
	[4th April]}

\index{Wroxham , Norfolk@Wroxham, \emph{Norfolk}}

Resignation of Tom Garrod (C).

\noindent
\begin{tabular*}{\columnwidth}{@{\extracolsep{\fill}} p{0.53\columnwidth} >{\itshape}l r @{\extracolsep{\fill}}}
Fran Whymark & C & 922\\
Stephen Heard & LD & 395\\
Jan Davis & Grn & 174\\
Julia Wheeler & Lab & 163\\
\end{tabular*}

\subsection*{Breckland}

At the May 2019 ordinary election there was an unfilled vacancy in Thetford Castle ward due to the death of John Newton (UKIP).
\index{Thetford Castle , Breckland@Thetford Castle, \emph{Breckland}}

\subsection*{Great Yarmouth}

At the May 2019 ordinary election there was an unfilled vacancy in Ormesby ward due to the death of Charles Reynolds (C).
\index{Ormesby , Great Yarmouth@Ormesby, \emph{Great Yarmouth}}

\subsection*{King's Lynn and West Norfolk}

At the May 2019 ordinary election there was an unfilled vacancy in West Winch ward due to the disqualification (non-attendance) of Baljinder Anota (C).
\index{West Winch , King's Lynn and West Norfolk@Didcot S., \emph{King's Lynn \& W. Norfolk}}

\subsubsection*{Upwell and Delph \hspace*{\fill}\nolinebreak[1]%
	\enspace\hspace*{\fill}
	\finalhyphendemerits=0
	[12th December]}

\index{Upwell and Delph , King's Lynn and West Norfolk@Upwell \& Delph, \emph{King's Lynn \& W. Norfolk}}

Resignation of David Pope (Ind).

\noindent
\begin{tabular*}{\columnwidth}{@{\extracolsep{\fill}} p{0.53\columnwidth} >{\itshape}l r @{\extracolsep{\fill}}}
Stewart Dickson & Lab & -\\
Terry Hipsey & Ind & -\\
Vivienne Spikings & C & -\\
\end{tabular*}

\subsection*{North Norfolk}

At the May 2019 ordinary election there was an unfilled vacancy in Mundesley ward due to the death of Wyndham Northam (C).
\index{Mundesley , North Norfolk@Mundesley, \emph{N. Norfolk}}

\subsubsection*{Sheringham North \hspace*{\fill}\nolinebreak[1]%
	\enspace\hspace*{\fill}
	\finalhyphendemerits=0
	[28th November]}

\index{Sheringham North , North Norfolk@Sheringham N., \emph{N. Norfolk}}

Resignation of Brian Hannah (LD).

\noindent
\begin{tabular*}{\columnwidth}{@{\extracolsep{\fill}} p{0.53\columnwidth} >{\itshape}l r @{\extracolsep{\fill}}}
Liz Withington & LD & 364\\
Richard Shepherd & C & 323\\
Sue Brisbane & Lab & 65\\
\end{tabular*}

\section{North Yorkshire}

\subsection*{County Council}

\subsubsection*{Upper Dales \hspace*{\fill}\nolinebreak[1]%
	\enspace\hspace*{\fill}
	\finalhyphendemerits=0
	[17th October; C gain from Ind]}

\index{Upper Dales , North Yorkshire@Upper Dales, \emph{N. Yorks.}}

Death of John Blackie (Ind).

\noindent
\begin{tabular*}{\columnwidth}{@{\extracolsep{\fill}} p{0.53\columnwidth} >{\itshape}l r @{\extracolsep{\fill}}}
Yvonne Peacock & C & 884\\
Jill McMullon & Ind & 741\\
Simon Crosby & LD & 204\\
Kevin Foster & Grn & 107\\
\end{tabular*}

\subsection*{Craven}

\subsubsection*{Upper Wharfedale \hspace*{\fill}\nolinebreak[1]%
	\enspace\hspace*{\fill}
	\finalhyphendemerits=0
	[2nd May]}

\index{Upper Wharfedale , Craven@Upper Wharfedale, \emph{Craven}}

Resignation of Tanya Graham (C).

\noindent
\begin{tabular*}{\columnwidth}{@{\extracolsep{\fill}} p{0.53\columnwidth} >{\itshape}l r @{\extracolsep{\fill}}}
Sue Metcalfe & C & 480\\
Anna Craven & Grn & 177\\
Siân Wheal & LD & 59\\
Virpi Kettu & Lab & 42\\
\end{tabular*}

\subsection*{Middlesbrough}

\subsubsection*{Park End and Beckfield \hspace*{\fill}\nolinebreak[1]%
	\enspace\hspace*{\fill}
	\finalhyphendemerits=0
	[4th July]}

\index{Park End and Beckfield , Middlesbrough@Park End \& Beckfield, \emph{Middlesbrough}}

Resignation of Jan Mohan (Ind).

\noindent
\begin{tabular*}{\columnwidth}{@{\extracolsep{\fill}} p{0.53\columnwidth} >{\itshape}l r @{\extracolsep{\fill}}}
Stephen Hill & Ind & 511\\
Steven James & Ind & 303\\
Paul McGrath & Lab & 115\\
Val Beadnall & C & 23\\
Ian Jones & LD & 13\\
\end{tabular*}

\subsection*{Richmondshire}

\subsubsection*{Hawes, High Abbotside and Upper Swaledale \hspace*{\fill}\nolinebreak[1]%
	\enspace\hspace*{\fill}
	\finalhyphendemerits=0
	[17th October]}

\index{Hawes, High Abbotside and Upper Swaledale , Richmondshire@Hawes, High Abbotside \& Upper Swaledale, \emph{Richmondshire}}

Death of John Blackie (Ind).

\noindent
\begin{tabular*}{\columnwidth}{@{\extracolsep{\fill}} p{0.53\columnwidth} >{\itshape}l r @{\extracolsep{\fill}}}
Jill McMullon & Ind & 409\\
Pat Kirkbride & C & 231\\
Margaret Lowndes & Grn & 57\\
\end{tabular*}

\section{Northamptonshire}

\subsection*{County Council}

\subsubsection*{Oundle \hspace*{\fill}\nolinebreak[1]%
	\enspace\hspace*{\fill}
	\finalhyphendemerits=0
	[21st February]}

\index{Oundle , Northamptonshire@Oundle, \emph{Northamptonshire}}

Resignation of Heather Smith (Ind elected as C).

\noindent
\begin{tabular*}{\columnwidth}{@{\extracolsep{\fill}} p{0.53\columnwidth} >{\itshape}l r @{\extracolsep{\fill}}}
Annabel de Capell Brooke & C & 1864\\
Marc Folgate & LD & 1276\\
Harry James & Lab & 403\\
Allan Shipham & UKIP & 89\\
\end{tabular*}

\subsection*{Corby}

\subsubsection*{Beanfield \hspace*{\fill}\nolinebreak[1]%
	\enspace\hspace*{\fill}
	\finalhyphendemerits=0
	[10th October]}

\index{Beanfield , Corby@Beanfield, \emph{Corby}}

Death of Mary Butcher (Lab).

\noindent
\begin{tabular*}{\columnwidth}{@{\extracolsep{\fill}} p{0.53\columnwidth} >{\itshape}l r @{\extracolsep{\fill}}}
Alison Dalziel & Lab & 818\\
Roy Boyd & C & 497\\
Chris Stanbra & LD & 147\\
\end{tabular*}

\subsection*{Daventry}

\subsubsection*{Brixworth \hspace*{\fill}\nolinebreak[1]%
	\enspace\hspace*{\fill}
	\finalhyphendemerits=0
	[18th July; LD gain from C]}

\index{Brixworth , Daventry@Brixworth, \emph{Daventry}}

Resignation of Fabienne Fraser-Allen (C).

\noindent
\begin{tabular*}{\columnwidth}{@{\extracolsep{\fill}} p{0.53\columnwidth} >{\itshape}l r @{\extracolsep{\fill}}}
Jonathan Harris & LD & 817\\
Lauryn Harrington-Carter & C & 615\\
Stuart Coe & Lab & 218\\
\end{tabular*}

\subsubsection*{Abbey North \hspace*{\fill}\nolinebreak[1]%
	\enspace\hspace*{\fill}
	\finalhyphendemerits=0
	[24th October; C gain from Lab]}

\index{Abbey North , Daventry@Abbey N., \emph{Daventry}}

Resignation of Aiden Ramsey (Lab).

\noindent
\begin{tabular*}{\columnwidth}{@{\extracolsep{\fill}} p{0.53\columnwidth} >{\itshape}l r @{\extracolsep{\fill}}}
Lauryn Harrington-Carter & C & 376\\
Alan Knape & LD & 280\\
Emily Carter & Lab & 262\\
\end{tabular*}

\subsection*{East Northamptonshire}

\subsubsection*{Irthlingborough Waterloo \hspace*{\fill}\nolinebreak[1]%
	\enspace\hspace*{\fill}
	\finalhyphendemerits=0
	[8th August]}

\index{Irthlingborough Waterloo , East Northamptonshire@Irthlingborough Waterloo, \emph{E. Northants.}}

Resignation of Marika Hillson (C).

\noindent
\begin{tabular*}{\columnwidth}{@{\extracolsep{\fill}} p{0.53\columnwidth} >{\itshape}l r @{\extracolsep{\fill}}}
Lee Wilkes & C & 542\\
Caroline Cross & Lab & 478\\
\end{tabular*}

\subsubsection*{Higham Ferrers Chichele \hspace*{\fill}\nolinebreak[1]%
	\enspace\hspace*{\fill}
	\finalhyphendemerits=0
	[12th December]}

\index{Higham Ferrers Chichele , East Northamptonshire@Higham Ferrers Chichele, \emph{E. Northants.}}

Resignation of Anna Sauntson (C).

\noindent
\begin{tabular*}{\columnwidth}{@{\extracolsep{\fill}} p{0.53\columnwidth} >{\itshape}l r @{\extracolsep{\fill}}}
Bert Jackson & C & 1379\\
Suzanna Austin & LD & 800\\
\end{tabular*}

\subsubsection*{Higham Ferrers Lancaster \hspace*{\fill}\nolinebreak[1]%
	\enspace\hspace*{\fill}
	\finalhyphendemerits=0
	[12th December]}

\index{Higham Ferrers Lancaster , East Northamptonshire@Higham Ferrers Lancaster, \emph{E. Northants.}}

Resignation of Pam Whiting (C).

\noindent
\begin{tabular*}{\columnwidth}{@{\extracolsep{\fill}} p{0.53\columnwidth} >{\itshape}l r @{\extracolsep{\fill}}}
Peter Tomas & C & 1531\\
Simon Baylis & LD & 913\\
\end{tabular*}

\subsection*{Kettering}

\subsubsection*{Desborough St Giles \hspace*{\fill}\nolinebreak[1]%
	\enspace\hspace*{\fill}
	\finalhyphendemerits=0
	[12th December]}

\index{Desborough Saint Giles , Kettering@Desborough St Giles, \emph{Kettering}}

Death of David Soans (C).

\noindent
\begin{tabular*}{\columnwidth}{@{\extracolsep{\fill}} p{0.53\columnwidth} >{\itshape}l r @{\extracolsep{\fill}}}
Jim French & C & 1475\\
Phil Sawford & Lab & 1007\\
Alan Window & LD & 196\\
Daz Dell & Grn & 115\\
\end{tabular*}

\subsection*{South Northamptonshire}

\subsubsection*{Middleton Cheney \hspace*{\fill}\nolinebreak[1]%
	\enspace\hspace*{\fill}
	\finalhyphendemerits=0
	[12th September; LD gain from C]}

\index{Middleton Cheney , South Northamptonshire@Middleton Cheney, \emph{S. Northants.}}

Resignation of Jonathan Riley (C).

\noindent
\begin{tabular*}{\columnwidth}{@{\extracolsep{\fill}} p{0.53\columnwidth} >{\itshape}l r @{\extracolsep{\fill}}}
Mark Allen & LD & 384\\
Alison Eastwood & C & 345\\
Adam Sear & Grn & 89\\
Arthur Greaves & Lab & 59\\
\end{tabular*}

\subsection*{Wellingborough}

\subsubsection*{Finedon \hspace*{\fill}\nolinebreak[1]%
	\enspace\hspace*{\fill}
	\finalhyphendemerits=0
	[12th September]}

\index{Finedon , Wellingborough@Finedon, \emph{Wellingborough}}

Resignation of Barbara Bailey (C).

\noindent
\begin{tabular*}{\columnwidth}{@{\extracolsep{\fill}} p{0.53\columnwidth} >{\itshape}l r @{\extracolsep{\fill}}}
Andrew Weatherill & C & 547\\
Laurence Harper & Ind & 227\\
Marion Turner-Hawes & Grn & 134\\
Isobel Stevenson & Lab & 76\\
Chris Nelson & LD & 64\\
\end{tabular*}

\section{Northumberland}

\subsubsection*{Holywell \hspace*{\fill}\nolinebreak[1]%
	\enspace\hspace*{\fill}
	\finalhyphendemerits=0
	[2nd May]}

\index{Holywell , Northumberland@Holywell, \emph{Northumberland}}

Death of Bernard Pidcock (Lab).

\noindent
\begin{tabular*}{\columnwidth}{@{\extracolsep{\fill}} p{0.53\columnwidth} >{\itshape}l r @{\extracolsep{\fill}}}
Leslie Bowman & Lab & 916\\
Maureen Levy & C & 510\\
Anita Romer & LD & 164\\
\end{tabular*}

\section{Nottinghamshire}

\subsection*{Broxtowe}

\subsubsection*{Stapleford South East \hspace*{\fill}\nolinebreak[1]%
	\enspace\hspace*{\fill}
	\finalhyphendemerits=0
	[13th June; 2 LD gains from C]}

\index{Stapleford South East , Broxtowe@Stapeford S.E., \emph{Broxtowe}}

Ordinary election postponed from 2nd May: death of outgoing councillor Chris Rice (C) who was seeking re-election.

\noindent
\begin{tabular*}{\columnwidth}{@{\extracolsep{\fill}} p{0.53\columnwidth} >{\itshape}l r @{\extracolsep{\fill}}}
Tim Hallam & LD & 559\\
David Grindell & LD & 538\\
John Doddy & C & 380\\
Adam Stockwell & C & 331\\
Sue Paterson & Lab & 322\\
Eleanor Allan & Lab & 290\\
\end{tabular*}

\subsection*{Gedling}

At the May 2019 ordinary election there was an unfilled vacancy in Bestwood St Albans ward due to the death of Denis Beeston (Lab).
\index{Bestwood Saint Albans , Gedling@Bestwood St Albans, \emph{Gedling}}

\subsection*{Mansfield}

MIF = Mansfield Independent Forum

At the May 2019 ordinary election there was an unfilled vacancy in Grange Farm ward due to the disqualification (non-attendance) of Ron Jelley (MIF).
\index{Grange Farm , Mansfield@Grange Farm, \emph{Mansfield}}

\subsubsection*{Sandhurst \hspace*{\fill}\nolinebreak[1]%
	\enspace\hspace*{\fill}
	\finalhyphendemerits=0
	[27th June; MIF gain from Lab]}

\index{Sandhurst , Mansfield@Sandhurst, \emph{Mansfield}}

Election of Andy Abrahams (Lab) as Mayor of Mansfield.

\noindent
\begin{tabular*}{\columnwidth}{@{\extracolsep{\fill}} p{0.53\columnwidth} >{\itshape}l r @{\extracolsep{\fill}}}
Dave Saunders & MIF & 227\\
Michelle Swordy & Lab & 177\\
Cathryn Fletcher & C & 71\\
Daniel Hartshorn & UKIP & 56\\
\end{tabular*}

\section{Oxfordshire}

\subsection*{County Council}

\subsubsection*{Wallingford \hspace*{\fill}\nolinebreak[1]%
	\enspace\hspace*{\fill}
	\finalhyphendemerits=0
	[28th November; Grn gain from Ind]}

\index{Wallingford , Oxfordshire@Wallingford, \emph{Oxon.}}

Resignation of Lynda Atkins (Ind).

\noindent
\begin{tabular*}{\columnwidth}{@{\extracolsep{\fill}} p{0.53\columnwidth} >{\itshape}l r @{\extracolsep{\fill}}}
Pete Sudbury & Grn & 998\\
Adrian Lloyd & C & 755\\
Elaine Hornsby & Ind & 483\\
George Kneeshaw & Lab & 202\\
\end{tabular*}

\subsection*{Cherwell}

\subsubsection*{Kidlington West \hspace*{\fill}\nolinebreak[1]%
	\enspace\hspace*{\fill}
	\finalhyphendemerits=0
	[2nd May]}

\index{Kidlington West , Cherwell@Kidlington W., \emph{Cherwell}}

Resignation of Alaric Rose (LD).

Combined with the 2019 ordinary election; see page \pageref{CherwellKidlingtonWest} for the result.

\subsection*{South Oxfordshire}

At the May 2019 ordinary election there were unfilled vacancies in Didcot South, and Kidmore End and Whitchurch wards due to the resignations of Anthony Nash and Robert Simister (both C) respectively.
\index{Didcot South , South Oxfordshire@Didcot S., \emph{S. Oxon.}}
\index{Kidmore End and Whitchurch , South Oxfordshire@Kidmore End \& Whitchurch, \emph{S. Oxon.}}

\section{Rutland}

At the May 2019 ordinary election there were unfilled vacancies in Cottesmore and Oakham South West wards due to the resignations of Andrew Stewart (C) and Richard Alderman (Ind) respectively.
\index{Cottesmore , Rutland@Cottesmore, \emph{Rutland}}
\index{Oakham South West , Rutland@Oakham S.W., \emph{Rutland}}

\subsubsection*{Ryhall and Casterton \hspace*{\fill}\nolinebreak[1]%
	\enspace\hspace*{\fill}
	\finalhyphendemerits=0
	[12th September; C gain from Ind]}

\index{Ryhall and Casterton , Rutland@Ryhall \& Casterton, \emph{Rutland}}

Disqualification (failure to sign acceptance of office) of Chris Parsons (Ind).

\noindent
\begin{tabular*}{\columnwidth}{@{\extracolsep{\fill}} p{0.53\columnwidth} >{\itshape}l r @{\extracolsep{\fill}}}
Richard Coleman & C & 357\\
Beverley Wrigley-Pheasant & LD & 156\\
Steve Fay & Grn & 121\\
\end{tabular*}

\section{Shropshire}

\subsection*{Shropshire}

\subsubsection*{Belle Vue \hspace*{\fill}\nolinebreak[1]%
	\enspace\hspace*{\fill}
	\finalhyphendemerits=0
	[25th April]}

\index{Belle Vue , Shropshire@Belle Vue, \emph{Shropshire}}

Resignation of Harry Taylor (Lab).

\noindent
\begin{tabular*}{\columnwidth}{@{\extracolsep{\fill}} p{0.53\columnwidth} >{\itshape}l r @{\extracolsep{\fill}}}
Mary Halliday & Lab & 603\\
James McLeod & LD & 403\\
Ross George & C & 152\\
Dave Latham & Grn & 65\\
Bob Oakley & UKIP & 58\\
\end{tabular*}

\subsubsection*{Meole \hspace*{\fill}\nolinebreak[1]%
	\enspace\hspace*{\fill}
	\finalhyphendemerits=0
	[15th August]}

\index{Meole , Shropshire@Meole, \emph{Shropshire}}

Resignation of Nic Laurens (C).

\noindent
\begin{tabular*}{\columnwidth}{@{\extracolsep{\fill}} p{0.53\columnwidth} >{\itshape}l r @{\extracolsep{\fill}}}
Gewndoline Burgess & C & 438\\
Adam Fejfer & LD & 309\\
Darrell Morris & Lab & 286\\
Emma Bullard & Grn & 131\\
\end{tabular*}

\subsubsection*{Bishop's Castle \hspace*{\fill}\nolinebreak[1]%
	\enspace\hspace*{\fill}
	\finalhyphendemerits=0
	[12th September]}

\index{Bishop's Castle , Shropshire@Bishop's Castle, \emph{Shropshire}}

Resignation of Jonathan Keeley (LD).

\noindent
\begin{tabular*}{\columnwidth}{@{\extracolsep{\fill}} p{0.53\columnwidth} >{\itshape}l r @{\extracolsep{\fill}}}
Ruth Houghton & LD & 838\\
Edward Thompson & C & 229\\
Andy Stelman & Lab & 107\\
\end{tabular*}

\section{Somerset}

\subsection*{South Somerset}

At the May 2019 ordinary election there was an unfilled vacancy in Ilminster ward due to the resignation of Carol Goodall (LD).
\index{Ilminster , South Somerset@Ilminster, \emph{S. Somerset}}

\subsection*{Taunton Deane}

At the abolition of Taunton Deane council in April 2019 there was an unfilled vacancy in Taunton Pyrland and Rowbarton, and Wellington Rockwell Green and West wards due to the deaths of Tom Davies and Bob Bowrah (both C) respectively.
\index{Taunton Pyrland and Rowbarton , Taunton Deane@Taunton Pyrland \& Rowbarton, \emph{Taunton Deane}}
\index{Wellington Rockwell Green and West , Taunton Deane@Wellington Rochwell Green \& W., \emph{Taunton Deane}}

\subsection*{Somerset West and Taunton}

\subsubsection*{Vivary \hspace*{\fill}\nolinebreak[1]%
	\enspace\hspace*{\fill}
	\finalhyphendemerits=0
	[19th September; LD gain from C]}

\index{Vivary , Somerset West and Taunton@Vivay, \emph{Somerset W. \& Taunton}}

Resignation of Catherine Herbert (C).

\noindent
\begin{tabular*}{\columnwidth}{@{\extracolsep{\fill}} p{0.53\columnwidth} >{\itshape}l r @{\extracolsep{\fill}}}
Derek Perry & LD & 648\\
Sharon Fussell & C & 307\\
Neil Rudram & Ind & 155\\
Robert Noakes & Lab & 32\\
Marguerite Paffard & Grn & 30\\
\end{tabular*}

\subsubsection*{Norton Fitzwarren and Staplegrove \hspace*{\fill}\nolinebreak[1]%
	\enspace\hspace*{\fill}
	\finalhyphendemerits=0
	[3rd October; LD gain from Ind]}

\index{Norton Fitzwarren and Staplegrove , Somerset West and Taunton@Norton Fitzwarren \& Staplegrove, \emph{Somerset W. \& Taunton}}

Death of Jean Adkins (Ind).

\noindent
\begin{tabular*}{\columnwidth}{@{\extracolsep{\fill}} p{0.53\columnwidth} >{\itshape}l r @{\extracolsep{\fill}}}
Andy Sully & LD & 686\\
Rob Williams & C & 493\\
Alan Debenham & Grn & 67\\
Michael McGuffie & Lab & 31\\
\end{tabular*}

\section{Staffordshire}

\subsection*{County Council}

\subsubsection*{Watling South \hspace*{\fill}\nolinebreak[1]%
	\enspace\hspace*{\fill}
	\finalhyphendemerits=0
	[12th December]}

\index{Watling South , Staffordshire@Watling S., \emph{Staffs.}}

Death of Michael Greatorex (C).

\noindent
\begin{tabular*}{\columnwidth}{@{\extracolsep{\fill}} p{0.53\columnwidth} >{\itshape}l r @{\extracolsep{\fill}}}
Richard Ford & C & -\\
Roger Jones & LD & -\\
Simon Peaple & Lab & -\\
\end{tabular*}

\subsection*{Lichfield}

At the May 2019 ordinary election there was an unfilled vacancy in Summerfield and All Saints ward due to the resignation of Richard Mosson (C).
\index{Summerfield and All Saints , Lichfield@Summerfield \& All SS., \emph{Lichfield}}

\subsection*{Newcastle-under-Lyme}

\subsubsection*{Holditch and Chesterton \hspace*{\fill}\nolinebreak[1]%
	\enspace\hspace*{\fill}
	\finalhyphendemerits=0
	[21st March; Ind gain from Lab]}

\index{Holditch and Chesterton , Newcastle-under-Lyme@Holditch \& Chesterton, \emph{Newcastle-under-Lyme}}

Resignation of Chris Spence (Lab).

\noindent
\begin{tabular*}{\columnwidth}{@{\extracolsep{\fill}} p{0.53\columnwidth} >{\itshape}l r @{\extracolsep{\fill}}}
Kenneth Owen & Ind & 282\\
Peter Radford & Lab & 268\\
Mark Barlow & UKIP & 86\\
Lawrence Whitworth & C & 49\\
Carol Lovatt & SDP & 14\\
\end{tabular*}

\subsubsection*{Maer and Whitmore \hspace*{\fill}\nolinebreak[1]%
	\enspace\hspace*{\fill}
	\finalhyphendemerits=0
	[2nd May]}

\index{Maer and Whitmore , Newcastle-under-Lyme@Maer \& Whitmore, \emph{Newcastle-under-Lyme}}

Resignation of David Harrison (C).

\noindent
\begin{tabular*}{\columnwidth}{@{\extracolsep{\fill}} p{0.53\columnwidth} >{\itshape}l r @{\extracolsep{\fill}}}
Graham Hutton & C & 629\\
Salwa Booth & LD & 97\\
\end{tabular*}

\subsubsection*{Holditch and Chesterton \hspace*{\fill}\nolinebreak[1]%
	\enspace\hspace*{\fill}
	\finalhyphendemerits=0
	[12th December]}

\index{Holditch and Chesterton , Newcastle-under-Lyme@Holditch \& Chesterton, \emph{Newcastle-under-Lyme}}

Resignation of Emily Horsfall (Lab).

\noindent
\begin{tabular*}{\columnwidth}{@{\extracolsep{\fill}} p{0.53\columnwidth} >{\itshape}l r @{\extracolsep{\fill}}}
Lillian Barker & Ind & -\\
David Grocott & Lab & -\\
Peter Hailstones & C & -\\
Aidan Jenkins & LD & -\\
\end{tabular*}

\subsection*{South Staffordshire}

At the May 2019 ordinary election there was an unfilled vacancy in Trysull and Seisdon ward due to the death of Robert McCardle (C).
\index{Trysull and Seisdon , South Staffordshire@Trysull \& Seisdon, \emph{S. Staffs.}}

\subsubsection*{Wombourne South West \hspace*{\fill}\nolinebreak[1]%
	\enspace\hspace*{\fill}
	\finalhyphendemerits=0
	[6th June]}

\index{Wombourne South West , South Staffordshire@Wombourne S.W., \emph{S. Staffs.}}

Ordinary election postponed from 2nd May: death of outgoing councillor Mary Bond (C) who was seeking re-election.

\noindent
\begin{tabular*}{\columnwidth}{@{\extracolsep{\fill}} p{0.53\columnwidth} >{\itshape}l r @{\extracolsep{\fill}}}
Mike Davies & C & 359\\
Vince Merrick & C & 325\\
Claire McIlvenna & Grn & 90\\
Pete Stones & LD & 79\\
Adam Freeman & Lab & 47\\
Michael Vaughan & Lab & 43\\
\end{tabular*}

\subsection*{Staffordshire Moorlands}

At the May 2019 ordinary election there was an unfilled vacancy in Cheadle South East ward due to the resignation of Deb Grocott (C).
\index{Cheadle South East , Staffordshire Moorlands@Cheadle S.E., \emph{Staffs. Moorlands}}

\subsection*{Tamworth}

At the May 2019 ordinary election there was an unfilled vacancy in Castle ward due to the death of Steve Claymore (C).
\index{Castle , Tamworth@Castle, \emph{Tamworth}}

\subsubsection*{Mercian \hspace*{\fill}\nolinebreak[1]%
	\enspace\hspace*{\fill}
	\finalhyphendemerits=0
	[12th December]}

\index{Mercian , Tamworth@Mercian, \emph{Tamworth}}

Death of Michael Greatorex (C).

\noindent
\begin{tabular*}{\columnwidth}{@{\extracolsep{\fill}} p{0.53\columnwidth} >{\itshape}l r @{\extracolsep{\fill}}}
Gordon Moore & Lab & -\\
Steven Pritchard & C & -\\
\end{tabular*}

\section{Suffolk}

\subsection*{Ipswich}

\subsubsection*{Alexandra \hspace*{\fill}\nolinebreak[1]%
	\enspace\hspace*{\fill}
	\finalhyphendemerits=0
	[26th September]}

\index{Alexandra , Ipswich@Alexandra, \emph{Ipswich}}

Resignation of Adam Leeder (Lab).

\noindent
\begin{tabular*}{\columnwidth}{@{\extracolsep{\fill}} p{0.53\columnwidth} >{\itshape}l r @{\extracolsep{\fill}}}
Adam Rae & Lab & 734\\
Henry Williams & LD & 287\\
Lee Reynolds & C & 278\\
Tom Wilmot & Grn & 164\\
\end{tabular*}

\subsection*{Mid Suffolk}

At the May 2019 ordinary election there was an unfilled vacancy in Eye ward due to the death of Michael Burke (C).
\index{Eye , Mid Suffolk@Eye, \emph{Mid Suffolk}}

\subsection*{St Edmundsbury}

At the abolition of St Edmundsbury council in April 2019 there was an unfilled vacancy in Haverhill North ward due to the resignation of Anthony Williams (Ind elected as UKIP).
\index{Haverhill North , Saint Edmundsbury@Haverhill N., \emph{St Edmundsbury}}

\section{Surrey}

\subsection*{County Council}

\subsubsection*{Warlingham \hspace*{\fill}\nolinebreak[1]%
	\enspace\hspace*{\fill}
	\finalhyphendemerits=0
	[31st January]}

\index{Warlingham , Surrey@Warlingham, \emph{Surrey}}

Resignation of David Hodge (C).

\noindent
\begin{tabular*}{\columnwidth}{@{\extracolsep{\fill}} p{0.53\columnwidth} >{\itshape}l r @{\extracolsep{\fill}}}
Becky Rush & C & 1199\\
Charles Lister & LD & 990\\
Martin Haley & UKIP & 176\\
Michael Snowden & Lab & 126\\
\end{tabular*}

\subsubsection*{Haslemere \hspace*{\fill}\nolinebreak[1]%
	\enspace\hspace*{\fill}
	\finalhyphendemerits=0
	[2nd May; Ind gain from C]}

\index{Haslemere , Surrey@Haslemere, \emph{Surrey}}

Resignation of Richard Hampson (C).

\noindent
\begin{tabular*}{\columnwidth}{@{\extracolsep{\fill}} p{0.53\columnwidth} >{\itshape}l r @{\extracolsep{\fill}}}
Nikki Barton & Ind & 2665\\
Malcolm Carter & C & 1159\\
Adrian la Porta & Lab & 263\\
\end{tabular*}

\subsection*{Reigate and Banstead}

Nork = Nork Residents Association

At the May 2019 ordinary election there were unfilled vacancies in Kingswood with Burgh Heath and Nork wards due to the death of Ros Mill (C) and the resignation of Jonathan White (Nork).
\index{Kingswood with Burgh Heath , Reigate and Banstead@Kingswood with Burgh Heath, \emph{Reigate \& Banstead}}
\index{Nork , Reigate and Banstead@Nork, \emph{Reigate \& Banstead}}

\subsection*{Spelthorne}

At the May 2019 ordinary election there was an unfilled vacancy in Stanwell North ward due to the resignation of Kevin Flurry (C).
\index{Stanwell North , Spelthorne@Stanwell N., \emph{Spelthorne}}

\subsection*{Waverley}

At the May 2019 ordinary election there was an unfilled vacancy in Hindhead ward due to the disqualification (non-attendance) of Christiaan Hesse (C).
\index{Hindhead , Waverley@Hindhead, \emph{Waverley}}

\section{Warwickshire}

\subsection*{Rugby}

\subsubsection*{Rokeby and Overslade \hspace*{\fill}\nolinebreak[1]%
	\enspace\hspace*{\fill}
	\finalhyphendemerits=0
	[22nd August]}

\index{Rokeby and Overslade , Rugby@Rokeby \& Overslade, \emph{Rugby}}

Resignation of Nick Long (LD).

\noindent
\begin{tabular*}{\columnwidth}{@{\extracolsep{\fill}} p{0.53\columnwidth} >{\itshape}l r @{\extracolsep{\fill}}}
Glenda Allanach & LD & 963\\
Deborah Keeling & C & 346\\
Beck Hemsley & Lab & 165\\
Richard Hartland & Brexit & 163\\
Becca Stevenson & Grn & 79\\
\end{tabular*}

\subsection*{Stratford-on-Avon}

At the May 2019 ordinary election there was an unfilled vacancy in Alcester Town ward due to the death of Eric Payne (C).
\index{Alcester Town , Stratford-on-Avon@Alcester Town, \emph{Stratford-on-Avon}}

\subsection*{Warwick}

\subsubsection*{Leamington Lillington \hspace*{\fill}\nolinebreak[1]%
	\enspace\hspace*{\fill}
	\finalhyphendemerits=0
	[Tuesday 29th October]}

\index{Leamington Lillington , Warwick@Leamington Lillington, \emph{Warwick}}

Resignation of Heather Calver (LD).

\noindent
\begin{tabular*}{\columnwidth}{@{\extracolsep{\fill}} p{0.53\columnwidth} >{\itshape}l r @{\extracolsep{\fill}}}
Daniel Russell & LD & 1296\\
Hayley Key & C & 664\\
Luc Lowndes & Lab & 384\\
\end{tabular*}

\subsubsection*{Warwick Myton and Heathcote \hspace*{\fill}\nolinebreak[1]%
	\enspace\hspace*{\fill}
	\finalhyphendemerits=0
	[12th December]}

\index{Warwick Myton and Heathcote , Warwick@Warwick Myton \& Heathcote, \emph{Warwick}}

Disqualification (bankruptcy restrictions order) of Sakhi Sanghera (C).

\noindent
\begin{tabular*}{\columnwidth}{@{\extracolsep{\fill}} p{0.53\columnwidth} >{\itshape}l r @{\extracolsep{\fill}}}
Paul Atkins & Grn & -\\
Bob Dhillon & Ind & -\\
Hugh Foden & C & -\\
Curtis Oliver-Smith & Lab & -\\
\end{tabular*}

\section{West Sussex}

Justice = Justice Party

\subsection*{County Council}

\subsubsection*{Northgate and West Green \hspace*{\fill}\nolinebreak[1]%
	\enspace\hspace*{\fill}
	\finalhyphendemerits=0
	[2nd May]}

\index{Northgate and West Green , West Sussex@Northgate \& West Green, \emph{W. Sussex}}

Resignation of Sue Mullins (Lab).

\noindent
\begin{tabular*}{\columnwidth}{@{\extracolsep{\fill}} p{0.53\columnwidth} >{\itshape}l r @{\extracolsep{\fill}}}
Karen Sudan & Lab & 1293\\
Jan Tarrant & C & 839\\
David Anderson & LD & 268\\
Richard Kail & Grn & 246\\
\end{tabular*}

\subsubsection*{Three Bridge \hspace*{\fill}\nolinebreak[1]%
	\enspace\hspace*{\fill}
	\finalhyphendemerits=0
	[26th September]}

\index{Three Bridges , West Sussex@Three Bridges, \emph{W. Sussex}}

Death of Charles Petts (C).

\noindent
\begin{tabular*}{\columnwidth}{@{\extracolsep{\fill}} p{0.53\columnwidth} >{\itshape}l r @{\extracolsep{\fill}}}
Brenda Burgess & C & 1102\\
Angela Malik & Lab & 628\\
David Anderson & LD & 257\\
Danielle Kail & Grn & 136\\
Arshad Khan & Justice & 9\\
\end{tabular*}

\subsubsection*{Bourne \hspace*{\fill}\nolinebreak[1]%
	\enspace\hspace*{\fill}
	\finalhyphendemerits=0
	[21st November]}

\index{Bourne , West Sussex@Bourne, \emph{W. Sussex}}

Resignation of Viral Parikh (Brexit elected as C).

\noindent
\begin{tabular*}{\columnwidth}{@{\extracolsep{\fill}} p{0.53\columnwidth} >{\itshape}l r @{\extracolsep{\fill}}}
Mike Magill & C & 1368\\
Andrew Kerry-Bedell & LD & 1009\\
Michael Neville & Grn & 250\\
Jane Towers & Lab & 161\\
Andrew Emerson & Patria & 12\\
\end{tabular*}

\subsection*{Chichester}

At the May 2019 ordinary election there was an unfilled vacancy in Selsey North ward due to the resignation of Darren Wakeham (C).
\index{Selsey North , Chichester@Selsey N., \emph{Chichester}}

\subsubsection*{Loxwood \hspace*{\fill}\nolinebreak[1]%
	\enspace\hspace*{\fill}
	\finalhyphendemerits=0
	[21st November; C gain from LD]}

\index{Loxwood , Chichester@Loxwood, \emph{Chichester}}

Resignation of Natalie Hume (Grn elected as LD).

\noindent
\begin{tabular*}{\columnwidth}{@{\extracolsep{\fill}} p{0.53\columnwidth} >{\itshape}l r @{\extracolsep{\fill}}}
Janet Duncton & C & 1005\\
Alexander Jeffery & LD & 486\\
Francesca Chetta & Grn & 126\\
Andrew Emerson & Patria & 9\\
\end{tabular*}

\subsection*{Crawley}

\subsubsection*{Tilgate \hspace*{\fill}\nolinebreak[1]%
	\enspace\hspace*{\fill}
	\finalhyphendemerits=0
	[26th September]}

\index{Tilgate , Crawley@Tilgate, \emph{Crawley}}

Death of Charles Petts (C).

\noindent
\begin{tabular*}{\columnwidth}{@{\extracolsep{\fill}} p{0.53\columnwidth} >{\itshape}l r @{\extracolsep{\fill}}}
Maureen Mwagale & C & 741\\
Kiran Khan & Lab & 396\\
Angharad Old & LD & 82\\
Derek Hardman & Grn & 75\\
Arshad Khan & Justice & 5\\
\end{tabular*}

\subsection*{Horsham}

\subsubsection*{Storrington and Washington \hspace*{\fill}\nolinebreak[1]%
	\enspace\hspace*{\fill}
	\finalhyphendemerits=0
	[12th December]}

\index{Storrington and Washington , Horsham@Storrington \& Washington, \emph{Horsham}}

Resignation of Paul Marshall (C).

\noindent
\begin{tabular*}{\columnwidth}{@{\extracolsep{\fill}} p{0.53\columnwidth} >{\itshape}l r @{\extracolsep{\fill}}}
James Wright & C & 3283\\
Alex Beveridge & LD & 1344\\
Jim Monaghan & Lab & 924\\
\end{tabular*}

\subsection*{Worthing}

\subsubsection*{Salvington \hspace*{\fill}\nolinebreak[1]%
	\enspace\hspace*{\fill}
	\finalhyphendemerits=0
	[12th December]}

\index{Salvington , Worthing@Salvington, \emph{Worthing}}

Resignation of Anthony Baker (C).

\noindent
\begin{tabular*}{\columnwidth}{@{\extracolsep{\fill}} p{0.53\columnwidth} >{\itshape}l r @{\extracolsep{\fill}}}
Emma Norton & LD & -\\
Richard Nowak & C & -\\
Gill Poole & Lab & -\\
\end{tabular*}

\section{Wiltshire}

\subsection*{Wiltshire}

\subsubsection*{Trowbridge Drynham \hspace*{\fill}\nolinebreak[1]%
	\enspace\hspace*{\fill}
	\finalhyphendemerits=0
	[4th July; LD gain from C]}

\index{Trowbridge Drynham , Wiltshire@Trowbridge Drynham, \emph{Wilts.}}

Death of Graham Payne (C).

\noindent
\begin{tabular*}{\columnwidth}{@{\extracolsep{\fill}} p{0.53\columnwidth} >{\itshape}l r @{\extracolsep{\fill}}}
Andrew Bryant & LD & 431\\
Kam Reynolds & C & 316\\
John Knight & Ind & 246\\
Shaun Henley & Lab & 44\\
\end{tabular*}

\subsubsection*{Westbury North \hspace*{\fill}\nolinebreak[1]%
	\enspace\hspace*{\fill}
	\finalhyphendemerits=0
	[18th July]}

\index{Westbury North , Wiltshire@Westbury N., \emph{Wilts.}}

Resignation of David Jenkins (LD).

\noindent
\begin{tabular*}{\columnwidth}{@{\extracolsep{\fill}} p{0.53\columnwidth} >{\itshape}l r @{\extracolsep{\fill}}}
Carole King & LD & 488\\
Ian Cunningham & Ind & 231\\
Antonio Piazza & C & 140\\
Jane Russ & Lab & 57\\
Francis Morland & Ind & 16\\
\end{tabular*}

\subsubsection*{Ethandune \hspace*{\fill}\nolinebreak[1]%
	\enspace\hspace*{\fill}
	\finalhyphendemerits=0
	[19th September]}

\index{Ethandune , Wiltshire@Ethandune, \emph{Wilts.}}

Death of Jerry Wickham (C).

\noindent
\begin{tabular*}{\columnwidth}{@{\extracolsep{\fill}} p{0.53\columnwidth} >{\itshape}l r @{\extracolsep{\fill}}}
Suzanne Wickham & C & 778\\
Alan Rankin & LD & 587\\
\end{tabular*}

\subsubsection*{Melksham Without South \hspace*{\fill}\nolinebreak[1]%
	\enspace\hspace*{\fill}
	\finalhyphendemerits=0
	[24th October]}

\index{Melksham Without South , Wiltshire@Melksham Wt. S., \emph{Wilts.}}

Resignation of Roy While (C).

\noindent
\begin{tabular*}{\columnwidth}{@{\extracolsep{\fill}} p{0.53\columnwidth} >{\itshape}l r @{\extracolsep{\fill}}}
Nick Holder & C & 593\\
Vanessa Fiorelli & LD & 388\\
\end{tabular*}

\subsubsection*{Trowbridge Lambrok \hspace*{\fill}\nolinebreak[1]%
	\enspace\hspace*{\fill}
	\finalhyphendemerits=0
	[28th November; LD gain from C]}

\index{Trowbridge Lambrok , Wiltshire@Trowbridge Lambrok, \emph{Wilts.}}

Resignation of Deborah Halik (C).

\noindent
\begin{tabular*}{\columnwidth}{@{\extracolsep{\fill}} p{0.53\columnwidth} >{\itshape}l r @{\extracolsep{\fill}}}
Jo Trigg & LD & 622\\
David Cavill & C & 455\\
\end{tabular*}

\section{Worcestershire}

\subsection*{County Council}

\subsubsection*{Bromsgrove South \hspace*{\fill}\nolinebreak[1]%
	\enspace\hspace*{\fill}
	\finalhyphendemerits=0
	[31st October; C gain from Lab]}

\index{Bromsgrove South , Worcestershire@Bromsgrove S., \emph{Worcs.}}

Resignation of Chris Bloore (Lab).

\noindent
\begin{tabular*}{\columnwidth}{@{\extracolsep{\fill}} p{0.53\columnwidth} >{\itshape}l r @{\extracolsep{\fill}}}
Kyle Daisley & C & 769\\
Rachel Jsnkins & Ind & 436\\
Joshua Robinson & LD & 357\\
Bren Henderson & Lab & 351\\
\end{tabular*}

\subsection*{Malvern Hills}

At the May 2019 ordinary election there was an unfilled vacancy in Pickersleigh ward due to the resignation of Leanne Halling (C).
\index{Pickersleigh , Malvern Hills@Pickersleigh, \emph{Malvern Hills}}

\subsection*{Worcester}

\subsubsection*{Claines \hspace*{\fill}\nolinebreak[1]%
	\enspace\hspace*{\fill}
	\finalhyphendemerits=0
	[8th August; LD gain from C]}

\index{Claines , Worcester@Claines, \emph{Worcester}}

Death of Stuart Denleigh-Maxwell (C).

\noindent
\begin{tabular*}{\columnwidth}{@{\extracolsep{\fill}} p{0.53\columnwidth} >{\itshape}l r @{\extracolsep{\fill}}}
Mel Allcott & LD & 1307\\
Jules Benham & C & 1252\\
Stephen Dent & Grn & 125\\
Saiful Islam & Lab & 60\\
\end{tabular*}

\subsection*{Wychavon}

At the May 2019 ordinary election there were unfilled vacancies in Harvington and Norton, and Lovett and North Claines wards due to the resignations of Charles Homer and Lynne Duffy (both C) respectively.
\index{Harvington and Norton , Wychavon@Harvington \& Norton, \emph{Wychavon}}
\index{Lovett and North Claines , Wychavon@Lovett \& North Claines, \emph{Wychavon}}

\subsection*{Wyre Forest}

At the May 2019 ordinary election there was an unfilled vacancy in Areley Kings and Riverside ward due to the resignation of Jamie Shaw (Lab).
\index{Areley Kings and Riverside , Wyre Forest@Areley Kings \& Riverside, \emph{Wyre Forest}}

\section{Glamorgan}

\subsection*{Cardiff}

\subsubsection*{Ely \hspace*{\fill}\nolinebreak[1]%
	\enspace\hspace*{\fill}
	\finalhyphendemerits=0
	[21st February; PC gain from Lab]}

\index{Ely , Cardiff@Ely, \emph{Cardiff}}

Death of Jim Murphy (Lab).

\noindent
\begin{tabular*}{\columnwidth}{@{\extracolsep{\fill}} p{0.53\columnwidth} >{\itshape}l r @{\extracolsep{\fill}}}
Andrea Gibson & PC & 831\\
Irene Humphreys & Lab & 779\\
Gavin Brookman & C & 271\\
Richard Jerrett & LD & 46\\
\end{tabular*}

\subsubsection*{Cyncoed \hspace*{\fill}\nolinebreak[1]%
	\enspace\hspace*{\fill}
	\finalhyphendemerits=0
	[Tuesday 16th July]}

\index{Cyncoed , Cardiff@Cyncoed, \emph{Cardiff}}

Death of Wendy Congreve (LD).

\noindent
\begin{tabular*}{\columnwidth}{@{\extracolsep{\fill}} p{0.53\columnwidth} >{\itshape}l r @{\extracolsep{\fill}}}
Robert Hopkins & LD & 1920\\
Peter Hudson & C & 838\\
Madhu Khanna-Davies & Lab & 560\\
Morgan Rogers & PC & 152\\
\end{tabular*}

\subsubsection*{Whitchurch and Tongwynlais \hspace*{\fill}\nolinebreak[1]%
	\enspace\hspace*{\fill}
	\finalhyphendemerits=0
	[3rd October]}

\index{Whitchurch and Tongwynlais , Cardiff@Whitchurch \& Tongwynlais, \emph{Cardiff}}

Death of Tim Davies (C).

\noindent
\begin{tabular*}{\columnwidth}{@{\extracolsep{\fill}} p{0.53\columnwidth} >{\itshape}l r @{\extracolsep{\fill}}}
Mia Rees & C & 1544\\
Marc Palmer & Lab & 1190\\
Dan Allsobrook & PC & 674\\
Sian Donne & LD & 588\\
David Griffin & Grn & 248\\
\end{tabular*}

\subsubsection*{Llanishen \hspace*{\fill}\nolinebreak[1]%
	\enspace\hspace*{\fill}
	\finalhyphendemerits=0
	[21st November; C gain from Lab]}

\index{Llanishen , Cardiff@Llanishen, \emph{Cardiff}}

Resignation of Phil Bale (Lab).

\noindent
\begin{tabular*}{\columnwidth}{@{\extracolsep{\fill}} p{0.53\columnwidth} >{\itshape}l r @{\extracolsep{\fill}}}
Siân-Elin Melbourne & C & 1566\\
Garry Hunt & Lab & 1254\\
Will Ogbourne & LD & 387\\
Chris Haines & PC & 209\\
Michael Cope & Grn & 138\\
Lawrence Gwynn & Ind & 59\\
\end{tabular*}

\subsection*{Merthyr Tydfil}

\subsubsection*{Cyfarthfa \hspace*{\fill}\nolinebreak[1]%
	\enspace\hspace*{\fill}
	\finalhyphendemerits=0
	[11th April]}

\index{Cyfarthfa , Merthyr Tydfil@Cyfarthfa, \emph{Merthyr Tydfil}}

Death of Paul Brown (Ind).

\noindent
\begin{tabular*}{\columnwidth}{@{\extracolsep{\fill}} p{0.53\columnwidth} >{\itshape}l r @{\extracolsep{\fill}}}
Michelle Jones & Ind & 861\\
Mark Prevett & Lab & 330\\
David Griffiths & Ind & 180\\
Paul Phillips & C & 48\\
\end{tabular*}

\subsection*{Neath Port Talbot}

\subsubsection*{Resolven \hspace*{\fill}\nolinebreak[1]%
	\enspace\hspace*{\fill}
	\finalhyphendemerits=0
	[23rd May]}

\index{Resolven , Neath Port Talbot@Resolven, \emph{Neath Port Talbot}}

Death of Des Davies (Lab).

\noindent
\begin{tabular*}{\columnwidth}{@{\extracolsep{\fill}} p{0.53\columnwidth} >{\itshape}l r @{\extracolsep{\fill}}}
Dean Lewis & Ind & 699\\
Mark Francis & Lab & 293\\
Andrew Hippsley & PC & 121\\
Jonathan Jones & C & 34\\
Sheila Kingston-Jones & LD & 23\\
\end{tabular*}

\subsubsection*{Pelenna \hspace*{\fill}\nolinebreak[1]%
	\enspace\hspace*{\fill}
	\finalhyphendemerits=0
	[20th June]}

\index{Pelenna , Neath Port Talbot@Pelenna, \emph{Neath Port Talbot}}

Death of Martin Ellis (Ind).

\noindent
\begin{tabular*}{\columnwidth}{@{\extracolsep{\fill}} p{0.53\columnwidth} >{\itshape}l r @{\extracolsep{\fill}}}
Jeremy Hurley & Ind & 251\\
Hywel Miles & PC & 120\\
Peter Hughes & Ind & 105\\
Andrew Jones & Lab & 43\\
Frank Little & LD & 6\\
\end{tabular*}

\subsubsection*{Rhos \hspace*{\fill}\nolinebreak[1]%
	\enspace\hspace*{\fill}
	\finalhyphendemerits=0
	[14th November; PC gain from Lab]}

\index{Rhos , Neath Port Talbot@Rhos, \emph{Neath Port Talbot}}

Resignation of Alex Thomas (Lab).

\noindent
\begin{tabular*}{\columnwidth}{@{\extracolsep{\fill}} p{0.53\columnwidth} >{\itshape}l r @{\extracolsep{\fill}}}
Marcia Spooner & PC & 359\\
Yvonne Lewis & C & 162\\
Rupert Denholm-Hall & Lab & 145\\
\end{tabular*}

\subsection*{Rhondda Cynon Taf}

\subsubsection*{Rhondda \hspace*{\fill}\nolinebreak[1]%
	\enspace\hspace*{\fill}
	\finalhyphendemerits=0
	[4th July; PC gain from Lab]}

\index{Rhondda , Rhondda Cynon Taf@Rhondda, \emph{Rhondda Cynon Taf}}

Death of Robert Smith (Lab).

\noindent
\begin{tabular*}{\columnwidth}{@{\extracolsep{\fill}} p{0.53\columnwidth} >{\itshape}l r @{\extracolsep{\fill}}}
Eleri Griffiths & PC & 404\\
Loretta Tomkinson & Lab & 266\\
Alexander Davies & C & 145\\
Karen Roberts & LD & 127\\
Adrian Dunphy & Comm & 18\\
\end{tabular*}

\subsubsection*{Ynyshir \hspace*{\fill}\nolinebreak[1]%
	\enspace\hspace*{\fill}
	\finalhyphendemerits=0
	[5th December; Lab gain from PC]}

\index{Ynyshir , Rhondda Cynon Taf@Ynyshir, \emph{Rhondda Cynon Taf}}

Resignation of Darren Macey (PC).

\noindent
\begin{tabular*}{\columnwidth}{@{\extracolsep{\fill}} p{0.53\columnwidth} >{\itshape}l r @{\extracolsep{\fill}}}
Julie Edwards & Lab & 407\\
Adrian Parry & PC & 331\\
\end{tabular*}

\subsection*{Vale of Glamorgan}

\subsubsection*{Rhoose \hspace*{\fill}\nolinebreak[1]%
	\enspace\hspace*{\fill}
	\finalhyphendemerits=0
	[14th February]}

\index{Rhoose , Vale of Glamorgan@Rhoose, \emph{Vale of Glamorgan}}

Resignation of Matthew Lloyd (C).

\noindent
\begin{tabular*}{\columnwidth}{@{\extracolsep{\fill}} p{0.53\columnwidth} >{\itshape}l r @{\extracolsep{\fill}}}
Andrew RT Davies & C & 1140\\
John Hartland & Lab & 368\\
Samantha Campbell & Ind & 345\\
\end{tabular*}

\section{Mid and West Wales}

\subsection*{Ceredigion}

\subsubsection*{Llanbadarn Fawr Sulien \hspace*{\fill}\nolinebreak[1]%
	\enspace\hspace*{\fill}
	\finalhyphendemerits=0
	[18th July]}

\index{Llanbadarn Fawr Sulien , Ceredigion@Llanbadarn Fawr Sulien, \emph{Ceredigion}}

Death of Paul James (PC).

\noindent
\begin{tabular*}{\columnwidth}{@{\extracolsep{\fill}} p{0.53\columnwidth} >{\itshape}l r @{\extracolsep{\fill}}}
Matthew Woolfall Jones & PC & 186\\
Michael Chappell & LD & 93\\
Richard Layton & Lab & 15\\
\end{tabular*}

\subsection*{Pembrokeshire}

\subsubsection*{Hundleton \hspace*{\fill}\nolinebreak[1]%
	\enspace\hspace*{\fill}
	\finalhyphendemerits=0
	[7th November]}

\index{Hundleton , Pembrokeshire@Hundleton, \emph{Pembrokeshire}}

Death of Margot Bateman (Ind).

\noindent
\begin{tabular*}{\columnwidth}{@{\extracolsep{\fill}} p{0.53\columnwidth} >{\itshape}l r @{\extracolsep{\fill}}}
Steve Alderman & Ind & 220\\
Jacob Taylor & C & 128\\
Daphne Bush & Ind & 58\\
Shirley Hammond-Williams & LD & 57\\
Nicky Hancock & Ind & 46\\
Tony Stenson & Ind & 37\\
David Edwards & Ind & 35\\
Barry Grange & Ind & 25\\
Jonathan Nutting & Ind & 1\\
\end{tabular*}

\subsection*{Powys}

\subsubsection*{Llandrindod North \hspace*{\fill}\nolinebreak[1]%
	\enspace\hspace*{\fill}
	\finalhyphendemerits=0
	[24th October; LD gain from C]}

\index{Llandrindod North , Powys@Llandrindod N., \emph{Powys}}

Resignation of Gary Price (Ind elected as C).

\noindent
\begin{tabular*}{\columnwidth}{@{\extracolsep{\fill}} p{0.53\columnwidth} >{\itshape}l r @{\extracolsep{\fill}}}
Jake Berriman & LD & 226\\
Tom Turner & C & 164\\
Rosie McConnell & Lab & 89\\
\end{tabular*}

\subsubsection*{Newtown South \hspace*{\fill}\nolinebreak[1]%
	\enspace\hspace*{\fill}
	\finalhyphendemerits=0
	[24th October]}

\index{Newtown South , Powys@Newtown S., \emph{Powys}}

Resignation of Alan Morrison (C).

\noindent
\begin{tabular*}{\columnwidth}{@{\extracolsep{\fill}} p{0.53\columnwidth} >{\itshape}l r @{\extracolsep{\fill}}}
Les Skilton & C & 134\\
Kelly Healy & LD & 110\\
Richard Edwards & PC & 64\\
\end{tabular*}

\subsubsection*{St Mary \hspace*{\fill}\nolinebreak[1]%
	\enspace\hspace*{\fill}
	\finalhyphendemerits=0
	[14th November; Lab gain from C]}

\index{Saint Mary , Powys@St Mary, \emph{Powys}}

Resignation of Sarah Lewis (C).

\noindent
\begin{tabular*}{\columnwidth}{@{\extracolsep{\fill}} p{0.53\columnwidth} >{\itshape}l r @{\extracolsep{\fill}}}
Liz Rijnenberg & Lab & 344\\
Alan Roberts & C & 244\\
Grenville Ham & PC & 130\\
Gareth Roberts & LD & 102\\
Gareth Phillips & Ind & 101\\
\end{tabular*}

\section{North Wales}

\subsection*{Flintshire}

\subsubsection*{Bagillt West \hspace*{\fill}\nolinebreak[1]%
	\enspace\hspace*{\fill}
	\finalhyphendemerits=0
	[24th October]}

\index{Bagillt West , Flintshire@Bagillt W., \emph{Flintshire}}

Resignation of Mike Reece (Lab).

\noindent
\begin{tabular*}{\columnwidth}{@{\extracolsep{\fill}} p{0.53\columnwidth} >{\itshape}l r @{\extracolsep{\fill}}}
Kevin Rush & Lab & 251\\
David Stanley & Ind & 144\\
\end{tabular*}

\subsubsection*{Trelawnyd and Gwaenysgor \hspace*{\fill}\nolinebreak[1]%
	\enspace\hspace*{\fill}
	\finalhyphendemerits=0
	[12th December]}

\index{Trelawnyd and Gwaenysgor , Flintshire@Trelawnyd \& Gwaenysgor, \emph{Flintshire}}

Resignation of Andrew Holgate (C).

\noindent
\begin{tabular*}{\columnwidth}{@{\extracolsep{\fill}} p{0.53\columnwidth} >{\itshape}l r @{\extracolsep{\fill}}}
Tim Roberts & C & 296\\
Helen Papworth & Lab & 283\\
David Ellis & Ind & 230\\
Heather Prydderch & LD & 59\\
\end{tabular*}

\subsection*{Gwynedd}

\subsubsection*{Morfa Nefyn \hspace*{\fill}\nolinebreak[1]%
	\enspace\hspace*{\fill}
	\finalhyphendemerits=0
	[23rd May]}

\index{Morfa Nefyn , Gwynedd@Morfa Nefyn, \emph{Gwynedd}}

Resignation of Siân Wyn Hughes (PC).

\noindent
\begin{tabular*}{\columnwidth}{@{\extracolsep{\fill}} p{0.53\columnwidth} >{\itshape}l r @{\extracolsep{\fill}}}
Gareth Jones & PC & \emph{unop.}\\
\end{tabular*}

\section{Aberdeenshire Councils}

\subsection*{Aberdeen}

Red = Red Party of Scotland

\subsubsection*{Bridge of Don (2) \hspace*{\fill}\nolinebreak[1]%
	\enspace\hspace*{\fill}
	\finalhyphendemerits=0
	[3rd October]}

\index{Bridge of Don , Aberdeen@Bridge of Don, \emph{Aberdeen}}

Resignation of Brett Hunt (C) and death of Sandy Stuart (SNP).

\noindent
\begin{tabular*}{\columnwidth}{@{\extracolsep{\fill}} p{0.53\columnwidth} >{\itshape}l r @{\extracolsep{\fill}}}
	\emph{First preferences}\\
Sarah Cross & C & 1857\\
Jessica Mennie & SNP & 1797\\
Michal Skoczykloda & LD & 929\\
Graeme Lawrence & Lab & 305\\
Sylvia Hardie & Grn & 140\\
Philip Clarke & UKIP & 55\\
Simon McLean & Ind & 43\\
Max McKay & Red & 9\\
\end{tabular*}

\emph{Quota = 1712.  Cross and Mennie elected, count complete.}

\subsubsection*{Torry\slash Ferryhill \hspace*{\fill}\nolinebreak[1]%
	\enspace\hspace*{\fill}
	\finalhyphendemerits=0
	[21st November]}

\index{Torry/Ferryhill , Aberdeen@Torry \slash Ferryhill, \emph{Aberdeen}}

Resignation of Catriona Mackenzie (SNP).

\noindent
\begin{tabular*}{\columnwidth}{@{\extracolsep{\fill}} p{0.53\columnwidth} >{\itshape}l r @{\extracolsep{\fill}}}
	\emph{First preferences}\\
Audrey Nicoll & SNP & 1618\\
Neil Murray & C & 972\\
Willie Young & Lab & 395\\
Gregory McAbery & LD & 315\\
Betty Lyon & Grn & 304\\
Simon McLean & Ind & 86\\
Roy Hill & UKIP & 53\\
\end{tabular*}

\emph{McLean and Hill eliminated}: Nicoll 1624 Murray 996 Young 402 Lyon 347 McAbery 324

\emph{McAbery eliminated}: Nicoll 1673 Murray 1045 Young 462 Lyon 429

\noindent
\begin{tabular*}{\columnwidth}{@{\extracolsep{\fill}} p{0.53\columnwidth} >{\itshape}l r @{\extracolsep{\fill}}}
	\emph{Lyon eliminated}\\
	Audrey Nicoll & SNP & 1819\\
	Neil Murray & C & 1083\\
	Willie Young & Lab & 533\\
\end{tabular*}

\section{Clyde Councils}

\subsection*{North Lanarkshire}

\subsubsection*{Thorniewood \hspace*{\fill}\nolinebreak[1]%
	\enspace\hspace*{\fill}
	\finalhyphendemerits=0
	[19th September]}

\index{Thorniewood , North Lanarkshire@Thorniewood, \emph{N. Lanarks.}}

Resignation of Hugh Gaffney MP (Lab).

\noindent
\begin{tabular*}{\columnwidth}{@{\extracolsep{\fill}} p{0.53\columnwidth} >{\itshape}l r @{\extracolsep{\fill}}}
\emph{First preferences}\\
Norah Mooney & Lab & 1362\\
Eve Cunnington & SNP & 1202\\
Lorraine Nolan & C & 296\\
Colin Robb & LD & 168\\
Rosemary McGowan & Grn & 46\\
\end{tabular*}

\emph{Robb and McGowan eliminated}: Mooney 1424 Cunnington 1245 Nolan 335

\noindent
\begin{tabular*}{\columnwidth}{@{\extracolsep{\fill}} p{0.53\columnwidth} >{\itshape}l r @{\extracolsep{\fill}}}
	\emph{Nolan eliminated}\\
	Norah Mooney & Lab & 1528\\
	Eve Cunnington & SNP & 1271\\
\end{tabular*}

\subsection*{South Lanarkshire}

Libtn = Scottish Libertarian Party

\subsubsection*{East Kilbride Central North \hspace*{\fill}\nolinebreak[1]%
	\enspace\hspace*{\fill}
	\finalhyphendemerits=0
	[29th August]}

\index{East Kilbride Central North , South Lanarkshire@East Kilbride C.N., \emph{S. Lanarks.}}

Death of Sheena Wardhaugh (Ind elected as SNP).

\noindent
\begin{tabular*}{\columnwidth}{@{\extracolsep{\fill}} p{0.53\columnwidth} >{\itshape}l r @{\extracolsep{\fill}}}
\emph{First preferences}\\
Grant Ferguson & SNP & 1582\\
Kirsty Williams & Lab & 690\\
Graham Fisher & C & 498\\
Paul McGarry & LD & 422\\
Antony Lee & Grn & 153\\
David Mackay & UKIP & 48\\
Stephen McNamara & Libtn & 12\\
\end{tabular*}

\noindent
\begin{tabular*}{\columnwidth}{@{\extracolsep{\fill}} p{0.53\columnwidth} >{\itshape}l r @{\extracolsep{\fill}}}
\emph{Lee, Mackay and McNamara eliminated}\\
Grant Ferguson & SNP & 1650\\
Kirsty Williams & Lab & 715\\
Graham Fisher & C & 519\\
Paul McGarry & LD & 456\\
\end{tabular*}

\noindent
\begin{tabular*}{\columnwidth}{@{\extracolsep{\fill}} p{0.53\columnwidth} >{\itshape}l r @{\extracolsep{\fill}}}
\emph{Lee, Mackay and McNamara eliminated}\\
Grant Ferguson & SNP & 1743\\
Kirsty Williams & Lab & 837\\
Graham Fisher & C & 606\\
\end{tabular*}

\section{Forth Councils}

\subsection*{Clackmannanshire}

\subsubsection*{Clackmannanshire Central \hspace*{\fill}\nolinebreak[1]%
\enspace\hspace*{\fill}
\finalhyphendemerits=0
[28th March]}

\index{Clackmannanshire Central , Clackmannanshire@Clackmannanshire C., \emph{Clackmannanshire}}

Resignation of Phil Fairlie (SNP).

\noindent
\begin{tabular*}{\columnwidth}{@{\extracolsep{\fill}} p{0.53\columnwidth} >{\itshape}l r @{\extracolsep{\fill}}}
\emph{First preferences}\\
Jane McTaggart & SNP & 865\\
Margaret Brookes & Lab & 675\\
William Marlin & C & 419\\
Dawson Michie & UKIP & 69\\
Marion Robertson & Grn & 53\\
John Biggam & LD & 36\\
\end{tabular*}

\noindent
\begin{tabular*}{\columnwidth}{@{\extracolsep{\fill}} p{0.53\columnwidth} >{\itshape}l r @{\extracolsep{\fill}}}
	\emph{Marlin, Michie, Robertson and Biggam eliminated}\\
	Jane McTaggart & SNP & 933\\
	Margaret Brookes & Lab & 814\\
\end{tabular*}

%\noindent
%\begin{tabular*}{\columnwidth}{@{\extracolsep{\fill}} p{0.53\columnwidth} >{\itshape}l r @{\extracolsep{\fill}}}
%\emph{Khanam, Sherwood-Johnson and Robertson eliminated}\\
%Helen Lewis & SNP & 980\\
%Alex Stewart & C & 784\\
%\end{tabular*}

\subsection*{East Lothian}

\subsubsection*{Haddington and Lammermuir \hspace*{\fill}\nolinebreak[1]%
	\enspace\hspace*{\fill}
	\finalhyphendemerits=0
	[9th May]}

\index{Haddington and Lammermuir , East Lothian@Haddington \& Lammermuir, \emph{E. Lothian}}

Resignation of Brian Small (C).

\noindent
\begin{tabular*}{\columnwidth}{@{\extracolsep{\fill}} p{0.53\columnwidth} >{\itshape}l r @{\extracolsep{\fill}}}
\emph{First preferences}\\
Craig Hoy & C & 2212\\
Lorraine Glass & SNP & 1866\\
Neal Black & Lab & 1359\\
Stuart Crawford & LD & 774\\
David Sisson & UKIP & 108\\
\end{tabular*}

\emph{Crawford and Sisson eliminated}: Hoy 2428 Glass 2044 Black 1589

\noindent
\begin{tabular*}{\columnwidth}{@{\extracolsep{\fill}} p{0.53\columnwidth} >{\itshape}l r @{\extracolsep{\fill}}}
	\emph{Black eliminated}\\
	Craig Hoy & C & 2759\\
	Lorraine Glass & SNP & 2469\\
\end{tabular*}



\subsection*{Edinburgh}

ForBritn = For Britain Movement

\subsubsection*{Leith Walk \hspace*{\fill}\nolinebreak[1]%
	\enspace\hspace*{\fill}
	\finalhyphendemerits=0
	[11th April; SNP gain from Lab]}

\index{Leith Walk , Edinburgh@Leith Walk, \emph{Edinburgh}}

Resignation of Marion Donaldson (Lab).

\noindent
\begin{tabular*}{\columnwidth}{@{\extracolsep{\fill}} p{0.53\columnwidth} >{\itshape}l r @{\extracolsep{\fill}}}
\emph{First preferences}\\
Rob Munn & SNP & 2596\\
Lorna Slater & Grn & 1855\\
Nick Gardner & Lab & 1123\\
Dan McCroskrie & C & 777\\
Jack Caldwell & LD & 623\\
Kevin Illingworth & Ind & 110\\
Steven Alexander & UKIP & 85\\
David Jacobsen & SocLab & 56\\
John Scott & Ind & 16\\
Paul Stirling & ForBritn & 14\\
Tom Laird & Libtn & 12\\
\end{tabular*}

\emph{Illingworth, Alexander, Jacobsen, Scott, Stirling and Laird eliminated:} Munn 2630 Slater 1904 Gardner 1157 McCroskrie 825 Caldwell 652

\emph{Caldwell eliminated:} Munn 2721 Slater 2093 Gardner 1320 McCroskrie 912

\emph{McCroskrie eliminated:} Munn 2763 Slater 2223 Gardner 1497

\noindent
\begin{tabular*}{\columnwidth}{@{\extracolsep{\fill}} p{0.53\columnwidth} >{\itshape}l r @{\extracolsep{\fill}}}
	\emph{Gardner eliminated}\\
	Rob Munn & SNP & 3021\\
	Lorna Slater & Grn & 2765\\
\end{tabular*}

\subsection*{Fife}

\subsubsection*{Dunfermline Central \hspace*{\fill}\nolinebreak[1]%
	\enspace\hspace*{\fill}
	\finalhyphendemerits=0
	[14th November; SNP gain from C]}

\index{Dunfermline Central , Fife@Dunfermline C., \emph{Fife}}

Resignation of Alan Craig (C).

\noindent
\begin{tabular*}{\columnwidth}{@{\extracolsep{\fill}} p{0.53\columnwidth} >{\itshape}l r @{\extracolsep{\fill}}}
\emph{First preferences}\\
Derek Glen & SNP & 1526\\
Chloe Dodds & C & 1142\\
Aude Boubaker-Calder & LD & 1050\\
Michael Boyd & Lab & 621\\
Fiona McOwan & Grn & 235\\
Keith Chamberlain & Libtn & 28\\
\end{tabular*}

\emph{Boyd, McOwan and Chamberlain eliminated}: Glen 1761 Boubaker-Caldrer 1343 Dodds 1202

\noindent
\begin{tabular*}{\columnwidth}{@{\extracolsep{\fill}} p{0.53\columnwidth} >{\itshape}l r @{\extracolsep{\fill}}}
	\emph{Dodds eliminated}\\
	Derek Glen & SNP & 1798\\
	Aude Boubaker-Calder & LD & 1796\\
\end{tabular*}

\subsubsection*{Rosyth \hspace*{\fill}\nolinebreak[1]%
	\enspace\hspace*{\fill}
	\finalhyphendemerits=0
	[14th November]}

\index{Rosyth , Fife@Rosyth, \emph{Fife}}

Resignation of Sam Steele (SNP).

\noindent
\begin{tabular*}{\columnwidth}{@{\extracolsep{\fill}} p{0.53\columnwidth} >{\itshape}l r @{\extracolsep{\fill}}}
	\emph{First preferences}\\
Sharon Green-Wilson & SNP & 1347\\
Margaret Fairgrieve & C & 768\\
Billy Pollock & Lab & 480\\
Jill Blair & LD & 249\\
Alastair Macintyre & Ind & 157\\
Craig McCutcheon & Grn & 132\\
Calum Paul & Libtn & 16\\
\end{tabular*}

\emph{McCutcheon and Paul eliminated}: Green-Wilson 1406 Fairgrieve 774 Pollock 498 Blair 275 Macintyre 157

\emph{Macintyre eliminated}: Green-Wilson 1429 Fairgrieve 822 Pollock 526 Blair 291

\noindent
\begin{tabular*}{\columnwidth}{@{\extracolsep{\fill}} p{0.53\columnwidth} >{\itshape}l r @{\extracolsep{\fill}}}
	\emph{Blair eliminated}\\
	Sharon Green-Wilson & SNP & 1486\\
	Margaret Fairgrieve & C & 885\\
	Billy Pollock & Lab & 591\\
\end{tabular*}

\section{Highland Councils}

\subsection*{Highland}

\subsubsection*{Inverness Central \hspace*{\fill}\nolinebreak[1]%
	\enspace\hspace*{\fill}
	\finalhyphendemerits=0
	[14th November]}

\index{Inverness Central , Highland@Inverness C., \emph{Highland}}

Resignation of Richard Laird (SNP).

\noindent
\begin{tabular*}{\columnwidth}{@{\extracolsep{\fill}} p{0.53\columnwidth} >{\itshape}l r @{\extracolsep{\fill}}}
	\emph{First preferences}\\
Emma Roddick & SNP & 1015\\
Rachael Hatfield & C & 345\\
Richie Patton & Ind & 277\\
Mary Dormer & LD & 237\\
Russell Deacon & Grn & 220\\
Ardalan Eghtedar & Lab & 154\\
\end{tabular*}

\emph{Eghtedar eliminated}: Roddick 1033 Hatfield 349 Patton 303 Dormer 266 Deacon 238

\noindent
\begin{tabular*}{\columnwidth}{@{\extracolsep{\fill}} p{0.53\columnwidth} >{\itshape}l r @{\extracolsep{\fill}}}
	\emph{Deacon eliminated}\\
	Emma Roddick & SNP & 1115\\
	Rachael Hatfield & C & 360\\
	Richie Patton & Ind & 338\\
	Mary Dormer & LD & 325\\
\end{tabular*}

\subsection*{Moray}

\subsubsection*{Keith and Cullen \hspace*{\fill}\nolinebreak[1]%
	\enspace\hspace*{\fill}
	\finalhyphendemerits=0
	[21st November; C gain from Ind]}

\index{Keith and Cullen , Moray@Keith \& Cullen, \emph{Moray}}

Resignation of Ron Shepherd (Ind).

\noindent
\begin{tabular*}{\columnwidth}{@{\extracolsep{\fill}} p{0.53\columnwidth} >{\itshape}l r @{\extracolsep{\fill}}}
	\emph{First preferences}\\
Laura Powell & C & 1142\\
Jock McKay & SNP & 1047\\
Rob Barsby & Ind & 349\\
Ian Aitchison & LD & 212\\
\end{tabular*}

\noindent
\begin{tabular*}{\columnwidth}{@{\extracolsep{\fill}} p{0.53\columnwidth} >{\itshape}l r @{\extracolsep{\fill}}}
	\emph{Barsby and Aitchison eliminated}\\
	Laura Powell & C & 1339\\
	Jock McKay & SNP & 1184\\
\end{tabular*}

\section{Island Councils}

\subsection*{Shetland}

\subsubsection*{Lerwick South \hspace*{\fill}\nolinebreak[1]%
	\enspace\hspace*{\fill}
	\finalhyphendemerits=0
	[7th November]}

\index{Lerwich South , Shetland@Lerwick S., \emph{Shetland}}

Resignation of Beatrice Wishart MSP (Ind).

\noindent
\begin{tabular*}{\columnwidth}{@{\extracolsep{\fill}} p{0.53\columnwidth} >{\itshape}l r @{\extracolsep{\fill}}}
\emph{First preferences}\\
Gary Robinson & Ind & 374\\
Stephen Flaws & Ind & 350\\
Frances Valente & Ind & 154\\
Arwed Wenger & Ind & 116\\
Caroline Henderson & Ind & 73\\
\end{tabular*}

\emph{Henderson eliminated:} Robinson 385 Flaws 368 Valente 172 Wenger 126

\noindent
\begin{tabular*}{\columnwidth}{@{\extracolsep{\fill}} p{0.53\columnwidth} >{\itshape}l r @{\extracolsep{\fill}}}
	\emph{Valente and Wenger eliminated}\\
	Stephen Flaws & Ind & 498\\
	Gary Robinson & Ind & 451\\
\end{tabular*}

\subsubsection*{Shetland Central \hspace*{\fill}\nolinebreak[1]%
	\enspace\hspace*{\fill}
	\finalhyphendemerits=0
	[7th November]}

\index{Shetland Central , Shetland@Shetland C., \emph{Shetland}}

Resignation of Mark Burgess (Ind).

\noindent
\begin{tabular*}{\columnwidth}{@{\extracolsep{\fill}} p{0.53\columnwidth} >{\itshape}l r @{\extracolsep{\fill}}}
\emph{First preferences}\\
Moraig Lyall & Ind & 344\\
Julie Buchan & Ind & 116\\
Stewart Douglas & SNP & 111\\
Gordon Laverie & Ind & 84\\
Johan Adamson & Ind & 77\\
\end{tabular*}

\noindent
\begin{tabular*}{\columnwidth}{@{\extracolsep{\fill}} p{0.53\columnwidth} >{\itshape}l r @{\extracolsep{\fill}}}
	\emph{Adamson eliminated}\\
	Moraig Lyall & Ind & 366\\
	Julie Buchan & Ind & 131\\
	Stewart Douglas & SNP & 119\\
	Gordon Laverie & Ind & 93\\
\end{tabular*}

\section{Tay Councils}

\subsection*{Dundee}

Cit1st = Citizens First

\subsubsection*{North East \hspace*{\fill}\nolinebreak[1]%
	\enspace\hspace*{\fill}
	\finalhyphendemerits=0
	[2nd May]}

\index{North East , Dundee@North East, \emph{Dundee}}

Death of Brian Gordon (Lab).

\noindent
\begin{tabular*}{\columnwidth}{@{\extracolsep{\fill}} p{0.53\columnwidth} >{\itshape}l r @{\extracolsep{\fill}}}
\emph{First preferences}\\
Steven Rome & SNP & 1507\\
Jim Malone & Lab & 1224\\
Robert Lindsay & C & 271\\
Michael Taylor & TUSC & 91\\
Alison Orr & Grn & 77\\
Roger Keech & Cit1st & 45\\
\end{tabular*}

\emph{Keech eliminated}: Rome 1510 Malone 1229 Lindsay 274 Taylor 100 Orr 82

\emph{Taylor and Orr eliminated}: Rome 1555 Malone 1284 Lindsay 290

\noindent
\begin{tabular*}{\columnwidth}{@{\extracolsep{\fill}} p{0.53\columnwidth} >{\itshape}l r @{\extracolsep{\fill}}}
	\emph{Lindsay eliminated}\\
	Steven Rome & SNP & 1576\\
	Jim Malone & Lab & 1388\\
\end{tabular*}

\end{resultsiii}

% Index:
\clearpage
\phantomsection
%\addcontentsline{toc}{chapter}{Index of Wards}
{\scriptsize%\raggedright
\frenchspacing\printindex}
\thispagestyle{plain}

%% GNU Free Documentation License
\chapter*{{GNU Free Documentation License}}
\phantomsection % so hyperref creates bookmarks
\addcontentsline{toc}{chapter}{GNU Free Documentation License}
%\label{label_fdl}
\pagestyle{plain}

 Version 1.3, 3 November 2008


 Copyright \copyright{} 2000, 2001, 2002, 2007, 2008 Free Software Foundation, Inc.
 
 \bigskip
 
 <\url{http://fsf.org/}>
 
 \bigskip
 
 Everyone is permitted to copy and distribute verbatim copies
 of this license document, but changing it is not allowed.

\begin{results}
\tiny

\subsubsection*{Preamble}

The purpose of this License is to make a manual, textbook, or other
functional and useful document ``free'' in the sense of freedom: to
assure everyone the effective freedom to copy and redistribute it,
with or without modifying it, either commercially or noncommercially.
Secondarily, this License preserves for the author and publisher a way
to get credit for their work, while not being considered responsible
for modifications made by others.

This License is a kind of ``copyleft'', which means that derivative
works of the document must themselves be free in the same sense. It
complements the GNU General Public License, which is a copyleft
license designed for free software.

We have designed this License in order to use it for manuals for free
software, because free software needs free documentation: a free
program should come with manuals providing the same freedoms that the
software does. But this License is not limited to software manuals;
it can be used for any textual work, regardless of subject matter or
whether it is published as a printed book. We recommend this License
principally for works whose purpose is instruction or reference.

\subsubsection*{1. Applicability and definitions}

This License applies to any manual or other work, in any medium, that
contains a notice placed by the copyright holder saying it can be
distributed under the terms of this License. Such a notice grants a
world-wide, royalty-free license, unlimited in duration, to use that
work under the conditions stated herein. The ``\emph{Document}'', below,
refers to any such manual or work. Any member of the public is a
licensee, and is addressed as ``\emph{you}''. You accept the license if you
copy, modify or distribute the work in a way requiring permission
under copyright law.

A ``\emph{Modified Version}'' of the Document means any work containing the
Document or a portion of it, either copied verbatim, or with
modifications and/or translated into another language.

A ``\emph{Secondary Section}'' is a named appendix or a front-matter section of
the Document that deals exclusively with the relationship of the
publishers or authors of the Document to the Document's overall subject
(or to related matters) and contains nothing that could fall directly
within that overall subject. (Thus, if the Document is in part a
textbook of mathematics, a Secondary Section may not explain any
mathematics.) The relationship could be a matter of historical
connection with the subject or with related matters, or of legal,
commercial, philosophical, ethical or political position regarding
them.

The ``\emph{Invariant Sections}'' are certain Secondary Sections whose titles
are designated, as being those of Invariant Sections, in the notice
that says that the Document is released under this License. If a
section does not fit the above definition of Secondary then it is not
allowed to be designated as Invariant. The Document may contain zero
Invariant Sections. If the Document does not identify any Invariant
Sections then there are none.

The ``\emph{Cover Texts}'' are certain short passages of text that are listed,
as Front-Cover Texts or Back-Cover Texts, in the notice that says that
the Document is released under this License. A Front-Cover Text may
be at most 5 words, and a Back-Cover Text may be at most 25 words.

A ``\emph{Transparent}'' copy of the Document means a machine-readable copy,
represented in a format whose specification is available to the
general public, that is suitable for revising the document
straightforwardly with generic text editors or (for images composed of
pixels) generic paint programs or (for drawings) some widely available
drawing editor, and that is suitable for input to text formatters or
for automatic translation to a variety of formats suitable for input
to text formatters. A copy made in an otherwise Transparent file
format whose markup, or absence of markup, has been arranged to thwart
or discourage subsequent modification by readers is not Transparent.
An image format is not Transparent if used for any substantial amount
of text. A copy that is not ``Transparent'' is called ``\emph{Opaque}''.

Examples of suitable formats for Transparent copies include plain
ASCII without markup, Texinfo input format, \LaTeX{} input format, SGML
or XML using a publicly available DTD, and standard-conforming simple
HTML, PostScript or PDF designed for human modification. Examples of
transparent image formats include PNG, XCF and JPG. Opaque formats
include proprietary formats that can be read and edited only by
proprietary word processors, SGML or XML for which the DTD and/or
processing tools are not generally available, and the
machine-generated HTML, PostScript or PDF produced by some word
processors for output purposes only.

The ``\emph{Title Page}'' means, for a printed book, the title page itself,
plus such following pages as are needed to hold, legibly, the material
this License requires to appear in the title page. For works in
formats which do not have any title page as such, ``Title Page'' means
the text near the most prominent appearance of the work's title,
preceding the beginning of the body of the text.

The ``\emph{publisher}'' means any person or entity that distributes
copies of the Document to the public.

A section ``\emph{Entitled XYZ}'' means a named subunit of the Document whose
title either is precisely XYZ or contains XYZ in parentheses following
text that translates XYZ in another language. (Here XYZ stands for a
specific section name mentioned below, such as ``\emph{Acknowledgements}'',
``\emph{Dedications}'', ``\emph{Endorsements}'', or ``\emph{History}''.) 
To ``\emph{Preserve the Title}''
of such a section when you modify the Document means that it remains a
section ``Entitled XYZ'' according to this definition.

The Document may include Warranty Disclaimers next to the notice which
states that this License applies to the Document. These Warranty
Disclaimers are considered to be included by reference in this
License, but only as regards disclaiming warranties: any other
implication that these Warranty Disclaimers may have is void and has
no effect on the meaning of this License.

\subsubsection*{2. Verbatim copying}

You may copy and distribute the Document in any medium, either
commercially or noncommercially, provided that this License, the
copyright notices, and the license notice saying this License applies
to the Document are reproduced in all copies, and that you add no other
conditions whatsoever to those of this License. You may not use
technical measures to obstruct or control the reading or further
copying of the copies you make or distribute. However, you may accept
compensation in exchange for copies. If you distribute a large enough
number of copies you must also follow the conditions in section~3.

You may also lend copies, under the same conditions stated above, and
you may publicly display copies.

\subsubsection*{3. Copying in quantity}

If you publish printed copies (or copies in media that commonly have
printed covers) of the Document, numbering more than 100, and the
Document's license notice requires Cover Texts, you must enclose the
copies in covers that carry, clearly and legibly, all these Cover
Texts: Front-Cover Texts on the front cover, and Back-Cover Texts on
the back cover. Both covers must also clearly and legibly identify
you as the publisher of these copies. The front cover must present
the full title with all words of the title equally prominent and
visible. You may add other material on the covers in addition.
Copying with changes limited to the covers, as long as they preserve
the title of the Document and satisfy these conditions, can be treated
as verbatim copying in other respects.

If the required texts for either cover are too voluminous to fit
legibly, you should put the first ones listed (as many as fit
reasonably) on the actual cover, and continue the rest onto adjacent
pages.

If you publish or distribute Opaque copies of the Document numbering
more than 100, you must either include a machine-readable Transparent
copy along with each Opaque copy, or state in or with each Opaque copy
a computer-network location from which the general network-using
public has access to download using public-standard network protocols
a complete Transparent copy of the Document, free of added material.
If you use the latter option, you must take reasonably prudent steps,
when you begin distribution of Opaque copies in quantity, to ensure
that this Transparent copy will remain thus accessible at the stated
location until at least one year after the last time you distribute an
Opaque copy (directly or through your agents or retailers) of that
edition to the public.

It is requested, but not required, that you contact the authors of the
Document well before redistributing any large number of copies, to give
them a chance to provide you with an updated version of the Document.

\subsubsection*{4. Modifications}

You may copy and distribute a Modified Version of the Document under
the conditions of sections 2 and 3 above, provided that you release
the Modified Version under precisely this License, with the Modified
Version filling the role of the Document, thus licensing distribution
and modification of the Modified Version to whoever possesses a copy
of it. In addition, you must do these things in the Modified Version:

\begin{itemize}
\item[A.] 
 Use in the Title Page (and on the covers, if any) a title distinct
 from that of the Document, and from those of previous versions
 (which should, if there were any, be listed in the History section
 of the Document). You may use the same title as a previous version
 if the original publisher of that version gives permission.
 
\item[B.]
 List on the Title Page, as authors, one or more persons or entities
 responsible for authorship of the modifications in the Modified
 Version, together with at least five of the principal authors of the
 Document (all of its principal authors, if it has fewer than five),
 unless they release you from this requirement.
 
\item[C.]
 State on the Title page the name of the publisher of the
 Modified Version, as the publisher.
 
\item[D.]
 Preserve all the copyright notices of the Document.
 
\item[E.]
 Add an appropriate copyright notice for your modifications
 adjacent to the other copyright notices.
 
\item[F.]
 Include, immediately after the copyright notices, a license notice
 giving the public permission to use the Modified Version under the
 terms of this License, in the form shown in the Addendum below.
 
\item[G.]
 Preserve in that license notice the full lists of Invariant Sections
 and required Cover Texts given in the Document's license notice.
 
\item[H.]
 Include an unaltered copy of this License.
 
\item[I.]
 Preserve the section Entitled ``History'', Preserve its Title, and add
 to it an item stating at least the title, year, new authors, and
 publisher of the Modified Version as given on the Title Page. If
 there is no section Entitled ``History'' in the Document, create one
 stating the title, year, authors, and publisher of the Document as
 given on its Title Page, then add an item describing the Modified
 Version as stated in the previous sentence.
 
\item[J.]
 Preserve the network location, if any, given in the Document for
 public access to a Transparent copy of the Document, and likewise
 the network locations given in the Document for previous versions
 it was based on. These may be placed in the ``History'' section.
 You may omit a network location for a work that was published at
 least four years before the Document itself, or if the original
 publisher of the version it refers to gives permission.
 
\item[K.]
 For any section Entitled ``Acknowledgements'' or ``Dedications'',
 Preserve the Title of the section, and preserve in the section all
 the substance and tone of each of the contributor acknowledgements
 and/or dedications given therein.
 
\item[L.]
 Preserve all the Invariant Sections of the Document,
 unaltered in their text and in their titles. Section numbers
 or the equivalent are not considered part of the section titles.
 
\item[M.]
 Delete any section Entitled ``Endorsements''. Such a section
 may not be included in the Modified Version.
 
\item[N.]
 Do not retitle any existing section to be Entitled ``Endorsements''
 or to conflict in title with any Invariant Section.
 
\item[O.]
 Preserve any Warranty Disclaimers.
\end{itemize}

If the Modified Version includes new front-matter sections or
appendices that qualify as Secondary Sections and contain no material
copied from the Document, you may at your option designate some or all
of these sections as invariant. To do this, add their titles to the
list of Invariant Sections in the Modified Version's license notice.
These titles must be distinct from any other section titles.

You may add a section Entitled ``Endorsements'', provided it contains
nothing but endorsements of your Modified Version by various
parties---for example, statements of peer review or that the text has
been approved by an organization as the authoritative definition of a
standard.

You may add a passage of up to five words as a Front-Cover Text, and a
passage of up to 25 words as a Back-Cover Text, to the end of the list
of Cover Texts in the Modified Version. Only one passage of
Front-Cover Text and one of Back-Cover Text may be added by (or
through arrangements made by) any one entity. If the Document already
includes a cover text for the same cover, previously added by you or
by arrangement made by the same entity you are acting on behalf of,
you may not add another; but you may replace the old one, on explicit
permission from the previous publisher that added the old one.

The author(s) and publisher(s) of the Document do not by this License
give permission to use their names for publicity for or to assert or
imply endorsement of any Modified Version.

\subsubsection*{5. Combining documents}

You may combine the Document with other documents released under this
License, under the terms defined in section~4 above for modified
versions, provided that you include in the combination all of the
Invariant Sections of all of the original documents, unmodified, and
list them all as Invariant Sections of your combined work in its
license notice, and that you preserve all their Warranty Disclaimers.

The combined work need only contain one copy of this License, and
multiple identical Invariant Sections may be replaced with a single
copy. If there are multiple Invariant Sections with the same name but
different contents, make the title of each such section unique by
adding at the end of it, in parentheses, the name of the original
author or publisher of that section if known, or else a unique number.
Make the same adjustment to the section titles in the list of
Invariant Sections in the license notice of the combined work.

In the combination, you must combine any sections Entitled ``History''
in the various original documents, forming one section Entitled
``History''; likewise combine any sections Entitled ``Acknowledgements'',
and any sections Entitled ``Dedications''. You must delete all sections
Entitled ``Endorsements''.

\subsubsection*{6. Collections of documents}

You may make a collection consisting of the Document and other documents
released under this License, and replace the individual copies of this
License in the various documents with a single copy that is included in
the collection, provided that you follow the rules of this License for
verbatim copying of each of the documents in all other respects.

You may extract a single document from such a collection, and distribute
it individually under this License, provided you insert a copy of this
License into the extracted document, and follow this License in all
other respects regarding verbatim copying of that document.

\subsubsection*{7. Aggregation with independent works}

A compilation of the Document or its derivatives with other separate
and independent documents or works, in or on a volume of a storage or
distribution medium, is called an ``aggregate'' if the copyright
resulting from the compilation is not used to limit the legal rights
of the compilation's users beyond what the individual works permit.
When the Document is included in an aggregate, this License does not
apply to the other works in the aggregate which are not themselves
derivative works of the Document.

If the Cover Text requirement of section~3 is applicable to these
copies of the Document, then if the Document is less than one half of
the entire aggregate, the Document's Cover Texts may be placed on
covers that bracket the Document within the aggregate, or the
electronic equivalent of covers if the Document is in electronic form.
Otherwise they must appear on printed covers that bracket the whole
aggregate.

\subsubsection*{8. Translation}

Translation is considered a kind of modification, so you may
distribute translations of the Document under the terms of section~4.
Replacing Invariant Sections with translations requires special
permission from their copyright holders, but you may include
translations of some or all Invariant Sections in addition to the
original versions of these Invariant Sections. You may include a
translation of this License, and all the license notices in the
Document, and any Warranty Disclaimers, provided that you also include
the original English version of this License and the original versions
of those notices and disclaimers. In case of a disagreement between
the translation and the original version of this License or a notice
or disclaimer, the original version will prevail.

If a section in the Document is Entitled ``Acknowledgements'',
``Dedications'', or ``History'', the requirement (section~4) to Preserve
its Title (section~1) will typically require changing the actual
title.

\subsubsection*{9. Termination}

You may not copy, modify, sublicense, or distribute the Document
except as expressly provided under this License. Any attempt
otherwise to copy, modify, sublicense, or distribute it is void, and
will automatically terminate your rights under this License.

However, if you cease all violation of this License, then your license
from a particular copyright holder is reinstated (a) provisionally,
unless and until the copyright holder explicitly and finally
terminates your license, and (b) permanently, if the copyright holder
fails to notify you of the violation by some reasonable means prior to
60 days after the cessation.

Moreover, your license from a particular copyright holder is
reinstated permanently if the copyright holder notifies you of the
violation by some reasonable means, this is the first time you have
received notice of violation of this License (for any work) from that
copyright holder, and you cure the violation prior to 30 days after
your receipt of the notice.

Termination of your rights under this section does not terminate the
licenses of parties who have received copies or rights from you under
this License. If your rights have been terminated and not permanently
reinstated, receipt of a copy of some or all of the same material does
not give you any rights to use it.

\subsubsection*{10. Future revisions of this License}

The Free Software Foundation may publish new, revised versions
of the GNU Free Documentation License from time to time. Such new
versions will be similar in spirit to the present version, but may
differ in detail to address new problems or concerns. See
\url{http://www.gnu.org/copyleft/}.

Each version of the License is given a distinguishing version number.
If the Document specifies that a particular numbered version of this
License ``or any later version'' applies to it, you have the option of
following the terms and conditions either of that specified version or
of any later version that has been published (not as a draft) by the
Free Software Foundation. If the Document does not specify a version
number of this License, you may choose any version ever published (not
as a draft) by the Free Software Foundation. If the Document
specifies that a proxy can decide which future versions of this
License can be used, that proxy's public statement of acceptance of a
version permanently authorizes you to choose that version for the
Document.

\subsubsection*{11. Relicensing}

``Massive Multiauthor Collaboration Site'' (or ``MMC Site'') means any
World Wide Web server that publishes copyrightable works and also
provides prominent facilities for anybody to edit those works. A
public wiki that anybody can edit is an example of such a server. A
``Massive Multiauthor Collaboration'' (or ``MMC'') contained in the
site means any set of copyrightable works thus published on the MMC
site.

``CC-BY-SA'' means the Creative Commons Attribution-Share Alike 3.0
license published by Creative Commons Corporation, a not-for-profit
corporation with a principal place of business in San Francisco,
California, as well as future copyleft versions of that license
published by that same organization.

``Incorporate'' means to publish or republish a Document, in whole or
in part, as part of another Document.

An MMC is ``eligible for relicensing'' if it is licensed under this
License, and if all works that were first published under this License
somewhere other than this MMC, and subsequently incorporated in whole
or in part into the MMC, (1) had no cover texts or invariant sections,
and (2) were thus incorporated prior to November 1, 2008.

The operator of an MMC Site may republish an MMC contained in the site
under CC-BY-SA on the same site at any time before August 1, 2009,
provided the MMC is eligible for relicensing.

\end{results}

\end{document}



\index{Folkestone and Hythe|see{Shepway}}

% Index:
\clearpage
\phantomsection
%\addcontentsline{toc}{chapter}{Index of Wards}
{\scriptsize%\raggedright
\frenchspacing\printindex}
\thispagestyle{plain}

\end{document}